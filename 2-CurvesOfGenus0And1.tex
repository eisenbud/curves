%header and footer for separate chapter files

\ifx\whole\undefined
\documentclass[12pt, leqno]{book}
\usepackage{graphicx}
\input style-for-curves.sty
\usepackage{hyperref}
\usepackage{showkeys} %This shows the labels.
%\usepackage{SLAG,msribib,local}
%\usepackage{amsmath,amscd,amsthm,amssymb,amsxtra,latexsym,epsfig,epic,graphics}
%\usepackage[matrix,arrow,curve]{xy}
%\usepackage{graphicx}
%\usepackage{diagrams}
%
%%\usepackage{amsrefs}
%%%%%%%%%%%%%%%%%%%%%%%%%%%%%%%%%%%%%%%%%%
%%\textwidth16cm
%%\textheight20cm
%%\topmargin-2cm
%\oddsidemargin.8cm
%\evensidemargin1cm
%
%%%%%%Definitions
%\input preamble.tex
%\input style-for-curves.sty
%\def\TU{{\bf U}}
%\def\AA{{\mathbb A}}
%\def\BB{{\mathbb B}}
%\def\CC{{\mathbb C}}
%\def\QQ{{\mathbb Q}}
%\def\RR{{\mathbb R}}
%\def\facet{{\bf facet}}
%\def\image{{\rm image}}
%\def\cE{{\cal E}}
%\def\cF{{\cal F}}
%\def\cG{{\cal G}}
%\def\cH{{\cal H}}
%\def\cHom{{{\cal H}om}}
%\def\h{{\rm h}}
% \def\bs{{Boij-S\"oderberg{} }}
%
%\makeatletter
%\def\Ddots{\mathinner{\mkern1mu\raise\p@
%\vbox{\kern7\p@\hbox{.}}\mkern2mu
%\raise4\p@\hbox{.}\mkern2mu\raise7\p@\hbox{.}\mkern1mu}}
%\makeatother

%%
%\pagestyle{myheadings}

%\input style-for-curves.tex
%\documentclass{cambridge7A}
%\usepackage{hatcher_revised} 
%\usepackage{3264}
   
\errorcontextlines=1000
%\usepackage{makeidx}
\let\see\relax
\usepackage{makeidx}
\makeindex
% \index{word} in the doc; \index{variety!algebraic} gives variety, algebraic
% PUT a % after each \index{***}

\overfullrule=5pt
\catcode`\@\active
\def@{\mskip1.5mu} %produce a small space in math with an @

\title{Personalities of Curves}
\author{\copyright David Eisenbud and Joe Harris}
%%\includeonly{%
%0-intro,01-ChowRingDogma,02-FirstExamples,03-Grassmannians,04-GeneralGrassmannians
%,05-VectorBundlesAndChernClasses,06-LinesOnHypersurfaces,07-SingularElementsOfLinearSeries,
%08-ParameterSpaces,
%bib
%}

\date{\today}
%%\date{}
%\title{Curves}
%%{\normalsize ***Preliminary Version***}} 
%\author{David Eisenbud and Joe Harris }
%
%\begin{document}

\begin{document}
\maketitle

\pagenumbering{roman}
\setcounter{page}{5}
%\begin{5}
%\end{5}
\pagenumbering{arabic}
\tableofcontents
\fi


\chapter{Curves of genus 0 and 1}\label{genus 0 and 1 chapter}

In this chapter, we'll begin our project of describing curves in projective space with the simplest cases, that of curves of genus 0 and 1. Despite the relative simplicity of these curves\footnote{In the case of genus 1 especially, the theory we will treat is ``simple'' because we are operating over $\CC$; the arithmetic theory occupies a major part of modern number theory!}, there are many interesting statements to make about the geometry of their embeddings in $\PP^r$, as well as many conjectures and open problems.

On a curve of genus 0, there is a unique invertible sheaf of given degree $d$; and on a curve of genus 1 all invertible sheaves of given degree $d$ are congruent modulo the automorphism group of the curve. Thus, all invertible sheaves of given degree $d$ behave in the same way. This is no longer true for curves of higher genus!

\section{Curves of genus 0} 

As we saw in more generality in Example~\ref{linear systems on Pr}, there is for each $d \in \ZZ$  a unique invertible sheaf $\cO_{\PP1} (d)$
of degree $d$ on $\PP^1$, and $H^0(\cO_{\PP1} (d))$ may be identified with the $(d+1)$-dimensional vector space of
forms of degree $d$ in 2 variables.
The fact that there are so many sections characterizes $\PP^1$:

\begin{theorem}\label{characterization of P1}
Let $C$ be a reduced, irreducible projective curve and let $\cL$ be an invertible sheaf of degree $d>0$ on $C$. If $\h^0(\cL) \geq d+1$ then
$C \cong \PP^1$, so $\cL \cong \cO_{\PP^1}(d)$, and $\h^0(\cL) = d+1$. 
\end{theorem}

\begin{proof}
Let $p_1,\dots p_{d-1}$ be general points of $C$, and set $\cL':=\cL(-p_1-\cdots-p_{d-1})$. From the correspondence between divisors and
invertible sheaves, we see that the degree of $\cL'$ is $1$.
 Since $\cL$ is locally isomorphic to the sheaf of functions on $C$, the condition of vanishing at a point imposes at most 1 linear condition on 
the global sections of $\cL$, and thus $H^0(\cL') \geq 2$, so we may assume from the outset that $d =1$.

The linear system $(\cL, H^0(\cL))$ cannot have any base points, since
otherwise after subtracting one, we would get an invertible sheaf of degree $\leq 0$ with two independent global sections. This is impossible, since some linear combination of the sections would vanish at any given point, showing that the degree would be
$\geq 1$.

Thus we see that the linear system $(\cL, H^0(\cL))$ defines a morphism $\phi: C\to \PP^1$ of degree 1 whose fibers are the divisors defined by
the vanishing of sections of $\cL$, and which are thus of degree 1. Thus if $p\in C$ is the preimage of $q\in \PP^1$, the induced map of local rings
$\phi^*:\cO_{\PP^1, q} \to \cO_{C, p}$ is a finite, birational map. Since $\cO_{\PP^1, q}$ is integrally closed, this is an isomorphism. Thus 
$\phi$ is an isomorphism, as required. 
 \end{proof}

Note that we used the algebraic closure of the ground field in choosing points on $C$, but not characteristic 0.

\begin{corollary}
 Every smooth curve $C$ of genus 0 over an algebraically closed field is isomorphic to $\PP^1$.
\end{corollary}

\begin{proof}
 By \trr, any linear system $\cL$ of degree $d$ on $C$ has $h^0\cL \geq d+1$.
\end{proof}


\begin{fact}
Over a non-algebraically closed field, a curve $C$ of genus 0 need not have any points, or any line bundles of odd degree (since the canonical bundle $K_C$ has degree $-2$, there do necessarily exist line bundles of every even degree; thus an arbitrary curve of genus 0 is isomorphic to a conic plane curve). 
The classification of curves of genus 0 over non-algebraically closed fields is a subject that goes back to Gauss.

Any smooth projective curve of genus 0 over a field $k$ is a \emph{form} of $\PP^1$ in the sense that they become isomorphic after extension
of scalars to
the algebraic closure $\overline k$. The unique example with $k = \RR$ is the conic $x^2+y^2+z^2 = 0$. 

Noncommutative algebras enter the subject of forms of of $\PP^1$ (and $\PP^n$ more generally) in a surprising way: The curve $\PP_k^1$ itself may be described as the scheme of left ideals of $k$-vector-space dimension 2 in the ring of
$2\times 2$ matrices over $k$ (such an ideal can be embedded in the matrix ring as a linear combination of the 2 columns in an appropriate sense). More generally, any scheme that is a form of $\PP^1$ over $k$
may be described as the scheme of 2-dimensional left ideals in a 4-dimensional central simple ($=$ Azumaya) algebra over $k$. For example, the
conic $x^2+y^2+z^2 = 0$ with no points over $\RR$ is the scheme of left ideals in the algebra of quaternions. See~\cite[]{Serre-Local Fields?}
\end{fact}

\section{Rational Normal Curves}\label{rational normal curves section}

Recall from Example~\ref{Veronese definition} that the image of the $d$-th \emph{Veronese map}  
$$
\phi_d: \PP^1 \to \PP(H^0((\cO_{\PP^1}(d)) \cong \PP^d \quad, (s,t) \mapsto (s^d, s^{d-1}t, \dots, t^d)
$$
corresponding to by the complete linear system $|\cO_{\PP^1}(d)|$ is called the \emph{rational normal curve} of degree $d$. Rational normal curves arise often in the literature because of their they have many extremal properties. 

\subsubsection{Rational normal curves have minimal degree}

\begin{proposition}
If $C$ is a reduced, irreducible nondegenerate curve in $\PP^d$ then $\deg C \geq d$, with equality if and only if $C$ is a  rational normal curve.
\end{proposition}

\begin{proof}
By the correspondence between morphisms and linear systems, the invertible sheaf $\cL$ corresponding to the morphism $C \hookrightarrow \PP^d$ has degree $d$ and
 $h^0(\cL) \geq d+1$. The conclusion follows from Theorem~\ref{characterization of P1}.
\end{proof}

We will see \fix{do it here. Quote a good version of Bertini?} more generally that, if $X$ is a non-degenerate variety in $\PP^d$ of dimension $k$, then $\deg(X) \geq d-k+1$; and we will describe the varieties that achieve the minimum in Chapter~\ref{ScrollsChapter}.

\subsubsection{Points on a rational normal curve are maximally independent}

\begin{proposition}\label{independence on rnc}
If $C\subset \PP^d$ is a rational normal curve of degree $d$ and $\Gamma\subset C$ is a subscheme of length $\ell \leq d+1$, then
$\Gamma$ lies on no plane of dimension $<\ell$. In particular, any $m \leq d+1$ distinct points on a rational normal curve $C \subset \PP^d$ are linearly independent
and span $\PP^d$.
\end{proposition}

The rational normal curve is the unique smooth curve with this property, as we shall see in Chapter~\ref{InflectionsChapter} \fix{give precise reference}. Since the linear span of a double point on $C$ is the tangent line, the Proposition implies that the tangent lines at a collection of $\leq (d+1)/2$ points are
independent as well.

\begin{proof}
 We can reduce to the case $\ell = d+1$ by adding points to $\Gamma$, so it suffices to do that case. Suppose that $C$ is the rational normal curve.
If $\Gamma$ is contained in a hyperplane then by B\'ezout's Theorem the degree of $C$ would be $\geq d+1$, a contradiction.
\end{proof}

For a related result see Corollary~\ref{uninflected}.

In the case of distinct points it is easy to make a direct argument: In affine coordinates chosen so that none of the points are
at infinity we can identify the points $\lambda_1,\dots,\lambda_{d+1} \in C \cong \PP^1$ with complex numbers, and the statement (for $\ell = d+1$) is tantamount to the nonvanishing of the Vandermonde determinant
$$
\begin{vmatrix}
1 & \lambda_1 & \lambda_1^2 & \dots & \lambda_1^d \\
1 & \lambda_2 & \lambda_2^2 & \dots & \lambda_2^d \\
\vdots & & & & \vdots \\
1 & \lambda_{d+1} & \lambda_{d+1}^2 & \dots & \lambda_{d+1}^d \\
\end{vmatrix}
= \prod_{1 \leq i < j \leq d+1} (\lambda_j - \lambda_i)
$$

\fix{edited to here 8/20/22}
\subsubsection{The equations defining a rational normal curve}

Choosing a basis $s,t$ for the linear forms on $\PP^1$, we can write
$$
\phi_d : (s,t) \mapsto (s^d, s^{d-1}t,\dots t^d)
$$
from which we see that $C$ lies in the zero locus of the homogeneous quadratic polynomial $z_iz_j - z_kz_l$ for every $i+j=k+l$. As a convenient way to package these, we can realize these forms the $2\times 2$ minors of the matrix
$$
M \; = \; \begin{pmatrix}
x_0 & x_1 & \dots & x_{d-1} \\
x_1 & x_2 & \dots & x_d
\end{pmatrix}.
$$
Note that if we substitute $s^it^{(d-i)}$ for $x_i$ and identify $H^0(\cO_{\PP^1}(i)$ with $\CC[s,t]_i$, this becomes the multiplication table
$$
H^0(\cO_{\PP^1}(i)) \times H^0(\cO_{\PP^1}(d-i-1)) \to H^0(\cO_{\PP^1}(d));
$$
we shall see a general version of this in Chapter~\ref{ScrollsChapter}, where we shall also prove that 
the minors of this matrix generate the ideal of forms on $\PP^d$ vanishing on $C$ (Proposition~\ref{RNC generators}).
For now we prove two slightly weaker results:

First, $C$ is set-theoretically defined by the $2\times 2$ minors of $M$. Explicitly, suppose that $p = (x_0,\dots,x_d) \in \PP^d$ is any point, and all the polynomials $Q_{ijkl}$ above vanish at $p$. If $x_0 = 0$, then from the vanishing of 
$\det \begin{pmatrix}
x_0 & x_1  \\
x_1 & x_2 
\end{pmatrix}$ 
we see that $x_1 = 0$, and similarly we have $x_2 = \dots = x_{d-1}=0$; this the point $p = (0,\dots,0,1)$, which is a point on the rational normal curve. On the other hand, if $x_0 \neq 0$, set $\lambda = x_1/x_0$; we see in turn that $x_2/x_1 = \dots = x_d/x_{d-1} = \lambda$; thus $p = (1, \lambda, \dots,\lambda^d)$, again a point of the rational normal curve.

Second, the ${d\choose 2}$ distinct $2\times 2$ minors of $M$ are linearly independent. To see this, write $\overline M$
for the matrix obtained by reducing $M$
modulo $x_0$ and $x_d$. It suffices to show that the minors of $\overline M$ generate the ${d\choose 2}$ quadratic monomials in $(x_1, \dots, x_{d-1})^2$. We may see this by induction on $d$: the minors of the reduced matrix
that involve the first column are the monomials 
$x_1x_{d-1}, x_2x_{d-1}, \dots, x_{d-1}^2)$, and modulo $x_{d-1}$ we get a matrix of the same form as $\overline M$ with one fewer column, whose
minors thus generate $(x_1, \dots, x_{d-2})^2$.

Note that
the restriction map
$$
H^0(\cO_{\PP^d}(2)) \; \to \; H^0(\cO_{C}(2)) = H^0(\cO_{\PP^1}(2d))
$$
 is surjective  because every monomial of degree $2d$ on $\PP^1$ is a product of two monomials of degree $d$. Comparing dimensions, we see that the dimension of the kernel---that is, the space of quadratic polynomials on $\PP^d$ vanishing on $C$---has dimension
$$
\binom{d+2}{2} - (2d+1) \; = \; \binom{d}{2}.
$$


In fact, this gives us another characterization of rational normal curves as extremal: rational normal curves lie on more quadric hypersurfaces than any other irreducible, nondegenerate curve in $\PP^d$.

\begin{proposition}\label{rnc on most quadrics}
If $C \subset \PP^d$ is any irreducible, nondegenerate curve, then
$$
h^0(\cI_{C/\PP^d}(2)) \leq  \binom{d}{2};
$$
and if equality holds then $C$ is a rational normal curve
\end{proposition}

\begin{proof}
Consider the restriction of the quadrics containing $C$ to a general hyperplane $H \cong \PP^{d-1} \subset \PP^d$, and let $\Gamma = H \cap C$. We have an exact sequence:
$$
0 \to \cI_{C/\PP^d}(1) \to \cI_{C/\PP^d}(2) \to \cI_{\Gamma/\PP^{d-1}}(2) \to 0.
$$ 
Since $C$ is nondegenerate, $h^0(\cI_{C/\PP^d}(1)) = 0$, and since $\deg C \geq d$, the hyperplane section $\Gamma$ of $C$ must contain at least $d$ linearly independent points. Since linearly independent points impose independent conditions on quadrics, we have
$$
h^0(\cI_{\Gamma/\PP^{d-1}}(2)) \leq h^0(\cO_{\PP^{d-1}}(2)) - d,
$$
establishing the desired inequality.
\end{proof}


From the projective normality of the rational normal curve (Example~\ref{rnc is projectively normal}), we know the Hilbert function of a rational normal curve $C \subset \PP^d$, and hence the dimensions of the graded pieces of its homogeneous ideal: we have
$$
h^0(\cI_{C/\PP^d}(m)) =  \binom{d+m}{m} - (md+1)
$$


\subsubsection{Rational normal curves are projectively homogeneous}

Another important property of rational normal curves $C \subset \PP^d$ is that they are \emph{projectively homogeneous} in the sense that the subgroup $G$ of the automorphism group $PGL_{d+1}$ of automorphisms of $\PP^d$ that carries $C$ to itself acts transitively on $C$. More generally,

\begin{proposition}\label{Veronese is projectively homogeneous}
If $X\subset \PP^n$ is the image of $\PP^r$ by the Veronese map corresponding to $|\sO_{\PP^r}(d)|$, then $X$ is projectively homogeneous.
\end{proposition}
\begin{proof}
First, $\PP^r$ is a homogeneous variety in the sense that $\Aut \PP^r$ acts transitively. If $\sigma$ is an automorphism then,
 because $\cO_{\PP^r}(d)$ is the unique
invertible sheaf of degree $d$ on $\PP^r$,  we have $\sigma^*\cO_{\PP^r}(d) = \cO_{\PP^r}(d)$ so $\sigma$ induces an automorphism $\phi$ on $H^0(\cO_{\PP^r}(d))$, and an automorphism $\overline \phi$ on the ambient space $\PP H^0(\cO_{\PP^r}(d))$ of the target of the $d$-th Veronese map. If $\alpha$
is a rational function with divisor $D$, then $\phi(\alpha) = \alpha\circ \sigma$ has divisor $\sigma^{-1}(D)$, so $\overline\phi^{-1}$ induces $\sigma$ on $\PP^r$. 
\end{proof}

The rational normal curve $C \subset \PP^r$ can  be characterized among irreducible, nondegenerate curves as the unique projectively homogeneous curve in $\PP^r$ (Corollary~\ref{no more projectively homogeneous curves}.

\subsubsection{Interpolation for rational normal curves}

Another remarkable property of rational normal curves is expressed in the following Proposition. Recall that a collection of points (or a finite subscheme)
of $\PP^n$ is said to be in \emph{linearly general position} if no $k+1$ of them lie in a $(k-1)$-plane with $k\leq n$. 

\begin{proposition}\label{points on rnc}
If $p_1,\dots, p_{n+3} \in \PP^n$ are any $n+3$ points in $\PP^n$ in linearly general position, then there exists a unique rational normal curve $C \subset \PP^n$ containing them.
 \end{proposition}

\begin{proof}
To start, we observe that there is an automorphism $\Phi : \PP^n \to \PP^n$ carrying the points $p_1,\dots,p_{n+1}$ to the coordinate points $(0,\dots,0,1,0,\dots,0)\in \PP^n$ and the point $p_{n+2}$ to the point $(1,1\dots,1)$.  denote the images of the remaining  points $p_{n+3}$  by $[\alpha_0,\dots,\alpha_n]$. We consider maps $\PP^1 \to \PP^n$ given in terms of an inhomogeneous coordinate $z$ on $\PP^1$ by
$$
z \mapsto \left( \frac{1}{z - \nu_0}, \frac{1}{z - \nu_1} , \dots, \frac{1}{z - \nu_n}  \right)
$$
with $\nu_0,\dots,\nu_n$ any distinct scalars. Clearing denominators, we see that the image of such a map is a rational normal curve, and it passes through the $n+1$ coordinate points of $\PP^n$, which are the images of the points $z = \nu_0, \dots, \nu_n \in \PP^1$. Moreover, the image of the point $z = \infty$ at infinity is the point $(1,1, \dots,1)$; and we can adjust the values of $\nu_0,\dots,\nu_n$ so that the image of the point $z = 0$ is $(\beta_0,\dots,\beta_n)$. This proves existence; we leave uniqueness as the Exercise~\ref{Castelnuovo uniqueness}. 
\end{proof}


There is another way to prove Proposition~\ref{points on rnc} that may provide more insight (it actually produces the equations defining the rational normal curve through the points $p_1,\dots,p_{n+3}$); this is described in \cite{Montreal}. 

We remark that for any family of curves in projective space $\PP^r$ (for example, a component of the family of all smooth, irreducible, nondegenerate curves of degree $d$ and genus $g$ in $\PP^r$) and any integer $m$, we can ask whether there exists a curve in the family passing through $m$ given general points of $\PP^r$. This is the only example we know of curves other than complete intersections for which there is a \emph{unique} such curve.


\subsection{Other rational curves}

What about other rational curves in projective space? 

Any linear series $\cD$ of degree $d$ on $\PP^1$ is a subseries of the complete series $|\cO_{\PP^1}(d)|$, so any map $\phi : \PP^1 \to \PP^r$ of degree $d$ may be given as the
composition of the embedding $\phi_d : \PP^1 \to \PP^d$ of $\PP^1$ as a rational normal curve with a linear projection $\pi : \PP^d \to \PP^r$. Since the natural degree $d$ Veronese embedding of $\PP^1 = \PP(V)$ as the rational normal curve is the map
$\PP(V) \to \PP(\Sym^d(V))$, the projection is a map into $\PP(W)$, where $W\subset \Sym^d(V)$. 

We can make this more explicit in several ways: first, choosing a basis $s,t$ for $V$ and a basis of forms $F_i$ of degree $d$ on $\PP^1$ for $W$, we may write the map as
$$
(s,t) \; \mapsto \; (F_0(s,t), \dots, F_r(s,t)).
$$
If the $F_i$ have a greatest common factor $G(s,t)$ then the set $G=0$ will be the base locus
of the linear system, and the map
$$
(s,t) \; \mapsto \; ( \frac{F_0(s,t)}{(G(s,t)}, \dots, \frac{F_r(s,t)}{(G(s,t)}).
$$
gives the same map, represented as a linear system of lower degree $d-\deg G$. Thus we
will now assume that the forms in $W$ have no common factor.

Perhaps even simpler, we can pass to an open affine set $\AA^1$ given by $t=1$ iin $\PP^1$, with coordinate function $z = s/t$ and
dehomogenize the $F_i$ to get a vector space of polynomials $f_i = F(s,1)$ of degree $\leq d$. Then we may write the
map as
$$
z \; \mapsto \; (f_0(z), \dots, f_r(z)).
$$
Since the $F_i$ have no common factor, and in particular are not all divisible by $t$, at least one of the
$f_i$ will have degree exactlly $d$; and since $\CC[z]$ is a principle ideal domain, any set of 
polynomials with no common divisor generats an ideal containing 
1 so after reordering the coefficients of $\PP^r$ write the map as
$$
z \; \mapsto \; (1, f'_1(z), \dots, f'_r(z)).
$$
or even
$$
\AA^1 \ni z \mapsto \; (f_1(z), \dots, f_r(z))\in \AA^n
$$
where again the polynomials $f_i'$ have degrees $\leq d$, and at least one has degree $d$.
For example, the twisted cubic itself can be represented by the map
$z \mapsto (z,z^2,z^3)$.

Here, we think of a polynomial $f(z) = f(s/t)$ of degree $\leq d$ as a rational function on $\PP^1$ having
a pole of order at most $d$ at the point at infinity $(1,0)\in \PP^1$; but we could also take rational
functions $F_i(s,t)/G_i(s,t)$ of total degree $\deg F_i-\deg G_i = d$ or, dehomogenizing, $\phi_i(z) = f_i(z)/g_i(z)$.

\subsection *{Smooth rational quartics}
Given how easy it is to describe rational curves in projective space in this way, it is surprising how many open questions there are. We begin with
one of the simplest cases: a smooth nondegenerate rational curve $C$ of degree $4$ in $\PP^3$.
As described abover, such a curve is the image of $\PP^1$ under a map given by 4 quartic forms,
or, in the most coordinate-free formulation, a codimension 1 subspace of $\Sym^4(V)$, where
$V\cong \CC^2$.

Our analysis of this case follows a pattern that we will repeat in many other situations in the next
chapters. We first ask what surfaces contain such a curve $C$---that is, what forms on $\PP^3$ vanish on $C$. 

\begin{proposition}
If $C\subset \PP^3$ is a smooth rational quartic curve, then $C$ lies on a smooth quadric
surface $Q=0$ and the homogenous ideal of $C$ is minimally
generated by $Q$ and three cubic forms.
\end{proposition}


\begin{proof}
We first consider the maps restriction maps
$$
\rho_e: H^0(\op3e) \to H^0(\cO_C(e)).
$$
Since we assume that $C$ is smooth, thus $C\cong \PP^1$,
we may identify $H^0(\cO_C(e))$ with $H^0(\op1{4e})$.
 Since we assume that $C$ is nondegenerate, it does not lie on a hyperplane,
 so for $\rho_1$ is a monomorphism. 
 
The source $H^0(\op32)$ of $\rho_2$ is 10-dimensional, and the target $H^0(\op18)$ is
9-dimensional, so $\rho_2$ has a kernel of dimension at least 1; that is, $C$ lies on
a quadric surface, and since $C$ is irreducible and nondegenerate, any quadric surface containing
$C$ is irreducible, and must in fact be smooth. This is a special case of 
Exercise~\ref{curves on cones}, and the reader may work it out in Exercise~\ref{quartic on quadric} below.

If $C$ lay on a second quadric surface, then, since the quadrics must be irreducible,
$C$ would be contained in the complete intersection of the two, which has degree 4, so 
$C$ would be the complete intersection of the two quadrics.

Given that $C$ lies on a smooth quadric surface $S$, we consider its divisor class $(a,b)$ in the 
Picard group $\Pic(Q) = \ZZ\oplus \ZZ$. The complete intersection of $S$ with another
quadric, lying in the divisor class of twice a hyperplane section, is $(2,2)$. But a curve
in the class $(a,b)$ has degree $a+b$ and genus $(a-1)(b-1) = 0$ (see Example~\ref{Div of quadric}), so a curve in the class $(2,2)$
has genus 1, not zero. In fact solving the equations $a+b=4, (a-1)(b-1)=0$we see that $C\sim (1,3)$ or $C\sim (3,1)$. Since the cases
are symmetric we assume $C\sim(1,3)$. 

The source of $\rho_3$ is 20 dimensional and the target is 13 dimensional so there are at least 7
cubics in the ideal of $C\subset \PP^3$. Four of these come from multiplying the quadric
by the 4 variables on $\PP^3$, so there are at least 3 more cubic generators in $I_{C/\PP^3}$,
 the homogeneous ideal of $C$. 

We now use the exact sequences 
$$
0\to I_{S/\PP^3} \to I_{C/\PP^3} \to I_{C/S} \to 0.
$$
and
$$
0\to \sI_{S/\PP^3} \to \sI_{C/\PP^3} \to \sI_{C/S} \to 0.
$$
Note that $\sI_{S/\PP^3} \cong \op3{-2}$, and since $H^1(\op3d)=0$ for all $d$, 
the sequences
$$
0\to H^0(\sI_{S/\PP^3}(d)) \to H^0(\sI_{C/\PP^3}(d)) \to H^0(\sI_{C/S}(d)) \to 0.
$$
are exact for all $d$. 

Since $C$ lies in class $(1,3)$ we see that 
$$
\sI_{C/S}(d) = \sI_{(-1+d,-3+d)/\PP^1\times \PP^1}.
$$
Thus 
$$
h^0(\sI_{C/S}(3)) = h^0(\sI_{(2,0)/\PP^1\times \PP^1} = h^0(\op12)\cdot h^0(\op10) = 3
$$
and we see that $I(C)$ contains just 3 cubic generators. 

Moreover, the restriction
of $H^0(\op31)$ to the quadric is $H^0(\sI_{(2,0)/\PP^1\times \PP^1}(1,1)$,
so $H^0(\op31) \otimes H^0(\sI_{(2,0)/\PP^1\times \PP^1}(a,b) \to 
H^0(\sI_{(2,0)/\PP^1\times \PP^1}(a+1, b+1)$
is surjective whenever both $a$ and $b$ are non-negative. In particular
$$
I_{C/S} = \oplus_{d=0}^\infty H^0(\sI_{C/S}(d)
$$
 is generated in degree 3. Since $I_{S/\PP^3}$ is generated by one quadric, this
  completes the proof.
\end{proof}

%One way to do this would be to carry out an analysis, along the lines of the one above in the case $Q$ is smooth, on the desingularization $\tilde Q$ of a quadric cone $Q$. (This is obtained simply by  blowing up $Q$ at the vertex.) In other words, we could determine the Picard group $\Pic(\tilde Q)$, with its intersection pairing and canonical class $K_{\tilde Q}$; we could then ask what the class of the proper transform $\tilde C$ of $C$ in $\tilde Q$ could be and arrive in this way at a contradiction. We'll outline a proof along these lines in Exercise~\ref{F2} below; but this chapter's long enough already, so we'll take an ad-hoc approach.

%Our approach will proceed in three steps:
%
%\begin{enumerate}
%\item a smooth curve $C \subset Q$ of even degree cannot contain the vertex of $Q$;
%\item a curve $C \subset Q$ not containing the vertex is the intersection of $Q$ with a surface $S \subset \PP^3$; and
%\item a smooth  intersection of two quadrics in $\PP^3$ has genus 1.
%\end{enumerate}
%
%For the first assertion, suppose that the curve $C$ passes through the vertex and meets a general line $L \subset Q$ in $l$ points other than the vertex. Then the intersection of $C$ with a general plane $H$ through the vertex would be transverse and consist of $2l+1$ points, contradicting the hypothesis that $C$ had even degree. Thus in particular our quartic curve $C$ does not pass through the vertex, and hence is a Cartier divisor on $Q$.
%
%The second observation follows from the fact that for any line $L \subset Q$, \emph{the complement $Q \setminus L$ is isomorphic to $\AA^2$}. Thus any invertible sheaf on $Q$ is trivial on the complement of $L$; accordingly, any Cartier divisor on $Q$ is linearly equivalent to a multiple of $L$. But odd multiples of $L$ cannot be Cartier divisors on $Q$, and $2L$ is a hyperplane section of $Q$, so we must have $\cO_Q(C) = \cO_Q(m)$ for some $m$. From the exact sequence
%$$
%0 \to \cO_{\PP^3}(m-2) \to \to \cO_{\PP^3}(m) \to \cO_{Q}(m)  \to 0
%$$
%and the vanishing of $H^1(\cO_{\PP^3}(m-2))$, we deduce that $C$ must be the complete intersection of $Q$ with a surface of degree $m$.
%
%Finally, if $C = Q \cap Q'$ is the smooth intersection of two quadrics, we can (after replacing $Q$ and $Q'$ with general linear combinations of the two) assume that $Q$ and $Q'$ are smooth; applying adjunction twice (to $Q \subset \PP^3$ and then to $C \subset Q$), we arrive at $K_C = \cO_C$, so the genus of $C$ is 1.

\subsection{Problems on rational curves}

We can say far less about rational curves of higher degree, even in $\PP^3$.  For arbitrary $d$ and $r$, we don't even know the possible Hilbert functions of a rational curve of degree $d$ in $\PP^r$. Even in the $\PP^3$ we don't know the answers for higher degree.

However, we do know the Hilbert function of a \emph{general} rational curve $C \subset \PP^r$ of degree $d$. To make sense of this, let $C_0 \subset \PP^d$ be a rational normal curve of degree $d$. 
As described above, any rational curve of degree $d$ in $\PP^r$ is the image of $\PP^1$ under the map defined by
a linear system $(\op1d, V)$ where $V$ is an $(r+1)$-dimensional subspace of the space of
forms of degree $d$ in 2 variables, $H^0(\op1d)$. Thus we can talk about a \emph{general rational curve} of degree $d$ in $\PP^r$ by considering general subspaces of this type, and we can ask:
What is the Hilbert function of a general rational curve $C \subset \PP^r$ of degree $d$? 

As in the example, this is tantamount to looking at the restriction maps
$$
\rho_m : H^0(\cO_{\PP^r}(m) \to H^0(\cO_C(m)) = H^0(\cO_{\PP^1}(md)).
$$
Equivalently, we're asking: if $V$ is a general  $(r+1)$-dimensional vector space of homogeneous polynomials of degree $d$, what is the dimension of the space of polynomials spanned by $m$-fold products of polynomials in $V$? 

We might naively guess that the answer is, ``as large as possible," meaning that the rank of $\rho_m$ is $\binom{m+r}{r}$ when that number is less than $md+1$, and equal to $md+1$ when it is greater---in other words, the map $\rho_m$ is either injective or surjective for each $m$. This was proven in~\cite{Ballico-Ellia83}. 

As we will see in subsequent Chapters it is possible to speak of a general curve of genus $g$
and a general invertible sheaf of degree $d$ on such a curve; and the analogous statement 
about Hilbert functions  was proven in \cite{ELarson2018}; see Chapter~\ref{Brill-Noether}.

Nevertheless, even the degrees of the generators of the homogeneous ideal of a general
rational curve of degree $d$ in $\PP^3$ is unknown for larger $d$. \fix{DE wrote to Iarrobino 8/13 to ask whether he knows.}

\subsubsection{The secant plane conjecture}

Given a smooth curve $C \subset \PP^r$, we say that an $e$-secant $s$-plane to $C$ is an $s$-plane $\Lambda \cong \PP^s \subset \PP^r$ such that the intersection $\Lambda \cap C$ has degree $\geq e$; if we exclude degenerate cases (for example, where $\Lambda \cap C$ fails to span $\Lambda$), this is the same as saying we have a divisor $D \subset C$ of degree $e$ whose span is contained in an $s$-plane.

Should you expect a curve $C \subset \PP^r$ to have any $e$-secant $s$-planes? The set of $s$-planes in $\PP^r$ is parametrized by the Grassmannian $\GG = \GG(s,r)$, which had dimension $(s+1)(r-s)$. Inside $\GG$, the locus of planes that meet $C$ has codimension $r-s-1$ (the locus of planes containing a given point of $C$ has codimension $r-s$); so the naive expectation might be that the locus of $e$-secant $s$-planes would have codimension $e(r-s-1)$ in $\GG$. Thus one might conjecture that a curve $C \subset \PP^r$ to have $e$-secant $s$-planes when 
$$
e \; \leq \; (s+1)\frac{r-s}{r-s-1},
$$
perhaps with a few low degree exceptions. Is this true of a general rational curve? For most $e$, $r$ and $s$, we don't know!

\section{Curves of genus 1}

We will describe the maps of a curve of genus 1 given by
the complete linear systems in the lowest degree cases of interest, 2,3,and 4 and 5. Along the
way we will see evidence for the assertion that there is a 1-parameter family of non-isomorphic
curves of genus 1, an assertion that we will make precise in~\ref{ModuliChapter}.


On a curve $E$ of genus 1 the canonical sheaf has degree 0; and since it has 1 section, it must be $\sO_C$.
Since invertible sheaves of negative degree cannot have any sections, the \trr shows that
$h^0( \sL) = \deg \sL$ for any $\sL$ of positive degree. Among the surprising consequences is that, once we choose a point as 
origin (technically, making our curve of genus 1 into an \emph{elliptic curve}) the  the curve becomes
an algebraic group. 

The group operation is easy to describe:
Let $E$ be an elliptic curve with origin $o\in E$ chosen arbitrarily. If $p,q$ are points of $E$ then $\sO_E(p+q-o)$ has degree 1, and
thus has a unique global section. This vanishes at a unique point $r$, which may also be described as the unique
effective divisor linearly equivalent to $p+q-o$. For example if $r$ is the  unique point
linearly equivalent to $2o-p$ then $p+r-o\sim o$, so that $r$ is the inverse $-p$ of $p$. Note that this operation is commutative.

\begin{proposition}\label{group law} Let $E$ be an elliptic curve with origin $o\in E$.
If we set $p+q = r$ where $r$ is the unique effective divisor linearly equivalent to $p+q-o$, then $E$ becomes an algebraic group
and the group of divisor classes is divisible, in the sense that for divisor $D$ of degre $n>0$
 there is a point $p$ such that $D\sim np$.
 \end{proposition}

\begin{remark}
From the definition it is obvious that 
the map
$E \to \Pic_0(E)$ sending $p$ to $\sO_E(p-o)$ is an isomorphism of groups.
 In Chapter~ref{JacobianChapter} we will see a similar construction: the Picard groups can be made into
varieties, and for a curve $C$ of genus $g$ the divisors
of degree $g$ form a group that is birational to $Pic_g(C)$; for smooth curves any birational map is an isomorphism.
For the moment we simply note that there is a natural sense in which the invertible sheaves of degree 0 on $E$
form a 1-dimensional family; and multiplying by any fixed divsor of degree $d$, we see similarly
that $\Pic^d(E)$ is naturally a 1-dimensional family. 
\end{remark}
 
\begin{proof}
To show that the group operation is given by regular functions, we map $E$ to $\PP^2$ by the complete linear system $|3o|$. Since
$h^0(3o) = 3$ but $h^0(3o-p-q) = 1$ for any points $p,q$, this is an embedding. Moreover, there is a line in $\PP^2$ that meets
$E$ triply at $o$, which is thus an inflection point, and nowhere else. The three points $p,q,r$ in which any other line in the plane
meet $E$ sum to a divisor linearly equivalent to $3o$, and thus sum to zero in the group law, that is $r = -p-q$ Since $r$ is the unique
point in which the line
through $p,q$ meets  E, it follows that the coordinates of $r$ are polynomial functions of those of $p,q$, so the operation
$(p,q) \to -p-q$ is algebraic. But by the same token, the tangent line to $E$ at $r$ meets $E$ again at the point $-r$,
so the operation $r\to -r$ is also algebraic.

Given the group operation, we see that multiplication by a positive integer $n$ defines a non-constant map of 
curves $E\to E$. Since $E$ is projective, this map is surjective, proving the divisibility of the divisor classes of degree 0. 
Given a divisor class $D$ of degree $n$, the class $D -no$has degree 0, and thus can be written as $n(r-o)$, so
$D\sim nr$.
\end{proof}

\begin{corollary}\label{equivalence of sheaves}
Given two invertible sheaves $\sL, \sL'$ on an of the same degree on a curve $E$ genus 1, there is an automorphism $\sigma: E\to E$
such that $\sigma^*\sL = \sL'$.
\end{corollary}

\begin{proof}
By Proposition~\ref{group law} we may write $\sL \cong \sO_E(np)$ and $\sL'\cong \sO_E(np')$ for some points $p,p'$; and tranlation by $p-p'$
is an automorphism of $E$ carrying one into the other.
\end{proof}


For the rest of the chapter, we fix a smooth irreducible projective curve $E$ of genus 1.

\subsection{Double covers of $\PP^1$}

Let $E$ be a smooth projective curve of genus 1 and let  $\sL$ be an invertible sheaf of degree 2 on $E$. By \trr{},\kern -3pt $h^0(\sL) = 2$ and the linear series $|\sL|$ is base point free, so we get a map $\phi : E \to \PP^1$ of degree 2. By \trh the map $\phi$ will have 4 branch points, which must be distinct because in a degree 2 map
only simple branching is possible. By Corollary~\ref{equivalence of sheaves}, these four points are determined, up to automorphisms of $\PP^1$ by the curve $E$, and are independent of the choice of $\sL$.

After composing with an automorphism of $\PP^1$ we can take these four points to be $0, 1, \infty$ and $\lambda$ for some $\lambda \neq 0, 1 \in \CC$. Since there is a unique double cover of $\PP^1$ with given branch divisor (see Lemma~\ref{branched cover classification} it follows that $E \cong E_\lambda$, where $E_\lambda$ is the curve given by the affine equation
$$
y^2 = x(x-1)(x-\lambda).
$$

When are two curves $E_\lambda$ and $E_{\lambda'}$ isomorphic? By what we've said, this will be the case if and only if there is an automorphism of $\PP^1$ carrying the points $\{0,1,\infty,\lambda\}$ to $\{0,1,\infty,\lambda'\}$, in any order. This will be the case if and only if $\lambda$ and $\lambda'$ belong to the same orbit under the action of the group $G \cong S_3 \subset PGL(3)$ of automorphisms of $\PP^1$ permuting the three points $0, 1$ and $\infty$. Direct computation shows that the orbit of $\lambda$ is
$$
\lambda' \in \{\lambda, \; 1-\lambda, \; \frac{1}{\lambda},\;  \frac{1}{1-\lambda}, \; \frac{\lambda - 1}{\lambda}, \; \frac{\lambda}{\lambda - 1} \}.
$$
We now use L\"uroth's Theorem: any subfield of $\CC(x)$ having transcendence degree 1 over $\CC$ is equal to $\CC(y)$ for some rational function $j$.
Since the quotient of a normal variety by a group is normal, this implies that any quotient of $\PP^1$ by a finite group is again isomorphic to $\PP^1$.
One can check that $j$ can be taken to be
$$
j \; = \; 256\cdot \frac{\lambda^2 - \lambda + 1}{\lambda^2(\lambda - 1)^2}.
$$
(the factor of 256 is there for arithmetic reasons). Thus there is a unique smooth projective curve of genus 1 for each value of $j$, and in particular, the family of
curves of genus 1 is parametrized by points on a line.

\subsection{Plane cubics}

Let $\sL$ be an invertible sheaf of degree 3 on $E$. As in the proof of Proposition~\ref{group law} the linear system $|\sL|$ gives an embedding of $E$ as a smooth plane cubic curve of degree 3; conversely, the genus formula tells us that a smooth plane cubic curve indeed has genus 1. 

The space of plane cubic curves is parametrized by the space of cubic forms in 3 variables up to 
scalars, a  $\PP^9$. The locus of forms defining smooth curves is a Zariski open subset. On the other hand, by what we've said, two plane cubics are isomorphic iff they are congruent under the group $PGL_3$ of automorphisms of $\PP^2$. Since the group $PGL_3$ has dimension 8, one should expect that the family of such curves up to isomorphism has dimension 1.



\subsection{Quartics in $\PP^3$} 

Again, let $E$ be a smooth projective curve of genus 1, and consider now the embedding of $E$ into $\PP^3$ given by the sections of an invertible sheaf $\sL$ of degree 4. To understand the ideal of $E$ we consider the restriction map
$$
\rho_2 \;  : \; H^0(\cO_{\PP^3}(2)) \; \to \; H^0(\cO_{E}(2)) = H^0(\sL^2).
$$
The space on the left---the space of homogeneous polynomials of degree 2 in four variables---has dimension 10, while by Riemann-Roch the space $H^0(\sL^2)$ has dimension 8. It follows that $E$ lies on at least two linearly independent quadrics $Q$ and $Q'$. Since $E$ does not lie in any plane, neither $Q$ nor $Q'$ can be reducible; thus by \bt we see that
$$
E = Q \cap Q'
$$
is the complete intersection of two quadrics in $\PP^3$. Moreover, we also see from the Lasker-Noether ``AF+BG" theorem that the kernel of $\rho_2$ is exactly the span of $Q$ and $Q'$. Thus $E$ determines a point in the Grassmannian $G(2, H^0(\cO_{\PP^3}(2))) = G(2, 10)$ of pencils of quadrics; and by Bertini's Theorem, a Zariski open subset of that Grassmannian correspond to smooth quartic curves of genus 1. We can use this to once more calculate the dimension of the family of curves of genus 1: the Grassmannian $G(2,10)$ has dimension 16, while the group $PGL_4$ of automorphisms of $\PP^3$ has dimension 15, so again one should expect that the family of curves of genus 1 up to isomorphism has dimension 1.

There is a direct way to go back and forth between the representation of the smooth genus 1 curve $E$ as the intersection of two quadrics in $\PP^3$ and the representation of $E$ as a double cover
of $\PP^1$ branched at 4 distinct points. First, by Bertini's Theorem, we may take the two quadrics to be nonsingular, since they must meet transversely along $E$, and elsewhere the
pencil of quadrics they span has no base points. Representing the quadrics as symmetric matrices $A,B$, the pencil of all quadrics containing $E$ can be 
written as $sA+tB$. A quadric in the pencil is singular at the points $(s,t)$ such that the quartic polynomial $det(sA+tB)$ vanishes; thus at 4 points.

 A smooth quadric has two rulings by lines; a cone has one. Thus the family
$$
\Phi := \{ (\lambda, \sL) \mid \sL \in \Pic(Q_\lambda) \text{ is the class of a ruling of } Q_\lambda \}
$$
is---at least set-theoretically---a 2-sheeted cover of $\PP^1$, branched over the four values of $\lambda$ corresponding to singular quadrics in the pencil. In fact, we claim

\begin{proposition}\label{rulings on pencil}
There is a natural identification of $\Phi$ with $E$, and thus the branch points of $\Phi$ over $\PP^1$---that is, the set of singular elements of the pencil of quadrics---are the same, up to automorphisms of $\PP^1$ as the four points over which a double cover of $\PP^1$ by $E$ are ramified.
\end{proposition} 


\begin{proof}
First, choose a base point $o \in E$. We will construct inverse maps $E \to \Phi$ and $\Phi \to E$ as follows:
\begin{enumerate}
\item Suppose $q \in E$ is any point other than $o$, and let $M = \overline{o,q}$. Every quadric $Q_\lambda$ contains the two points $o, q \in M$; if $r \in M$ is any third point, there will be a unique $\lambda$ such that $r$, and hence all of $M$, lies in $Q_\lambda$. Thus the choice of $q$ determines both one of the quadrics $Q_\lambda$ of the pencil, and a ruling of that quadric, giving us a map $E \to \Phi$.
\item The other direction is similarly straightforward. Given a quadric $Q_\lambda$ and a choice of ruling of $Q_\lambda$, there will be a unique line $M \subset Q_\lambda$ of that ruling passing through $o$, and that line $M$ will meet the curve $E$ in one other point $q$; this gives uf the inverse map $\Phi \to E$.
\end{enumerate}
\end{proof}

There is a beautiful extension of this result to pencils of quadrics in any odd-dimensional projective space. Briefly: a smooth quadric $Q \subset \PP^{2g+1}$ has two rulings by $g$-planes, which merge into one family when the quadric specializes to quadric of rank $2g+1$, that is, a cone over a smooth quadric in $\PP^{2g}$. If $\{Q_\lambda\}_{\lambda \in \PP^1}$ is a pencil with smooth base locus $X = \cap_{\lambda \in \PP^1} Q_\lambda$, then exactly $2g+2$ of the quadrics will be singular, and they will all be of rank $2g+1$. The space $\Phi$ of rulings of the quadrics $Q_\lambda$ is thus a double cover of $\PP^1$ branched at $2g+2$  points; that is, a hyperelliptic curve of genus $g$. For a proof see for example~\cite[Proposition 22.34]{Harris1995}.
 This shows in particular that the polynomial $\det(sA+tB)$ has $2g+2$ \emph{distinct} roots. 


There is also a remarkable analogue of Proposition~\ref{rulings on pencil} for any $g$: the variety $F_{g-1}(X)$ of $g-1$-planes in the base locus $X$ of the pencil is isomorphic to the Jacobian of the  curve $\Phi$. A proof of this in case $g=2$ can be found in~\cite{Griffiths-Harris1978}; for all $g$ it is done in~\cite{Donagi}

We can continue to describe the ideals of the complete embeddings of genus 1 curves in any degree because they lie on 2-dimensional rational
normal curves---see Appendix 2. However, there is a particularly nice description in $\PP^4$. 

\begin{fact}[Quintic curves of genus 1 in $PP^4$]
Let $E$ be a smooth curve of genus 1. Any invertible sheaf $\sL$ of degree $5$ on $E$ is very ample, and considering the map $H^0(\op42) \to H^0(\sL^2)$
we see that $E\subset \PP^4$ lies on (at least 5 quadrics). 
Recall that if $A$ is a skew-symmetric matrix
then the determinant of $A$ is the square of a polynomial in the entries of $A$ called the Pfaffian of $A$. For example, if
$$
M = \begin{pmatrix}
0&x_{1,1}&x_{1,2}&x_{1,3}\\
-x_{1,1}&0&x_{2,2}&x_{2,3}\\
-x_{1,2}&-x_{2,2}&0&x_{3,3}\\
-x_{1,3}&-x_{2,3}&-x_{3,3}&0\\
\end{pmatrix}
$$
then the Pfaffian of $M$ is by definition $x_{1,1}x_{3,3}-x_{1,2}x_{2,3}+x_{1,3}x_{2,2}$.

\begin{proposition} \cite[Theorem11]{Eisenbud1995}
If $E\subset \PP^4$ is a smooth curve of genus 1 and degree 5, then there is a skew-symmetric $5\times 5$ matrix of linear forms $A$
such that the homogeneous ideal of $E$ is generated by the  Pfaffians of the 5 $4\times 4$ submatrices of $A$ obtained by removing
a given row and the corresponding column.
\end{proposition}
\end{fact}

\section{Exercises}


\begin{exercise}
 Let $\nu_d: \PP^r \to \PP^{\binom{r+d}{r}-1}$ be the $d$-Veronese map, and let $C\subset \PP^r$ be the rational normal curve of degree $r$. Is $\nu_d(C)$ nondegenerate? If not, what is the dimension of its linear span (that is, of the smallest linear
 space that contains it?
\end{exercise}

\begin{exercise}
Establish the analog of Proposition~\ref{rnc on most quadrics} for hypersurfaces of any degree $m$, that is to say no irreducible, nondegenerate curve in $\PP^r$ lies on more hypersurfaces of degree $m$ than the rational normal curve.
To do this, let $C\subset \PP^d$ be any irreducible nondegenerate curve, and use the exact sequences
$$
0 \to \cI_{C/\PP^d}(l-1) \to \cI_{C/\PP^d}(l) \to \cI_{\Gamma/\PP^{d-1}}(l) \to 0.
$$ 
with $2 \leq l \leq m$ to show that
$$
h^0(\cI_{C/\PP^d}(m)) \leq  \binom{d+m}{m} - (md+1)
$$
with equality only if $C$ is a rational normal curve.
\end{exercise}

\begin{exercise}
Prove directly  the special case $r=3$: that the twisted cubic is the unique irreducible, nondegenerate space curve lying on three quadrics. 
\end{exercise}


\begin{exercise}\label{F2}
Let $Q \subset \PP^3$ be a cone over a smooth conic curve in $\PP^2$; let $\pi : S \to Q$ be the blow-up of $Q$ at the vertex and $E \subset S$ the exceptional divisor of the blow-up.
\begin{enumerate}
\item Show that $S$ is smooth.
\item Show that the Picard group $\Pic(S)$ is freely generated by two classes, the class $f$ of the proper transform of a line in $Q$ and the class $e$ of the exceptional divisor.
\item\label{intersections on a quadric} Show that the intersection pairing on $\Pic(S)$ is given by
$$
f \cdot f = 0; \quad f \cdot e = 1 \quad \text{and} \quad e \cdot e = -2
$$
(Hint: one way to do the last of these is to show that a hyperplane section of $Q$ has class $h = e + 2f$ and use $h^2 = 2$.)
\item Show that the canonical class $K_S = -2e-4f$.
\item Show that if $\tilde C \subset S$ is a curve with class $ae + bf$, and
$C = \pi(\tilde C) \subset Q \subset \PP^3$, then
$$
\deg(C) = b \quad \text{and} \quad g(\tilde C) = a(b-a) -b +1
$$
\item Deduce that there does not exist a smooth rational quartic curve on a quadric cone.
\end{enumerate}
\end{exercise}

\begin{exercise}
As a consequence of our description of rational quartic curves, show that a general $g^3_4$ on $\PP^1$ is uniquely expressible as a sum of the $g_1^1$ and a $g^1_3$
(in other words, a general 4-dimensional vector space of quartic polynomials on $\PP^1$ is uniquely expressible as the product of a 2-dimensional vector space of cubics and the 2-dimensional space of linear forms.
\end{exercise}

\begin{exercise}
Show that, up to projective equivalence, there is a 1-parameter family of rational quartic curves in $\PP^3$ 
by constructing an invariant that distinguishes them. (Hint: think of a curve of type $(1,3)$ on a quadric as a the graph of a degree 3 map $\PP^1 \to \PP^1$, and use Hurwitz' Theorem.)
\end{exercise}






\begin{exercise}\label{Castelnuovo uniqueness}
Show that if $C, C' \subset \PP^n$ are two rational normal curves and $\#(C \cap C') \geq n+3$, then $C = C'$. (Hint: use induction on $n$.)
\end{exercise}


\begin{exercise}\label{rnc and representations}
Let $V = \CC\cdot e_1\oplus \CC\cdot e_2$ be a 2-dimensional vector space. 

The group $SL_2= SL(V)$ acts on the rational normal curve of degree $d$ through automorphisms induced from its action on
 on the ambient space $\PP^d$ of the rational normal curve, which may be identified with $\PP(\Sym^d(V))$.

In~\cite[pp. 146--150]{Fulton-Harris} it is shown that
 every finite dimensional rational 
representation of $V$ is a direct sum of representations of the form $\Sym^e(V)$ for various $e\geq 0$. Moreover, it is often easy to understand
how a given representation decomposes by looking at the action of
$$
\alpha := \begin{pmatrix}
t&0\\
0&t^{-1}
\end{pmatrix}
\in SL(V).
$$
Note that $\Sym^e(V)$ is spanned by ``weight vectors" ($\equiv$ eigenvectors of $\alpha$) $w_s := e_1^{e-s} e_2^{s}$ 
which satisfy $\alpha w_s = t^{e-2s}$ for $s = 0, \dots e$.
To decompose an arbitrary representation $W$, knowing that $W$ is a direct sum of $\Sym^{e_i}V$, it is enough to know the 
eigenvalues for the action of $\alpha$: We begin by finding an element $w\in W$ that
is an eigenvector of $\alpha$ and transforms by $\alpha$ as
as $\alpha w = t^mw$ with the highest possible $m$ (this is called a ``highest weight vector''). This element $w$ must be contained
in a summand $\Sym^m(V)$, and after removing the eigenvalues of the action of $SL_2$ on $\Sym^m(V)$, we continue. 
\begin{enumerate}
 \item Use this method to show that 
\begin{align*}
&\Sym^d(V)\otimes Sym^d(V)= \Sym^{2d}(V) \oplus  \Sym^{2d-2}(V) \oplus \Sym^{2d-4}(V) \cdots\\
 &\Sym^2(\Sym^d(V))= \Sym^{2d}(V) \oplus \Sym^{2d-4}(V)\oplus \Sym^{2d-8}(V) \cdots\\
 &\bigwedge^2(\Sym^d(V))= \Sym^{2d-2}(V) \oplus \Sym^{2d-6}(V)\oplus \Sym^{2d-10}(V) \cdots
\end{align*}
  (where we take $\Sym^{m}(V)=0$ when $m<0$
 \item Show that the space of quadrics containing the rational normal curve is a representation of $SL_2$ of the form
 $$
 \Sym^{2d-4}(V)\oplus \Sym^{2d-8}(V) \cdots
 $$
  \item Show  there is a distinguished nonsingular skew-symmetric form (up to scalars) on the ambient space of the twisted cubic; in particular
  is, given a twisted cubic in $\PP^3$ there is a distinguished plane containing each point of $\PP^3$.
 \item Show that if $d$ is divisible by 4 there is a distinguished quadric in the ideal of the rational normal curve.
\end{enumerate}
\end{exercise}

\begin{exercise}
Let $\PP^1 \hookrightarrow C \subset \PP^3$ be a twisted cubic. Show that the normal bundle $\cN_{C/\PP^3}$ (defined to be the quotient of the restriction $T_{\PP^3}|_C$ to $C$ of the tangent bundle  of $\PP^3$  by the tangent bundle $T_C$) is 
$$
\cN_{C/\PP^3} \cong \cO_{\PP^1}(5) \oplus  \cO_{\PP^1}(5).
$$
Hint: for any point $p \in C$, let $L_p \subset \cN_{C/\PP^3}$ be the sub-line bundle of $\cN_{C/\PP^3}$ whose fiber over any point $q \neq p \in C$ is the one-dimensional subspace of $(\cN_{C/\PP^3})_q$ spanned by the line $\overline{p,q}$. (This of course only defines a sub-line bundle of $\cN_{C/\PP^3}$ over $C \setminus \{p\}$, but there is a unique extension to a sub-line bundle of $\cN_{C/\PP^3}$ over all of $C$.) Show that for $p \neq p'$ we have
$$
\cN_{C/\PP^3} = L_p \oplus L_{p'}.
$$
\end{exercise}

\begin{exercise}
Let $\PP^1 \hookrightarrow C \subset \PP^d$ be a rational normal curve. Show that the normal bundle $\cN_{C/\PP^d}$  is 
$$
\cN_{C/\PP^d} \cong \bigoplus_{i=1}^{d-1} \cO_{\PP^1}(d+2).
$$
\end{exercise}

\begin{exercise}
In the situation of the preceding problem, the set  of direct summands of $\cN_{C/\PP^d} $ is a projective space $\PP^{d-2}$. How does the  group of automorphisms of $\PP^d$ carrying $C$ to itself act on this $\PP^{d-2}$?

\end{exercise}

\input footer.tex


