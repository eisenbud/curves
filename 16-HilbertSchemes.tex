%header and footer for separate chapter files

\ifx\whole\undefined
\documentclass[12pt, leqno]{book}
\usepackage{graphicx}
\input style-for-curves.sty
\usepackage{hyperref}
\usepackage{showkeys} %This shows the labels.
%\usepackage{SLAG,msribib,local}
%\usepackage{amsmath,amscd,amsthm,amssymb,amsxtra,latexsym,epsfig,epic,graphics}
%\usepackage[matrix,arrow,curve]{xy}
%\usepackage{graphicx}
%\usepackage{diagrams}
%
%%\usepackage{amsrefs}
%%%%%%%%%%%%%%%%%%%%%%%%%%%%%%%%%%%%%%%%%%
%%\textwidth16cm
%%\textheight20cm
%%\topmargin-2cm
%\oddsidemargin.8cm
%\evensidemargin1cm
%
%%%%%%Definitions
%\input preamble.tex
%\input style-for-curves.sty
%\def\TU{{\bf U}}
%\def\AA{{\mathbb A}}
%\def\BB{{\mathbb B}}
%\def\CC{{\mathbb C}}
%\def\QQ{{\mathbb Q}}
%\def\RR{{\mathbb R}}
%\def\facet{{\bf facet}}
%\def\image{{\rm image}}
%\def\cE{{\cal E}}
%\def\cF{{\cal F}}
%\def\cG{{\cal G}}
%\def\cH{{\cal H}}
%\def\cHom{{{\cal H}om}}
%\def\h{{\rm h}}
% \def\bs{{Boij-S\"oderberg{} }}
%
%\makeatletter
%\def\Ddots{\mathinner{\mkern1mu\raise\p@
%\vbox{\kern7\p@\hbox{.}}\mkern2mu
%\raise4\p@\hbox{.}\mkern2mu\raise7\p@\hbox{.}\mkern1mu}}
%\makeatother

%%
%\pagestyle{myheadings}

%\input style-for-curves.tex
%\documentclass{cambridge7A}
%\usepackage{hatcher_revised} 
%\usepackage{3264}
   
\errorcontextlines=1000
%\usepackage{makeidx}
\let\see\relax
\usepackage{makeidx}
\makeindex
% \index{word} in the doc; \index{variety!algebraic} gives variety, algebraic
% PUT a % after each \index{***}

\overfullrule=5pt
\catcode`\@\active
\def@{\mskip1.5mu} %produce a small space in math with an @

\title{Personalities of Curves}
\author{\copyright David Eisenbud and Joe Harris}
%%\includeonly{%
%0-intro,01-ChowRingDogma,02-FirstExamples,03-Grassmannians,04-GeneralGrassmannians
%,05-VectorBundlesAndChernClasses,06-LinesOnHypersurfaces,07-SingularElementsOfLinearSeries,
%08-ParameterSpaces,
%bib
%}

\date{\today}
%%\date{}
%\title{Curves}
%%{\normalsize ***Preliminary Version***}} 
%\author{David Eisenbud and Joe Harris }
%
%\begin{document}

\begin{document}
\maketitle

\pagenumbering{roman}
\setcounter{page}{5}
%\begin{5}
%\end{5}
\pagenumbering{arabic}
\tableofcontents
\fi


\chapter{Hilbert Schemes I: Examples}
\label{HilbertSchemesChapter}

In earlier Chapters, we described some  degree $d$ embeddings curves of  genus $g$ in projective spaces for small $d,g$. In this chapter, we'll try to describe the family of all such curves in $\PP^3$---that is, the Hilbert schemes $\cH  := Hilb_{dm-g+1}$.

We will be primarily interested in 
the open subset parametrizing smooth, irreducible, nondegenerate curves $C \subset \PP^r$ (sometimes called the \emph{restricted Hilbert scheme}), which we denote by $\cH^\circ \subset \cH$ . 


Three basic questions about the schemes $\cH^\circ$ are:

\begin{enumerate}
\item[$\bullet$] Is $\cH^\circ$ irreducible? and
\item[$\bullet$]  What is its dimension or dimensions?
\item[$\bullet$] Where is it smooth, and where is it singular?
\end{enumerate}

Of course, there are many more questions about the geometry of $\cH^\circ$: for example,  what is the closure $\overline{\cH^\circ} \subset \cH$ in the whole Hilbert scheme? (In other words, when is a subscheme $X \subset \PP^r$ with Hilbert polynomial $dm-g+1$ \emph{smoothable}, in the sense that it is the flat limit of a family of smooth curves?) What is the Picard group of $\cH^\circ$ or of its closure? We will for the most part not address these, though we will indicate the answers in special cases.

In Examples~\ref{Hilb for plane curves} and \ref{Hilb for plane curves-continued} we discussed the (deceptively simple) Hilbert scheme of plane curves. In this chapter we will work with the next case, nondegenerate curves in $\PP^3$. 

\section{Degree 3}

The first case to consider is that of the Hilbert scheme containing twisted cubics, $Hilb_{3m+1}(\PP^3).$

In Proposition~\ref{hilb of twisted cubics} we showed that the open set of smooth twisted cubics has dimension 12,
and we gave another proof, based on linkage, in Chapter~\ref{LinkageChapter}. 
By Exercise~\ref{twisted cubic normal bundle}, the normal bundle of a twisted cubic $C$ is $\sN_{C/\PP^3}=\sO_{\PP^1}(5)\oplus \sO_{\PP^1}(5)$
so by Theorem~\ref{tangent space of Hilb} the tangent space to the Hilbert scheme at $C$ is
$H^0(\sN_{C/\PP^3}) = \CC^12$, so $Hilb$ is smooth at this point.

\subsection{The other component}

There is  one other component of $Hilb_{3m+1}(\PP^3)$, which we think of as an \emph{extraneous component}; that is, a component whose general point does \emph{not} correspond to a smooth, irreducible nondegenerate curve. 

\begin{theorem}
$Hilb_{3m+1}(\PP^3)$ has two irreducible components. One is 12-dimensional with generic point corresponding to  a twisted cubic,
while the other is 15-dimensional with generic point corresponding to the union of a smooth plane cubic and a point outside the plane.
\end{theorem}

\begin{proof}
Let $C'$ be the purely 1-dimensional scheme defined by the intersection of the primary ideals in the decomposition of $I_C$ that correspond to
components of dimension 1. Note that if the hilbert polynomials of $C$ and $C'$ are $p(m)$ and  $p'(m)$ then
$p(m) \geq p'(m)$ for all large $m$, and equality for large $m$ implies that $C'=C$.

Since $\deg C' = \deg C = 3$, the curve $C'$ is either irreducible or the union of two or three irreducible components. In the first case $C'$ is either nondegenerate, in which case it is a twisted cubic by Theorem~\ref{characterization of P1} and $C' = C$; or a plane cubic. If $C'$ is a plane cubic, then it has Hilbert function $3m$, so $\sI_{C'}/\sI_C$
corresponds to a point in $\PP^3$, either embedded in $C'$ or not. Such a union is specified by the choice of the
plane, the cubic in it, and the point, and thus has dimension $3 + 9+3 = 15.$

On the other hand, if $C$ is not planar and not irreducible, then $C'$ consists of 3 lines or the union of a (planar) conic
and a line not in the plane. In this case each connected component has arithmetic genus 0 and thus Hilbert polynomial
with constant term 1; so the curve must be connected. All such curves can be realized as divisors of type $(1,2)$
on a quadric, and thus have Hilbert function equal to that of $C$, whence again $C' = C$.
\end{proof}


In~\cite{Piene-Schlessinger} it is also shown that each component individually is smooth and rational and the intersection,
which is 11-dimensional with generic point corresponding to a plane nodal cubic (see Exercise~\ref{hilb intersection} with an embedded point at the node, is also
smooth and rational. 


\section{Extraneous components}

 Let $\cH = \cH_d(\PP^n)$ be the Hilbert scheme of subschemes of $\PP^n$ with Hilbert polynomial the constant $d$, with open subset $\cH^\circ \subset \cH$ whose points correspond to reduced $d$-tuples of points in $\PP^n$. This open subset is the complement of the diagonal in the $d$th symmetric power of $\PP^n$. We call the closure of this open set the \emph{principal component} of $\cH$. It has dimension $dn$.

Iarrobino in \cite{Iarrobino1985} first proved that, for any $n \geq 3$ and any sufficiently large $d$, there are components of $\cH_d(\PP^n)$ having dimension strictly larger than $dn$; see Exercise~ref{bigger component}. There are also examples of extraneous components
of dimension $<dn$. No one knows how many irreducible components the Hilbert scheme $\cH = \cH_d(\PP^n)$ has, or what their dimensions might be.

This in turn infects the Hilbert schemes of curves. For example, the Hilbert scheme $\cH_{dm-g+1}$ parametrizing curves of degree $d$ and genus $g$ in $\PP^3$, has a component whose general point corresponds to a union of a plane curve of degree $d$ and $\binom{d-1}{2} - g$ points; moreover, if $\Gamma$ is any irreducible component of the Hilbert scheme of zero-dimensional subschemes of degree $\binom{d-1}{2} - g$ in $\PP^3$, there is a component of $\cH_d(\PP^n)$ whose  general point corresponds to a union of a plane curve of degree $d$ and the subscheme corresponding to a general point of $\Gamma$. Further, we can replace the plane curves in this construction with any component of the Hilbert scheme of curves of degree $d$ and genus $g' > g$. There can also be components of $\cH_{dm-g+1}$ whose general point corresponds to a subscheme of $\PP^3$ with an embedded point---see~\cite{Chen-Nollet}

we don't know (see the paper by Dawei Chen and Scott Nollet, at https://arxiv.org/abs/0911.2221).

Bottom line, it's a mess. For many $g,d$ the Hilbert scheme $\cH_{dm-g+1}(\PP^3)$ has many components. In most cases no one knows how many, or what their dimensions are.

\section{Degree 4}

By Clifford's Theorem  an irreducible nondegenerate curve of degree 4 in $\PP^{3}$ must have genus 0 or 1; we consider these cases in turn.

\subsection{Genus 0}\label{degree 4 genus 0}

We can deal with rational quartics by a slight variant of the first method we used to deal with twisted cubics. A rational curve of degree 4 is the image of a map $\phi_F : \PP^1 \to \PP^3$ given by a four-tuple $F = (F_0,F_1,F_2,F_3)$ with $F_i \in H^0(\cO_{\PP^1}(4))$. The space of all such four-tuples up to scalars is a projective space of dimension $4 \times 5 - 1 = 19$; let $U \subset \PP^{19}$ be the open subset of four-tuples such that the map $\phi$ is a nondegenerate embedding. We then have a surjective map $\pi : U \to \cH^\circ$, whose fiber over a point $C$ is the space of maps with image $C$. Since any two such maps differ by an automorphism of $\PP^1$---that is, an element of $\PGL_2$---the fibers of $\pi$ are three-dimensional; we conclude that \emph{$\cH^\circ_{0,3,4}$ is irreducible of dimension 16}.

The same analysis can be used on rational curves of any degree $d$: the space $U$ of nondegenerate embeddings $\PP^1 \to \PP^3$ of degree $d$ is an open subset of the projective space $\PP^{4(d+1)-1}$ of four-tuples of homogeneous polynomials of degree $d$ on $\PP^1$ modulo scalars; and the fibers of the corresponding map $U \to \cH^\circ_{dm+1}$ are copies of $\PGL_2$. This yields the

\begin{proposition}\label{dimension of rational curves}
The open set $\cH^\circ \subset \cH_{0,3,d}$ parametrizing smooth, irreducible nondegenerate rational curves $C \subset \PP^3$ is irreducible of dimension $4d$.
\end{proposition}

\begin{exercise}
Give an argument for Proposition~\ref{dimension of rational curves} in case $d=4$ using linkage. 
\end{exercise}

One further remark. Following our discussion of twisted cubics, we were able to see in Exercise~\ref{twisted cubic normal bundle} that the restricted Hilbert scheme of twisted cubics is smooth by identifying the normal bundle of a twisted cubic and determining the dimension of its space of global sections. In fact, the same is true for rational curves of any degree, as the following exercise shows.

\begin{exercise}
Let $C \cong \PP^1 \subset \PP^3$ be a smooth rational curve of any degree $d$. 
\begin{enumerate}
\item Show that $h^1(\cN_{C/\PP^3}) = 0$; that is, the normal bundle of $C$ is nonspecial.
\item Using this, the Riemann-Roch formula for vector bundles on a curve and Proposition~\ref{dimension of rational curves}, show that the Hilbert scheme $\cH$ is smooth at the point $[C]$.
\end{enumerate} 
\end{exercise}

We should point out that, in contrast to the case of twisted cubics, smooth rational curves in $\PP^r$ of the same degree may have different normal bundles. This gives an interesting stratification of the restricted Hilbert scheme of rational curves; see \cite{Riedl} for a discussion.

\subsection{Genus 1}
 As we saw in Section~\ref{}, a quartic curve $C \subset \PP^3$ of genus 1 is the intersection of two quadric surfaces, and by Lasker's theorem, every quadric containing $C$ is a linear combination of those two. Conversely, the intersection of two general quadrics in $\PP^3$ is a quartic curve of genus 1. We can thus construct a family of quartics of genus 1: let $V = H^0(\cO_{\PP^3}(2))$ be the 10-dimensional vector space of homogeneous quadric polynomials in $\PP^3$ and $G(2,V)$ the Grassmannian of 2-planes in $V$, and consider the incidence correspondence
$$
\Gamma = \{ (\Lambda, p) \in G(2,V) \times \PP^3 \mid F(p)=0 \; \forall \; F \in \Lambda \}.
$$
The fiber of $\Gamma$ over a point $\Lambda \in G(2,V)$ is thus the base locus of the pencil of quadrics represented by $\Lambda$; let $B \subset G(2,V)$ be the Zariski open subset over which the fiber is smooth, irreducible and nondegenerate of dimension 1. By the universal property of Hilbert schemes, the family $\pi_1 : \Gamma_B \to U$ induces a map $\phi : B \to \cH^\circ$ that is one-to-one on points; it follows that the reduced subscheme of $\cH^\circ$ is birational to an open subset of the Grassmannian $G(2,10)$, and we conclude that \emph{$\cH^\circ_{1,3,4}$ is irreducible of dimension 16}. Exercise~\ref{hilb 1,3,4} shows that this map is actually an isomorphism.

\begin{exercise}\label{hilb 1,3,4}
Let $C = Q \cap Q' \subset \PP^3$ be a smooth curve of degree 4 and genus 1. Identify the normal bundle $\cN_{C/\PP^3}$ of $C$, and use this to conclude that $\cH^\circ_{1,3,4}$ is itself reduced, and even smooth, and thus isomorphic to an open subset of the Grassmannian $G(2,10)$.
\end{exercise}

The  argument  here---where we constructed a family $\cC \to B$ of curves of given type, and then invoked the universal property of the Hilbert scheme to get a map $B \to \cH$ is typical in analyses of Hilbert schemes. Here here are two slightly more general cases:

\begin{exercise}\label{complete intersection open}
Let $m \geq n >0$ be two positive integers. Show that the locus $U_{n,m} \subset \cH^\circ$ of curves $C \subset \PP^3$ that are smooth complete intersections of surfaces of degrees $n$ and $m$ is an open subset of the Hilbert scheme.
\end{exercise}

\begin{exercise}\label{first complete intersection exercise}
Consider  the locus $U_{n,n} \subset \cH^\circ$ of curves $C \subset \PP^3$ that are smooth complete intersections of two surfaces of degrees $n$. Show that $U_{n,n}$ 
is isomorphic to an open subset of the Grassmannian $G(2, H^0(\cO_{\PP^3}(n))$.
\end{exercise}

\section{Degree 5}

Let $C \subset \PP^3$ be a smooth, irreducible, nondegenerate quintic curve of genus $g$. By Clifford's theorem the bundle $\cO_C(1)$ must be nonspecial, so  by the Riemann-Roch theorem we must have $0\leq g \leq 2$. We have already seen that the space $\cH^\circ_{5m+1}$ of rational quintic curves is irreducible of dimension 20. We will treat the case $g=2$ in detail, and leave the case $g=1$ as an exercise. This case will be covered in a different way in Section~\ref{estimating dim hilb}.

\subsection{Genus 2}

We have considered curves of genus 2 in Section~\ref{}.  To recap the analysis, let $C \subset \PP^3$ be a smooth, irreducible, nondegenerate curve of degree 5 and genus 2. By the Riemann-Roch theorem,  $h^0(\cO_C(2)) = 10-2+1 = 9<h^0(\cO_{\PP^3}(2)) = 10$  so the restriction map
$$
H^0(\cO_{\PP^3}(2)) \to H^0(\cO_C(2))
$$
has a kernel. Since $\deg C = 5 > 2\times 2$, the curve $C$ cannot lie on two independent quadrics; thus $C$ lies on a unique quadric surface $Q$. Similarly, the restriction map
$$
H^0(\cO_{\PP^3}(3)) \to H^0(\cO_C(3))
$$
has at least a 6-dimensional kernel; since cubics of the form $LQ$ span only a 4-dimensional space, we see that $C$ lies on a cubic surface $S$ not containing $Q$. The intersection $Q\cap S$
has degree 6, and is thus the union of $C$ and a line. If $Q$ is smooth then, in terms of the isomorphism $Q \cong \PP^1 \times \PP^1$, we can say $C$ is a curve of type $(2,3)$ on the quadric $Q$. Note that conversely if $L \subset \PP^3$ is a line and $Q$ and $S \subset \PP^3$ are general quadric and cubic surfaces containing $L$, and if we write
$$
Q \cap S = L \cup C
$$ 
then the curve $C$ is a curve of type $(2,3)$ on the quadric $Q$ and hence, by the adjunction formula,
 a quintic of genus 2.

This suggests two ways of describing the family $\cH^\circ \subset \cH_{5m-1}$ of such curves. First, we can use the fact that $C$ is linked to a line to make an incidence correspondence
$$
\Psi = \{ (C, L, Q, S) \in \cH^\circ \times \GG(1,3) \times \PP^9 \times \PP^{19} \; \mid \; Q \cap S = C \cup L \},
$$
where the $\PP^9$ (respectively, $\PP^{19}$) is the space of quadric (respectively, cubic) surfaces in $\PP^3$. Given a line $L \in \GG(1,3)$, the space of quadrics containing $L$ is a $\PP^6$, and the space of cubics containing $L$ is a $\PP^{15}$; thus the fiber of the projection $\pi_2 : \Psi \to \GG(1,3)$ over $L$ is an open subset of $\PP^6 \times \PP^{15}$, and we see that \emph{$\Psi$ is irreducible of dimension $4 + 6 + 15 = 25$}.

On the other hand, the fiber of $\Psi$ over a point $C \in \cH^\circ$ is an open subset of the $\PP^5$ of cubics containing $C$; and we conclude that \emph{$\cH^\circ$ is irreducible of dimension $20$}.

\begin{exercise}
Let $C \subset \PP^3$ be a smooth curve of degree 5 and genus 2, and assume that the quadric surface $Q$ containing $C$ is smooth. From the exact sequence
$$
0 \to \cN_{C/Q} \to  \cN_{C/\PP^3} \to  \cN_{Q/\PP^3}|_C \to 0,
$$
calculate $h^0()$ and deduce that \emph{$\cH^\circ_{2,3,5}$ is smooth at the point $[C]$}. Does  this conclusion still hold if $Q$ is singular?
\end{exercise}

Another, in some ways more direct, approach to describing the restricted Hilbert scheme $\cH^\circ_{2,3,5}$ would be to use the fact that the quadric surface $Q$ containing a quintic curve $C \subset \PP^3$ of genus 2 is unique. We thus have a map
$$
\cH^\circ \to \PP^9,
$$
whose fiber over a point $Q \in \PP^9$ is the space of quintic curves of genus 2 on $Q$. 

The problem is, the space of quintic curves of genus 2 on a given quadric $Q$ is not in general irreducible: for a general, and thus smooth quadric $Q$ it consists of the disjoint union of the open subsets of smooth elements in the two linear series of curves of type $(2,3)$ and $(3,2)$ on $Q$, each of which is a $\PP^{11}$. We can conclude immediately that $\cH^\circ$ is of pure dimension 20; but to conclude that it is irreducible we need to verify that, in the family of all smooth quadric surfaces, the monodromy exchanges the two rulings. \fix{ refer to the place---earlier---where monodromy is discussed, and say this follows from the irreducibility of an appropriately modified incidence correspondence. Do this example where the monodromy if first discussed, too.} This is not hard: it amounts to the assertion that the family
$$
\Gamma = \{ (Q,L) \in \PP^9 \times \GG(1,3) \; \mid \; L \subset Q \}
$$
is irreducible, which can be seen via projection on the second factor.

%There is another approach to the problem of describing $\cH^\circ$, which is to describe such curves parametrically rather than via the equations defining them as subsets of $\PP^3$, which is a direct generalization of the approach we took to the proof of Proposition~\ref{dimension of rational curves} above. We'll describe this in general in Section~\ref{estimating dim hilb}. It covers the case of quintics of genus 1, so we won't deal with that case separately, except in the form of an exercise:

\begin{exercise}
Show that a smooth, irreducible, nondegenerate curve $C \subset \PP^3$ of degree 5 and genus 1 is residual to a rational quartic in the complete intersection of two cubics, and use the result of subsection~\ref{degree 4 genus 0} to deduce that the space of genus 1 quintics is irreducible of dimension 20.
\end{exercise}

\section{Degree 6}

Again the Clifford and Riemann-Roch theorems suffice to compute the possible genera of a curve of degree 6. To start with,  if the line bundle $\cO_C(1)$ is nonspecial, then by the Riemann-Roch theorem we have $g \leq 3$. Suppose on the other hand that $\cO_C(1)$ is special. Since   $h^{0}(\cO_C(1)) \geq 4$, we have equality in Clifford's theorem, and either $C$ is hyperelliptic and $\cO_C(1)$ is a multiple of the $g^{1}_{2}$ or  $C$ is  a canonically embedded curve of genus 4. The first case cannot occur, since no special multiple of the hyperelliptic series of degree $\leq 2g-2$ can be very ample; thus $C$ must be a canonical curve of genus 4. In sum, by applying Clifford's Theorem and the Riemann-Roch Theorem, we see that a smooth irreducible, nondegenerate curve of degree 6 in $\PP^3$ has genus at most 4.

\begin{exercise}
\begin{enumerate}
\item Show that all genera $g \leq 4$ do occur; that is, there exists a smooth irreducible, nondegenerate curve of degree 6 and genus $g$ in $\PP^3$ for all $g \leq 4$.
\item What is the largest possible genus of a smooth irreducible, nondegenerate curve $C \subset \PP^3$ of degree $d=7$? Can you do this with Clifford and Riemann-Roch, or do you need to invoke Castelnuovo?
\end{enumerate}
\end{exercise}

The cases of genera 0, 1 and 2 are covered under Proposition~\ref{nonspecial Hilbert}, leaving us the cases $g = 3$ and 4. Both are well-handled by the Cartesian approach of describing their ideals.

\subsection{Genus 4}

As we've seen in Section\ref{????} a canonical curve of genus 4 is the complete intersection of a (unique) quadric $Q$ and a cubic surface $S$. We thus have a map
$$
\alpha : \cH^\circ \rTo \PP^9
$$
sending a curve $C$ to the quadric $Q$ containing it. Moreover, the fibers of this map are open subsets of the projective space $\PP V$, where $V$ is the quotient
$$
V = \frac{H^0(\cO_{\PP^3}(3))}{H^0(\cI_{Q/\PP^3}(3))}
$$
of the space of all cubic polynomials modulo cubics containing $Q$. Since this vector space has dimension 16, the fibers of $\alpha$ are irreducible of dimension 15, and we deduce that \emph{the space $\cH^\circ_{6m-3}$ is irreducible of dimension 24}.

In fact, Exercise~\ref{first complete intersection exercise} can be generalized in this way to smooth complete intersections of surfaces of any degree:

\begin{exercise}\label{second complete intersection exercise}
As before, let $U_{n,m} \subset \cH^\circ$ be the locus of curves $C \subset \PP^3$ that are smooth complete intersections of surfaces of degrees $n$ and $m$.
 In case $m > n$, show that $U_{m,n}$ is isomorphic to an open subset of a projective bundle over the projective space $\PP(H^0(\cO_{\PP^3}(n))) \cong \PP^{\binom{n+3}{3}-1}$ of surfaces of degree $n$, with fiber over the point $[S] \in \PP(H^0(\cO_{\PP^3}(n)))$ the projective space $\PP(H^0(\cO_{\PP^3}(m))/H^0(\cI_{S/\PP^3}(m)) \cong \PP^{\binom{m+3}{3} - \binom{m-n+3}{3} - 1}$.
\end{exercise}


\subsection{Genus 3}
We leave this to the reader to complete as follows:

\begin{exercise}
Let $C$ be a curve of degree 6 and genus 3, and assume that $C$ does not lie on any quadric surface. Show that $C$ is residual to a twisted cubic in the complete intersection of two cubic surfaces, and use this to deduce that the space of such curves is irreducible of dimension 24.
\end{exercise}


\begin{exercise}
Now let $C$ again be a curve of degree 6 and genus 3, but now assume that $C$ \emph{does} lie on a quadric surface $Q$. Show that such a curve is a flat limit of curves of the type described in the last exercise, and conclude that $\cH^\circ_{3,3,6}$ is irreducible of dimension 24. (Hint: Let $L$, $Q$ and $F$ denote a general linear form, a general quadratic form and a general cubic form, and consider the pencil of surfaces $S_t = V(tF + LQ) \subset \PP^3$ specializing from the cubic surface $V(F)$ the to reducible cubic $V(LQ)$.)

\end{exercise}



\section{Why  $4d$?}\label{estimating dim hilb}

The sharp-eyed reader will have noticed that, in every case analyzed so far,  the Hilbert scheme parametrizing smooth curves of degree $d$ and genus $g$ in $\PP^3$ has dimension $4d$. While this is not the case in general (we will see shortly an example where it fails), $4d$ is indeed the ``expected dimension'' from certain points of view. In the following subsections we'll describe two such computations. For the remainder of this section, we will step outside $\PP^3$ and consider, more generally, the restricted Hilbert scheme $\cH^\circ$ of smooth, irreducible, nondegenerate curves in $\PP^r$.

\subsection{Estimating $\dim \cH^\circ$ by Brill-Noether}

One method of estimating  the dimension of $\cH^\circ$ is a generalization of the proof of Proposition~\ref{dimension of rational curves}, with two additional wrinkles: First, since not all line bundles of degree $d$ on a curve $C$ of genus $g > 0$ are linearly equivalent, we must invoke the Picard variety $\Pic_d(C)$ parametrizing line bundles of degree $d$ on a given curve $C$, discussed in Chapter~\ref{new Jacobians chapter}. Second, since not all curves of genus $g > 0$ are isomorphic, we must involve the moduli space  $M_g$ parametrizing abstract curves of genus $g$, discussed in Chapter~\ref{Moduli chapter}.

To begin with a simple example, let $\cH^\circ$ again be the space of smooth, irreducible, nondegenerate curves $C \subset \PP^3$ of degree 5 and genus 2. By the property of $M_{2}$ as a coarse moduli space, we get a map
$$
\mu : \cH^\circ \rTo M_2.
$$
To analyze the fiber $\Sigma_C =\mu^{-1}(C)$ of the map $\mu$ over a point $C \in M_2$ we first use the map
$$
\nu : \Sigma_C \rTo \Pic_5(C),
$$
obtained by sending a point in $\Sigma_C$ to the line bundle $\cO_C(1)$. Proposition~\ref{**}, implies that any line bundle of degree 5 on a curve of genus 2 is very ample, so this map is surjective. Note that 
$h^0(\cL) = 4$, so the linear series  giving the embedding is complete. Thus, once we have specified the abstract curve $C$, and the line bundle $\cL \in \Pic_5(C)$ the embedding is determined by giving a basis for $H^0(\cL)$, up to scalars. In other words, each fiber of $\nu$ is isomorphic to $\PGL_4$. We can now work our way up from $M_2$:

\begin{enumerate}

\item[$\bullet$] We know that $M_2$ is irreducible of dimension 3.

\item[$\bullet$] It follows that the space of pairs $(C,\cL)$ with $C \in M_2$ a smooth curve of genus 2 and $\cL \in \Pic_5(C)$ is irreducible of dimension 3 + 2 = 5; and finally

\item[$\bullet$] It follows that $\cH^\circ$ is irreducible of dimension $5 + 15 = 20$.

\end{enumerate}

In fact, this approach applies to a much wider range of examples: whenever $d \geq 2g+1$ and $r \leq d-g$, we can look at the tower of spaces

\begin{diagram}
\cH^\circ = \cH^\circ_{dm-g+1}(\PP^r) \\
\dTo \\
\cP_{d,g} = \{(C,\cL) \mid \cL \in \Pic_d(C) \} \\
\dTo \\
M_g.
\end{diagram}

Exactly as in the special case $(d,g,r) = (5,2,3)$ above, we can work our way up the tower:


\begin{enumerate}

\item[$\bullet$]  $M_g$ is irreducible of dimension $3g-3$;

\item[$\bullet$] it follows from the fact that the Picard variety is irreducible of dimension $g$ that $\cP_{d,g}$ is irreducible of dimension $3g-3+g = 4g-3$; and finally

\item[$\bullet$] since the fibers of $\cH^\circ \to \cP_{d,g}$ consist of $(r+1)$-tuples of linearly independent sections of $\cL$ (mod scalars), and $h^0(\cL) = d-g+1$, it follows that $\cH^\circ$ is irreducible of dimension $\dim(\cP_{d,g}) + (r+1)(d-g+1) = 4g-3 + (r+1)(d-g+1) - 1$.

\end{enumerate}

In sum, we have the

\begin{proposition}\label{nonspecial Hilbert}
Whenever $d \geq 2g+1$, the space $\cH^\circ$ of smooth, irreducible, nondegenerate curves $C \subset \PP^r$ is either empty (if $d-g < r$) or irreducible of dimension $4g-3 + (r+1)(d-g+1) - 1$; in particular, if $r=3$, the dimension of $\cH^\circ$ is $4d$.
\end{proposition}

\begin{exercise}
By analyzing the geometry of linear series of degrees $2g-1$ and $2g$ on a curve of genus $g$, extend Proposition~\ref{nonspecial Hilbert} to the cases $d = 2g-1$ and $2g$. What goes wrong if $d \leq 2g-2$?
\end{exercise}

Proposition~\ref{nonspecial Hilbert} gives a simple and clean answer to our basic questions about the dimension and irreducibility of the restricted Hilbert scheme $\cH^\circ$ in case $d \geq 2g-1$. But what happens outside of this range? In fact, we  can use Brill-Noether theory to modify this analysis to extend this beyond the range $d \geq 2g+1$.

Basically, what's different in general is that the map $\cH^\circ \to \cP_{d,g}$ is no longer dominant; rather, over a point $[C] \in M_g$, its image is open in the subvariety $W^r_d(C) \subset \Pic_d(C)$ parametrizing line bundles $\cL$ on $C$ of degree $d$ with at least $r+1$ sections. Now, as long as the Brill-Noether number $\rho(d,g,r)$ is non-negative, the Brill-Noether theorem tells us that for a general curve $C$, the variety $W^r_d(C)$ has dimension $\rho$, and (assuming $r \geq 3$) the general point of $W^r_d(C)$ corresponds to a very ample line bundle with exactly $r+1$ sections. In this situation, there is a unique component of $\cH_0 \subset \cH^\circ$ dominating $M_g$, and the map $\cH^\circ \to \cP_{d,g}$ carries this component to a subvariety $\cW^r_d \subset \cP_{d,g}$ of dimension $3g-3 + \rho$. In sum, then, we have the basic theorem

\begin{theorem}\label{principal component}
Let $g, d$ and $r$ be any nonnegative integers, with Brill-Noether number  $\rho(g,r,d) = g - (r+1)(g-d+r) \geq 0$. There is then a unique component $\cH_0$ of the restricted Hilbert scheme $\cH^\circ_{g,r,d}$ dominating the moduli space $M_g$; and this component has dimension
$$
\dim \cH_0 = 3g-3+\rho + (r+1)^2 - 1 = 4g-3 + (r+1)(d-g+1) - 1.
$$
\end{theorem}

 The component $\cH_0$ identified in Theorem~\ref{principal component} is called the \emph{principal component} of the Hilbert scheme; there may be others as well, of possibly different dimension, and we do not know precisely for which $d,g$ and $r$ these occur. Finally, in case $\rho < 0$, the Brill-Noether theorem tells us only that there is no component of $\cH^\circ_{g,r,d}$ dominating $M_g$; we'll discuss some of the outstanding questions in this range in Section~\ref{open problems} below. 

%\begin{exercise}
%Use the argument for Proposition~\ref{nonspecial Hilbert} to cover the case $d=2g$, and use this to deduce again that $\cH^\circ_{dm-2}(\PP^3)$ is irreducible of dimension 24.
%\end{exercise}

\subsection{Estimating $\dim \cH^\circ$ by the Euler characteristic of the normal bundle}

It is interesting to compare the estimate of  $\dim \cH^\circ$ above with what we get from deformation theory. Let $\cH$ be a component of the scheme $\cH^\circ$, with $C \subset \PP^r$ a curve corresponding to a general point $[C]$ of $\cH$.

We start with the idea that the dimension of the scheme $\cH$ is approximated by the dimension of its Zariski tangent space $T_{[C]}\cH$ at a general point $[C]$. In Section~\ref{???} we saw that the tangent space to $\cH$ at $[C]$ is the space $H^0(\cN_{C/\PP^r})$ of global sections of the normal bundle $\cN = \cN_{C/\PP^r}$. We can think of the dimension $h^0(\cN)$ as approximated by the Euler characteristic $\chi(\cN)$, with ``error term" $h^1(\cN)$ coming from its first cohomology group.

Given these two approximations, we arrive at a number we can  compute. From the exact sequence
$$
0 \to T_C \to T_{\PP^r}|_C \to \cN \to 0
$$
we deduce that
\begin{align*}
c_1(\cN) &= c_1(T_{\PP^r}|_C) - c_1(T_C) \\
&= (r+1)d - (2-2g).
\end{align*}

Now we can apply the Riemann-Roch Theorem for vector bundles on curves (\cite[Theorem ???]{3264}) to conclude that
\begin{align*}
\chi(\cN) &= c_1(\cN) - \rank(\cN)(g-1) \\
&= (r+1)d - (r-3)(g-1).
\end{align*}

Note that our two ``estimates" are actually inequalities. But, unfortunately, they go in opposite directions: we have
$$
\dim \cH \leq \dim T_{[C]}\cH,
$$
but 
$$
\dim T_{[C]}\cH \geq \chi(\cN).
$$
Nonetheless, one can show that if $C \subset \PP^r$ is a smooth curve then the versal deformation space of $C \subset \PP^r$ has dimension at least $\chi(\cN)$. If we consider the family of Picard varieties over the family of smooth curves in a neighborhood of $C$ and we can deduce that for any component of $\cH^\circ$ containing $C$ we have
$$
\dim \cH^\circ \; \geq \; (r+1)d - (r-3)(g-1)
$$

\subsection{They're the same!} Proposition~\ref{nonspecial Hilbert} suggests that the ``expected dimension" of the restricted Hilbert scheme $\cH^\circ$ of curves of degree $d$ and genus $g$ in $\PP^r$ should be 
$$
h(g,r,d) := 4g-3 + (r+1)(d-g+1) - 1.
$$
But the calculation immediately above suggests it should be $(r+1)d - (r-3)(g-1)$. Which is it? The answer is both: they're the same number!

\section{Exercises}
\begin{exercise}\label{twisted cubic normal bundle}
Let $C \cong \PP^1 \subset \PP^3$. Show that the normal bundle $\cN_{C/\PP^3} \cong \cO_{\PP^1}(5)^{\oplus 2}$; that is, the normal bundle of a twisted cubic is the direct sum of two line bundles of degree 5. Use this to prove that the restricted Hilbert scheme $\cH^\circ$ of twisted cubics is everywhere smooth. \fix{everywhere?  or just on the locus of smooth twisted cubics?}
\end{exercise}

Hint: Restricting the presentation matrix of $I_C$ to $C$---that is, substituting the forms of degree 3 in 2 variables for the variables in the presentation matrix---we get a presentation 
$$
R(-9)^2 \rTo^A R(-6)^3 \to I_C/I_C^2 \to 0 
$$
The kernel of the dual of $A$ is the normal bundle; it has 2 linear generators, 
$$
\begin{pmatrix}
t&0\\
-s&t\\
0&-s
\end{pmatrix}
$$
as a module over $\CC[s,t].$

Alternate hint: consider the line sub-bundle of the normal bundle obtained from the secant lines through a given point of 
$C$, and show that it has degree 5 \fix{???}.

\begin{exercise}\label{hilb intersection}
Show that the locus $\Sigma$ of schemes $X$ consisting of a nodal plane cubic curve $C$ with a spatial embedded point of multiplicity 1 at the node is dense in the intersection $\overline{\cH^\circ} \cap \overline{\cH'}$.
\end{exercise}

\begin{exercise}
 Compute the dimension of each of the following subsets of $Hilb_{3m+1}$:
 
\begin{enumerate}
 \item unions of a conic and a line meeting it in 1 point.
 \item the connected union of 3 lines not all contained in the same plane
 \item nodal plane cubics together with an embedded point at the node that is not contained in the plane of
 the cubic.
\end{enumerate}
\end{exercise}

\begin{exercise}\label{bigger component}(Iarrobino)
Show that ideals generated by $a$ independent forms of degree $d$ together with all the forms of degree $d+1$ in 3 variables
form a family of 0-dimensional schemes of degree $d:={3+d\choose 3} -a$ which can be interpreted as a subvariety
of the Hilbert scheme $Hilb_d(\PP^3)$. Show that the dimension of this subvariety is $h := a({2+d\choose 2}-a)$. Find values of
$d$ and $a$ such that $h>3d$, and conclude that the Hilbert scheme $Hilb_d(\PP^3)$ has more than one component.
\end{exercise}

%footer for separate chapter files

\ifx\whole\undefined
%\makeatletter\def\@biblabel#1{#1]}\makeatother
\makeatletter \def\@biblabel#1{\ignorespaces} \makeatother
\bibliographystyle{msribib}
\bibliography{slag}

%%%% EXPLANATIONS:

% f and n
% some authors have all works collected at the end

\begingroup
%\catcode`\^\active
%if ^ is followed by 
% 1:  print f, gobble the following ^ and the next character
% 0:  print n, gobble the following ^
% any other letter: normal subscript
%\makeatletter
%\def^#1{\ifx1#1f\expandafter\@gobbletwo\else
%        \ifx0#1n\expandafter\expandafter\expandafter\@gobble
%        \else\sp{#1}\fi\fi}
%\makeatother
\let\moreadhoc\relax
\def\indexintro{%An author's cited works appear at the end of the
%author's entry; for conventions
%see the List of Citations on page~\pageref{loc}.  
%\smallbreak\noindent
%The letter `f' after a page number indicates a figure, `n' a footnote.
}
\printindex[gen]
\endgroup % end of \catcode
%requires makeindex
\end{document}
\else
\fi



