%header and footer for separate chapter files

\ifx\whole\undefined
\documentclass[12pt, leqno]{book}
\usepackage{graphicx}
\input style-for-curves.sty
\usepackage{hyperref}
\usepackage{showkeys} %This shows the labels.
%\usepackage{SLAG,msribib,local}
%\usepackage{amsmath,amscd,amsthm,amssymb,amsxtra,latexsym,epsfig,epic,graphics}
%\usepackage[matrix,arrow,curve]{xy}
%\usepackage{graphicx}
%\usepackage{diagrams}
%
%%\usepackage{amsrefs}
%%%%%%%%%%%%%%%%%%%%%%%%%%%%%%%%%%%%%%%%%%
%%\textwidth16cm
%%\textheight20cm
%%\topmargin-2cm
%\oddsidemargin.8cm
%\evensidemargin1cm
%
%%%%%%Definitions
%\input preamble.tex
%\input style-for-curves.sty
%\def\TU{{\bf U}}
%\def\AA{{\mathbb A}}
%\def\BB{{\mathbb B}}
%\def\CC{{\mathbb C}}
%\def\QQ{{\mathbb Q}}
%\def\RR{{\mathbb R}}
%\def\facet{{\bf facet}}
%\def\image{{\rm image}}
%\def\cE{{\cal E}}
%\def\cF{{\cal F}}
%\def\cG{{\cal G}}
%\def\cH{{\cal H}}
%\def\cHom{{{\cal H}om}}
%\def\h{{\rm h}}
% \def\bs{{Boij-S\"oderberg{} }}
%
%\makeatletter
%\def\Ddots{\mathinner{\mkern1mu\raise\p@
%\vbox{\kern7\p@\hbox{.}}\mkern2mu
%\raise4\p@\hbox{.}\mkern2mu\raise7\p@\hbox{.}\mkern1mu}}
%\makeatother

%%
%\pagestyle{myheadings}

%\input style-for-curves.tex
%\documentclass{cambridge7A}
%\usepackage{hatcher_revised} 
%\usepackage{3264}
   
\errorcontextlines=1000
%\usepackage{makeidx}
\let\see\relax
\usepackage{makeidx}
\makeindex
% \index{word} in the doc; \index{variety!algebraic} gives variety, algebraic
% PUT a % after each \index{***}

\overfullrule=5pt
\catcode`\@\active
\def@{\mskip1.5mu} %produce a small space in math with an @

\title{Personalities of Curves}
\author{\copyright David Eisenbud and Joe Harris}
%%\includeonly{%
%0-intro,01-ChowRingDogma,02-FirstExamples,03-Grassmannians,04-GeneralGrassmannians
%,05-VectorBundlesAndChernClasses,06-LinesOnHypersurfaces,07-SingularElementsOfLinearSeries,
%08-ParameterSpaces,
%bib
%}

\date{\today}
%%\date{}
%\title{Curves}
%%{\normalsize ***Preliminary Version***}} 
%\author{David Eisenbud and Joe Harris }
%
%\begin{document}

\begin{document}
\maketitle

\pagenumbering{roman}
\setcounter{page}{5}
%\begin{5}
%\end{5}
\pagenumbering{arabic}
\tableofcontents
\fi


\chapter{Curves of genus 0 and 1}\label{genus 0 and 1 chapter}

In this chapter, we'll begin our project of describing curves in projective space with the simplest cases, that of curves of genus 0 and 1. In the case of genus 1 especially, the theory we will treat is ``simple'' because we are operating over $\CC$; the arithmetic theory occupies a major part of modern number theory. Even here, there are interesting statements to make about the geometry of their embeddings in $\PP^r$, and there are many open problems.


\section{Curves of genus 0} 

Using Theorem~\ref{characterization of P1} and the Riemann-Roch theorem, we can show that (over an algebraically closed field) $\PP^1$
is the only curve of arithmetic genus 0:

\begin{corollary}
 Every reduced irreducible curve $C$ of arithmetic genus 0 over an algebraically closed field is isomorphic to $\PP^1$.
 \end{corollary}

\begin{proof}
The curve $C$ is smooth since otherwise its normalization would have negative genus, which is absurd.
By \trr, any linear series $\cL$ of degree $d$ on $C$ has $h^0\cL \geq d+1$, so we may use Theorem~\ref{characterization of P1}
to conclude that $C\cong \PP^1$.
\end{proof}

Even images of genus 0 curves must have genus 0. The following proof works as well if we replace $\CC$ by any algebraically
closed field.

\begin{theorem}(\cite{Luroth})\label{Lueroth}
\begin{enumerate}
\item If $C\to D$ is a non-constant map of projective curves, then the geometric genus of $C$ must be at least that of $D$.
In particular, if $C$ has geometric genus 0,  
then $D$ has geometric genus 0.
 \item If $K$ is a field with $\CC \subsetneq K \subset \CC(x)$
  then $K = \CC(y)$ for some $y \in \CC(x)$. 
\end{enumerate}
\end{theorem}

\begin{proof}
\noindent 1. Normalizing, we get a map  $ \tilde C \to \tilde D$, and the first statement follows from Hurwitz' Theorem.

\medbreak

\noindent 2. The second item follows from the first: 
Let $x_1\in K\setminus \CC$ so that $\CC(x_1)$ has transcendence degree 1 over $\CC$. Thus $K$ is
an algebraic extension of $\CC(x_1)$. If $\CC(x_1) \neq K$ we can choose $f\in K\setminus \CC(x_1)$ .
Since $f$ is algebraic over $\CC(x_1)$, the field $\CC(x_1,f)$ is the field of fractions of a curve $D$.
 Applying part 1, we see that the normalization of $D$
is $\PP^1$, and thus $\CC(x_1,f) = \CC(x_2)$ for some element $x_2$.

Continuing in this way we produce a tower of fields
$\CC \subsetneq \CC(x_1)\subsetneq \CC(x_2)\cdots \subset K$. Since $\CC(x)$ is finitely generated,
it has finite degree over $\CC(x_1)$, and this bounds the degree of $\CC(x_i)$ over $\CC(x_1)$.
Thus the sequence of fields must terminate, and it can only terminate when $\CC(x_i) = K$.
\end{proof}
 
\begin{fact}
The theorem remains true if we replace $\CC$ by any  field; see for example \cite[Section 8.13]{JacobsonII} for a elementary proof.
There is an analogue for rational surfaces proven by Castelnuovo: every complex surface admitting a surjective
map from $\PP^2$ is rational (see for example \cite[Corollary V.5]{Beauville}). 
\end{fact}

\begin{fact}
Over a non-algebraically closed field, a curve $C$ of genus 0 need not have any points, or any line bundles of odd degree (since the canonical bundle $K_C$ has degree $-2$, there do necessarily exist line bundles of every even degree; thus an arbitrary curve of genus 0 is isomorphic to a conic plane curve). 
The classification of curves of genus 0 over non-algebraically closed fields is a subject that goes back to Gauss.

A smooth projective curve of genus 0 over a field $k$ is a \emph{form} of $\PP^1$ if it becomes isomorphic to $\PP^1$ after extension
of scalars to
the algebraic closure of $k$. The unique example with $k = \RR$ that is not isomorphic over $\RR$ to $\PP^1$is the conic with no $\RR$-rational points, $x^2+y^2+z^2 = 0$. 

Noncommutative algebras enter the subject of forms of of $\PP^1$ (and $\PP^n$ more generally) in a surprising way: The curve $\PP_k^1$ itself may be described as the scheme of left ideals of $k$-vector-space dimension 2 in the ring of
$2\times 2$ matrices over $k$ (such an ideal can be embedded in the matrix ring as a linear combination of the 2 columns in an appropriate sense). More generally, any scheme that is a form of $\PP^1$ over $k$
may be described as the scheme of 2-dimensional left ideals in a 4-dimensional central simple ($=$ Azumaya) algebra over $k$. For example, the
conic $x^2+y^2+z^2 = 0$ with no points over $\RR$ is the scheme of left ideals in the algebra of quaternions. See~\cite[Section X.6]{Serre1979}.
\end{fact}

\section{Rational Normal Curves}\label{rational normal curves section}


\subsubsection{The equations defining a rational normal curve}

Choosing a basis $s,t$ for the linear forms on $\PP^1$, we can write
$$
\phi_d : (s,t) \mapsto (s^d, s^{d-1}t,\dots t^d)
$$
from which we see that the image $C$ of $\phi_d$ lies in the zero locus of the homogeneous quadratic polynomial $x_i x_j - x_{i+1}x_{j-1}$ for every $i+j$. We can realize these quadratic forms as the $2\times 2$ minors of the matrix
$$
M \; = \; \begin{pmatrix}
x_0 & x_1 & \dots & x_{d-1} \\
x_1 & x_2 & \dots & x_d
\end{pmatrix}.
$$
Note that if we substitute $s^it^{(d-i)}$ for $x_i$ and identify $H^0(\cO_{\PP^1}(i))$ with $\CC[s,t]_i$, this becomes the multiplication table
$$
H^0(\cO_{\PP^1}(i)) \times H^0(\cO_{\PP^1}(d-i-1)) \to H^0(\cO_{\PP^1}(d));
$$

It is easy to see that $C$ is set-theoretically defined by the $2\times 2$ minors of $M$: the affine set $s=1$ in $\PP^1$ maps
to the affine set $x_0 = 1$ in $\PP^d$, and the affine form of the map is $t \mapsto (t, t^2, \dots, t^d)$. But if $x_1 = t$ then from 
the equations $x_0x_i = x_1x_{i-1}$ we see successively that $x_i = t^i$; that is, the vanishing of the minors of $M$ at a point $p$
implies that $p$ lies on $C$.

A much stronger statement is true:

\begin{proposition}\label{RNC generators} The homogeneous ideal of the rational normal curve 
$$
\PP^1 \rOnto C \subset \PP^d\quad (s,t) \mapsto (s^d, s^{d-1}t, \dots t^d)
$$ 
 is generated by the ideal $I$ of
 $2\times 2$ minors of $M$.
  \end{proposition}
  
 By way of notation, we write
 $\CC[s,t]_{(d)}$ for the subring of $\CC[s,t]$ generated by all forms whose degree is a multiple of $d$. 
 Since any monomial of degree $nd$ is a product of $n$ monomials of degree $d$, the maps
 $$
 H^0(\sO_{\PP^d}(n)) \to H^0(\sO_{\PP^1}(nd)) = \CC[s,t]_{nd}
 $$
 corresponding to $\phi_d$ is surjective, so $\CC[s,t]_{(d)}$ is the homogeneous coordinate ring of $C$,
 and the Proposition amounts to saying that $\CC[x_0,\dots, x_d]/I \cong \CC[s,t]_{(d)}$.
 
\begin{proof}
Let $I\subset \CC[x_0,\dots, x_d]$ be the ideal generated by the $2\times 2$ minors of $M$,
and write ,
so that $\CC[s,t]_{(d)}$ is the homogeneous coordinate ring of $C$.

As we have seen, $I$ is  contained in the kernel of the surjection 
$$
\phi: \CC[x_0, \dots, x_d]/I \rOnto \CC[s,t]_{(d)}:\quad x_i\mapsto s^{d-i}t^i
$$
and we must show that this is a monomorphism, or equivalently that the source of $\phi$ in degree $n$ has
(at most) the same dimension $nd+1$ as the target of $\phi$ in degree $nd$.

If $0<i\leq j<d$ then 
$$
x_ix_j = x_{i-1}x_{j+1} \mid \mod I.
$$
Thus every monomial in the $x_i$ of degree $t$ is equivalent, modulo $I$, to a monomial of the form
 $$
 x_0^ax_1^{\epsilon_1}\cdots x_{d-1}^{\epsilon_{d-1}}x_d^b\leqno(*)
 $$
 where at most on $\epsilon_i$ is 1, and the rest are 0. There are $n+1$ such elements of degree $n$ with all the $\epsilon_i = 0$
 and $n(d-1)$ elements with one of the $\epsilon_i = 1$. Thus there are $nd+1$ such elements in all, proving that $\phi$ is
 an isomorphism.
  \end{proof}


\begin{corollary}\label{forms vanishing on the RNC}
The dimension of the degree $n$ part of the homogeneous ideal of the rational normal curve of degree $d$ is
$$
H^0(\sI_{C/\PP^d}(n)) = \binom{n+d}{n} - (nd+1).
$$
and in particular that the $\binom{d}{ 2}$ minors of the matrix $M$ are linearly independent.
\end{corollary}

\begin{proof}
The homogeneous coordinate ring of the rational normal curve is $\CC[s,t]_{(d)}\subset \CC[s,t]$. Comparing the dimension
of $\CC[x_0,\dots,x_d]_n$ with the dimension of $\CC[s,t]_{nd}$ gives the result.
\end{proof}

The number of quadrics containing the rational normal curve is extremal:

\begin{proposition}\label{rnc on most quadrics}
If $C \subset \PP^d$ is any irreducible, nondegenerate curve, then
$$
h^0(\cI_{C/\PP^d}(2)) \leq  \binom{d}{2};
$$
and if equality holds then $C$ is a rational normal curve
\end{proposition}

\begin{proof}
Consider the restriction of the quadrics containing $C$ to a general hyperplane $H \cong \PP^{d-1} \subset \PP^d$, and let $\Gamma = H \cap C$. We have an exact sequence:
$$
0 \to \cI_{C/\PP^d}(1) \to \cI_{C/\PP^d}(2) \to \cI_{\Gamma/\PP^{d-1}}(2) \to 0.
$$ 
Since $C$ is nondegenerate, $h^0(\cI_{C/\PP^d}(1)) = 0$, and since $\deg C \geq d$, the hyperplane section $\Gamma$ of $C$ must contain at least $d$ linearly independent points. Since linearly independent points impose independent conditions on quadrics, we have
$$
h^0(\cI_{\Gamma/\PP^{d-1}}(2)) \leq h^0(\cO_{\PP^{d-1}}(2)) - d,
$$
establishing the desired inequality. 

To prove that the equality can be achieved only with the rational normal curve, note that because  $\Gamma\subset \PP^{d-1}$
is a general hyperplane section, it
contains a collection of $d$  points $\Gamma'$ that span $\PP^{d-1}$. Suppose that $\Gamma$ contained another point $p$.
It suffices to show that $p$ imposes a nontrivial vanishing condition of quadrics containing $\Gamma'$.
The intersection of the $d$ hyperplanes in $\PP^{d-1}$ that are spanned by subsets of $d-1$ of the
points of $\Gamma'$ is empty, so the span of one of these subsets $\Gamma''$ does not contain $p$. The union of 
the span of $\Gamma''$ with a general hyperplane containing the point $\Gamma' \setminus \Gamma''$
is a quadric containing $\Gamma'$ but not $p$, as required. 

Thus, in the case of equality, $\Gamma$ consists of $d$ points, so $\deg C = d$.
Corollary~\ref{minimal degree curves} implies that $C$ is the rational normal curve.
\end{proof}

In fact, for each $m$, a rational normal curves lies on more hypersurfaces of each degree $m$ than any other irreducible, nondegenerate curve in $\PP^d$; see Exercise~\ref{extremal m-ics}. 

\subsubsection{Rational normal curves are projectively homogeneous}

Another important property of rational normal curves $C \subset \PP^d$ is that they are \emph{projectively homogeneous} in the sense that the subgroup $G$ of the automorphism group $PGL_{d+1}$ of automorphisms of $\PP^d$ that carries $C$ to itself acts transitively on $C$. More generally,

\begin{proposition}\label{Veronese is projectively homogeneous}
If $X\subset \PP^n$ is the image of $\PP^r$ by the Veronese map corresponding to $|\sO_{\PP^r}(d)|$, then $X$ is projectively homogeneous.
\end{proposition}
\begin{proof}
First, $\PP^r$ is a homogeneous variety in the sense that $\Aut \PP^r$ acts transitively. If $\sigma$ is an automorphism then,
 because $\cO_{\PP^r}(d)$ is the unique
invertible sheaf of degree $d$ on $\PP^r$,  we have $\sigma^*\cO_{\PP^r}(d) = \cO_{\PP^r}(d)$ so $\sigma$ induces an automorphism $\phi$ on $H^0(\cO_{\PP^r}(d))$, and an automorphism $\overline \phi$ on the ambient space $\PP H^0(\cO_{\PP^r}(d))$ of the target of the $d$-th Veronese map. If $\alpha$
is a rational function with divisor $D$, then $\phi(\alpha) = \alpha\circ \sigma$ has divisor $\sigma^{-1}(D)$, so $\overline\phi^{-1}$ induces $\sigma$ on $\PP^r$. 
\end{proof}

The rational normal curve $C \subset \PP^r$ can  be characterized among irreducible, nondegenerate curves as the unique projectively homogeneous curve in $\PP^r$ (Corollary~\ref{uninflected curves}).

\subsubsection{Interpolation for rational normal curves}

Another remarkable property of rational normal curves is expressed in the following Proposition. Recall that a collection of points (or a finite subscheme)
of $\PP^n$ is said to be in \emph{linearly general position} if no $k+1$ of them lie in a $(k-1)$-plane with $k\leq n$. 

\begin{proposition}(Castelnuovo's Lemma)\label{points on rnc}
If $p_1,\dots, p_{n+3} \in \PP^n$ are any $n+3$ points in $\PP^n$ in linearly general position, then there exists a unique rational normal curve $C \subset \PP^n$ containing them.
 \end{proposition}

\begin{proof}
To start, we observe that there is an automorphism $\Phi : \PP^n \to \PP^n$ carrying the points $p_1,\dots,p_{n+1}$ to the coordinate points $(0,\dots,0,1,0,\dots,0)\in \PP^n$ and the point $p_{n+2}$ to the point $(1,1\dots,1)$.  denote the images of the remaining  point $p_{n+3}$  by $[\alpha_0,\dots,\alpha_n]$. We consider maps $\PP^1 \to \PP^n$ given in terms of an inhomogeneous coordinate $z$ on $\PP^1$ by
$$
z \mapsto \left( \frac{1}{z - \nu_0}, \frac{1}{z - \nu_1} , \dots, \frac{1}{z - \nu_n}  \right)
$$
with $\nu_0,\dots,\nu_n$ any distinct scalars. Clearing denominators, we see that the image of such a map is a rational normal curve, and it passes through the $n+1$ coordinate points of $\PP^n$, which are the images of the points $z = \nu_0, \dots, \nu_n \in \PP^1$. Moreover, the image of the point $z = \infty$ at infinity is the point $(1,1, \dots,1)$. Because the points are assumed linearly independent, the $\alpha_i$ must all be nonzero, and thus if we take  $\nu_i = -1/\alpha_i$, the image of the point $z = 0$ is $(\alpha_0,\dots,\alpha_n)$. This proves existence; we leave uniqueness to Exercise~\ref{Castelnuovo uniqueness}. 
\end{proof}

\begin{fact}
A generalization of Proposition~\ref{points on rnc} is given, with a proof that is different
even in our case, in \cite[Proposition 3.19]{Montreal}.  There is also a version replacing the set of $n+3$ distinct points by a  finite
scheme of degree $n+3$ satsfying necessary conditions, in \cite{EHLGP}.
\end{fact}

We remark that for any family of curves in projective space $\PP^r$ (for example, a component of the family of all smooth, irreducible, nondegenerate curves of degree $d$ and genus $g$ in $\PP^r$) and any integer $m$, we can ask whether there exists a curve in the family passing through $m$ given general points of $\PP^r$. This is the only example we know of curves other than complete intersections for which there is a \emph{unique} such curve.


\section{Other rational curves}

What about other rational curves in projective space? 

Any linear series $\cD$ of degree $d$ on $\PP^1$ is a subseries of the complete series $|\cO_{\PP^1}(d)|$, so any map $\phi : \PP^1 \to \PP^r$ of degree $d$ may be given as the
composition of the embedding $\phi_d : \PP^1 \to \PP^d$ of $\PP^1$ as a rational normal curve with a linear projection $\pi : \PP^d \to \PP^r$. Since the natural degree $d$ Veronese embedding of $\PP^1 = \PP(V)$ as the rational normal curve is the map
$\PP(V) \to \PP(\Sym^d(V))$, the projection is a map into $\PP(W)$, where $W\subset \Sym^d(V)$. 

We can make this more explicit in several ways: first, choosing a basis $s,t$ for $V$ and a basis of forms $F_i$ of degree $d$ on $\PP^1$ for $W$, we may write the map as
$$
(s,t) \; \mapsto \; (F_0(s,t), \dots, F_r(s,t)).
$$
If the $F_i$ have a greatest common factor $G(s,t)$ then the set $G=0$ will be the base locus
of the linear series, and the map
$$
(s,t) \; \mapsto \; ( \frac{F_0(s,t)}{G(s,t)}, \dots, \frac{F_r(s,t)}{G(s,t)}).
$$
gives the same map, represented as a linear series of lower degree $d-\deg G$. Thus we
will now assume that the forms in $W$ have no common factor.

Perhaps even simpler, we can pass to an open affine set $\AA^1$ given by $t=1$ in $\PP^1$, with coordinate function $z = s/t$ and
dehomogenize the $F_i$ to get a vector space of polynomials $f_i = F(s,1)$ of degree $\leq d$. Then we may write the
map as
$$
z \; \mapsto \; (f_0(z), \dots, f_r(z)).
$$
Since the $F_i$ have no common factor, and in particular are not all divisible by $t$, at least one of the
$f_i$ will have degree exactlly $d$; and since $\CC[z]$ is a principle ideal domain, any set of 
polynomials with no common divisor generates an ideal containing 
1 so after reordering the coefficients of $\PP^r$ write the map as
$$
z \; \mapsto \; (1, f'_1(z), \dots, f'_r(z)).
$$
or even
$$
\AA^1 \ni z \mapsto \; (f_1(z), \dots, f_r(z))\in \AA^n
$$
where again the polynomials $f_i'$ have degrees $\leq d$, and at least one has degree $d$.
For example, the twisted cubic itself can be represented by the map
$z \mapsto (z,z^2,z^3)$.

Here, we think of a polynomial $f(z) = f(s/t)$ of degree $\leq d$ as a rational function on $\PP^1$ having
a pole of order at most $d$ at the point at infinity $(1,0)\in \PP^1$; but we could also take rational
functions $F_i(s,t)/G_i(s,t)$ of total degree $\deg F_i-\deg G_i = d$ or, dehomogenizing, $\phi_i(z) = f_i(z)/g_i(z)$.

\subsection *{Smooth rational quartics}
Given how easy it is to describe rational curves in projective space in this way, it is surprising how many open questions there are. We begin with
one of the simplest cases: a smooth nondegenerate rational curve $C$ of degree $4$ in $\PP^3$.
As described above, such a curve is the image of $\PP^1$ under a map given by 4 quartic forms,
or, in the most coordinate-free formulation, a codimension 1 subspace of $\Sym^4(V)$, where
$V\cong \CC^2$. 

\begin{example}
In Exercise~\ref{distinguishing rational quartics} the reader may prove that there is a 1-parameter family of distinct embeddings of smooth rational quartic curves in $\PP^3$. Perhaps the simplest is given by the parameterization 
$$
\PP^1 \ni (s,t) \mapsto (s^4, s^3t, st^3, t^4)\in \PP^3,
$$
or, more simply, $t\mapsto(t, t^3, t^4)$. 
\end{example}


One of the most famous open problems in the theory of curves is whether the curve above
can be written \emph{set theoretically} as the intersection of two surfaces. For a sample of the recent
work on this, see~\cite{MR3356940}. However, it is relatively easy to determine generators for the
ideal of a smooth rational quartic in $\PP^3$:


\begin{proposition}\label{ideal of rational quartic}
If $C\subset \PP^3$ is a smooth rational quartic curve, then $C$ lies on a smooth quadric
surface $S$ in the divisor class $(1,3)$, and the homogenous ideal of $C$ is minimally
generated by the quadric defining $S$ and three cubic forms.
\end{proposition}

\begin{proof}
We first consider the maps restriction maps
$$
\rho_e: H^0(\op3e)\to H^0(\cO_C(e)).
$$
Since $C\cong \PP^1$,
we may identify $H^0(\cO_C(e))$ with $H^0(\op1{4e})$.
 Since we assume that $C$ is nondegenerate, it does not lie on a hyperplane,
 so  $\rho_1$ is a monomorphism. 
 
The source $H^0(\op32)$ of $\rho_2$ is 10-dimensional, and the target $H^0(\op18)$ is
9-dimensional, so $\rho_2$ has a kernel of dimension at least 1; that is, $C$ lies on
a quadric surface, and since $C$ is irreducible and nondegenerate, any quadric surface containing
$C$ is irreducible. This quadric is smooth, as we will show in Corollary~\ref{curves on a singular scroll}.

If $C$ lay on a second quadric surface, then, since the quadrics must be irreducible,
$C$ would be contained in the complete intersection of the two, which also has degree 4, so 
$C$ would be the complete intersection of the two quadrics.

Given that $C$ lies on a smooth quadric surface $S$, we consider its divisor class $(a,b)$ in the 
Picard group $\Pic(S) = \ZZ\oplus \ZZ$. The complete intersection of $S$ with another
quadric, lying in the divisor class of twice a hyperplane section, is $(2,2)$. But a curve
in the class $(a,b)$ has degree $a+b$ and genus $(a-1)(b-1)$ (see Example~\ref{Div of quadric}), so a curve in the class $(2,2)$
has genus 1, not zero. In fact solving the equations $a+b=4, (a-1)(b-1)=0$ we see that $C\sim (1,3)$ or $C\sim (3,1)$. Since the cases
are symmetric we assume $C\sim(1,3)$. 

The source of $\rho_3$ is 20 dimensional and the target is 13 dimensional so there are at least 7
cubics in the ideal of $C\subset \PP^3$. Four of these come from multiplying the quadric
by the 4 variables on $\PP^3$, so there are at least 3 more cubic generators in $I_{C/\PP^3}$,
 the homogeneous ideal of $C$. 

We now use the exact sequences 
$$
0\to I_{S/\PP^3} \to I_{C/\PP^3} \to I_{C/S} \to 0.
$$
and
$$
0\to \sI_{S/\PP^3} \to \sI_{C/\PP^3} \to \sI_{C/S} \to 0.
$$
Note that $\sI_{S/\PP^3} \cong \op3{-2}$, and since $H^1(\op3d)=0$ for all $d$, 
the sequences
$$
0\to H^0(\sI_{S/\PP^3}(d)) \to H^0(\sI_{C/\PP^3}(d)) \to H^0(\sI_{C/S}(d)) \to 0.
$$
are exact for all $d$. 

Since $C$ lies in class $(1,3)$ we see that 
$$
\sI_{C/S}(d) = \sI_{(-1+d,-3+d)/\PP^1\times \PP^1}.
$$
Thus by the K\"unneth formula,
$$
h^0(\sI_{C/S}(3)) = h^0(\sI_{(2,0)/\PP^1\times \PP^1}) = h^0(\op12)\cdot h^0(\op10) = 3
$$
and we see that $I(C)$ contains just 3 cubic generators. 

Moreover, the restriction
of $H^0(\op31)$ to the quadric is $H^0(\sO_{\PP^1\times \PP^1}(1,1))$,
so 
$$
H^0(\op31) \otimes H^0(\sI_{(2,0)/\PP^1\times \PP^1}(a,b)) \to 
H^0(\sI_{(2,0)/\PP^1\times \PP^1}(a+1, b+1))
$$
is surjective whenever both $a$ and $b$ are non-negative. In particular
$$
I_{C/S} = \oplus_{d=0}^\infty H^0(\sI_{C/S}(d))
$$
 is generated in degree 3. Since $I_{S/\PP^3}$ is generated by one quadric, this
  completes the proof.
\end{proof}

\subsection{Problems on rational curves}

We can say far less about rational curves of higher degree, even in $\PP^3$. For example, when $d$ is large, we
don't know the possible Hilbert functions for curves of degree $d$ in $\PP^3$, and the situation in $\PP^r$ for
$r>3$ is even worse.

However, we do know the Hilbert function of a \emph{general} rational curve $C \subset \PP^r$ of degree $d$. To make sense of this, let $C_0 \subset \PP^d$ be a rational normal curve of degree $d$. 
As described above, any rational curve of degree $d$ in $\PP^r$ is the image of $\PP^1$ under the map defined by
a linear series $(\op1d, V)$ where $V$ is an $(r+1)$-dimensional subspace of the space of
forms of degree $d$ in 2 variables, $H^0(\op1d)$. Thus we can talk about a \emph{general rational curve} of degree $d$ in $\PP^r$ by considering general subspaces of this type.

As in Proposition~\ref{ideal of rational quartic}, this is tantamount knowing the ranks of the restriction maps
$$
\rho_m : H^0(\cO_{\PP^r}(m)) \to H^0(\cO_C(m)) = H^0(\cO_{\PP^1}(md)).
$$
Equivalently we ask: if $V$ is a general  $(r+1)$-dimensional vector space of homogeneous polynomials of degree $d$, what is the dimension of the space of polynomials spanned by $m$-fold products of polynomials in $V$? 

We might guess that the answer is, ``as large as possible," meaning that the rank of $\rho_m$ is $\binom{m+r}{r}$ when that number is less than $md+1$, and equal to $md+1$ when it is greater---in other words, the map $\rho_m$ is either injective or surjective for each $m$. This was proven in~\cite{Ballico-Ellia83}. 

As we will see in subsequent Chapters it is possible to speak of a general curve of genus $g$
and a general invertible sheaf of degree $d$ on such a curve; and the analogous statement 
about Hilbert functions  was proven in \cite{ELarson2018}; see Chapter~\ref{Brill-Noether}.

Nevertheless, even the degrees of the generators of the homogeneous ideal of a general
rational curve of degree $d$ in $\PP^3$ is unknown for larger $d$. 
%\fix{DE wrote to Iarrobino 8/13 to ask whether he knows.}

\subsubsection{The secant plane conjecture}

Given a smooth curve $C \subset \PP^r$, we say that an $e$-secant $s$-plane to $C$ is an $s$-plane $\Lambda \cong \PP^s \subset \PP^r$ such that the intersection $\Lambda \cap C$ has degree $\geq e$; if we exclude degenerate cases (for example, where $\Lambda \cap C$ fails to span $\Lambda$), this is the same as saying we have a divisor $D \subset C$ of degree $e$ whose span is contained in an $s$-plane.

Should you expect a curve $C \subset \PP^r$ to have any $e$-secant $s$-planes? The set of $s$-planes in $\PP^r$ is parametrized by the Grassmannian $\GG = \GG(s,r)$, which has dimension $(s+1)(r-s)$. Inside $\GG$, the locus of planes that meet $C$ has codimension $r-s-1$ (the locus of planes containing a given point of $C$ has codimension $r-s$). Thus one might conjecture that a curve $C \subset \PP^r$ to have $e$-secant $s$-planes when 
$$
e \; \leq \; (s+1)\frac{r-s}{r-s-1},
$$
perhaps with a few low degree exceptions. Is this true of a general rational curve? For most $e$, $r$ and $s$, we don't know.

\section{Curves of genus 1}

We will describe the maps of a curve of genus 1 given by
the complete linear series in the lowest degree cases of interest, 2,3,and 4 and 5. Along the
way we will see several ways of parametrizing curves of genus 1 by one-dimensional varieties,
forerunners of the moduli spaces that we will introduce in~\ref{ModuliChapter}.


On a curve $E$ of genus 1 the canonical sheaf has degree 0; and since it has 1 section, it must be $\sO_C$.
Since invertible sheaves of negative degree cannot have any sections, \trr shows that
$h^0( \sL) = \deg \sL$ for any $\sL$ of positive degree. Among the surprising consequences is that, once we choose a point as 
origin (technically, making our curve of genus 1 into an \emph{elliptic curve}) the curve becomes
an algebraic group. 

The group operation is easy to describe:
Let $E$ be an elliptic curve with origin $o\in E$ chosen arbitrarily. If $p,q$ are points of $E$ then $\sO_E(p+q-o)$ has degree 1, and
thus has a unique global section. This vanishes at a unique point $r$, which may also be described as the unique
effective divisor linearly equivalent to $p+q-o$. For example if $r$ is the  unique point
linearly equivalent to $2o-p$ then $p+r-o\sim o$, so that $r$ is the inverse $-p$ of $p$. Note that this operation is commutative.

\begin{proposition}\label{group law} Let $E$ be an elliptic curve with origin $o\in E$.
If we set $p+q = r$ where $r$ is the unique effective divisor linearly equivalent to $p+q-o$, then $E$ becomes an algebraic group
and the group of divisor classes is divisible, in the sense that for divisor $D$ of degre $n>0$
 there is a point $p$ such that $D\sim np$.
 \end{proposition}

\begin{remark}
From the definition it is obvious that 
the map
$E \to \Pic_0(E)$ sending $p$ to $\sO_E(p-o)$ is an isomorphism of groups, and adding multiples of $o$
induces an isomorphism with each $\Pic_d(E)$ as well. This provides a natural sense
in which the family of invertible sheaves on $C$ can be treated as a smooth curve.
 In Chapter~\ref{JacobianChapter} we will see a general construction: the Picard groups can be made into
varieties, and for a curve $C$ of genus $g$ the divisors
of degree $g$ form a group that surjects birationally to $\Pic_g(C)$. 
\end{remark}
 
\begin{proof}
To show that the group operation is given by regular functions, we map $E$ to $\PP^2$ by the complete linear series $|3o|$. Since
$h^0(3o) = 3$ but $h^0(3o-p-q) = 1$ for any points $p,q$, this is an embedding. Moreover, there is a line in $\PP^2$ that meets
$E$ triply at $o$, which is thus an inflection point, and nowhere else. The three points $p,q,r$ in which any other line in the plane
meet $E$ sum to a divisor linearly equivalent to $3o$, and thus sum to zero in the group law, that is $r = -p-q$ Since $r$ is the unique
point in which the line
through $p,q$ meets  E, it follows that the coordinates of $r$ are polynomial functions of those of $p,q$, so the operation
$(p,q) \to -p-q$ is algebraic. But by the same token, the tangent line to $E$ at $r$ meets $E$ again at the point $-r$,
so the operation $r\to -r$ is also algebraic.

Given the group operation, we see that multiplication by a positive integer $n$ defines a non-constant map of 
curves $E\to E$. Since $E$ is projective, this map is surjective, proving the divisibility of the divisor classes of degree 0. 
Given a divisor class $D$ of degree $n$, the class $D -no$ has degree 0, and thus can be written as $n(r-o)$, so
$D\sim nr$.
\end{proof}

\begin{corollary}\label{equivalence of sheaves}
Given two invertible sheaves $\sL, \sL'$ on an of the same degree on a curve $E$ genus 1, there is an automorphism $\sigma: E\to E$
such that $\sigma^*\sL = \sL'$.
\end{corollary}

\begin{proof}
By Proposition~\ref{group law} we may write $\sL \cong \sO_E(np)$ and $\sL'\cong \sO_E(np')$ for some points $p,p'$; and tranlation by $p-p'$
is an automorphism of $E$ carrying one into the other.
\end{proof}


For the rest of the chapter, we fix a smooth irreducible projective curve $E$ of genus 1.

\subsection{Double covers of $\PP^1$}

Let $E$ be a smooth projective curve of genus 1 and let  $\sL$ be an invertible sheaf of degree 2 on $E$. By \trr{},\kern -3pt $h^0(\sL) = 2$ and the linear series $|\sL|$ is base point free, so we get a map $\phi : E \to \PP^1$ of degree 2. By \trh the map $\phi$ will have 4 branch points, which must be distinct because in a degree 2 map
only simple branching is possible. By Corollary~\ref{equivalence of sheaves}, these four points are determined, up to automorphisms of $\PP^1$ by the curve $E$, and are independent of the choice of $\sL$.

After composing with an automorphism of $\PP^1$ we can take these four points to be $0, 1, \infty$ and $\lambda$ for some $\lambda \neq 0, 1 \in \CC$. Since there is a unique double cover of $\PP^1$ with given branch divisor (see Lemma~\ref{branched cover classification} it follows that $E \cong E_\lambda$, where $E_\lambda$ is the curve given by the affine equation
$$
y^2 = x(x-1)(x-\lambda).
$$

The two curves $E_\lambda$ and $E_{\lambda'}$ are isomorphic if and only if there is an automorphism of $\PP^1$ carrying the points $\{0,1,\infty,\lambda\}$ to $\{0,1,\infty,\lambda'\}$, in any order. This will be the case if and only if $\lambda$ and $\lambda'$ belong to the same orbit under the action of the group $G \cong S_3 \subset PGL(3)$ of automorphisms of $\PP^1$ permuting the three points $0, 1$ and $\infty$. Direct computation shows that the orbit of $\lambda$ is
$$
\lambda' \in \{\lambda, \; 1-\lambda, \; \frac{1}{\lambda},\;  \frac{1}{1-\lambda}, \; \frac{\lambda - 1}{\lambda}, \; \frac{\lambda}{\lambda - 1} \}.
$$
We now use Theorem~\ref{Lueroth}: any subfield of $\CC(x)$ having transcendence degree 1 over $\CC$ is equal to $\CC(j)$ for some rational function $j$ .

The quotient of a normal variety by a finite group is normal (Reason: If $R$ is normal  with quotient field $Q$ and $G$ is a finite group,
then $R^G = R\cap Q^G$, and it is immediate that the intersection of normal rings is normal). Thus any quotient of $\PP^1$ by a finite group is again isomorphic to $\PP^1$.
One can check that $j$ can be taken to be
$$
j \; = \; 256\cdot \frac{\lambda^2 - \lambda + 1}{\lambda^2(\lambda - 1)^2}.
$$
(the factor of 256 is superfluous for us, but makes things work over the integers). Thus there is a unique smooth projective curve of genus 1 for each value of $j$, and in particular, the family of
curves of genus 1 is parametrized by points on a line.

\subsection{Plane cubics}

We can also represent an arbitrary smooth curve of genus 1 as a plane cubic:
Let $\sL$ be an invertible sheaf of degree 3 on $E$. As in the proof of Proposition~\ref{group law} the linear series $|\sL|$ gives an embedding of $E$ as a smooth plane cubic curve of degree 3; conversely, the adjunction formula implies that any smooth plane cubic curve has genus 1. 

The space of plane cubic curves is parametrized by the space of cubic forms in 3 variables up to 
scalars, a  $\PP^9$. The locus of forms defining smooth curves is a Zariski open subset. If two plane cubics are abstractly
isomorphic, that is if we have two different degree 3 linear series $|\sL|, |\sL|$ mapping a given genus 1 curve $C$ to the plane, then by
Proposition~\ref{equivalence of sheaves} we may  precompose one of the maps with an automorphism of $C$
and suppose that $\sL = \sL'$. Thus the two curves differ by an element of $PGL_3$ of automorphisms of $\PP^2$. Since the group $PGL_3$ has dimension 8, one should expect that the family of such curves up to isomorphism has dimension 1, which accords with the identification of the set of such
curves with the $j$-line in the previous section.



\subsection{Genus 1 quartics in $\PP^3$ and quintics in $\PP^4$} 

Again, let $E$ be a smooth projective curve of genus 1, and consider now the embedding of $E$ into $\PP^3$ given by the sections of an invertible sheaf $\sL$ of degree 4. To understand the ideal of $E$ we consider the restriction map
$$
\rho_2 \;  : \; H^0(\cO_{\PP^3}(2)) \; \to \; H^0(\cO_{E}(2)) = H^0(\sL^2).
$$
The space on the left---the space of homogeneous polynomials of degree 2 in four variables---has dimension 10, while by Riemann-Roch the space $H^0(\sL^2)$ has dimension 8. It follows that $E$ lies on at least two linearly independent quadrics $Q$ and $Q'$. Since $E$ does not lie in any plane, neither $Q$ nor $Q'$ can be reducible; thus by \bt\ we see that
$$
E = Q \cap Q'
$$
Since $Q,Q'$ form a regular sequence, the ideal $(Q,Q')$ is unmixed, and thus the homogeneous ideal $I(E)$ is generated by
these two quadrics. In this way, $E$ determines a point in the Grassmannian $G(2, H^0(\cO_{\PP^3}(2))) = G(2, 10)$ of pencils of quadrics; and by Bertini's Theorem, a Zariski open subset of that Grassmannian correspond to smooth quartic curves of genus 1. We can use this to once more calculate the dimension of the family of curves of genus 1: the Grassmannian $G(2,10)$ has dimension 16, while the group $PGL_4$ of automorphisms of $\PP^3$ has dimension 15, so again one should expect that the family of curves of genus 1 up to isomorphism has dimension 1.

There is a direct way to go back and forth between the representation of the smooth genus 1 curve $E$ as the intersection of two quadrics in $\PP^3$ and the representation of $E$ as a double cover
of $\PP^1$ branched at 4 distinct points. First, by Bertini's Theorem, we may take the two quadrics to be nonsingular, since they must meet transversely along $E$, and elsewhere the
pencil of quadrics they span has no base points. Representing the quadrics as symmetric matrices $A,B$, the pencil of all quadrics containing $E$ can be 
written as $sA+tB$. A quadric in the pencil is singular at the points $(s,t)$ such that the quartic polynomial $det(sA+tB)$ vanishes; thus at 4 points.

 A smooth quadric has two rulings by lines; a cone has one. Thus the family
$$
\Phi := \{ (\lambda, \sL) \mid \sL \in \Pic(Q_\lambda) \text{ is the class of a ruling of } Q_\lambda \}
$$
is---at least set-theoretically---a 2-sheeted cover of $\PP^1$, branched over the four values of $\lambda$ corresponding to singular quadrics in the pencil. In fact, we claim

\begin{proposition}\label{rulings on pencil}
There is a natural identification of $\Phi$ with $E$, and thus the branch points of $\Phi$ over $\PP^1$---that is, the set of singular elements of the pencil of quadrics---are the same, up to automorphisms of $\PP^1$ as the four points over which a double cover of $\PP^1$ by $E$ are ramified.
\end{proposition} 


\begin{proof}
First, choose a base point $o \in E$. We will construct inverse maps $E \to \Phi$ and $\Phi \to E$ as follows:
\begin{enumerate}

\item Suppose $q \in E$ is any point other than $o$, and let $M\subset \PP^3$ be the line $\overline{o,q}$ spanned by $o,q$. Every quadric $Q_\lambda$ contains the two points $o, q \in M$; if $r \in M$ is any third point, there will be a unique $\lambda$ such that $r$, and hence all of $M$, lies in $Q_\lambda$. Thus the choice of $q$ determines both one of the quadrics $Q_\lambda$ of the pencil, and a ruling of that quadric, giving us a map $E \to \Phi$.

\item  Given a quadric $Q_\lambda$ and a choice of ruling of $Q_\lambda$, there will be a unique line $M \subset Q_\lambda$ of that ruling passing through $o$, and that line $M$ will meet the curve $E$ in one other point $q$; this gives uf the inverse map $\Phi \to E$.
\end{enumerate}
\end{proof}

\begin{fact}
 There is a beautiful extension of this result to pencils of quadrics in any odd-dimensional projective space. Briefly: a smooth quadric $Q \subset \PP^{2g+1}$ has two rulings by $g$-planes, which merge into one family when the quadric specializes to quadric of rank $2g+1$, that is, a cone over a smooth quadric in $\PP^{2g}$. If $\{Q_\lambda\}_{\lambda \in \PP^1}$ is a pencil with smooth base locus $X = \cap_{\lambda \in \PP^1} Q_\lambda$, then exactly $2g+2$ of the quadrics will be singular, and they will all be of rank $2g+1$. The space $\Phi$ of rulings of the quadrics $Q_\lambda$ is thus a double cover of $\PP^1$ branched at $2g+2$  points; that is, a hyperelliptic curve of genus $g$. For a proof see for example~\cite[Proposition 22.34]{Harris1995}.
 This shows in particular that the polynomial $\det(sA+tB)$ has $2g+2$ \emph{distinct} roots. 

 There is also a remarkable analogue of Proposition~\ref{rulings on pencil} for any $g$: the variety $F_{g-1}(X)$ of $g-1$-planes in the base locus $X$ of the pencil is isomorphic to the Jacobian of the  curve $\Phi$. A proof of this in case $g=2$ can be found in~\cite{Griffiths-Harris1978}; for all $g$ it is done in~\cite{Donagi}
\end{fact}

The complete embeddings of genus 1 curves of any degree lie on certain ruled surfaces called rational
normal scrolls, and we will describe them in these terms in Chapter~\ref{scrolls Chapter}. However, there is a particularly nice description in $\PP^4$. 

\begin{fact}[Quintic curves of genus 1 in $\PP^4$]
Let $E$ be a smooth curve of genus 1. Any invertible sheaf $\sL$ of degree $5$ on $E$ is very ample, and considering the map $H^0(\op42) \to H^0(\sL^2)$
we see that $E\subset \PP^4$ lies on (at least) 5 quadrics. 
Recall that if $A$ is a skew-symmetric matrix of even size,
then the determinant of $A$ is the square of a polynomial in the entries of $A$ called the Pfaffian of $A$. For example, if
$$
M = \begin{pmatrix}
0&x_{1,1}&x_{1,2}&x_{1,3}\\
-x_{1,1}&0&x_{2,2}&x_{2,3}\\
-x_{1,2}&-x_{2,2}&0&x_{3,3}\\
-x_{1,3}&-x_{2,3}&-x_{3,3}&0\\
\end{pmatrix}
$$
then the Pfaffian of $M$ is by definition $x_{1,1}x_{3,3}-x_{1,2}x_{2,3}+x_{1,3}x_{2,2}$.

As a special case of the main theorem of ~\cite{MR453723} we have:
\begin{proposition} \cite[Theorem11]{Eisenbud1995}
If $E\subset \PP^4$ is a smooth curve of genus 1 and degree 5, then the presentation matrix of $I(E)$ is
a $5\times 5$ matrix of linear forms $A$. With a suitable choice of bases and variables, it can be put in the form
$$
A = 
\begin{pmatrix}
0&0&x_0&x_1&x_2\\
0&0&x_1&x_2&x_3\\
-x_0&-x_1&0&\ell_1&\ell_2\\
-x_1&-x_2&-\ell_1&0&\ell_3\\
-x_2&-x_3&-\ell_2&-\ell_3&0
\end{pmatrix}
$$
and
the homogeneous ideal of $E$ is generated by the  Pfaffians of the five $4\times 4$ submatrices of $A$ obtained by removing
a row and the corresponding column.
\end{proposition}
\end{fact}

\section{Exercises}

\begin{exercise}\label{veronese inverse}
With notation as in Section~\ref{rational normal curves section}, show that the sheaf associated to the graded module $\coker M$,
that is, the cokernel of the map $\sO_{\PP^d}^d(-1) \to \sO_{\PP^d}^2$ defined by $M$, is the unique invertible sheaf of degree 1
on the rational normal curve $C$, and that thus the associated complete linear series defines the isomorphism $C\to \PP^1$ inverse
to the Veronese map.
\end{exercise}

\begin{exercise}\label{equations of Veroneses}
Considering $\PP^n$ as $\Proj \CC[x_0,\dots,x_n]$, we may index the variables $z_p$ of $\PP^{\binom{n+d}{d}-1}$ by  monomials $p$
of degree $d$ in the $x_i$. Let $M_{n,d}$ be an $(n+1)\times \binom{n+d-1}{n}$ matrix of linear forms
on $\PP^{\binom{n+d}{d}-1}$ whose rows are indexed by the variables $x_i$, whose columns are indexed by the monomials $m$ of degree $d-1$ in the $x_i$ and
whose $(i,m)$ entry is $z_{x_im}$. (For example the matrix
$M$ of Section~\ref{rational normal curves section} is the matrix $M_{1,d}$.) Show that the $2\times 2$ minors of $M_{n,d}$ generate the ideal of the image $V_{n,d}$ of the Veronese map 
$\PP^n\to \PP^{\binom{n+d}{d}-1}$, and that the cokernel of $M_{n,d}$ is the unique invertible sheaf of degree 1 supported on $V_{n,d}$.
\end{exercise}

\begin{exercise}
 Let $\nu_d: \PP^r \to \PP^{\binom{r+d}{r}-1}$ be the $d$-Veronese map, and let $C\subset \PP^r$ be the rational normal curve of degree $r$. Is $\nu_d(C)$ nondegenerate? If not, what is the dimension of its linear span (that is, of the smallest linear
 space that contains it?
\end{exercise}


\begin{exercise}
Show that the twisted cubic is the unique irreducible, nondegenerate space curve lying on three quadrics by considering the possible
intersections of two of the quadrics.
\end{exercise}

\begin{exercise}\label{decomposition of a $g^3_4$}
As a consequence of our description of rational quartic curves on a smooth quadric in Proposition~\ref{ideal of rational quartic},
show that a general $g^3_4$ on $\PP^1$ is uniquely expressible as a sum of the $g_1^1$ and a $g^1_3$
(in other words, a general 4-dimensional vector space of quartic polynomials on $\PP^1$ is uniquely expressible as the product of a 2-dimensional vector space of cubics and the 2-dimensional space of linear forms.
\end{exercise}

\begin{exercise}\label{distinguishing rational quartics}
Show that, up to projective equivalence, there is a 1-parameter family of embeddings of $\PP^1$ as a 
smooth quartic curve in $\PP^3$ 
by constructing an invariant that distinguishes them. 
\end{exercise}

\begin{exercise}\label{Castelnuovo uniqueness}
Complete the proof of Proposition~\ref{points on rnc} by showing that if $C, C' \subset \PP^n$ are two rational normal curves meeting in at least $n+3$ distinct points, then $C = C'$. 
\end{exercise}


\begin{exercise}\label{rnc and representations}
Let $V = \CC\cdot e_1\oplus \CC\cdot e_2$ be a 2-dimensional vector space. 

The group $SL_2= SL(V)$ acts on the rational normal curve of degree $d$ through automorphisms induced from its action on
 on the ambient space $\PP^d$ of the rational normal curve, which may be identified with $\PP(\Sym^d(V))$.

In~\cite[pp. 146--150]{Fulton-Harris} it is shown that
 every finite dimensional rational 
representation of $SL(V)$ is a direct sum of representations of the form $\Sym^e(V)$ for various $e\geq 0$. Moreover, it is often easy to understand
how a given representation decomposes by looking at the action of
$$
\alpha := \begin{pmatrix}
t&0\\
0&t^{-1}
\end{pmatrix}
\in SL(V).
$$
Note that $\Sym^e(V)$ is spanned by ``weight vectors" ($\equiv$ eigenvectors of $\alpha$) $w_s := e_1^{e-s} e_2^{s}$ 
which satisfy $\alpha w_s = t^{e-2s}$ for $s = 0, \dots e$.
To decompose an arbitrary representation $W$, knowing that $W$ is a direct sum of $\Sym^{e_i}V$, it is enough to know the 
eigenvalues for the action of $\alpha$: We begin by finding an element $w\in W$ that
is an eigenvector of $\alpha$ and transforms by $\alpha$ as
as $\alpha w = t^mw$ with the highest possible $m$ (this is called a ``highest weight vector''). This element $w$ must be contained
in a summand $\Sym^m(V)$, and after removing the eigenvalues of the action of $SL_2$ on $\Sym^m(V)$, we continue. 
\begin{enumerate}
 \item Use this method to show that 
\begin{align*}
&\Sym^d(V)\otimes \Sym^d(V)= \Sym^{2d}(V) \oplus  \Sym^{2d-2}(V) \oplus \Sym^{2d-4}(V) \cdots\\
 &\Sym^2(\Sym^d(V))= \Sym^{2d}(V) \oplus \Sym^{2d-4}(V)\oplus \Sym^{2d-8}(V) \cdots\\
 &\bigwedge^2(\Sym^d(V))= \Sym^{2d-2}(V) \oplus \Sym^{2d-6}(V)\oplus \Sym^{2d-10}(V) \cdots
\end{align*}
  (where we take $\Sym^{m}(V)=0$ when $m<0$
 \item Show that the space of quadrics containing the rational normal curve is a representation of $SL_2$ of the form
 $$
 \Sym^{2d-4}(V)\oplus \Sym^{2d-8}(V) \cdots
 $$
  \item Show  there is a distinguished nonsingular skew-symmetric form (up to scalars) on the ambient space of the twisted cubic; in particular
  is, given a twisted cubic in $\PP^3$ there is a distinguished plane containing each point of $\PP^3$.
 \item Show that if $d$ is divisible by 4 there is a distinguished quadric in the ideal of the rational normal curve.
\end{enumerate}
\end{exercise}

%\begin{fact}
% In the case of the quartic in $\PP^4$, the quadric is the set of quartic polynomials whose zeros have
% j-invariant 0.
% \fix{ref or proof}
%\end{fact}

\begin{exercise}\label{Normal bundle of cubic}
Let $\PP^1 \hookrightarrow C \subset \PP^3$ be a twisted cubic. Show that the normal bundle $\cN_{C/\PP^3}$ (defined to be the quotient of the restriction $T_{\PP^3}|_C$ to $C$ of the tangent bundle  of $\PP^3$  by the tangent bundle $T_C$) is 
$$
\cN_{C/\PP^3} \cong \cO_{\PP^1}(5) \oplus  \cO_{\PP^1}(5).
$$
Hint: for any point $p \in C$, let $L_p \subset \cN_{C/\PP^3}$ be the sub-line bundle of $\cN_{C/\PP^3}$ whose fiber over any point $q \neq p \in C$ is the one-dimensional subspace of $(\cN_{C/\PP^3})_q$ spanned by the line $\overline{p,q}$. (This of course only defines a sub-line bundle of $\cN_{C/\PP^3}$ over $C \setminus \{p\}$, but there is a unique extension to a sub-line bundle of $\cN_{C/\PP^3}$ over all of $C$.) Show that for $p \neq p'$ we have
$$
\cN_{C/\PP^3} = L_p \oplus L_{p'}.
$$
\end{exercise}

\begin{exercise}
Let $\PP^1 \hookrightarrow C \subset \PP^d$ be a rational normal curve. Show that the normal bundle $\cN_{C/\PP^d}$  is 
$$
\cN_{C/\PP^d} \cong \bigoplus_{i=1}^{d-1} \cO_{\PP^1}(d+2).
$$
\end{exercise}

\begin{exercise}
In the situation of the preceding problem, the set  of direct summands of $\cN_{C/\PP^d} $ is a projective space $\PP^{d-2}$. How does the  group of automorphisms of $\PP^d$ carrying $C$ to itself act on this $\PP^{d-2}$?
(For more on normal bundles of rational curves, see for example~\cite{MR3778979}.
\end{exercise}

\input footer.tex


