%header and footer for separate chapter files

\ifx\whole\undefined
\documentclass[12pt, leqno]{book}
\usepackage{graphicx}
\input style-for-curves.sty
\usepackage{hyperref}
\usepackage{showkeys} %This shows the labels.
%\usepackage{SLAG,msribib,local}
%\usepackage{amsmath,amscd,amsthm,amssymb,amsxtra,latexsym,epsfig,epic,graphics}
%\usepackage[matrix,arrow,curve]{xy}
%\usepackage{graphicx}
%\usepackage{diagrams}
%
%%\usepackage{amsrefs}
%%%%%%%%%%%%%%%%%%%%%%%%%%%%%%%%%%%%%%%%%%
%%\textwidth16cm
%%\textheight20cm
%%\topmargin-2cm
%\oddsidemargin.8cm
%\evensidemargin1cm
%
%%%%%%Definitions
%\input preamble.tex
%\input style-for-curves.sty
%\def\TU{{\bf U}}
%\def\AA{{\mathbb A}}
%\def\BB{{\mathbb B}}
%\def\CC{{\mathbb C}}
%\def\QQ{{\mathbb Q}}
%\def\RR{{\mathbb R}}
%\def\facet{{\bf facet}}
%\def\image{{\rm image}}
%\def\cE{{\cal E}}
%\def\cF{{\cal F}}
%\def\cG{{\cal G}}
%\def\cH{{\cal H}}
%\def\cHom{{{\cal H}om}}
%\def\h{{\rm h}}
% \def\bs{{Boij-S\"oderberg{} }}
%
%\makeatletter
%\def\Ddots{\mathinner{\mkern1mu\raise\p@
%\vbox{\kern7\p@\hbox{.}}\mkern2mu
%\raise4\p@\hbox{.}\mkern2mu\raise7\p@\hbox{.}\mkern1mu}}
%\makeatother

%%
%\pagestyle{myheadings}

%\input style-for-curves.tex
%\documentclass{cambridge7A}
%\usepackage{hatcher_revised} 
%\usepackage{3264}
   
\errorcontextlines=1000
%\usepackage{makeidx}
\let\see\relax
\usepackage{makeidx}
\makeindex
% \index{word} in the doc; \index{variety!algebraic} gives variety, algebraic
% PUT a % after each \index{***}

\overfullrule=5pt
\catcode`\@\active
\def@{\mskip1.5mu} %produce a small space in math with an @

\title{Personalities of Curves}
\author{\copyright David Eisenbud and Joe Harris}
%%\includeonly{%
%0-intro,01-ChowRingDogma,02-FirstExamples,03-Grassmannians,04-GeneralGrassmannians
%,05-VectorBundlesAndChernClasses,06-LinesOnHypersurfaces,07-SingularElementsOfLinearSeries,
%08-ParameterSpaces,
%bib
%}

\date{\today}
%%\date{}
%\title{Curves}
%%{\normalsize ***Preliminary Version***}} 
%\author{David Eisenbud and Joe Harris }
%
%\begin{document}

\begin{document}
\maketitle

\pagenumbering{roman}
\setcounter{page}{5}
%\begin{5}
%\end{5}
\pagenumbering{arabic}
\tableofcontents
\fi


\chapter{Proof of the Brill Noether Theorem}\label{Brill Noether proof chapter}
\label{BrillNoetherproofChapter}

Our goal in this chapter is to give a proof of the Brill-Noether theorem based on the analysis
of inflections of of linear series on $\PP^{1}$ of Theorem~\ref{osculating intersection}. We will focus on proving what we call the ``Basic Brill-Noether" theorem (Theorem~\ref{basic BN}) of Chapter~\ref{Brill-Noether}), which we reproduce here for convenience:

\begin{theorem}[Basic Brill Noether]\label{BN-basic}
If $r\geq 0$ and
 $$
 \rho(g,r,d) := g - (r+1)(g-d+r) \geq 0.
$$
then every smooth projective curve of genus $g$  possesses a $g^r_d$; and for a general curve $C$,  $\dim W^r_d(C) = \rho$. Conversely, if $\rho < 0$ then a general curve $C$ of genus $g$ does not possess a $g^r_d$.
\end{theorem}

In fact, a closer examination of our proof will yield some of the assertions of the stronger Theorems~\ref{Wrd omnibus} and~\ref{grd omnibus}, as well as additional results on the existence of linear series with specified inflectionary behavior; we will discuss these at the end of the chapter.


\section{Castelnuovo's approach}

The original argument of Brill and Noether reduced the proof to the assertion that a matrix depending
on an invertible sheaf on a general curve
behaves in a certain sense like a matrix of indeterminates. This was finally proven in~\cite{Griffiths-Harris-BN} using an idea that goes back to Castelnuovo in~\cite{zbMATH02692307}.

Interestingly, Castelnuovo's goal was not to prove the Brill-Noether theorem, which was considered established at the time (or at least not in need of further demonstration); rather, he asked a follow-up question. If indeed a general curve $C$ of genus $g = 2d-2$ has a finite number of $g^1_d$s, as the Brill-Noether theorem asserts, Castelnuovo wanted to know: how many? We have answered this question in cases $g = 2, 4$ and 6 in earlier chapters, and these cases were certainly known to Castelnuovo, but the cases of higher $g$ was unknown.
 His idea was to specialize to a general curve $C_0$ with $g$ nodes  whose normalization is $\PP^1$, and count the number of $g^1_d$s on that curve. Appealing to the (then) vague ``principle of conservation of number'', Castelnuovo felt that the result probably reflected the number an a general smooth curve as well.\footnote{Castelnuovo presented his computation as heuristic, not claiming it was a full proof, and the reviewer of Castelnuovo's paper in the Zentralblatt wrote very politely:
``Das Resultat, welches er bekommen hat, gibt mit grosser Wahrscheinlichkeit den wahren Wert von N \dots; daher sind wir mit dem Verf. einverstanden, wenn er seinen Versuch nicht für wertlos hält''
(The result that he obtained gives the true value of [the number of $g^1_d$s] with high probability, and thus we agree with the author that his work is not worthless\dots)}  

By way of notation, we'll write $r_1,\dots,r_g$ for the nodes of $C_0$, and let $p_i, q_i \in \PP^1$ be the two points lying over $r_i$.
Castelnuovo counted the $g^1_d$s on $C_0$ by observing that any pencil on $C_0$ can be pulled back to a pencil on the normalization $\PP^1$; if we embed $\PP^1$ in $\PP^d$ as a rational normal curve of degree $d$, such a pencil is cut out by the hyperplanes containing
 a $(d-2)$-plane $\Lambda \subset \PP^d$. Moreover, to say that such a pencil is a pullback from $C_0$ means that every divisor of the pencil that contains $p_i$ contains $q_i$ and vice versa. This  is the condition that $\Lambda \cap \overline{p_i,q_i} \neq \emptyset$.

In the Grassmannian $G(d-1, d+1)$ of $(d-2)$-planes in $\PP^d$, the locus of those that meet the line $L_i = \overline{p_i,q_i}$ is what we called in the last chapter the Schubert cycle $\Sigma_1(L_i)$. In these terms the set of $g^1_d$s on $C_0$ is the intersection
$$
W^1_d(C_0) \; = \; \bigcap_{i=1}^{2d-2} \Sigma_1(L_i).
$$

Castelnuovo proposed that if the points $p_i, q_i\in \PP^1$ were chosen generally, then the Schubert cycles
$L_i$ would meet transversely, and thus that the cardinality of this intersection is the power $\sigma_1^{2d-2}$ in the Chow ring $A(G(d-1, d+1))$. Castelnuovo evaluated this power, and came to the conclusion that a general curve $C$ of genus $g=2d-2$ has 
$$
\#W^1_{d+1}(C) \; = \; \frac{(2d-2)!}{(d-1)!d!}
$$
pencils of degree $d$. Indeed, we see
 in Exercise~\ref{secant general position} that if the points $p_i, q_i$ are general, then the Schubert cycles $\Sigma_1(L_i)$ at least intersect properly, so the given number is the number of $g^1_d$ counted with appropriate multiplicities.

%Now, we saw in Corollary~\ref{secant schubert proper} that if the points $p_i, q_i$ are general, then the Schubert cycles $\Sigma_1(L_i)$ intersect properly; if we assume in addition that they intersect transversely, then the cardinality of this intersection is the power $\sigma_1^{2d-2}$ in the Chow ring $A(G(d-1, d+1))$. (Indeed, since the Schubert cycles $\Sigma_1(L_i)$ are hyperplane sections of the Grassmannian in the Pl\"ucker embedding, this is equal to the degree of the Grassmannian.)
%Castelnuovo evaluated this power, and came to the conclusion that a general curve $C$ of genus $g=2d-2$ has 
%$$
%\#W^1_{k+1}(C) \; = \; \frac{(2d-2)!}{(d-1)!d!}
%$$
%pencils of degree $d$.

The papers \cite{MR323792}, \cite{MR0357398} and \cite{Kempf} use this degeneration idea
to prove the ``existence'' part of Theorem~\ref{BN-basic}---that is, that $W^{r}_{d}(C)$ is nonempty when
$\rho\geq 0$---and in~\cite{Kleiman-special},  the general ``non-existence'' part is reduced to a the proof of Exercise~\ref{secant general position} and completed for $r=1$. The general statement was finally proven in~\cite{Griffiths-Harris-BN}.

We will also adopt this general approach, and we will prove the existence and non-existence parts together
for all $g,d,r$. However, the strength of Theorem~\ref{osculating intersection} compared to Exercise~\ref{secant general position} suggests specializing to a $g$-cuspidal curve $C_0$ rather than a $g$-nodal one, and we will use
this refinement. 

\subsection{Upper bound on the codimension of $W^r_d(C)$}

Let $C$ be a reduced irreducible projective curve of arithmetic genus $g$.  In Section~\ref{BN by divisors}  how we might arrive at the ``expected" dimension of the locus $W^r_d(C)$ by estimating the dimension of the subvariety $C^r_d \subset C_d$ of divisors moving in an $r$-dimensional linear series; here we'll give a similar argument using the Picard variety $\pic(C)$. Again the heart of the proof is the estimation of the
codimension of an ideal of minors of a certain matrix of functions. Since the proof uses degeneration
to a cuspidal curve, we will work with flat families of curves that may include singular fibers. For simplicity
we will take the base of the family to be a complex disk $\Delta$ centered at the origin in $\CC$.

\begin{theorem}\label{local existence}
Let $\sC/\Delta$ be a family of reduced irreducible projective curves of arithmetic genus $g$. If
the fiber  $C_0$ of the family has $\dim W^r_d(C_0) = \rho(p,r,d) \geq 0$
locally at a particular invertible sheaf $\sL_{0}$,  then $\dim W^r_d(C_b) = \rho$ locally for all $b$ in a neighborhood of $0 \in \Delta$ and invertible sheaves in a neighborhood of $\sL_{0}$
in the relative Picard variety $\pic_{d}(\sC/\Delta)$. In particular, $W^r_d(C_b)$ is nonempty for all $b$ in a neighborhood of $0 \in \Delta$.
\end{theorem}

The idea of the following proof is a modern form of the discussion in the original paper of Brill and Noether. 

\begin{proof}  Possibly after pulling back the family $\sC/\Delta$ along a ramified covering $\Delta\to \Delta$
and restricting to a smaller disk around 0, we may choose $m$ sections
$p_{i}(b)$ with distinct values in the smooth locus of each fiber, and $m$ as large as we like. We 
define a family of divsors $D_{b} = \sum_{i}p_{i}(b)$.
By the semicontinuity of fiber dimension, we may choose $m$ so large that $h^{1}(\sO_{C_{b}}(D_{b})) = 0$
and that for every invertible sheaf $\sL_{b}$ of degree $d$ on $C_{b}$ we have
$h^{1}(\sL_{b}(D_{b}) = 0$. 
By Theorem~\ref{easy RR} we have
$$
h^0(\cL_{b}(D_{b})) =  d+m - g + 1.
$$

The pushforward of the Poincar\'e bundle $\cP$ on $\pic_{d+m}(\sC/\Delta) \times \sC$ to $\pic_{d+m}(\sC/\Delta)$ is thus a vector
bundle $\sE$ of rank $d + m - g + 1$ on $\Delta$ whose fiber over a point representing $\sL_{b}$ is the vector space $H^0(\cL(D_{b}))$. Similarly,
the pushforward 
${\pi_1}_*(\cP|_{D_{b}})$
is a trivial vector bundle $\sF$ whose fiber at every point is the $m$-dimensional vector space $\oplus_{i = 1}^{m} \cL(E)|_{p_i}$. Regarding $D$ as a family of finite schemes inside $\sC$, the restriction map
$$
\cP  \rTo \cP_{D}
$$
pushes forward to give a map of vector bundles $\phi : \cE \to \cF$ on $\sC$ which, on a fiber over $b$, is the evaluation of sections $\sigma \in H^0(\cL_{b}(D_{b}))$ at the points $p_i$.

If $\cL_{b}$ is an invertible sheaf of degree $d$ on $C_{b}$ then the space $H^0(\cL_{b})$ is the kernel of the map $\phi$ at the point corresponding to $\cL_{b}(D_{b}) \in \pic_{d+m}(\sC/\Delta)$. Locally on $\sC$ the map $\phi$ can be defined by a $(d+m) \times (d+m-g+1)$ matrix of regular functions. The locus $W^r_d(C_{b})$ is 
thus the translate
by $\otimes \sO_{\sC}(D)$ of the locus where $\phi$ has rank $d+m-g-r$ or less, and by \cite[Exercise 10.9]{Eisenbud1995} this locus is either empty or its components have codimension $\leq (r+1)(g-d+r)$ in $\pic_{d+m}(C)$, which has dimension $g$. Consequently, every component of $W^r_d(C)$ has dimension at least $\rho$. 
\end{proof}

 We will see that if $C_{0}$ is any rational curve
 with $g$ cusps, and $\rho(g,r,d)\geq 0$, then $\dim W^r_d(C_0) = \rho$ locally at every invertible sheaf
 on $C_{0}$.


\section{Specializing to a $g$-cuspidal curve}

Our first goal is to find a family $\{C_t\}$ of curves of arithmetic genus $g$, with $C_t$ smooth for $t \neq 0$ and $C_0$ a rational curve with $g$ cusps. To do this we show how to construct a rational curve $C_0$ with $g$ cusps and deform it to a smooth curve.

\subsection{Constructing curves with cusps}

\begin{proposition}
Let $C$ be any curve and $p \in C$ a smooth point. There exists a curve $C_0$ and a bijective morphism $f : C \to C_0$ such that  $f$ maps $C \setminus \{p\}$ isomorphically to $C_0 \setminus \{r\}$ and the image $r=f(p) \in C_0$ is a cusp of $C_0$.
\end{proposition}

We will say that we \emph{crimp} $C$ at $p$ to obtain $C_{0}$. For the corresponding result with nodes instead of cusps, see Exercise~\ref{independent secants}. 

\begin{proof}
We can construct $C_0$ explicitly as a topological space homeomorphic to $C$, with structure sheaf $\cO_{C_0}$ that is
the subsheaf of $\cO_C$ consisting of functions on $C$ whose derivative at $p$ is 0.
\end{proof}

If we start with $\PP^1$, pick any $g$ points $p_1,\dots, p_g \in \PP^1$ and crimp at each $p_i$, we arrive at a $g$-cuspidal curve $C_0$.


\subsection{Smoothing a cuspidal curve}  Given a curve $C_0$ with a finite number of cusps and no other singularities, we can find a proper flat family $\cC \to \Delta$ with special fiber $C_0$ and all other fibers smooth;
that is, we can smooth $C_0$. 

To begin with, we can do this locally in the complex analytic setting: if $p \in C_0$ is a cusp, we can find an analytic neighborhood of $p$ in which $C_0$ is given by the equation $y^2 = x^3$; we can smooth this by taking the family
$$
y^2 = x(x-1)(x-t)
$$
for $t\in \Delta$.
\pict{local picture of a family of smooth curves degenerating to a cusp}

The next step is to argue that we can glue together these local smoothings to obtain a proper family $\cC \to \Delta$, and this is where we need to invoke a result from deformation theory:

\begin{lemma}\label{specialization to cuspidal curve}
Let $p_1,\dots,p_g \in \PP^1$ be distinct points, and $C_0$ the curve obtained by 
crimping $\PP^1$ at each $p_i$. There exists a family of curves $\pi : \cC \to \Delta$, where
\begin{enumerate}
\item $\Delta$ is a disk centered at the origin in $\CC$.
\item for all $b \neq 0 \in \Delta$, the fiber $C_b = \pi^{-1}(b)$ is a smooth, projective curve of genus $g$;  and
\item the fiber over $0$ is the curve $C_0$.
\end{enumerate}
\end{lemma}

\begin{proof}[Proof Sketch]
The cuspidal curve $C_{0}$ can be embedded in projective space $\PP^{n}$ and there
it is locally a complete intersection so, as with the case of one cusp, above, the local obstructions
to smoothing vanish. The global obstruction is the first cohomology of a sheaf supported at the cusps,
and therefore vanishes as well.
\end{proof}
%\fix{Give a reference.
%Ravi sent us references!}


%
%
%\subsubsection{Step 2: Castelnuovo's specialization}
%
%Next, Castelnuovo proposed analyzing a family of smooth curves specializing to a $g$-nodal one; in order to use this construction in a proof of Brill-Noether, we have to prove that such families exist. We'll state the lemma we need here:
%
%\begin{lemma}\label{specialization to nodal curve}
%Let $p_1,\dots,p_g, q_1,\dots, q_g \in \PP^1$ be distinct points, and $C_0$ the curve obtained by identifying $p_i$ with $q_i$ for $i = 1,\dots,g$. There exists a family of curves $\pi : \cC \to B$, where
%\begin{enumerate}
%\item $B$ is a smooth curve, with distinguished point $0 \in B$;
%\item for all $b \neq 0 \in B$, the fiber $C_b = \pi^{-1}(b)$ is a smooth, projective curve of genus $g$;  and
%\item the fiber over $0$ is the curve $C_0$.
%\end{enumerate}
%\end{lemma}
%
%This lemma will follow from the local geometry of Severi varieties, as worked out in Chapter~\ref{PlaneCurvesChapter}, and we defer the proof to that chapter.

%\begin{lemma}\label{BN in family}
%If $p_1,\dots,p_g, q_1,\dots, q_g \in \PP^1$ are general points and $\cC \to B$ is a family of curves as described in Lemma~\ref{specialization to nodal curve} above, then for general $b \in B$ the fiber $C_b$ does not possess a $g^r_d$ with $\rho < 0$.
%\end{lemma}
%
%From this, we deduce the basic
%
%\begin{theorem}\label{bare-bones BN}
%A general curve $C$ of genus $g$ does not possess a $g^r_d$ with $\rho(g,r,d) < 0$.
%\end{theorem}

\section{The family of Picard varieties}\label{Picard family}

By Lemma~\ref{specialization to cuspidal curve} there is a flat family $\cC \to \Delta$ of curves, specializing from a smooth curve of genus $g$ to a $g$-cuspidal curve. The next step is to relate linear series on the general fiber of our family to their limits on $C_0$.

\subsection{The Picard variety of a cuspidal curve}

%
%To start, let $C$ be a curve with a node $p$, and $\widetilde C \rTo^\nu C$ the normalization of $C$ at $p$; we'll denote by $q,r \in \widetilde C$ the points lying over $p$. If $\cL$ is an invertible sheaf on $C$, and $\cM := \nu^*(\cL)$ the pullback of $\cL$ to $\widetilde C$, then $\cM$ is an invertible sheaf on $\widetilde C$. Its fibers over $q$ and $r$ are both identified with the fiber $\cL_p$ of $\cL$ at $p$, and hence with each other. Conversely, given an invertible sheaf $\cM$ on $\widetilde C$ and an identification of the fibers $\cM_q$ and $\cM_r$, we can form an invertible sheaf $\cL$ on $C$ by taking the subsheaf of $\nu_*\cM$ whose sections ``agree" at $q$ and $r$, in terms of the identification. 
%
%\pict{line bundle on a nodal curve and it's pullback, indicating the identification of the fibers over the
%preimages of the node}
%
%Thus the family of invertible sheaves $\cL$ on $C$ whose pullback is isomorphic to a given $\cM$ may be identified with the set of isomorphisms $\cM_q \cong \cM_r$ of the two one-dimensional vector spaces $\cM_q$ and $\cM_r$, so that  we have an exact sequence of groups
%$$
%0 \rTo  \CC^* \rTo \pic_0(C) \rTo^{\nu^*}  \pic_0(\widetilde C) \rTo 0
%$$
%and similarly for $\pic_d$ for any degree $d$.
%
%One way to express this---which will apply as well in the cuspidal case---is that \emph{the data of an invertible sheaf $\cL$ on $C$ is equivalent to the data of an invertible sheaf $\widetilde \cL$ on $\widetilde C$, together with a choice of trivialization of $\widetilde \cL$ on the preimage $\nu^{-1}(p)$, up to scalar multiplication}.
%
%
%\def\wL{{\widetilde\sL}}
%
%Similarly, 

We can describe the invertible sheaves on a cuspidal curve $C_{0}$ in terms of the invertible sheaves on the
normalization:

\begin{proposition}\label{torsion-free on cuspidal}
Let $p\in C_{0}$ be a cusp that is the result of crimping a reduced irreducible projective curve $C$ at a smooth point $q\in C$,
and let $\nu: C\to C_{0}$ be the natural morphism.
There is an exact sequence
$$
0 \rTo \CC \rTo \pic_0(C_{0}) \rTo^{\nu^*} \pic_0(C) \rTo 0.
$$
\end{proposition}

\begin{proof}
The preimage $\nu^{-1}(p)$ is the nonreduced scheme $2q \subset C$. If $\cL$ is an invertible sheaf on 
$C_{0}$ then $\nu^{*}(\cL)$ is invertible on $C$, and thus locally trivial near $q$ and, in particular, trivial
when restricted to the subscheme $2q$. Conversely,
if $\cL'$ is an invertible sheaf on $C$ and we choose a trivialization of $a: \cL'|_{U}\cong \sO_{U}$ on a neighborhood $ U\subset C$
of $q$, then there is a unique invertible sheaf $\cL$ on $C_{0}$ together with a  trivialization
$a: \cL|_{\nu(U)}\cong \sO_{\nu(U)}$
n $\nu(U)$ that
pulls back to $\cL'$ with the given trivialization $a$, up to an automorphims of $\cL'$, that is, multiplication by a scalar.
Thus the data of an invertible sheaf $\cL$ on $C_0$ is equivalent to the data of an invertible sheaf $\cL'$ on $C$, together with a trivialization of $\widetilde \cL$ on the preimage $\nu^{-1}(p) = 2q$ up to multiplication by a nonzero scalar.

Of course every invertible sheaf on the zero-dimensional scheme $2q$ is trivial, and a trivialization
thus corresponds to an automorphism of the structure sheaf $\sO_{2q} \cong \CC[\epsilon]/(\epsilon^{2})$---not as an algebra, but as a rank 1 free module over $\CC[\epsilon]/(\epsilon^{2})$. Such a map is determined by
its effect on the generator 1, and can take this element to any other generator $a+b\epsilon$ with
$a\neq 0$ and $b\in \CC$ arbitrary. After multiplication
by a scalar, which has no effect, we may suppose that $a =1$ and thus the automorphisms, mod nonzero
Up to multiplication by a scalar, we may take the isomorphism to induce the identity on $\CC$
In this case, the family of such trivializations, modulo multiplcation by a nonzero scalar, is parametrized by
the additive group $\CC$.
\end{proof}

\begin{corollary}
If $C_{0}$ is a $g$-cuspidal rational curve, then $\pic_{d}(C_{0}) \cong \CC^{g}$. More precisely,
$\pic_{d}(C_{0})$ is a principal homogeneous space for the additive group $\CC^{g}$, in the sense that
this group acts faithfully and transitively.
\end{corollary}
\begin{proof}
We can construct one invertible sheaf on $C_{0}$ as the inverse of the ideal sheaf of a divisor of $d$ smooth points, and any other will differ from this one by the choice of an element of $\pic_{0}(C_{0})$,
corresponding to a choice of $g$ trivializations, as in the proof of the Proposition. \end{proof}

Thus we see that $\pic_{d}(C_{0})$ has dimension $g$ just as does the Picard variety of a smooth curve
of genus $g$, an important difference being that $\pic_{d}(C_{0})$ is not compact.
%
%Suppose that
%$\sL$ is an invertible sheaf on $C$, and let $\wL = \pi^*(\sL) = \sO_{\widetilde C}\otimes_{\sO_C} \sL$ be its pullback to $\widetilde C$. The inclusion $\gm_{C,p}\subset \sO_C \subset \pi_*(\sO_{\widetilde C})$ induces proper inclusions
%$$
%\gm_{C,p}\wL  = \wL(-2p) \subset \sL \subset \wL;
%$$
%where for simplicity we have written $p$ again for the preimage of $p$ in $\widetilde C$ and $\wL$ in place of $\pi_*(\wL)$.
%This is the same as saying that local sections of $\sL$ may be identified with the local sections of $\wL$ that
%are either units or vanish to order 2 at $p$.
%
%Fixing the sheaf $\wL\in \Pic_0(\widetilde C)$ we note that  $\wL/\wL(-2p) \cong \CC[x]/(x^2)$. The 1-dimensional subspaces of $\CC[x]/(x^2)$ that contain a unit
%are those of the form $\CC(1+rx)$ for $r\in \CC$. Identifying the image of a local generator of $\sL$ in $\wL/\wL(-2p)$
% with $1\in \CC[x]/(x^2)$, we see that there are invertible sheaves $\sL_r$ on $C$ whose local generator
% at $p$ goes to $1+rx\in \CC[x]/(x^2)$. These are non-isomorphic on $C$ because
% any isomorphism would induce an automorphism of $\wL(-2)$, and such an automorphism would be given by multiplication
% with an element of $\CC^*$. Once again, these are sheaves that become isomorphic when restricted to the
% complement of $p$ (since they are isomorphic as sheaves on $\widetilde C$. Thus we have an exact sequence 
%$$
%0 \rTo \CC \rTo \pic_0(C) \rTo^{\nu^*} \pic_0(\widetilde C) \rTo 0,
%$$
%and similarly for $\Pic_d$.

% If the curve $C$ has a cusp at $p$, the analogue of the class of identifications $\cM_q \cong \cM_r$ above is an equivalence class of trivializations of the invertible sheaf $\cM$ over the preimage $\nu^{-1}(p) \subset \widetilde C$, where two trivializations are equivalent if they differ by multiplication by a constant. Stated that way, the same conclusion holds if $\nu : \widetilde C \to C$ is the normalization of a curve $C$ at a cusp $p$; the difference is that the preimage $\nu^{-1}(p)$ is a nonreduced scheme isomorphic to $\Spec \CC[\epsilon]/(\epsilon^2)$ rather than two reduced points. In this case, the family of equivalence classes of trivializations of $\cM$ over $\nu^{-1}(p)$ is a copy of $\CC$ rather than $\CC^*$, and we have correspondingly an 
 


%In the case of the curves we'll be dealing with, we can say that for a $g$-cuspidal curve $C$,
%$$
%\pic_0(C) \; \cong \CC^g
%$$
%and for a $g$-nodal curve $C$, 
%$$
%\pic_0(C) \; \cong (\CC^*)^g.
%$$
%Note that both of these are irreducible of dimension $g$, just like the Picard variety of a smooth curve of genus $g$. In neither case, however, is $\pic_0(C)$ proper. We will deal with this fact in Section~\ref{line bundle limits} below.

\subsection{The relative Picard variety}

%\fix{the next para should  be factified in the Jacobian chapter where we state the existence of Pic_d in a cheerful fact.} 
Returning to the family $\pi : \cC \to \Delta$ at the beginning of Section~\ref{Picard family}, the Picard varieties $\Pic_d(C_t)$ form a family $\pic_d(\cC/\Delta)$, and the varieties $W^r_d(C_t)$ form a subfamily $\cW^r_d(\cC/\Delta)$.  In the argument
of Theorem~\ref{local existence} bounding  the dimensions of the $W^r_d$
we may replace the points $p_i$ by sections of the family, and thus
the codimension of $\cW^r_d(\cC/\Delta) \subset \pic(\cC/\Delta)$ is $\leq (r+1)(g-d+r)$ locally at each point.

We will soon show that $W^{r}_{d}(C_{0})$, the fiber of $\cW^{r}_{d}(\cC)$ over 0,  is nonempty of dimension $\rho(g,r,d)$ when $\rho(g,r,d)\geq 0$ and otherwise empty. If the family of Picard varieties over $\Delta$ were proper,
this would prove Theorem~\ref{BN-basic}; but it is not proper, and thus it is, for example, a priori possible when $\rho(g,r,d)<0$ and thus $W^{r}_{d}(C_{0})=\emptyset$ that $W^{r}_{d}(C_{t})\neq \emptyset$ for $t\neq 0$; this would mean that families of invertible sheaves $\sL_{t}$ on $C_{t}$ would simply not have a limit
that is an invertible sheaf on $C_{0}$. Thus we next must describe such limits.

%But if our goal is to say something about the varieties $W^r_d(C_t)$ by studying the fiber of $\cW^r_d(\cC/\Delta)$ over $t=0$, we have a problem: as we've noted, the map $\cW^r_d(\cC/\Delta) \to \Delta$ is not proper, so it is a priori possible that the fiber of $\cW^r_d(\cC/\Delta)$ over $t=0$ is empty, whatever the geometry of $W^r_d(C_t)$ for $t \neq 0$. In other words,  if we're going to analyze linear series on the general curve of our family $\{C_t\}$ of smooth curves specializing to a $g$-cuspidal or $g$-nodal curve by taking limits, we have to describe the possible limits of linear series on $C_t$ as $t\to 0$; we will take this up in the next section.


\subsection{Limits of invertible sheaves}\label{invertible sheaf limits}


Suppose that $0\in \Delta$ is a point of a smooth curve, and that  $\pi : \cC \to \Delta$ is a family of smooth genus $g$ curves specializing to a (necessarily rational) curve $C_0$ with $g$ cusps as in Lemma~\ref{specialization to cuspidal curve}. 
 Let 
$$
\pi^\circ: \cC^\circ := \cC \setminus C_0\to \Delta^\circ := \Delta\setminus 0,
$$
and suppose that there is a invertible sheaf $\cL^\circ$ on $\cC^\circ$ such that $h^0(\cL^\circ|_{C_b}) \geq r+1$ for each $b \neq 0 \in \Delta$ so that $\cL^\circ|_{C_b}\in W^{r}_{d}(C_{b})$. We would like to describe the ``limit" of $\sL$ as $b \to 0$. We can extend $\cL^\circ$  to a rank 1 torsion-free sheaf on all of $\cC$, and we regard
the torsion-free sheaf  on $C_{0}$


\begin{lemma}\label{limit sheaf}
In the situation above, there exists a torsion-free sheaf $\cL$ of rank 1 on $\cC$, flat over $\Delta$ and locally isomorphic to
an ideal sheaf of $\cC$, such that $\cL|_{\cC^\circ} \cong \cL^\circ$.
\end{lemma}

In fact, any torsion free sheaf of rank 1 on $\cC$ is locally isomorphic
to an ideal sheaf; this follows from the fact that $\cC$ is generically Gorenstein. However, the argument we give 
shows directly that the extension is isomorphic to an ideal sheaf tensored with an invertible sheaf, and the fact that
it is locally isomorphic to an ideal sheaf follows at once.

\begin{proof} Choose an auxiliary invertible sheaf $\cM$ on $\cC$ with relative degree $e > d + 2g$ and let $\cM^\circ$ be the restriction of $\cM$ to $\cL^\circ$. Consider the invertible sheaf 
$$
\cN^\circ = (\cL^\circ)^* \otimes \cM^\circ.
$$
%The bundle $\cN^\circ$ has lots of sections: the direct image, as a sheaf on $B$, is locally free of rank $e-g+1 > 0$, and after restricting to an open neighborhood of $0 \in B$ we can assume it's generated by them \fix{This seems to require that the
%original fibers had exactly $r+1$ independent sections. Also, we are still in a punctured neighborhood of $b$, so this might need some further argument}. 
Choose a section $\sigma$ of $\cN^\circ$; let $D^\circ \subset \cC^\circ$ be its divisor of zeros, and let $D \subset \cC$ be the closure of $D^\circ$ in $\cC$. Because it i the closure, the scheme $D$ has no embedded 
0-dimensional components, and thus $\sO_{\cC}/\sI_{D/\cC}$ has no $\sO_{\Delta}$-torsion. 

Since $\Delta$ is a smooth
curve, this implies that $\sO_{\cC}/\sI_{D/\cC}$ is flat over $\Delta$, and thus $\sI_{D/\cC}|_{C_{0}}$
is an ideal sheaf, the kernel of $\sO_{\cC} \to \sO_{\cC}/\sI_{D/\cC}$. Thus
$$
\cL := \cI_{D/\cC} \otimes \cM
$$
has the desired properties. 
\end{proof}

\pict{divisor on total space of a family of curves degenerating to one with several cusps. The divisor
should go through some of the cusps, some not.}
%Even if the family we originally started with had smooth total space, the base change called for in the first step would yield a family $\cC$ with  total space singular at the nodes of $C_0$. If $D$ passes through any of these points it need not be Cartier, 
%so $\cL|_{C_0}$, though torsion-free, may not be invertible.
%
%In sum, if the general fiber $C_b$ of our family has a $g^r_d$, we can conclude that the special fiber $C_0$ has a torsion-free sheaf $\cL_0$ with 
%$$
%c_1(\cL_0) = d;
%$$
%\fix{where did the reader learn about Chern classes of torsion-free sheaves?? Might be better to say degree and explain what that means for a torsion-free sheaf.}
%and, by upper-semicontinuity of cohomology,
%$$
%h^0(\cL_0) \geq r+1.
%$$

%\fix{clarify that this is locally iso to an ideal, though maybe not globally}
Fortunately the ideal sheaves on cuspidal curves have a simple local structure. The reason lies in the relation of $R$ to its integral closure. 

\begin{definition}
The \emph{conductor} of an integral domain $R_{0}$ is the annihilator of the $R_{0}$-module
$R/R_{0}$, where $R$ is the integral closure of $R_{0}$.
\end{definition}

It follows at once from the definition that the conductor of $R_{0}$ is also an ideal of $R$, and that it is the largest such ideal. The following result is a restatement of example 2 after Proposition~\ref{Leray}:
%\fix{introduce the word conductor in Ch 2 where this is done}

\begin{proposition}\label{conductor of node and cusp}
If $R_{0}$ is the local ring of an ordinary cusp singularity of a curve, then  $R/R_{0} \cong k$, the residue field of $R$, and thus the conductor of $R_{0}$ is the
maximal ideal $\gm_{0}$ of $R_{0}$. \qed
\end{proposition}

\begin{proof} These properties can be verified after completing at the maximal ideal of $R$.
To say 
%that $R_0$ has an node singularity means that the completion of $R_0 \subset R$ at the maximal ideal $\gm_0$ of $R_0$ is $k[[x,y]]/(xy)\subset k[[x]]\times k[[y]]$. Since  $k[[x,y]]/(xy)$ contains every $x^n$ and $y^n$ with $n>0$,
%$(x,y) (k[[x]]\times k[[y]]) \subset k[[x,y]]/(xy)$, so 
%$(k[[x]]\times k[[y]])/ k[[x,y]]/(xy) \cong k$.
%
%Similarly, to 
that $R_0$ has an ordinary cusp singularity means that the completion of 
$R_0 \subset R$ is $k[[x^2,x^3]]\subset k[[x]]$, and the quotient is $kx \cong k$.
\end{proof}

%\fix{I changed torsion-free sheaf to ideal sheaf. While it's true that any torsion-free sheaf of rank 1 over
%a generically Gorenstein ring is an ideal
%sheaf, there are subtleties that I think are best avoided.}

\begin{theorem}\label{torsion free at node}
Let $p$ be an ordinary cusp of a curve $C_0$ with normalization $\pi: C \to C_0$. Let $R_0 = \sO_{C_0,p}$ be the local ring of the singular point, with maximal ideal $\gm_0 = \sI_{p}R_{0}$,
and let $R_{0}\rTo^{\pi^{*}} R$ be its normalization.  If $I_{0}\neq 0$ is an ideal of $R_{0}$ then 
either $I_{0}\cong R_{0}$ or $I_{0}\cong \gm_0$.

 In the latter case, $I_{0} \cong Ra$, and if $R_{0}/I_{0}$ has length $v$ then $R/I_{0} = R/RI_{0}$ has length $v+1$.
 Moreover,
there is a split exact sequence
$$
0\to R/\gm_0 R \to I_{0} \otimes R  \to R I_{0} \to 0
$$
with $RI_{0}  \cong R$.
\end{theorem}

Interpreting Theorem~\ref{torsion free at node} in the context of a an ideal sheaf on a cuspidal curve
we get:

\begin{corollary}
Let $p\in C_{0}$ be an ordinary cusp in a reduced irreducible curve and let $\pi:C \to C_{0}$ be 
the partial normalization of $C_{0}$ at $p$, so that $p_a(C) = p_a(C_{0}) -1$.
Let $q\in C$ be the point lying over $p\in C_{0}$.

If $\sF_0$ is locally
isomorphic to a nonzero ideal sheaf on $C_{0}$
and $\sF_0$ is not locally free at $p$, then there is 
a unique locally free sheaf $\sF$ on $C$ and a short exact sequence
$$
0\to \sO_{\pi^{-1}(p)} \to \pi^{*} \sF_{0} \to \sF \to 0.
$$
Thus $\chi(\sF) = \chi(\sF_{0}) -2$, so 
$$
\deg \sF = \chi(\sF) - \chi(\widetilde \sO_{C}) = 
\deg(\sF_{0})-1.
$$
Moreover the map $H^{0}(\sF_{0}) \rTo^{\pi^{*}} H^{0}(\sF(-q))$ is a monomorphism. 
\qed
\end{corollary}

\begin{proof}[Proof of Theorem~\ref{torsion free at node}] 
%We may work over the local ring $R_0 := \cO_{C_{0},p}$, and replace $\cF$ with 
%its localization $I\subset R_0$ at $p$. Write
%$\gm_0$ for the maximal ideal of $R_0$ which by Proposition~\ref{conductor of node and cusp} is also an ideal of $S$.
%We must show that $I$ is isomorphic as an $R_0$-module to either $R_0$ or $\gm_0$.

\def\End{{\rm End}}

Consider the endomorphism ring of $I_0$, and note that it is commutative, contains $R_0$,  and is integral over $R_0$ so 
$$
R_0 \subset \End(I_0) \subset R.
$$
Since
$R/R_0 \cong k$, the ring $\End(I_0)$ is equal to either 
$R_0$ or $R$. 

First, suppose
$\End(I_{0})=R$, which is a discrete valuation ring. Every ideal of $R$ is principal, so we may write $I_{0} = Ra$.
 Since $\gm_0$ is also an ideal of $R$, it is isomorphic to $R$
as $R$-module, and since $R_0\subset R$,
$I \cong \gm_0$ as $R_0$-modules. Moreover, after completing we have $\widehat R_{0} \cong \CC[[t^{2}, t^{3}, \dots]]$ and it is evident that if $a$ has valuation $d$ in $R$ then the length of $R_{0}/(aR)$ is $d-1$, proving the claims in this case.

On the other hand., suppose
$\End(I)=R_0$.
 and consider the inclusions
$$
\gm_0 I \subset I \subset R I.
$$
%We have
%$$
%\gm_0 I = (\gm_0 R) I =  \gm_0 (R I) \cong \gm_0 R \cong R.
%$$
%Thus
%$$
%\gm_0 I \subset I \subset R.
%$$
The left and right hand modules both have endomorphism ring $R$,
so both containments must be strict. Since $R/\gm_0$ has length 2,
we see that $I/\gm_0 I$ is principal, so $I\cong R_0$.
\end{proof}

Returning once again to the family $\cC \to \Delta$ at the beginning of Section~\ref{Picard family}, assume that for general $t$ the scheme $W^{r}_{d}(C_t)\neq \emptyset$. After replacing $\Delta$ with a ramified covering
we may assume that $\sW^{r}_{d}(\sC) \to \Delta$ has a section, that is, an invertible sheaf $\sL$ on $\sC$
 such that the restriction of $\sL$ to each fiber $C_{t}$ with $t\neq 0$ has at least $r+1$ sections, so the limit of this $g$-cuspidal curve $C_0$ has a torsion-free sheaf $\cL_0$ of degree $d$. By the semi-continuity
 of cohomology, $\sL_{0}$ also has at least $r+1$ sections. 

At each cusp $r_i$ of $C_0$, $\cL_0$ is either locally free or isomorphic to the maximal ideal of the cuspidal curve. Thus, if we denote by $p_1,\dots, p_k$ the cusps of $C_0$ at which $\cL_0$ is isomorphic to the maximal ideal, and let $C'_{0}$ be the partial normalization of $C_0$ at those cusps, then $C'_{0}$ will be a curve of arithmetic genus $g-k$ having $g-k$ cusps, and the pullback of $\cL_0$ to $C'_{0}$ has at least $r$ sections with base points at the points lying over $p_1,\dots,p_k$;  this is a $g^r_{d-k}$ on the $(g-k)$-cuspidal curve  $D_0$.



\section{Putting it all together}

\subsection{Non-existence}

We can now prove that if $\rho(g,r,d) < 0$  then $W^{r}_{d}(C) = \emptyset$ for curves in an open dense subset of $M_{g}$---that is, a general curve of genus $g$ does not posses a $g^r_d$.
Here we use Theorem~\ref{osculating intersection} from the last chapter, which implies exactly this statement for invertible sheaves on an arbitrary $g$-cuspidal curve $C_0$. For if it were the case that a general curve possessed a $g^r_d$ with $\rho(g,r,d) < 0$, then by the results of Section~\ref{Picard family}, there would be
 an invertible sheaf of degree $d$ with $\geq r+1$ sections on a $(g-k)$-cuspidal curve  $ C'_0$ for some $k \geq 0$. But since
$$
\rho(g-k, r, d-k) = \rho(g,r,d) - k < 0
$$
this is impossible.

\subsection{Existence}

Once more we let $\cC \to \Delta$ be a family of smooth curves specializing to a $g$-cuspidal curve $C_0$. We can use Corollary~\ref{intersection with sigma nonzero} in combination with Theorem~\ref{osculating intersection} to say that the variety $W^r_d(C_0)$ is nonempty of dimension $\rho(g,r,d)$; since the codimension of $\cW^r_d(\cC/\Delta) \subset \pic_d(\cC/\Delta)$ is at most $(r+1)(g-d+r)$ everywhere, any we may conclude that likewise for general $t$ the variety $W^r_d(C_t)$ is nonempty of dimension exactly $\rho(g,r,d)$. 

\section{Brill-Noether with inflection}

The approach we've taken here to the proof of Brill-Noether is well-suited to analyzing the inflectionary behavior of linear series on a general curve; indeed, a small modification of the argument above allows us to prove a stronger form of the Brill-Noether statement, concerning the existence of $g^r_d$s on a general curve $C$ that are required to
 have inflection points of given weights.

\begin{definition}
Let $C$ be a smooth curve of genus $g$ and $p_1,\dots,p_n \in C$ distinct points of $C$. If $\cD = (L,V)$ is a linear system on $C$ of degree $d$ and dimension $r$, we define the \emph{adjusted Brill-Noether number} of $\cD$ relative to the points $p_k$ to be
$$
\rho(\cD; p_1,\dots,p_k) := g - (r+1)(g-d+r) - \sum_{k=1}^n w(\cD,p_k).
$$
\end{definition}

In these terms, we can prove the following generalization of the ``nonexistence" half of Brill-Noether:

\begin{theorem}\label{Brill-Noether with inflection}
Let $(C;p_1,\dots,p_n)$ be a general $n$-pointed curve of genus $g$ (that is, let $C$ be a general curve and $p_1,\dots,p_n \in C$ general points. If $\cD$ is any linear system on $C$, then
$$
\rho(\cD; p_1,\dots,p_k) \geq 0.
$$
\end{theorem}

\begin{proof}
To start, let $\cC \to B$ be a family of curves as in the proof of Lemma~\ref{cusp smoothing lemma}. Let $\sigma_1, \dots, \sigma_n : B \to \cC$ be sections of $\cC \to B$ with $\sigma_k(0)$ a smooth point of $C_0$ for all $k$ (such sections can always be found after passing to an analytic open neighborhood of $0 \in B$ 
Exactly as in the proof of Lemma~\ref{BN in family}, if the general curve $C_b$ in our family admits a $g^r_d$ $\cD$ with
$$
\rho(\cD;\sigma_1(b),\dots,\sigma_n(b)) < 0
$$
we can choose a family $\{\cD_b\}$ of such linear series 
on the fibers $C_b$ for $b \neq 0$ and, taking limits, we arrive at a $g^r_d$ $\cD_0$ on $\PP^1$ with
$$
w(\cD_0, q_i) \geq r
$$
for each of the $g$ points $q_i \in \PP^1$ lying over the cusps of $C_0$, and in addition
$$
w(\cD_0, r_k) \geq w(\cD_b,\sigma_k(b))
$$
where $r_k \in \PP^1$ is the point in $\PP^1$ lying over $\sigma_k(0) \in C_0$. Adding up, we have
\begin{align*}
\sum_{i=1}^g w(\cD_0, q_i) + \sum_{k=1}^n w(\cD_0, r_i) &\geq rg + \sum_{k=1}^n w(\cD_b,\sigma_k(b)) \\
&> rg + g - (r+1)(g-d+r) = (r+1)(d-r)
\end{align*}
since we assumed that 
$$
\rho(\cD_b;\sigma_1(b),\dots,\sigma_n(b)) = g - (r+1)(g-d+r) - \sum_{k=1}^n w(\cD_b,\sigma_k(b)) < 0.
$$
But as before the Pl\"ucker formula for $\PP^1$ tells us that
$$
\sum_{p \in \PP^1} w(\cD_0, p) = (r+1)(d-r),
$$
a contradiction.
\end{proof}

This extension of the ``nonexistence" part of Brill-Noether raises the question of a converse: if $(C;p_1,\dots,p_n)$ is a general $n$-pointed curve of genus $g$, and we specify ramification sequences $\alpha^1, \dots, \alpha^n$, can we say that there exists a $g^r_d$ $\cD$ on $C$ with $\alpha_i(\cD, p_k) \geq \alpha^k_i$ for $k=1,\dots,n$ and $i = 0, \dots, r$? If the product of the corresponding Schubert classes in $G(d-r, d+1)$ is nonzero, then we can indeed assert the existence of such a linear series; and if the product is 0, we can deduce that no such linear series exists.

Here is one way to state what we know without getting lost in the thicket of Schubert calculus:

\begin{theorem}\label{BN with inflection and dimension}
Let $C$ be a smooth curve of genus $g$ and $p_1,\dots,p_n \in C$ distinct points; for $k = 1,\dots,n$ let $\alpha^k = (\alpha^k_0,\dots\alpha^k_r)$ be a nondecreasing sequence of nonnegative integers, and let
$$
G^r_d(p_1,\dots,p_n; \alpha^1,\dots,\alpha^n) = \{\cD \in G^r_d(C) \mid \alpha_i(\cD, p_k) \geq \alpha^k_i \}.
$$
If $(C, p_1,\dots,p_n)$ is a general $n$-pointed curve, then either $G^r_d(p_1,\dots,p_n; \alpha^1,\dots,\alpha^n)$ is empty or
$$
\dim G^r_d(p_1,\dots,p_n; \alpha^1,\dots,\alpha^n) = \rho(g,r,d) - \sum_{k+1}^n \sum_{i=0}^r \alpha^k_i.
$$
\end{theorem}

Finally, from the codimensions of the Schubert varieties  deduce the ramification behavior of general
linear series:

\begin{theorem}
If $\cD$ is a general $g^r_d$ on a general curve, then $\cD$ has only simple ramification; that is,	
$$
w(\cD, p) \leq 1 \quad \text{for all } p \in C.
$$
\end{theorem}

Applying this in case $d=2g-2$ and $r = g-1$, we arrive at the statement made in Section~\ref{Weierstrass points}: that a general curve $C$ of genus $g$ has only Weierstrass points of weight 1.

\section{Exercises}


Here is a geometric way to construct a curve $C_0$ by identifying $k$ pairs of points on a smooth curve $C$ of genus $h$: We embed the curve $C$ in projective space by the sections of a invertible sheaf of high degree $d$, and then we project.
%\fix{There is a commented out "Constructing nodal curves"}

\begin{exercise}\label{independent secants} Let $C \subset \PP^N$ be a smooth curve of genus $h$, embedded in projective space by the complete linear system associated to a invertible sheaf $\cL$ of degree $d > 2g + 2k$. Show that any $k+1$ secant or tangent lines to $C$  having pairwise disjoint intersection with $C$ are linearly independent; that is, they span a $\PP^{2k-1}$.
\end{exercise}


\begin{exercise}
  By Theorem~\ref{BN with inflection and dimension}, there is a $g^1_4$ on a general curve of genus 3 ramified at any 2 general points. Construct it.
\end{exercise}

The next series of exercises shows that, like curves with arbitrary sets of $g$ ordinary cusps,
curves with $g$ \emph{general} nodes also satisfy Theorem~\ref{BN-basic}; in fact these were
used in the original proof in ~\cite{Griffiths-Harris-BN}. We will deduce this from the 
cuspidal case; the difficulty of dealing with the (necessary)
condition that the nodes be general makes a direct proof much harder.

\begin{exercise}\label{construction of nodal curves}
Prove the following existence result:

\begin{proposition} \label{Constructing nodal curves}
 Given a (possibly singular) curve $C$ and a pair of distinct smooth points $p,q\in C$, there is a map
 $C\to C_0$ that is an isomorphism away from $p,q$ and identifies $p,q$ to a point $r_{i}\in C_{0}$
 in such a way
 that the completion of the local ring at $r_{i}$ satisfies
 $$
\widehat\cO_{C_{0}, r_i} \cong k[[x,y]]/(xy).
$$
\end{proposition}

Hint:
We may suppose that the curve $C$ is affine.
For each $i$ we let $\cO_{r_i}$  be the set of germs of sections of $\cO_C$ that are
 defined at both $p_i$ and $q_i$, and have the same value. Thus, regarding everything
 as subsets of the quotient field of $\sO_{C}$, and identifying $\sO_{C}$ with its ring of global sections,
 $$
 \cO_{r_i} = k+(\gm_{\sO_{C},p_i} \cap \gm_{\sO_{C},q_i}) \subset \cO_{C,p_i}\cap \cO_{C,q_i}.
 $$
 Finally, we set 
 $$
 \cO_{C_0} = \sO_{C}\bigcap_{i=1}^g \cO_{r_i}.
 $$
Show that $C_{0} := \Spec \sO_{C_{0}}$ is a curve with the desired properties.
\end{exercise}

\begin{exercise}\label{linear series on a nodal curve}
Imitate the proof of Proposition~\ref{torsion-free on cuspidal} to show that if $R_{0}$ is the local ring of a node---that is, the completion $\widehat R$ is isomorphic to $\CC[x,y]/(xy)$---then every nonzero ideal of $R$ is isomorphic
either to $R$ or to the maximal ideal of $R$, Deduce that if $C\to C_{0}$ is the
partial normalization of a single node on a reduced irreducible projective curve, then there is an exact sequence
$0\to \CC^{*} \to \pic_{0}(C_{0}) \to \pic_{0}(C) \to 0$.

Hint:
To start, let $C$ be a curve with a node $p$, and $\widetilde C \rTo^\nu C$ the normalization of $C$ at $p$; we'll denote by $q,r \in \widetilde C$ the points lying over $p$. If $\cL$ is an invertible sheaf on $C$, and $\cM := \nu^*(\cL)$ the pullback of $\cL$ to $\widetilde C$, then $\cM$ is an invertible sheaf on $\widetilde C$. Its fibers over $q$ and $r$ are both identified with the fiber $\cL_p$ of $\cL$ at $p$, and hence with each other. Conversely, given an invertible sheaf $\cM$ on $\widetilde C$ and an identification of the fibers $\cM_q$ and $\cM_r$, we can form an invertible sheaf $\cL$ on $C$ by taking the subsheaf of $\nu_*\cM$ whose sections ``agree" at $q$ and $r$, in terms of the identification. 
\end{exercise}
 
 \fix{add:}
\begin{exercise}\label{secant general position}
 
\end{exercise}
%\section{potential exercises}
%\begin{enumerate}
%\item Show that there is a unique quadric in $\PP^3 containing 3 skew lines.
%\item use the previous to show that there are exactly 2 lines meeting each of 4 general lines
%\item use the previous and
% \item Use Castelnuovo's specialization to compute a lower bound for the number of $g^1_3$s on a curve of genus 4.
%
%\end{enumerate}

%footer for separate chapter files

\ifx\whole\undefined
%\makeatletter\def\@biblabel#1{#1]}\makeatother
\makeatletter \def\@biblabel#1{\ignorespaces} \makeatother
\bibliographystyle{msribib}
\bibliography{slag}

%%%% EXPLANATIONS:

% f and n
% some authors have all works collected at the end

\begingroup
%\catcode`\^\active
%if ^ is followed by 
% 1:  print f, gobble the following ^ and the next character
% 0:  print n, gobble the following ^
% any other letter: normal subscript
%\makeatletter
%\def^#1{\ifx1#1f\expandafter\@gobbletwo\else
%        \ifx0#1n\expandafter\expandafter\expandafter\@gobble
%        \else\sp{#1}\fi\fi}
%\makeatother
\let\moreadhoc\relax
\def\indexintro{%An author's cited works appear at the end of the
%author's entry; for conventions
%see the List of Citations on page~\pageref{loc}.  
%\smallbreak\noindent
%The letter `f' after a page number indicates a figure, `n' a footnote.
}
\printindex[gen]
\endgroup % end of \catcode
%requires makeindex
\end{document}
\else
\fi
