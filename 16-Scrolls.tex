% !TEX TS-program = pdflatexmk
%header and footer for separate chapter files

\ifx\whole\undefined
\documentclass[12pt, leqno]{book}
\usepackage{graphicx}
\usepackage{eps-to-pdf}
\input style-for-curves.sty
%\input sl-macros.sty
\usepackage{hyperref}
\usepackage{showkeys} %This shows the labels.
\usepackage{msribib}
\usepackage{pdfpages}
\usepackage{draftwatermark}
\SetWatermarkText{DRAFT:\ \today}
\SetWatermarkScale{2}
\SetWatermarkColor[gray]{0.9}

%\usepackage{SLAG,msribib,local}
%\usepackage{amsmath,amscd,amsthm,amssymb,amsxtra,latexsym,epsfig,epic,graphics}
%\usepackage[matrix,arrow,curve]{xy}
%\usepackage{graphicx}
%\usepackage{diagrams}
%
%%\usepackage{amsrefs}
%%%%%%%%%%%%%%%%%%%%%%%%%%%%%%%%%%%%%%%%%%
%%\textwidth16cm
%%\textheight20cm
%%\topmargin-2cm
%\oddsidemargin.8cm
%\evensidemargin1cm
%
%%%%%%Definitions
%\input preamble.tex
%\input style-for-curves.sty
%\def\TU{{\bf U}}
%\def\AA{{\mathbb A}}
%\def\BB{{\mathbb B}}
%\def\CC{{\mathbb C}}
%\def\QQ{{\mathbb Q}}
%\def\RR{{\mathbb R}}
%\def\facet{{\bf facet}}
%\def\image{{\rm image}}
%\def\cE{{\cal E}}
%\def\cF{{\cal F}}
%\def\cG{{\cal G}}
%\def\cH{{\cal H}}
%\def\cHom{{{\cal H}om}}
%\def\h{{\rm h}}
% \def\bs{{Boij-S\"oderberg{} }}
%
%\makeatletter
%\def\Ddots{\mathinner{\mkern1mu\raise\p@
%\vbox{\kern7\p@\hbox{.}}\mkern2mu
%\raise4\p@\hbox{.}\mkern2mu\raise7\p@\hbox{.}\mkern1mu}}
%\makeatother

%%
%\pagestyle{myheadings}

%\input style-for-curves.tex
%\documentclass{cambridge7A}
%\usepackage{hatcher_revised} 
%\usepackage{3264}
   
\errorcontextlines=1000
%\usepackage{makeidx}
\let\see\relax
\usepackage{makeidx}
\makeindex
% \index{word} in the doc; \index{variety!algebraic} gives variety, algebraic
% PUT a % after each \index{***}

\overfullrule=5pt
\catcode`\@\active
\def@{\mskip1.5mu} %produce a small space in math with an @

\title{A Chapter from ``The Practice of Algebraic Curves"}
\author{\copyright David Eisenbud and Joe Harris}
%%\includeonly{%
%0-intro,01-ChowRingDogma,02-FirstExamples,03-Grassmannians,04-GeneralGrassmannians
%,05-VectorBundlesAndChernClasses,06-LinesOnHypersurfaces,07-SingularElementsOfLinearSeries,
%08-ParameterSpaces,
%bib
%}

\date{\today}
%%\date{}
%\title{Curves}
%%{\normalsize ***Preliminary Version***}} 
%\author{David Eisenbud and Joe Harris }
%
%\begin{document}

\begin{document}
\maketitle

\pagenumbering{roman}
\setcounter{page}{5}
%\begin{5}
%\end{5}
\pagenumbering{arabic}
\tableofcontents
\fi


\chapter{Scrolls and the Curves They Contain}
\label{ScrollsChapter}


\begin{verbatim}
 The naming of cats is a difficult matter,
 It isn't just one of your everyday games.
 You may think that I am as mad as a hatter,
 When I tell you each cat must have three different names.
 The first is the name that the family use daily ...
 But I tell you, a cat needs a name that's particular ...
 But above and beyond there's still one name left over,  ...
 [his] deep and inscrutable, singular name.
\end{verbatim}
--T. S. Eliot, Old Possum's Book of Practical Cats\footnote{\cite{PracticalCats}}

\section*{}
Some of the simplest subvarieties in projective space are the \emph{rational normal scrolls}. They appear in many contexts in algebraic geometry, and are useful for describing the embeddings of curves of low degree and genus. 

We begin this chapter by giving three different characterizations of these varieties, each useful in a different context: First a classical geometric construction, then an algebraic description that allows one to ``find" the scrolls containing a given variety, and then a more modern geometric definition that makes it easy to understand the divisors on a scroll. Finally, we turn to some of the applications to the embeddings of curves. We will focus on the 2-dimensional case because this is the case that occurs in our applications.

In this chapter we will refer to rational normal scrolls simply as scrolls. The third characterization we will give lends itself to a natural generalization to  irrational ruled varieties, In the literature the word ``scroll'' is often used for this wider class.

\section{Some classical geometry (the name the family use daily)}\label{daily name}

To construct a scroll of dimension 2 in $\PP^n$, we start by choosing integers $0\leq a_1 \leq a_2$ with $a_1 + a_2 = n-1$, and consider  a pair of complementary linear subspaces $\PP^{a_1}$ and $\PP^{a_2} \subset \PP^n$---that is, we express an $(n+1)$-dimensional vector space as a direct sum $ V_1 \oplus V_2$ of subspaces $V_1, V_2 \subset V$ of dimensions $a_1+1$ and $a_2+1$, and let $\PP^{a_1} = \PP V_1$ and $\PP^{a_2} = \PP V_2 \subset \PP (V_1\oplus V_2)$.

Next, for $i=1,2$, we take $\phi_i : \PP^1 \to \PP^{a_i}$ to be the parametrization of the rational normal curve of degree $a_i$ given by a basis of homogeneous polynomials of degree $a_i$ (if $a_i = 0$ this is just the constant map from $\PP^1$ to a point.) Finally, we define the scroll $S(a_1, a_2)$ to be the union of the lines
$$
S(a_1,a_2) := \bigcup_{t\in \PP^1} \overline{\phi_1(p), \phi_2(p)}.
$$

If $a_1\leq a_2$ we call the curve $C_{a_{1}}$ the \emph{directrix} of the scroll, and we call the lines $ \overline{\phi_1(p), \phi_2(p)}$ the \emph{rulings} of the scroll. For example we can realize a smooth quadric surface in $\PP^3$ as the union $S(1,1)$ of the lines joining corresponding points on two skew lines. 

\fix{Insert picture!}

In the degenerate case $a_{1}= 0$, the surface $S(0,a_{2})$ is the cone
in $\PP^{a_{2}+1}$ over a rational normal curve of degree $a_{2}$. Since $S(0,a_2)$ is singular when $a_2\geq 2$, it is useful to consider the surface
$$
\tilde S(0, a_2) := \left\{ (t, q) \in \PP^1 \times \PP^n  \mid q \in \overline{\phi_1(t), \phi_2(t)}\right)\}.
$$
This is the blow-up of the cone $S(0, a_2)$ at its vertex; like the surfaces $S(a_1,a_2)$ with $a_1 > 0$ it is a $\PP^1$ bundle over $\PP^1$ and thus is smooth. As we shall see, $\tilde S(0, a_2)$ is isomorphic to the scroll $S(1, a_2+1)$.

It is not hard to prove directly that $S(a_1,a_2)$ is an algebraic variety, and we shall soon write down its defining equations.

From the description above we can immediately deduce the dimension and degree of a scroll:

\begin{proposition}
\begin{enumerate}
\item $S(a_1,a_2)$ is a nondegenerate surface.
 \item $S(a_1,a_2)$ has degree $a_1+a_2$, and codimension $a_1+a_2-1.$
 \item $S(a_{1},a_{2})$ is non-singular if $0<a_{1}, a_{2}$.
 \end{enumerate}
\end{proposition}\label{deg and codim}

\begin{proof}
 The rational normal curves separately span the spaces $\PP^{a_i}$, so a hyperplane containing both of them would contain $\overline{\PP^{a_1}, \PP^{a_{2}}} = \PP$, proving nondegeneracy. 
 
 It is clear from our description that $S$ is 2-dimensional, and thus of
codimension $a_{1}+a_{2}+1 -2 = a_{1}+a_{2}-1$. 

To compute the degree, we choose a general hyperplane $H$ containing $\PP^{a_{1}}$. The intersection $H\cap C_{2}$ consists of $a_{2}$ reduced points. Thus the intersection $H\cap S$ consists of $C_{1}$ and the $a_{2}$ reduced lines connecting 
the points of $H\cap C_{2}$ with their corresponding points on $C_{1}$; this union has degree $a_{1}+a_{2}$.

If $0< a_{1}$ we also see from this argument that, given any point  $p\in S(a_{1},a_{2})$, there is
a hyperplane section that is non-singular at $p$, and thus $S(a_{1},a_{2})$ is nonsingular at $p$.
\end{proof}

A completely parallel construction creates rational normal scrolls of dimension $r$. Start with a series of integers $0 \leq a_1 \leq \dots \leq a_r$;
set $N = \sum_{i=1}^{r}(a_{i}+1)$,  and
decompose $\CC^{N}$ as
$$
\CC^{N} = \bigoplus_{i=1}^{r}\CC^{a_{i}+1}.
$$
Let $\PP^{a_{i}}\subset \PP^{N-1}$ be the subspaces corresponding to the summands,  choose
maps $\phi_i : \PP^1 \to \PP^{a_{i}}$ be a map given by a basis of homogeneous polynomials of degree $a_i$, and define the scroll $S \subset \PP^{N-1}$ by
$$
S:=S(a_{1}, \dots, a_{r}) = \bigcup_{p\in C_{1}}\overline{\phi_1(p), \phi_{2}(p), \dots, \phi_{r}(p)}.
$$
The variety $S$ is nondegenerate of codimension $N-1-r$ and degree $\sum a_{i} = N-r$. The proof is similar to the one we gave for $r=2$.

%As in the surface case, if one or more of the indices $a_i$ are equal to 0, the resulting variety $S$ is a cone. In these cases, it is again useful in some settings to introduce the blow-up of this cone along its vertex, which we can realize as the variety
%$$
%\tilde S := \left\{ (t, q) \in \PP^1 \times \PP^n  \mid q \in \overline{\phi_1(t), \dots, \phi_r(t)}\right)\}.
%$$

To put this construction in context, we recall an elementary fact of projective geometry:
 
\begin{proposition}\label{minimal degree}
 Any irreducible, nondegenerate variety $X$ of codimension $c$ in $\PP^{N}$ has degree $\geq c +1$.
\end{proposition}

\begin{proof} We do induction on $\dim X$. If $H\cong \PP^{N-1}\subset \PP^N$ is a hyperplane then $H\cap X$ spans
$H$ by Proposition~\ref{arbitrary hyperplane}. Bertini's Theorem shows that if $H$ is a general hyperplane then $H\cap X$ is either an irreducible variety of 
(if $\dim X\geq 2$) or a set of reduced points. In the former case we are done by induction. In the latter case
we appeal to the fact that fewer than $c+1$ points could not span $\PP^c$.
 \end{proof}

Thus scrolls are \emph{varieties of minimal degree}. The reader already knows that the rational normal curves of degree $a$ in $\PP^{a}$ are the only irreducible, nondegenerate curves of degree $a$ and codimension $a-1$. A celebrated theorem of del Pezzo (for surfaces) and Bertini (in general) generalizes this statement:

\begin{fact}\label{classification of scrolls} 
Any irreducible, nondegenerate variety $X\subset \PP^{N}$  with $\deg X = \codim X+1$, is either a quadric hypersurface, a scroll, the Veronese surface in $\PP^{5}$, or a cone over the Veronese surface.
\end{fact}

A proof may be found in \cite{Eisenbud-Harris-Centennial}.

One interesting way to view the construction of a scroll is that we chose subvarieties $C_{i}\subset \PP^{a_{i}}$ and a one-to-one correspondence between them, that is, a subscheme
$\Gamma\subset \prod_{i}C_{i}$ that projects isomorphically onto each $C_{i}$; the scroll is then the
union of the planes spanned by sets of points $p_{i}\in C_{i}$ that are ``in correspondence''. There are other interesting varieties constructed starting with other choices of subvarieties $C_{i}$ and subschemes---not necessarily reduced---of $\prod_{i}C_{i}$. See \cite{Eisenbud-Sammartano} for an exploration of this idea.

We tend to speak of ``the'' rational normal scroll rather than ``a'' rational normal scroll'', despite the choices made in the definition, for the following reason:

\begin{proposition}\label{uniqueness of scrolls}
The scroll $S(a_1,a_2)$ is, up to a linear automorphism of $\PP^{a_1+a_2+1}$, independent of the choices made in its
 definition. 
\end{proposition}

\begin{proof} 
To simplify the notation, set $S := S(a_{1}, a_{2})$ and $\PP := \PP^{a_1+a_2+1}$.
To construct $S$ we chose 
\begin{enumerate}
 \item disjoint subspaces $\PP^{a_i}\subset \PP$;
 \item a rational normal curve in each subspace; and
 \item an isomorphism between these curves.
\end{enumerate}
Elementary linear algebra shows that there are automorphisms of $\PP$ carrying any choice of disjoint subspaces to any other choice. Further, since the rational normal curve of degree $a$ is unique up to an automorphism of $\PP^{a}$, the choice in (2) can be undone by a linear automorphism. Finally, any automorphism of $C_{a_{2}}\cong \PP^{1}$ extends to an automorphism of $\PP^{a_{2}} = |\cO_{\PP^{1}}(a_{2})|$, and this extends to an automorphism of $\PP$ fixing $\PP^{a_{1}}$ pointwise,
showing that $S(a_{1}, a_{2})$ is independent, up to an automorphism of the ambient space, of the choice in (3)  as well.
\end{proof}



\section{1-generic matrices and the equations of scrolls
(the name that's particular)}\label{particular name}

Suppose that a scheme $X $ is embedded in $PP^n$ by a complete linear series, and that
$\sO_X(1)$ can be ``factored'' as a tensor product $\sL\otimes \sM$ of invertible sheaves on $X$. If we pick sets of $p$ independent elements $\{\ell_i\}\subset H^0(\sL)$ and  $q$ independent elements $\{m_i\} \subset H^0(\sM)$ then the multiplication map 
$$
\mu: H^0(\sL) \otimes H^0(\sM) \to H^0(\sO_{\PP^n}(1))
$$
 gives rise to 
a $p\times q$ matrix $M_\mu$ of linear forms on $\PP^n$ whose $i,j$ entry is $\mu(\ell_im_j)$.
More abstractly, this is a linear space of matrices obtained from the ``adjunction'' isomorphism 
$\Hom(A\otimes B, C)\cong \Hom(A, \Hom(B,C))$.

Regarding the sections of invertible sheaves as sheaves of functions on $X$, we see from the commutativity of
multiplication that the $2\times 2$ minors
of 
$$
\det \begin{pmatrix}
\ell_{i_1}m_{j_1} & \ell_{i_1}m_{j_2}\\
\ell_{i_2}m_{j_1} &\ell_{i_2}m_{j_2}  
\end{pmatrix},
$$
vanish on $X$---that is, the ideal of $2\times 2$ minors $I_2(M_\mu)$ is contained in the homogeneous ideal
of $X$. 
We will see that, if $p=2$, then the ideal $I_2(M_\mu)$
is the ideal of a rational normal scroll of codimension $q-1$.

For example, the rational normal curve $C_a\subset \PP^a$ is $X = \PP^1$ embedded by the complete
linear series $|\sO_{\PP^n}(a)|$, and $\sO_{\PP^1}(a) = \sO_{\PP^1}(1)\otimes \sO_{\PP^1}(a-1)$.
If we take bases $s^it^j$ in each of $H^0(\O_{\PP^1}(1)$ and  $H^0(\O_{\PP^1}(a)$ and use the parametrization
$x_i = s^it^{a-i}$ we get
the $2\times a$ matrix
$$
M_\mu := 
\begin{pmatrix}
x_0&x_1&\dots&x_{a-1}\\
x_1&\dots&x_{a-1}&x_a
\end{pmatrix}.
$$
When restricted to $\PP^1$, this becomes
$$
M_a = \bordermatrix{
& s^{a-1}&s^{a-2}t&\dots&t^{a-1}\cr
s&  s^{a}& s^{a-1}t&\dots&st^{a-1}\cr
t&  s^{a-1}t& s^{a-2}t^{2}&\dots&t^{a}\cr
}$$
where we have written $s,t$ for the basis of $H^)(\sO_{\PP^1}(1))$, and bordered the matrix
with the corresponding bases of $H^0(\sO_{\PP^1}(1))$ and $H^0(\sO_{\PP^1}(a-1))$, and it is obvious
that the minors of $M_a$ are 0.

By a \emph{generalized row} of $M_{a}$, we mean a $\CC$-linear combination of the given rows of $M_{a}$. Note that the points at which the $2\times 2$ minors of $M_{a}$ vanish are the points at which the evaluations of the two rows are linearly dependent; that is, the points at which some
generalized row of $M_{a}$ vanishes identically. From Proposition~\ref{RNC generators}, we see that the points of the rational normal curve are exactly the points where all the linear forms in some generalized row
of $M_{a}$ vanish.


The matrix $M_{\mu}$ shares some properties with the generic $p\times q$ matrix:

\begin{definition}
 A matrix of linear forms $M$ is  \emph{1-generic} if every generalized row of $M$
 consists of $\CC$-linearly independent forms.. 
 \end{definition}

 For example, the matrix 
$$
M = \begin{pmatrix}
 x &y\\
 z&x
\end{pmatrix}
$$
over $\CC[x,y,z]$ is  1-generic, since if a row and column transformation produced a 0 the determinant would be a product of linear forms, whereas
$\det M = x^2-yz$ is irreducible. 

On the other hand, the matrix
$$
M' = \begin{pmatrix}
 x &y\\
 -y&x
\end{pmatrix}
$$
over $\CC[x,y]$ is not 1-generic, since
$$
\begin{pmatrix}
1&0\\
-i&1 
\end{pmatrix}
M'
\begin{pmatrix}
 1&0\\
 i&1
\end{pmatrix}
= 
\begin{pmatrix}
 x+iy&0\\
 0&x-iy
\end{pmatrix}
$$
(but note that it would be 1-generic if we restricted scalars to $\RR$---thus the definition depends on the field).

Here is another way of seeing that $M'$ is not 1-generic over $\CC$:

\begin{lemma}\label{variables needed}
  \label{size of 1-generic} There exist 1-generic $p\times q$ matrices of linear forms in $n+1$ variables over $\CC$ if and only if $n\geq p+q$.
In particular, the dimension of the space of linear forms spanned by the entries of a  1-generic matrix $M$ of size $p\times q$ is at least $p+q-1$. Moreover, if this space of linear forms has dimension $>p+q-1$, then the restriction of $M$ to a general hyperplane is still 1-generic.
\end{lemma}
\begin{proof} Consider any map of vector spaces $A\otimes B \to C$, where we regard $C$ as a 
space of linear forms.
With notation as above, if $\ell_i\otimes m_j\in \ker \mu$, then the $i,j$ entry of $M_\mu$ is 0 and similarly for
any \emph{pure} tensor, $\ell\otimes m\in A\otimes B$. Thus $M_\mu$ is 1-generic if and only if the linear subspace
$\ker \mu \subset  A\otimes B$
is disjoint from the set of pure tensors. Under the isomorphism $A\otimes B \cong \Hom (A^*, B)$
the set of pure tensors corresponds to matrices of rank 1, and this set has codimension $(m-1)(n-1) = mn-m-n+1$
in $\Hom(A^*, B)$ (proof: a rank 1 matrix corresponds to the choice of a 1-quotient of $A^*$ and a 1-dimensional subspace
of $B$, thus a point in $\PP^m \times \PP^n$).
Thus
for $\mu$ to correspond to a  1-generic matrix,  we must have $\codim \ker \mu \geq m+n-1$. Since $\codim \ker \mu = \dim \im \mu\subset C$ we see that any 1-generic matrix must involve at least $m-n+1$ variables. 

Furthermore, the restriction of $M_\mu$ to a hyperplane corresponds to the composite homomorphism
$A\otimes B \to C \to C/\langle x \rangle$, or equivalently to the addition of 1 element to $\ker \mu$, and thus
if $M_\mu$ is 1-generic and involves $>m+n-1$ variables, then the restriction to a general hyperplane
is again 1-generic.
\end{proof}

%\begin{proof}
%If we think of a polynomial ring $\CC[z_0,\dots,z_n]$ as the symmetric algebra
%of a vector space $V$ of rank $n+1$, then we may regard a $p\times q$ matrix of
%linear forms $M$ as coming from a map $m: \CC^{p}\otimes \CC^{q}\to V$. The matrix is 1-generic
%if and only if no ``pure'' tensor $r\otimes s$ goes to zero, that is, iff the kernel $K$ of $m$ intersects the cone of
%pure tensors only in 0. The cone of pure tensors is the cone over the Segre embedding of $\PP^{p-1}\times \PP^{q-1}$, 
%and thus has dimension $(p-1)+(q-1)+1$. Thus a general subspace $K$ of codimension $\geq p+q-1$ will intersect the cone
%only in 0, but any larger subspace $K$ will intersect the cone non-trivially,  and the first two statements follow.
%
%Moreover, if $K$ is any space of codimension $>p+q-1$ that intersects the cone only in 0, then the general subspace $K'\subset K$
%of dimension one larger still intersects the cone only in 0, proving the last statement.
%\end{proof}

\begin{proposition}\label{some generators}
Let $X$ be
an irreducible, reduced variety, and suppose that $\sL,\sM$ are invertible sheaves on $X$.
The matrix $M_\mu$ coming from the map $\mu:H^0(\sL) \otimes H^0(\sM) \to H^0(\sL\otimes \sM)$
is 1-generic.
\end{proposition}

\begin{proof} The entries of the generalized row of $M_\mu$ corresponding to $s\in H^0(\sL)$
are a basis of $s\cdot H^0(\sM) \cong H^0(\sM)$, and are thus
linearly independent.
\end{proof}

The 1-generic matrices $M$ of size $2\times a$ have a simple classification. The beginning of the story is the
 calculation of the codimension of the ideal $I_2(M)$ generated by the $2\times 2$ minors of $M$:

\begin{lemma}\label{codim of 2,n 1-generic}
If $M$ is a $2\times a$ matrix of linear forms in $\CC[x_0,\dots, x_n]$, and $M$ is 1-generic, then 
and  $V(I_2(M))$ is irreducible of codimension $a-1$.
\end{lemma}

\begin{proof}
The algebraic set defined by $I_2(M)$ is the set of points on which a generalized row $r_\lambda, \ \lambda\in \PP^1$ of $M$ vanishes.
Because $M$ is 1-generic, each $V(r_\lambda)$ has codimension $a$. Thus $V(I_2(M))$ is fibered by projective
spaces of dimension $n-a$ over $\PP^1$, and is thus irreducible of dimension either $n-a$ or $n-a+1$. In the first
case all the spaces $V(r_\lambda)$  would be equal, contradicting Lemma~\ref{variables needed}.
\end{proof}


We have seen in Proposition~\ref{RNC generators} that the ideal of minors of
$$
M_{a}:= 
\begin{pmatrix}
 x_0&x_1&\dots&x_{a-1}\\
 x_1&x_2&\dots&x_{a}\\
\end{pmatrix}
$$ 
generates the ideal of the rational normal curve of degree $a$ in $\PP^a$, and is thus prime. More generally:

\begin{theorem}\label{1-generic basics}  
Let $I = I_2(M)$  be the ideal generated by the $2\times 2$ minors of  a 1-generic, $2\times a$ matrix $M$
of linear forms in $S = \CC[x_0,\dots, x_n]$.
 \begin{enumerate}

\item The ideal $I$ is prime, and $V(I)$ either is smooth, or is a cone over a smooth variety.

\item If $a=n$ then $V(I)$ is a rational normal curve. More generally, the variety $V = V(I) \subset \PP^n$ has degree $a$ and codimension $a-1$ and is thus a variety of minimal degree.
\end{enumerate}
\end{theorem}

\begin{proof}  By Lemma~\ref{codim of 2,n 1-generic} the set that $V(I)$ has codimension $a-1$.

%By Lemma~\ref{variables needed}, the scheme defined by $I$ spans $\PP^a$.

If the span of the linear forms in $M$ is not the whole space of linear forms on $\PP^n$, then $V(I)$ is a cone,
so we may assume that the entries of $M$ generate the maximal ideal.

 Let $p\in V(I)$ be a point, so that $M$ has a generalized row---without loss of generality the second row---whose 
 entries vanish
at $p$. Since not all the linear forms
of $S$ can vanish at a point of $\PP^n$, one of the entries of the first row, say $\ell_{1,1}$ is nonzero
at $p$. By making row and column transformations we may reduce to the case where
$\ell_{1,j}$ vanishes at $p$ for all $j\neq 1$ and also $\ell_{2,1}$ vanishes at $p$. Since all the minors
vanish at $p$, we see that all the entries $\ell_{2,j}$ must vanish at $p$ as well. Since the entries
of each generalized row are linearly independent, the ideal generated by the entries of the second row define a plane of
$\Lambda$ codimension $a$ containing $p$. 

Over the local ring $\sO_{\PP^n, p}$ at $p$ the element $\ell_{1,1}$ becomes a unit.
Thus the element $\ell_{2,j}$ together with the minors 
$$
\det 
\begin{pmatrix}
\ell_{1,1}& \ell_{1,j}\\
\ell_{2,1}& \ell_{2,j}
\end{pmatrix}
$$
for $j\neq 1$ generate the ideal of $\Lambda$ locally at $p$. Since the products
$\ell_{2,1}\ell_{1,j}$ vanish to order 2 at $p$, we see that the zariski tangent space to the scheme $V(I)$
has codimension at most $1$ in $\Lambda$, and thus codimension $\leq a-1 = \codim (V(I))$ in $\PP^n$. It follows that $V(I)$ is smooth at $p$, and since $V(I)$ is
irreducible, the ideal $I$ is prime.

If $a=n$ then $V(I)$ is a smooth, nondegenerate curve. In Chapter~\ref{HomologicalChapter} we will construct the Eagon-Northcott complex $EN(M)$. By Theorem~\ref{EN}, the complex $EN(M)$ is a free resolution,
and if $M'$ is the ideal of the rational normal curve, then $EN(M')$ is again a resolution. Since the Hilbert
function of $V(I)$ can be computed from the free resolution, it follows that $V(I)$ has the same Hilbert function
as $V(I_2(M)$, and thus $V(I)$ is a rational normal curve.

If $a<n$ then by Lemma~\ref{some generators} there is linear form $\ell$ such that $M$ remains 1-generic modulo $\ell$.
Since $I_2(M)$ is prime it has the same degree and codimenion as $I_2(M) +(x)/(x) \subset S/(x)$, so by induction its degree
is $a$.
\end{proof}

\begin{corollary}\label{equations of scrolls} Let $a_{1}, \dots, a_{r}$ be non-negative integers, and let $N = r-1+\sum_{i=1}^{r} a_{i}$.
The ideal of $S(a_{1},\dots,a_{r})\subset \PP^{N}$ is generated by the $2\times 2$ minors of the matrix
{\footnotesize
$$
\setcounter{MaxMatrixCols}{20}
M = \begin{pmatrix}
x_{1,0}&x_{1,1}&\dots&x_{1, a_{1}-1}&|&x_{2,0}&\dots&x_{2, a_{2}-1}&|&\dots&|&x_{r,0}&\dots&x_{r, a_{r}-1}\\
x_{1,1}&x_{1,2}&\dots&x_{1, a_{1}}.  &|&x_{2,1}&\dots&x_{2, a_{2}}&|&\dots&|&x_{r,1}&\dots&x_{r, a_{r}}
\end{pmatrix}
$$
}
Moreover, the scroll admits a linear projection to each $C_{a_i}$.
\end{corollary}

\begin{proof} We may think of the matrix $M$ as consisting of $r$ blocks, $M_{a_{i}}$. These blocks are 1-generic by Proposition~\ref{some generators}. Since they involve distinct variables, it follows that $M$ is 1-generic. Thus by
Theorem~\ref{}, the ideal $I_{2}(M)$ is prime and of codimension $\sum a_{i}-1$, as is the ideal of the scroll. Thus it suffices to show that the minors of $M$ vanish on the scroll.

Let $C_{i}$ be the rational normal curve in the subspace $\PP^{a_{i}}\subset\PP^{N}$.
As always, the set $V(M)$ is the union of the linear spaces on which generalized rows of $M$ vanish; and each such space is the space spanned by the points in the curves $C_{a_{i}}$ corresponding to the part of that row in the block $M_{a_{i}}$---that is, $V(I_{2}(M))$ is the union of the spans of sets of corresponding points on the $C_{a_{i}}$, as required.
\end{proof}

More is true: 
\begin{theorem}\label{matrix pencils}
 Every
 1-generic $2 \times N$ matrix of linear forms can be transformed by row and column transformations and a linear change
 of variables to one of the type shown in
Corollary~\ref{equations of scrolls}, and thus the minors of any 1-generic matrix defines a scroll. 
\end{theorem}

\begin{proof}[Proof sketch]
A $2\times a$ matrix of linear forms in $N+1$ variables may be thought of as a tensor
in $\CC^{2}\otimes \CC^{a}\otimes \CC^{N+1}$, or, equivalently, as an $a\times (N+1)$ matrix of linear forms in 2 variables. This, in turn is equivalent to a \emph{pencil} (that is, a projective line) in the vector space of scalar $a\times (N+1)$ matrices. Such ``matrix pencils'' were first classified by Kronecker; see 
\cite[Theorems *** and ***]{Gantmacher} for an exposition, and ~\cite{Eisenbud-Harris-Centennial} for a geometric approach.
\end{proof}




\section{Scrolls as Images of Projective Bundles (the deep and inscrutable name)}\label{inscrutable name}

Our third description of scrolls is that they are the images of projective space bundles on $\PP^{1}$ under the map given by the complete series associated to the tautological line bundle. When both $a_{1}$ and $a_{2}$ are strictly positive, we will see that this is an embedding. For simplicity we focus on the 2-dimensional case; the case of higher dimensional scrolls is similar.

Fix integers $a_1\leq a_2$, and let $\sE = \sO_{\PP^1}(a_1) \oplus \sO_{\PP^1}(a_2)$. Recall (from \cite[***]{Hartshorne1977}) that the projectivized vector bundle
$\pi: X := \PP(\sE) \to \PP^1$ is a ruled surface: over each point of $\PP^1$ the fiber is $\pi: \PP(\CC\oplus \CC) = \PP^1$. 
Furthermore, tensoring
$\sE$ with an invertible sheaf $\sO_{\PP^1}(b)$ on $\PP^1$ does not change $X$, so the isomorphism type of the surface depends only on the difference $a_2-a_1$, but $\sO_{\PP(\sE(b))}(1) = \sO_{\PP(\sE)}(1) \otimes \pi^*(\sO_{\PP^1}(b)$
(\cite[****]{Hartshorne1977}). The Picard group of $\PP(\sE)$ is isomorphic to $\ZZ \oplus \ZZ$
and is generated by $\sO_{\PP(\sE)}(1)$ and $\pi^*(\sO_{\PP^1}(1))$.

\begin{theorem}
Let 
$$
\sE = \sO_{\PP^1}(a_1) \oplus \sO_{\PP^1}(a_2)
$$
and suppose that $0\leq a_1\leq a_2$. Let $X = \PP(\sE)$  be the corresponding $\PP^1$ bundle over $\PP^1$
and let $\pi: X \to \PP^1$ be the natural projection. Set $\sL =   \sO_{\PP(\sE)}(1)$.

The complete linear series $|\sL|$ is base-point free, and is very ample if $a_1>0$. 
Let $\phi:X\to \PP^{a_1+a_2+1} = \PP^N$ be the corresponding morphism. The image of $\phi$ is the rational normal scroll $S(a_1,a_2).$
More explicitly:
\begin{enumerate}
 \item If $C_1, C_2\subset X$ be the curves defined by the vanishing
of the sections of $\sO_{\PP^1}(a_2)$ and  $\sO_{\PP^1}(a_1)$ respectively, then the restriction of 
$\sL$ to $C_i \cong \PP^1$ is $\sO_{\PP^1}(a_i)$, and the restriction of $\sL$ to the fiber $\PP^1$ of $\pi$ is $\sO_{\PP^1}(1)$.
\item The map $\pi:X\to \PP^1$ restricts to an isomorphism on each $C_i$; thus the fibers of $\pi$ meet each $C_i$ in a point. The images of $C_1, C_2$ are contained in disjoint subspaces of $\PP^N$, and the fibers of $\pi$ are mapped 
to lines joining the corresponding points of the $C_i$.
\end{enumerate}
\end{theorem}

\fix{Put in a proof that is mostly easy remarks/quotes from Hartshorne}

\begin{corollary}\cite[****]{Hartshorne1977}
Let $0\leq a_1\leq a_2$ be integers. The divisor class group of the 
rational normal scroll 
$$
X:=S(a_1,a_2) \subset \PP^N = \PP^{a_1+a_2+1}
$$
is generated by the class of the hyperplane section and the class
of a ruling. If $a_1 = 0$, then the blowup of $X$ at its singular point is $S(1, a_2+1)\subset \PP^{N+2}$,
and $C_1\subset S(1, a_2+1)$ is the exceptional divisor. The blowup map $S(1, a_2+1) \to S(0,a_2)$
corresponds to the isomorphism 
$$
\PP(\sO_{\PP^1}(1) \oplus \sO_{\PP^1}(a_2+1)) \to \sO_{\PP^1} \oplus \sO_{\PP^1}(a_2)
$$
induced by tensoring with $\pi^(\sO_{\PP^1}(-1)$.
\end{corollary}

%
%Now consider the complete linear series $|\sO_\PP{\sE}(1)|$, which has dimension $N := a_1+a_2+1$. Restricted to
%$C_{a_i}\cong \PP^1$ this is $\sO_{\PP^1}(a_i)$, which is base-point free. Restricted to a fiber of $\pi$, this is
%$\sO_{\PP^1}(1)$, which is very ample. It follows that $|\sO_\PP{\sE}(1)|$ defines a morphism $X\to \PP^{N}$,
%carrying the curves $C_i$ to rational normal curves in disjoint subspaces of $\PP^N$, and the fibers of $\pi$ to
%lines joining corresponding points of these two curves; that is, the rational normal scroll $S(a_1,a_2)$. Furthermore, 
%$|\sO_\PP{\sE}(1)$ is very ample if $a_1>0$ and maps $C_{a_1}$ to a point if $a_1 = 0$.
% 
%
%We start from the description of 
%$
%X:=S(a_{1}, a_{2}),
%$
%with $0\leq a_{1}, a_{2}$ and not both 0,
%as the vanishing locus of the minors of the matrix
%$$
%\setcounter{MaxMatrixCols}{20}
%M:= M_{a_{1}, a_{2}} = 
%\begin{pmatrix}
%x_{1,0}&x_{1,1}&\dots&x_{1, a_{1}-1}&|&x_{2,0}&\dots&x_{2, a_{1}-1}\\
%x_{1,1}&x_{1,2}&\dots&x_{1, a_{1}}.  &|&x_{2,1}&\dots&x_{2, a_{1}}
%\end{pmatrix}
%$$
%of Section~\ref{particular name}. For $p = (s,t) \in \PP^{1}$ we write
%$R_{p}$ for the locus where the linear forms in $s$ times the first row plus $t$ times the second row, 
%$$
%sx_{1,0}+tx_{1,1}, \dots, sx_{2, a_{1}-1}+ tx_{2, a_{1}},
%$$
%all vanish, so that $R_{p}$ is a ruling of $X$ in $\PP^{N}$
%
%
%\begin{theorem}\label{scroll as proj}
%Let $X = S(a_{1}, a_{2})\subset \PP^{N}$, with $N = a_{1}+a_{2}+1$, and assume $0<a_{1}, a_{2}$. The rulings $R_{p}$ of $X$ are the preimages of points under a morphism $\pi: X\to \PP^{1}$. Furthermore, the line bundle 
%$$
%\sL := \sO_{\PP^{N}}(1)|_{X}
%$$ 
%restricts to $\sO_{\PP^{1}}(1)$ on each $R_{p} \cong \PP^{1}$, and the pushforward
%$\sE := \pi_{*}\sL$ is isomorphic to 
%$\sO_{\PP^{1}}(a_{1})\oplus \sO_{\PP^{1}}(a_{2})$. 
%
%Moreover, $X\cong \PP(\cE)$ and the embedding $X\subset \PP^N$ corresponds to the complete linear series $|\cO_{\PP(\cE)}(1)|$.
%\end{theorem} 
%
%In Proposition~\ref{singular scrolls} we will extend this to singular scrolls, showing that the scroll $S(0,a)$, 
%\fix{add?: is the image of $\PP(cE)$ for $E = \sO_{\PP^1}\oplus \sO_{\PP^1}(a)$,
%and is} which is the cone over the rational normal curve 
%of degree $a$, is the image of $S(1,a+1)$ under a map that blows down the line $C_{1}$.
%
%\begin{proof} We write $M$ for the matrix $M_{a_{1}, a_{2}}$.
%Since there are $N+1$ independent entries of $M$ the intersection
%of $R_{p}$ with $R_{q}$ is empty when $p\neq q$, so there is at least a set-theoretic map $X\to \PP^{1}$ sending the points of $R_{p}$ to $p$. To see that this is really a morphism, consider the sheaf
%$$
%\sL = \coker \phi: \sO_{\PP^{N}}(-1)^{a+b} \to \sO_{\PP^{N}}^{2}
%$$
%given by the matrix $M$. Let $p,q$ be distinct points of $\PP^{1}$ and  let
%$\tilde p$ be a point in the ruling $L_p$. Since $L_q$ is disjoint from $L_{p}$, some linear form in the generalized row corresponding to $q$ does not vanish at $p$. Thus
%the restriction of $M$ to a neighborhood of the point $p$ is equivalent to the matrix
%$$
%\begin{pmatrix}
%0&0&\dots&0 \\
%1&0&\dots&0 
%\end{pmatrix}
%$$
%so $\sL$ becomes free of rank 1 when restricted to this neighborhood; that is, $\sL$ is an invertible sheaf.
%
%The two basis vectors of $\sO_{\PP^{N}}^{2}$ map to two global sections
%$\sigma_{1},\sigma_{2}$ of $\sL$. By the argument above these two sections generateß¡ $\sL$ locally everywhere on $X$, and indeed $\sigma_{1}$ fails to generate $\sL$ locally precisely at the points where the second row of $M$ vanishes. Thus the linear series
%defined by these sections corresponds to a morphism $\pi: X \to \PP^{1}$ whose fibers are exactly the rulings of $X$. 
%
%\fix{it seems from here on $\sL$ should be the restriction of $\sO_{\PP^N}(1)$, different (?) from the $\sL$ that defines
%the map to $\PP^1$.}
%
%Because  $L_{p}$ is a linear space, the general hyperplane in $\PP^{N}$
%meets $L_{p}$ in a point; that is $\sO_{\PP^{N}}(1)$ restricts to $\sO_{\PP^{1}}(1)$ as claimed.
%
%Since $\sL$ is an invertible sheaf and $X$ is a variety, $\sL$ is flat over $\PP^{1}$. Since the restriction of $\sL$ to each fiber is $\sO_{\PP^{1}}(1)$, which has two global sections, we see that $\sE:= \pi_{*}\sL$ is a vector bundle of rank 2 \cite[Theorem III.12.9]{Hartshorne1977}.
%Moreover, since $X\subset \PP^{N}$ is non-degenerate, we see that $\sL$, and therefore also $\sE$, has at least $N+1 = a_{1}+a_{2}+2$ independent global sections. \fix{?? I could see this being true of the first twist of the subsheaf, that is,
%$\sL^{-1}(1)$...}
%
%Now consider the directrices $C_{i}:= C_{a_{i}}\subset X$.  The restriction
%$\pi|_{C_{i}}$ is an isomorphism inverse to the parametrization $\PP^{1} \to C_{i}$, and $\sO_{\PP^{N}}(1)|_{C_{i}}$ pulls back to $\sO_{\PP^{1}}(a_{i})$, so $\pi_{*}(\sL|_{C_i}) = \sO_{\PP^{1}}(a_{i})$. Thus the maps $\sL \to \sL|_{C_{a_{i}}}$
%induce maps $\sE = \pi_{*}\sL \to \pi_{*}( \sL|_{C_i})$. Putting this together, we get a map
%$$
%\sE \to \sO_{\PP^{1}}(a_{1}) \oplus \sO_{\PP^{1}}(a_{2}) =: \sE'.
%$$
%Since the spaces spanned by $C_{1}$ and $C_{2}$ are complementary, this map of rank 2 vector bundles is an inclusion. Since $\sE'$ has only $a_{1}+a_{2}+2$ independent global sections, and is generated by them, the map is an isomorphism, proving that $\sE \cong \sO_{\PP^{1}}(a_{1}) \oplus \sO_{\PP^{1}}(a_{2})$
%
%It remains to show that $X \cong \PP(\sE)$, embedded by $|\cO_{\PP(\cE)}(1)|$. Maps $\phi: X \to \PP (\sE)$ correspond to surjections 
%$\alpha: \pi^* \sE \to \sM$ for some line bundle $\sM$ on $X$ in such a way that $\sM = \phi^*\sO_{\PP(\sE)}(1)$ (after passing to an open
%cover of $\PP^1$ the proof is almost the same as in the case of a map to projective space; see \cite[II.7.12]{Hartshorne1977}) . We may take $\alpha$ to be the 
%natural map $\pi^*\sE = \pi^*\pi_*\sO_X(1)$, and the corresponding map $\beta$ then makes the diagram
% \newarrow{Eq}{}{=}{=}{=}{}
%$$
%\begin{diagram}
% X&\rTo^{\phi_{|\sO_{X}(1)|}}&\PP^N\\
%\dTo^\beta&&\dEq \\
%\PP\sE &\rTo^{\phi_{|\cO_{\PP(\cE)}(1)|}}&\PP^N
%\end{diagram}
%$$
%commute, proving the desired assertion.
%\end{proof}
% 
\begin{proposition} Suppose that $0<a_1\leq a_2$ and
$\sE = \sO_{\PP^1}(a_1)\oplus \sO_{\PP^1}(a_2)$. Let $\pi: X:= \PP(\sE) \to \PP^1$ be the projection.
The sections $\sigma:\PP^1 \to C\subset X$ of $\pi$---that is, maps $\sigma$ such that $\pi\sigma$ is the 
identity---correspond to surjections
$\sE \to \sL$ for some line bundle $\sL$ on $\PP^1$. Considering  $X$ as embedded in 
$\PP^{a_1+a_2+1}$, we have $\sL = \sigma^*\sO_C(1)$, so the degree of $C$ is equal
to the degree of $\sL$.

Thus $\pi$ admits a section of degree $e$ as a curve in $\PP^{a_1+a_2+1}$ if and only if
$0<e = a_1$ or $e\geq a_2$.
\end{proposition}

\begin{proof}
Apply \cite[II.7.12]{Hartshorne1977}).
\end{proof}

\section{Curves on a 2-dimensional scroll}\label{curves on scrolls}

We may begin with the curve, and try to ``find'' the scroll; or we may begin with the scroll and ask
what curves it contains. We begin with the first approach:

\subsection{Finding a scroll containing a given curve}
\begin{proposition}
Suppose that $C\subset \PP^n$ is a linearly normal integral curve, and $D$ is a base-point-free Cartier divisor on $C$. If the linear span of $D$ is $t$-dimensional with $t\leq n-2$, then $C$ lies on a scroll of dimension $t+1$.
\end{proposition}

\begin{proof}
Choose a base-point-free pencil $\CC^2\cong V \subset H^0(\sO_C(D)$, and let $H$ be a hyperplane section of $C$. Since the span of $D$ is $t$-dimensional, $W:=H^0(\sO_C(H-D))$ has dimension $n-t$, and the natural 1-generic mapping
$V\otimes W \to H^0(\sO_C(H)$ corresponds, as in Theorem~\ref{1-generic basics} and Theorem~\ref{matrix pencils}, to the desired scroll.
\end{proof}

\begin{corollary}\label{hyperelliptic and trigonal} Suppose that one of the following holds:
\begin{enumerate}
 \item  $C\subset \PP^n$ is a linearly normal hyperelliptic curve, and  
$$
\{D_\lambda \mid \lambda \in \PP^1\}
$$
are the divisors of the $g^1_2$ on $C$; or

\item $C\subset \PP^{g-1}$ is a trigonal canonical curve, and  
$\{D_\lambda \mid \lambda \in \PP^1\}$
are the divisors of the $\g^1_3$.
\end{enumerate}

The union of the lines spanned by the $D_\lambda$
is a scroll $S(a_1,a_2)$ and $\max\{a_1, a_2\}$ is the maximal number of
$D_\lambda$ that are contained in a single hyperplane.
\end{corollary}

\begin{proof}
Let $\sL$ be the invertible sheaf $\cO_C(D_\lambda)$ corresponding to the $g^1_2$ in case 1 or
the $g_3^1$ in case 2 and let $s_\lambda$ be
the section vanishing on $D_\lambda$. Setting $\sM =  \sL^{-1}\otimes \sO_C(1)$, we see that
$s_\lambda\cdot H^0(\sM) \subset H^0(\sO_C(1)$ is the space of linear forms vanishing on
$D_\lambda$. In both cases, this space is a line: this is obvious in case 1, and follows from the 
geometric Riemann-Roch theorem in case 2. 

These forms make up the
generalized rows of the matrix $M_\mu$ corresponding to the multiplication
$\mu: H^0(\sL)\otimes H^0(\sM) \to H^0(\sO_C(	1))$, we see that the union of the lines is a
scroll $S(a_1,a_2)$ cut out by $I_2(M_\mu)$.

The proof is completed by the more general Proposition~\ref{which scroll}.
\end{proof}

\begin{proposition}\label{which scroll}
Let $S(a_1,a_2)\subset \PP^{a_1+a_2+1}$ be a scroll with $a_1\leq a_2$. The maximal number of rulings contained in
a proper subspace of $ \PP^{a_1+a_2+1}$, is $a_2$. 

Equivalently, if the scroll is defined from a multiplication
map $V\otimes H^0(\sL_2) \to H^0(\sO_X(1))$, where $V$ is a basepoint-free pencil in $H^0(\sL_1)$,
then $a_2$ is the maximal integer such that $\sL_1^{-a_2}\\sO_X(1)$ is effective.
\end{proposition}

\begin{proof}
A hyperplane containing $C_{a_1}$ meets $C_{a_2}$ in $a_2$
points, and thus contains $a_2$ rulings of the scroll. If $H$ were a hyperplane containg more than $a_2$
rulings, then $H$ would meet each curve $C_{a_i}$ in more than $a_i$ points, and thus $H$ would contain
both these curves, so that $S(a_1,a_2)\subset H$. Since $S(a_1,a_2)$ is nondegenerate, this is impossible.

The second characterization follows because the rulings of the scroll are the divisors of elements of $V.$
\end{proof}

In the simplest case, the scroll might be a rational normal curve. There is a famous result of Castelnuovo
in this direction. Recall that a set of $d$ points in linearly general position that spans $\PP^r$ imposes
at least $min(d, 2r+1)$ conditions on quadrics (Proposition~\ref{min hilb}).

\begin{theorem}\label{2n+3}
If $\Gamma\subset \PP^n$ is a set of $d\geq 2n+3$ points in linearly general position, and $\Gamma$ imposes only
$2n+1$ conditions on quadrics, then $\Gamma$ lies on rational normal curve.
\end{theorem}
\begin{proof}
 See~\cite[Lemma 3.9]{MR685427}. The proof is broken into hints in Exercise~\ref{2n+3 exercise}
\end{proof}

The bound $2n+3$ is sharp; for example, in $\P^3$ the complete intersection of $3 = 10 -(2n+1)$ quadrics
is $8=2n+2$ points, and these do not lie on a rational normal curve (Exercise~\ref{2n+3 is sharp}).


\begin{theorem}\label{Castelnuovo examples}
If $C\subset \PP^r$ is an integral curve of degree $d\geq 2r+1$ with arithmetic genus equal to $\pi(r,d)$, the
Castelnuovo bound, then $C$ lies on a 2-dimensional scroll or on a cone over the Veronese surface
\end{theorem}

\begin{proof}
From the proof of Castelnuovo's theorem (Theorem~\ref{Castelnuovo's bound}) we see that a general hyperplane
section of $C$, which is a set of point imposes exactly $2(r-1)+1$ conditions on quadrics, and moreover these
quadrics are the restriction of quadrics containing the curve. By Theorem~\ref{2n+1} the hyperplane
section lies on a rational normal curve whose ideal consists of all the quadrics containing the points;
Thus these quadrics intersect in a surface of minimal degree containing $C$.
\end{proof}

Both scrolls and the Veronese surface can occur; see Exercise~\ref{Castelnuovo Veronese}.

\subsection{Finding curves on a given scroll}

We now turn to the reverse approach: given a 2-dimensional scroll, what are the curves that lie on it?
The key is to understand the divisor class group and the canonical class. We begin with the nonsingular case.

\noindent{\bf Notation:} Throughout this section we consider the vector bundle 
$$
\sE = \sO_{\PP^1}(a_1) \oplus\sO_{\PP^1}(a_2)
$$
with $0\leq a_1\leq a_2$ and the 
scroll $ S(a_1, a_2) \subset \PP^N$, where $N = a_1+a_2+1$. This scroll is the image of $X = \PP(\sE)$ by a map that is an isomorphism
if $0<a_1$ and is the blow-down of the $C_1$ if $a_1=0$ \fix{This requires defining $C_1,C_2$ inside $\PP(\sE)$}.  We write $\pi:X \to \PP^1$ for the projection, and
$C_{a_i}\subset X$ for the directrix of degree $a_i$. The degree of $S(a,b)$ is $d := a_1+a_2$.


\begin{theorem}\label{pic of scroll}
\begin{enumerate} Suppose in addition that $a_{1}>0$.

\item The Picard group of $X$ is $\Pic X \cong \ZZ^2$, freely generated by  the class $F$ of a ruling and the class $H$ of a  hyperplane section. 
\item The
intersection form on $\Pic(X)$ is given by
$$
\bordermatrix{\kern 10pt\cdot&F&H\cr
F&0&1\cr
H&1&d
}
$$

\item The canonical class of the scroll is $K := -2H +(d-2)F$, so $K^2 = 8$.

\item The degree of a curve in the class $pH+qF$ is $pd+q$, while its arithmetic genus is
${p\choose 2}d+pq-p-q+1$.

\item The class of $C_{a_1}$
is $H-a_2F$, and the class of $C_{a_2}$ is 
is $H-a_1F$. 
\item If $C \subset \PP^{g-1}$ is a trigonal canonical curve and $X$ is the scroll swept out by the trisecants of $C$, then the class of $C$ is $3H+(4-g)F = H-K$. (Note that in this case we necessarily have $a_1 > 0$.)
\end{enumerate}
\end{theorem}

\begin{proof}
We have 
$$
H^2 = \deg X = a_1+a_2 = d; \quad H.F = \deg F = 1; \quad F^2 = 0
$$
where the last equality follows because any two fibers are linearly equivalent and disjoint.
It follows in particular that $H$ and $F$ are linearly independent.

We next show that $\Pic X$ is generated by $H$ and $F$. If $D$ is any divisor,
then $D' = D - (F.D)H$ meets $F$ in degree 0, and it now suffices to show that $D'\sim aF$ for
some integer $a$.
Since $\sO_X(D')|_F = \sO_F$ for any fiber $F$, we see that
$\pi_*(\sO_X(D'))$ is a torsion-free sheaf on $\PP^1$ whose fiber at each point is one-dimensional, 
so $\sL$ is a line bundle on $\PP^1$.  Possibly replacing $D'$ by $-D'$, and using the fact that
$\pi_*(\sO_X(-D')) = (\pi_*(\sO_X(D'))^{-1}$ we may assume that $\sL$ is globally generated, and it follows that  the natural map of line bundles $\pi^*\sL = \pi^*\pi_* \sO_X(D') \to \sO_X(D') $ is a surjection, whence 
$\sO_X(D') \cong \pi^*\sL$. Thus if $q = \deg \sL$ then
$D' \sim qF$. Note that we could also recover $q$ as $H.D'$. This completes the proof of parts
1 and 2.

To compute the canonical class $K_X = pH+qF$ we use the adjunction formula on the rational curves
$H$ and and $F$. Thus $-2 = (F+K).F = p $ and 
$$
-2 = (H+K).H = d + pd+q = d + (-2)d+q
$$
whence $q = d-2$ as required for part 3.
 
 Part 4 is a direct computation from the adjunction formula.
 
For part 5 we observe that a hyperplane containing $C_{a_1}$ meets $X$ in $C_{a_1}$ plus
$a_2$ rulings; thus $C_{a_1}\sim H-a_2F$.

Finally from the adjunction formula $K_C = (K_X+C)\mid_C$ we see that $H+K$ is the class of
the canonical curve. Moreover, any other curve of the same degree would differ by
a divisor meeting $H$ with multiplicity 0, that is, a divisor of the form $E = p(H-dF)$.
But we would also have to have $(H-K+E).(H+E) = 2g-2$, and a straightforward
computation shows that there is no solution to this equation with $p\neq 0$ and $g>3$.
\end{proof}

Now we can say exactly which classes on the scroll contain curves:

\begin{theorem}\label{where are the curves?} Again, suppose that $0<a_{1}$.
There are reduced  curves in the class $D = pH+qF$ if and only if one of the following holds:

\begin{enumerate}
\item $D\sim qF$; that is, $p=0, q>0$; or
\item $D\sim C_{a_{1}}$; that is, $p=1, q=-a_{2}$; or
\item $p\geq 0$ and $D\cdot C_{a_{1}}> 0$; that is, $q \geq -pa_1.$
\end{enumerate}
In case (3) the linear series $|D|$ is basepoint free. When, in addition, $a_2>a_1$ or $q>-pa_1$ the class $|D|$ contains irreducible smooth curves.
\end{theorem}

Note that in case (1) we have $D^{2} = 0$, because any two fibers of $\pi$ are disjoint; in case (2) we have $D^{2}= a_{1}-a_{2}\leq 0$ and in case (3) we have $D^{2}\geq 0$, and $D^2=0$ only if
$d=2, q = -pa_1$. In particular, no irreducible curve
on $X$ other than $C_{a_1}$ can have negative self-intersection.

\begin{proof}
The existence of smooth curves of types 1 and 2 is obvious; the following result will show that
the ones of type 3 move in a base-point-free linear series. By Bertini's theorem \ref{***} such a series must contain smooth curves unless the associated map factors through a curve, in which case $D^2 = dp^2-2pq = p(pa_1+pa_2 -2 q) = 0$, which implies that $a_2=a_1$ and $q= -pa_1$.

The next result, Theorem~\ref{global sections}, thus completes the proof.
\end{proof}

\begin{theorem}\label{global sections} Again, suppose that $0<a_{1}$.
Suppose that $D$ is a divisor on the scroll $X$ as above. If $D \sim pH+qF$, then 
\begin{align*}
 H^{0}(\sO_{X}(D)) &= H^{0}(\sO_{\PP^{1}}(q) \otimes \Sym^{p} \sE)\\
 &= 
\bigoplus_{0\leq i\leq p}H^{0}\bigl(\sO_{\PP^{1}}(q + (p-i)a_{1}+i a_{2})\bigr).
\end{align*}
and $|D|$ is basepoint free iff every summand is nonzero.
Thus, numerically,
$$
h^{0}(\sO_{X}(D)) = 
%\sum_{\{i\ \mid\ q+(p-i)a_{1}+i a_{2} \geq 0\}}H^{0}\bigl(\sO_{\PP^{1}}(q + (p-i)a_{1}+i a_{2})\bigr)
%= 
\sum_{\{i\ \mid\ q+(p-i)a_{1}+i a_{2} \geq 0\}}1+(q + (p-i)a_{1}+i a_{2}),
$$
and
$|D|$ is base point free iff $p\geq 0$ and $q\geq -pa_{1}$.
\end{theorem}

\begin{proof} First, If $q<-pa_{1}$, then 
$$
D\cdot C_{a_{1}} = (pH+qF) \cdot (H-a_{2}F) = p(a_{1}+a_{2}) -pa_{1}+q = pa_{1}+q < 0.
$$
so any effective divisor in the class of $D$ must have a component in common with $C_{a_{1}}$.

Let $\pi:X\to \PP^{1}$ be the structure map of the projective bundle $X = \PP_{\PP^{1}}(\sE)$.
We have $H^{0}(\sO_{X}(pH+qF)) = H^{0}(\pi_{*}(\sO_{X}(pH+qF)))$. Also, 
We may write $\sO_{X}(pH+qF)$ as $\sO_{X}(p) \otimes \pi^{*}\sO_{\PP^{1}}(q)$, and since
$sO_{\PP^{1}}(q)$ is a line bundle we see that 
$$
\pi_{*}\bigl(\sO_{X}(p) \otimes \pi^{*}\sO_{\PP^{1}}(q)\bigr) 
 = \pi_{*}\bigl(\sO_{X}(p)\bigr)\otimes \sO_{\PP^{1}}(q).
$$

The projective bundle $\PP(\sE)$ is
by definition Proj of the symmetric algebra $\PP(\sE):=\Proj(\Sym(\sE))$. Over any open 
subset $U$ of the base $\PP^1$ over which $\sE$ is free, this is $\pi^{-1}(U) = U\times \PP^2$,
and it follows that $\pi_*(\sO_{\PP(\sE)}(p)) = \Sym^p\sE$. 
Thus 
\begin{align*}
\pi_{*}(\sO_{X}(pH+qF)) &= 
\pi_{*}\bigl(\sO_{X}(p) \otimes \pi^{*}\sO_{\PP^{1}}(q)\bigr) \\
 &= \pi_{*}\bigl(\sO_{X}(p)\bigr)\otimes \sO_{\PP^{1}}(q)\\
&=  \Sym^{p}(\sE)\otimes \sO_{\PP^{1}}(q)\\
&=  \bigl(\bigoplus_{0\leq i\leq p} \sO_{\PP^{1}}((p-i)a_{1}+i a_{2})\bigr) \otimes \sO_{\PP^{1}}(q),
\end{align*}
and the first formula follows. 

Clearly, every term 
$H^{0}(\bigl(\sO_{\PP^{1}}(q + (p-i)a_{1}+i a_{2})\bigr))$ is nonzero iff and only if 
$H^{0}(\sO_{\PP^{1}}(q + pa_{1})$ is nonzero iff $q\geq -pa_{1}$.
If $\sigma = \sum \sigma_i$ is a section of $\sO_X(D)$ written according to the decomposition
above, then the restriction of $\sigma$ to  the rational normal curve $C_{a_1} = \PP(\sO_{\PP^1}(a_1)$ is the component $\sigma_0$, and similarly for $C_{a_2}$ and $\sigma_p$. Thus when  all the summands are nonzero
there are sections  vanishing on $C_{a_{1}}$ but not $C_{a_{2}}$, and vice versa, so the system is base point free. 
\end{proof}

 
We can easily compute the degrees and genera of curves that lie on scrolls:

\begin{proposition}Again, suppose that $0<a_{1}$.
 Suppose that $D\sim pH+qF$ is a smooth irreducible curve on $S(a_{1}, a_{2})$ as in Theorem~\ref{where are the curves?}. 
\begin{itemize}
 \item The degree of $D$ is $p(a_{1}+a_{2}) +q$.
 \item The genus of $D$ is ${p\choose 2}(a_{1}+a_{2}) + (p-1)(q-1)$.
\end{itemize}
\end{proposition}
  
\begin{proof} The degree of $D$ is $H\cdot D$, yielding the given formula. Let $g$ be the genus of $D$. 
By the adjunction formula
\begin{align*}
2g-2 =  \bigl((p-2)H+&(q+a_{1}+a_{2}-2)F\bigr)\cdot (pH+qF)\\ 
 &= (p^{2}-p)(a_{1}+a_{2})+2(pq-p-q)
\end{align*}
so $g = {p\choose 2}(a_{1}+a_{2}) + (p-1)(q-1)$ as required.
\end{proof}

\begin{fact}
A general curve $C$ of  genus $\geq 22$ does not lie on any 2-dimensional scroll.
\end{fact}
\begin{proof}
Except when $D\sim C_{a_{1}}$, a rational curve of negative self-intersection, every nonsingular curve on $X$ moves in a non-trivial linear series. However the moduli space of curves of genus $\geq 22$ is of general type, and this implies in particular that there there is no nontrivial rational family of curves containing a general curve of 
genus $\geq 22$. But if the linear series containing $C$ had all nonsingular fibers isomorphic to $C$, then
$X$ would be birationally isomorphic to $C\times \PP^{1}$, and thus not rational, a contradiction.
\end{proof}

\fix{add: hyperelliptic curves. Refer to the g=2 chapter}

Finally, we treat the case of the cone over a rational normal curve:
\begin{proposition}\label{singular scrolls}
If $X$ is a scheme, $\sE$ a locally free sheaf on $X$, and $\sL$ an invertible sheaf on $X$, 
and Let $\pi: \PP(\sE)\to X$ the natural projection.
$$
\PP(\sL \otimes \sE)\cong \PP(\sE).
$$
Under this isomorphism the invertible sheaf $\sO_{\PP(\sE\otimes \sL)}(1)$ corresponds to the invertible sheaf
$\sO_{\PP(\sE)}(1)\otimes \pi^{*}\sL$. 

Moreover, the singular scroll $S(0,a_{2})$, which is the cone over a rational normal curve of degree $a_{2}$, is the image of
$S(1, a_{2}+1)$ under the map corresponding to the complete linear series 
$$
|\sO_{S(1, a_{2}+1)}(1) \otimes \pi^{*}(\sO_{\PP^{1}}(-1))|,
$$
which blows down the line $C_{1}$.
\end{proposition}

Note that $S(1,a_2+1)$ is isomorphic to the surface $\tilde S(0, a_2)$ of Section~\ref{daily name}.

\begin{proof} 
The desired map $\PP(\sE) \to \PP(\sE\otimes \sL)$ corresponds to the surjection 
$\pi^*(\sE \otimes\sL) \ \pi^*(\sE) \otimes \pi^*(\sL) \to \sO_{\PP(\sE)}(1) \otimes \pi^*(\sL)$, 
and the inverse map is formed similarly.

When $\sE = \sO_{\PP^1}(a_1)\oplus \sO_{\PP^1}(a_2+1)$ and $a_1 = 1$ the
invertible sheaf
$\sO_{S(a_1, a_{2}+1)}(1) \otimes \pi^{*}(\sO_{\PP^{1}}(-1))$ corresponds to the divisor class $H-F$, which meets $C_{a_1}$  in degree 0, so the image
of $C_{a_1}$ under the corresponding linear series is a point. However the description of  $H^0(\sE)$ in
Theorem~\ref{global sections} shows that the restriction to $F$
is still the complete linear series $|\sO_{\PP^1}(1)|$, and the restriction to $C_{{a_2}+1}\cong \PP^1$
is $|\sO_{\PP^1}(a_2)|$.
\end{proof}

\begin{corollary}\label{curves on a singular scroll}
 The smooth curves that lie on the cone $S(0,a_2)$ with $a_2>1$ are isomorphic to their strict transforms
 on $S(1,a_2+1)$, and on $S(1,a_2+1)$ these lie in the classes $mH-mF$ and $mH-(m-1)F$ for $m\geq 1$, 
 corresponding to a multiple of the hyperplane section on $S(0,a_2)$ or one ruling plus a multiple of the hyperplane section;
 in particular, the degree is congruent to 0 or 1 mod $a_2$. 
 In the latter case, the curve is a Weil divisor that is not a Cartier divisor on $S(0,a_2)$.
\end{corollary}

\begin{proof}
These are the classes that intersect $C_{a_1}$ in degree 0 or 1.
\end{proof}


In these terms we can also see which classes on a scroll correspond to hyperelliptic or trigonal curves in the way described by Corollary~\ref{hyperelliptic and trigonal}:

\begin{corollary}\label{which class}
In situation 1 of Corollary~\ref{hyperelliptic and trigonal}, the class of $C\subset \PP^n$ on $S(a_1,a_2)$ is
$2H - mF$, where $m =  2n-2 - \deg C$, which is one less than the number of quadrics needed to generated $I_{C/S(a_1,a_2)}$.

In situation 2, the class of $C$ is 
$3H -  mF$ where $m = g-4$, which is one less than the number of degree 3 minimal generators of $I_{C}.$
\end{corollary}

\begin{proof} Write $d=2$ in the first case and $d=3$ in the second.
Since the lines of the scroll are spanned by the $D_\lambda$, they meet $C$ in d points, 
giving the coefficient of $H$. 

The coefficient of $F$ is $d\deg S(a_1,a_2) - \deg C$ in both cases, and 
$\deg S(a_1,a_2)= 1+ \codim  S(a_1,a_2)$. By Theorem~\ref{global sections}, 
$h^0(dH-mF) = m+1$, so $m$ is one less than the number of minimal generators
of $I_{C/S(a_1,a_2)}$. In case 2, the ideal of the canonical curve $C$ is minimally
generated by cubics together with  the quadrics in the ideal of the scroll, completing the proof. \end{proof}

\fix{2n+3 is now stated as a fact with ref in the hyperplane section chapter}
%\begin{exercise}\label{linear bound is sharp}
%Let $D \subset \PP^n$ be a rational normal curve. If $\Gamma \subset D$ is any collection of $d$ points on $D$ (or for that matter any subscheme of $D$ of degree $d$) then the Hilbert function of $\Gamma$ is
%$$
%h_\Gamma(m) = \min\{d, mn+1\}
%$$
%\end{exercise} 
%Castelnuovo in fact proved a converse to this observation:
%\begin{fact}
%If $d \geq 2n+3$, then any collection of $d$ points in $\PP^n$ in linearly general position with Hilbert function $h_\Gamma(2) = \min\{d, 2n+1\}$ must lie on a rational normal curve (and then $h_\Gamma(m) = \min\{d, mn+1\}$ for all $m$.) See~\cite[Lemma3.9]{MR685427} for a proof.
%\end{fact}
%\fix{should we include the proof?}
%
\section{Exercises}

\begin{exercise}
 Give a proof of Proposition~\ref{minimal degree} without using Bertini's Theorem, by projecting $X$ from a general point and using induction on $\codim X$.
 Hint: Let $p\in L\cap X$ be a point. If every secant to $X$ through $p$ lies entirely in $X$, then $X$ is a cone over $p$; but since $p$ was a general point, this would imply that $X$ is a linear space, contradicting non-degeneracy. 

It follows that the projection $\pi_{p}:X \to \PP^{N-1}$ is a generically finite (rational) map from $X$ to $X' := \pi_{p}(X)$,
and thus $\dim X' = \dim X$ and $\codim X' = \codim X-1$. The plane 
$\pi_{p}(L)$ meets $X'$ in the images of the points of $L\cap X$ other than $p$, so
$\deg X\geq \deg X'+1$. By induction, $\deg X' \geq \codim X'+1 = \codim X$, completing the argument.
\end{exercise}

\begin{exercise}\label{special projections}
Suppose $1\leq a_1 < a_2$. Show that the projection of $S(a_1,a_2)$ from any point on $C_{a_1}$ is 
$S(a_1-1, a_2)$, and that the projection from any point on $C_{a_2}$ is $S(a_1, a_2-1)$. (If $a_1 = a_2$, projection of $S(a_1,a_2)$ from any point is $S(a_1-1, a_2)$.)

Conclude, in particular, that the blowup of the cone $S(0,a_2)$ over the rational normal curve of degree $a_2$ is $S(1,a_2+1)$.  
\end{exercise}

\begin{exercise}
Show that a matrix $M$ of linear forms is 1-generic iff, even after arbitrary row and column transformations, its entries are all non-zero.
\end{exercise}

\begin{exercise}
Let $M$ be a 1-generic $p\times q$ matrix of linear forms, with $p\leq q$. Show that the codimension of
$I_p(M)$ is $q-p+1$. Hint: imitate the proof of lemma~\ref{codim of 2,n 1-generic}.
\end{exercise}

\begin{exercise}
Show that if $X\subset \PP^n$ is a projective variety whose homogeneous ideal $I$ contains $m$ independent quadrics, then the ideal of the general hyperplane section of $X$ in $\PP^{n-1}$
contains at least $m$ independent quadrics. Use this to prove that if $X$ has codimension $c$ then $m\leq {c+1\choose 2}$, and that equality holds if and only if
$X$ is a variety of minimal degree (that is, $\deg X = c+1$.)
\end{exercise}

\begin{exercise}\label{many quadrics}
 If $X \subset \PP^n$ is an irreducible, nondegenerate projective variety of codimension $c$ whose homogeneous ideal
 contains ${c\choose 2}$ independent quadratic forms, then $X$ is a variety of degree $c$.
 
Hint: Let $Y = X \cap H$ be a general hyperplane section of $X$ and consider the restriction map $H^0(\cI_{X/\PP^n}(2)) \to H^0(\cI_{Y/\PP^{n-1}}(2))$; repeat $n-c$ times.
\end{exercise}

\begin{exercise}\label{Castelnuovo Veronese}
Find all the curves on the Veronese surface that satisfy the conditions of 
 Theorem~\ref{Castelnuovo examples}. Do any of them also lie on a 2-dimensional scroll?
\end{exercise}


\begin{exercise}
State and prove an analogue of Proposition~\ref{which scroll} for curves on scrolls of dimension $>2$.
Hint: $a_1$ is again the maximal number of rulings. What's the cleanest statement after that? Does one have
to project the scroll to $S(a_2, \dot)$ to see the rest?
\end{exercise}


\begin{exercise}\label{general projections}
Improve the result of Exercise~\ref{special projections} by showing that, if $0\leq a_1,\leq a_2$, then
 the projection of $S(a_1,a_2)$ from any point not on $C_{a_1}$ (or not the vertex, in case $a_1=0$) is 
 is $S(a_1, a_2-1)$.
\end{exercise}

\fix{maybe integrate this with Corollary~\ref{curves on a singular scroll}}
Exercise~\ref{F2}
\begin{exercise}\label{curves on cones}
\item If $C \subset S(0, a_2) \subset \PP^{a_2+1}$ is a smooth curve with class $m$ times the hyperplane section (that is, corresponding to a curve on $S(1,a_2+1)$ with class $mH - mF$), show that
$$
\deg(C) = ma_2 \quad \text{and} \quad g(C) = \binom{m}{2}a_2 - m + 1.
$$
\item If $C \subset S(0, a_2) \subset \PP^{a_2+1}$ is a smooth curve with class $m$ times the hyperplane section plus a ruling (that is, corresponding to a curve on $S(1,a_2+1)$ with class $mH - (m-1)F$), show that
$$
\deg(C) = ma_2 + 1 \quad \text{and} \quad g(C) = \binom{m}{2}a_2.
$$
\end{exercise}

\begin{exercise}~\label{2n+3 exercise}
Prove that if $\Gamma\subset \PP^n$ is a set of $d\geq 2n+3$ points in linearly general position, and $\Gamma$ imposes only
$2n+1$ conditions on quadrics, then $\Gamma$ lies on rational normal curve, using the following steps:
\fix{add an exercise version of the proof}
\end{exercise}

\begin{exercise}~\label{2n+3 is sharp}
Show that 3 general quadrics meet in 8 general points in $\PP^3$, and deduce that the bound in Exercise~\ref{2n+3 exercise}
is sharp.
\end{exercise}

\fix{Revise the following, moved from the genus 4,5 chapter and referred to there}

\begin{exercise} Trigonal curves of genus 5.\label{trigonal genus 5} 
 Returning to the canonical curve $C \subset \PP^4$, suppose that the intersection $X = Q_1 \cap Q_2 \cap Q_3$ of the three quadrics containing $C$ has dimension 2. If $C$ were a component of $X$, then the sum of the degrees of the irreducible components of $X$ would be strictly greater than 8, which Fulton's theorem doesn't allow. Thus $C$ must be contained in a 2-dimensional irreducible component  $S$ of $X$, and this surface $S$ is necessarily nondegenerate. 

 a rational normal scroll in $\PP^4$ is either
 the union of the lines joining corresponding points on a line and a conic contained in a complementary plane, denoted $S(1,2)$, or the cone over a twisted cubic curve in $\PP^3$, denoted $S(0,3)$. The latter case does not occur here, because by Corollary~\ref{curves on a singular scroll} the smooth curves on $S(0,3)$ are either hypersurface sections, and thus of degree $3$ for
some integer $a$, or in the rational equivalence class of a hypersurface section plus one line,
of degree $3a+1$, and thus not of degree 8. One can see this directly too: if the scroll were the cone over
a twisted cubic, then the lines on the cone would cut out the $g^1_3$ on $C$. Since a pair of lines on the cone span
only a 2-plane, the sum $D$ of the two divisors would be a divisor of degree 6  and thus $K-D$ would be a $g^1_2$, 
so the curve would be hyperelliptic.

In the case when $C$ lies on $S := S(1,2)$, we may write the class of $C$ in terms of the hyperplane class $H$ and the class $F$ of a ruling, $C\sim pH+qF$, and we see from the 
projection of $S$ to $\PP^1$ that $C$ admits
a degree $p$ covering of $\PP^1$. Since we have assumed that $C$ is not hyperelliptic,
we must have $p\geq 3$. By Theorem~\ref{where are the curves?} we must have
$q\geq -p$. Since $\deg C = C\cdot H = 3p+q = 8$, we must have either 
$C\sim 3H-F$ or $C\sim 4H-4F$. By the adjunction formula, in the second case,
$$
2g(C)-2 = 8 = (C+K_S)\cdot C = (4H-4F)+(-2H+F))\cdot(4H-4F) =4
$$
whereas a similar computation in the first case yields 8; thus $C\sim 3H-F$.

The key thing to note here is that the curve $C$ has intersection number 3 with the lines of the ruling of $S$, meaning that $C$ is a trigonal curve. (We also see that the $g^1_3$ on $C$ is unique: if $D = p + q + r$ is a divisor moving in a pencil, the points $p, q$ and $r$ must lie on a line; since the surface $S$ is the intersection of quadrics, those lines must lie on $S$ and so must be  lines  of the ruling.) We see also that the locus $W^1_4(C)$ has two components: there are pencils with a base point, that is, consisting of the $g^1_3$ plus a base point; and there are the residual series $K_C - g^1_3 - p$. Each of these components of $W^1_4(C)$ is a copy of the curve $C$ itself, and they meet in two points, corresponding to the points of intersection of $C$ with the directrix of the scroll $S$.\fix{why is it clear that there are no others?}
\end{exercise}
\fix{did we prove the last statement in general in section 4-5 or only in special cases? I believe it
follows from Petri's theorem.}

\fix{consider an exercise to find the automorphism group}

%\section{Appendix: Varieties of minimal degree}
%\includepdf[pages=1-11]{Centennial.pdf}


%footer for separate chapter files

\ifx\whole\undefined
\makeatletter\def\@biblabel#1{#1]}\makeatother
\gdef\urlhook{\url}
\bibliography{slag}
\bibliographystyle{msribib}


%%%% EXPLANATIONS:

% f and n
% some authors have all works collected at the end

\catcode`\^\active
%if ^ is followed by 
% 1:  print f, gobble the following ^ and the next character
% 0:  print n, gobble the following ^
% any other letter: print letter
\makeatletter
\def^#1{\ifx1#1f\expandafter\@gobbletwo\else
        \ifx0#1n\expandafter\expandafter\expandafter\@gobble\else#1\fi\fi}
\makeatother
\let\moreadhoc\relax
\def\indexintro{%An author's cited works appear at the end of the
%author's entry; for conventions
%see the List of Citations on page~\pageref{loc}.  
%\smallbreak\noindent
The letter `f' after a page number indicates a figure, `n' a footnote.}
\printindex[gen]
%requires makeindex
\end{document}
\else
\fi


%Let $B= \PP^{1}$, $\sE = \sO_{\PP^{1}}(a_{1}) \oplus \sO_{\PP^{1}}(a_{2})$ and  $\pi: X = \PP_{B}(\sE) \to B$ the structure map.
%
% From the general formula for the Picard group of a projective bundle above, and the fact that the Picard group of
% $\PP^{1}$ is $\ZZ$ we see that the divisor class group of $X$ is the free abelian group on the class of a divisor that belong to $\sO_{X}(1)$, namely the hyperplane section $H$, and $\pi^{*}(\sO_{\PP^{1}}(1)$, namely
% $\pi^{-1}(x)$ for any point $x\in \PP^{1}$, the fiber, proving the first formula.

%
%This is a special case of a very general situation, where, among other things, the Picard group is easy to compute, and which we now explain. 
%
%
%Recall that the projective space $\PP^{n}$ may be defined as $\Proj \Sym_{\CC}(\CC^{n+1})$. The inclusion
%of rings $\CC = \Sym_{\CC}(\CC^{n+1})_{0}\subset \Sym_{\CC}(\CC^{n+1})$ induces a structure map
%$\pi: \PP^{n}\to \Spec \CC$. 
%The variety $\PP^{n}$ comes equipped with a tautological line bundle $\sO_{\PP^{N}}(1)$, which is associated to the graded module $(\Sym_{\CC} \CC^{n+1})(1)$, and a tautological map 
%$$
%\CC^{n+1}\otimes \sO_{\PP^{N}} =\pi^{*}(\CC^{n+1}) \to \sO_{\PP^{N}}(1)
%$$
%that induces an isomorphism on global sections.
%
%\begin{fact}\label{projective space bundles}
%In an exactly parallel way, we may make a projective space bundle $\PP_B(\sE)$ over a variety $B$ from a vector bundle $\sE$ on  $B$ 
%by taking $\PP_B(\sE) = \Proj \Sym_{\sO_{B}} (\sE)$.
%The inclusion of sheaves of rings
%$\sO_{B}  = (\Sym_{\sO_{B}}(\sE))_{0} \hookrightarrow \Sym_{\sO_{B}}(\sE)$ induces a structure map
%$\pi: \PP_B(\sE) \to B$. If $\sE$ has rank $n+1$, then over any closed point $b\in B$ we have
%$\sE_{b} \cong \CC^{n+1}$, and so the fiber $\pi^{-1}(b)$ is $\PP^{N}$. The restriction of 
%$\sO_{\PP_B(\sE)}(1)$ to $\pi^{-1}(b)$ is $\PP^{N}$ is $\sO_{\PP^{N}}(1)$.
%
%The variety $\PP_{B}(\sE)$ comes equipped with a tautological line bundle $\sO_{\PP_{B}}(\sE)(1)$, which is associated to the graded module $(\Sym_{\sO_{B} (\sE)}(1)$, and a tautological map 
%$$
%\pi^{*}(\sE) \to \sO_{\PP_{B}(\sE)}(1)
%$$
%that induces an isomorphism on global sections. Furthermore, 
%$$
%\pi_{*}\sO_{\PP_{B}(\sE)}(p)) = \Sym^{p}(\sE)
%$$
%for every $p$.
%
%Thus the pair $(\PP_{B}(\sE), \sO_{\PP_{B}(\sE)}(1))$ determines $\sE$; but 
%$\PP_{B}(\sE)$ alone determines $\sE$ only up to twisting with a line bundle on $B$. For example, 
%if $\sE$ is itself a line bundle on $B$, then $\PP_{B}(\sE) \cong  B$, but $\sO_{\PP_{B}(\sE)}(1)) \cong \sE$.
%
%Conversely, if $\pi: X\to B$ is a map whose fibers are isomorphic to $\PP^{N}$, and if $X$ carries a line bundle $\sL$ whose restriction to each fiber of $\pi$ is $\sO_{\PP^{N}}(1)$, then $X\cong \PP_{B}(\sE)$ and $\sL \cong \sO_{\PP_{B}(\sE)}(1)$,
%where $\sE = \pi_{*}(\sL)$.
%\end{fact}

%
%Finally, the Picard group of invertible sheaves on $X$ is
%$\Pic X \cong \Pic B \oplus \ZZ h$, where $h$ is the class of the tautological bundle, and
%the map $\Pic B\to \Pic X$ is pull-back by $\pi$.
%The case of scrolls is the case where $B =\PP^{1}$. The situation is simpler than the general case, 
%because every vector bundle
%on $\PP^{1}$ is a sum of line bundles $\sO_{\PP^{1}}(a_{i})$. For simplicity of notation and concreteness, we will concentrate on the case of 2-dimensional scrolls. The case of $r$-dimensional scrolls is exactly parallel, and we give some references. 
% 
%Let $X := S(a_{1}, a_{2})\subset \PP^{N}$ be a nonsingular scroll of degree $a=a_{1}+a_{2}$ and $N = a_{1}+a_{2}+1$. In terms of the geometry of Section~\ref{daily name}, the variety $X$ is fibered over
%$B:=\PP^{1}$ with fibers being the lines joining the corresponding points of the directrices $C_{a_{1}}$ and 
%$C_{a_{2}}$. More precisely, in terms of the algebra of Section~\ref{particular name}, if $M$ is a
%a $2\times a$, 1-generic, matrix of linear forms
%$$
%\begin{pmatrix}
% \ell_0&\ell_{1}&\dots &\ell_{a-1}\\
% \ell_1&\ell_{2}&\dots &\ell_{a}\\ 
%\end{pmatrix}
%$$
% whose $2\times 2$ minors generate the ideal of $X$,
%then the map 
%$$
%\sO_{\PP^{N}}^{a}(-1)\to \sO_{\PP^{N}}^{2}
%$$
% defined by $M$ has rank 1 everywhere
%on $X$, so its cokernel is a line bundle $\sL$ with two global sections. The zero locus
%of the image of the first generator is the set where the second row vanishes, that is, 
%one of the planes of the scroll, and similarly for any scalar linear combination of the
%two generators; that is, $\sL$ defines a morphism $\pi: X \to \PP^{1}$ whose fibers are
%projective lines.
%
%Because the fibers of $\pi$ are embedded as linear spaces, the line bundle $\sO_{\PP^{N}}(1)$
%restricts to a line bundle on each fiber $\PP^{1}$ of $\pi$ that is the  equal to $\sO_{\PP^{1}}(1)$.
%
%
%To check these statements, we reverse the process: Let 
%$\sE =  \sO_{\PP^{1}}(a_{1}) \oplus \sO_{\PP^{1}}(a_{2})$
%and consider the complete linear series  
%$
%|\sO_{\PP_B(\sE)}(1)|.
%$
%Because $\sE$ is generated by global sections, this linear series is base point free and thus defines
%a morphism
%$$
%\phi:  \PP_B(\sE) \to \PP_B(H^{0}(\sE)) = \PP^{N}.
%$$
%
%The variety $\PP_B(\sE)$ contains subvarieties corresponding to the rank 1 quotients 
%$\sO_{\PP^{1}}(a_{i})$ of $\sE$, and the bundle $\sO_{\PP_B(\sE)}(1)$ restricts to
%the bundle $\sO_{\PP_B(\sO_{\PP^{1}}(a_{i}))}(1)$. The sections of $\sO_{\PP^{1}}(a_{i})$ restrict
%as well, and we see that the image of $\phi$ contains the rational normal curves
%$C_{a_{i}}\subset \PP^{a_{i}}$. Since all the sections from $\sO_{\PP^{1}}(a_{1})$ vanish on the $C_{a_{2}}$, and similarly for $\sO_{\PP^{1}}(a_{2})$ and $C_{a_{1}}$, we see that the two curves are embedded in  disjoint subspaces spaces. Furthermore, since the structure maps $C_{a_{i}} = \PP_{B}(\sO(a_{i})) \to \PP^{1}$ are
%isomorphisms, we see that each
% fiber of $\PP_B(\sE)$ meets each $C_{a_{i}}$ in a single point. Thus the embedded
% variety $X\subset \PP^{N}$ is a scroll, as claimed.
%
%
%
%
%Putting this together we have outlined part of the proof of the following:
%
%\begin{fact}\label{push-forward formula}
% If $X := S(a_{1}, a_{2})\subset \PP^{N}$ is a nonsingular rational normal scroll, then
% $X$ is isomorphic to a projective space bundle $\PP_{B}(\sE)$, where $B = \PP^{1}$, and the restriction of $\sO_{\PP^{N}}(1)$ to $X$
% is $\sO_{\PP_B(\sE)}(1)$. Further, writing $\pi: X\to \PP^{1}$ for the structure map, we have
%$$
%\sE = \pi_{*}(\sO_{X}(1)) = \sO_{\PP^{1}}(a_{1}) \oplus sO_{\PP^{1}}(a_{2}),
%$$
%and more generally 
%$$
%\pi_{*}(\sO_{X}(p)) = \Sym^{p}\sE = \sO_{\PP^{1}}(pa_{1}) \oplus \sO_{\PP^{1}}((p-1)a_{1}+a_{2})
%\oplus \cdots\oplus \sO_{\PP^{1}}(pa_{2}).
%$$
%
%\end{fact}
%
%In particular, since $\Pic(\PP^{1}) = \ZZ$, we see that the divisor class group
%of a scroll $S(a_{1}, a_{2})$ is freely generated by the class $H$ of a hyperplane section and the class $F$ of a ruling. The intersection form, is now easy to compute. If $C,D$ are divisor classes on the scroll, we write $C\cdot D\in \ZZ$ for their intersection number.
% 



