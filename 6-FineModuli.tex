%header and footer for separate chapter files

\ifx\whole\undefined
\documentclass[12pt, leqno]{book}
\usepackage{graphicx}
\input style-for-curves.sty
\usepackage{hyperref}
\usepackage{showkeys} %This shows the labels.
%\usepackage{SLAG,msribib,local}
%\usepackage{amsmath,amscd,amsthm,amssymb,amsxtra,latexsym,epsfig,epic,graphics}
%\usepackage[matrix,arrow,curve]{xy}
%\usepackage{graphicx}
%\usepackage{diagrams}
%
%%\usepackage{amsrefs}
%%%%%%%%%%%%%%%%%%%%%%%%%%%%%%%%%%%%%%%%%%
%%\textwidth16cm
%%\textheight20cm
%%\topmargin-2cm
%\oddsidemargin.8cm
%\evensidemargin1cm
%
%%%%%%Definitions
%\input preamble.tex
%\input style-for-curves.sty
%\def\TU{{\bf U}}
%\def\AA{{\mathbb A}}
%\def\BB{{\mathbb B}}
%\def\CC{{\mathbb C}}
%\def\QQ{{\mathbb Q}}
%\def\RR{{\mathbb R}}
%\def\facet{{\bf facet}}
%\def\image{{\rm image}}
%\def\cE{{\cal E}}
%\def\cF{{\cal F}}
%\def\cG{{\cal G}}
%\def\cH{{\cal H}}
%\def\cHom{{{\cal H}om}}
%\def\h{{\rm h}}
% \def\bs{{Boij-S\"oderberg{} }}
%
%\makeatletter
%\def\Ddots{\mathinner{\mkern1mu\raise\p@
%\vbox{\kern7\p@\hbox{.}}\mkern2mu
%\raise4\p@\hbox{.}\mkern2mu\raise7\p@\hbox{.}\mkern1mu}}
%\makeatother

%%
%\pagestyle{myheadings}

%\input style-for-curves.tex
%\documentclass{cambridge7A}
%\usepackage{hatcher_revised} 
%\usepackage{3264}
   
\errorcontextlines=1000
%\usepackage{makeidx}
\let\see\relax
\usepackage{makeidx}
\makeindex
% \index{word} in the doc; \index{variety!algebraic} gives variety, algebraic
% PUT a % after each \index{***}

\overfullrule=5pt
\catcode`\@\active
\def@{\mskip1.5mu} %produce a small space in math with an @

\title{Personalities of Curves}
\author{\copyright David Eisenbud and Joe Harris}
%%\includeonly{%
%0-intro,01-ChowRingDogma,02-FirstExamples,03-Grassmannians,04-GeneralGrassmannians
%,05-VectorBundlesAndChernClasses,06-LinesOnHypersurfaces,07-SingularElementsOfLinearSeries,
%08-ParameterSpaces,
%bib
%}

\date{\today}
%%\date{}
%\title{Curves}
%%{\normalsize ***Preliminary Version***}} 
%\author{David Eisenbud and Joe Harris }
%
%\begin{document}

\begin{document}
\maketitle

\pagenumbering{roman}
\setcounter{page}{5}
%\begin{5}
%\end{5}
\pagenumbering{arabic}
\tableofcontents
\fi


\chapter{Fine Moduli spaces} 
\label{Moduli chapter}\label{ModuliChapter}

\section{What is a moduli problem?}

Algebraic geometry is almost unique among geometric theories in that the objects---varieties,  schemes or maps between them---can be parametrized by other varieties or schemes. The set of submanifolds of a given manifold, or more generally of maps between two given manifolds, seems too large to be given the structure of a finite-dimensional manifold itself. By contrast, any algebraic variety is specified by a finite collection of polynomials, which in turn have a finite number of coefficients, so it's not too far-fetched that the collection of all varieties with specified numerical invariants, or morphisms between two given varieties, could be given the structure of a ``moduli space'' that is a variety (or scheme or\dots) in its own right.

For example, perhaps the original one, projective plane curves of degree $d$
are in natural one-to-one correspondence with the forms of degree $d$ modulo the group of nonzero scalars---that is, with the points of the dual of the projective space
$ \PP(H^0(\sO_{\PP^2}(d)))=\PP^{\binom{d+2}{2}-1} $.
In this chapter, we'll give a general framework for the notion of moduli space, introducing the main examples that we will treat in this book.

There are several ways in which the possibility of making moduli spaces has been useful in algebraic geometry. First, the existence of a moduli space that  parameterizes objects of a certain type allows us to speak of ``the general object'', meaning that we allow ourselves to avoid the ``special'' properties of objects parameterized by closed subvarieties of the moduli space. We have already used this
possibility in many places in this book. 

Second, it allows us to speak coherently about
families of objects. Some moduli spaces carry \emph{universal families}, and every nice family of the sort of objects
they parameterize is pulled back from this one by a unique map.  

This idea was already exploited informally in the nineteenth century in the guise of ``preservation of number'', used to count configurations of points or curves with a given property by specializing the 
data, and we have also exploited this idea in
Chapter~\ref{JacobianChapter} to explain the count of odd and even theta characteristics on a a general curve by appealing to the existence of a specialization to a hyperelliptic curve. In a related manner, we have already seen how the fact that invertible sheaves of degree $d$ on a curve $C$ are parametrized by a $g$-dimensional variety allows us to prove the ``$g+3$" theorem (Theorem~\ref{g+3 theorem}).

Third, to the extent that we can describe the intersection theory of a moduli space, it opens up the possibility of doing enumerative geometry on it to count solutions of geometric problems---and in particular to prove the existence of solutions. For example, knowing that the parameter space for lines in $\PP^3$ is the Grassmannian $\GG(1,3)$, a projective variety of dimension 4, and that the condition that the set of lines meeting a given curve of degree $d$
is a divisor linearly equivalent to  $d$ times the hyperplane sections, we can conclude  that there exists a line in $\PP^3$ meeting any any four given curves \cite[****]{3264}. We will use the same idea to prove the much deeper existence of certain linear series on all curves (the ``existence" half of the Brill-Noether theorem, discussed in Chapter~\ref{Brill-Noether}). 

In modern terms, a \emph{moduli problem} consists of a class of objects in algebraic geometry---schemes, subschemes of a given scheme, sheaves on schemes, maps of schemes, typically defined by some common attributes---and a notion of what it means to have a \emph{family} of these objects parametrized by a scheme $B$. The notion is formalized in the idea of a \emph{moduli functor}, 
which associates to each scheme $B$ the set of families over $B$ of the given sort. Examples will make this vague notion more concrete.

\subsection{Some moduli problems}

\begin{enumerate}\label{list of moduli problems}

\item \emph{Effective divisors on a given curve}. The objects are effective divisors of given degree on a given smooth, projective curve $C$. A family of such divisors is a subscheme $\cD \subset B \times C$, flat of degree $d$ over $B$. 
Here we are using
the equivalence between divisors of degree $d$ on a smooth curve and degree $d$ subschemes of the curve. The moduli space is the $d$-th symmetric power $C_d$ of $C$, discussed in Section~\ref{symmetric section}.

\item \emph{Line bundles on a given curve}. The objects are invertible sheaves on a given smooth projective curve $C$. A family over a scheme $B$ is an equivalence class of invertible sheaves $\cL$ on $B \times C$ whose restriction to each fiber of $B \times C$ over $B$ has degree $d$, where two families $\cL$ and $\cL'$ on $B \times C$ are equivalent if $\cL\cong \cL' \otimes \cM$, where $\cM$ is  an invertible sheaf pulled back from $B$. Usually one restricts attention to invertible sheaves 
whose restriction to each fiber $b\times C$ has a given degree $d$.
The moduli spaces are the Jacobian and Picard varieties, discussed in Section~\ref{Picard section}.

\item \emph{Hurwitz spaces}. The objects are curves of a given genus, together with maps to $\PP^1$ of given degree, up to isomorphism of the curves that commute with the maps. Often the allowable ramification indices are specified. We will discuss the simplest Hurwitz scheme in Chapter~\ref{CurvesModuliChapter}.

\item \emph{Severi varieties}. The objects are plane curves
of given degree $d$ and geometric genus $g$. If $g \neq {d-1\choose 2}$
the curves will necessarily be singular but are often
constrained to have only mild singularities, usually only nodes. We will discuss the Severi variety at the end of this chapter.

\item \emph{Hilbert schemes}. The objects are curves of given degree $d$ and arithmetic genus $g$ in $\PP^r$.  A family over $B$ is a subscheme $\cC \subset B \times \PP^r$, flat over $B$,  whose fibers are smooth, projective curves of genus $g$. There are many useful extensions of this notion. For example, there are Hilbert schemes
for subschemes $X$ contained in a given projective scheme $Y$ and having specified Hilbert polynomial or Hilbert function. 
A discussion of the Hilbert scheme is the main business of this Chapter.

\item \emph{Moduli of smooth curves}. The objects are isomorphism classes of smooth, projective curves of genus $g$. A family over $B$ is an equivalence class of smooth, projective morphisms $f : \cC \to B$ whose fibers are curves of genus $g$, where two such families $f, f'$
are equivalent if there is an isomorphism from the source of $f$ to the source
of $f'$ making the diagram
$$
\begin{diagram}[small]
\cC && \rTo^\cong && \cC'\\
&\rdTo_f&&\ldTo_{f'}\\
&&B
\end{diagram}
$$
commute.
\end{enumerate}

The moduli spaces of curves are discussed in Chapter~\ref{CurvesModuliChapter}.

\section{What is a solution to a moduli problem?}

Typically what we want from a solution to a moduli problem is to understand all possible families of the objects
in question. In particular, the individual objects should be in one-to-one correspondence with the closed points of the
moduli space.  The word \emph{natural} is the key. Most of the time, the set of objects we are interested in has cardinality $2^{\aleph_0}$, as do all positive-dimensional varieties $M$ over $\CC$, so a mere  bijection between the points of $M$ and the objects to be parametrized is meaningless.

In the nicest situations there is a \emph{universal} family $\phi: \cX\to M$ of these objects over $M$,
such that any family over a scheme $B$,  is pulled back from the one on $M$ via a unique morphism $B\to M.$  Such a space $M$ with its universal family $\phi$, if it exists, is called a \emph{fine moduli space}. This can be expressed more abstractly but more succinctly by saying that there is an isomorphism of functors:
$$
\{ B \mapsto \text{families of } X \text{ over } B \} \cong \{ B\mapsto {\rm{Mor}}_{{\rm Schemes}}(B, M) \}.
$$
If $M$ is a fine moduli space then the identity map $M\to M$ corresponds to the ``universal family'' $\phi: \cX \to M$. 
For example, the Hilbert scheme, as well as $Div_d(C)$ and $\Pic_d(C)$ are fine moduli spaces; the moduli space of curves
and the Hurwitz schemes are not.

If a fine moduli space and its universal family exist, then it is unique up to unique isomorphism: given two avatars $M$ and $M'$
the universal family on $M$ corresponds to a map $M\to M'$, and we similarly produce a map $M'\to M$. The pullback of the universal family on $M$ by the composition of these two maps is again the universal family, so the composition is the identity map.

\def\eps{{\epsilon}}
One of the useful features of a fine moduli space is that it makes the computation of tangent spaces relatively easy.
Recall that if $(R,\gm)$ is the local ring at a point $m$ on a variety $\cM$ then the (Zariski) tangent
space to $M$ at $m$ is the vector space of linear functionals $Hom_{R/\gm}(\gm/\gm^2, R/\gm)$.   Assuming that
$R$ contains its residue field $R/\gm$, such functionals
are precisely the restrictions to $\gm/\gm^2$ of the ring homomorphisms $R \to R/\gm[\eps]/(\eps^2)$ inducing the identity
$R/\gm \to R/\gm$. 
 If $\cM$ is a fine moduli space for some functor $F$, then such maps are in one-to-one correspondence
with the set $F(E)$ of families over $E := \Spec(R/\gm[\eps]/(\eps^2))$. 


The vector space structure on the tangent space is also accessible from this description: since the representable
functor $\rm{Mor}_{\rm Schemes}(B, M)$ preserves products, the
 sum of tangent vectors corresponding to families $X_i \to E$ is the restriction to the diagonal
 $E \subset E\times E$
 of the product family $X_1 \times X_2 \to E\times E$.
Often one can to compute the tangent space to a moduli space before even knowing that the moduli space exists!

%Hilbert schemes are fine moduli spaces for subschemes of a given scheme, but 
%fine moduli spaces do not exist for all moduli problems. As we'll explain in Section~\ref{coarse moduli}, there does not exist a universal family of abstract curves of genus $g$: the moduli space of curves is
%not a fine moduli space, but has a slightly weaker property.

\section{Hilbert schemes}\label{hilbert scheme section}

The Hilbert scheme is a fine moduli space representing the functor of flat families of subschemes of $\PP^r$,
that is, the functor that takes a scheme $S$ over $\CC$ to the set of subschemes $\sX \subset S\times \PP^r$
that are flat over $S$. It is a central example of in the theory, one in terms of which many others are defined.

\begin{example}\label{Hilb for plane curves}
 For a simple example, consider the family of schemes $X$ in the plane
$Hilb_p(\PP^2)$
having the same Hilbert polynomial $p_X(m)  = p(m) := dm+1-{d-1\choose 2}$ as a plane curve of degree $d$. Since 
$\dim X = \deg p$, $X$ has at least a component that is 1-dimensional, and this component must have degree
equal to $d$, the leading coefficient of $p$. Since the Hilbert polynomial  of this component is already equal
to $p_X(m)$ we see that $X$ is in fact a plane curve of degree $d$. The space $Hilb_{p}(\PP^2)$ is thus the projective space $\PP^{{d+2\choose 2}-1}$ of forms of degree $d$,
and the universal family is the projection 
$$
\{(x,F) \in \PP^2 \times \PP^{{d+2\choose 2}-1} \mid F(x) = 0\} \to \PP^{{d+2\choose 2}-1}.
$$
\end{example}
 
Since the Hilbert polynomials of the fibers of a flat family are all equal (\cite[Section III.9]{Hartshorne1977}, the Hilbert scheme is the disjoint union of components corresponding to particular Hilbert polynomials. For example, the set of curves $C \subset \PP^r$ of degree $d$ and genus $g$ corresponds to a subset of the Hilbert scheme parametrizing subschemes of $\PP^r$ with Hilbert polynomial $p(m) = dm - g + 1$, though not all schemes with this hilbert polynomial are purely one-dimensional subschemes, as we shall see. In this section, we'll describe
 the tangent spaces to the Hilbert scheme and sketch the construction. For a rigorous treatment of the construction of the Hilbert scheme, including many generalizations,  see~\cite{HomogHilbert} or \cite{MR2222646}. In Chapters~\ref{HilbertSchemesChapter} and~\ref{HilbertSchemesCounterexamplesChapter}, we'll describe  the Hilbert schemes of curves of low degree and genus in $\PP^3$ in more detail. 

\subsection{The tangent space to the Hilbert scheme}\label{tan hilbert section}

Following the general recipe for tangent spaces to fine moduli spaces, we need to understand flat families
of projective schemes over $E := \Spec \CC[\eps]/(\eps^2)$. Recall (\cite[****]{Hartshorne1977} that if $X\subset \PP^r$ is a smooth subscheme then
the dual of the exact sequence
$$
0\to \sI_{X/\PP^r}/\sI_{X/\PP^r}^2 \to \Omega_{\PP^r}\mid_X \to \Omega_X \to 0
$$
identifies $\sHom(\sI_{X/\PP^r}/\sI_{X/\PP^r}^2, \sO_X)$ with the \emph{normal bundle} of $X$ in $\PP^r$.
By analogy, for any subscheme $X\subset \PP^r$ we define the \emph{normal sheaf} to be
$$
\sN_{X/\PP^r} := \sHom(\sI_{X/\PP^r}/\sI_{X/\PP^r}^2, \sO_X).
$$

\begin{theorem}\label{tangent space of Hilb}
Let $X\subset \PP^r$ be a projective variety with Hilbert polynomial $p(m)$. The flat families 
$\sX \subset \PP^r\times E$ specializing to $X$ at the closed point defined by $(\eps)$
are in natural one-to-one correspondence with the vector space $Hom_{\PP^r}(\sI_{X/\PP^r}, \sO_X) = H^0(\sN_{X/\PP^r})$, which is thus the tangent space to the Hilbert scheme $Hilb_{p}(\PP^r)$ at $[X]$.
\end{theorem}

\begin{proof}
We will actually treat the analogous result for an affine scheme $X \subset \AA^r$; since our construction is natural, it will patch on an affine cover to give the projective case stated above. To simplify the notation, set $U = \CC[\eps]/(\eps^2)= S \oplus \eps S$. 

A $U$-module $M$ is flat  over $U$ if and only if $\Tor_1^U(\CC,M) = 0$. 
Since the free resolution of $\CC$ as a $U$-module has the form
$$
\cdots \rTo^\eps U \rTo^\eps U \rTo^\eps U \rTo \CC \rTo 0
$$
we have $\Tor_1^U(\CC,M)=0$ if and only if the submodule of $M$ annihilated by $\eps$ is $\eps M$.

%The algebra $T/I'$ is flat over $U$ if and only if $\Tor_1^U(T/I', \CC) = 0$.
%Let 
%$$
%F_2 \to F_1 \to T \to T/I' \to 0
%$$
%be the beginning of a $T$-free resolution of $T/I'$. Since $-\otimes_U\CC$ is right exact, 
%$\Tor_1^U(T/I', \CC)$  is the kernel of $F_1 \otimes \CC \to I$ modulo the image of
%$F_2 \otimes \CC \to F_1\otimes C$. Thus $\Tor_1^U(T/I', \CC) = 0$ if and only if
%all the relations on generators of $I$ can be lifted to relations on generators of $I'$.

We now construct a flat family from a homomorphism: Let $I = (g_1,\dots, g_t)\subset S = \CC[x_1,\dots, x_r]$ be the ideal defining $X$. 
Note that $Hom_S(I/I^2, S/I) = Hom_S(I,S/I)$. Let $\phi: I\to S/I$ be a homomorphism, and let $h_i\in S$ be
any element reducing to $\phi(g_i)$ modulo $I$.  The ideal
$\tilde I := (g_1+\eps h_1,\dots, g_t+\eps h_t)\subset S[\eps]/(\eps^2)$
defines a scheme over $\Spec U$ which restricts to $X$ modulo $(\eps)$; that is, $I' +(\eps) = I$. 

To see that $\tilde I$ is 
independent of the lifting chosen, suppose that $k_1,\dots,k_t\in I$ so that $g_i+\eps(h_i+k_i)$ are a different lifting,
generating an ideal $\tilde I'$.
Writing $k_i = \sum r_{i,j}g_j$ we have $g_i+\eps(h_i+k_i) = g_i+\eps h_i+ \sum \eps r_ij(g_i+\eps h_i)$, so 
$\tilde I' \subset \tilde I$, and symmetrically $\tilde I \subset \tilde I'$.

To prove flatness, suppose that $a\in T$ and  $\eps a  \in I'$,
so that we can write  $\eps a= \sum(r_i+\eps s_i)(g_i+\eps h_i) \in T$.
It follows that 
$\sum r_i\g_i = 0$. Since $\phi$ is a homomorphism this implies $\sum r_i h_i \in I.$ Thus
$\eps a \in  \eps I\subset T$, whence $a\in I + (\eps)$.  Writing $a =\sum p_i g_i+\eps b'$
and using the relations $g_i \equiv \eps h_i \hbox{ mod } I'$ we get
 $a \equiv \eps (\sum p_i h_i+b') \hbox{ mod } I'$, as required.

Finally, starting from a flat family $T$ over $U$ we must construct a homomorphism $I/I^2 \to S/I$  in a way that is inverse to the construction above: Given an ideal $\tilde I \subset T$ such that $T/\tilde I$ is flat over $U$ and $\tilde I + (\eps) = I +\eps$, 
let $J$ be the image of $\tilde I$ in $T/(\eps I) = S \oplus (\eps S)/(\eps I)$. We claim that $J$ is the graph of a homomorphism $\phi: I \to  (\eps S)/(\eps I) \cong S/I$. 

Since $\tilde I + \eps S = I +\eps S$,  the projection  to the first  summand of $S'$ maps $J$ onto $I$. 
To prove that $J$ is the graph of a homomorphism, it suffices to show that this projection is an isomorphism.
The kernel is
the intersection of $J$ with the second factor $(\eps S)/(\eps I )$, so we must show that
if $r\in S$ and $\eps r \in \tilde I$ then $\eps r \in \eps I$. 

Since $\eps r \in \tilde I$, the condition of flatness implies
that the image of $r$ in $T/\tilde I$ is in $\eps(T/\tilde I)$, which is to say that $r \in \tilde I + \eps S = I +\eps S$.
Thus $r \in  I + \eps S,$ whence $\eps  r \in \eps I$ as required.

Thus
$J$ defines a homomorphism $\phi$ as claimed, which means that
 $\tilde I$ is generated by $\{g+\phi g\mid g\in I\}$, so the two constructions are inverse to one another.
\end{proof}

\begin{example}\label{Hilb for plane curves-continued}
If $C$ is the plane curve of degree $d$ defined by a form $F$, then the ideal sheaf of $C$ is $\sO_{\PP^2}(-d)$, and thus
$$
\sI_{C/\PP^2}/\sI_{C/\PP^2}^2 = \sO_{\PP^2}(d)\mid_C = \sO_C(d).
$$
From the exact sequence 
$$
0\rTo\sO_{\PP^2}\rTo^F\sO_{\PP^2}(d) \rTo\sO_C(d)\rTo 0
$$
we deduce that the dimension of the tangent space to $Hilb_{p(m)}(\PP^2)=\PP^d$  at $C$
with $p(m) = dm+1-{d-1\choose 2}$,
is $h^0(\sO_C(d)) = h^0(\sO_{\PP^2}(d))-1 = \dim \PP^d$, as expected.
\end{example}

\

\subsection{Parametrizing twisted cubics} By Lemma~\ref{smooth is open} below, the set of twisted cubics---that is, smooth curves of degree 3 in $\PP^3$---is an open subset of the Hilbert scheme $Hilb_{3m+1}(\PP^3)$. As we've seen, a twisted cubic curve $C \subset \PP^3$ can be described as the zero locus of three homogeneous quadratic polynomials $Q_1, Q_2$ and $Q_3$ in the homogeneous coordinates on $\PP^3$; to specify the twisted cubic we could just list the $3 \times 10 = 30$ coefficients of these. But of course we could replace the three quadrics $Q_i$ with any three independent linear combinations of them; what matters---and what is is naturally associated to $C$---is the vector space $V = \langle Q_1, Q_2, Q_3 \rangle \subset H^0(\cO_{\PP^3}(2))$ that they span. This suggests that we consider the map of sets
$$
h : \{ \text{twisted cubic curves } C \subset \PP^3 \} \to G = G(3, H^0(\cO_{\PP^3}(2)))
$$
obtained by associating to a twisted cubic $C$ the second graded piece of its homogeneous ideal. 

This differs significantly from the example of plane curves given at the beginning of this chapter: there, the objects to be parametrized were the zero locus of a single polynomial, and we could vary those coefficients arbitrarily and still have a plane curve; thus, the image of the analogous map was open in the projective space $\PP^N$. In the present situation, though, if we vary the coefficients of the three quadratic polynomials $Q_i$ generally, the resulting quadrics will no longer intersect in a twisted cubic curve, but rather in eight points. Thus the image of the map $h$ is more complicated.
In fact, we'll see in Section~\ref{*****} below that the image of this map is a locally closed subvariety of the Grassmannian (we'll give the equations cutting it out in $G$), and the image can lay claim to being the moduli space of twisted cubics in $\PP^3$.

In the meantime, we can analyze the open subset of the Hilbert scheme consisting of twisted cubics. First a basic lemma:

\begin{lemma}\label{smooth is open}
Suppose that $X \to B$ is a flat family of projective schemes (that is, the projection to the second factor of 
$X\subset B\times \PP^n$ over a smooth base $B$. The points $b\in B$ such that the fiber $X_b$ is smooth and connected is an open set.
\end{lemma}

\begin{proof}
Since $B$ is smooth, it's closed points are locally complete intersections, so if $p\in X$ is a singular point of $X$ then the fiber through $p$ is singular. Since the singular subset of $X$ is closed and the family is projective, the image of the singular set in $X$ is closed, and we may remove it and assume that $X$ is smooth from the start. Now the condition that a point on a fiber is singular is a closed condition on $X$, so again its image in $B$ is closed. Finally, a smooth fiber $X_b$ is connected if and only if $h^0\sO_{X_b})$ and by the flat base change theorem~\cite[***]{Hartshorne1977} this set is open as well.
\end{proof}


\begin{proposition}\label{hilb of twisted cubics}
The open subset $\cH^\circ$ of the Hilbert scheme $Hilb_{3m+1}(\PP^3)$ parametrizing twisted cubics is irreducible of dimension 12.
\end{proposition}

\begin{proof}  Let $C_0 \subset \PP^3$ be a twisted cubic, and consider the family of translates of $C_0$ by automorphisms $A \in \PGL_4$ of $\PP^3$: that is, the family
$$
\cC = \{ (A, p) \in \PGL_4 \times \PP^3 \; \mid \; p \in A(C_0) \}.
$$
Via the projection $\pi : \cC \to \PGL_4$, this is a family of twisted cubics, and so it induces a map
$$
\phi : \PGL_4 \to \cH^\circ.
$$
Since every twisted cubic is a translate of $C_0$, this is surjective, with fibers isomorphic to the stabilizer of $C_0$, that is, the subgroup of $\PGL_4$ of automorphisms of $\PP^3$ carrying $C_0$ to itself. By the discussion in Section~\ref{linear series 1}, every automorphism of $C_{0}$ is induced by an automorphism of $\PP^{3}$, so the stabilizer is isomorphic to $\PGL_2$ and  thus has dimension 3. Since $\PGL_4$ is irreducible of dimension 15, we conclude that \emph{$\cH^\circ$ is irreducible of dimension 12}.
\end{proof}


\subsection{Construction of the Hilbert scheme in general}

The Hilbert scheme is more complicated than would appear from the examples above, starting with the Hilbert polynomial $3m+1$. There are many subschemes of $\PP^3$ that have the same Hilbert polynomial $3m+1$ as a twisted cubic---for example, the union of a plane cubic and a point---and are not the intersection of the quadrics containing them. (See Exercises \ref{characterization of degree} and Exercise~\ref{deg of disjoint union}). In Chapters~\ref{HilbertSchemesChapter}
and \ref{HilbertSchemesCounterexamplesChapter} we will discuss many more components of Hilbert schemes.

A fundamental result of~\cite{Matsusaka} provides a place to start:

\begin{lemma}\label{matsusaka}
Let $p(m) \in \QQ[m]$ be a polynomial. There exists an integer $m_0$ such that

\begin{enumerate}  

\item For any subscheme $X \subset \PP^r$ with Hilbert polynomial $p_X = p$ we have
$$
h^0(\cI_{X/\PP^r}(m)) = \binom{m+r}{r} - p(m) \quad \text{for all } m \geq m_0
$$
or in other words the Hilbert function of $X$ agrees with the Hilbert polynomial $p_X = p$ for all $m \geq m_0$; and

\item For any subscheme $X \subset \PP^r$ with Hilbert polynomial $p_X = p$ and for all $m \geq m_0$, $X$ is the intersection of the hypersurfaces of degree $m$ containing it.
\end{enumerate}
\end{lemma}

In fact, with $m_0$ sufficiently large more is true:  these forms generate the truncation of the saturated ideal (see~\cite{Gotzmann})

Note that  for any given $X$ the existence of an $m_0$ satisfying the statement of the lemma is immediate by Serre's vanishing theorem~\ref{Serre-Grothendieck vanishing}. The point of the lemma is that we can find one value of $m_0$ that works for all $X$ with Hilbert polynomial $p$. The following result of Gotzmann provides a method for determining $m_0$. 

\begin{theorem}
The Hilbert polynomial  of the homogeneous coordinate ring of any scheme $X\subset \PP^r$ can be written uniquely in the form
$$
\chi(\sO_X(m) = {m+a_1\choose a_1}+ {m+a_2 -1\choose a_2}+ \cdots+{m+a_s -(s-1)\choose a_s},
$$
with 
$$
a_1\geq \cdots \geq a_s \geq 0
$$
where the binomial coefficients are interpreted as polynomials in $m$. Moreover, the homogeneous ideal of $X$ is
then generated in degrees $\leq s$, and one can take $m_0 = s$ in the construction of the Hilbert Scheme, above.
\end{theorem}
See~\cite{MR1023391} %Green-Gotzmann
for an exposition and a proof. From the coefficients $a_j$ one can read off uniform vanishing theorems for $H^i(\sI_X)$
 as well.
 
 For example, the Hilbert polynomial $3m+1$ of the twisted cubic may be written as
 $$
 3m+1 =  {m+1\choose 1}+ {m+1 -1\choose 1}+{m+1 -2\choose 1}+{m+0 -(3)\choose 0},
 $$
 Here $s=4$, and indeed the homogeneous ideal of the union of a plane cubic with a point, also in the plane,
 requires equations of degree 4.
 
The first item allows us to define a  map of sets
$$
h : \left\{ \text{subschemes $X \subset \PP^r$ with $p_X=p$} \right\}  \to G\big(\binom{m_0+r}{r} - p(m_0),\binom{m_0+r}{r} \big)
$$
by sending $X$ to $H^0(\cI_{X/\PP^r}(m_0))$; the second implies that this map is injective.  In Section~\ref{eqns of Hilb} we give a set of equations on $G = G(\binom{m_0+r}{r} - p(m_0), \binom{m_0+r}{r})$ with common zero locus the image $\im(h)$, showing that $\im(h)$ is closed and giving it the structure of a scheme; this is the Hilbert scheme we seek.


We observed above that if $Q_1, Q_2$ and $Q_3$ were general quadrics, their intersection would be
8 points,  not a twisted cubic. What we want to know is how to tell these cases apart algebraically. One way to do this is to consider the multiplication map
$$
V \otimes H^0(\cO_{\PP^3}(1)) \to H^0(\cO_{\PP^3}(3)).
$$
We saw in Chapter~\ref{genus0And1Chapter} that the cokernel of this map is the 10-dimensional space $H^0(\sO_{\PP^1}(9))$, so the image of this map is 10-dimensional, whereas
3 general quadrics form a complete intersection and would have only Koszul syzygies, so
in the case of general quadrics this map would have 12-dimensional image.
This is a map from a 12-dimensional vector space to a 20-dimensional one, and what we've seen is that if $V$ is the net of quadrics containing a twisted cubic, it has a 2-dimensional kernel; that is, it has rank 10. 

Thus if $S$ is the universal subbundle on $G = G(3, H^0(\cO_{\PP^3}(2))$, and  $H^0(\cO_{\PP^3}(d))\otimes \sO_G$ is the trivial bundle, then the multiplication map above gives a map of vector bundles
$$
\mu: S \otimes H^0(\cO_{\PP^3}(1)) \to H^0(\cO_{\PP^3}(3))
$$
We can represent this locally as a matrix of functions, and the minors of this matrix vanish on the points of
of the Hilbert scheme: in a neighborhood of a point in $G$ corresponding to a twisted cubic, the common zero locus of these minors is the locus of nets of quadrics containing a twisted cubic

\subsection{Equations defining the Hilbert scheme}\label{eqns of Hilb}

In fact, the construction of the Hilbert scheme in general is no more structurally complicated than this special case. Given a polynomial $p(m)$, we find a value of $m_0$ that satisfies the statement of Lemma~\ref{matsusaka}; we let
$$
G = G\big(\binom{m_0+r}{r} - p(m_0), \binom{m_0+r}{r} \big)
$$
be the Grassmannian, and let $h$ be the map from the set of subschemes of $\PP^r$ with Hilbert polynomial $p$ to $G$ sending $X$ to $H^0(\cI_{X/\PP^r}(m_0))$. We then get a map of vector bundles  on $G$
$$
S \otimes H^0(\cO_{\PP^r}(1)) \to H^0(\cO_{\PP^r}(m_0+1)),
$$
and indeed in a neighborhood of any point of $G$ in the image of $h$, the common zero locus of the minors of size $\binom{r+m_0+1}{r} - p(m_0+1)$ of a matrix representative of this map is the image of $h$; and these functions define the Hilbert scheme.

\subsection{Subschemes of a given scheme}

Though the Hilbert scheme is a priori about subschemes of projective space, it is easy to see that if $X\subset \PP^n$,
then the family of subschemes $Y$ of $X$ with given Hilbert function $p$ is a closed subscheme of $Hilb_p(\PP^n)$: we simply
add the condition that the vector space of forms of high degree defining $Y$  contain the vector space defining $X$---this is also a determinantal condition.

\section{Hurwitz spaces}\label{Hurwitz spaces}

Hurwitz spaces are spaces parametrizing branched covers. They are fascinating objects; we know quite a bit about their geometry but there is much that is unknown as well. In this discussion, we'll stick to the simplest case, that of the \emph{small Hurwitz spaces}, parametrizing simply branched covers of $\PP^1$.

To start with the definition: the small Hurwitz space $\cH^\circ_{d,g}$ parametrizes pairs $(C, f)$ where $C$ is a smooth curve of genus $g$ and $f : C \to \PP^1$ a map of degree $d$ with simple branching; that is,
$$
\cH^\circ_{d,g} = \{ (C, f) \mid C \in M_g \text{ and } f:C \to \PP^1 \text{ simply branched of degree } d\}.
$$

There are two natural maps from the Hurwitz space to other spaces. First, we can ``project on the first factor;" that is, simply forget the map $f$ to arrive at a map $\pi : \cH^\circ_{d,g} \to M_g$. Secondly, we can associate to a point $(C,f) \in \cH^\circ_{d,g}$ the branch divisor $B \subset \PP^1$, which is an unordered $b$-tuple of distinct points in $\PP^1$, which we can think of as a point in the $b$th symmetric product $(\PP^1)_b  \cong \PP^b$. We thus have a diagram
$$
\begin{diagram}
& & \cH^\circ_{d,g} & & \\
& \ldTo^\pi & & \rdTo^\beta & \\
M_g & & & & U \subset \PP^b
\end{diagram}
$$
where $U \subset \PP^b$ is the complement of the discriminant hypersurface. Thus the Hurwitz space is positioned between an object $U$ we understand relatively well, and an object $M_g$ about which we would like to know more; this accounts for the historical important of Hurwitz spaces. We'll now illustrate how this works.

To begin with, by the analysis in Section~\ref{}, we see that \emph{the map $\beta$ is a covering space}: for any reduced divisor $B \subset \PP^1$ there are a finite number of simply branched covers of $\PP^1$ with branch divisor $B$; and as we vary the points of $B$ locally we can deform the cover along with them. This allows us to give the Hurwitz space $\cH^\circ_{d,g}$ the structure of a smooth variety, and also tells us that
$$
\dim(\cH^\circ_{d,g}) = b = 2d+2g-2
$$

Next, we look at the projection $\pi : \cH^\circ_{d,g} \to M_g$. To start, let's assume $d$ is large relative to $g$; $d \geq g+1$ suffices, but you can take $d$ as large as you like; taking $d > 2g$ may make the argument simpler. We have then the

\begin{proposition}
If $d \geq g+1$, the map $\pi : \cH^\circ_{d,g} \to M_g$ is surjective, with fibers of dimension $2d-g+1$.
\end{proposition}

\begin{proof}
The question is, given a curve $C$, how many simply branched maps $f : C \to \PP^1$ of degree $d$ are there? To begin with, the $g+1$ theorem (\ref{}) tells us that there is one, whence we see that $\pi$ is surjective. As for estimating the dimension of the fibers, this is straightforward. To specify a map $f : C \to \PP^1$, we can start by choosing a divisor $D \in C_d$, which will be the divisor $f^{-1}(\infty)$; this can be a general divisor of degree $d$ on $C$. Second, we choose a divisor $E$ which will be $f^{-1}(0)$; this can be a general member of the linear system $|D|$, which has dimension $d-g$. Finally, specifying $f^{-1}(\infty)$ and $f^{-1}(0)$ determines the map $f$ up to scalar multiplication on $\PP^1$; adding up the degrees of freedom, we see that the fibers of $\pi$ have dimension
$$
d + (d-g) + 1 = 2d-g+1.
$$ 
\end{proof}

Finally, we conclude that
$$
\dim(M_g) = (2d+2g-2) - (2d - g + 1) = 3g-3.
$$

We can use this in turn to analyze the cases of smaller $d$. As a basic application, note that the group $PGL_2$ of automorphisms of $\PP^1$ acts on the Hurwitz space: given $\varphi \in PGL_2$, we can send $(C,f)$ to $(C, \varphi \circ f)$. Moreover, the orbits of this action lie in fibers of the projection $\pi : \cH^\circ_{d,g} \to M_g$, meaning that \emph{the fibers of $\pi$ have dimension at least 3}. Thus we can deduce the corollary

\begin{corollary}
If $d < \lceil \frac{g}{2} \rceil + 1$, then a general curve $C$ of genus $g$ does not admit a map of degree $d$ to $\PP^1$.
\end{corollary}

This is one-half of the case $r=1$ of the Brill-Noether theorem, about which we will say much more later.

This is just one example of an application of Hurwitz spaces to the study of $M_g$. Another one worth mentioning is the original proof of the irreducibility of $M_g$: in~\cite{}, Hurwitz analyzes the monodromy of the map $\beta: \cH^\circ_{d,g} \to U \subset \PP^b$---what happens, in other words, when you let the branch points of a cover wander around in $U$ before coming back to their original locations. He proves that the monodromy is transitive, and hence that ther Hurwitz space $\cH^\circ_{d,g}$ is irreducible; since $\cH^\circ_{d,g}$ dominates $M_g$ for $d$ large, he deduces that $M_g$ must be irreducible as well.

Hurwitz' argument illustrates a fundamental point: in practice, moduli spaces of curves ``with extra structure," such as a map to projective space, are often easier to work with, and provide a useful tool for getting inside the geometry of abstract moduli spaces. For example, if we're given an abstract curve $C$ of genus $g$, it's hard---without developing a fair amount of deformation theory---to show that $C$ varies in a nontrivial family. But if $C$ is expressed as a branched cover, we can find such families just by varying the branch points.

There are many more problems about Hurwitz spaces that we won't get into here: notably, finding a good compactification; describing the divisor class group (and more generally the cycle class theory) of the Hurwitz spaces, and calculating the degrees of the maps $\cH^\circ_{d,g} \to \PP^b$, called Hurwitz numbers; see~\cite{}, \cite{} and \cite{} for more.

\section{The Severi variety}\label{severi variety}

Despite its antiquity, many questions about the family of plane curves, such as which ones degenerate into which others, and in what way, remain open. All plane curves of degree $d$ have the same Hilbert function, and thus the same arithmetic genus
$\binom{d-1}{2}$, but since curves of degree $d$ can have different sorts and numbers of singularities, they can have geometric genera from 0 to $\binom{d-1}{2}$. In this section we will explore the subset of of (reduced, irreducible) curves of degree $d$ with a fixed geometric genus. We will focus on the open set consisting of curves with only nodes as singularities, which we call \emph{nodal curves}, and compute its dimension. We will also prove the existence of specializations from a smooth projective curve of genus $g$ to a general $g$-nodal curve, a result used in Chapter~\ref{InflectionsChapter}.

\def\Vdg{{V_{d,g}}}
%\def\Vdgtilde{{\widetilde{V}_{d,g}}} 
\def\Vdgbar{{\overline{V}_{d,g}}} 

Let $\PP^N := \PP^{{d_1+2\choose 2} - 1}$ be the projective space parametrizing plane curves of degree $d$.
Within $\PP^N$ the set of reduced irreducible curves is open---it is the complement of the union of the images of the maps 
$$
\PP^{{d_1+2\choose 2} - 1}\times\PP^{{d_2+2\choose 2}-1} \to \PP^N
$$ 
with $d_1+d_2 = d$ given by multiplication of forms. 

\fix{I got rid of $\widetilde V_{d,g}$ leaving just the small Severi variety $Vdg$, now without the tilde and $Vdgbar$}

\begin{propdef}
The \emph{Severi variety} $V_{d,g} \subset \Vdgbar$ is the locus of plane curves of degree $d$ with $\delta = \binom{d-1}{2} - g$ nodes and no other singularities. This is a locally closed subset of $\PP^N$. It is sometimes
called the \emph{small Severi variety}, since we are excluding curves with more complicated singularities.
\end{propdef}

\begin{proof}
\fix{ How about proving that $V_{d,g}$ is locally closed, and at least mentioning that $V_{d,g}$ is open in $\overline V_{d,g}$}
\end{proof}

We will see that in a neighborhood of  $ {V}_{d,g}$,  the closure $\overline V_{d,g}$  is well behaved; but away from this,
even the singularities of $\overline V_{d,g}$  are not well understood. It is is an interesting open problem to find a better partial compactification of $ V_{d,g}$. 


\begin{fact}
As we shall see, the variety $V_{d,g}$ is smooth. In 1921 F. Severi gave an incorrect proof that $\Vdg$ was connected, and thus irreducible. This was finally proven in~\cite{MR837522}.
\end{fact}
\fix{add: Zariski: plane curves of degree 6 with 6 cusps and no other singularities has two components, distinguished
by the fact that on one component the cusps lie on a conic.This is in Z's Algebraic Surface}
\subsection{Local geometry of the Severi variety}\label{local severi geometry}

We first consider the \emph{universal singular point}
$$
\Phi := \left\{ (C, p) \in \PP^N \times \PP^2 \mid p \in C_{sing} \right\}
$$
and its image $\Delta\subset \PP^N$, the \emph{discriminant} variety. 

\begin{proposition}
 $\Phi$ is smooth and irreducible of dimension $N-1$, and the discriminant $\Delta$ is a hypersurface in $\PP^N$.
\end{proposition}
\begin{proof}
Projection on the second factor expresses $\Phi$ as a $\PP^{N-3}$-bundle over $\PP^2$. Explicitly, if $[X,Y,Z]$ are homogeneous coordinates on $\PP^2$, and $\{a_{i,j,k} \mid i+j+k = d \}$ are homogeneous coordinates on $\PP^N$, then the universal curve 
$$
\CC := \left\{ (C, p) \in \PP^N \times \PP^2 \mid p \in C \right\}
$$
is given as the zero locus of the single bihomogenous polynomial of bidegree $(1, d)$
$$
F([a_{i,j,k}], [X,Y,Z] ) = \sum a_{i,j,k} X^iY^jZ^k;
$$
and the universal singular point is the common zero locus of the three partial derivatives $\partial F/\partial X$, $\partial F/\partial Y$ and  $\partial F/\partial Z$. 

The set of forms $F$ that define curves singular at a given point is defined by 3 independent linear conditions, and since the set of 
points is 2-dimensional, the set $\Delta$ of singular forms has dimension $N-1$.
\end{proof}
 
We next compute the differential of the map $\pi : \Phi \to \PP^N$:

\begin{lemma}\label{tangent space to discriminant}
Let $C \subset \PP^2$ be a plane curve of degree $d$, regarded as a point in $\PP^N$,  having a node at a point $p$. The differential 
$$
d\pi : T_{(C,p)}\Phi \to T_C \PP^N
$$
is injective, with image the hyperplane $H_p \subset \PP^N$ of plane curves containing the point $p$.
\end{lemma}

Thus, if $p$ is the only singularity of $C$, then $\Delta$ is smooth at $C$; and more generally the image of a small analytic neighborhood of $(C,p) \in \Phi$ is smooth, and we can identify its tangent space at $p$ with the hyperplane $H_p$. 

\begin{proof}
We will prove this using affine coordinates on $\PP^2$ and $\PP^N$. Changing coordinates if necessary, we may assume that the point $[1,0,0] \notin C$, and that the point $p$ is $[0,0,1]$. let $x = X/Z$ and $y = Y/Z$ be coordinates on the affine plane $Z \neq 0$ and write the polynomial $F(x,y,1)$ above as
$$
f(x,y) = \sum_{i+j \leq d} a_{i,j} x^iy^j
$$
with $a_{d,0}$ normalized to 1. 

Let $g,h$ be the two partial derivatives of $f$, that is:
$$
g(x,y) := \binom{\partial f}{\partial x} = \sum_{i+j \leq d} i a_{i,j} x^{i-1}y^j
$$
and
$$
h(x,y) := \binom{\partial f}{\partial y} = \sum_{i+j \leq d} j a_{i,j} ix^{i}y^{j-1}.
$$
The functions $f, g$ and $h$ are local defining equations for $\Phi$; we consider their partial derivatives with respect to $x, y$ and $a_{0,0}$, evaluated at the point $(C,p)$, as in in figure~\ref{tang to Delta}.

\begin{table}[h!]\label{tang to Delta}
  \begin{center}
     \begin{tabular}{c|c|c|c} % <-- Alignments: 1st column left, 2nd middle and 3rd right, with vertical lines in between
            & $f$ & $g$ & $h$ \\
      \hline
$\frac{\partial}{\partial x}$ & 0 & $a_{2,0}$ & $a_{1,1}$ \\
$\frac{\partial}{\partial y}$ & 0 & $a_{1,1}$ & $a_{0,2}$ \\
$\frac{\partial}{\partial a_{0,0}}$ & 1 & 0 & 0 
    \end{tabular}
  \end{center}
\end{table}

The point $p$ is a node of $C$ (and not a more complicated singularity) if and only if the upper right $2 \times 2$ submatrix is nonsingular, which shows that the differential $d\pi$ is injective, and its image is the hyperplane $a_{0,0} = 0$ in $\PP^N$, which is exactly the hyperplane of curves containing $p$.
\end{proof}

\begin{lemma}\label{adjoint independent}
The nodes $q_i$ of an irreducible nodal plane curve $C$ of degree $d$ impose independent conditions on curves of degree $d-3$, and hence on curves of any degree $m \geq d-3$.
\end{lemma}
\begin{proof}
We will prove in Chapter~\ref{PlaneCurveChapter} that the $g$ sections of the canonical sheaf on the normalization $\widetilde C$ of
$C$ are the preimages of the sections of $\sO_C(d-3)$ that vanish at the nodes. On the other hand, 
$h^0(\sO_C(d-3) = \binom{d-1}{2}$, and the difference is exactly the number of nodes.
\end{proof}

\begin{corollary}\label{local geometry of Severi}
If $C$ is a nodal curve of degree $d$ with geometric genus $g = \binom{d-1}{2}-\delta$, then in a neighborhood of $C\in \PP^N$
the discriminant hypersurface of all singular curves consists of $\delta$ smooth sheets, meeting transversely, and hence
$V_{d,g}$ is smooth. 

Moreover, in a neighborhood  $C \in \PP^N$ 
the variety $\overline V_{d,g'}$ with $g' =  \binom{d-1}{2}-\delta' > g$ is the union of $\binom{\delta}{\delta'}$ smooth branches, each of dimension $N - \delta'$, corresponding bijectively with subsets of $\{p_1,\dots,p_{\delta'}\}$ of cardinality $\delta'$.
\end{corollary}
\begin{proof}
Lemma~\ref{tangent space to discriminant} shows that in an analytic neighborhood of $C\in \PP^N$ the discriminant hypersurface $\Delta$ will consist of $\delta$ smooth sheets, each corresponding to one node, and Lemma~\ref{adjoint independent} implies that the tangent spaces to these sheets are linearly independent. 
\end{proof}


\begin{corollary}\label{dim Severi}
The  Severi variety $V_{d,g}$ has pure dimension $N - \delta$, where $\delta = \binom{d-1}{2} - g$.
\end{corollary}

In Section~\ref{estimating dim hilb}, we give a heuristic calculation of the ``expected dimension'' $h(d,g,r)$ of the variety parametrizing curves of degree $d$ and genus $g$ in $\PP^r$
$$
h(g,r,d) := 4g-3 + (r+1)(d-g+1) - 1.
$$
The actual dimension of the restricted Hilbert scheme may be quite different. But  Corollary~\ref{dim Severi} shows that in case $r=2$ (as in the case of $r=1$), the actual dimension is always the expected.

%Corollary~\ref{local geometry of Severi} has an important consequence that we use in Chapter~\ref{InflectionsChapter}: the existence of a families of smooth curves of genus $g$ specializing to a $g$-nodal rational curve. 

%In fact, such families are already explicit in Lemma~\ref{local geometry of Severi}: if we start with a rational nodal  curve $C_0 \subset \PP^2$---for example, the general projection of a rational normal curve $\PP^1 \hookrightarrow \PP^d$ to $\PP^2$---the lemma says that we can find a family of nodal curves $C_t$ of geometric genus $g$ specializing to $C_0$, in which the  $\delta = \binom{d-1}{2} - g$ nodes of $C_t$ specialize to a subset of $\delta$ of the nodes of $C_0$. Taking the normalization of the total space of this family yields a family of smooth curves of genus $g$ specializing to a $g$-nodal rational curve, as desired. \fix{watch out! the normalization of a flat family
%is not always the normalization of each fiber! -- eg there are normal surfaces with both smooth and singular fibers.}
%
%To carry this out in detail, let us first state precisely the lemma we seek to establish:

%\begin{corollary}\label{smoothing nodes}
%Let $p_1, \dots, p_g, q_1, \dots, q_g \in \PP^1$ be any $2g$ distinct points, and let $C := \PP^1/p_i \sim q_i$ be the nodal curve of arithmetic genus $g$ obtained by identifying $p_i$ with $q_i$ for $i = 1,\dots,g$. There exists a family $\cC \to  \Delta $ of curves, parametrized by the disc $\Delta$, with $C_t$ a smooth projective curve of genus $g$ for all $t \neq 0$ and $C_0 = C$ as above.
%\end{corollary}
%
%\begin{proof}
%Let's start by realizing the curve $C_0$ in projective space. This is straightforward: to begin, choose a degree $d > 2g+2$, and embed $\PP^1$ as a rational normal curve in $\PP^d$ via the Veronese map. Let $L_i = \overline{p_i,q_i} \subset \PP^d$ be the line spanned by $p_i$ and $q_i$, and let $\Lambda \cong \PP^{g-1} \subset \PP^d$ be the $(g-1)$-plane spanned by a general point on each of these
%lines.
%Let $\pi_\Lambda : \PP^d \to \PP^{d-g}$ be the projection with center $\Lambda$, and let $C \subset \PP^{d-g}$ be the image of our rational normal curve. We claim that the curve $C$ has degree $d$; that  $C$ has nodes at the points corresponding to the pairs $p_i,q_i$; and that $C$ has no other singularities.
%
%Thus we have to check that
%\begin{enumerate}
%\item $\Lambda$ is disjoint from the rational normal curve, so that the map $\pi_\Lambda$ is regular on $\PP^1$;
%\item $\Lambda$ does not meet any secant or tangent lines to $\PP^1$ other than the $L_i$, so that the map $\pi_\Lambda$ is an isomorphism away from the points $p_i, q_i$; and
%\item for each $i$, the 3-plane $\overline{2p_i + 2q_i}$ spanned by the tangent lines to $\PP^1$ at $p_i$ and $q_i$ meets $\Lambda$ in only one point (the point of intersection of $\Lambda$ with $L_i$), so that the common image point $\pi_\Lambda(p_i) = \pi_\Lambda(q_i) \in C$ is indeed a node of $C$.
%\end{enumerate}
%
%Each of these assertions can be verified by a  dimension count. Denote by $\Sigma_{d-g}(L_i) \subset \GG(g-1, d)$ the Schubert cycle of $(g-1)$-planes in $\PP^d$ meeting $L_i$. The cycles $\Sigma_{d-g}(L_i)$ have codimension $d-g$ in $\GG(g-1,d)$. By Proposition~\ref{independence on rnc} the lines $L_i$ are linearly independent , the Schubert cycles intersect properly in a subvariety
%$$
%X := \bigcap_{i=1}^g \Sigma_{d-g}(L_i) \subset \GG(g-1,d)
%$$
%of dimension $g(d-g+1) - g(d-g) = g$. (Thus a $(g-1)$-plane $\Lambda \in X$ is determined by its points of intersection with the $L_i$.) Since $\Lambda$ corresponds to a general point of $X$, we 
% have to show that the locus of $(g-1)$-planes violating each of the conditions above has strictly smaller dimension; we leave these verifications as Exercise~\ref{} below.
%
%Next, we project again: we take $\Gamma \cong \PP^{d-g-3} \subset \PP^{d-g}$ a general $(d-g-3)$-plane, and consider the projection $\pi_\Gamma : C \to \PP^2$. We claim that this map is regular and birational onto a rational nodal plane curve $C_0$, so that $C_0$ has $\binom{d-1}{2}$ nodes $r_1,\dots,r_{\binom{d-1}{2}}$, of which $r_1,\dots,r_g$ are the images of the nodes of $C$. Again, this requires a series of verifications: we have to show that the locus of planes $\Gamma \cong \PP^{d-g-3} \subset \PP^{d-g}$ for which each part of the statement fails has dimension strictly less than $\dim \GG(d-g-3, d-g) = 3(d-g-2)$; again, we leave this as an exercise.
%
%Finally, given all this, Lemma~\ref{local geometry of Severi} implies that we can find an arc $\{C_t\}$ in the Severi variety $\overline V_{d,g}$ whose general member has $\binom{d-1}{2} - g$ nodes and whose special member $C_0$ is the curve we've just constructed, with the $\binom{d-1}{2} - g$ nodes of $C_t$ specializing to the nodes $r_{g+1}, \dots r_{\binom{d-1}{2}} \in C_0$.
%
% Taking the normalization of the total space of this family yields a family of smooth curves of genus $g$ specializing to a $g$-nodal rational curve, as desired. 
% 
% %\fix{this NEEDS a proof. the normalization of a flat family
%%is not always the normalization of each fiber! In this case the general fiber gets normalized, but not the new nodes in the
%%special fiber.}
%\end{proof}

\section{Exercises}

%\begin{exercise}\label{symmetric power vs Hilbert scheme}
%\begin{enumerate}
% \item If $X$ is a smooth curve, then the Hilbert scheme of finite subschemes of $X$ of degree $d$ is
% isomorphic to the symmetric product of $d$ copies of $X$.
% \item If $X$ is a singular curve or any variety of dimension $r \geq 2$, the symmetric power $X^{(d)}$ is \emph{not} the Hilbert scheme of subschemes of dimension 0 and degree $d$ on $X$. 
% 
% %\fix{maybe needs a hint, especially since we can't do even the first part!}
%\end{enumerate}
% \end{exercise}


\begin{exercise}
It is not an accident that we can characterize a fine moduli space $M$ in terms of the maps into it. 
 Let $X$ be a category, and $F,G$ two functors from $X$ to the category of sets.
 A morphism $\eta: F\to G$ in the category of functors is what is called a \emph{natural transformation}:
 for every object $a\in X$ there is a morphism $\eta_a:F(a) \to G(a)$ such that for every
 morphism $f: a\to b$ in $X$ the compositions $G(f)\circ \eta_a$ and $\eta_b\circ F(f)$
 are equal. 
\begin{enumerate}
 \item Prove Yoneda's Lemma: If $X$ is any category, and $F$ is a contravariant functor from $X$ to the category of sets, then 
 $$
 \Hom_{\hbox{\scriptsize Functors on $X$}}(\Hom_X( -, Z), F) = F(Z)
 $$
 \item Conclude that if the functors $\Hom_X( -, Z)$ and $\Hom_X( -, Z')$ are isomorphic in the functor category, 
 then $Z \cong Z'$ in $X$; that is, the functor $\Hom_X( -, Z)$ determines the object $Z$ up to isomorphism.
 \end{enumerate}
\end{exercise}

\begin{exercise}\label{deg of disjoint union}
Suppose that a scheme $X\subset \PP^n$ is the disjoint union of subschemes $Y,Z$. Show that the Hilbert polynomial of
$X$ is the sum of the Hilbert polynomials of $Y$ and $Z$. What statement can you make about the Hilbert functions?
\end{exercise}

\begin{exercise}
More generally, suppose that a scheme $X\subset \PP^n$ is the union of subschemes $Y,Z$. Show that the Hilbert polynomial of
$X$ is the sum of the Hilbert polynomials of $Y$ and $Z$ minus the Hilbert polynomial of $Y\cap Z$. 
\end{exercise}

\begin{exercise}
Let $H \subset \PP^3$ be a 2-plane; let $C \subset H$ be a plane cubic curve and $p \in H \setminus C$ and point in $H$ not on $C$; let $X = C \cup \{p\}$.
\begin{enumerate}
\item Show that the Hilbert polynomial of $X$ is $p_X(m) = 3m+1$.
\item Show that the smallest value of $m_0$ satisfying the statement of Lemma~\ref{} is 4.
\end{enumerate}
\end{exercise}

\begin{exercise}\label{rational normal hilbert}
Use an  argument like that of Proposition~\ref{hilb of twisted cubics} to show that the restricted Hilbert scheme $\cH^\circ \subset \cH_{0,r,r}$ of rational normal curves $C \subset \PP^r$ is irreducible of dimension $r^2+2r-3$.
\end{exercise}

\begin{exercise}\label{hilb at a ci}
If $C = X\cap Y\subset \PP^3$ is a complete intersection of surfaces of degrees $d,e$, then
$Hilb$ is smooth at the point $[C]$, of dimension $2\binom{3+d,3}-4$ if $d=e$
or $\binom{3+d}{3} +\binom{3+e}{3} -\binom{3+e-d}{3} -3$ if $d<e$.

Hint: The normal bundle is $\sN = \sO_C(d)+\sO_C(e)$. To prove smoothness, use
Exercise~\ref{ci is acm} to compute $H^0(\sN)$.
\end{exercise}

%footer for separate chapter files

\ifx\whole\undefined
%\makeatletter\def\@biblabel#1{#1]}\makeatother
\makeatletter \def\@biblabel#1{\ignorespaces} \makeatother
\bibliographystyle{msribib}
\bibliography{slag}

%%%% EXPLANATIONS:

% f and n
% some authors have all works collected at the end

\begingroup
%\catcode`\^\active
%if ^ is followed by 
% 1:  print f, gobble the following ^ and the next character
% 0:  print n, gobble the following ^
% any other letter: normal subscript
%\makeatletter
%\def^#1{\ifx1#1f\expandafter\@gobbletwo\else
%        \ifx0#1n\expandafter\expandafter\expandafter\@gobble
%        \else\sp{#1}\fi\fi}
%\makeatother
\let\moreadhoc\relax
\def\indexintro{%An author's cited works appear at the end of the
%author's entry; for conventions
%see the List of Citations on page~\pageref{loc}.  
%\smallbreak\noindent
%The letter `f' after a page number indicates a figure, `n' a footnote.
}
\printindex[gen]
\endgroup % end of \catcode
%requires makeindex
\end{document}
\else
\fi


