%header and footer for separate chapter files

\ifx\whole\undefined
\documentclass[12pt, leqno]{book}
\usepackage{graphicx}
\input style-for-curves.sty
\usepackage{hyperref}
\usepackage{showkeys} %This shows the labels.
%\usepackage{SLAG,msribib,local}
%\usepackage{amsmath,amscd,amsthm,amssymb,amsxtra,latexsym,epsfig,epic,graphics}
%\usepackage[matrix,arrow,curve]{xy}
%\usepackage{graphicx}
%\usepackage{diagrams}
%
%%\usepackage{amsrefs}
%%%%%%%%%%%%%%%%%%%%%%%%%%%%%%%%%%%%%%%%%%
%%\textwidth16cm
%%\textheight20cm
%%\topmargin-2cm
%\oddsidemargin.8cm
%\evensidemargin1cm
%
%%%%%%Definitions
%\input preamble.tex
%\input style-for-curves.sty
%\def\TU{{\bf U}}
%\def\AA{{\mathbb A}}
%\def\BB{{\mathbb B}}
%\def\CC{{\mathbb C}}
%\def\QQ{{\mathbb Q}}
%\def\RR{{\mathbb R}}
%\def\facet{{\bf facet}}
%\def\image{{\rm image}}
%\def\cE{{\cal E}}
%\def\cF{{\cal F}}
%\def\cG{{\cal G}}
%\def\cH{{\cal H}}
%\def\cHom{{{\cal H}om}}
%\def\h{{\rm h}}
% \def\bs{{Boij-S\"oderberg{} }}
%
%\makeatletter
%\def\Ddots{\mathinner{\mkern1mu\raise\p@
%\vbox{\kern7\p@\hbox{.}}\mkern2mu
%\raise4\p@\hbox{.}\mkern2mu\raise7\p@\hbox{.}\mkern1mu}}
%\makeatother

%%
%\pagestyle{myheadings}

%\input style-for-curves.tex
%\documentclass{cambridge7A}
%\usepackage{hatcher_revised} 
%\usepackage{3264}
   
\errorcontextlines=1000
%\usepackage{makeidx}
\let\see\relax
\usepackage{makeidx}
\makeindex
% \index{word} in the doc; \index{variety!algebraic} gives variety, algebraic
% PUT a % after each \index{***}

\overfullrule=5pt
\catcode`\@\active
\def@{\mskip1.5mu} %produce a small space in math with an @

\title{Personalities of Curves}
\author{\copyright David Eisenbud and Joe Harris}
%%\includeonly{%
%0-intro,01-ChowRingDogma,02-FirstExamples,03-Grassmannians,04-GeneralGrassmannians
%,05-VectorBundlesAndChernClasses,06-LinesOnHypersurfaces,07-SingularElementsOfLinearSeries,
%08-ParameterSpaces,
%bib
%}

\date{\today}
%%\date{}
%\title{Curves}
%%{\normalsize ***Preliminary Version***}} 
%\author{David Eisenbud and Joe Harris }
%
%\begin{document}

\begin{document}
\maketitle

\pagenumbering{roman}
\setcounter{page}{5}
%\begin{5}
%\end{5}
\pagenumbering{arabic}
\tableofcontents
\fi


\chapter{Fine moduli spaces} 
\label{Moduli chapter}\label{ModuliChapter}

\section{What is a moduli problem?}

Algebraic geometry is almost unique among geometric theories in that the objects involved---varieties,  schemes or maps between them---can be parametrized by other varieties or schemes. The set of submanifolds of a given manifold, or more generally of maps between two given manifolds, seems too large to be given the structure of a finite-dimensional manifold itself. By contrast, any algebraic variety is specified by a finite collection of polynomials, which in turn have a finite number of coefficients, so it's not too far-fetched that the collection of all varieties with specified numerical invariants, or morphisms between two given varieties, could be given the structure of a ``moduli space'' that is a variety (or scheme or\dots) in its own right.

For an example---perhaps the original one---projective plane curves of degree $d$
are in natural one-to-one correspondence with the forms of degree $d$ modulo the group of nonzero scalars---that is, with the points of the dual of the projective space
$ \PP(H^0(\sO_{\PP^2}(d)))=\PP^{\binom{d+2}{2}-1} $.
Thus, for example, plane cubics are parametrized by $\PP^{9}$, and a  family $\cC \to B$ of cubics corresponds to a map $\phi: B \to \PP^9$.
In this chapter, we'll give a general framework for the notion of moduli space, introducing the main examples that we will treat in this book.

\begin{figure}
\vskip-10pt
\inprogress
\scalebox{1.25}{%
$
  \vcenter{\hbox{\includegraphics[height=0.76in,width=0.90in,trim=0 40 20 40,clip]{"main/Fig06-1a"}}}\quad
  \vcenter{\hbox{\includegraphics[height=0.68in]{"main/Fig06-1b"}}}\quad
  \vcenter{\vskip-1pt\hbox{\includegraphics[height=0.7in,width=0.85in]{"main/Fig06-1c"}}\vskip1pt}\quad
  \vcenter{\vskip-3pt\hbox{\includegraphics[height=1.28in,trim=0 0 27 0,clip]{"main/Fig06-1d"}}\vskip3pt}
$
}
\vskip-10pt
 \caption{One-parameter family of plane cubics.}
\end{figure}
 
There are several ways in which the possibility of making moduli spaces has been useful in algebraic geometry. First, the existence of a moduli space that  parameterizes objects of a certain type allows us to speak of ``the general object," meaning that we allow ourselves to avoid the ``special'' properties of objects parameterized by closed subvarieties of the moduli space. We have already used this
possibility in many places in this book. 

Second, it allows us to speak coherently about
families of objects. Some moduli spaces carry \emph{universal families}, and every nice family of the sort of objects
they parameterize is pulled back from this one by a unique map.  

This idea was already exploited informally in the nineteenth century in the guise of ``preservation of number," used to count configurations of points or curves with a given property by specializing the 
data, and we have also exploited this idea in
Chapter~\ref{JacobianChapter} to explain the count of odd and even theta characteristics on a general curve by appealing to the existence of a specialization to a hyperelliptic curve. In a related manner, we have already seen how the fact that invertible sheaves of degree $d$ on a curve $C$ are parametrized by a $g$-dimensional variety allows us to prove the ``$g+3$" theorem (Theorem~\ref{g+3 theorem}).

Third, to the extent that we can describe the intersection theory of a moduli space, it opens up the possibility of doing enumerative geometry on it to count solutions of geometric problems---and in particular to prove the existence of solutions. For example, knowing that the parameter space for lines in $\PP^3$ is the Grassmannian $\GG(1,3)$, a projective variety of dimension 4, and that the condition that the set of lines meeting a given curve of degree $d$
is a divisor linearly equivalent to  $d$ times the hyperplane section, we can conclude  that there exists a line in $\PP^3$ meeting any four given curves \cite[Section 3.4.1]{3264}. We will use the same idea to prove the much deeper existence of certain linear series on all curves (the ``existence" half of the Brill-Noether theorem, discussed in Chapter~\ref{Brill-Noether}). 

\begin{figure}
\centerline {\includegraphics[height=1.6in]{"main/Fig06-2"}}
 \caption{A general  one-parameter linear family of lines in $\PP^3$---that is, the family of lines
 contained in a general plane and passing through a general point in that plane---meets a space curve $C$ in
 $\deg C$ points.}
\end{figure}

In modern terms, a \emph{moduli problem} consists of a class of objects in algebraic geometry---schemes, subschemes of a given scheme, sheaves on schemes, maps of schemes, typically defined by some common attributes---and a notion of what it means to have a \emph{family} of these objects parametrized by a scheme $B$. The notion is formalized in the idea of a \emph{moduli functor}, 
which associates to each scheme $B$ the set of families over $B$ of the given sort. Examples will make this vague notion more concrete.

\subsection{Some moduli problems}

\begin{enumerate}\label{list of moduli problems}

\item \emph{Effective divisors on a given curve}. The objects are effective divisors of given degree on a given smooth, projective curve $C$. A family of such divisors is a subscheme $\cD \subset B \times C$, flat of degree $d$ over $B$. 
Here we are using
the equivalence between divisors of degree $d$ on a smooth curve and degree $d$ subschemes of the curve. The moduli space is the $d$-th symmetric power $C_d$ of $C$, discussed in Section~\ref{symmetric section}.

\item \emph{Invertible sheaves on a given curve}. The objects are invertible sheaves on a given smooth projective curve $C$. A family over a scheme $B$ is an equivalence class of invertible sheaves $\cL$ on $B \times C$ whose restriction to each fiber of $B \times C$ over $B$ has degree $d$, where two families $\cL$ and $\cL'$ on $B \times C$ are equivalent if $\cL\cong \cL' \otimes \cM$, where $\cM$ is  an invertible sheaf pulled back from $B$. Usually one restricts attention to invertible sheaves 
whose restrictions to each fiber $b\times C$ have a given degree $d$.
The moduli spaces are the Jacobian and Picard varieties, discussed in Section~\ref{Picard section}.

\item \emph{Moduli of smooth curves}. The objects are isomorphism classes of smooth, projective curves of genus $g$. A family over $B$ is an equivalence class of smooth, projective morphisms $f : \cC \to B$ whose fibers are curves of genus $g$, where two such families $f, f'$
are equivalent if there is an isomorphism from the source of $f$ to the source
of $f'$ making the diagram
$$
\begin{diagram}[small]
\cC && \rTo^\cong && \cC'\\
&\rdTo_f&&\ldTo_{f'}\\
&&B
\end{diagram}
$$
commute.

The moduli spaces $M_g$ of curves are harder to construct, and we will have a separate discussion of them in the following chapter. Nonetheless, it will be useful at several points in this chapter to assume their existence.

\item \emph{Hurwitz spaces}. An object is a smooth projective curve $C$ of a given genus, together with a map $f: C\to \PP^1$ of given degree, up to isomorphisms of the curves that commute with the map, as in the diagram
$$
\begin{diagram}[small]
C && \rTo^\cong && C'\\
&\rdTo_f&&\ldTo_{f'}\\
&&\PP^{1}
\end{diagram}.
$$
A family over B is a family $\cC \to B$ of smooth projective curves  of a given genus, together with flat map $\cC \to B\times \PP^{1}$ of degree $d$.
 Often the allowable ramification indices are specified. We will discuss the simplest Hurwitz scheme in Chapter~\ref{CurvesModuliChapter}.

\item \emph{Severi varieties}. The objects are plane curves
of given degree $d$ and geometric genus $g$. If $g \neq {d-1\choose 2}$
the curves will necessarily be singular but are often
constrained to have only mild singularities, usually only nodes. We will discuss the Severi variety in Chapter~\ref{CurvesModuliChapter}.

\item \emph{Hilbert schemes}. The objects are subschemes of a given projective space;   a family over $B$ is a subscheme $\cC \subset B \times \PP^r$, flat over $B$. Since the Hilbert polynomials of the fibers of a flat family are all equal (\cite[Section III.9]{Hartshorne1977}, the Hilbert scheme is the disjoint union of subschemes corresponding to particular Hilbert polynomials, and these subschemes are of finite type.
If $B$ is reduced then the flatness condition is equivalent to the  constancy of the Hilbert polynomial \cite[]{Hartshorne1977}.

In Chapter~\ref{HilbertSchemesChapter} we will study the
open set consisting of smooth curves $C\subset \PP^{r}$ of given degree and genus.

\end{enumerate}

\section{What is a solution to a moduli problem?}

Typically what we want from a solution to a moduli problem is to understand all possible families of the objects
in question. In particular, the individual objects should be in natural one-to-one correspondence with the closed points of the
moduli space.  The word \emph{natural} is the key. Most of the time, the set of objects we are interested in has cardinality $2^{\aleph_0}$, as do all positive-dimensional varieties $M$ over $\CC$, so a mere  bijection between the points of $M$ and the objects to be parametrized is meaningless.

In the nicest situations there is a \emph{universal} family $\phi: \cX\to M$ of these objects over $M$,
such that any family over a scheme $B$  is pulled back from the one on $M$ via a unique morphism $B\to M.$  Such a space $M$ with its universal family $\phi$, if it exists, is called a \emph{fine moduli space}. This can be expressed more abstractly but more succinctly by saying that the moduli scheme \emph{represents the moduli functor}, which means there is an isomorphism of functors:
$$
\{ B \mapsto \text{families of } X \text{ over } B \} \cong \{ B\mapsto {\rm{Mor}}_{{\rm Schemes}}(B, M) \}.
$$
If $M$ is a fine moduli space then the identity map $M\to M$ corresponds to the ``universal family'' $\phi: \cX \to M$. 
The Hilbert scheme, as well as $Div_d(C)$ and $\Pic_d(C)$ are fine moduli spaces but the moduli space of curves
and the Hurwitz schemes are not.

If a fine moduli space and its universal family exist, then it is unique up to unique isomorphism: given two avatars $M$ and $M'$
the universal family on $M$ corresponds to a map $M\to M'$, and we similarly produce a map $M'\to M$. The pullback of the universal family on $M$ by the composition of these two maps is again the universal family, so the composition is the identity map.

Although the moduli space of curves and the Hurwitz spaces are not fine moduli spaces, they are still defined
in a way that makes them unique, as we shall see below.

\def\eps{{\epsilon}}
One of the useful features of a fine moduli space is that it makes the computation of tangent spaces relatively easy.
Recall that if $(R,\gm)$ is the local ring at a point $m$ on a variety $\cM$ then the (Zariski) tangent
space to $M$ at $m$ is the vector space of linear functionals $Hom_{R/\gm}(\gm/\gm^2, R/\gm)$.   Assuming that
$R$ contains its residue field $k := R/\gm$, such functionals
are precisely the restrictions to $\gm/\gm^2$ of the ring homomorphisms $R \to k[\eps]/(\eps^2)$ inducing the identity on $k$.
 If $\cM$ is a fine moduli space for some functor $F$, then such maps are in one-to-one correspondence
with the set $F(E)$ of families over $E := \Spec(k[\eps]/(\eps^2))$. 


The vector space structure on the tangent space is also accessible from this description:  the
 sum of tangent vectors corresponding to families $X_i \to E$ is the restriction to the diagonal
 $E \subset E\times E$
 of the product family $X_1 \times X_2 \to E\times E$.
Often one can to compute the tangent space to a moduli space before even knowing that the moduli space exists!

%Hilbert schemes are fine moduli spaces for subschemes of a given scheme, but 
%fine moduli spaces do not exist for all moduli problems. As we'll explain in Section~\ref{coarse moduli}, there does not exist a universal family of abstract curves of genus $g$: the moduli space of curves is
%not a fine moduli space, but has a slightly weaker property.

\section{Hilbert schemes}\label{hilbert scheme section}

The Hilbert scheme is a fine moduli space representing the functor of flat families of subschemes of $\PP^r$,
that is, the functor that takes a scheme $B$ over $\CC$ to the set of subschemes $\sX \subset B\times \PP^r$
that are flat over $B$; a morphism $f: B'\to B$ induces a map carrying a family to the pullback of the family by $f$.

\begin{example}[The Hilbert scheme of plane curves]\label{Hilb for plane curves}
Let $p(m) := dm+1-{d-1\choose 2}$ be the Hilbert polynomial of a plane curve of degree $d$, and consider the family $Hilb_{p(m)}(\PP^2)$ of schemes $X \subset \PP^{2}$ having Hilbert polynomial $p$. Since 
$\dim X = \deg p = 1$, $X$ has at least one component that is 1-dimensional. The union of such components must have degree
equal to $d$, the leading coefficient of $p(m)$. Since the Hilbert polynomial  of this union is also equal
to $p(m)$ we see that $X$ is in fact a (possibly nonreduced and/or reducible) plane curve of degree $d$. The space $Hilb_{p(m)}(\PP^2)$ is the projective space $\PP^{{d+2\choose 2}-1}$ of forms of degree $d$,
and the universal family is the projection 
$$
\{(x,F) \in \PP^2 \times \PP^{{d+2\choose 2}-1} \mid F(x) = 0\} \to \PP^{{d+2\choose 2}-1}.
$$

More typically, the set of curves $C \subset \PP^r$ of degree $d$ and genus $g$ corresponds to a subset of the Hilbert scheme parametrizing subschemes of $\PP^r$ with Hilbert polynomial $p(m) = dm - g + 1$, though not all schemes with this Hilbert polynomial are purely one-dimensional subschemes, as we shall see.
\end{example}
 
 In this section, we'll compute 
 the Zariski tangent space to the Hilbert scheme at a point,  and we'll sketch the construction of the scheme itself. For a rigorous treatment including many generalizations,  see~\cite{HomogHilbert} or \cite{MR2222646}. In Chapter~\ref{HilbertSchemesChapter}, we'll describe  the
 open subsets of smooth curves in the Hilbert schemes of curves of low degree and genus in $\PP^3$ in more detail. 

\subsection{The tangent space to the Hilbert scheme}\label{tan hilbert section}

Following the general recipe for tangent spaces to fine moduli spaces, we need to understand flat families
of projective schemes over $E := \Spec \CC[\eps]/(\eps^2)$. Recall (\cite[p.182]{Hartshorne1977}) that if $X\subset \PP^r$ is a smooth subscheme then
 the normal bundle $\sN_{X/\PP^r}$ of $X$ in $\PP^r$ is defined in terms of the tangent bundles
 of $X$ and $\PP^r$ by the exact sequence: 
$$
0\rTo \sT_X \rTo^{\phi} \sT_{\PP^r}\mid_X \rTo \sN_{X/\PP^r} \rTo 0.
$$
Thus a global section of 
$\sN_{X/\PP^r}$
can be thought of as an infinitesimal motion of $X$.

\begin{figure}
\centerline {\includegraphics[width=3in]{"main/Fig06-0"}}
 \caption{Infinitesimal perturbation of a curve by a normal vector field.}
\end{figure}
\fix{Silvio: This picture (spuriously) suggests a trivial deformation. There was an earlier picture of this, for plane curves, showing the perturbed curve crossing the original one.}

To define the normal sheaf for locally complete intersection subschemes $X$ we first define the \emph{conormal sheaf}
to be the kernel of the dual  of $\phi$, which is the natural map $\Omega_{\PP^r}|_X \to \Omega_X$,
and since $X$ is locally a complete intersection, this corresponds to the exact sequence:
$$
0\to \sI_{X/\PP^r}/\sI_{X/\PP^r}^2 \to \Omega_{\PP^r}|_X \to \Omega_X \to 0.
$$
We thus define $\sN_{X/\PP^r} := \sHom(\sI_{X/\PP^r}/\sI_{X/\PP^r}^2, \sO_X)$.

By analogy we define the \emph{normal sheaf} of any subscheme $X\subset \PP^r$
$$
\sN_{X/\PP^r} := \sHom(\sI_{X/\PP^r}/\sI_{X/\PP^r}^2, \sO_X).
$$
The following theorem shows that this
has the same connection with infinitesimal motions for arbitrary subschemes $X$.


\begin{theorem}\label{tangent space of Hilb}
Let $X\subset \PP^r$ be a subscheme with Hilbert polynomial $p(m)$ and let
$E = \Spec\CC[\eps]/(\eps)^{2}$. The flat families 
$\sX \subset \PP^r\times E$ specializing to $X$ at the closed point defined by $(\eps)$
are in natural one-to-one correspondence with the vector space $Hom_{\PP^r}(\sI_{X/\PP^r}, \sO_X) = H^0(\sN_{X/\PP^r})$, which is thus the tangent space to the Hilbert scheme $Hilb_{p}(\PP^r)$ at $[X]$.
\end{theorem}

\begin{proof}
We will actually treat the analogous result for an affine subscheme $X \subset \AA^r = \Spec S$, where
$S = \CC[x_{1}, \dots, x_{r}]$; since our construction is natural, it will patch on an affine cover to give the projective case stated above. 

An $E$-module $M$ is flat  over $E$ if and only if $\Tor_1^E(\CC,M) = 0$. 
Since the free resolution of $\CC$ as an $E$-module has the form
$$
\cdots \rTo^\eps E \rTo^\eps E \rTo^\eps E \rTo \CC \rTo 0
$$
we have $\Tor_1^E(\CC,M)=0$ if and only if the submodule of $M$ annihilated by $\eps$ is $\eps M$.

We first construct a flat family from a homomorphism: Let $I = (g_1,\dots, g_t)\subset S = \CC[x_1,\dots, x_r]$ be the ideal defining $X$. 
Note that $Hom_S(I/I^2, S/I) = Hom_S(I,S/I)$. Let $\phi: I\to S/I$ be a homomorphism, and let $h_i\in S$ be
any element reducing to $\phi(g_i)$ modulo $I$.  The ideal
$$
I' := (g_1+\eps h_1,\dots, g_t+\eps h_t)\subset S[\eps]/(\eps^2) =: S'
$$
defines a scheme $\Spec S'/I'$  over $E$ which restricts to $X$ modulo $(\eps)$; that is, $I' +(\eps) = I$. 

To see that $I'$ is 
independent of the lifting chosen, suppose that $k_1,\dots,k_t\in I$ so that the elements $g_i+\eps(h_i+k_i)$ are a different lifting,
generating an ideal $I''$.
Writing $k_i = \sum r_{i,j}g_j$ we have 
$$
g_i+\eps(h_i+k_i) = g_i+\eps h_i+ \sum \eps r_{i,j}(g_j+\eps h_j)
$$
 so 
$I'' \subset I$, and symmetrically $I' \subset I''$.

To prove flatness, suppose that $a\in S'$ and  $\eps a  \in I'$,
so that we can write  $\eps a= \sum(r_i+\eps s_i)(g_i+\eps h_i)$ for some $r_{i}, s_{i} \in S$.
It follows that 
$\sum r_i\g_i = 0$. Since $\phi$ is a homomorphism this implies $\sum r_i h_i \in I.$ Thus
$\eps a \in  \eps I\subset S'$, whence $a\in I + (\eps)$.  Writing $a =\sum p_i g_i+\eps b'$
and using the relations $g_i \equiv -\eps h_i \hbox{ mod } I'$ we get
 $a \equiv \eps (-\sum p_i h_i+b') \hbox{ mod } I'$, as required.

Finally, starting from a flat family $S'/I'$ over $E$ with $I' + (\eps) = I +\eps$, 
let $J$ be the image of $I'$ in $S'/(\eps I) = S \oplus (\eps S)/(\eps I)$. We claim that $J$ is the graph of a homomorphism $\phi: I \to  (\eps S)/(\eps I) \cong S/I$. 

Since $I' + \eps S = I +\eps S$,  the projection  to the  summand $S\subset S'$ maps $J$ onto $I$. 
To prove that $J$ is the graph of a homomorphism, it suffices to show that this projection is an isomorphism.
The kernel is
the intersection of $J$ with $(\eps S)/(\eps I )$, so we must show that
if $r\in S$ and $\eps r \in I'$ then $\eps r \in \eps I$. 

Since $\eps r \in I'$, the condition of flatness implies
that the image of $r$ in $S'/I'$ is in $\eps(S'/I')$, which is to say that $r \in I' + \eps S = I +\eps S$.
Thus $r \in  I + \eps S,$ whence $\eps  r \in \eps I$ as required.

This shows that
$J$ is the graph of a homomorphism $\phi$ such that
 $I'$ is generated by $\{g+\phi g\mid g\in I\}$, so the two constructions are inverse to one another.
\end{proof}

\begin{example}\label{Hilb for plane curves-continued}
If $C$ is the plane curve of degree $d$ defined by a form $F$, then the ideal sheaf of $C$ is $\sO_{\PP^2}(-d)$, and thus
$$
\sHom(\sI_{C/\PP^2}/\sI_{C/\PP^2}^2, \sO_{C}) = \sO_{\PP^2}(d)|_C = \sO_C(d).
$$
From the exact sequence 
$$
0\rTo\sO_{\PP^2}\rTo^F\sO_{\PP^2}(d) \rTo\sO_C(d)\rTo 0
$$
we deduce that the dimension of the tangent space to $Hilb_{p(m)}(\PP^2)=\PP^{\binom{d+2}{2}-1}$  at $C$
with $p(m) = dm+1-{d-1\choose 2}$,
is $h^0(\sO_C(d)) = h^0(\sO_{\PP^2}(d))-1 = \dim \PP^{\binom{d+2}{2}-1}$, as expected.
\end{example}

\

\subsection{Parametrizing twisted cubics} By Lemma~\ref{smooth is open} below, the set of twisted cubics---that is, smooth, irreducible, nondegenerate curves of degree 3 in $\PP^3$---is an open subset of the Hilbert scheme $Hilb_{3m+1}(\PP^3)$. As we've seen, a twisted cubic curve $C \subset \PP^3$ can be described as the zero locus of three homogeneous quadratic polynomials $Q_1, Q_2$ and $Q_3$ in the homogeneous coordinates on $\PP^3$; to specify the twisted cubic we could just list the $3 \times 10 = 30$ coefficients of these. But of course we could replace the three quadrics $Q_i$ with any three independent linear combinations of them; what matters---and what is naturally associated to $C$---is the vector space $V = \langle Q_1, Q_2, Q_3 \rangle \subset H^0(\cO_{\PP^3}(2))$. This suggests that we consider the map of sets
$$
h : \{ \text{twisted cubic curves } C \subset \PP^3 \} \to G = G(3, H^0(\cO_{\PP^3}(2)))
$$
obtained by associating to a twisted cubic $C$ the second graded piece of its homogeneous ideal. 

This differs significantly from the example of plane curves given at the beginning of this chapter: there, the objects to be parametrized were the zero locus of a single polynomial, and we could vary those coefficients arbitrarily and still have a plane curve; thus, the image of the analogous map was open in the projective space $\PP^{\binom{d+2}{2}-1}$. But if we generically
perturb the coefficients of the three quadratic polynomials $Q_i$ defining the twisted cubic the resulting quadrics will
be a complete intersection, generating
 the ideal of a set of eight points. Thus the image of the map $h$ does not contain an open set of $G$.
We will give equations defining the image in Section~\ref{hilb construction}, and we can consider it
the Hilbert scheme of twisted cubics.

As in this case, we will mainly be   interested in the subsets of the Hilbert schemes corresponding to smooth irreducible curves:

\begin{lemma}\label{smooth is open}
Suppose that $X \to B$ is a flat family of projective schemes (that is, the projection to the second factor of 
$X\subset B\times \PP^n$  is flat). The points $b\in B$ such that the fiber $X_b$ is smooth and irreducible form an open set.
\end{lemma}

\begin{proof}
To prove the Lemma we may assume that $B = \Spec A$ is affine and irreducible. Let $x_{0}\dots, x_{r}$ be homogeneous coordinates
on $\PP^{r}$, and suppose that 
$X\subset \PP^{r}_{\CC}$ is defined by the ideal $I_{X} = (F_{1}, \dots, F_{n}) \in A[x_{0}, \dots x_{r}]$.
Since $X$ is flat over $B$ the fiber dimension $d$ is constant, and the singular locus 
of a given fiber $X_{b}$ is  the subscheme of $X_{b}$ defined by the ideal $J$ of
$(r-d)\times (r-d)$ minors of the Jacobian matrix $(\partial F_{i}/\partial x_{j})$.

Let $Y\subset B\times \PP^{r}$ be the closed subscheme defined by $J$ alone. 
It follows that $Y\cap X \subset X$ defines a scheme of singular points of fibers of the family, so this set is closed
in $X$.
Since the map $X \to B$ is projective, the image of $X\cap Y$ is closed, and its complement is open.
 Finally, a smooth fiber $X_b$ is irreducible if and only if it is connected. This holds if and only if $h^0(\sO_{X_b}) <2$ and by the flat base change theorem~\cite[Theorem 12.11]{Hartshorne1977} this set is open as well.
\end{proof}


%\begin{proof}
%Since $B$ is smooth, it's closed points are locally complete intersections, so if $p\in X$ is a singular point of $X$ then the fiber through $p$ is singular. Since the singular subset of $X$ is closed and the family is projective, the image of the singular set in $X$ is closed, and we may remove it and assume that $X$ is smooth from the start. Now the condition that a point on a fiber is singular is a closed condition on $X$, so again its image in $B$ is closed. Finally, a smooth fiber $X_b$ is irreducible if and only if it is connected. This holds if and only if $h^0(\sO_{X_b}) = 1$ and by the flat base change theorem~\cite[Theorem 12.11]{Hartshorne1977} this set is open as well.
%\end{proof}


\begin{proposition}\label{hilb of twisted cubics}
The open subset $\cH^\circ$ of the Hilbert scheme $Hilb_{3m+1}(\PP^3)$ parametrizing twisted cubics is irreducible of dimension 12.
\end{proposition}

\begin{proof}  Let $C_0 \subset \PP^3$ be a twisted cubic, and consider the family of translates of $C_0$ by automorphisms $A \in \PGL_4$ of $\PP^3$: that is, the family
$$
\cC = \{ (A, p) \in \PGL_4 \times \PP^3 \; \mid \; p \in A(C_0) \}.
$$
Via the projection $\pi : \cC \to \PGL_4$, this is a family of twisted cubics, and so it induces a map
$$
\phi : \PGL_4 \to \cH^\circ.
$$
Since every twisted cubic is a translate of $C_0$, this is surjective, with fibers isomorphic to the stabilizer of $C_0$, that is, the subgroup of $\PGL_4$ of automorphisms of $\PP^3$ carrying $C_0$ to itself. By Exercise~\ref{projective automorphism}, every automorphism of $C_{0}$ is induced by an automorphism of $\PP^{3}$, so the stabilizer is isomorphic to $\PGL_2$ and  thus has dimension 3. Since $\PGL_4$ is irreducible of dimension 15, we conclude that $\cH^\circ$ is irreducible of dimension 12.
\end{proof}


\subsection{Construction of the Hilbert scheme in general}\label{hilb construction}

The Hilbert scheme is more complicated than would appear from the examples above, and this is even true
for the Hilbert polynomial $3m+1$. There are many subschemes of $\PP^3$ that have the same Hilbert polynomial $3m+1$ as a twisted cubic---for example, the union of a plane cubic and a point---and are not the intersection of the quadrics containing them. (See Exercises \ref{characterization of degree} and~\ref{deg of disjoint union}). In Chapter~\ref{HilbertSchemesChapter} we will discuss many more components of Hilbert schemes.

A fundamental result of~\cite{Matsusaka} provides a place to start:

\begin{lemma}\label{matsusaka}
Let $p(m) \in \QQ[m]$ be a polynomial. There exists an integer $m_0$ such that

\begin{enumerate}  

\item For any subscheme $X \subset \PP^r$ with Hilbert polynomial $p_X = p$ we have
$$
h^0(\cI_{X/\PP^r}(m)) = \binom{m+r}{r} - p(m) \quad \text{for all } m \geq m_0
$$
or in other words the Hilbert function of $X$ agrees with the Hilbert polynomial $p_X = p$ for all $m \geq m_0$; and

\item For any subscheme $X \subset \PP^r$ with Hilbert polynomial $p_X = p$ and
the saturated ideal of $X$ is defined by forms of degree $\leq m$.
\end{enumerate}
\end{lemma}

Note that  for any given $X$ the existence of an $m_0$ satisfying the statement of the lemma is immediate by  Theorem~\ref{Serre-Grothendieck vanishing}. The point of the lemma is that we can find one value of $m_0$ that works for all $X$ with Hilbert polynomial $p$. The following result of~\cite{Gotzmann} provides a method for determining $m_0$. 

\begin{theorem}
The Hilbert polynomial  of the homogeneous coordinate ring of any scheme $X\subset \PP^r$ can be written uniquely in the form
$$
\chi(\sO_X(m) = {m+a_1\choose a_1}+ {m+a_2 -1\choose a_2}+ \cdots+{m+a_s -(s-1)\choose a_s},
$$
with 
$$
a_1\geq \cdots \geq a_s \geq 0
$$
where the binomial coefficients are interpreted as polynomials in $m$. Moreover, the saturated homogeneous ideal of $X$ is
 generated in degrees $\leq s$, and one can take $m_0 = s$ in the construction of the Hilbert Scheme, above.
\end{theorem}

See~\cite{MR1023391} %Green-Gotzmann
for an exposition and a proof. From the coefficients $a_j$ one can read off uniform vanishing theorems for $H^i(\sI_X(m))$
 as well.
 
 For example, the Hilbert polynomial $3m+1$ of the twisted cubic may be written as
 $$
 3m+1 =  {m+1\choose 1}+ {m+1 -1\choose 1}+{m+1 -2\choose 1}+{m+0 -(3)\choose 0},
 $$
 Here $s=4$, and indeed the homogeneous ideal of the union of a plane cubic with a point, also in the plane,
 requires equations of degree 4.
 
 \subsection{Grassmannians}\label{Grassmannian section}
 
 The most fundamental (and simple) of Hilbert schemes are the Grassmannians (including the projective spaces themselves), which parameterize the families of linear subspaces of given dimension in vector spaces or in projective spaces.
For $0\leq k\leq r$ we write $\GG(k,r)$ for the set of $k$-planes in $\PP^{r}$ and identify it with
$G(k+1,r+1)$, the set of $(k+1)$-dimensional vector subspaces of an $(r+1)$-dimensional vector space.
When we want to make the $(r+1)$-dimensional vector space $V$ explicit, we write $G(k+1, V)$ instead.

We embed $G(k+1,V)$ in $\PP(\wedge^{r-k}V) = \PP^{\binom{r+1}{r-k}-1}$ by sending a subspace $W\subset V$ to the 1-quotient $\wedge^{r-k}V \to \wedge^{r-k}(V/W)$. This map is a monomorphism called the \emph{Pl\"ucker embedding}, and its image is an algebraic subvariety, which we take to be the algebraic structure
of the Grassmannian. 


Concretely, if
we choose a basis of $V/W$ so that the projection map $V \to V/W$ is given by an $(r-k)\times (r+1)$
matrix $A$ then the coordinates of the image of $W$ are the $(r-k)\times (r-k)$ minors of $A$,
called \emph{Pl\"ucker coordinates} of $W$. On the open set where the first $(r-k)\times (r-k)$
minor is nonzero, we may multiply by its inverse, and it is not hard to check that the
minors become the entries of the complementary $k \times (r-k)$ submatrix; thus
 $G(k,V)$ is covered by open sets isomorphic to affine $k(r-k)$-space, and  thus $G(k+1,r+1) = \GG(k,r)$ is smooth and
 irreducible of dimension $k(r-k)$. 
 
 
\begin{example}
The Grassmannian $\GG(1,3)$ of lines in $\PP^{3}$ has Pl\"ucker coordinates of a line
$L$ that is the span of points $q,r\in \PP^{3}$ the $2\times 2$ minors of the $2\times 4$ matrix
whose rows are the coordinates of the points:
$$
\begin{pmatrix}
q_{0}&q_{1}&q_{2}&q_{3}\\
r_{0}&r_{1}&r_{2}&r_{3}\\
\end{pmatrix}.
$$
Indexing the minors $p_{i,j}$ by pairs of distinct column indices one can easily prove the
\emph{Pl\"ucker relation}
$$
p_{0,1}p_{2,3} - p_{0,2}p_{1,3}+p_{0,3}p_{1,2} = 0,
$$
and this defines $\GG(1,3)\subset \PP^{5}$ as a quadric hypersurface.
\end{example}.


\def\sW{{\mathcal W}}

The \emph{universal sub-bundle} $\sW \subset V\times G(k+1,r+1)$, also called the \emph{tautological sub-bundle},  is
the vector bundle with fiber
 $W\subset V$ over the point of $G(k+1,V)$ corresponding to $W$. It is universal in the sense that
 given any scheme $X$ and a $(k+1)$-dimensional sub-bundle $\sW'$ of the trivial bundle $V\times X$
there is a unique morphism $X\to G(k+1,V)$ such that the pullback of $\sW$ is $\sW'$.

For a thorough introduction to the Grassmannian, see for example~\cite[Chapters 3--5]{3264}.

 \subsection{Equations defining the Hilbert scheme}\label{eqns of Hilb}

Matsusaka's theorem allows us to define an injective map of sets
$$
h : \left\{ \text{subschemes $X \subset \PP^r$ with $p_X=p$} \right\}  \to G\left(\binom{m_0+r}{r} - p(m_0),\binom{m_0+r}{r} \right)
$$
by sending $X$ to $H^0(\cI_{X/\PP^r}(m_0))$, and its image is the set of closed points of the Hilbert scheme.
It remains to describe the scheme structure.

We observed above that though there are vector spaces $V$ of 3 quadrics in $\PP^{3}$ that define
twisted cubics, a general such vector space  would generate the ideal of
8 points,  not a twisted cubic. What we want to know is how to tell these cases apart algebraically. Consider the multiplication map
$$
V \otimes H^0(\cO_{\PP^3}(1)) \to H^0(\cO_{\PP^3}(3)).
$$
We saw in Chapter~\ref{genus 0 and 1 chapter} that the cokernel of this map is the 10-dimensional space $H^0(\sO_{\PP^1}(9))$, so the image of this map is 10-dimensional, whereas
3 general quadrics form a complete intersection and would have only Koszul syzygies, so
in the case of general quadrics this map would have 12-dimensional image.
This is a map from a 12-dimensional vector space to a 20-dimensional one, and what we've seen is that if $V$ is the net of quadrics containing a twisted cubic, it has a 2-dimensional kernel; that is, it has rank 10. 

Thus if $\cE$ is the universal subbundle on $G = G(3, H^0(\cO_{\PP^3}(2))$, and  $H^0(\cO_{\PP^3}(d))\otimes \sO_G$ is the trivial bundle, then the multiplication map above gives a map of vector bundles
$$
\mu: \cE \otimes H^0(\cO_{\PP^3}(1)) \to H^0(\cO_{\PP^3}(3)).
$$
We can represent this locally as a matrix of functions, and the $11\times 11$ minors of this matrix
define the rank 10 locus, and thus vanish on the points of
of the Hilbert scheme: in a neighborhood of a point in $G$ corresponding to a twisted cubic, the common zero locus of these minors is the locus of nets of quadrics containing a twisted cubic


In fact, the construction of the Hilbert scheme in general is no more structurally complicated than this special case. Given a polynomial $p(m)$, we find a value of $m_0$ that satisfies the statement of Lemma~\ref{matsusaka}; we let
$$
G = G\left(\binom{m_0+r}{r} - p(m_0), \binom{m_0+r}{r} \right)
$$
be the Grassmannian, and let $h$ be the map from the set of subschemes of $\PP^r$ with Hilbert polynomial $p$ to $G$ sending $X$ to $H^0(\cI_{X/\PP^r}(m_0))$. We then get a map of vector bundles  on $G$
$$
\sE \otimes H^0(\cO_{\PP^r}(1)) \to H^0(\cO_{\PP^r}(m_0+1)).
$$
In a neighborhood of a point of $G$ in the image of $h$, the common zero locus of the minors of size $\binom{r+m_0+1}{r} - p(m_0+1)$ of a matrix representative of this map is the image of $h$. Thus these functions define the Hilbert scheme.

\section{Bounding the number of maps between curves}\label{maps between curves}

A priori, the Hilbert scheme parametrizes subschemes of projective space. But the construction is adaptable to many other situations. In this section we'll sketch a proof of such an application. As we have seen, there can be infinitely many maps of given degree from a curve to $\PP^{1}$, and
this is also the case for maps to a curve of genus 1, even modulo the automorphisms of the target. But this is not
the case in higher genus: given two smooth projective curves $C$ and $D$ of genera $g, h \geq 2$, we'll show that there are at most a finite number of nonconstant morphisms $C \to D$. In fact, the number is bounded purely in terms of $g$ and $h$:

\begin{theorem}\label{bounded maps}
Given integers $g,h\geq 2$ there is a bound $N(g,h)$ on the number of distinct non-constant morphisms
from a smooth projective curve $C$ of genus $g$ to a smooth projective curve $D$ of genus $h$.
\end{theorem}

A special case of this result is a bound on the size of the group of automorphisms of a curve of genus $g\geq 2$. We will give a second proof of that result, which doesn't rely on the Hilbert scheme, in Theorem~\ref{finite autos}.

 \begin{proof}
 Hurwitz' Theorem~\ref{Hurwitz} implies a bound on the degree $d$ of a morphism $f : C \to D$ so it suffices to 
 bound the number of morphisms of a fixed degree $d$.
 
 We will use the Hilbert scheme in the relative setting, as Grothendieck originally defined it: Given a base scheme $S$,
the scheme $Hilb_{p(m)}(\PP_{S}^{r})$ represents the functor on $S$-schemes $X\to S$ that associates to 
$X$ the set of subschemes of $\PP_{X}^{r}$, flat over $X$, whose fibers have Hilbert polynomial $p$.

We can construct a family of products of pairs of curves of genera $g$ and $h$ embedded in projective space as follows
(the details don't matter---only the existence):
Let $S_{g}\to S$ be the Hilbert scheme of smooth curves of genus $g$ embedded by invertible sheaves of degree $2g+1$
in $\PP^{g+1}_{S}$
and similarly for $S_{h}$ and curves of genus $h$, and write $([C],[D])$ for the corresponding point in the fiber
of $S_{g}\times_{S}S_{h}$. From the universal families of curves over $S_{g}$ and
$S_{h}$ we may construct a family $\sC \to S= S_{g}\times_{S} S_{h}$ whose fiber over $s$ includes all products of 
pairs of smooth curves of genera $g$ and $h$, embedded in $\PP_{s}^{g+1}\times_{S}\PP_{s}^{h+1}$. Finally we embed
this product of projective spaces by the Segre embedding in $\PP_{S}^{N}$, with $N = (g+2)(h+2)-1$. 

If $\Gamma \subset C \times D \subset \PP^N$ is the graph of a morphism $f : C \to D$ of degree $d$, then $\Gamma$ has genus $g$ and degree $2g+1 + d(2h+1)$, so we know its Hilbert polynomial $p$. The set of
subschemes in this Hilbert scheme that project isomorphically to $C$ and $d$-to-1 to $D$ correspond to the
points of a  locally closed subset, of this Hilbert scheme, and therefore are a scheme of finite type. The fibers
of this scheme over $S$ are the sets of morphisms of degree $d$ from curves of genus $g$ to curves of genus $h$.

We first prove that each fiber is finite---that is, there can only be finitely many maps of degree $d$ from
one fixed curve to another.  By Theorem~\ref{tangent space of Hilb} it suffices for this to show that if we fix the curves $C,D$ and a map $\phi$
with graph $\Gamma_{\phi}$
then the normal bundle 
$$
\sN_{\Gamma_{\phi}/(C\times D} = \sT_{C\times D}/\sT_{\Gamma_{\phi}}
$$
has no sections. The projection onto the first factor identifies $\sT_{\Sigma_{\phi}}$ with $\sT_{C}$,
and the quotient is thus $\phi^{*} \sT_{D}$, which has degree $2-2h<0$. Thus, fiber by fiber,
the scheme of morphisms is finite. 

Since the base $S$ of the family is also a scheme of finite type and the degree of the fiber is semicontinuous,
there is an absolute bound $N(g,h)$ as claimed.
 \end{proof}

A small variation of this construction, essentially the case $g=h$ and $d=1$, parametrizes families of
isomorphisms of curves: given two families of projective curves $\cX \subset B \times \PP^r_{S}$ and $\mathcal{Y} \subset B\times \PP^s_{S}$, the result is a scheme $Isom(\cX, \mathcal{Y}) \to S$ whose
fiber over a point of $S$ is the set of isomorphisms of $X_{b}\to Y_{b}$ for $b$ in the fiber over $s$.
This turns out to be useful in describing the properties of the moduli space $M_{g}$ treated in the
next chapter.


%\subsection{Subschemes of a given scheme}
%
%It is easy to see that if $X\subset \PP^n$,
%then the family of subschemes $Y$ of $X$ with given Hilbert polynomial $p$ is a closed subscheme of $Hilb_p(\PP^n)$: we simply
%add the condition that the vector space of forms of high degree defining $Y$  contain the vector space defining $X$---this is also a determinantal condition---and denote the resulting scheme $Hilb_p(X)$.
%
%The proof of Theorem~\ref{tangent space of Hilb} is independent of the nature of the ambient scheme, and shows that if $Y \subset X$ is any subscheme, it has a normal sheaf $\cN_{Y/X}= \Hom_{X}(\sI_{Y/X}, \sO_{X}/\sI_{Y/X})$, and the tangent space to $Hilb_p(X)$ at $[Y]$ is  $H^0(\cN_{Y/X})$.
%
%\subsection{Nested pairs} A variant of the Hilbert scheme construction can be used to parametrize \emph{nested pairs} of subschemes of $\PP^r$; that is, given a pair of polynomials $p(m)$ and $q(m)$ we can consider pairs $(X,Y)$ with $X \subset Y \subset \PP^r$, with $X$ having Hilbert polynomial $p$ and $Y$ having Hilbert polynomial $q$. To construct it we find a suitably large $m$, and associate to any such pair $X \subset Y \subset \PP^r$ the point
%$$
%H^0(\cI_Y(m)) \subset H^0(\cI_X(m)) \subset H^0(\cO_{\PP^r}(m))
%$$
%in the flag manifold $F(\binom{r+m}{r} - q(m), \binom{r+m}{r} - p(m), \binom{r+m}{r})$.
%
%For example, we could take $p=1$; the resulting space would parametrize pointed subschemes $(y,Y)$ with $Y \subset \PP^r$ having Hilbert polynomial $q$ and $y \in Y$. Indeed, this is the same scheme as the universal family over $Hilb_p(\PP^n)$.
%
%We can also identify the tangent spaces to Hilbert schemes of nested pairs, at least when the schemes involved are reasonable: if $X \subset Y \subset \PP^r$ is any nested pair of smooth subschemes of $\PP^r$, we have morphisms of normal sheaves
%$$
%\begin{diagram}
%\cN_{X/\PP^r} & \rTo^\alpha & \cN_{Y/\PP^r}|_X \\
%& & \uTo^\beta \\
%& & \cN_{Y/\PP^r}
%\end{diagram}
%$$
%and the tangent space to the Hilbert scheme of nested pairs at the point $[X,Y]$ is the space of pairs $(\sigma, \tau) \in H^0(\cN_{X/\PP^r}) \times H^0(\cN_{Y/\PP^r})$ with $\alpha(\sigma) = \beta(\tau)$. See Lemma 6.23 of~\cite{3264} for details.
%
%\subsection{Relative Hilbert schemes}\label{Relative Hilbert schemes} Suppose now that $\sY \to B$ is a flat family of projective schemes over some base $B$; that is, $\sY$ is a subscheme of $B \times \PP^r$ flat over $B$. For any Hilbert polynomial $p(m)$, we can construct the relative Hilbert scheme $Hilb_p(\sY/B)$ parametrizing pairs $(b, X)$ with $b \in B$ and $X \subset Y_b$ a subscheme of the fiber $Y_b$ having Hilbert polynomial $p$. In a sense, the last construction is a special case of this: if we take the family $\sY \to B$ to be the universal family over $B = Hilb_q(\PP^r)$, we get the Hilbert scheme of nested pairs.
%
%Relative Hilbert schemes can behave in unexpected ways. For example, suppose $B$ is a smooth curve and $\sY \to B$ a family of smooth curves over $B \setminus \{b_0\}$ specializing to a nodal curve $Y_{b_0}$. We've seen that for $b \neq b_0$, the Hilbert scheme of subschemes of $Y_b$ of dimension 0 and degree $d$ is the $d$th symmetric power $Y_b^{(d)}$, and we might want to complete this to a family proper over all of $B$. Taking the relative relative Hilbert scheme $Hilb_d(\sY/B)$ yields a very different result than taking the relative $d$th symmetric power; see~\cite{MR2172162} for an example.
%
%%\fix{I'm thinking you might want to axe the preceding paragraph; I won't object, but let's talk}
%%\fix{I just made it a little less enthusiastic; I looked up Ran's paper, and it didn't seem to me quite up to the hype. -- DE}
%
%\subsection{Morphisms}\label{Hilbert morphisms} Suppose we want to parametrize morphisms between two given projective schemes $X$ and $Y$. We can embed the product $X \times Y$ in a larger projective space $\PP^N$, and then consider the Hilbert scheme $\cH(X \times Y)$ parametrizing subschemes of the product $X \times Y$. Among such subschemes are the graphs of maps $f: X \to Y$; since a map is determined by its graph, this expresses the family of morphisms from $X$ to $Y$ as a locally closed subscheme $\cH(X \to Y) \subset \cH(X \times Y)$.
%
%%\fix{the observing the two simplys in the next sentence puts me off: its a worthwhile statement, so either say why or make it an unproved assertion.}
%We can also describe the tangent space to $\cH(X \to Y)$ at a point $f : X \to Y$ by observing that the normal bundle of the graph $\Gamma_f \subset X \times Y$ is simply the pullback $f^*T_Y$ of the tangent bundle of $Y$; thus the tangent space to $\cH(X \to Y)$ at $f$ is simply $H^0(f^*T_Y)$.
%
%\subsection{The Isom functor}\label{Isom}
%Finally, we can combine the last two constructions
%%fix{I didn't see a construction of the relative Hilbert scheme at all---just the description of a set}
% to construct what is called the Isom scheme. The situation is this: we are given families of projective varieties $\cX \subset B \times \PP^r$ and $\mathcal{Y} \subset B\times \PP^s$, and we first parametrize the set of morphisms between members of these two families; that is, the set
%$$
%\{ (b, f) \mid b \in B \text{ and } f: X_b \to Y_b \}
%$$
%The techniques of the last two sections allow us to do exactly that; and we can further restrict to the locus where $f$ is an isomorphism to arrive at a scheme, called the Isom scheme, parametrizing isomorphisms between corresponding fibers of $\cX$ and $\mathcal Y$.
%
%If we are given families of projective varieties $\cX \subset B \times \PP^r$ and $\mathcal{Y} \subset B' \times \PP^s$---not necessarily over the same base---we can parametrize the set of all morphisms $f : X_b \to Y_{b'}$ between members of these two families over any points $b \in B$ and $b' \in B'$. We simply apply the preceding construction to the products
%$$
%\sZ := \cX \times B' \quad \text{and} \quad \sW : = \sY \times B,
%$$
%both viewed as families over the product $B \times B'$; the corresponding Isom scheme parametrizes all triples $(b, b', f)$ with $f : X_b \to Y_{b'}$.
%
%\subsection{The Quot scheme} Subschemes of $\PP^{r}$ correspond to the quotients of $\sO_{\PP^{r}}$. Why not quotients of some other coherent sheaf? Indeed, the original construction by Grothendieck starts with a coherent sheaf $\sE$
%on a relative scheme $X/S$, and forms the functor of families of surjections $\sE \to \sF$ such that $\sF$ is flat over $X$. See for example~\cite{MR2222646}.
% For variants involving Hilbert functions instead of Hilbert polynomials, and other gradings, see~\cite{MR2073194}.
% 
% \subsection{An application}
%  
% \begin{proposition}\label{finite morphisms}
% If $C$ and $D$ are smooth, projective curves of genera $g, h \geq 2$, then there are only finitely many nonconstant maps $f : C \to D$.
% \end{proposition}
% 
% \begin{proof}
% Hurwitz' Theorem~\ref{Hurwitz} implies a bound on the degree $d$ of a morphism $f : C \to D$ so we may assume that $d$ is fixed.
% 
%We embed the product $C \times D$ in projective space $\PP^N$ by sections of the invertible sheaf $\pi_1^* \cL \otimes \pi_2^* \cM$, where $L$ and $M$ are very ample invertible sheaves on $C$ and $D$ of degrees $l$ and $m$ respectively. If $\Gamma \subset C \times D \subset \PP^N$ is the graph of a morphism $f : C \to D$ of degree $d$, then $\Gamma$ has genus $g$ and degree $l + dm$, so we know its Hilbert polynomial $p$; thus the Hilbert scheme $\cH(C \to D)$ is a scheme of finite type.
% 
% It now suffices to show that $\cH(C \to D)$ is zero-dimensional. Tangent space to the Hilbert scheme at a morphism $f$ is
% $$
% T_{[f]}\cH(C \to D) = H^0(f^*T_D).
% $$
% Since $f^*T_D$ is an invertible sheaf of negative degree, it cannot have nonzero global sections. 
% \end{proof}
% 
% 
%\fix{The following seems to me far too coy. It repeats how you got interested, but that's not the reader's problem! I suggest making one Theorem
%saying that given $g,h$ there is a uniform bound.}
%
% Proposition~\ref{finite morphisms} raises an interesting question: given that the number of morphisms $f : C \to D$ is finite, is it uniformly bounded? In other words, we ask whether there exists a number $N = N(g,h)$ such that for \emph{any} smooth projective curves of genera $g$ and $h$,
% $$
%\# (\cH(C \to D)) \leq N.
% $$
%Indeed, we can prove that there does exist a uniform upper bound $N(g,h)$ by combining the above with the construction of Section~\ref{Isom}. To do this, we start with the Hilbert schemes $\cH_1 = \cH_{(2g+1)m -g + 1}(\PP^{g+1})$ and $\cH_2 = \cH_{(2h+1)m -h + 1}(\PP^{h+1})$ parametrizing curves of degree $2g+1$ in $\PP^{g+1}$ and curves of degree $2h+1$ in $\PP^{h+1}$, or rather the locally closed subset of each parametrizing smooth, irreducible curves; let $\cX \to \cH_1$ and $\sY \to \cH_2$ be the corresponding universal families. We then apply the Isom functor to these two families and argue as before.
%
%At present, we have no idea what $M(g,h) := \max_{C, D}\#\cH(C \to D)$ might be.

\section{Exercises}


\begin{exercise}\label{deg of disjoint union}
Suppose that a scheme $X\subset \PP^n$ is the disjoint union of subschemes $Y,Z$. Show that the Hilbert polynomial of
$X$ is the sum of the Hilbert polynomials of $Y$ and $Z$. What statement can you make about the Hilbert functions?

Hint: the Hilbert polynomials satisfy $p_X = p_Y + p_Z$, which follows from the vanishing of $h^1(\cI_{Y\cup Z} (m))$ for large $m$; the Hilbert functions satisfy $h_X \leq h_Y + h_Z$. (When $h_X(m) = h_Y(m) + h_Z(m)$, we say that $Y$ and $Z$ \emph{impose independent conditions} on $|\cO_{\PP^n}(m)|$.)
\end{exercise}

\begin{exercise}
More generally, suppose that a scheme $X\subset \PP^n$ is the union of subschemes $Y,Z$. Show that the Hilbert polynomial of
$X$ is the sum of the Hilbert polynomials of $Y$ and $Z$ minus the Hilbert polynomial of $Y\cap Z$. 

Hint: use the exact sequence
$$
0 \to \cI_{Y\cup Z} \to \cI_{Y} \oplus \cI_{Z} \to \cI_{Y \cap Z} \to 0
$$
\end{exercise}

\begin{exercise}
Let $H \subset \PP^3$ be a 2-plane; let $C \subset H$ be a plane cubic curve and $p \in H \setminus C$ and point in $H$ not on $C$; let $X = C \cup \{p\}$.
\begin{enumerate}
\item Show that the Hilbert polynomial of $X$ is $p_X(m) = 3m+1$.
\item Show that the smallest value of $m_0$ satisfying the statement of Lemma~\ref{matsusaka} is 4.

Hint: any cubic vanishing on $X$ vanishes identically on $H$.
\end{enumerate}
\end{exercise}

\begin{exercise}\label{rational normal hilbert}
Use an  argument like that of Proposition~\ref{hilb of twisted cubics} to show that the restricted Hilbert scheme $\cH^\circ$ of rational normal curves $C \subset \PP^r$ is irreducible of dimension $r^2+2r-3$.

Hint: As in the twisted cubic case, the group $PGL_{r+1}$ acts transitively on $\cH^\circ$ with stabilizer $PGL_2$
\end{exercise}

\begin{exercise}\label{hilb at a ci}
If $C = X\cap Y\subset \PP^3$ is a complete intersection of surfaces of degrees $d,e$, then
$Hilb$ is smooth at the point $[C]$, of dimension $2\binom{3+d}{3}-4$ if $d=e$
or $\binom{3+d}{3} +\binom{3+e}{3} -\binom{3+e-d}{3} -2$ if $d<e$.

Hint: The normal bundle is $\sN = \sO_C(d)+\sO_C(e)$. To prove smoothness, use
Exercise~\ref{ci is acm} to compute $H^0(\sN)$.
\end{exercise}

%\begin{section} possible exercises:
%\begin{enumerate}
%\item Iarrobino: large component
%\item hilb of quartics in P3
%\item hilb of all rational curves in P3
%\item hilb of pairs: define; prove "line on quadric" is irreducible.
%\item hilb for linear space is the Grassmannian
%\item describe an analogue of the construction of hilb for the case of nested pairs of schemes
%\item in a hilb \times hilb, show that the locus of pairs of schemes that intersect in closed. (Hint intersect universal families, push forward)
%\end{enumerate}
%\end{section}
%footer for separate chapter files

\ifx\whole\undefined
%\makeatletter\def\@biblabel#1{#1]}\makeatother
\makeatletter \def\@biblabel#1{\ignorespaces} \makeatother
\bibliographystyle{msribib}
\bibliography{slag}

%%%% EXPLANATIONS:

% f and n
% some authors have all works collected at the end

\begingroup
%\catcode`\^\active
%if ^ is followed by 
% 1:  print f, gobble the following ^ and the next character
% 0:  print n, gobble the following ^
% any other letter: normal subscript
%\makeatletter
%\def^#1{\ifx1#1f\expandafter\@gobbletwo\else
%        \ifx0#1n\expandafter\expandafter\expandafter\@gobble
%        \else\sp{#1}\fi\fi}
%\makeatother
\let\moreadhoc\relax
\def\indexintro{%An author's cited works appear at the end of the
%author's entry; for conventions
%see the List of Citations on page~\pageref{loc}.  
%\smallbreak\noindent
%The letter `f' after a page number indicates a figure, `n' a footnote.
}
\printindex[gen]
\endgroup % end of \catcode
%requires makeindex
\end{document}
\else
\fi


