%header and footer for separate chapter files

\ifx\whole\undefined
\documentclass[12pt, leqno]{book}
\usepackage{graphicx}
\usepackage{eps-to-pdf}
\input style-for-curves.sty
%\input sl-macros.sty
\usepackage{hyperref}
\usepackage{showkeys} %This shows the labels.
\usepackage{msribib}
\usepackage{pdfpages}
\usepackage{draftwatermark}
\SetWatermarkText{DRAFT:\ \today}
\SetWatermarkScale{2}
\SetWatermarkColor[gray]{0.9}

%\usepackage{SLAG,msribib,local}
%\usepackage{amsmath,amscd,amsthm,amssymb,amsxtra,latexsym,epsfig,epic,graphics}
%\usepackage[matrix,arrow,curve]{xy}
%\usepackage{graphicx}
%\usepackage{diagrams}
%
%%\usepackage{amsrefs}
%%%%%%%%%%%%%%%%%%%%%%%%%%%%%%%%%%%%%%%%%%
%%\textwidth16cm
%%\textheight20cm
%%\topmargin-2cm
%\oddsidemargin.8cm
%\evensidemargin1cm
%
%%%%%%Definitions
%\input preamble.tex
%\input style-for-curves.sty
%\def\TU{{\bf U}}
%\def\AA{{\mathbb A}}
%\def\BB{{\mathbb B}}
%\def\CC{{\mathbb C}}
%\def\QQ{{\mathbb Q}}
%\def\RR{{\mathbb R}}
%\def\facet{{\bf facet}}
%\def\image{{\rm image}}
%\def\cE{{\cal E}}
%\def\cF{{\cal F}}
%\def\cG{{\cal G}}
%\def\cH{{\cal H}}
%\def\cHom{{{\cal H}om}}
%\def\h{{\rm h}}
% \def\bs{{Boij-S\"oderberg{} }}
%
%\makeatletter
%\def\Ddots{\mathinner{\mkern1mu\raise\p@
%\vbox{\kern7\p@\hbox{.}}\mkern2mu
%\raise4\p@\hbox{.}\mkern2mu\raise7\p@\hbox{.}\mkern1mu}}
%\makeatother

%%
%\pagestyle{myheadings}

%\input style-for-curves.tex
%\documentclass{cambridge7A}
%\usepackage{hatcher_revised} 
%\usepackage{3264}
   
\errorcontextlines=1000
%\usepackage{makeidx}
\let\see\relax
\usepackage{makeidx}
\makeindex
% \index{word} in the doc; \index{variety!algebraic} gives variety, algebraic
% PUT a % after each \index{***}

\overfullrule=5pt
\catcode`\@\active
\def@{\mskip1.5mu} %produce a small space in math with an @

\title{A Chapter from ``The Practice of Algebraic Curves"}
\author{\copyright David Eisenbud and Joe Harris}
%%\includeonly{%
%0-intro,01-ChowRingDogma,02-FirstExamples,03-Grassmannians,04-GeneralGrassmannians
%,05-VectorBundlesAndChernClasses,06-LinesOnHypersurfaces,07-SingularElementsOfLinearSeries,
%08-ParameterSpaces,
%bib
%}

\date{\today}
%%\date{}
%\title{Curves}
%%{\normalsize ***Preliminary Version***}} 
%\author{David Eisenbud and Joe Harris }
%
%\begin{document}

\begin{document}
\maketitle

\pagenumbering{roman}
\setcounter{page}{5}
%\begin{5}
%\end{5}
\pagenumbering{arabic}
\tableofcontents
\fi


\chapter{Brill--Noether theory and applications to genus 6}\label{Brill--Noether}\label{BNChapter}

\section{What linear series exist?}

Let's start with a naive question: when does there exist a curve $C$ of genus $g$ and a $g^r_d$ on $C$\emdash equivalently, a line bundle $\cL$ of degree $d$ on $C$ with $h^0(\cL) \geq r+1$? The Riemann--Roch and Clifford theorems together provide a complete answer to this question:

\begin{theorem}\label{arbitrary linear series}
There exists a curve $C$ of genus $g$ and a line bundle $\cL$ of degree $d$ on $C$ with $h^0(\cL) \geq r+1$ if and only if
$$
r \leq
\begin{cases}
d-g, \quad \text{if } d \geq 2g-1; \text{ and} \\
d/2,  \quad \text{if } 0 \leq d \leq 2g-2.
\end{cases}
$$
\end{theorem}


For the\emdash perhaps more interesting\emdash question of when  there exists a curve of genus $g$ with a birationally very ample $g^r_d$, Castelnuovo's theorem gives a quadratic bound, roughly $d \geq \sqrt{g(2r-2)}$.

In both these situations, the curves that achieve the bounds are quite special. Perhaps the most interesting question of all is, for which $r,d$ do \emph{all} curves of genus $g$ have a $g^r_d$, and what is the
behavior of these series on a general curve? Brill--Noether theory provides some answers to both these questions.

\section{Brill--Noether theory}

The following result was stated by Brill and Noether in 1874, and finally proven in a series of works by
\cite{Kempf}, \cite{MR323792}, \cite{MR0357398}, \cite{Kleiman-special} culminating in a paper by
Griffiths and the second author~\cite{Griffiths-Harris-BN}.

\begin{theorem}[Basic Brill Noether]\label{basic BN}
If $r\geq 0$ and
 $$
 \rho(g,r,d) := g - (r+1)(g-d+r) \geq 0,
$$
then every smooth projective curve of genus $g$  possesses a $g^r_d$. Conversely, if $\rho < 0$ then a general curve $C$ of genus $g$ does not possess a $g^r_d$.
\end{theorem}

%\begin{theorem}[Basic Brill Noether]\label{basic BN}
%A general curve $C$ of genus $g$  possesses a linear series of degree $d$ and dimension $r>d-g$ if and only if
%$$
% \rho(g,r,d) := g - (r+1)(g-d+r) \geq 0.
%$$
%\end{theorem}

It is interesting to compare the values of $d,r$ that are possible on special and general curves; see Figure~\ref{Clifford-Castelnuovo-BrillNoether comparison}.

\begin{figure}
\inprogress     
\centerline {\includegraphics[height=3in]{"main/Fig11-1-Clifford-Castelnuovo-Brill-Noether"}}
\caption{For smooth curves of genus 100, 
these are bounds on $(d,r)$ for all linear series (Clifford), 
birationally very ample series (Castelnuovo), and all linear series
on general curves (Brill--Noether). }
\label{Clifford-Castelnuovo-BrillNoether comparison}
\end{figure}
%\fix{vert axis should be labeled $r$, horizontal labeled $d$. Color might
%be good to make the regions clearer.}

Gathering the inequalities, and putting them all in terms of lower bounds on $d$ given $g, r$,
we get \goodbreak
%$$
\begin{align*}
 d &\geq \min\{r+g, 2r\} \hbox{ by the Riemann--Roch and Clifford theorems}\\
 d &\geq \sqrt{(2r-2)g} \hbox{ by an approximation to the Castelnuovo theorem}\\
 d &\geq r+g-\frac{g}{r+1} \hbox{ for a general curve.}
\end{align*}
%$$

In the following sections, we'll explain the heuristic argument that led Brill and Noether to the statement of Theorem~\ref{basic BN} and discuss some refinements.   In Chapter~\ref{InflectionsChapter} we'll give a proof based on the study
of inflections and on families of Jacobians.% \fix{char 0?}.

The case $r=1$ is already interesting:

\begin{corollary}
If $C$ is any curve of genus $g$, then $C$ admits a map  to $\PP^1$ of degree $d$ for some $d \leq \lceil \frac{g+2}{2}\rceil$.
\end{corollary}

Thus any curve of genus 2 is hyperelliptic, any curve of genus 3 or 4 is either hyperelliptic or trigonal  (admits a 3-1 map to $\PP^1$), and so on. We have already verified this assertion in genus $g \leq 5$ by analyzing the geometry of the canonical map; for higher genera, though, this is not feasible.

Note also that this is exactly the converse to Corollary~\ref{branched cover BN} of Chapter~\ref{ModuliChapter}.


\subsection{A Brill--Noether inequality}\label{BN by divisors}

The proof of the Brill--Noether theorem starts with a dimension estimate that was first carried out by Brill and Noether in 1874 \cite{Brill-NoetherOriginal}. The estimate provides an inequality on the dimension
of the variety $W^r_d$, and the assertion of the theorem is that this is sharp for a general curve.

%From Kleiman-Laksov:  For r= 1, the matter is treated in section 4 of Riemann's " Theorie der Abel'schen Functionen" [11] (1857) and in lecture 31 of Hensel-Landsberg(1902) 1; the general case is treated in Brill--Noether [1](1874) and in lecture 57 and appendix G of Severi [13].(1921)

Let $C$ be a smooth projective curve of genus $g$, and $D = p_1 + \dots + p_d$ a divisor on $C$. Assume for simplicity that  the points $p_i$ are distinct; the same argument  can be carried out in general, but requires more complicated notation.

When does the divisor $D$ move in an $r$-dimensional linear series? By the Riemann--Roch theorem $h^0(D) \geq r+1$ if and only if the vector space $H^0(K-D)$ of 1-forms vanishing on $D$ has dimension at least $g-d+r$\emdash that is, if and only if the  evaluation map
$$
H^0(K) \to H^0(K|_D) = \bigoplus k_{p_i}
$$
has rank at most $d-r$. 

We can represent this map by a $g \times d$ matrix. Choose a basis $\omega_1,\dots,\omega_g$ for the space $H^0(K)$ of 1-forms on $C$; choose an analytic open neighborhood $U_j$ of each point $p_j \in D$ and choose a local coordinate $z_j$ in $U_j$ around each point $p_j$, and write
$$
\omega_i = f_{i,j}(z_j)dz_j
$$
in $U_j$. We  have $r(D) \geq r$ if and only if the  matrix-valued function
$$
A(z_1,\dots,z_d) = 
\begin{pmatrix}
f_{1,1}(z_1) & f_{2,1}(z_1) & \dots & f_{g,1}(z_1) \\
f_{1,2}(z_2) & f_{2,2}(z_2) & \dots & f_{g,2}(z_2) \\
\vdots & \vdots &  & \vdots \\
f_{1,d}(z_d) & f_{2,d}(z_d) & \dots & f_{g,d} (z_d)
\end{pmatrix}
$$
has rank $d-r$ or less at $(z_1,\dots,z_d) = (0,\dots,0)$.


In the space $M_{d,g}$ of $d \times g$ matrices, the subset of matrices of rank $d-r$ or less has codimension $r(g-d+r)$ (\cite[Exercise 10.9]{Eisenbud1995}. 
It follows 
that if  an effective divisor $D$ of degree $d$ with $h^0(D) \geq r+1$ exists, then in a neighborhood of the point $D \in C_d$ the locus $C^r_d$ of such divisors must have dimension at least $d - r(g-d+r)$, with equality if the map $A$ is dimensionally transverse to the locus in $M_{d,g}$ of matrices of rank at most $d-r$. Since a general fiber of the map $\mu : C^r_d \to W^r_d(C)$ has dimension $r$, it follows that 
$$
\dim W^r_d(C) \geq d - r(g-d+r) - r = g - (r+1)(g-d+r)
$$
and this is exactly the  Brill--Noether (in)equality. We will give a proof of the Brill--Noether theorem in Chapter~\ref{BrillNoetherproofChapter}.


\subsection{Refinements of the Brill--Noether theorem}

Theorem~\ref{basic BN} suggests a slew of questions, both about the geometry of the schemes $W^r_d(C)$ parametrizing linear series on a general curve $C$ (are they irreducible? what are their singular loci,\dots), and about the geometry of the linear systems themselves (do they give embeddings? what's the Hilbert function of the image? \dots). This is an active area of research. Here is some of what is currently known, starting with results about the geometry of $W^r_d(C)$:

\begin{theorem}\label{Wrd omnibus}
Let $C$ be a general curve of genus $g$. If we set $\rho = g - (r+1)(g-d+r)$, then for $d \leq g+r$,
\begin{enumerate}

\item $\dim(W^r_d(C)) = \rho$ (\cite{Griffiths-Harris-BN});\label{GH}

\item\label{sing wrd} the singular locus of $W^r_d(C)$ is exactly $W^{r+1}_d(C)$
(\cite{Gieseker-Petri}, \cite{Lazarsfeld-Petri};
\label{irr wrd} 

\item if $\rho > 0$ then $W^r_d(C)$ is irreducible (\cite{MR611386});

\item\label{rho=0} if $\rho = 0$ then $W^r_d(C)$ consists of a finite set of  points of cardinality
$$
\#W^r_d(C) = g! \prod_{\alpha=0}^r \frac{\alpha!}{(g-d+r+\alpha)!}
$$
and the monodromy of the generically finite covering of  $M_g$ by the universal family
$\cW^r_d$ of $W^r_d$s is transitive.
(\cite{zbMATH04014883}).

\item\label{Petri} if  $\sL$ is an invertible sheaf on $C$, then the multiplication map
$$
m : H^0(L) \otimes H^0(\omega_C\otimes L^{-1}) \rTo H^0(\omega_C)
$$
is injective, and the Zariski tangent space to the scheme $W^r_d(C)$ at the point $L$, as a subspace
of the tangent space $T_L\pic_d(C) = H^0(\omega_C)^*` `$, is the annihilator of the image of $m$
or, equivalently, the kernel of the dual of $m$ (\cite{Gieseker-Petri}).
\end{enumerate}
\end{theorem}

\begin{corollary}\label{2L nonspecial}
If $C$ is a general curve and $\sL$ is a general point of $W^r_d(C)$ with $r\geq 2$,
 then $\sL^m$ is nonspecial for all $m \geq 2$.
\end{corollary}

\begin{proof}
If $\sL^m$ were special\emdash that is, if $\omega_C\otimes \sL^{-m} = E$ were effective\emdash then we would have an inclusion $H^0(\sL) = H^0(\omega_C\otimes \sL^{-m+1}(-E)) \hookrightarrow H^0(\omega_C\otimes \sL^{-m+1})$. By Part~\ref{Petri} of Theorem~\ref{Wrd omnibus}, the map 
 $$
m : H^0(\sL^{m-1}) \otimes H^0(\omega_C\otimes \sL^{-m+1}) \rTo H^0(\omega_C)
$$
is injective, so the map
$$
H^0(\sL^{m-1}) \otimes H^0(\sL) \subset H^0(\sL^{m-1}) \otimes H^0(\omega_C\otimes \sL^{-m+1})
$$
obtained by restriction would likewise be injective.
However if $\sigma, \tau \in H^0(\sL)$ are two linearly independent sections, then $\sigma^{m-1} \otimes \tau - \sigma^{m-2}\tau \otimes \sigma$ lies in the kernel, contradicting the specialness of $\sL^m` `$.
\end{proof}

\begin{remark}

\begin{enumerate}
 \item As a special case of Part~\ref{rho=0} of the Theorem we see that the number of $g^{1}_{d}$s
 in the case $\rho=0$, that is, $g=2d-2$, is the Catalan number $C_{d-1}:= \frac{1}{d}\binom{2d}{d}$.

We have already seen this in the first two cases: in genus 2, it says the canonical series $|K|$ is the unique $g^1_2$ on a curve of genus 2, and in the case of genus 4 we have already seen  that there are exactly two $g^1_3$s on a general curve of genus 4. In genus 6, it says that a general curve of genus 6 has 5 $g^1_4$s; we'll describe these in Section~\ref{general genus 6} below.  

\item Part~\ref{Petri} and Part~\ref{GH} imply Part~\ref{sing wrd}. \cite[Section IV.4]{ACGH} shows that at a point $\sL  \in W^r_d(C) \setminus W^{r+1}_d(C)$, the tangent space to $W^r_d$ at the point $\sL $ is the annihilator
in $(H^0(\omega_C))^*$ of the image of $\mu$; given that $\mu$ is injective, we can compare dimensions and deduce that $W^r_d$ is smooth at $\sL $.

\item For any curve $C$, there exists a scheme $G^r_d(C)$ parametrizing linear series of degree $d$ and dimension $r$; that is, in set-theoretic terms,
$$
G^r_d(C) = \left\{ (\sL , V) \mid \sL  \in Pic_d(C), \text{ and } V \subset H^0(\sL ) \text{ with } \dim V = r+1 \right\}.
$$
$G^r_d(C)$ maps to $W^r_d(C)$; the map is an isomorphism over the open subset $W^r_d(C) \setminus W^{r+1}_d(C)$ and has positive-dimensional fibers over $W^{r+1}_d(C)$. It was conjectured
by Petri and proven in \cite{Gieseker-Petri} that for a general curve the scheme $G^r_d(C)$ is smooth for any $d$ and $r$.
\end{enumerate}
\end{remark}


Recall that  in theorems~\ref{g+1 theorem}, \ref{g+2 theorem}, \ref{g+3 theorem} we proved that
general invertible sheaves of degrees $g+1$, $g+2$ and $g+3$ on any curve
give the nicest possible maps to (respectively) $\PP^1, \PP^2$ and $\PP^3.$ These
linear series, being general of degree $\geq g$, are  nonspecial and have respectively
2, 3, or 4-dimensional spaces of sections. The following result shows that something
similar is true on a general curve for general linear series with 2,3, or 4-dimensional
spaces of sections, though they may have degrees much less than $g+1, g+2, g+3$:

\begin{theorem}\label{grd omnibus}(\cite[Proposition 5.4]{Eisenbud-Harris83}
Let $C$ be a general curve of genus $g$, and suppose that
$|D|$ is a general $g^r_d$ on $C$.

 \begin{enumerate}
\item If $r \geq 3$ then $D$ is very ample; that is, the map $\phi_D : C \to \PP^r$   embeds $C$ in $\PP^r` `$;
\item If $r=2$ the map $\phi_D : C \to \PP^2$ gives a birational embedding of $C$ as a nodal plane curve; and 
\item If $r=1$, the map $\phi_D : C \to \PP^1$ expresses $C$ as a simply branched cover of $\PP^1` `$.
\end{enumerate}
\end{theorem}

In case $\rho = 0$\emdash so that there are a finite number of $g^r_d$s on a general curve $C$\emdash these statements hold for \emph{all} the $g^r_d$s on $C$.

In the course of investigating embeddings of a curve $C\subset \PP^n$ we have again and again
asked about the ranks of the maps $H^0(\sO_{\PP^n}(d)) \to H^0(\sO_C(d))$. In the case of
a general curve, the following theorem of \cite{Larson} gives a comprehensive answer; in particular, it gives
 the Hilbert function of any general embedding:
 
\begin{theorem}[E. Larson](Maximal Rank theorem)\label{maximal rank}
If $C$ is a general curve of genus $g$ and $\sL  \in W^r_d(C)$ is a general point, then for each $m > 0$ the multiplication map
$$
\rho_m : \Sym^m H^0(\sL ) \to H^0(\sL ^m)
$$
has maximal rank; that is, it is injective if $\binom{m+r}{r} \leq h^0(\sL ^m)$ and surjective if $\binom{m+r}{r} \geq h^0(\sL ^m)$.
\end{theorem}


If $\sL\in W^r_d(C)$ is a general point, then Corollary~\ref{2L nonspecial} shows that 
$h^0(\sL ^m) = md-g+1$ for all $m \geq 2$
and this allows us to compute the Hilbert function of a general embedding as a curve
of degree $d$ as 
 $$
 h_C(m) = \min\left(\binom{m+r}{r},\ md-g+1\right).
 $$
 
A key step in Larson's proof is an interpolation theorem~\cite{MR3908670}, later extended in~\cite{MR4653767} to show:

\begin{theorem}[\cite{MR4653767}]\label{Larson-Vogt}
Let $d, g$ and $r$
be nonnegative integers with $\rho(d, g, r) \geq 0$. There is a general curve of degree $d$ and genus $g$ through $n$ general
points in $\PP^r$
if and only if
$$
(r-1)n \leq (r + 1)d-(r-3)(g-1)
$$
except in the four cases $(d, g, r) = (5, 2, 3)$,
$(6, 4, 3)$, $(7, 2, 5)$ and $(10, 6, 5)$.
 \end{theorem}
 
There is a possible extension of the maximal rank theorem. If $C \subset \PP^r$ is a general curve embedded by a general linear series, the maximal rank theorem tells us the dimension of the $m$th graded piece of the ideal of $C$, for any $m$: this is just the dimension of the kernel of $\rho_m$. But it doesn't tell us the degrees of generators of the homogeneous ideal of $C$. For example, if $m_0$ is the smallest $m$ for which $I(C)_m \neq 0$, or numerically the smallest $m$ such that $\binom{m+r}{r} > md-g+1$, we can ask: is the homogeneous ideal $I(C)$ generated by $I(C)_{m_0}$? This can't always be the case, since 
there are examples where the  smallest nonzero graded piece of $I(C)$ has dimension 1. But one might conjecture that $I(C)$ is always be generated by its graded pieces of degrees $m$ and $m+1$; this is an open problem.

To answer this\emdash given that we know the dimensions of $I(C)_m$ for every $m$\emdash we would need to know the ranks of the multiplication maps
$$
\sigma_m : I(C)_m \otimes H^0(\cO_{\PP^r}(1)) \to I(C)_{m+1}
$$
for each $m$. In particular, we may conjecture that \emph{the maps $\sigma$ have maximal rank}; if this were true we could deduce the degrees of a minimal set of generators for the homogeneous ideal $I(C)$.

Another recent strand of work on Brill--Noether theory was developed in the thesis
\cite{HLarson} and in \cite{arXiv:2008.10765}, providing  analogues of many of the parts of the classical Brill--Noether theorem
for general curves of given gonality in the cases $\rho\geq 0$.


There are many remaining questions! One is the question of \emph{secant planes}: a naive dimension count would suggest that an irreducible, nondegenerate curve $C \subset \PP^r$ should have an $s$-secant $t$-plane if and only if $s(r-t-1) \leq (t+1)(r-t)$
(for example a curve $C \subset \PP^3$ has 4-secant lines, but no 5-secant lines). Is this true for a general curve embedded in $\PP^r$ by a general linear series?

\begin{exercise}
We saw in Chapter~\ref{JacobianChapter} that if $C$ is any curve of genus $g$ and $D$ a general divisor of degree $g+3$ on $C$, then $\phi_D : C \hookrightarrow \PP^3$ is an embedding. Using the Brill--Noether theorem, show that if $C$ is general then the image curve in $\PP^3$ has no 5-secant lines.
\end{exercise}

\section{Linear series on curves of genus 6}\label{genus 6 section}\label{general genus 6}
%\fix{This will come after the plane curves: we get the 5-ic del Pezzo for free, given the "conditions of adjunction", and then
%the special cases will give us exercise in what the conditions of adjunction are.}

We have seen in our analysis of curves of genus up to 5 that curves of the same genus can look quite different from the point of view of the linear series they possess: the existence of $g^r_d$s,  the geometry of the schemes $W^r_d(C)$ parametrizing them, and the geometry of the associated maps to projective space, can look quite different on different curves.

The variety of possible behaviors has increased modestly with the genus. Genus 6 is a tipping point: we could still enumerate all the possible behaviors of the schemes $W^r_d(C)$\emdash as distinguished by the number of components, dimension and singularities of the various schemes $W^r_d(C)$, and the geometry of the associated maps\emdash but it's quite a long list, and we will actually study just a few cases. For genus 7 and higher a full analysis has probably never been carried out. 

In lower genus we tacitly verified the statements of the Brill--Noether theorem from our descriptions of the canonical models. In genus 6, by contrast, we cannot  deduce the Brill--Noether theorem from studying the geometry of the canonical curve\emdash though we can easily see that a canonical curve $C \subset \PP^5$ of genus 6 lies on a 6-dimensional vector space of quadrics, that doesn't tell us much about its geometry.

Instead we will appeal directly to the Brill--Noether theorems. Here is a summary of what we will use:

\begin{theorem}\label{BN consequences}
Every smooth curve $C$ of genus 6 has at least one $g^{2}_{6}$. If $C$ is general, then
$W^{2}_{6}(C)$ and $W^{1}_{4}(C)$ each consist of 5 reduced points, while $W^{2}_{5} = W^{1}_{3} = \emptyset$.  Less formally, C has precisely 5 $g^{2}_{6}$s and 5 $g^{1}_{4}$s, but no $g^{2}_{5}$s and no $g^{1}_{3}$s. The image of the map associated to each $g^{2}_{6}$ is a nodal plane curve and its nodes are in linearly general position, that is, no three are collinear.
\end{theorem}

All these assertions except for the linear general position of the nodes follow immediately from 
Theorem~\ref{basic BN} and
Theorem~\ref{Wrd omnibus}; we will deduce the linear general position of the nodes from the relationship of the different
linear series on a general curve that are given by these theorems.

%\subsection{Linear series on general curves of genus 6}\label{general genus 6}
\subsection{General curves of genus 6}
\emph{We suppose for the rest of this section that $C$ is a general smooth curve of genus 6.}

By Theorem~\ref{BN consequences} we can map $C$ birationally to a plane sextic $C_0$ with only nodes as singularities. Since a plane sextic has arithmetic genus $\binom{6-1}{2} = 10$, the curve $C_{0}$
must have exactly 4 nodes.

Once we have exhibited one birational map of $C$ to a plane sextic with 4 nodes, we can describe all five $g^2_6$s and all five $g^1_4$s in terms of this plane model. For example, composing a $g^1_6$ corresponding to $f: C\to \PP^1$ with the projections from the 4 nodes gives four $g^1_4$s. To see the 5th $g^{1}_{4}$ we introduce some terminology:

Suppose $f : X \to S$ is a regular map from any smooth curve $X$ to a surface $S$. If $\sL $ is a line bundle on $S$ and $V \subset H^0(\sL )$ a vector space of sections, we can associate to them a linear system on $X$ by taking the pullback linear system $f^*V \subset H^0(f^*\sL )$ on $X$ and subtracting the basepoints; this is called the \emph{linear series cut out on $X$ by $V$}. 

To see the $g^{1}_{4}$s on $C$ in this way,  suppose again that $C$ is a general curve of genus 6 as above and $f : C \to \PP^2$ is a birational map onto a sextic curve $C_0$ with four nodes; let $p \in C_0$ be one of the nodes and consider the linear system $(\cO_{\PP^2}(1),V)$ of lines in $\PP^2$ through $p$. The pullback $f^*\cO_{\PP^2}(1)$ of course has degree 6, but the pullback linear series $f^*V$ has two basepoints, at the points $q, r \in C$ lying over $p$. The linear series cut on $C$ by $V$ is thus a $g^1_4$. 

To produce the fifth $g^1_4$, consider the linear series cut on $C$ by conic plane curves passing through all four nodes of $C_0$. There is a pencil of such conics, and the pullback $f^*\cO_{\PP^2}(2)$ has degree 12.
If the nodes are linearly independent then the pullback series has eight basepoints; thus we arrive at another $g^1_4$ on $C$.  Not all the nodes can be contained in a line, since then, by B\'ezout's theorem, the line would be a component of $C_0$. Thus if the nodes are linearly dependent, then exactly 3 lie on a line
so the linear series cut by the conics containing the nodes coincides with the projection from the 4th node. 
This would represent a nonreduced point of the scheme $W^1_4(C)$, the subject of Exercise~\ref{nonreduced Wrd} below. Thus the nodes are independent.

For another example, consider the linear system cut on $C$ by cubics passing through all four nodes. This has degree $3\cdot 6 - 8 = 10$ and dimension 6. It follows that this is the complete canonical series on $C$. (In Chapter~\ref{PlaneCurvesChapter} we will see directly that this is the case.)

Given the degree six map $f : C \to C_0 \subset \PP^2$ corresponding to one $g^2_6$ we can use the fact that the five $g^2_6$s on $C$ are residual to the five $g^1_4$s in the canonical series to construct the other four $g^2_6$s: they are cut out on $C$ by the linear system of plane conics passing through three of the four nodes of $C_0$. Equivalently, their images are the curves obtained from $C_{0}$ by the quadratic transformation
of $\PP^{2}$ centered at 3 of the 4 nodes, which blows up these 3 nodes and blows down the three lines
 joining them.
 
 In previous chapters we have seen that in genus $\leq 5$ a general canonical curve is  a complete intersection, but this fails for a canonical curve $C$ of genus 6. There is a 21-dimensional vector space of
quadratic forms on $\PP^5` `$, and $h^0(\sO_C(2)) = 2(2g-2)-g+1 = 15$, so $C$ lies on at least 6 quadrics, and we will show that its ideal sheaf is generated by exactly 6 quadrics. Since $6>\codim C$, the canonical curve of genus 6 is not a complete intersection. However, such curves lie on a quintic del Pezzo surface, which may be described as follows.


%There is a further consequence of this description: the four nodes of $C_0$ are in linear general position; that is, no three are collinear. 
%By parts~(\ref{rho=0}) and~\ref{Petri} of Theorem~\ref{Wrd omnibus}, $C$ must have 5 distinct $g^1_4$s, and if three of the four nodes of $C_0$ were collinear, the $g^1_4$ cut on $C$ by lines through the fourth node would coincide with the $g^1_4$ cut on $C$ by conics through all four.  
%
 
\begin{figure}
\centerline {\includegraphics[height=2in]{"main/Fig11-2"}}
\caption{A $g^1_4$ as the projection from a node of a plane sextic.
\marginparhere{Silvio: both lines of the projection should meet the sextic 4 times; the dashed line could be tangent. In general, perhaps our convention for showing a map by projection should be to show more than two projection lines (and note that they are all on the same footing).}
}
\end{figure}

\begin{figure}
\centerline {\includegraphics[height=2in]{"main/Fig11-3"}}
\caption{A $g^1_4$ as a pencil of hyperbolas through the four nodes of a plane sextic.}
\end{figure}
 
 
\begin{figure}
\centerline {\includegraphics[height=2in]{"main/Fig11-4"}}
\caption{A sextic with 4 nodes and the fundamental triangle of the quadratic transformation giving
a different $g^{2}_{6}$. 
\marginparhere{Silvio: nice picture. Might be better to use color for
  the three lines of the distinguished triangle. Is the sextic
  mathematically correct? If not, it might be clearer if the lines of
  the triangle weren't so tight to the sextic\emdash they are a little
  hard to distinguish.}
}
\end{figure}


\subsection{Del Pezzo surfaces}\label{Del Pezzo sketch}

We  met the del Pezzo surface of degree 5 in Section~\ref{Genus 1 quintics in P4}. 
We briefly sketch, without proofs, a little of the 
 rich classical theory of del Pezzo surfaces in general. The basics are well treated in \cite[pp. 45--50]{Beauville}; for more, see the
beautiful book \cite{Manin}, which also goes into some of the arithmetic theory. We will use only the case of the del Pezzo surface of degree 4, which lies in $\PP^{5}$

By definition,
a \emph{del Pezzo} surface is a smooth surface embedded in $\PP^n$  by its complete anticanonical series $-K_S$. These exist only for $3\leq n\leq 9$. The best-known example is a smooth cubic surface in $\PP^3` `$. That it is a del Pezzo surface follows from the adjunction formula.

A del Pezzo surface in $\PP^n$ has degree $n$, and is isomorphic to the blow-up of $\PP^2$ at $9-n$ points of which no 3 lie on a line and no 6 lie on a conic, embedded by the linear series on $\PP^2$
consisting of the cubics passing through the $9-n$ points\emdash except when $n=8$, in which case the
linear series of curves of type $(2,2)$ on $\PP^1\times \PP^1$ provides another example.

Comparing the linear series  of cubic forms containing $p_1,\dots,p_4$ with the linear series  of sextic forms vanishing to order 2 at $p_1,\dots,p_4$, we see that a quintic del Pezzo surface $S \subset \PP^5$ lies on at least $5$ quadrics. In fact, its homogeneous ideal is generated by exactly 5 quadrics.

A quintic del Pezzo surface $S \subset \PP^5$ contains exactly 10 lines, which (in terms of the description of $S$ as the blow up of $\PP^2$ at four points $p_1,\dots,p_4 \in \PP^2$) are the 4 exceptional divisors and the 6 proper transforms of the lines joining the $p_i$ pairwise. 
It is the intersection of a $\PP^5$ with the Grassmannian $G(2,5) \subset \PP^9` `$, and correspondingly the five quadrics containing $S$ can be realized as the Pfaffians of a  $5\times 5$ skew-symmetric matrix of linear forms on $\PP^5` `$.

\begin{figure}
\centerline {\includegraphics[height=1.6in]{"main/Fig11-5"}}
\caption{Dual graph of the configuration of 10 lines on a quintic del Pezzo surface, the plane blown up
at 4 points showing 4 pairwise disjoint exceptional divisors.}
\label{dual graph of the configuration of 10 lines on a quintic del Pezzo surface}
\end{figure}

There is also a notion of a \emph{weak del Pezzo} surface; this is a smooth surface whose anticanonical bundle is nef but not necessarily ample. We get such a surface if we blow up $\PP^2$ at a configuration of points of which three are collinear; in this circumstance the anticanonical bundle on the blow-up $S$ has degree 0 on the proper transform of the line containing the three points, and this proper transform is correspondingly collapsed to a rational double point of the image $\phi_{-K}(S)$. In general, the description of del Pezzo surfaces as blow-ups of the plane extends to the case of weak del Pezzos.



\subsection{The canonical image of a general curve of genus 6}

Using the Brill--Noether theorem, we have seen that a general curve $C$ of genus 6 is the normalization of a plane sextic $C_0$ with four nodes, and that the canonical series on $C$ is cut out by cubics in the plane passing through the four nodes. Thus the canonical model lies on the surface $S \subset \PP^5$ that is the image of the plane under the (rational) map given by cubics through these four points, which we now recognize as a quintic del Pezzo surface.

\begin{theorem}
A general canonical curve $C$ of genus 6 is the intersection of a quintic del Pezzo surface and a quadric. 
\end{theorem}

\begin{proof}
We have seen that $C \subset \PP^5$ lies on a quintic del Pezzo surface $S \subset \PP^5` `$. The surface is cut out by 5 quadrics, and we know that $C$ lies on 6 independent quadrics,
so $C$ is contained in the complete intersection of $S$ with a quadric. Since this scheme has degree 10, which is the degree of $C$, they are equal.
\end{proof}

For corresponding theorems for general curves with $g=7,8,9$ see \cite{Mukai1}, \cite{Mukai2}, and \cite{Mukai3}.


\section{Other curves of genus 6}

From Theorem~\ref{BN consequences} we know that every curve $C$ of genus 6 has  a $g^{2}_{6}$. For a general curve the corresponding morphism $\phi_{D}$ maps $C$ birationally onto a plane sextic with 4 nodes and exactly 5 $g^{1}_{4}$s. 
In this  section we will analyze some of the other curves of genus 6, starting with the question  ``what could go wrong?"
with the birational map to $\PP^2` `$. 

Since we already have a complete picture in the hyperelliptic case, we'll assume that $C$ is nonhyperelliptic. Clifford's theorem then rules out
the existence of a $g^3_6$ so  the $g^2_6$ is a complete linear series  $|D|$ for some divisor $D$ of degree 6.

Further analysis can be divided as follows:
\begin{enumerate}
\item $C$ is not trigonal. Then either
\begin{enumerate}
 \item $|D|$ has a basepoint; the canonical image of $C$ lies on the Veronese surface.
\item $\phi_{D}$ maps $C$ two to one onto a cubic $E \subset \PP^2$ of genus 1; the canonical image is the complete intersection of a quadric and the cone over the elliptic quintic $E\subset \PP^{4}$.
\item $\phi_{D}$ is birational onto a plane sextic with no triple point.
 \end{enumerate}
 \item $C$ is trigonal. The canonical image lies on a 
rational normal scroll $S(a,4-a)$; see Chapter~\ref{ScrollsChapter}.
In this case $|D|$ is base-point free and $\phi_{D}$  either maps $C$ three to one onto a conic
 or birationally onto a sextic with a triple point. 
\end{enumerate}

In the rest of this section we will examine cases 1a and 1b. Case 2 can be further divided by the value of $a$.
Case 1c may be divided into many parts according to the various configurations
of double points of the plane sextic, which may be nodes, cusps, tacnodes or higher-order double points,  corresponding to various possible schemes $W^{1}_{4}$. 

The analysis of the other cases is lengthy, but largely accessible with the tools we've introduced.


\subsection{$|D|$ has a basepoint}\label{g26 has a basepoint}
Clifford's theorem shows that a nonhyperelliptic curve of genus 6 cannot have a $g^2_4$, so   
$|D|$ has exactly one basepoint, and when we subtract the basepoint we get a base-point-free $g^2_5$. 

If $C \to C_0 \subset \PP^r$ is the map given by a base-point free $g^r_d$, the degree $d$ of the linear series is the degree of the image curve $C_0$ times the degree of the map $C \to C_0$. Since 5 is prime, the associated map $\phi_D : C \to \PP^2$ is birational onto a quintic curve. Moreover, since plane quintic curves have arithmetic genus 6, the image $\phi_D(C)$ is smooth; thus $C$ is isomorphic to a smooth plane quintic.

This allows us to describe the other special linear series on $C$. By the adjunction formula, the canonical series $|K_C|$ is cut on $C$ by conics in the plane. 

The plane $\PP^2$ is embedded by the complete linear series of quadrics as the Veronese surface in $\PP^5` `$, and since the canonical series of $C$ is cut out by quadrics, the canonical model of $C$
lies on this surface, and the canonical ideal is generated by the 6-dimensional family of quadrics containing the Veronese surface\emdash the $2\times 2$ minors of the generic symmetric $3\times 3$ matrix
corresponding to the multiplication map 
$$
H^{0}(\sO_{\PP^{2}}(1)) \otimes H^{0}(\sO_{\PP^{2}}(1)) \to H^{0}(\sO_{\PP^{2}}(2))
$$
as explained in Proposition~\ref{some equations}-- together with a 3-dimensional family of cubics,  the image of the 3 dimensional family of forms of degree 6 that are multiples of the quintic form defining $C$ in $\PP^2` `$.

As we showed in Chapter~\ref{3b}, the $g^1_4$s on $C$ are exactly the projections from points of $C$, so $W^1_4(C)\cong C$, and there are no 
$g^{1}_{3}$s.


\subsection{$C$ is not trigonal and the image of \texorpdfstring{$\phi_{D}$}{phi(D)} is two to one onto a  plane curve of degree 3.}

The cubic curve $E$ is smooth since otherwise it would have geometric genus 0 and $C$ would be  hyperelliptic. Thus $C$ is a double cover of a smooth curve of genus 1; we say that $C$ is \emph{bielliptic}.

In this case the canonical divisor class $K_C$ is the pullback of an invertible sheaf $\cO_E(F)$ for some divisor class of degree 5 on $E$. But it is not the case that the canonical series $|K_C|$ is the pullback of the linear series $|\cO_E(F)|$: by the Riemann--Roch theorem, the latter has dimension 4, rather than 5. Indeed, if we recall that the target of the canonical map $\phi_K : C \to \PP^5$ is the projective space $\PP H^0(K_C)$, there is be a point $X \in \PP^5$ corresponding to the hyperplane $\pi^*H^0(F) \hookrightarrow H^0(K_C)$, and projection of the canonical curve from this point maps $C$ 2-to-1 onto the image $\phi_F(E) \subset \PP^4` `$. In other words, the canonical model of $C$ lies on a cone $S = \overline{X, E}$ over an elliptic normal quintic curve $E \subset \PP^4` `$. 

As we saw in Chapter~\ref{3b}, the quintic curve $\phi_F(E) \subset \PP^4$ lies on 5 quadrics, as does the cone $S \subset \PP^5$ over it. Thus there is a quadric $Q \subset \PP^5$ containing $C$ but not containing $S$. B\'ezout's theorem shows that in this case, the canonical model of a bielliptic curve of genus 6 is the intersection of the cone over an elliptic quintic curve with a quadric.

\section{Exercises}

\begin{exercise}
Let $C$ be a bielliptic curve of genus 6; that is, a double cover of a  smooth projective curve $E$ of genus 1. 
\begin{enumerate}
\item Show that $C$ cannot be hyperelliptic (going forward, we will identify $C$ with its canonical image in $\PP^5$).
\item Let $F$ an invertible sheaf of degree 5 on $E$ and $\phi_F(E) \subset \PP^4$ the corresponding elliptic normal quintic curve. Show that $\phi_F(E)$ lies on 5 quadrics, as does the cone $S \subset \PP^5$ over it
\item Deduce that there is a quadric $Q \subset \PP^5$ containing $C$ but not containing $S$. Now invoke B\'ezout's theorem to deduce that in this case, the canonical model of a bielliptic curve of genus 6 is the intersection of the cone over an elliptic quintic curve with a quadric.
\end{enumerate}

Hint: For the second part, we know that $\phi_F(E)$ lies on at least 5 quadrics by the usual restriction sequence; if it lay on 6 or more it would be a rational normal curve. (Alternatively, see Section~\ref{g=1 in P4}.)
\end{exercise}


\begin{exercise}
Use the preceding exercise to show that if $C$ is a bielliptic curve of genus 6, a 2-sheeted cover of an elliptic curve $E$, then every $g^1_4$ on $C$ is the pullback of a $g^1_2$ on $E$, and likewise  every $g^2_6$ on $C$ is the pullback of a $g^2_3$ on $E$. Deduce that in this case, $W^1_4(C)$ and $W^2_6(C)$ are each isomorphic to $E$.

Hint: Use the description of the canonical model of $C$ and the geometric Riemann--Roch theorem.
\end{exercise}


\begin{exercise}
Let $C$ be a trigonal curve of genus 6 with $g^1_3$ $|E|$, and $p \in C$ a general point. Show that the linear series $|K_C - E-p|$ is a $g^2_6$, and that the corresponding map $C \to \PP^2$ maps $C$ birationally onto a plane sextic curve with a triple point.
\end{exercise}


\begin{exercise}\label{plane models}
Let $C$ be the normalization of a plane sextic $C_0$ with four nodes, three of which are colinear. Show that by choosing a different $g^2_6$ on $C$, we can express it as the normalization of a plane sextic with two nodes and a tacnode.

Hint: Take the image of $C$ under the map associated to the $g^2_6$ cut out by conics through two of the three collinear nodes and the one remaining node.
\end{exercise}


\begin{exercise}
Show that if $C$ is the normalization of a plane sextic $C_0$ with only double points, then $W^1_4(C) \cong W^2_6(C)$ is zero-dimensional (so in particular this case does not overlap with any of the previous cases)
\end{exercise}


\begin{exercise}
Find an example of a curve $C$ of genus 6 such that $W^1_4(C)$ consists of one point of degree 5. Bonus points for showing that in this case the scheme $W^1_4(C) \cong \Spec \CC[\epsilon]/(\epsilon^5)$.
\end{exercise}

Hint: the curve in question is the normalization of a plane sextic curve with one double point, consisting of two smooth branches with contact of order 4 with each other and contact of order 3 with their common tangent line. The exercise asks you to both prove that such a curve exists, and that the $g^1_4$ cut out by lines through the double point is the unique $g^1_4$ on $C$. 

\begin{exercise}
Show that if $C$ is a smooth plane quintic, then the $g^2_6$s on $C$ all have a basepoint; that is, they are all of the form $|K_C| + p$ for $p \in C$. 

Furthermore, the canonical model of $C$ will lie on a quadratic Veronese surface $S$; and the six
quadrics containing the canonical curve $C$ are the six quadrics containing $S$ (in particular, the intersection of the quadrics containing $C$ will be $S$, so
the ideal of $C$ requires generators of degree $>2$.
\end{exercise} 

\begin{exercise}\label{nonreduced Wrd}
Show that if the nodes of the curve $C_0$ are in linear general position\emdash that is, no three collinear\emdash then indeed the map $\mu : H^0(D) \otimes H^0(K-D) \to H^0(K)$ is an isomorphism for each of the five $g^1_4$s on $C$.
\end{exercise}





\input footer.tex