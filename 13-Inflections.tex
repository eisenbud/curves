%header and footer for separate chapter files

\ifx\whole\undefined
\documentclass[12pt, leqno]{book}
\usepackage{graphicx}
\input style-for-curves.sty
\usepackage{hyperref}
\usepackage{showkeys} %This shows the labels.
%\usepackage{SLAG,msribib,local}
%\usepackage{amsmath,amscd,amsthm,amssymb,amsxtra,latexsym,epsfig,epic,graphics}
%\usepackage[matrix,arrow,curve]{xy}
%\usepackage{graphicx}
%\usepackage{diagrams}
%
%%\usepackage{amsrefs}
%%%%%%%%%%%%%%%%%%%%%%%%%%%%%%%%%%%%%%%%%%
%%\textwidth16cm
%%\textheight20cm
%%\topmargin-2cm
%\oddsidemargin.8cm
%\evensidemargin1cm
%
%%%%%%Definitions
%\input preamble.tex
%\input style-for-curves.sty
%\def\TU{{\bf U}}
%\def\AA{{\mathbb A}}
%\def\BB{{\mathbb B}}
%\def\CC{{\mathbb C}}
%\def\QQ{{\mathbb Q}}
%\def\RR{{\mathbb R}}
%\def\facet{{\bf facet}}
%\def\image{{\rm image}}
%\def\cE{{\cal E}}
%\def\cF{{\cal F}}
%\def\cG{{\cal G}}
%\def\cH{{\cal H}}
%\def\cHom{{{\cal H}om}}
%\def\h{{\rm h}}
% \def\bs{{Boij-S\"oderberg{} }}
%
%\makeatletter
%\def\Ddots{\mathinner{\mkern1mu\raise\p@
%\vbox{\kern7\p@\hbox{.}}\mkern2mu
%\raise4\p@\hbox{.}\mkern2mu\raise7\p@\hbox{.}\mkern1mu}}
%\makeatother

%%
%\pagestyle{myheadings}

%\input style-for-curves.tex
%\documentclass{cambridge7A}
%\usepackage{hatcher_revised} 
%\usepackage{3264}
   
\errorcontextlines=1000
%\usepackage{makeidx}
\let\see\relax
\usepackage{makeidx}
\makeindex
% \index{word} in the doc; \index{variety!algebraic} gives variety, algebraic
% PUT a % after each \index{***}

\overfullrule=5pt
\catcode`\@\active
\def@{\mskip1.5mu} %produce a small space in math with an @

\title{Personalities of Curves}
\author{\copyright David Eisenbud and Joe Harris}
%%\includeonly{%
%0-intro,01-ChowRingDogma,02-FirstExamples,03-Grassmannians,04-GeneralGrassmannians
%,05-VectorBundlesAndChernClasses,06-LinesOnHypersurfaces,07-SingularElementsOfLinearSeries,
%08-ParameterSpaces,
%bib
%}

\date{\today}
%%\date{}
%\title{Curves}
%%{\normalsize ***Preliminary Version***}} 
%\author{David Eisenbud and Joe Harris }
%
%\begin{document}

\begin{document}
\maketitle

\pagenumbering{roman}
\setcounter{page}{5}
%\begin{5}
%\end{5}
\pagenumbering{arabic}
\tableofcontents
\fi


\chapter{Inflection points}\label{inflections chapter}
\label{InflectionsChapter}


Generalizing the ramification points of a map from a smooth curve $C$ to $\PP^1$, there are finitely
many ``special'' points determined by any linear series on any curve, the \emph{inflection points}.
We love this topic for its echos of classical algebraic geometry and it will provide the tools to give a proof of the Brill-Noether theorem in the following chapter. In characteristic 0 every linear series has finitely many inflection points, and the number of these, properly counted, depends only on the genus of the curve and the degree of the linear series.

\section{Inflection points,  Pl\"ucker formulas and Weierstrass points}

\subsection{Definitions}
To characterize the inflection points of a linear series $\sD = (\sL, V)$ on a curve $C$, we will use the following result:

\begin{proposition}\label{vanishing sequence} Let $V$ be a finite-dimensional vector space of global sections of an invertible sheaf $\sL$ on a smooth curve $C$, and let $p \in C$ be a point. There exists a basis $\sigma_0, \dots, \sigma_r$ of $V$ consisting of sections vanishing to different orders at $p$. Thus the set
$$
\{ \ord_p(\sigma) \mid \sigma \neq 0 \in V \}
$$
 has cardinality $\dim V$.
\end{proposition}

\begin{proof} Let $\tau_0, \dots, \tau_r$ be any basis of $V$.  If  $\tau_i$ and $\tau_j$ vanish to the same order at $p$, then 
some nonzero linear combination $\tau_i' := a\tau_i+b\tau_j$   vanishes to strictly higher order. Since the coefficients $a$ and $b$ are both necessarily nonzero we may modify the basis, replacing $\tau_i$ with $\tau_i'$, strictly increasing the sum of the orders. 
The order of vanishing of each $\sigma_i$ is bounded above by $\deg \sL$, so the sum of the orders is bounded by $(r+1)\deg \sL$. Thus the process must terminate, and when it does,
 the orders must be distinct. \end{proof}

According to  Proposition~\ref{vanishing sequence}, if $\cD = (\cL, V)$ is any $g^r_d$ on $C$, we may write
$$
\{ \ord_p(\sigma) \mid \sigma \neq 0 \in V \} = \{a_0,\dots,a_r\} \; \text{ with } \; 0\leq a_0 < a_1 < \dots < a_r.
$$
The sequence $a_i = a_i(\cD,p)$ is called the \emph{vanishing sequence} of $\cD$ at $p$.  Since $a_i \geq i$, the numbers $\alpha_i = \alpha_i(\cD,p) := a_i - i$ are often more interesting, and the sequence $0 \leq \alpha_0 \leq \alpha_1 \leq \dots \leq \alpha_r$ is called the \emph{ramification sequence} of $\cD$ at $p$. 

We say that $p$ is an \emph{inflection point} of the linear series $\cD$ if $(\alpha_0,\dots,\alpha_r) \neq (0,\dots,0)$---equivalently, if $\alpha_r > 0$---and we define the \emph{weight} of $p$ to be
$$
w(\cD, p) = \sum_{i=0}^r \alpha_i(\cD, p).
$$
If $\cD$ is very ample, so that it may be viewed as the linear series cut on $C$ by hyperplanes for some embedding $C \subset \PP^r$, then $p$ is an inflection point if $a_r > r$; that is, if there is a hyperplane $H \subset \PP^r$ having contact of order $r+1$ or more with $C$ at $p$. 

The first two terms in the ramification sequence are particularly important: $\alpha_0(\cD, p)$ is nonzero if and only if $p$ is a base point of $\cD$; and if $\alpha_0(\cD, p)=0$, then $\alpha_1(\cD, p) = 0$ if and only if, in addition, the map $\phi_\cD$ is an immersion (that is, has nonzero derivative) at $p$.


\subsection{The Pl\"ucker formula}

In characteristic 0, a linear series on a smooth projective curve can have only finitely many inflection points (this can be proven
directly just as in Lemma~\ref{finite inflections}), and indeed the sum of the weights of all the inflection points depends only on the genus of the curve and the degree and dimension of the the linear series. This is the Pl\"ucker formula:

\begin{theorem}\label{Plucker}
If $C$ be a smooth curve of genus $g$ and $\cD$ is a
linear series of degree $d$ and dimension $r$, then
 \begin{equation}\label{Plucker formula}
\sum_{p \in C} w(\cD, p) \; = \; (r+1)d + r(r+1)(g-1).
\end{equation}
\end{theorem}
For a proof, see for example~\cite[Theorem 7.13]{allthat}.

Theorem~\ref{Plucker} also holds in positive characteristic under the hypothesis that the number of inflection points is finite (equivalently, not every point is an inflection point). This may seem like an unnecessary hypothesis---it's hard to imagine a plane curve in which every point is a flex!---but in positive characteristic there are such curves; see Exercise~\ref{inseparable Gauss}.

As an immediate consequence of the Pl\"ucker formula, we have

\begin{corollary}\label{uninflected curves}
 If $C\subset \PP^r$ is a smooth nondegenerate curve with no inflection points, then $C$ is the rational normal curve of degree $r$. 
\end{corollary}

\begin{proof}
Suppose $C \subset \PP^r$ is a curve of degree $d$ and genus $g$. If $C$ has no inflection points then, by the Pl\"ucker formula, we must have
$$
(r+1)d + r(r+1)(g-1) = 0.
$$
This immediately implies that $g=0$, so that we must have $(r+1)(d-r) = 0$ and hence $d=r$; thus $C$ is a rational normal curve.
\end{proof}

\begin{corollary}
A nondegenerate smooth  curve $C\subset \PP^{r}$ is the rational normal curve
of degree $r$ if and only if any of the following equivalent properties holds:
\begin{enumerate}
 \item $C$ is projectively homogeneous.
 \item $C$ has no inflection points.
 \item Every subscheme of length $r+1$ contained in $C$ spans $\PP^{r}$.
\end{enumerate}
\end{corollary}

\begin{proof}
Corollary~\ref{uninflected curves} implies that the
rational normal curves are the only ones without any inflection points.

By Proposition~\ref{Veronese is projectively homogeneous}, the rational normal curve 
is projectively homogeneous. On the other hand, since not every point on $C$ is an inflection point, a curve with an inflection point cannot be projectively homogeneous.

If $p \in C$ is an inflection point, then $(r+1)p$ is  a subscheme that lies in a hyperplane
whereas in Proposition~\ref{independence of points on a RNC} we showed that if $C \subset \PP^r$ is a rational normal curve, and $\Gamma \subset C$ any proper subscheme of $C$ of degree $r+1$, then $\Gamma$ spans $\PP^r$. 
\end{proof}

The Pl\"ucker formula leaves many questions about the possible configurations of flex points unanswered. 
For example, how many cusps can there be on a plane curve of geometric genus $g$ and degree $d$? 
The answer is known only up to degree 8: \cite{Zariski1931} proves that a plane curve of degree 8 can have 15 but not 16 cusps.
See \cite{Calabri} and \cite{Kulikov} for recent contributions and \cite{Kharlamov-Sottile} for a study of what is possible
over the real numbers.

\begin{figure}\label{3-cusp quartic}
\centerline{ \includegraphics[scale=.5]{"main/Fig12-1-threeCusps"}}
 \caption{A quartic plane curve with 3 cusps. {Silvio: doesn't need to be blue.
 could be reproduced. the credit could be less important (footnote?) (Image credit F. Sottile)}}
\end{figure}

One thing we do know is the behavior of the inflection points for a general linear series on a general curve; we will state this here and prove it as a corollary to the proof of the Brill-Noether theorem in Chapter~\ref{BrillNoetherproofChapter}:

\begin{theorem}\label{Brill Noether Plucker}
If $C$ is a general curve of genus $g$, $\sL \in W^r_d(C) \subset \pic_d(C)$ a general line bundle of degree $d$ with $h^0(\sL) = r+1$ and $V = H^0(\sL)$, then every inflection point of the linear series $\cD = (\sL, V)$ has weight 1 and hence ramification sequence $(0, \dots, 0, 1)$.
\end{theorem}


\subsection{Flexes of plane curves}\label{plane curve pluecker}

Specializing Theorem~\ref{Plucker} to a smooth curve of degree $d$ in the plane, and using the formula
$g= {d-1\choose 2}$, we see that the total number of flexes is $3(d-2)d$. 

It turns out that the ``flex divisor''
$\sum w(C, p)p$
is the intersection of $C$ with a curve of degree $3(d-2)$, called the \emph{Hessian}: If $C$ is defined by a form $F(x_0, x_1, x_2)$ of degree $d$, then
the Hessian of $C$ is the curve defined by the determinant of the \emph{Hessian matrix} of partial derivatives
$$
Hess(C) :=
\begin{pmatrix}
 \partial^2 F/\partial x_0 \partial x_0 & \partial^2 F/\partial x_0 \partial x_1 & \partial^2 F/\partial x_0 \partial x_2 \\
\partial^2 F/\partial x_1 \partial x_0 & \partial^2 F/\partial x_1 \partial x_1 & \partial^2 F/\partial x_1 \partial x_2 \\
\partial^2 F/\partial x_2 \partial x_0 & \partial^2 F/\partial x_2 \partial x_1 & \partial^2 F/\partial x_2 \partial x_2 
\end{pmatrix}
$$
\begin{theorem}\label{Hessian} If $C$ is a smooth plane curve then the flex divisor of $C$ is the intersection 
of $C$ with the Hessian curve defined by $\det Hess(C)$.
\end{theorem}
The proof is an exercise in Euler's formula and matrix manipulation; see Exercise~\ref{Hessian exercise}.

\subsection{Weierstrass points}\label{Weierstrass points}

As with any extrinsic invariant of a curve in projective space, we can derive an intrinsic invariant of an abstract curve by applying the invariant to the canonical linear series. We define a \emph{Weierstrass point} of a curve $C$ to be an inflection point of the canonical linear series $|K_C|$. 

Thus $p$ is a Weierstrass point of $C$ if there exists a  differential form on $C$ vanishing to order $g$ or more at $p$. The \emph{weight} $w_p$ of a Weierstrass point $p \in C$  is defined to be the weight $w(|K_C|,p)$ of $p$ as an inflection point of the canonical series. 

The Pl\"ucker formula tells us  the total weight of the Weierstrass points on a given curve $C$:

\begin{corollary}\label{plucker formula}
The sum of the weights of the Weierstrass points on a curve $C$ of genus $g$ is
$$
\sum_{p \in C} w_p = g^3-g.
$$\qed
\end{corollary}

Note that Theorem~\ref{Brill Noether Plucker} implies that on a general curve $C$ of genus $g$, every Weierstrass point has weight 1; thus there are $g^3-g$ distinct Weierstrass points on $C$.

%Since there are only finitely many ramification points of the canonical series, a general point $p$ on any curve $C$ has gap sequence $(1,2,\dots,g)$, and correspondingly its Weierstrass semigroup
%is $W_p = (0, g+1, g+2, \dots)$. A Weierstrass point is called \emph{normal} if it has weight 1; this is tantamount to saying that the gap sequence is $(1,2,\dots,g-1,g+1)$, or that the semigroup is $(0, g, g+2, g+3, \dots)$. (The full Brill-Noether theorem tells us that a general curve $C$ has only normal Weierstrass points; this will be a consequence of Theorem~\ref{BN with inflection and dimension} below.)

For example, suppose $C$ is a curve of genus 2. The canonical series on $C$ gives a map $\phi_K : C \to \PP^1$ of degree 2; the Weierstrass points of $C$ are the 6 ramification points of this map. 

In genus 3, if $C$ is hyperelliptic then the Weierstrass points are exactly the 8 ramification points of the 2-sheeted cover $C \to \PP^1$, and each has weight 3. If $C$ is non-hyperelliptic, then it is a plane quartic curve. A general such curve has 24 ordinary flexes, which are Weierstrass points of weight 1; special quartics may have some number $\alpha$ of \emph{hyperflexes}---points where the tangent line has contact of order 4 with the curve---which are Weierstrass points of weight 2; in this case $C$ has $\alpha$ Weierstrass points of weight 2 and $24-2\alpha$ Weierstrass points of weight 1. (It has been shown that $\alpha$ cannot be 11, but  all  other values  between 0 and 12 occur (\cite{Vermeulen}).)

\subsection{Another characterization of Weierstrass points}

The Riemann-Roch formula tells us that
$$
h^0(\cO_C(gp)) = g - g + 1 + h^0(K_C(-gp))
$$
so the condition $h^0(K_C(-gp)) \neq 0$ that $p$ be a Weierstrass point is equivalent to the condition $h^0(\cO_C(gp)) > 1$; in other words, we have

\begin{proposition}\label{Weierstrass characterization}
A point $p \in C$ is a Weierstrass point if and only if there exists a nonconstant rational function on $C$, regular on $C \setminus \{p\}$ and having a pole of order at most $g$ at $p$.
\end{proposition} 

Indeed, this is how Weierstrass points were characterized when they were first studied.

\subsubsection{The Weierstrass semigroup} 

Proposition~\ref{Weierstrass characterization} suggests that we look at the set of all possible orders of pole at $p$ of rational functions regular on $C \setminus \{p\}$; that is,
$$
W(C,p) := \left\{ -\ord_p(f) \mid f \in K(C) \text{ with $f$ regular on } C \setminus \{p\} \right\}.
$$
This is clearly a sub-semigroup of the natural numbers $\NN$; it is called the \emph{Weierstrass semigroup} of the point $p$.  

Another way to characterize the condition that there exists a rational function on $C$, regular on $C \setminus \{p\}$, with a pole of order exactly $k$ at $p$ is to say that
$$
h^0(\cO_C(kp)) = h^0(\cO_C((k-1)p)) + 1.
$$
Applying Riemann-Roch to both sides of this equation, we see that it is equivalent to the condition
$$
h^0(K_C(-kp)) = h^0(K_C((-k+1)p)).
$$

In English: there exists a rational function on $C$, regular on $C \setminus \{p\}$, with a pole of order exactly $k$ at $p$, if and only if there does \emph{not} exist a regular differential on $C$ with a zero of order exactly $k-1$ at $p$.
 In other words, the complement $\NN \setminus W(C,p)$ is exactly the vanishing sequence of the canonical series at $p$, shifted by 1; in particular, it has cardinality  exactly $g$. This is called the \emph{Weierstrass gap sequence} of the point $p$.

There is still much we don't know about Weierstrass points in general. Most notably, we don't know what semigroups of finite index in $\NN$ occur as Weierstrass semigroups; an example of Buchweitz shows that not all semigroups occur, but there are also positive results, such as the statement ([EH]) that every semigroup of weight $w \leq g/2$ occurs, and its refinement and strengthening in~\cite{MR3892968}).


\section{Finiteness of the automorphism group}\label{finiteness section}

Because the Weierstrass points of a smooth projective curve $C$ are defined intrinsically, any automorphism of $C$ must carry Weierstrass points to Weierstrass points. We can use this observation to give a direct proof of the

\begin{theorem}\label{finite autos}
If $C$ is a smooth curve of genus $\geq 2$ then the automorphism group $\Aut(C)$ is finite.
\end{theorem}

We will actually prove a strong form of the assertion: the subgroup of $Aut(C)$ of automorphisms that fix each  of the finitely many Weierstrass points is either
 trivial or, in the case of hyperelliptic curves, just the $\ZZ/2$ generated by the hyperelliptic involution.
    
We will study the fixed points of automorphisms by studying the intersection of the graph of
an automorphism with the diagonal divisor in $C\times C$ (an alternative proof uses the Lefschetz fixed point formula). For this we use the
\emph{Neron-Severi group} $N(S)$ of a surface $S$. This consists of divisors modulo \emph{numerical equivalence}---that is, in $N(S)$ we identify divisors $H, H'$ if for all divisors $D$ on $S$ we have $H\cdot D = H'\cdot D$. The group
$N(S)$ is a  finitely generated free abelian group for any surface---in characteristic 0 it is a subgroup of the quotient of the second integral homology group by its torsion elements. The rank of $N(S)$ is called the \emph{Picard number} of $S$. 

\begin{theorem}[Hodge index theorem]\label{hodge index}
If $H\subset S$ is an ample divisor on a smooth projective surface, and $D \neq 0 \in N(S)$ is a divisor class with $D\cdot H = 0$, then 
$D^2<0$; that is, the intersection pairing is negative definite on the orthogonal complement of an
ample divisor.
\end{theorem}
\begin{proof}
The result follows easily from the Riemann-Roch theorem for surfaces; see \cite[Theorem V.1.9]{Hartshorne1977}.  
\end{proof}

There is also a stronger form of the Hodge index theorem, in which the condition that $D$ is ample is weakened to the requirement that $D^2 > 0$. The proof of this stronger form relies on Hodge theory; see for example~\cite{MR1997577} or~\cite{Griffiths-Harris1978}.

Together, the following two lemmas imply the stronger form of Theorem~\ref{finite autos}:

\begin{lemma}\label{2g+3fixedpoints}
Let $C$ be a smooth projective curve of genus $g \geq 2$, and $f: C \to C$ an automorphism of $C$.
 If $f$ has $2g+3$ or more distinct fixed points, then $f$ is the identity.
\end{lemma}

\begin{proof}
Let $S = C\times C$, and let $\Delta$ and $\Gamma \subset S$ be the diagonal and the graph of $f$ respectively, and let $\Phi_1$ and $\Phi_2 \subset S$ be fibers of the two projection maps. Let $\delta, \gamma, \varphi_1$ and $\varphi_2$ be the classes of these curves in  $N(S)$. The number of fixed points of $f$ (counted with multiplicities) is the intersection number  $b = \delta \cdot \gamma$.

We know all the other pairwise intersection numbers of these classes. To begin with, the ones involving $\varphi_1$ or $\varphi_2$ are obvious. From the sequence
$$
0\to \sT_{\Delta} \to \sT_{C\times C}|_{\Delta} \to \sN_{\Delta/C\times C} \to 0
$$
we see that the normal bundle of the diagonal is isomorphic to the tangent bundle of $C$, so
$$
\delta^2 = 2 - 2g.
$$
Since the automorphism $id_C \times f : C\times C \to C \times C$ carries $\Delta$ to $\Gamma$, we see that $\gamma^2 = 2-2g$ as well.

We can now apply the index theorem for surfaces to deduce our inequality. To keep things relatively simple, let's introduce two new classes: set
$$
\delta' = \delta - \varphi_1 - \varphi_2 \quad \text{and} \quad \gamma' = \gamma - \varphi_1 - \varphi_2,
$$
so that $\delta'$ and $\gamma'$ are orthogonal to the class $\varphi_1 + \varphi_2$. Since $\varphi_1 + \varphi_2$ has positive self-intersection, the index theorem  tells us that the intersection pairing must be negative definite on the span $\langle \delta',\gamma' \rangle \subset N(S)$. In particular, the determinant of the intersection matrix
\begin{center}
\begin{tabular}{c|c|c}
& $\delta'$ &  $\gamma'$  \\
\hline
$\delta'$ & $-2g$ & $b-2$ \\
\hline
$\gamma'$ & $b-2$ & $-2g$ 
\end{tabular}
\end{center}
(where again $b = \gamma \cdot \delta$) is nonnegative, or equivalently, $b\leq 2g+2$.
\end{proof}

Having established an upper  bound on the number of fixed points an automorphism $f$ of $C$ (other than the identity) may have, it remains to find a lower bound on the number of distinct Weierstrass points; this is the content of the next lemma.


\begin{lemma}\label{2g+2Weierstrass}
If $C$ is a smooth projective curve of genus $g \geq 2$, then $C$ has at least $2g+2$ distinct Weierstrass points; and if it has exactly $2g+2$ Weierstrass points it is hyperelliptic.
\end{lemma}

\begin{proof}
Let $p \in C$ be any point, and $w_1=w_1(p),\dots,w_g = w_g(p)$ the ramification sequence of the canonical series $|K_C|$ at $p$. By definition, 
$$
h^0(K_C(-(w_i+i)p)) = g - i.
$$
Applying Clifford's theorem we have
$$
g-i \leq \frac{2g - 2 - w_i - i}{2} + 1;
$$
solving, we see that
$$
w_i \leq i
$$
and hence
$$
w_p \leq \binom{g}{2}
$$
where $w_p$ is the total weight of $p$ as a Weierstrass point. Since the total weight of the Weierstrass points on $C$ is $g^3-g$ by the Pl\"ucker formula Theorem~\ref{Plucker}, the number of distinct Weierstrass points is at least
$$
\frac{g^3-g}{\binom{g}{2}} = 2g+2.
$$
Finally, by the strong form of Clifford's theorem~(\ref{Clifford equality}), equality implies that the curve is hyperelliptic.
\end{proof}

We can deduce Theorem~\ref{finite autos} from Lemmas~\ref{2g+3fixedpoints} and~\ref{2g+2Weierstrass} as follows. If $C$ is non-hyperelliptic, then Lemma~\ref{2g+2Weierstrass} says it must have at least $2g+3$ Weierstrass points, and then Lemma~\ref{2g+3fixedpoints} says that any automorphism fixing all the Weierstrass points must be the identity. If, on the other hand, $C$ is hyperelliptic, with degree 2 map $\pi : C \to \PP^1$, then since the $g^1_2$ on $C$ is unique, any automorphism of $C$ must carry fibers of $\pi$ to fibers of $\pi$; that is, it must commute with an automorphism of $\PP^1$. In other words, we have an exact sequence
$$
0 \to \ZZ/2 \to Aut(C) \to G \to 0
$$
where $G$ is a subgroup of the group of automorphisms of $\PP^1$ preserving  the set of branch points of $\pi$. Since there are at least 6 such branch points, this group is finite.

Note that the argument here gives an explicit bound for the size of $Aut(C)$, namely $(g^3-g)!$. In fact, we can do much better: in Exercise~\ref{84(g-1)}, the reader is invited to prove the inequality $|Aut(C)| \leq 84(g-1)$, which is in fact sharp for infinitely many $g$.

\section{Curves with automorphisms are special}\label{curves with automorphisms}

In genus $g > 2$, curves with nontrivial automorphisms are rare. The general curve has none, and more is true:

\begin{lemma}
In the moduli space $M_g$ of smooth curves of genus $g \geq 3$, the locus of curves with nontrivial automorphisms has codimension $g-2$; and the only component of this codimension is the locus
of hyperelliptic curves.
\end{lemma}

\begin{proof} 
Let $C$ be a smooth curve of genus $g$ admitting a nontrivial automorphism $\phi : C \to C$. Taking a power, we may assume $\phi$ has prime order $p$. 
%Since the number of automorphisms is bounded,
%there are only finitely many primes $p$ to consider. 
Let $\langle \phi \rangle = \{1, \phi, \phi^2,\dots,\phi^{p-1} \}$ be the group of automorphisms of $C$ generated by $\phi$, and let $B = C/\langle \phi \rangle $ be the quotient of $C$ by this group, so that the quotient map $\pi : C \to B$ expresses $C$ as a $p$-sheeted branched cover of $B$, having $b$ branch points $q_1,\dots, q_b$ of multiplicity $p-1$ and otherwise unramified.
As we have seen in Chapter~\ref{genus 2 and 3 chapter}, if we know $B$ and the set
of branch points on $B$ then we know $C$ up to a finite set of choices. 

The point is, in a neighborhood of $[C] \in M_g$, the locus of curves with an automorphism of order $p$ has dimension at most $3h-3 + b$. Our claim is thus that
$$
3g-3 - (3h-3+b) \geq g-2
$$
or equivalently
$$
2g - 1 - (3h-3) - b \geq 0.
$$
By applying the Riemann-Hurwitz formula to the map $C \to B$, we get
$$
2g-2 = p(2h-2) + b(p-1).
$$
Plugging this in, our claim is equivalent to the assertion that
$$
p(2h-2) + b(p-1) + 1 - (3h-3) - b \geq 0
$$
that is,
$$
(2p-3)h -2p + b(p-2) + 4 \geq 0.
$$
Now, the expression on the left could be negative only if $h=0$, in which case the expression reduces to
$$
(b-2)(p-2) \geq 0;
$$
and since $h=0$ the number $b$ of branch points must be at least 2, so we're done.
\end{proof}

%\begin{lemma}
%In the moduli space $M_g$ of smooth curves of genus $g \geq 3$, the locus of curves with nontrivial automorphisms has codimension $g-2$; and the only component of this codimension is the locus
%of hyperelliptic curves.
%\end{lemma}
%
%\begin{proof} 
%Let $C$ be a smooth curve of genus $g$ admitting a nontrivial automorphism $\phi : C \to C$. Taking a power, we may assume $\phi$ has prime order $p$. 
%%Since the number of automorphisms is bounded,
%%there are only finitely many primes $p$ to consider. 
%Let $\langle \phi \rangle = \{1, \phi, \phi^2,\dots,\phi^{p-1} \}$ be the group of automorphisms of $C$ generated by $\phi$, and let $B = C/\langle \phi \rangle $ be the quotient of $C$ by this group. As we have seen in Chapter~\ref{genus 2 and 3 chapter}, if we know $B$ and the set
%of branch points on $B$ then we know $C$ up to a finite set of choices.  Moreover, the number $b$ of 
%branch points is determined from Hurwitz's theorem by the genera of $C$ and $B$. Note that every branch point must have multiplicity $p - 1$, so that the number of branch points is
%$$
%b \; = \; \frac{2g-2 - p(2h-2)}{p-1}
%% \leq 2g-2 - p(2h-2).
%$$ 
%where $h$ is the genus of $B$. Thus we see that the curves with an automorphism of order $p$
%form a family of dimension at most
%$$
%\dim M_{h}+ b = 3h-3 + \frac{2g-2 - p(2h-2)}{p-1}.
%$$ 
%It is not hard to see that the maximum is taken for $h=0$ and $p=2$, whence the codimension in $M_g$
%of any component of the locus of curves of genus $g$ with a nontrivial  automorphism is $\geq 3g+3 - (2g+4+1) = g-2$,
%taken on only by double coverings of $\PP^{1}$.
%\end{proof}
%
%
%
%Finally, let $b$ be the number of fixed points of $\phi$, so that the quotient map $\pi : C \to B$ expresses $C$ as a $p$-sheeted branched cover of $B$, having $b$ branch points $q_1,\dots, q_b$ of multiplicity $p-1$ and otherwise unramified. Thus, to specify the curve $B$ and the $b$ points $q_i$ determines $C$ up to a finite number of choices. 
%
%The point is, in a neighborhood of $[C] \in M_g$, the locus of curves with an automorphism of order $p$ has dimension at most $3h-3 + b$. Our claim is thus that
%$$
%3g-3 - (3h-3+b) \geq g-2
%$$
%or equivalently
%$$
%2g - 1 - (3h-3) - b \geq 0.
%$$
%By applying the Riemann-Hurwitz formula to the map $C \to B$, we get
%$$
%2g-2 = p(2h-2) + b(p-1).
%$$
%Plugging this in, our claim is equivalent to the assertion that
%$$
%p(2h-2) + b(p-1) + 1 - (3h-3) - b \geq 0
%$$
%that is,
%$$
%(2p-3)h -2p + b(p-2) + 4 \geq 0.
%$$
%Now, the expression on the left could be negative only if $h=0$, in which case the expression reduces to
%$$
%(b-2)(p-2) \geq 0;
%$$
%and since $h=0$ the number $b$ of branch points must be at least 2, so we're done.

\section{Inflections of linear series on $\PP^1$}

We are in a position to describe all possible inflectionary behavior of linear series on $\PP^1$. This will provide the essential ingredient in our proof of the Brill-Noether theorem in the next chapter.

%\def\grd{{g^{r}_{d}}}
%\def\balpha{{\boldsymbol{\alpha}}}
%\def\bbeta{{\boldsymbol{\beta}}}

Giving a $\grd$ on $\PP^{1}$ amounts to choosing an $(r+1)$-dimensional subspace $W$ of  $H^{0}(\sO_{\PP^{1}}(d))$ and thus the family of $\grd$s is parametrized by the
Grassmannian $G(r+1, H^{0}(\sO_{\PP^{1}}(d))) = G(r+1, d+1)$.

\goodbreak

\begin{theorem}\label{transversality of ramification}
The set of $\grd$s on $\PP^{1}$  with ramification sequence that is termwise at least as big as $\balpha = (\alpha_{0}, \dots, \alpha_{r})$ 
at a
given point $p\in \PP^{1}$
is an irreducible subvariety of $G(r+1,d+1)$ of codimension $|\balpha| := \sum \alpha_{i}$. The intersection
of these subvarieties for any set of distinct points and any ramification sequences at those points
is either empty or dimensionally transverse, depending only on the collection of ramification sequences.
\end{theorem}

We will restate and prove this theorem, giving the combinatorial condition on the ramification indices,
as Theorem~\ref{osculating intersection} below. To simplify the notation, we will write $\ell := r+1, e:= d+1$. 


Let $V =H^{0}(\sO_{\PP^{1}}(d))$ and, for $p\in \PP^{1}$, write $V_{i}(p)$ for the space of
forms of degree $d$ that vanish to order $\geq e-i$ at $p$, and defined the \emph{vanishing flag} $\sV(p)$ at $p$
to be the chain of subspaces
$$
0\subset V_{1}(p) \subsetneq V_{2}(p) \subsetneq\cdots\subsetneq V_{e}(p) = V.
$$
Note that $\dim V_{i}(p) = i$.
A subspace $W$ of dimension $\ell = r+1$ has vanishing sequence termwise $\geq {\bf a} = a_{0}, \dots a_{r}$ and thus
ramification sequence termwise
$$
\geq \balpha = (\alpha_{0} = a_{0}-0, a_{1}-1 \dots, \alpha_{r}= a_{r}-r)
$$
 at $p$ if and only if
$\dim W\cap V_{e-a_{\ell-i}}(p)\geq  i$  for  $i = 0,\dots, \ell$.
Such a condition is called a Schubert condition
on $W$, and we pause to describe the \emph{Schubert varieties} in
the Grassmannian $G(\ell, e)$ that consist
of subspaces satisfying such a condition. See for example~\cite[Chapters 3 and 4]{3264} for a full exposition.
To accord with the notation there, which indexes Schubert cycles by certain decreasing sequences,
we define $\beta_{i} = \alpha_{\ell-i}$ so that
$$
d-r = e-\ell \geq \beta_{1} \geq \cdots \geq \beta_{\ell}\geq 0,
$$
and we write $\bbeta$ for the sequence of $\beta_{i}$. Putting this together, we have:
\begin{proposition}
The condition for an $\ell$-dimensional subspace $W \subset H^{0}(\sO_{\PP^{1}}(d))$
\marginpar{\redden{There were two propositions with the label `ramification': this one and Proposition \ref{ramification}. Since this label was getting overridden, I removed it, but you should check that none of the refs to  Proposition \ref{ramification} should be to this one.}}
to define a $\grd$ with ramification sequence at least $\balpha =(\alpha_{0},\dots, \alpha_{r})$ at $p$ is
$$
\dim W\cap V_{e-\ell+i-\beta_{i}}(p)\geq  i \hbox{ for } 0\leq i \leq \ell.
$$
where $e:=d+1$ and $\beta_{i} = \ell-\alpha_{i}$.
\end{proposition}


%$$\balpha, \bbeta$$
%dimensional subspacesn $\PP^{1}$ may be identified with an $r+1$ dimensional subs If we embed $\PP^1$ in $\PP^d$ as a rational normal curve $C$ then we c any $g^r_d$ on $\PP^1$ will be the linear system cut out on $C$ by hyperplanes containing a fixed $(d-r-1)$-plane $W$, and the inflectionary behavior of the $g^r_d$ is determined by how $W$ intersects the flag of \emph{osculating spaces}  to $C$ at a point, which we will define below. 

\subsection{Schubert cycles}\label{Schubert1}

\begin{definition}
A \emph{complete flag} $\cal V$  in an $e$-dimensional vector space $V$ is a nested sequence of vector spaces
$$
0 \subset V_1 \subset V_2 \subset \dots  \subset V_{e} = V.
$$
with $\dim V_i = i$.
\end{definition}

Given a flag in $V$ and any  $\ell$-dimensional subspace $W \subset V$, we can derive the nested sequence of $e$ subspaces of $W$:
$$
0 \; \subset \; W \cap V_1 \; \subset \;  W \cap V_2 \; \subset \;  \dots \; \subset \;  W \cap V_d \; \subset \;  W \cap V_{e} = W
$$
Each term in this sequence is either equal to the preceding one, or of dimension 1 greater; the former  occurs $e-\ell$ times, and the latter $\ell$ times. For a general $\ell $-plane $W$, the jumps occur at the end; that is, we have $W \cap V_{e-\ell} = 0$, and thereafter the dimension of the intersection goes up by 1 each time. This makes it natural
to describe the special position of a given $\ell $-plane by how early the $i$-th jump occurs: 

\begin{definition}
A \emph{Schubert index} for $G(\ell, e)$ is a sequence ${\bbeta} = (\beta_1,\dots,\beta_{\ell})$ of integers with $e-\ell \geq \beta_1 \geq \beta_2 \geq \dots \geq \beta_{\ell} \geq 0$.
The \emph{Schubert cycle $\Sigma_{\bbeta}({\cal V}) \subset G$} associated to a complete flat $\sV$ in $\CC^e$ and
Schubert index $\bbeta$  is 
$$
\Sigma_{\bbeta}({\cal V}) \; := \; \left\{ W \in G \mid \dim(W \cap V_{e-\ell+i-\beta_i}) \geq i \; \forall i \right\}
$$
%When $\cal V$ is clear from context (or immaterial) we sometimes drop it and write $\Sigma_{\bbeta}$.
\end{definition}
In particular, if $\sV(p)$ is the vanishing flag for forms of degree $d$ at a point $p\in \PP^{1}$, then
a $\grd$ has ramification sequence (termwise) $\geq \balpha$ at $p$ if and only if $W$ belongs
to the Schubert variety $\Sigma_{\bbeta}(\sV(p))$, where $\beta_{i} = \alpha_{\ell-i}$ as above.

%Thus the variety  $\Sigma_{\bbeta}({\cal V})$ is the set of $\ell $-planes $W$ for which the $i$-th jump in the sequence of intersections $W \cap V_{e-\ell} = 0$ occurs
%$\beta_i$ steps earlier than for the general $\ell $-plane. 

%Alternately, since a generic $\ell $-plane meets $V_s$ in dimension $\max\{0, \ell  - (d+1-s) = k-d+s\}$,  
%$\Sigma_{\bbeta}({\cal V})$ is the set of $\ell $-planes that meet $V_s$ in dimension $\geq k-d+s + a_

 
For example, the Schubert cycle $\Sigma_{0\dots,0}$ is the whole Grassmannian, 
and $\Sigma_{1,0\dots,0}(\cal V)$ is the set of $\ell$-planes that meet $V_{e-\ell}$ nontrivially, which is
a hyperplane section of the Grassmannian in its Pl\"ucker embedding. 
More generally, the
\emph{special Schubert cycle} 
$\Sigma_\gamma(\sV) := \Sigma_{\gamma,0,\dots, 0}(\sV)$ 
is the set of $\ell$-planes
meeting  $V_{\gamma-\ell - \beta+1}$ nontrivially.
Since this condition really involves only the single space $U = V_{e-\ell-\gamma+1}$, we sometimes 
 write it
as $\Sigma_\gamma(U)$. 

For any Schubert index $\bbeta$, the codimension of $\Sigma_{\bbeta}({\cal V})$ in $G(\ell, e)$ is $|{\bbeta}|:= \sum \beta_i$
(Exercise~\ref{codim Schubert}).

%The natural isomorphism $\phi$ between the Grassmannian $G(\ell, V)$ of $\ell$-planes in an $e$-dimensional vector space $V$, and the Grassmannian  $G(e-\ell, V^*)$ of $(e-\ell)$-planes in  the dual vector space $V^*$, sends a subspace $W \subset V$ to its annihilator in $V^*$. This isomorphism carries Schubert cycles in $G(\ell, V)$ defined relative to a flag $\cal V$ to Schubert cycles in $G(e-\ell, V^*)$ defined relative to the dual flag ${\cal V}^*$, but with different indices.
%
%To describe the correspondence between Schubert indices,
%suppose that ${\bbeta} = (a_1,\dots, a_{\ell})$ is a Schubert index in $G(\ell, V)$, so that $n-k \geq a_1 \geq a_2 \geq \dots \geq a_{k+1} \geq 0$. The \emph{transpose} Schubert index to be the sequence ${\bbeta}^*:  (b_1,\dots, b_{n-k})$ given by 
%$$
% b_j \; := \; \#\{ i \mid a_i \geq j \}
%$$
%For example, the transpose of the special Schubert cycle
%$\Sigma_{a,0,\dots,0}(\sV)$ is 
%$\Sigma_{1,1,\dots,1,0,\dots,0}(\sV)$ with $a$ 1s.
%For the properties of this operation, see Exercise~\ref{Schubert duality}.



\begin{figure}
\inprogress
\centerline {\includegraphics[height=2.8in]{"main/Fig12-2"}}
caption{
$\Sigma_{1}(L) =$ Lines meeting $L\subset\PP^3$.
$\Sigma_{2}(p)$: Lines passing through $p\in\PP^3$.
$\Sigma_{1,1}(\Lambda)$: Lines in the plane $\Lambda\subset \PP^3$.
{Silvio: I realize that this is in process, but here are some notes:
1. the lines meeting L should look like lines
2. all the lines in $\Lambda$ should meet each other
3. The point p is lowercase in the description, should be lowercase in the illustration
4. The lines through p should look as non-planar as possible.
5. {Silvio: let's use color here}}}
\label{Schubert cycles in G(2,4).}\end{figure}

\subsection{Special Schubert cycles and Pieri's formula}

\begin{fact}
The variety of complete flags in $\CC^e$ is rational, and it follows that the class of $\Sigma_{\bbeta}({\cal V})\subset G$
in the Chow ring $A^*(G(\ell, e))$ of the Grassmanian
%, which, in characteristic 0 is equal to the cohomology ring $H^*(G(\ell, e), \ZZ)$,  
is independent
of the flag $\sV$. It is typically denoted $\sigma_{\bbeta}$. Moreover, the classes $\sigma_{\bbeta}$ form a basis for the Chow ring as a free abelian group.
\end{fact}

Thus the product $\sigma_{\bbeta} \cdot \sigma_{\bbeta'}$ is a linear combination of Schubert classes,
given combinatorially by the \emph{Littlewood-Richardson rule}---see for example~\cite{MR2247964}.
\emph{Pieri's formula} is the special
case of the Littlewood-Richardson rule that expresses the product in the Chow ring $A^*(G(\ell, e))$ of an a special Schubert class with an arbitrary Schubert class.
In Chapter~\ref{BrillNoetherproofChapter} we will use it to describe the possible ramification behavior of rational curves with cusps.

%\def\bdelta{{\boldsymbol{\delta}}}
\begin{fact}
\begin{proposition}\label{Pieri}(Pieri's Formula)
If $\sigma_\gamma$ is a special Schubert class and $\sigma_{\bbeta}$ is an arbitrary Schubert class, then
$$
\sigma_\gamma \cdot \sigma_{\bbeta} \; = \; \sum \sigma_{\bdelta}
$$
where the sum ranges over all Schubert indices ${\bdelta} = (\delta_1, \dots \delta_{\ell})$ with
$$
\sum \delta_i = \gamma + \sum \beta_i \quad \text{and} \quad \beta_i \leq \delta_i \leq \beta_{i-1} \text{ for all } i
$$
\end{proposition}
For a proof, see for example \cite[Section 4.2.4]{3264}.
\end{fact}
Figure~\ref{intersection product} illustrates the degeneration of $\Sigma_1\cdot \Sigma_1$ to $\Sigma_2 \cup \Sigma_{1,1}$.
\begin{figure}
\inprogress
\centerline {\includegraphics[height=2.8in]{"main/Fig12-3"}}
\caption{When the lines $L_1$ and $L_2$ move to meet each other at $P$ they become coplanar in $\Lambda$,
and the intersection $\Sigma_1(L_1) \cap \Sigma(L_2)$
degenerates to the union $\Sigma_2(P) \cup \Sigma_{1,1}(\Lambda)$.
{Silvio: $L_1$ red, $L_2$ blue. The other lines black. You could add another line emanating from $p$, and maybe dotted extensions
of the lines coming out of $p$ to show that they continue behind the plane $\Lambda$ }}
\label{intersection product}
\end{figure}

To understand Proposition~\ref{Pieri}, represent the Schubert class $\sigma_{\bbeta}$ by stacks of coins, with $\beta_1$ coins in the first stack, $\beta_2$ coins in the second stack, and so on. We now want to add a total of $\gamma$ coins to the stacks; we can add any number of them to any stack (including a stack that was previously empty), with the one condition that the new height of each stack can't be larger than the previous height of the stack to its left. This interpretation makes the following corollary clear:

\begin{corollary}\label{intersection with sigma nonzero}
If $\sigma_\gamma$ is any special Schubert class, $\sigma_{\bbeta}$ is any Schubert class,
and $m\geq 0$ is an integer with $m \gamma + \sum \beta_i \leq \dim G(\ell, e) = \ell(e-\ell)$, then   
$$
(\sigma_\gamma)^l \cdot \sigma_{\bbeta} \neq 0 \in A^*(G(\ell, e), \ZZ).
$$
\end{corollary}

%\subsection{Osculating flags of a rational normal curve}
%
%To relate the discussion above to inflection points of linear series on $\PP^1$ we introduce the notion of the \emph{osculating flag} to a curve at a point.
%
%If $C\subset \PP^d$ is a rational normal curve of degree $d$ and $p \in C$ is a point on $C$ then by Corollary~\ref{independence of points on a RNC} the span $\overline{mp}$ is an $(m-1)$-plane in $\PP^d$.
%Thus we have a complete flag of linear subspaces:
%$$
%\{p\} \subset \overline{2p} \subset \overline{3p} \subset \dots \subset \overline{dp} \subset \overline{(d+1)p} = \PP^d
%$$
%which we call the \emph{osculating flag} to $C$ at $p$. Note that the second term $\overline{2p}$ is the tangent line to $C$ at $p$; in general, the $k$-plane $\overline{(k+1)p}$ is called the \emph{osculating $k$-plane to $C$ at $p$}.
%
%For any $(d-r-1)$-plane $W \subset \PP^d$, let $(\sO_C(1), V_W)$ be the linear series cut on the rational normal curve $C\subset \PP^d$ by hyperplanes in $\PP^d$ containing $W$. (Note that if $W \cap C \neq \emptyset$, the linear series $V_W$ will have base points; we do not throw those away, since we want $V_W$ to always be a $g^r_d$.) All $g^r_d$s on $\PP^1$ are given in this way.
%
\begin{proposition}\label{ramification}
Let $C \subset \PP^d$ be a rational normal curve, and $(V_W, \sO_C(1))$ be the $g^r_d$ on $\PP^1$ cut by hyperplanes in $\PP^d$ containing a plane $W \subset \PP^d$ of dimension $d-r-1$, and write $\sV = 0\subset V_1\subset \cdots \subset V_d = \PP^d$ for the osculating flag of $C$ at $p$. The ramification sequence $\alpha(V_W, p)$
is determined by the formula
%
%Let ${\bf a}$ be a Schubert index in the Grassmannian $G(d-r, d+1)$; that is ${\bf a} = (a_1, \dots a_{d-r})$ with $r+1 \geq a_1 \geq \dots \geq a_{d-r} \geq 0$, and let ${\bf a}^*$ be the transpose Schubert index. If $\cal V$ is the osculating flag to the rational normal curve $C \subset \PP^d$ at a point $p$,
%then for any $(d-r-1)$-plane $W \subset \PP^d$, 
$$
W \in \Sigma_{\bf a}({\cal V}) \; \iff \; \alpha_i(V_W, p) \geq {\bf a}^*_{r+1-i} = \#\{j\mid a_j\geq r+1-i\}.
$$
\end{proposition} 
\def\tL{{\widetilde W}}
\def\tsV{{\widetilde \sV}}
\def\tV{{\widetilde V}}
In other words, the ramification sequence $\alpha(V_W, p)$ of the linear series $V_W$ at $p$ is exactly the reverse of the transpose of the Schubert index of the smallest Schubert cycle containing $\tL$, the vector
space associated to $W$, in  $G(d-r, d+1)$.

%\$$van = (\nu_0< \dots< \nu_r);$$
%$$\alpha = (\alpha_0\leq \dots\leq \alpha_r);$$
%$$ {\bf a} = (r+1 \geq a_1\geq \cdots \geq a_{d-r});$$
%$$  {\bf a}^* = (d-r\geq b_1\geq\cdots \geq b_{r+1})$$
%


\begin{proof}
To keep the notation of Section~\ref{Schubert1}, we work with the affine versions
$\tL, \tsV = (\tV_1, \dots, \tV_{d+1})$ of the projection center and the osculating flag at $p$, respectively,
so that, for example $\dim \tL = d-r$. (Note that the space $\tV_W$ of the linear series 
denoted $(V_W, \sO_{\PP^1}(d))$ 
is already a vector space, but for consistency of notation we will write $\tV_\tL$ anyway.)

The condition $\tL \in \Sigma_{\bf a}({\tsV})$
 means that 
$$
\dim \tL \cap \tV_{r+1+i-a_i} \geq i
$$
for each $i$. 
It follows that the codimension of the space of hyperplanes containing $\tL$ has codimension $\leq r-a_i+1$.
%It follows that, 
%in the space of hyperplanes containing $\tL$, it is $\leq r-a_i+1$ further conditions to contain $\tV_{r+i-a_i}$. Equivalently,
since the projective space corresponding to $\tV_{r+i-a_i}$ is the linear span of the divisor $(r+i-a_i+1)p\subset C$,
a codimension $\leq r-a_i+1$ space of sections of $\tV_\tL$  vanishes to order $\geq r -a_i+1+i$ at $p$. 

If we write the distinct orders of vanishing of the sections in $\tV_\tL$ as
$b_0 < b_1<  \cdots < b_r$, we see that $\tL \in \Sigma_{\bf a}$  if and only if
$a_i$ of the $b_j$ are $\geq r+i+a_i+1$, that is,
$b_{r-a_i+1}\geq r-a_i+1+i$ or equivalently $\alpha_{r-a_i+1}\geq i$ for each $i$.
Since $\alpha_i\leq \alpha_{i+1} \leq \alpha_r$,
this is equivalent to saying that the number of $\{j \mid \alpha_j \geq i\}$ is at least $a_i$, or, for the
reverse sequence $\alpha' = \alpha_r \geq \alpha_{r-1} \geq \cdots \geq 0$, that the $a_i$-th term is $\geq i$.
On the other hand ${\bf a'}_{a_i} = \#\{j\mid a_j \geq a_i\} = i$ since the sequence ${\bf a}$ is weakly
decreasing, so $\alpha'$ is termwise $\geq {\bf a'}$.
\end{proof}

\subsection{Conclusion}

Using these ideas we  can rephrase and prove Theorem~\ref{transversality of ramification} in terms of Schubert cycles,
adding the precise condition for the existence of a $\grd$ with prescribed ramification sequences
at an arbitrary collection of distinct points in $\PP^{1}$:

\begin{theorem}\label{osculating intersection}
Let $p_1,\dots,p_\delta \in C$ be distinct points on a rational normal curve $C \subset \PP^d$, and ${\cal V}^1, \dots, {\cal V}^\delta$ the corresponding vanishing flags. If ${\bbeta}^1, \dots, {\bbeta}^\delta$ are $\delta$ Schubert indices for $G(d-r, d+1)$, the Schubert cycles $\Sigma_{{\bbeta}^1}({\cal V}^1), \dots, \Sigma_{{\bbeta}^\delta}({\cal V}^\delta) \subset G(d-r, d+1)$ intersect properly; that is, the intersection is either empty or has codimension exactly $\sum_{j =1}^{\delta}|\bbeta^{j}|$,
the sum of the codimensions of the cycles $\Sigma_{{\bbeta}^j}$. Moreover, the intersection is nonempty if and only if
 the intersection product of the classes $[\Sigma_{{\bbeta}^j}]$ is nonzero in $A^*(G(d-r, d+1))$.
\end{theorem}


\begin{proof} 
If the intersection is empty then the Chow class is 0, so it suffices to show that the intersection is proper,
and we may assume that it is non-empty. Because the Grassmannian is smooth, the codimension of the intersection of any subvarieties
 is at most the sum of their codimensions, so it is enough to show that the codimension of the
 intersection cannot be too small.

The Schubert cycle $\Sigma_1$ is a hyperplane section of the Grassmannian $G(d-r, r+1)$, so that if $\Phi \subset G$ is any subvariety of dimension $m$, its intersection with $m$ Schubert cycles $\Sigma_1$ is nonempty. Thus, if the intersection
$$
X := \bigcap_{i=1}^\delta \Sigma_{{\bbeta}^i}({\cal V}^i)
$$
had dimension strictly bigger than the expected
$$
\rho \; := \; (r+1)(d-r) - \sum_{i=1}^\delta |{\bbeta}^i|,
$$
we could choose $\rho + 1$ additional points $q_1,\dots,q_{\rho + 1}$ on $C$, with vanishing flags ${\cal V}^1, \dots, {\cal V}^{\rho + 1}$ and the intersection of $X$ with the Schubert cycles $\Sigma_1({\cal V}^i)$ would still be nonempty.

It thus suffices to show that the intersection is empty if
$$
\sum_{i=1}^\delta |{\bbeta}^i| \; > \; (r+1)(d-r) = \dim G(d-r, d+1).
$$
If on the contrary 
$$
W \; \in \; \bigcap_{i=1}^\delta \Sigma_{{\bbeta}^i}({\cal V}^i),
$$
then, by Proposition~\ref{ramification}, the linear series $(W, \sO_{\PP^{1}}(d))$  would have
ramification of weight $|{\bbeta_i}|$ at $p_i$, and the sum of the weights would be strictly greater than $(r+1)(d-r)$. 
This contradicts the Pl\"ucker formula, Theorem~\ref{Plucker}.
\end{proof}

Using Proposition~\ref{ramification} this result becomes a characterization of the sets of possible ramification
sequences for linear series at specified points on $\PP^1.$
 Explicitly, suppose we are given a collection of distinct points $p_1,\dots,p_\delta \in \PP^1$, and for each point $p_i$ a ramification sequence
$$
\alpha^i = (\alpha^i_0, \alpha^i_1, \dots, \alpha^i_r) \quad \text{with} \quad 0 \leq \alpha^i_0 \leq \alpha^i_1 \leq \dots \leq \alpha^i_r \leq d-r.
$$
Let ${\bbeta}^i$ be this sequence in reverse (and relabelled); that is
$$
{\bbeta}^i \; = \; (\alpha^i_r, \alpha^i_{r-1}, \dots, \alpha^i_1, \alpha^i_0)
$$
and finally let ${\bf b}^i$ be the transpose of the Schubert index ${\bbeta}^i$. 

\begin{corollary}
With $\alpha^i $ and $\bf b^i$ related as above,  there exists a $g^r_d$ on $\PP^1$ with ramification sequence at $p_i$ equal to $(\alpha^i_0, \alpha^i_1, \dots, \alpha^i_r)$ if and only if 
$$
\prod_{i=0}^\delta  \sigma_{{\bf b}^i} \; \neq \; 0 \quad \text{in} \quad A^*(G(d-r, r+1)).
$$
\end{corollary}

\begin{proof}
If the product is nonzero, Proposition~\ref{ramification} and Theorem~\ref{osculating intersection} immediately show the existence of a $g^r_d$ on $\PP^1$ with ramification sequence greater than or equal to $\alpha^i$ at $p_i$. But by Theorem~\ref{osculating intersection}, the ones with ramification strictly greater than the $\alpha^i$ form a family of strictly smaller dimension. Thus a general $g^r_d$ with ramification sequence greater than or equal to $\alpha^i$ at $p_i$ has  ramification sequence exactly equal to $\alpha^i$ at $p_i$.
\end{proof}

We can deduce a result about general secant loci strengthening one that was originally proven in \cite{Griffiths-Harris-BN}:
Given a curve $C \subset \PP^d$, we say that a \emph{secant flag} in $\PP^d$ is a flag
$$
0 \subset V_1 \subset V_2 \subset \dots \subset V_{d-1} \subset V_d = \PP^d
$$
where each $V_i$ is spanned by its (scheme-theoretic) intersection with $C$. In other words, there is a sequence of points $p_1, p_2, \dots, p_{d+1} \in C$ such that
$$
V_i \; = \; \overline{p_1+p_2+ \dots + p_{i+1}}
$$
(An osculating flag is just the special case where all $p_i$ are equal.) Since any secant flag can specialize to an osculating flag, we can deduce the

\begin{corollary}\label{secant schubert proper}
Schubert cycles defined relative to \emph{general} secant flags to a rational normal curve intersect properly.
\end{corollary} 

Unlike the case for osculating flags, the hypothesis of generality is necessary for secant flags; see Exercise~\ref{only general secants}



\section{Exercises}
\begin{exercise}\label{inseparable Gauss}
Let $k$ be a field of characteristic $p$. Show that every point of the affine curve $y = x^{p+1}+1$ over $k$ is a flex point.
 
\end{exercise}

\begin{exercise}\label{2g+2fixedpoints}
Show that if $C$ is a smooth projective curve of genus $g \geq 2$ and $f : C \to C$ an automorphism fixing $2g+2$ points, then $C$ must be hyperelliptic and $f$ the hyperelliptic involution.

Hint: Since we know that $f$ has finite order, we can take the quotient $B = C/\langle f \rangle$ of $C$ by the cyclic group $\langle f \rangle$; apply Riemann-Hurwitz to the quotient map $C \to B$.
\end{exercise}


\begin{exercise}\label{84(g-1)}
Show that if $C$ is a smooth projective curve of genus $g \geq 2$ then 
$$
|Aut(C)| \leq 84(g-1)
$$

Hint: Since we know that $Aut(C)$ has finite order, we can take the quotient $B = C/Aut(C)$ of $C$; again, apply Riemann-Hurwitz to the quotient map $C \to B$. (Warning: the idea is the same as in Exercise~\ref{2g+2fixedpoints}, but the execution is substantially more complicated.)
\end{exercise}


\begin{exercise}\label{codim Schubert}
Show that the codimension of $\Sigma_{\bbeta}({\cal V})$ in the Grassmannian $G$ is equal to $\sum \beta_i$.

Hint: for $\Lambda \in \Sigma_{\bbeta}({\cal V})$, consider bases of $\Lambda$ such that $\Lambda \cap V_i$ is the span of basis vectors.
\end{exercise}

\begin{exercise}\label{Schubert duality}
 If ${\bbeta}^*$ is the transpose Schubert index to $\bbeta$,
\begin{enumerate}
\item  Show that $|{\bbeta}^*| = |{\bbeta}|$
\item Show that $({\bbeta}^*)^* = {\bbeta}$
\item Show that the isomorphism $\phi : G(k+1, V) \rTo^\cong G(n-k, V^*)$ carries the Schubert cycle $\Sigma_{\bbeta}({\cal V})$ to the Schubert cycle $\Sigma_{{\bbeta}^*}({\cal V}^*)$.
\end{enumerate}
(Of course, the first two parts follow from the third; think of the first two as warmups.)
\end{exercise}

\begin{exercise}(Ramification and osculating planes)\label{osculating planes}
Let $p\in C\subset \PP^{d}$ be a point on a rational normal curve of degree $d$, and
let $\Lambda\subset \PP^{d}$ be a $d-r-1$ plane, and let  $\sU :=(\sO_{\PP^{1}}(d), V)$
be
the  $\grd$,  possibly with base points, cut out by hyperplanes containing $\Lambda$. 
Consider the complete flag 
$$
U(p) := \{U_{1}, \dots, U_{d}\}
$$
of \emph{osculating spaces} at $p$, where $U_{t}$ is the linear span of the divisor $tp$ considered
as a subscheme of length $t$ in $C$. Show that the ramification sequence of $\sU$ at $p$
is determined by the formula
$$
W \in \Sigma_{\bf a}({U(p)}) \; \iff \; \alpha_i(\sU, p) \geq {\bf a}^*_{r+1-i} = \#\{j\mid a_j\geq r+1-i\}.
$$
In other words, the ramification sequence $\alpha(\sU, p)$ of the linear series $\sU$ at $p$ is the reverse of the transpose of the Schubert index of the smallest Schubert cycle containing $L$, the vector
space associated to $\Lambda$, in  $G(d-r, d+1)$. Conclude from Theorem~\ref{osculating intersection}
that any collection of Schubert cycles associated to osculating flags of $C$ at distinct points have intersection
that is dimensionally transverse or empty.
\end{exercise}

\begin{exercise}
Let $C$ be the rational normal curve in $\PP^{r}$. And let $(p_{i}, q_{i})\in  C^{2}$ be $t$ pairs of points, all distinct.
\marginpar{\redden{There were two exercises with the label `independent secants': this one and Exercise \ref{independent secants}. Since this label was getting overridden, I removed it, but you should check that none of the refs to  Exercise \ref{independent secants} should be to this one.}}
For each $i$, let $S_{i}$ be the Schubert variety of $k$-planes in $\PP^{r}$ that meet the secant line
$\overline{p_{i}q_{i}}$. Show that if the $p_{i}, q_{i}$ are sufficiently general, then the intersection
of the $S_{i}$ is dimensionally transverse. Hint: Let each pair of points come together, and use the result
for the tangent lines, a special case of Theorem~\ref{osculating intersection}.
\end{exercise}

\begin{exercise}\label{only general secants}
Show by example that Schubert cycles defined relative to arbitrary secant flags to a rational normal curve may fail to intersect properly. (Hint: it's enough to look at the case $d=2$ and $r=1$.)
\end{exercise}

\begin{exercise}
We can define the osculating flag to any nondegenerate curve $C \subset \PP^d$ at any smooth point $p$. If $p$ is an inflection point then $\overline{mp}$ may not be an $(m-1)$-plane, but we can look at the nested sequence of subspaces
$$
\{p\} \subset \overline{2p} \subset \overline{3p} \subset \dots 
$$
and pick out exactly the terms whose dimension is strictly greater than the preceding term; this gives a complete flag
$$
\{p\} \subset \overline{b_2p} \subset \overline{b_3p} \subset \dots \subset \overline{b_{d+1}p} = \PP^d
$$
Show that the sum $\sum b_i - i$ is equal to the weight of the point $p$ as an inflection point of the hyperplane series.
\end{exercise}


\begin{exercise}
The Pl\"ucker formula shows that an elliptic normal curve $E\subset \PP^{n-1}$ has $n^2$ inflection points. Show that they are
all simple, and that if you choose one as the origin for the group law on $E$, then 
then they are the $n$-torsion points of  $E \cong Jac(E)$.

Hint: observe that the inflection points are the points $p \in E$ such that $\cO_E(np) \cong \cO_E(1)$.
\end{exercise}

\begin{exercise}(Buchweitz)
It was long believed that every semigroup $H \subset \NN$ of finite index $g = \#(\NN \setminus H)$ could occur as the Weierstrass semigroup of a Weierstrass
point on a curve of genus $g$, but this is not the case. The following counterexample for a curve of genus 16 was found by Buchweitz (unpublished): Show that on a curve of genus 16 there can be no Weierstrass point $p$ with ramification sequence
$(0^{12}, 6,7,9,9)$ by showing that there would be too many vanishing orders at $p$ of quadratic differentials.
 \end{exercise}

%footer for separate chapter files

\ifx\whole\undefined
%\makeatletter\def\@biblabel#1{#1]}\makeatother
\makeatletter \def\@biblabel#1{\ignorespaces} \makeatother
\bibliographystyle{msribib}
\bibliography{slag}

%%%% EXPLANATIONS:

% f and n
% some authors have all works collected at the end

\begingroup
%\catcode`\^\active
%if ^ is followed by 
% 1:  print f, gobble the following ^ and the next character
% 0:  print n, gobble the following ^
% any other letter: normal subscript
%\makeatletter
%\def^#1{\ifx1#1f\expandafter\@gobbletwo\else
%        \ifx0#1n\expandafter\expandafter\expandafter\@gobble
%        \else\sp{#1}\fi\fi}
%\makeatother
\let\moreadhoc\relax
\def\indexintro{%An author's cited works appear at the end of the
%author's entry; for conventions
%see the List of Citations on page~\pageref{loc}.  
%\smallbreak\noindent
%The letter `f' after a page number indicates a figure, `n' a footnote.
}
\printindex[gen]
\endgroup % end of \catcode
%requires makeindex
\end{document}
\else
\fi
