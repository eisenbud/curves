%header and footer for separate chapter files

\ifx\whole\undefined
\documentclass[12pt, leqno]{book}
\usepackage{graphicx}
\input style-for-curves.sty
\usepackage{hyperref}
\usepackage{showkeys} %This shows the labels.
%\usepackage{SLAG,msribib,local}
%\usepackage{amsmath,amscd,amsthm,amssymb,amsxtra,latexsym,epsfig,epic,graphics}
%\usepackage[matrix,arrow,curve]{xy}
%\usepackage{graphicx}
%\usepackage{diagrams}
%
%%\usepackage{amsrefs}
%%%%%%%%%%%%%%%%%%%%%%%%%%%%%%%%%%%%%%%%%%
%%\textwidth16cm
%%\textheight20cm
%%\topmargin-2cm
%\oddsidemargin.8cm
%\evensidemargin1cm
%
%%%%%%Definitions
%\input preamble.tex
%\input style-for-curves.sty
%\def\TU{{\bf U}}
%\def\AA{{\mathbb A}}
%\def\BB{{\mathbb B}}
%\def\CC{{\mathbb C}}
%\def\QQ{{\mathbb Q}}
%\def\RR{{\mathbb R}}
%\def\facet{{\bf facet}}
%\def\image{{\rm image}}
%\def\cE{{\cal E}}
%\def\cF{{\cal F}}
%\def\cG{{\cal G}}
%\def\cH{{\cal H}}
%\def\cHom{{{\cal H}om}}
%\def\h{{\rm h}}
% \def\bs{{Boij-S\"oderberg{} }}
%
%\makeatletter
%\def\Ddots{\mathinner{\mkern1mu\raise\p@
%\vbox{\kern7\p@\hbox{.}}\mkern2mu
%\raise4\p@\hbox{.}\mkern2mu\raise7\p@\hbox{.}\mkern1mu}}
%\makeatother

%%
%\pagestyle{myheadings}

%\input style-for-curves.tex
%\documentclass{cambridge7A}
%\usepackage{hatcher_revised} 
%\usepackage{3264}
   
\errorcontextlines=1000
%\usepackage{makeidx}
\let\see\relax
\usepackage{makeidx}
\makeindex
% \index{word} in the doc; \index{variety!algebraic} gives variety, algebraic
% PUT a % after each \index{***}

\overfullrule=5pt
\catcode`\@\active
\def@{\mskip1.5mu} %produce a small space in math with an @

\title{Personalities of Curves}
\author{\copyright David Eisenbud and Joe Harris}
%%\includeonly{%
%0-intro,01-ChowRingDogma,02-FirstExamples,03-Grassmannians,04-GeneralGrassmannians
%,05-VectorBundlesAndChernClasses,06-LinesOnHypersurfaces,07-SingularElementsOfLinearSeries,
%08-ParameterSpaces,
%bib
%}

\date{\today}
%%\date{}
%\title{Curves}
%%{\normalsize ***Preliminary Version***}} 
%\author{David Eisenbud and Joe Harris }
%
%\begin{document}

\begin{document}
\maketitle

\pagenumbering{roman}
\setcounter{page}{5}
%\begin{5}
%\end{5}
\pagenumbering{arabic}
\tableofcontents
\fi


\fix{notation: the normalization should always be $C$, the other models $C_0$ or...}
\chapter{What linear series exist?}\label{Brill-Noether}\label{BNChapter}

If we only wish to know when a curve $C$ of genus $g$ has a $g^r_d$, then for $d>2g-2$ (or more generally nonspecial linear series) the Riemann-Roch Theorem gives a complete answer. Any special series must have $d\leq2g-2$, and in this case Clifford's Theorem gives the bound $r< d/2$, except in the case of hyperelliptic curves, where the answer is different but also completely understood. Together, these cases cover all possibilities:

\begin{theorem}\label{arbitrary linear series}
There exists a curve $C$ of genus $g$ and line bundle $\cL$ of degree $d$ on $C$ with $h^0(\cL) \geq r+1$ if and only if
$$
r \leq
\begin{cases}
d-g, \quad \text{if } d \geq 2g-1; \text{ and} \\
d/2,  \quad \text{if } 0 \leq d \leq 2g-2.
\end{cases}
$$
\end{theorem}


If we ask the---perhaps more interesting---question of when there can be a $g^r_d$ that is birationally
very ample, then Castelnuovo's theorem gives a bound. 

In both these situations, the curves that achieve the bounds are quite special. Perhaps the most interesting question of all is, for which $r,d$ do \emph{all} curves of genus $g$ have a $g^r_d$, and what is the
behavior of these series on a general curve? Brill-Noether theory provides some answers to both these questions.

\section{Brill-Noether theory}

\begin{theorem}[Basic Brill Noether]\label{basic BN}
A general curve $C$ of genus $g$  possesses a linear series of degree $d$ and dimension $r>d-g$ if and only if
$$
 \rho(g,r,d) := g - (r+1)(g-d+r) \geq 0.
$$
\end{theorem}

It is interesting to compare the values of $d,r$ that are possible on special and general curves, and see how many fewer are possible for birationally very ample series and for general curves. The graphs in the following table compare the results of 
Clifford's theorem, Castelnuovo's theorem, and the Brill-Noether theorem applied to curves
of genus 100:

\centerline{ \includegraphics[height=4in]{"Clifford-Castelnuovo-Brill-Noether"}}

In the following sections, we'll see why we might naively expect Theorem~\ref{basic BN} to be true, and we'll also state some refinements of the theorem.   In Chapter~\ref{InflectionsChapter} we give a proof of the nonexistence part (the ``only if"). A proof of the existence half of the theorem (the ``if" part of the statement) may be found in \cite[Theorem ****]{3264}.

The case $r=1$ is already interesting:

\begin{corollary}
If $C$ is any curve of genus $g$, then $C$ admits a map of degree $d$ to $\PP^1$ for some $d \leq \lceil \frac{g+2}{2}\rceil$.
\end{corollary}

Thus any curve of genus 2 is hyperelliptic, any curve of genus 3 or 4 is either hyperelliptic or trigonal  (admits a 3-1 map to $\PP^1$), and so on.


\subsection{Heuristic argument leading to the statement of Brill-Noether}

The proof of the Brill-Noether theorem starts with a simple dimension estimate that was first carried out by Brill and Noether almost a century and a half ago. \fix{let's say more about the history here, or at the end of this section}

Let $C$ be a smooth projective curve of genus $g$, and $D = p_1 + \dots + p_d$ a divisor on $C$. We'll assume here the points $p_i$ are distinct; the same argument (with  more complicated notation) can be carried out in general.

When does the divisor $D$ move in an $r$-dimensional linear series? Riemann-Roch gives an answer: it says that $h^0(D) \geq r+1$ if and only if the vector space $H^0(K-D)$ of 1-forms vanishing on $D$ has dimension at least $g-d+r$---that is, if and only if the  evaluation map
$$
H^0(K) \to H^0(K|_D) = \bigoplus k_{p_i}
$$
has rank at most $d-r$. 

We can represent this map by a $g \times d$ matrix. Choose a basis $\omega_1,\dots,\omega_g$ for the space $H^0(K)$ of 1-forms on $C$; choose an analytic open neighborhood $U_j$ of each point $p_j \in D$ and choose a local coordinate $z_j$ in $U_j$ around each point $p_j$, and write
$$
\omega_i = f_{i,j}(z_j)dz_j
$$
in $U_j$. We will have $r(D) \geq r$ if and only if the  matrix-valued function
$$
A(z_1,\dots,z_d) = 
\begin{pmatrix}
f_{1,1}(z_1) & f_{2,1}(z_1) & \dots & f_{g,1}(z_1) \\
f_{1,2}(z_2) & f_{2,2}(z_2) & \dots & f_{g,2}(z_2) \\
\vdots & \vdots &  & \vdots \\
f_{1,d}(z_d) & f_{2,d}(z_d) & \dots & f_{g,d} (z_d)
\end{pmatrix}
$$
has rank $d-r$ or less at $(z_1,\dots,z_d) = (0,\dots,0)$.

The point is, we can think of $A$ as a matrix valued function in an open set $U = U_1 \times U_2 \times \dots \times U_d \subset C_d$; and for divisors $D \in U$, we have $r(D) \geq r$ if and only if $\rank(A(D)) \leq d-r$. Now, in the space $M_{d,g}$ of $d \times g$ matrices, the subset of matrices of rank $d-r$ or less has codimension $r(g-d+r)$ (\cite[Theorem ****]{Eisenbud1995},  so if the map $A$ is dimensionally transverse to this degeneracy locus in $M_{d,g}$, we could deduce that the locus of divisors with $r(D) \geq r$ has dimension $d - r(g-d+r)$, and such divisors would exist only if
$$
d - r(g-d+r) \; \geq \; r.
$$
At the same time, if any divisor of degree $d$ with $h^0(D) \geq r+1$ exists, then there must be at least an $r$-dimensional family of them; this is exactly the statement of the Brill-Noether Theorem.


\subsection{Refinements of the Brill-Noether theorem}

As we indicated, Theorem~\ref{basic BN} is just a starting point. It raises a slew of questions, both about the geometry of the schemes $W^r_d(C)$ parametrizing linear series on a general curve $C$ (what are their dimensions, are they irreducible, etc.), and about the geometry of the linear systems themselves (do they give embeddings; what's the Hilbert function of the image, and so on). A great deal of progress has been made on these fronts. Here are statements of some of the main results:

\begin{theorem}[Brill-Noether theorem, omnibus version]\label{BN omnibus}
Let $C$ be a general curve of genus $g$. If we set $\rho = g - (r+1)(g-d+r)$, then
\begin{enumerate}
\item $\dim(W^r_d(C)) = \rho$;
\item\label{sing wrd} the singular locus of $W^r_d(C)$ is exactly $W^{r+1}_d(C)$;
\item\label{irr wrd} if $\rho > 0$ then $W^r_d(C)$ is irreducible;
\item\label{rho=0} if $\rho = 0$ then $W^r_d$ consists of a finite set of  points of cardinality
$$
\#W^r_d = g! \prod_{\alpha=0}^r \frac{\alpha!}{(g-d+r+\alpha)!};
$$
\item\label{Petri} if $L$ is any invertible sheaf on $C$, the map
$$
\mu : H^0(L) \otimes H^0(\omega_CL^{-1}) \rTo H^0(\omega_C)
$$
is injective, and the Zariski tangent space to the scheme $W^r_d(C)$ at the point $L$, as a subspace
of $H^0(\omega_C)^*$, is the annihilator of the image of $\mu$;
\item\label{general va} if $|D|$ is a general $g^r_d$ on $C$, then
\begin{enumerate}
\item if $r \geq 3$ then $D$ is very ample; that is, the map $\phi_D : C \to \PP^r$   embeds $C$ in $\PP^r$;
\item if $r=2$ the map $\phi_D : C \to \PP^2$ gives a birational embedding of $C$ as a nodal plane curve; and 
\item if $r=1$, the map $\phi_D : C \to \PP^2$ expresses $C$ as a simply branched cover of $\PP^1$.
\end{enumerate}

\item\label{maximal rank}[The maximal rank theorem] If $L \in W^r_d(C)$ is a general point (or any point, if $\rho = 0$), then for each $m > 0$ the multiplication map
$$
\rho_m : \Sym^m H^0(L) \to H^0(L^m)
$$
has maximal rank; that is, it is either injective or surjective.
\item\label{interpolation}[The interpolation theorem]
Let $d, g$ and $r$
be nonnegative integers with $\rho(d, g, r) \geq 0$. There is a general curve of degree $d$ and genus $g$ through $n$ general
points in $\PP^r$
if and only if
$$
(r-1)n \leq (r + 1)d-(r-3)(g-1)
$$
except in the four exceptional cases $(d, g, r) = (5, 2, 3),(6, 4, 3),(7, 2, 5)$ or $(10, 6, 5)$.
\end{enumerate}

\fix{add references, attributions, etc.}

\end{theorem}

A few special cases are worth noting:
\begin{enumerate}

\item As a special case of Part~\ref{rho=0}, we see that the number of $g^1_{k+1}$s on a general curve of genus $g = 2k$ is the $k$th Catalan number 
$$
c_k = \frac{2k!}{k!(k+1)!}.
$$
We have already seen this in the first case: in genus 2, it says the canonical series $|K|$ is the unique $g^1_2$ on a curve of genus 2. In the next chapter, we'll also see how to verify this directly for curves of genus 4---where it says that there are exactly two $g^1_3$s on a general curve of genus 4---and in genus 6, where it says that a general curve of genus 6 has 5 $g^1_4$s.  In genus 8, it says that a general curve of genus 8 has 14 $g^1_5$s, but we don't know of any way of seeing this directly from the geometry of a general curve of genus 8; and we know even less for larger $k$.

\item Part~\ref{Petri} implies Part~\ref{sing wrd}. In fact, a fairly elementary argument shows that at a point $L \in W^r_d(C) \setminus W^{r+1}_d(C)$, the tangent space to $W^r_d$ at the point $L$ is the annihilator
in $(H^0(\omega_C))^*$ of the image of $\mu$; given that $\mu$ is injective, we can compare dimensions and deduce that $W^r_d$ is smooth at $L$.

\item Part~\ref{maximal rank} is the celebrated \emph{maximal rank theorem} of Eric Larson. It answers in general a question that has come up multiple times so far in this book: every time we've asked what hypersurfaces contain a curve $C \subset \PP^r$ embedded by a linear system $|L|$, we've looked at the maps $\rho_m$. Each time, we knew the dimensions of the spaces $\Sym^m H^0(L)$ and $H^0(L^m)$, and the question was the rank of $\rho_m$; now we know the answer for a general linear system of any degree and dimension.Thus the maximal rank theorem tells us the Hilbert function of a general curve $C \subset \PP^r$ embedded by a general linear system.

\item There is a possible extension of the maximal rank theorem of Part~\ref{maximal rank}. If $C \subset \PP^r$ is a general curve embedded by a general linear series, the maximal rank theorem tells us the dimension of the $m$th graded piece of the ideal of $C$, for any $m$: this is just the dimension of the kernel of $\rho_m$. But it doesn't tell us what a minimal set of generators for the homogeneous ideal of $C$ might look like. For example, if $m_0$ is the smallest $m$ for which $I(C)_m \neq 0$, or numerically the smallest $m$ such that $\binom{m+r}{r} > md-g+1$, we can ask: is the homogeneous ideal $I(C)$ generated by $I(C)_{m_0}$?

To answer this question---given that we know the dimensions of $I(C)_m$ for every $m$---we would need to know the ranks of the multiplication maps
$$
\sigma_m : I(C)_m \otimes H^0(\cO_{\PP^r}(1)) \to I(C)_{m+1}
$$
for each $m$. In particular, we may conjecture that \emph{the maps $\sigma$ have maximal rank}; if this were true we could deduce the degrees of a minimal set of generators for the homogeneous ideal $I(C)$.

\item There is another object worth mentioning: for any curve $C$, there exists a scheme $G^r_d(C)$ parametrizing linear series of degree $d$ and dimension $r$; that is, in set-theoretic terms,
$$
G^r_d = \left\{ (L, V) \mid L \in Pic^d(C), \text{ and } V \subset H^0(L) \text{ with } \dim V = r+1 \right\}.
$$

(This requires a little more technical machinery to construct, which is why we haven't introduced it.) $G^r_d(C)$ maps to $W^r_d(C)$; the map is a birational isomorphism, being an isomorphism over the open subset $W^r_d(C) \setminus W^{r+1}_d(C)$ and having positive-dimensional fibers over $W^{r+1}_d(C)$. In fact, another version of Part~\ref{sing wrd} is the statement that \emph{for a general curve $C$, the scheme $G^r_d(C)$ is smooth for any $d$ and $r$}.

\item Parts~\ref{general va} and~\ref{maximal rank} describe the geometry of a general curve $C$ as embedded in projective space by a general linear series. But there are many remaining questions! One is the question of secant planes: a naive dimension count would suggest that an irreducible, nondegenerate curve $C \subset \PP^3$ will have a finite number of 4-secant lines, but no 5-secant lines. Is this true for a general curve embedded in $\PP^3$ by a general linear series?
\end{enumerate}

\section{Exercises}
\begin{exercise}
Prove a slightly stronger version of Theorem~\ref{arbitrary linear series} in the range $d \leq g-1$: that under the hypotheses of Theorem~\ref{arbitrary linear series} there exists a \emph{complete} linear series of degree $d$ and dimension $r$ for any $r \leq d/2$.
\end{exercise}

\begin{exercise}\label{rarity of Castelnuovo}
We have seen that complete intersections $C = Q \cap S \subset \PP^3$ of a quadric surface $Q$ and a surface $S$ of degree $k$ achieve Castelnuovo's bound $g = \pi(2k, 3)$ on the genus of curves of degree $2k$ in $\PP^3$. In fact, we will see in Chapter~\ref{ScrollsChapter} that any curve $C \subset \PP^3$ of degree $2k$ and genus $g = \pi(2k, 3) = (k-1)^2$ is of this form.
\begin{enumerate}
\item Find the dimension of the subvariety $\Gamma \subset M_g$ consisting of Castelnuovo curves.
\item Find the dimension of the subvariety $H \subset M_g$ of hyperelliptic curves, and compare this to the result of the first part.
\end{enumerate}
\end{exercise}

\input footer.tex