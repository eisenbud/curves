%header and footer for separate chapter files

\ifx\whole\undefined
\documentclass[12pt, leqno]{book}
\usepackage{graphicx}
\input style-for-curves.sty
\usepackage{hyperref}
\usepackage{showkeys} %This shows the labels.
%\usepackage{SLAG,msribib,local}
%\usepackage{amsmath,amscd,amsthm,amssymb,amsxtra,latexsym,epsfig,epic,graphics}
%\usepackage[matrix,arrow,curve]{xy}
%\usepackage{graphicx}
%\usepackage{diagrams}
%
%%\usepackage{amsrefs}
%%%%%%%%%%%%%%%%%%%%%%%%%%%%%%%%%%%%%%%%%%
%%\textwidth16cm
%%\textheight20cm
%%\topmargin-2cm
%\oddsidemargin.8cm
%\evensidemargin1cm
%
%%%%%%Definitions
%\input preamble.tex
%\input style-for-curves.sty
%\def\TU{{\bf U}}
%\def\AA{{\mathbb A}}
%\def\BB{{\mathbb B}}
%\def\CC{{\mathbb C}}
%\def\QQ{{\mathbb Q}}
%\def\RR{{\mathbb R}}
%\def\facet{{\bf facet}}
%\def\image{{\rm image}}
%\def\cE{{\cal E}}
%\def\cF{{\cal F}}
%\def\cG{{\cal G}}
%\def\cH{{\cal H}}
%\def\cHom{{{\cal H}om}}
%\def\h{{\rm h}}
% \def\bs{{Boij-S\"oderberg{} }}
%
%\makeatletter
%\def\Ddots{\mathinner{\mkern1mu\raise\p@
%\vbox{\kern7\p@\hbox{.}}\mkern2mu
%\raise4\p@\hbox{.}\mkern2mu\raise7\p@\hbox{.}\mkern1mu}}
%\makeatother

%%
%\pagestyle{myheadings}

%\input style-for-curves.tex
%\documentclass{cambridge7A}
%\usepackage{hatcher_revised} 
%\usepackage{3264}
   
\errorcontextlines=1000
%\usepackage{makeidx}
\let\see\relax
\usepackage{makeidx}
\makeindex
% \index{word} in the doc; \index{variety!algebraic} gives variety, algebraic
% PUT a % after each \index{***}

\overfullrule=5pt
\catcode`\@\active
\def@{\mskip1.5mu} %produce a small space in math with an @

\title{Personalities of Curves}
\author{\copyright David Eisenbud and Joe Harris}
%%\includeonly{%
%0-intro,01-ChowRingDogma,02-FirstExamples,03-Grassmannians,04-GeneralGrassmannians
%,05-VectorBundlesAndChernClasses,06-LinesOnHypersurfaces,07-SingularElementsOfLinearSeries,
%08-ParameterSpaces,
%bib
%}

\date{\today}
%%\date{}
%\title{Curves}
%%{\normalsize ***Preliminary Version***}} 
%\author{David Eisenbud and Joe Harris }
%
%\begin{document}

\begin{document}
\maketitle

\pagenumbering{roman}
\setcounter{page}{5}
%\begin{5}
%\end{5}
\pagenumbering{arabic}
\tableofcontents
\fi


\chapter{Quadrics and Ribbons}
\label{Quadrics and Ribbons}

\begin{question}
 Which quadrics containing a rational normal curve contain a ribbon? ie are limits of quadrics containing nearby canonical curves?
\end{question}

Genus 4: the answer is the singular quadrics.


Genus 5: we don't know.

Since we don't know, we could instead about the locus of singular quadrics containing a rational normal curve of degree $d$.

For $d=3$, This is a double plane conic.

\def\cS{{\mathcal S}}
\begin{theorem}
 If $d=4$ then the space of singular quadrics in the variety
 $$
 Q :=\PP^5 = \PP(k^6) = \PP(\wedge^2 \cS_3(k^2)) = \PP(\cS_2(\cS_2(k^2)))
 $$
 of quadrics containing the rational normal curve $C$ of degree 4 is reducible: its components are a nonsingular quadric $X_0$ and a singular cubic $X_1$. The elements of $X_1$ correspond to quadrics with vertex on the secant variety of $C$, while the elements of $X_0\setminus X_1$ correspond to singular quadrics whose vertex is not on the secant locus.
 
 Moreover, $X_1$ is the secant locus of the "standard" Veronese surface in $\PP(\cS_2\cS_2(k^2)))$.
\end{theorem}

\begin{proof}
Let $L\subset Q$ be a general pencil of quadrics containing $C$. The base locus of $L$ is a del Pezzo surface $S$, the blow up of
$\PP^2$ at 5 points. We have
$$
\Pic(S) = \ZZ\langle \zeta, e_1,\dots, e_5\rangle
$$
where $\zeta$ is the pullback of a line from $\PP^2$ and the $e_i$ are the preimages of the blown up points. We may assume
that $[C]\sim 2\zeta-e_1-e_2$. There are 5 singular quadrics containing $S$, and each one corresponds to an equation
$A\otimes B = \cO_S(1)$ where $A,B$ are line bundles with two sections each. The five solutions may be written
$$
A \sim \zeta-e_i,\qquad B \sim 2\zeta - \sum_{j\in J}e_j
$$
where $J = \{1,\dots,5\} \setminus \{i\} \subset \{1,\dots,5\}$ has 4 elements. 

A section of $A$ or $B$ corresponds to a conic on $S$, and $A\cdot B = 2$.
The vertex of the singular quadric corresponding to such a pair $A,B$ is the intersection of all the lines joining the pair of points of intersection
of an element of $|A|$ and an element of $|B|$. This vertex lies on the secant locus of $C$ iff there are points $p_1,p_2$ of $C$ that both lie on some element of $|A|$ and on some element of the pencil $|B|$. 

If $i = 1$ or 2, the restriction to $C$ of the pencils $A$ and $B$ have degrees 1 and 3 respectively, and this cannot be the case. In contract, if $i = 3, 4$ or 5, the pencils $A$ and $B$ each have degree $2$ on $C$. It follows from the genus formula that the map 
$$
\phi_A \times \phi_B : C \to \PP^1 \times \PP^1
$$
has a double point; that is, there are a pair of points as above on $C$.

Macaulay 2 assures us that the singular locus of $X_1$ is two-dimensional (and non-degenerate). Given this, note that $X_1$, being cubic, contains the secant variety of its singular locus; but the only nondegenerate surface in $\PP^5$ whose secant variety is not all of $\PP^5$ is the Veronese surface. Thus $(X_1)_{sing}$ is a Veronese surface, and $X_1$ the secant variety of that Veronese surface.
\end{proof}

One thing we seem to need for this is the fact that the lines joining the pair of points of intersection of a curve from the pencil $|A|$ and a curve from the pencil $|B|$ are \emph{all} the secant lines to $S$ passing through the vertex of the corresponding quadric

\begin{proposition}
 If $R$ is a canonical ribbon supported on $C$, then $R$ lies on a net of quadrics; and the discriminant locus of this net is the plane quintic curve consisting of the union of two conics tangent at two points plus the line joining the two points. In particular, the $\PP^2 \subset Q$ of quadrics containing $R$ intersects $X_1$ in a reducible curve.
\end{proposition}


%footer for separate chapter files

\ifx\whole\undefined
%\makeatletter\def\@biblabel#1{#1]}\makeatother
\makeatletter \def\@biblabel#1{\ignorespaces} \makeatother
\bibliographystyle{msribib}
\bibliography{slag}

%%%% EXPLANATIONS:

% f and n
% some authors have all works collected at the end

\begingroup
%\catcode`\^\active
%if ^ is followed by 
% 1:  print f, gobble the following ^ and the next character
% 0:  print n, gobble the following ^
% any other letter: normal subscript
%\makeatletter
%\def^#1{\ifx1#1f\expandafter\@gobbletwo\else
%        \ifx0#1n\expandafter\expandafter\expandafter\@gobble
%        \else\sp{#1}\fi\fi}
%\makeatother
\let\moreadhoc\relax
\def\indexintro{%An author's cited works appear at the end of the
%author's entry; for conventions
%see the List of Citations on page~\pageref{loc}.  
%\smallbreak\noindent
%The letter `f' after a page number indicates a figure, `n' a footnote.
}
\printindex[gen]
\endgroup % end of \catcode
%requires makeindex
\end{document}
\else
\fi
