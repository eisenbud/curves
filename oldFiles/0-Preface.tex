\input header.tex

\setcounter{chapter} {-1}
\chapter{Basic Questions}

\fix{The following is more material for a preface than a preface...}
\begin{center}
\emph{I'm very well acquainted, too, with matters mathematical,\\
I understand equations, both the simple and quadratical,\\
About binomial theorem I am teeming with a lot o' news,\\
With many cheerful facts about the square of the hypotenuse.}

---Gilbert and Sullivan, Pirates of Penzance, Major General's Song


\emph{Be simple by being concrete. Listeners are prepared to
accept unstated (but hinted) generalizations much more than they are able, on the spur of the moment, to
decode a precisely stated abstraction and to re-invent the special cases that motivated it in the first place. }

--Paul Halmos, How to Talk Mathematics

\emph{Another damned thick book! Always scribble, scribble, scribble! Eh, Mr. Gibbon?} --- \scriptsize{Prince William Henry, upon receiving the second  volume of The History of the Decline and Fall of the Roman Empire from the author.}
\end{center}


The most primitive objects of algebraic geometry are affine algebraic sets---subsets of $\RR^{n}$ or $\CC^{n}$ defined by the vanishing of polynomial functions---and the maps between them. But already in the first half of the 19th century geometers realized that there was a great advantage in working with varieties in complex projective space, treating affine varieties as projective varieties minus the intersection with the plane at infinity and real varieties as the fixed points of the complex involution. One sees this in the simplest examples: the ellipses, hyperbolas and parabolas in the real affine plane are all the same in the complex projective plane; the difference is only in how they intersect the line at infinity. A difficulty with the projective point of view is that on a connected projective variety there are no non-constant functions at all (reason: a function on a projective variety is a map to the affine line; since the image of a projective variety is again projective, the image would be a single point.) 

Starting with Riemann in the 1860s and culminating in the scheme theory of Grothendieck in the 1950s, algebraic varieties were treated in a way independent of any embedding: An algebraic variety is is a topological space with a sheaf of locally
defined polynomial functions. Many interesting aspects of geometry have to do not with single abstract varieties, but with maps between them, and in particular with embeddings in projective spaces. In general, maps between varieties can be described by their graphs, which are again varieties.  But for the special case of maps to projective spaces, the theory of \emph{linear series} is usually a more convenient description. The collection of all linear series on a variety reflects some of its best understood invariants. 

The basic objects of study in this book are smooth, connected projective algebraic curves over an algebraically closed field of characteristic 0, which we take to be the complex numbers $\CC$. Though we assume that the reader has been exposed to this theory in some form before, perhaps from Chapter IV of Hartshorne's {\it Algebraic Geometry}, we will review the elements  in the form we will use. 

\section{Algebraic Curves and Riemann Surfaces}

These objects can be viewed in two distinct but equivalent ways: as \emph{compact Riemann surfaces}, or compact complex manifolds of dimension 1; and as \emph{smooth projective algebraic curves over $\CC$}. (Here, when we use the term projective variety, we mean a variety isomorphic to a closed subset of projective space, not a variety with a specified embedding in $\PP^n$.) There are advantages to each point of view---the complex analytic point of view is more concrete, and requires relatively minimal amount of preliminaries; the algebraic point of view is substantially broader. 

First, if $C \subset \PP^n$ is a smooth, projective curve over $\CC$, then it is a submanifold of complex projective space, and so a Riemann surface. 
\fix{discuss geometric genus vs. arithmetic genus here?}

The other direction---going from a compact Riemann surface $C$ to a smooth projective curve over $\CC$, or equivalently embedding $C$ as a complex submanifold of $\PP^n$, after which Chow's theorem says that it is in fact a projective variety---is much deeper. The first, and hardest step is to show that a compact Riemann surface admits a nonconstant meromorphic function $f:C \to \CC$, and the corresponding statement is not true in higher dimensions. The function $f$ can be viewed as
a rational map $f': C\to \PP^{1}$. The next step is to see that the field $K(C)$ of all meromorphic functions
on $\CC$ is a finite extension of the field of rational functions on $\PP^{1}$; the sheaf of regular functions on $C$ is then the integral closure of the sheaf of regular functions on $\PP^{1}$ in $K(C)$.

Though equivalent for curves defined over $\CC$, these approaches have a very different flavors. For example,
given a map  $f: C\to C'$ from a smooth curve $C$ to a possibly singular curve $C'$ that is generically one-to-one, we can reconstruct $C$.
From the algebraic point of view this can be done by \emph{normalization}, or more concretely by blowing up the singular points of $C'$. From the analytic point of view, we can use the Weierstrass preparation theorem, which implies that there is the neighborhood $U$ of any point $p\in C'$ such that the punctured neighborhood $U \setminus p$ is isomorphic to a disjoint union of punctured discs; and $C$ is obtained by completing this to the corresponding disjoint union of discs.

\fix{

\section{Basics to add}
 \begin{enumerate}

\item Background \begin{enumerate}

\item Lasker's Theorem (complete intersections are unmixed) -- sometimes incorrectly called ``$AF+BG$''
state ci implies unmixed; prove by $H^1$ of line bundle.

\item B\'ezout and the  weak B\'ezout (ex 8.4.6 in Fulton).
B\'ezout via Koszul complex, at least in codim 2.

\item meaning of ``number of conditions imposed"

\end{enumerate}

\begin{thebibliography}{ABC99}

%\bibitem[1965]{BR} D. Buchsbaum and D.S.Rim.
%A generalized Koszul complex. III. A remark on generic acyclicity.
%Proc. Amer. Math. Soc. 16 (1965) 555--558. 

\bibitem[Walker]{Walker} Walker.
\bibitem[Hartshorne]{Hartshorne} Hartshorne Ch 4
\bibitem[Fulton]{Fulton} Fulton Alg curves
\bibitem[Schemes]{Eisenbud-Harris} Schemes
\bibitem[3264]{Eisenbud-Harris} 3264
\bibitem[Griffiths-Harris]{Griffiths-Harris} (for the Abel-Jacobi stuff)
\bibitem[Griffiths]{Griffiths-Chinese}(for the Abel-Jacobi stuff)
\bibitem[Mumford]{Mumford}-Curves and their Jacobians
\bibitem[Voisin]{Voisin} Hodge Theory
%\bibitem[]{Smith} Smith, K.

\end{thebibliography}
\bigskip

\vbox{\noindent Author Addresses:\par
\smallskip
\noindent{David Eisenbud}\par
\noindent{Department of Mathematics, University of California, Berkeley,
Berkeley CA 94720}\par
\noindent{eisenbud@math.berkeley.edu}\par
}

\input footer.tex