 \input header.tex
 
 \chapter{Material extracted from MacaulayBook.tex by Eisenbud and Gray}
 
 For a long time the emphasis was on the properties of real curves in the usual plane. However, by the 1850s there was a growing appreciation of the projective plane as an object of study. First Ferdinand M\"obius\index{M\"obius, August Ferdinand (1790--1868)} (1827) and then Julius Pl\"ucker\index{Pl\"ucker, Julius (1801--1868)} (1830) introduced systems of homogenous coordinates, which were then more systematically used by Otto Hesse\index{Hesse, Otto (1811--1874)}, in e.g. his (1850). But it was only in the 1850s and 1860s, with responses to the work of Riemann\index{Riemann, Georg Friedrich Bernhard (1826--1866)}, that plane algebraic geometry was explicitly  regarded as  the study of curves in the complex plane $\CC^2$, and the space of homogeneous triples  became a complex object that we can fairly interpret as $\PP_{\CC}^2.$\footnote{Unsurprisingly, not all of this older work meets modern standards. For a modern account of  much of the material described in this chapter, readers are referred to (Brieskorn and Kn\"orrer 1986). A historically informative account of many of these developments, if sometimes not entirely rigorous, can be found in (Coolidge 1940) with additional mathematical details in (Coolidge 1931).}

\bigbreak\hrule\bigbreak

Singular points on a plane algebraic curve played an important part in the theory of dual curves that was developed by Pl\"ucker\index{Pl\"ucker, Julius (1801--1868)}  in his (1835). 
What we would today call Pl\"ucker Theory concerns the relationship between the characteristics of a curve $C$ defined by a homogeneous equation $F(x,y,z) = 0$ and its 
``dual'' curve $C^{*}.$

\bigbreak\hrule\bigbreak

More precisely, Pl\"ucker in his \emph{Theorie der algebraischen Curven} (1839, Chapter 4) gave a fairly complete analysis of the singularities of curves of low degree by looking at their duals. He considered a curve of degree $m$ and class (the order of the dual curve) $n$ that has $\delta$ double points, $\kappa$ cusps, $\tau$ bitangents, and $\iota$ inflections. Corresponding to each double point the dual curve has a bitangent, and to each cusp there corresponds an inflection, and vice versa. More precisely, he gave an argument to show that
\[n = m(m-1) - 2\delta - 3 \kappa,\]
\[\iota = 3m(m-2) - 6\delta - 8\kappa,\]
and another formula for $\tau$, along with the corresponding formula for the dual curve and its inflection points and bitangents.

\bigbreak\hrule\bigbreak


The number $\iota$ is most easily found using an argument due to Hesse in his (1850), where he showed that the set of flexes $\Gamma \subset G$ on a curve $G$ of degree $d$ with equation $F(x,y,z) = 0$ could be characterized as the intersection of $G$ with the curve $D$ defined by the vanishing of the Hessian.

\bigbreak\hrule\bigbreak

To an extent, this was provided by Cayley\index{Cayley, Arthur (1821--1895)} in his (1866), where he addressed the question of how to extend Pl\"ucker's equations to curves with higher singularities.  He observed that Pl\"ucker himself had found some more equations, but in Cayley\index{Cayley, Arthur (1821--1895)}'s opinion the theory was ``open to a grave objection''.\footnote{See  (Cayley 1866,  522).} Cayley\index{Cayley, Arthur (1821--1895)} assumed that any singular curve is equivalent to one with a certain number $\delta'$ of double points, $\kappa'$ of cusps, $\tau'$ of bitangents, and $\iota'$ of inflexions, and set himself the task of determining these numbers. 

The work just discussed raised the question: what does an arbitrary singular point on a plane curve look like?  In the 19th century this was tackled with varying degrees of rigour using a theory of Cremona transformations, chiefly in the special case of quadratic transformations.\footnote{In modern terms, a Cremona transformation
is any birational automorphism of the projective plane, that is, any automorphism of the field $\CC(x,y)$ fixing the subfield $\CC$.}
The quadratic transformation is the special case defined by the three quadrics through three noncolinear points. Here we give a brief indication of how and why they were introduced by Luigi Cremona\index{Cremona, Luigi (1830--1903)} and used by Noether\index{Noether, Max (1844--1921)} and others. 

\bigbreak\hrule\bigbreak


Luigi Cremona\index{Cremona, Luigi (1830--1903)} outlined a general theory of transformations of the plane in his papers (1863) and (1865). His aim was to find the general transformation of the plane that maps a given figure one-to-one onto the image figure, and reciprocally. The transformation should be geometrical, and his arguments involved transformations that map straight lines to curves of order $n$, for some $n.$  For example, as he showed, if $n=2$ (the case of quadratic transformations) almost all straight lines must have images that are conics passing through 3 fixed points. Such examples had been studied before, Cremona\index{Cremona, Luigi (1830--1903)} cited papers by Steiner, Magnus, and Schiaparelli, but for higher values of $n$ everything was new.

Noether\index{Noether, Max (1844--1921)} sought to give a more rigorous account in his (1876). He began by saying that the theory of singular points of a plane algebraic curve had been treated algebraically in (Puiseux 1850) and geometrically, and somewhat hypothetically,  in (Cayley 1866). But there was still a need, he said, for a general analytic method, for a good definition of a singular point, for a derivation of Cayley\index{Cayley, Arthur (1821--1895)}'s extended Pl\"ucker formulae from such a definition, etc.\footnote{Noether\index{Noether, Max (1844--1921)} cited several other authors in the fast-moving field, among them   (Hamburger 1871) and (Koenigsberger 1874).}  The `etc.' surely included the necessity of redefining the genus of a plane algebraic curve, and proving the invariance of the genus under birational  transformations.  A proof that the genus of a plane curve is invariant under birational transformations had been offered by Clebsch\index{Clebsch, Rudolf Friedrich Alfred (1833--1872)} and Gordan in their (1866, Chapter 3), but only for the case of a curve with simple cusps and double points. 
\bigbreak\hrule\bigbreak


The idea that algebraic curves of degrees $k$ and $m$ in the plane should meet in $km$ points was known to  Euler\index{Euler, Leonhard (1797--1783)}, who observed in the second volume of his \emph{Introductio in Analysin Infinitorum} that for it to be true it would be necessary to take care of multiple points (such as tangents) and points `at infinity', as well as allowing coordinates of intersection points to be complex. Both he and Cramer\index{Cramer, Gabriel (1704--1752)} made attempts at proving this claim, but the first proof to command adequate assent was given by \'Etienne B\'ezout in 1779, and the result has been called B\'ezout's theorem ever since.\footnote{See (B\'ezout 1779).} One of the themes of this book is that to state the theorem in general requires a notion of the multiplicity of an intersection of two plane curves without common components. Such a notion was developed by Max Noether\index{Noether, Max (1844--1921)}, and refined by Macaulay. A modern proof would invoke Noether's Fundamental Theorem\index{Noether's Fundamental Theorem} --- see below --- and a computation of the Hilbert polynomial. This is a story that  has been pursued, with successive new generalisations, right up to the present day. 

\bigbreak\hrule\bigbreak

The idea that a plane cubic curve should be determined by nine points in the plane, because nine equations are needed to determine the nine ratios between the ten coefficients that enter its equation, was stated by  Euler\index{Euler, Leonhard (1797--1783)} in the second volume of his \emph{Introductio} (1748)  but it was surely known before. He also said that the same is true for curves of degree $n$, which are determined by $ n(n+3)$ general points.
\bigbreak\hrule\bigbreak

 Euler\index{Euler, Leonhard (1797--1783)} also knew that there are problems, however, and this claim can sometimes be false, because Gabriel Cramer\index{Cramer, Gabriel (1704--1752)} had written to him in 1744 to point out that any two cubics will meet in nine points, and so these nine points do not determine a cubic, and indeed there will be infinitely many cubics through these nine points.\footnote{Cramer\index{Cramer, Gabriel (1704--1752)} also knew that MacLaurin had been aware of this problem in 1720, and had not been able to solve it.} This apparent contradiction with the claim that nine points in the plane always determine a unique cubic is today known as Cramer's paradox. Cramer\index{Cramer, Gabriel (1704--1752)} also raised the problem in his book (1748, Chapter 3) and suggested that the resolution was that  nine points determine a cubic unless they are nine points common to two cubics, but he had no proof of the claim. He did, however, also suggest that if the $ (n+1)(n+2)$ points needed the determine a curve of degree $n$ contain $tn$ common to a curve of degree $t$ then the curve through the  $ (n+1)(n+2)$ breaks up into two curves, one of which passes through the $tn$ points.

 Euler\index{Euler, Leonhard (1797--1783)} went on to make modest progress in resolving the paradox in his (1750). He argued that each time a point of the plane is said to lie on a given cubic curve it imposes a linear equation on the coefficients of the curve. In the simplest case, suppose that eight points in the plane are given that impose eight independent conditions on the cubic. If these eight points are eight points common to the given cubic and another, then the ninth point common to those cubics cannot impose a new condition on the coefficients, and so the equation it imposes on the coefficients must be a consequence of the first eight equations. Or, rather, because  Euler\index{Euler, Leonhard (1797--1783)} lived before the language of linear dependence was available, he thought that the ninth equation might be identical with one of the previous eight. He was only able to resolve the problem explicitly for those cases when a five points do not determine a conic, which happens when four lie on a line.

As we shall see, it follows immediately from Cramer\index{Cramer, Gabriel (1704--1752)}'s and   Euler\index{Euler, Leonhard (1797--1783)}'s observation that given eight points common to two cubic curves there is a unique ninth that they all pass through, but neither  said so explicitly, and Cayley\index{Cayley, Arthur (1821--1895)} seems to be the first person to have said so explicitly (see Theorem~\ref{8points9} below).  

%Cayley, and rather more clearly Salmon, \emph{Higher Plane Curves} 1879, p. 17,  argued that if $C_1 =  0$ and $C_2 = 0$ are the equations of two cubics through the  points $P_1, P_2, \ldots , P_9$ then  any cubic through the 8 points $P_1, P_2, \ldots , P_8$  and an arbitrary point $Q$ has an equation of the form $k_1 C_1 + k_2 C_2 = 0$, because it can be written uniquely as $C_2(Q) C_1 - C_1(Q) C_2 = 0.$  So $k_1 : k_2 = -C_2(Q) : C_1(Q).$ But such a curve necessarily passes through $P_9$, the only point for which  the ratio $k_1 : k_2$ is indeterminate.

There matters seem to have rested until Joseph Diaz Gergonne\index{Gergonne, Joseph Diaz (1771--1859)} in France and Pl\"ucker\index{Pl\"ucker, Julius (1801--1868)} in Germany began to push the theory of algebraic curves of degrees three and four beyond what had been known to Cramer\index{Cramer, Gabriel (1704--1752)} and  Euler\index{Euler, Leonhard (1797--1783)}. Gergonne\index{Gergonne, Joseph Diaz (1771--1859)}, in his (1826, 220) was the first to proclaim that if two curves of degree $n = p+q $ meet in $n^2$ points of which $np$ lie on a curve of degree $p$ then the remaining points lie on a curve of degree $q.$ However, his argument was little more than an assertion, and this drew a reply from Pl\"ucker\index{Pl\"ucker, Julius (1801--1868)} (1829), who had been interested in these questions for some time, indicating that more needed to be understood before such a claim could be proved.  
Pl\"ucker\index{Pl\"ucker, Julius (1801--1868)} went on  to deduce Pascal's theorem from the theory of cubic curves.\footnote{See Pl\"ucker\index{Pl\"ucker, Julius (1801--1868)} (1827, 267). Chasles\index{Chasles, Michel (1793--1880)} knew of MacLaurin's work, but apparently not the contributions of Cramer\index{Cramer, Gabriel (1704--1752)} or  Euler\index{Euler, Leonhard (1797--1783)}, although he does mention the relevant books by them in his \emph{Aper\c{c}u Historique}.} He showed  that if the nine points common to two cubic curves divide into six points lying on a conic then the remaining three points lie on a line, and conversely. From this he deduced Pascal's theorem.\footnote{We do not know if Pl\"ucker\index{Pl\"ucker, Julius (1801--1868)} also saw that this implies Pappus's theorem when the conic is a line pair; in any case,  he did not say so.}

\bigbreak\hrule\bigbreak


Pl\"ucker\index{Pl\"ucker, Julius (1801--1868)}'s  final account  of the intersection theory of algebraic curves came a decade later, in his \emph{Theorie der algebraischen Curven} (1839).
He wrote the equations of two curves of degree $n$ through $ n(n+3) - 1$ points as 
$\Omega_n = 0$ and $\Omega_n ' = 0$, 
and, like Gergonne\index{Gergonne, Joseph Diaz (1771--1859)},  wrote the equation of any other curve of degree $n$ through those points in the form
$\Omega_n + \mu \Omega_n ' = 0,$
where $\mu$ is an undetermined constant.\footnote{See Pl\"ucker 1839, 7--10).} He could make this claim because the number of points is one less than the number required to determine a curve of degree $n.$ It is then clear that a value of $\mu$ will be determined by a choice of one more point unless the point lies on every curve of degree $n$ through the given points. But, because any two curves of degree $n$ meet in $n^2$ points, it follows that all the curves through the given $ n(n+3) - 1$ points will meet again in the 
\[n^2 - (n(n+3) - 1) =  (n(n-3) + 1)\]
points determined by $\mu.$ In particular, this means when $n = 3$ that all cubic curves through eight points meet again in a common ninth point.

%He then extended this approach to look at curves of degree $n$ that depend linearly on $m$ given points, and claimed that once $m-1$ of those points are specified the curves will all pass through $n^2 - (m - 1)$ points whose positions are determined by the $m-1$ points.

%He then considered (1839, 9--10) the case of a singly infinite family of curves of degree $n$ through $n^2$ points, and supposed that $ p(p+3)$ of these points are selected ($p<n$). A curve of degree $p$ passes through these points. The number of remaining points is
%\[\frac{n(n+3)}{2} - \frac{p(p+3)}{2} - 1 = nq -\left(\frac{q(q-3)}{2} +1\right).\]
%where $q = p - n$, and he now investigated when these points lie on a curve of degree $q.$ For him, this was a question about how many points of intersection of two curves determine how many other points, and after some further analysis he concluded that, given  the $n^2$ points common to two curves of degree $n$, if $np - \left(\frac{p(p-3)}{2} + 1\right)$ of these points lie on a curve of degree $p$ then the remaining points lie on a curve of degree $n-p.$

For German geometers, and in due course Cayley\index{Cayley, Arthur (1821--1895)} too, Pl\"ucker\index{Pl\"ucker, Julius (1801--1868)}'s work was an inspiration, and they took the method of counting constants in the equations for the coefficients of curves subject to various conditions from him. How far that can go, and what it leads to, is one of the subjects of this book.

A little later, and apparently independently, Michel Chasles\index{Chasles, Michel (1793--1880)} went back to the theorems of Pappus\index{Pappus of Alexandria, 4th century CE} and Pascal\index{Pascal, Blaise (1623--1662)}, and extended them.  He showed  in his \emph{Aper\c{c}u Historique} (1837, 149) that if the six points on a hexagon and two of the three points where opposite sides of the hexagon meet all lie on a cubic curve $C$, then the third point lies on the same cubic. His  proof rested on  regarding the odd-numbered sides and the even-numbered sides as each defining cubic curves. If lines 1 and 4 meet at $P$ and lines 2 and 5 meet at $Q$, then $P$ and $Q$ and the six vertices of the hexagon form a set of eight points common to two cubic curves, and $C$ is another curve though these eight points. There is an infinity of curves passing through these eight points, said Chasles\index{Chasles, Michel (1793--1880)}, but it is a general property of  curves of the third degree that pass through eight points common to two cubic curves that they all pass through the ninth common point, and so Pascal's theorem follows (this is the remark was soon to catch Cayley\index{Cayley, Arthur (1821--1895)}'s attention). 

\input footer.tex
