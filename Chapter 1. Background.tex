\documentclass[12pt, leqno]{book}
\usepackage{amsmath,amscd,amsthm,amssymb,amsxtra,latexsym,epsfig,epic,graphics}
\usepackage[matrix,arrow,curve]{xy}
\usepackage{graphicx}
\usepackage{diagrams}
%\usepackage{amsrefs}
%%%%%%%%%%%%%%%%%%%%%%%%%%%%%%%%%%%%%%%%%
%\textwidth16cm
%\textheight20cm
%\topmargin-2cm
\oddsidemargin.8cm
\evensidemargin1cm

%%%%%Definitions
\input preamble.tex
\def\TU{{\bf U}}
\def\AA{{\mathbb A}}
\def\BB{{\mathbb B}}
\def\CC{{\mathbb C}}
\def\QQ{{\mathbb Q}}
\def\RR{{\mathbb R}}
\def\facet{{\bf facet}}
\def\image{{\rm image}}
\def\cE{{\cal E}}
\def\cF{{\cal F}}
\def\cG{{\cal G}}
\def\cH{{\cal H}}
\def\cHom{{{\cal H}om}}
\def\h{{\rm h}}
 \def\bs{{Boij-S\"oderberg{} }}

\makeatletter
\def\Ddots{\mathinner{\mkern1mu\raise\p@
\vbox{\kern7\p@\hbox{.}}\mkern2mu
\raise4\p@\hbox{.}\mkern2mu\raise7\p@\hbox{.}\mkern1mu}}
\makeatother

%%
%\pagestyle{myheadings}
\date{April 30, 2018}
%\date{}
\title{Curves}
%{\normalsize ***Preliminary Version***}} 
\author{David Eisenbud and Joe Harris }

\begin{document}

\chapter{Background}
\section*{Outline}
 Background (corresponds to material in Hartshorne Ch 4, sects 1,2,4(?)---all except elliptic curves and ``classification")
\begin{enumerate}

\item Lasker's Theorem (complete intersections are unmixed) -- sometimes incorrectly called ``$AF+BG$''
state ci implies unmixed; prove by $H^1$ of line bundle.

\item B\'ezout and the  weak B\'ezout (ex 8.4.6 in Fulton).
B\'ezout via Koszul complex, at least in codim 2.

\item Discussion of genus of smooth curves. Riemann-Roch. Hilbert coefficient.

\item Divisors and maps; canonical divisor -- cotangent bundle; canonical map as example
Exercise: Projections from on and off a curve

\item canonical series is bpf, $2g+1$ is very ample. Canonical Curves and geometric RR

\item Adjunction formula for curves in a surface (Quote from Hartshorne)

\item Clifford, including strong form, canonical series is va except in hyperelliptic case.

\item Riemann-Hurwitz (pull back a differential form)

\item Families. Define families. discuss "good and bad" families. example: symmetric product. example: plane curves of degree d. family of line bundles on a curve (or a family of curves.)


\end{enumerate}

\section{The Lasker-Noether Theorem}

A key element in the algebraic study of plane curves initiated by Brill and Noether following Riemann's discoveries was what Noether named the ``Fundamental Lemma on Holomorphic Functions.'' In the original language it said that if the homogeneous forms $F(x_{0},x_{1},x_{2})=0$ and $G(x_{0},x_{1},x_{2})=0$ are the equations of two curves in $\PP^{2}$ that have no component in common, then any form $H$ that locally represents a function (the ``holomorphic functions'' of the name) vanishing on all the points
of the intersection must have an expression $H = AF+BG$, where $A,B$ are also homogeneous forms. It was extended by Lasker to the case of many polynomials (homogeneous or not) in many variables. (This is sometimes called the ``$AF+BG$ Theorem''.) 

\begin{theorem}\label{Lasker}
Suppose that $I = (f_{1}, \dots, f_{c}) \subset \CC[x_{0},\dots,x_{n}]$ is an ideal generated by $c$ homogeneous forms in a polynomial ring. 
If $X:= V(I)$ has codimension $c$, then $I$ is saturated and $\HH^{i}(\cO_{X}(d)) = 0$ for all $0<i<\dim X$ and all $d\in \ZZ$; and if $\dim X\geq 1$
then the map
$\HH^{0}\cO_{\PP^{n}}(d) \to \HH^{0}(\cO_{X}(d)$ is surjective for every $d$.
\end{theorem}

In modern language, the hypothesis says that $f_{1}, \dots, f_{c}$ is a \emph{regular sequence} and $V(f_{1},\dots,f_{c})$ form a \emph{complete intersection}, while the concludsion says that the ring $S/I$ is Cohen-Macaulay. See for example \cite[Chapter 18]{Eisenbud95}, where the result it proven in this generality and more.
We will prove it here only for $c\leq 2$.

Theorem~\ref{Lasker} immediately implies the orginal version of the theorem because (by the Nullstellensatz) the set of forms vanishing on $V(f_{1}, \dots, f_{c})$ is the saturation 
of the ideal $(f_{1}, \dots, f_{c})$. 

\begin{proof} The vanishing in the case $c=0$ is the usual computation of the cohomology of line bundles on $\PP^{n}$.

Write $e_{i}$ for the degree of $f_{i}$. If $c = 1$ then, since the polynomial ring is a domain, we have a short exact sequence
$$
0\to \cO_{\PP^{n}}(-e_{1}) \rTo^{f_{1}}  \cO_{\PP^{n}} \to  \cO_{X}\to 0.
$$
We tensor this sequence with $\cO_{\PP^{n}}(d)$, and pass to cohomology, obtaining a long exact sequence that begins
\begin{align*}
0\to &\HH^{0}(\cO_{\PP^{n}}(d-e_{1})) \rTo^{f_{1}}  \HH^{0}(\cO_{\PP^{n}}(d)) \to  \HH^{0}(\cO_{X}(d))\to\\
&\HH^{1}(\cO_{\PP^{n}}(d-e_{1}))\to \cdots .
\end{align*}
We may break this into two exact sequences
$$
0\to \HH^{0}(\cO_{\PP^{n}}(d-e_{1})) \rTo^{f_{1}}  \HH^{0}(\cO_{\PP^{n}}(d)) \to \cI_{X}(d) \to 0
$$
and 
$$
0 \to \cI_{X}(d) \to \HH^{0}(\cO_{X}(d))\to\HH^{1}(\cO_{\PP^{n}}(d-e_{1}))\to \cdots .
$$ 
The direct sum of the spaces $\HH^{0}(\cO_{\PP^{n}}(d))$ is the polynomial ring, so the exactness of the first sequence shows that $f_{1}$ generates the homogeneous ideal of $X$. If $\dim X\geq 1$ then $n>1$ so the right hand term in the second  vanishes,
proving the last assertion of the Theorem.

Finally, suppose that $c=2$. Since $X$ has codimension 2 the hypersurfaces $f_{1}=0$ and $f_{2}=0$ have no common component---that is, $f_{1}$ and $f_{2}$ are relatively prime. Since the polynomial ring has unique factorization, this implies that the intersection of the ideals $(f_{1})$ and $(f_{2})$ is the ideal
$(f_{1}f_{2})$, or in other words that the natural complex
$$
0\to\cO_{\PP^{n}}(-e_{1}-e_{2}) \rTo^{
\begin{pmatrix}
 f_{2}\\-f_{1}
\end{pmatrix}}
 \cO_{\PP^{n}}(-e_{1})\oplus \cO_{\PP^{n}}(-e_{2}) \rTo^{
 \begin{pmatrix}
 f_{1}&f_{2}
\end{pmatrix}}
\cO_{\PP^{n}}
\to
\cO_{X}
\to 0
$$
is exact. As before, we tensor with $\cO_{\PP^{n}}(d)$ and break the result into the two exact sequences
$$
0\to\cO_{\PP^{n}}(-e_{1}-e_{2}) \rTo^{
\begin{pmatrix}
 f_{2}\\-f_{1}
\end{pmatrix}}
 \cO_{\PP^{n}}(-e_{1})\oplus \cO_{\PP^{n}}(-e_{2}) \rTo^{
 \begin{pmatrix}
 f_{1}&f_{2}
\end{pmatrix}}
\to \cI_{X}(d) \to 0
$$
and
$$
0\to \cI_{X}(d) \to \cO_{\PP^{n}}(d) \to \cO_{X}(d) \to 0.
$$
Taking cohomology we get:
\begin{align*}
0\to\HH^{0}(\cO_{\PP^{n}}(d-e_{1}-e_{2}))
&\rTo^{
\begin{pmatrix}
 f_{2}\\-f_{1}
\end{pmatrix}}
\HH^{0}( \cO_{\PP^{n}}(d-e_{1})\oplus \cO_{\PP^{n}}(d-e_{2}))\\
 &\rTo^{
 \begin{pmatrix}
 f_{1}&f_{2}
\end{pmatrix}}
\HH^{0}(\cI_{X}(d))\to \HH^{1}(\cO_{X}(d-e_{1}-e_{2})) \to \cdots
\end{align*}
and
\begin{align*}
0\to \HH^{0}(\cI_{X}(d))\to
\HH^{0}(\cO_{\PP^{n}}(d))
\to \HH^{0}(\cO_{X}(d)) \to
\HH^{1}(\cI_{X}(d))\to\cdots
\end{align*}
As above, 
\end{proof}
\end{document}