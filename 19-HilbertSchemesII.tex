%header and footer for separate chapter files

\ifx\whole\undefined
\documentclass[12pt, leqno]{book}
\usepackage{graphicx}
\input style-for-curves.sty
\usepackage{hyperref}
\usepackage{showkeys} %This shows the labels.
%\usepackage{SLAG,msribib,local}
%\usepackage{amsmath,amscd,amsthm,amssymb,amsxtra,latexsym,epsfig,epic,graphics}
%\usepackage[matrix,arrow,curve]{xy}
%\usepackage{graphicx}
%\usepackage{diagrams}
%
%%\usepackage{amsrefs}
%%%%%%%%%%%%%%%%%%%%%%%%%%%%%%%%%%%%%%%%%%
%%\textwidth16cm
%%\textheight20cm
%%\topmargin-2cm
%\oddsidemargin.8cm
%\evensidemargin1cm
%
%%%%%%Definitions
%\input preamble.tex
%\input style-for-curves.sty
%\def\TU{{\bf U}}
%\def\AA{{\mathbb A}}
%\def\BB{{\mathbb B}}
%\def\CC{{\mathbb C}}
%\def\QQ{{\mathbb Q}}
%\def\RR{{\mathbb R}}
%\def\facet{{\bf facet}}
%\def\image{{\rm image}}
%\def\cE{{\cal E}}
%\def\cF{{\cal F}}
%\def\cG{{\cal G}}
%\def\cH{{\cal H}}
%\def\cHom{{{\cal H}om}}
%\def\h{{\rm h}}
% \def\bs{{Boij-S\"oderberg{} }}
%
%\makeatletter
%\def\Ddots{\mathinner{\mkern1mu\raise\p@
%\vbox{\kern7\p@\hbox{.}}\mkern2mu
%\raise4\p@\hbox{.}\mkern2mu\raise7\p@\hbox{.}\mkern1mu}}
%\makeatother

%%
%\pagestyle{myheadings}

%\input style-for-curves.tex
%\documentclass{cambridge7A}
%\usepackage{hatcher_revised} 
%\usepackage{3264}
   
\errorcontextlines=1000
%\usepackage{makeidx}
\let\see\relax
\usepackage{makeidx}
\makeindex
% \index{word} in the doc; \index{variety!algebraic} gives variety, algebraic
% PUT a % after each \index{***}

\overfullrule=5pt
\catcode`\@\active
\def@{\mskip1.5mu} %produce a small space in math with an @

\title{Personalities of Curves}
\author{\copyright David Eisenbud and Joe Harris}
%%\includeonly{%
%0-intro,01-ChowRingDogma,02-FirstExamples,03-Grassmannians,04-GeneralGrassmannians
%,05-VectorBundlesAndChernClasses,06-LinesOnHypersurfaces,07-SingularElementsOfLinearSeries,
%08-ParameterSpaces,
%bib
%}

\date{\today}
%%\date{}
%\title{Curves}
%%{\normalsize ***Preliminary Version***}} 
%\author{David Eisenbud and Joe Harris }
%
%\begin{document}

\begin{document}
\maketitle

\pagenumbering{roman}
\setcounter{page}{5}
%\begin{5}
%\end{5}
\pagenumbering{arabic}
\tableofcontents
\fi

\chapter{Hilbert Schemes II: Counterexamples} 
\label{HilbertSchemesCounterexamplesChapter}


In the preceding chapter, we described a number of examples of Hilbert schemes, and observed some patterns in their behavior: in each case the restricted Hilbert scheme $\cH^\circ_{g,r,d}$ parametrizing smooth, irreducible and nondegenerate curves of degree $d$ and genus $g$ in $\PP^r$ was irreducible of the ``expected dimension" $h(g,r,d) :=  4g-3 + (r+1)(d-g+1) - 1$. In fact, Theorem~\ref{principal component} tells us that these patterns persist, for those components of $\cH^\circ$ dominating the moduli space $M_g$.

But what about other components of the Hilbert scheme---components with $\rho(g,r,d) < 0$, or for that matter components with $\rho(g,r,d) \geq 0$ that simply don't dominate $M_g$? None of the patterns we've observed so far hold in general, and we will give a series of examples of this variety, and open questions. We will conclude the chapter by analyzing $\cH^\circ_{24,3,24}$ one of whose components is Mumford's celebrated example of a component that is everywhere non-reduced.

\section{Degree 8, genus 9}\label{degree 8 section}

We start with an example of a component of the restricted Hilbert scheme $\cH^\circ$ whose dimension is strictly greater than $h(g,r,d)$, the space $\cH^\circ_{9,3,8}$ of smooth, irreducible, nondegenerate curves in $\PP^3$ of degree 8 and genus 9. Let $C$ be such a curve, and consider the restriction map
$$
\rho_2 : H^0(\cO_{\PP^3}(2)) \rTo H^0(\cO_C(2)).
$$
The source of $\rho_2$ has dimension 10, but the Riemann-Roch Theorem
\begin{align*}
h^0(\cO_C(2)) =
\begin{cases}
9, \quad &\text{if } \cO_C(2) \cong K_C; \\
8,  \quad &\text{if } \cO_C(2) \not\cong K_C
\end{cases}
\end{align*}
admits two possibilities for the dimension of target of $\rho_2$.
However, if $h^0(\cO_C(2))$ were 8 then $C$ would  lie on two distinct quadrics $Q$ and $Q'$. Since $C$ is nondegenerate, it cannot lie on any irreducible quadrics; thus $Q$ and $Q'$ would have to be irreducible, which would violate B\'ezout's Theorem. We deduce that $\cO_C(2) \cong K_C$, and thus that $C$ lies on a unique quadric surface $Q$ (which must be irreducible since $C$ is irreducible and doesn't lie on a plane).

Similarly, $C$ cannot lie on any cubic not containing $Q$. Moving on to quartics, we look again at the restriction map
$$
\rho_4 : H^0(\cO_{\PP^3}(4)) \rTo H^0(\cO_C(4)).
$$
The dimensions here are, respectively, 35 and $4\cdot 8 - 9 + 1 = 24$; and we deduce that $C$ lies on at least an 11-dimensional vector space of quartic surfaces. On the other hand, only a 10-dimensional vector subspace of these vanish on Q; and so we conclude that $C$ lies on a quartic surface not containing $Q$. It follows from B\'ezout's Theorem that $C = Q \cap S$. By Lasker's Theorem, the ideal $(Q,S)$ is saturated, so it is equal to the homogeneous ideal of $C$. Thus $\ker(\rho_4)$ has dimension exactly 11, and  $S$ is unique modulo quartics vanishing on $Q$.

From these facts it is easy to compute the dimension of  $\cH^\circ_{9,3,8}$: Associating to $C$ the unique quadric on which it lies gives a map $\cH^\circ_{9,3,8} \to \PP^9$ with dense image, and each fiber is an open subset of the projective space $\PP V$, where $V$ is the 25-dimensional vector space
$$
V = \frac{H^0(\cO_{\PP^3}(4))}{H^0(\cI_{Q/\PP^3}(4))}.
$$
By Exercise~\ref{hilb at a ci}, $\cH^\circ_{9,3,8}$ is generically smooth, as well.

In sum, we have proven:
\begin{proposition}
 The scheme $\cH^\circ_{9,3,8}$ of curves of genus  is generically smooth and irreducible of dimension 33---one larger than the ``expected'' $4d$.
\end{proposition}

This is a special case of Exercise~\ref{second complete intersection exercise}.

\section{Degree 9, genus 10}\label{deg9 section}

For the next example, consider the space $\cH^\circ_{9,3,10}$ of curves of degree 9 and genus 10. Once more, to describe such a curve $C$, we look to the restriction maps $\rho_m: H^0(\cO_{\PP^3}(m)) \rTo H^0(\cO_C(m))$. The Riemann-Roch Theorem tells us that
\begin{align*}
h^0(\cO_C(2)) =
\begin{cases}
10, \quad &\text{if } \cO_C(2) \cong K_C \; \text{(``the first case,") and } \\
9,  \quad &\text{if } \cO_C(2) \not\cong K_C  \; \text{(``the second case.")}
\end{cases}
\end{align*}
Unlike the the situation in degree 8, both are possible; we'll analyze each.

1. Suppose first that $C$ does not lie on any quadric surface (so that we are necessarily in the first case above), and consider the map $\rho_3 : H^0(\cO_{\PP^3}(3)) \to H^0(\cO_C(3))$. By the Riemann-Roch Theorem, the dimension of the target is $3\cdot 9 - 10 + 1 = 18$, from which we conclude that $C$ lies on at least a pencil of cubic surfaces. Since $C$ lies on no quadrics, all of these cubic surfaces must be irreducible, and it follows by B\'ezout's Theorem that the intersection of two such surfaces is exactly $C$. At this point, Lasker's Theorem assures us that $C$ lies on exactly two cubics.

By Exercise~\ref{first complete intersection exercise}, then, the space $\cH^\circ_1$ of curves of this type is thus an open subset of the Grassmannian $G(2,20)$ of pencils of cubic surfaces, which is irreducible of dimension 36. By Exercise~\ref{hilb at ci}, $\cH^\circ_1$ is generically smooth.

2. Next, suppose that $C$ does lie on a quadric surface $Q \subset \PP^3$; let $\cH^\circ_2 \subset \cH^\circ_{10,3,9}$ be the locus of such curves. In this case, we claim two things:
\begin{enumerate}
\item[a.] $Q$ must be smooth; and
\item[b.] $C$ must be a curve of type $(3,6)$ on $Q$
\end{enumerate}

For part (a), we claim that a smooth, irreducible nondegenerate curve $C$ of degree 9 lying on a singular quadric must have genus 12. We can see this by observing that $Q$ must be a cone over a smooth conic curve, and so its blow-up at the vertex is the Hirzebruch surface $\FF_2$, with the directrix $E \subset \FF_2$ the exceptional divisor of the blowup, and a line $L$ of the ruling of $\FF_2$ the proper transform of a line lying on $Q$. The pullback to $\FF_2$ of the hyperplane class has intersection number 1 with $L$ and 0 with $E$, from which it follows that its class must  be $H = 2L + E$

Now, the proper transform $\widetilde C$ of $C$ in $\FF_2$ has intersection number 1 with $E$, since $C$ passes through the vertex of $Q$ and is smooth there; given this, and the fact that it has intersection number 9 with $H = 2L+E$, we can deduce that the class of $\widetilde C$ is $9L + 4E$. Now, we know that $K_{\FF_2} = -2E - 4L$; by adjunction we deduce that  the genus of $C$ is 12.

For the second part, once we know that $Q$ is smooth, the genus formula on $Q$ tells us immediately that $C$ must be of type $(3,6)$ or $(6,3)$. Now, since the quadric $Q$ containing $C$ is unique, by B\'ezout, we have a map $\cH^\circ_2 \to \PP^9$ associating to each curve $C$ of this type the unique quadric containing it. The fiber of this map over a given quadric $Q$ is the disjoint union of open subsets of the projective spaces $\PP^{27}$ parametrizing curves of type $(3,6)$ and $(6,3)$ on $Q$, and we see that the locus $\cH^\circ_2$ again has dimension 36. We outline two proofs that 
$\cH^\circ_2$ is irreducible in Exercise~\ref{degree 9 type 2 is irreducible}.
We can show that $\cH^\circ_2$ is generically smooth by computing its tangent space
$H^0(\sN_{C/\PP^3}$. We start with the normal bundle sequence
$$
0\to \sN_{Q/\PP^3}\mid_C \to \sN_{C/\PP^3} \to \sN_{C/Q} \to 0
$$
and note that, by the adjunction theorem $\omega_C = \sO_Q(1,4)\mid_C$.
Thus both $\sN_{Q/\PP^3}\mid_C = \sO_Q(2,2)\mid_C$ and
$\sN_{C/Q} = \sO_Q(3,6)\mid_C$ are nonspecial, so 
$$
h^0 \sN_{C/\PP^3} = h^0(\sN_{\sO_Q(2,2)}\mid_C ) + h^0(\sO_Q(3,6)\mid_C
= 10-1 + (4*7-1) = 36
$$

In sum, we have proven:
\begin{proposition}
 The scheme $\cH^\circ_{9,3,10}$ of smooth, irreducible, nondegenerate curves $C \subset \PP^3$ of degree 9 and genus 10 has two irreducible components, each generically smooth of dimension 36. One has as generic point a curve that is the complete intersections of two cubics. The generic point of the other is a curve of type $(3,6)$ on a quadric surface. \end{proposition}


 \section{Special components in the nonspecial range}

If we ignore the finer points of the Brill-Noether theorem and focus just on the statement about the dimension and irreducibility of the variety of linear series on a curve, we can express it in a simple form: according to Theorem~\ref{principal component} 

\begin{corollary}
 There is a component of the restricted Hilbert scheme $\cH^\circ$ of curves of degree $d$ and genus $g$ that dominates the moduli space $M_g$ if and only if $\rho(g,r,d)\geq 0$. This component is unique and has the expected dimension 
$$
h(g,r,d) = 4g-3 + (r+1)(d-g+1) - 1 = (r+1)d - (r-3)(g-1)
$$
\end{corollary}
\begin{proof}
  This is immediate from the results of Section~\ref{estimating dim hilb}.
\end{proof}
 
If we restrict further to the nonspecial range $d \geq g + r$, we don't need the ghosts of Brill or Noether to tell us this: if $\cL$ is a general line bundle of degree $d$ on a general curve $C$ of genus $g$, and $V \subset H^0(\cL)$ a general $(r+1)$-dimensional subspace, the linear system $V$ will embed the curve $C$ as a nondegenerate curve of degree $d$ in $\PP^{r}$, and the curves obtained in this way will comprise an irreducible component of the restricted Hilbert scheme.

But that doesn't mean that there aren't other components of the restricted Hilbert scheme, even in the nonspecial range! In this section, we'll construct an example of a component of the restricted Hilbert scheme $\cH^\circ_{21,7,28}$,  (so $d = g+r$), that does not dominate $M_g$ and indeed has dimension different than the ``expected" $(r+1)d - (r-3)(g-1) = 8*28-(4*20) = 144$.

For our example, we'll take $d = 28$, $g = 21$ and $r=7$. By Theorem~\ref{degree g+3 very ample}), a general line bundle $\cL$ of degree 28 on a curve $C$ of genus 21 is  very ample. Curves of genus 21 embedded in $\PP^7$ in this way comprise a component $\cH_0$ of the Hilbert scheme $\cH^\circ_{21, 7, 28}$ having the expected dimension 
$$
\dim M_{21} + \dim \Pic^{28}(C) + \dim PGL_8=3*21-3+21 + 63 =  144.
$$

But here's another way to construct a curve of degree 28 and genus 21 in $\PP^7$, that will produce a larger family of such curves, 
all trigonal:

\begin{lemma}
Let $C$ be a general trigonal curve of genus $g$, $\cM$  the line bundle of degree 3 on $C$ having two sections, and $\cL = K_C \otimes \cM^{-l}$.
\begin{enumerate}
\item If $l \leq g/2$, then $h^0(\cL) = g - 2l$; and
\item If $l \leq (g-4)/2$, then $\cL$ is very ample.
\end{enumerate}
\end{lemma}

\begin{proof}
Both statements follow from our description of the geometry of canonical models of trigonal curves, Corollary~\ref{hyperelliptic and trigonal}. We observed there that a trigonal canonical curve lies on a rational normal scroll $S$, and that if $C$ is general, then the scroll $S$ is balanced %\fix{we didn't observe that there. Need to add something. How general should we make it?}. 
The linear system $|\cL| = | K_C \otimes \cM^{-l}|$ is then cut out by hyperplanes in $\PP^{g-1}$ containing  any $l$ chosen lines from the ruling of $S$; and the first part follows from the fact that on a balanced scroll $S \subset \PP^r$, any $\lfloor (r+1)/2\rfloor$ lines of the ruling are linearly independent.

The second part follows similarly, because if $S \subset \PP^r$ is a balanced rational normal scroll of degree $r-1$ with $r\geq 5$, and $L \subset S$ is a line of the ruling, then the projection $\pi_L : S \to \PP^{r-2}$ extends to a regular map on all of $S$, because $L$
is a Cartier divisor. Since the ruling touches both the directrix and a complementary rational normal curve in $S$, it maps $S$ to a balanced scroll in $\PP^{r-2}$. Restricting to any curve $C \subset S$, it follows that $\pi_L$ gives an embedding of $C$ in $\PP^{r-2}$ as well.
\end{proof}

Getting back to our present example, what we see is that if $C$ is a general trigonal curve of genus 21 with $g^1_3 = |\cM|$, and $\cL = K_C \otimes \cM^{-4}$, then the line bundle $\cL$ embeds $C$ as a curve of degree $2g-2-12 = 28$ in $\PP^{12}$. Now we consider the projection of the image curve in $\PP^{12}$ to $\PP^7$. The  family of such projections is parametrized by an open subset of the Grassmannian $\GG(4, 12)$, which has dimension 40. A generic trigonal curve of genus 21 is a 3 to 1 cover of
$\PP^1$ simply branched over 46 points, which are well-defined up to automorphisms of $\PP^1$, so the family has dimension 43. We thus have $2g+1 = 43$ degrees of freedom in choosing the general trigonal curve $C$, and 
another 40 degrees of freedom in choosing the projection (that is, the subseries $g^7_{28} \subset |\cL|$); together these determine the image curve $C \subset \PP^7$ up to automorphisms of $\PP^7$. In sum, we see that the family $\cH_1$ of curves $C \subset \PP^7$ described in this way has dimension
$$
\dim \cH_1 = 43 + 40 + 63 = 146.
$$
In particular, $\cH_1$ cannot be in the closure of $\cH_0$. Thus, even though we are in the nonspecial range $d \geq g+r$, there is at least one other irreducible component of the restricted Hilbert scheme, which maps to a proper subvariety of $M_g$ and has dimension strictly greater than the expected.


\section{Open problems}\label{open problems}

\subsection{Brill-Noether in low codimension}
 
In Section~\ref{degree 8 section} we saw an example of a component of the Hilbert scheme with larger than ``expected'' dimension, and it's not hard to produce lots of similar examples: components of the Hilbert scheme that parametrize complete intersections, or more generally determinantal curves, have in general dimension larger than $h(g,r,d)$, and Exercise~\ref{many large components} gives a way of generating many more.

 But observations suggest a possible pattern: perhaps components of $\cH^\circ$ whose image in $M_g$ have low codimension still have the expected dimension $h(g,r,d)$. 

The cases with codimension $\leq 2$ are already known: In~\cite{BrillNoether-1}, it is shown that if $\Sigma \subset M_g$ is any subvariety of codimension 1, then the curve $C$ corresponding to a general point of $\Sigma$ has no linear series with Brill-Noether number $\rho < -1$; and Edidin in ~\cite{Edidin} proves the analogous (and much harder) result for subvarieties of codimension 2. Moreover, among the examples we know of components of the Hilbert scheme whose dimension is strictly greater than the expected $h(g,r,d)$, there are none whose image in $M_g$ has codimension less than $g-4$. We could therefore make the conjecture:

\begin{conjecture}
If $\cK \subset \cH^\circ_{d,g,r}$ is any component of a restricted Hilbert scheme, and the image of $\cK$ in $M_g$ has codimension $\leq g-4$, then $\dim \cK = h(g,r,d)$.
\end{conjecture}

\subsection{Maximally special  curves} Most of Brill-Noether theory, and the theory of linear systems on curves in general, centers on the behavior of linear series on a general curve. The opposite end of the spectrum is also interesting, and we may ask: How special can a linear series on a special curve be?

To make such a question precise, let $\widetilde M^r_{g,d} \subset M_g$ be the closure of the image of the map $\phi : \cH^\circ_{d,g,r}\to M_g$ sending a curve to its isomorphism class. 
\begin{enumerate}
\item What is the smallest possible dimension of $\cH^\circ_{d,g,r}$? 
\item What is the smallest possible dimension of $\widetilde M^r_{g,d}$?
\item Modifying the last question slightly, let $M^r_{g,d} \subset M_g$ be the closure of the locus of curves $C$ that possess a $g^r_d$ (in other words, we are dropping the condition that the $g^r_d$ be very ample). We can ask what is the smallest possible dimension of $M^r_{g,d}$?
\end{enumerate}

One might suppose that the most special curves, from the point of view of questions 2 and 3, are hyperelliptic curves but the locus in $M_g$ of hyperelliptic curves has dimension $2g-1$. What about smooth plane curves? That's better -- in the sense that the locus in $M_g$ of smooth plane curves has dimension asymptotic to $g$ (Exercise~\ref{moduli of plane curves}) -- but there are still a lot of them.


Can we do better?  Consider the locus of smooth complete intersections of two surfaces of degree $m$ in $\PP^3$. As we saw in Exercise~\ref{complete intersection open}, these comprise an open subset $\cH^\circ_{ci}$ of the Hilbert scheme of curves of degree $d = m^2$, and genus $g$ given by the relation
$$
2g-2 = \deg K_C = m^2(2m-4),
$$
or, asymptotically,
$$
g \sim m^3.
$$

Moreover, the dimension of this component of the Hilbert scheme is easy to compute, since as we saw in Exercise~\ref{first complete intersection exercise} that it is isomorphic to an open subset of the Grassmannian $G(2, \binom{m+3}{3})$, and so has dimension
$$
2(\binom{m+3}{3} - 2) \; \sim \; \frac{m^3}{3}
$$
Finally, we observe that if $C \subset \PP^r$ is a complete intersection curve of genus $g >1$, the canonical bundle $K_C$ is a positive power of $\cO_C(1)$, and by Exercise~\ref{ci is acm}  $C$ is arithmetically Cohen-Macaulay.

 
For a given abstract curve $C$ there are only finitely many invertible sheaves having the canonical sheaf as a power; and since a complete intersection is necessarily given by a complete linear series, there are only finitely many embeddings of a given curve as a complete intersection, up to $PGL_{r+1}$. In other words, the fibers of $\cH^\circ_{ci}$ over $M_g$ have dimension $\dim(PGL_{r+1}) = r^2 + 2r$.

Thus, we have a sequence of components of the restricted Hilbert scheme $\cH^\circ$ whose images in $M_g$ have dimension tending asymptotically to $g/3$.

Exercise~\ref{balanced CI}  suggests why we chose complete intersections of surfaces of the same degree.


More generally, we can consider complete intersections of $r-1$ hypersurfaces of degree $m$ in $\PP^r$. Such curves have
genus $g = m^{r-1}((r-1)m-r-1)/2 +1$ in a similar fashion we can calculate that their images in $M_g$ have dimension asympotically approaching $2g/r!= (m-1)(r-1)/r!$
 as $m \to \infty$, as we ask you to verify in Exercise~\ref{balanced CI in higher codim}.


The question is, can we do better? For example, if we fix $r$, can we find a sequence of components $\cH_n$ of  restricted Hilbert schemes  $\cH^\circ_{g_n,r,d_n}$ of curves in $\PP^r$ such that
$$
\lim \frac{\dim \cH_n}{g_n} \; = \; 0?
$$

\subsection{Rigid curves?}

In the last section, we considered components of the restricted Hilbert scheme whose image in $M_g$ was ``as small as possible." Let's go now all the way to the extreme, and ask: is there a component of the restricted Hilbert scheme $\cH^\circ_{g,r,d}$ whose image in $M_g$ is a single point? (Of course $M_0$ itself is a single point, so we exclude genus 0.) We can give three flavors of this question, in order of ascending preposterousness.

\begin{enumerate} 
\item First, we'll say a smooth, irreducible and nondegenerate curve $C \subset \PP^r$ is \emph{moduli rigid} if it lies in a component of the restricted Hilbert scheme whose image in $M_g$ is just the point $[C] \in M_g$---in other words, if the linear series $|\cO_C(1)|$ does not deform to any nearby curves.

\item Second, we say that such a curve is \emph{rigid} if it lies in a component $\cH^\circ$ of the restricted Hilbert scheme such that $PGL_{r+1}$ acts transitively on $\cH^\circ$. This is saying that $C$ is moduli rigid, plus the line bundle $\cO_C(1)$ does not deform to any other $g^r_d$ on $C$.

\item Finally, we say that such a curve is \emph{deformation rigid} if the curve $C \subset \PP^r$ has no nontrivial infinitesimal deformations other than those induced by $PGL_{r+1}$---in other words, every global section of the normal bundle $\cN_{C/\PP^r}$ is the image of the restriction of a vector field on $\PP^r$.
\end{enumerate}

In truth, these are not so much questions as howls of frustration. The existence of irrational rigid curves seems outlandish; we don't know anyone who thinks there are such things. But then, why can't we prove that they don't exist?



\section{Degree 14, genus 24}\label{mumford example}

In this final section will analyze the three components of the restricted Hilbert scheme $\cH^\circ_{14,3,24}$ of curves of degree 14 and genus 24 in $\PP^3$ (that there are no more is proven, for example, in \cite{Nasu2008}.) This will serve as a sort of capstone; we'll use many of the ideas and techniques we've developed, and even introduce some new wrinkles. 

In many of the analyses above, we have used the identification of the tangent space to the Hilbert scheme $\cH$ at a point $[C]$ with the space $H^0(\cN_{C/\PP^3})$ of global sections of the normal bundle of $C$ to show that the Hilbert scheme was smooth. But, as Mumford discovered \cite{Mumford1962}, one of the components of $\cH_{14,3,24}$  is everywhere nonreduced.

We will begin by analyzing the possible degrees of generators of the ideal of $C$, for $C \subset \PP^3$ a smooth, irreducible curve of degree 14 and genus 24. By applying the genus formula for plane curves and curves on quadrics we see that $C$ cannot lie in a plane or on a quadric. By B\'ezout's Theorem, $C$ cannot lie on both a cubic and a quartic hypersurface, though we shall see that both possibilities are realized.

For $m\geq 3$ let
$
\rho_m : H^0(\cO_{\PP^3}(m)) \rTo H^0(\cO_C(m))
$
be the natural maps.
We will proceed by computing the size of the kernel of $\rho_m$ for $m\geq 3$.

For $m \geq 4$, the line bundle $\cO_C(m)$ has degree $>2g-2 = 46$ , so the Riemann-Roch Theorem gives an exact value of $h^0(\cO_C(m))$.
However, when $m= 3$ we have 
$$
h^0(\cO_C(3)) = 42-24+1+h^0(K_C(-3)).
$$
Since $d-g+1 = 14-24+1$ is negative, $C$ is embedded in $\PP^3$ by a special linear series, and it follows from Section~\ref{hyperelliptic special} that $C$ is not hyperelliptic. The special line bundle $K_C(-3)$ has degree $46-42 = 4$ so,
by Clifford's Theorem in the non-hyperelliptic case, $h^0(K_C(-3)) \leq 2$. Thus $h^0(\cO_C(3)) = 19, 20$ or 21.

 The ``postulation table'' (\ref{postulation table})
collects the dimensions of the source and target of  $\rho_m$ for $m = 3, \dots, 6$. 
\begin{table}\label{postulation table}
\begin{center}\begin{tabular}{ c | c | c }
 $m$ & $h^0(\cO_{\PP^3}(m))$&$h^0(\cO_C(m))$  \\
 \hline
 3 & 20& 19, 20 or 21  \\
 4 & 35& 33  \\
 5 & 56& 47  \\
 6 & 84& 61 
\end{tabular}
\end{center}
\caption{Postulation table\label{postulation table}}
\end{table}

\subsection{Case 1: $C$ does not lie on a cubic surface}

\begin{proposition}\label{mumford example H1}
The locus $\cH^\circ_1 \subset \cH^\circ_{14,3,24}$ parameterizing curves not lying on a cubic surface is dense in an irreducible component of  $\cH^\circ_{14,3,24}$. It has dimension 56, and is generically smooth.
\end{proposition} 
 
\begin{proof}
  Let $C$ be curve in $\cH^\circ_1$. Table \ref{postulation table} shows that $C$ lies on at least two linearly independent quartic surfaces $S$ and $S'$; and since $C$ does not lie on any surface of smaller degree, neither can be reducible. It follows that the intersection $S \cap S'$ must consist of the union of the curve $C$ and a curve $D$ of degree 2. The linkage formula~(\ref{linked genus formula}) says that
$$
p_a(C) - p_{a}(D) = (14 - 2)\frac{4+4-4}{2} = 24,
$$
so $D$ has arithmetic genus 0. Note that the proof above of formula~(\ref{linked genus formula}) requires that at least one of the quartic surfaces containing $C$ is smooth, which we don't a priori know in this setting; to apply it we need to invoke the more general Theorem~\ref{justification of general linkage} from Chapter~\ref{LinkageChapter}.

We can now invoke the following lemma:

\begin{lemma}\label{conics}
A subscheme $D \subset \PP^3$ of dimension 1, degree 2 and arithmetic genus 0 (that is, $\chi(\cO_D) = 1$) is necessarily a plane conic; that is, the complete intersection of a plane and a quadric.
\end{lemma}

We remark that the need to prove a lemma like this is one of the drawbacks of the method of liaison: even if we are a priori interested just in smooth, irreducible and nondegenerate curves in $\PP^3$, applying liaison can lead to  singular and/or nonreduced curves. There are some restrictions---by Theorem~\ref{justification of general linkage}, for example, says that a curve residual to a pure-dimensional scheme in a complete intersection is pure dimensional. For the present case, knowing even this is unnecessary because  a general curve in $\cH^\circ_1$ lies on a smooth quartic surface.

\begin{proof}[Proof of Lemma~\ref{conics}]
Let $H \subset \PP^3$ be a general plane, and set $\Gamma = C \cap H$. This is a scheme of dimension 0 and degree 2 in $H \cong \PP^2$, which is then either the union of two reduced points, or a single nonreduced point isomorphic to $\Spec k[\epsilon]/(\epsilon^2)$. Either way, we observe that the restriction map $H^0(\cO_{\PP^3}(m)) \to H^0(\cO_{\Gamma}(m))$ is surjective for all $m \geq 1$, and hence the map $H^0(\cO_{C}(m)) \to H^0(\cO_{\Gamma}(m))$ is as well. It follows that
    $$
    h^0(\cO_C(m)) \geq h^0(\cO_C(m-1)) + 2
    $$
    for all $m \geq 1$; since we know by  hypothesis that $h^0(\cO_C(m)) = 2m+1$ for $m$ large, we may conclude that $h^0(\cO_C(1)) \leq 3 < h^0(\cO_{\PP^3}(1))$---in other words, the scheme $C$ must be contained in a plane. It is thus a plane conic, without embedded points since any embedded points would mean $p_a(C) < 0$.
\end{proof}

Conversely, if $C$ is any curve residual to a conic $D$ in the complete intersection of two quartics, it must have degree 14 and genus 24, and by B\'ezout's Theorem it cannot lie on a cubic surface. We can thus compute the dimension of the family $\cH^\circ_1$ of smooth curves of degree 14 and genus 24 not lying on a cubic surface via the incidence correspondence
$$
\Phi = \{ (C, D, S, S') \in \cH^\circ_{14,3,24} \times \cH_{0,3,2} \times \PP^{34} \times \PP^{34} \mid S \cap S' = C \cup D\}.
$$
where $ \cH_{0,3,2}$ is the Hilbert scheme of conics in $\PP^3$. The Hilbert scheme $\cH_{0,3,2}$ is irreducible of dimension 8 (this is a special case $m=1$, $n=2$ of Exercise~\ref{second complete intersection exercise}); and for any conic $D = V(L,Q)$ given as the complete intersection of the plane $V(L)$ and the quadric $V(Q)$, Lasker's Theorem says that the homogeneous ideal of $D \subset \PP^3$ is generated by $L$ and $Q$; this allows us to see that  the space of quartic surfaces containing $D$ is a linear subspace of $\PP^{34}$ of dimension 26. The fibers of $\Phi$ over $\cH_{0,3,2}$ are thus open subsets of $\PP^{25} \times \PP^{25}$, and we deduce that $\Phi$ is irreducible of dimension 58. 

The general members of the family of quartic surfaces containing a smooth conic are themselves smooth (Exercise~\ref{smooth quartic surfaces}), so we see from considering $C,D$ as divisors on a smooth quartic, as in the derivation of the linkage formula, that $(C\cdot D) = 10$. It follows that any quartic surface containing $C$ must contain $D$ as well and so, by Lasker's Theorem, must be a linear combination of $S$ and $S'$.   The fibers of $\Phi$ over its image in $\cH_{0,2,2}$ are thus open subsets of $\PP^1 \times \PP^1$. 
The condition of not lying on a cubic surface is open, so $\cH_1$  is dense in an irreducible component of $\cH^\circ_{14,3,24}$ of dimension 56.

\subsection{Tangent space calculations}

It remains to show that $\cH_1$ is generically smooth. To do this, we have to show that,
at a general point $[C] \in \cH_{1}$, the dimension of the Zariski tangent space $H^0(\cN_{C/\PP^3})$  is 56. 
Let $S$ be a smooth quartic surface containing $C$, and consider the exact sequence 
\begin{equation}\label{normal bundle sequence}
 0 \to \cN_{C/S} \to \cN_{C/\PP^3} \to \cN_{S/\PP^3}|_C \to 0.
\end{equation}
The bundle $\cN_{S/\PP^3}|_C \cong \cO_C(4)$, which is nonspecial; we have $h^0(\cO_C(4)) = 33$ and $h^1(\cO_C(4)) = 0$. By the adjunction formula applied to $S$ we see that $K_S = \cO_S$, and applying the formula again on $S$ we see that $\cN_{C/S} \cong K_C$. Thus $h^0(\cN_{C/S}) = 24$ and $h^1(\cN_{C/S}) = 1$.

From the long exact sequence in cohomology associated to the sequence (*) we see that there are two possibilities for the dimension of $H^0(\cN_{C/\PP^3})$: 56 and 57, depending on whether the map $H^0(\cN_{C/\PP^3}) \to H^0(\cN_{S/\PP^3}|_C)$ is surjective or of corank 1.

To settle this question, we need to invoke a basic fact about deformations of subschemes of a given scheme. For this discussion, let $Z$ be an arbitrary fixed scheme, and $X \subset Y \subset Z$ a nested pair of subschemes. We can ask two questions:

\begin{enumerate}
\item Given a first-order deformation $\widetilde Y \subset \Spec k[\epsilon]/(\epsilon^2) \times Z$ of $Y$ in $Z$, does there exist a first-order deformation $\widetilde X \subset \Spec k[\epsilon]/(\epsilon^2) \times Z$ of $X$ contained in it? and
\item Given a first-order deformation $\widetilde X \subset \Spec k[\epsilon]/(\epsilon^2) \times Z$ of $X$ in $Z$, does there exist a first-order deformation $\widetilde Y \subset \Spec k[\epsilon]/(\epsilon^2) \times Z$ of $Y$ containing it?
\end{enumerate}

The answer is a basic fact from deformation theory. Let $\alpha, \beta$ be the natural maps in the following diagram:
\begin{diagram}
H^0(\cN_{X/Z}) & \rTo^\alpha & H^0(\cN_{Y/Z}|_X)  \\
& & \uTo^\beta & & \\
& & H^0(\cN_{Y/Z}).
\end{diagram}

\begin{lemma}\label{normal bundle and deformation}
The first-order deformation of $X$ corresponding to the global section $\sigma \in H^0(\cN_{X/Z})$ is contained in the first-order deformation of $Y$ corresponding to the global section $\tau \in H^0(\cN_{Y/Z})$ if and only if $\alpha(\sigma) = \beta(\tau)$. In particular, every first-order deformation of $Y$ contains a first-order deformation of $X$ if and only if $\im(\beta) \subset \im(\alpha)$.
\end{lemma} 

For a proof of this lemma, see Chapter 6 of~\cite{3264}.

We apply this construction to $Z = \PP^3$, $Y = S \subset \PP^3$ a smooth quartic surface, and $X = D \subset S$ a smooth plane conic curve. We start with the sequence
$$
0 \to \cN_{D/S} \to \cN_{D/\PP^3} \to \cN_{S/\PP^3}|_D \to 0.
$$
Identifying $D$ with $\PP^1$, we have by adjunction that $\cN_{D/S} \cong \cO_{\PP^1}(-2)$, and $S$ being a quartic, we have $\cN_{S/\PP^3}|_D \cong \cO_{\PP^1}(8)$. Moreover, since $D$ is the complete intersection of a quadric and a plane, we have $\cN_{D/\PP^3} \cong \cO_{\PP^1}(2) \oplus  \cO_{\PP^1}(4)$, so that the sequence above looks like
$$
0 \to \cO_{\PP^1}(-2) \to \cO_{\PP^1}(2) \oplus \cO_{\PP^1}(4) \to \cO_{\PP^1}(8) \to 0
$$
Now, we know that $H^1(\cO_{\PP^1}(-2)) \neq 0$, while $H^1(\cO_{\PP^1}(2) \oplus \cO_{\PP^1}(4)) = 0$, so we conclude by Lemma~\ref{normal bundle and deformation} that the map
$H^0(\cN_{D/\PP^3}) \to H^0(\cN_{S/\PP^3}|_D)$ cannot be surjective; in other words, there exist first-order deformations of $S$ that contain no first-order deformation of $D$.

The same argument works if $D$ is the union of two lines meeting at a point.

We need to introduce one more element into the argument, expressed in the following proposition.

\begin{proposition}\label{special NL}
Let $S$ be a smooth quartic surface, and $C$ and $D \subset S$ a pair of curves forming the complete intersection of $S$ with another quartic surface $S'$, with $D$ a plane conic curve. A first-order deformation $\widetilde S$ of $S$ contains a first-order deformation of $C$ if and only if it contains a first-order deformation of $D$.
\end{proposition}

\begin{proof}
The key ingredient is the observation that $H^1(\cO_S(D)) = H^1(\cO_S(C)) = 0$. What this says is that a first-order deformation $\widetilde S$ of $S$ contains a first-order deformation of $D$ if and only if it contains a first-order deformation of the line bundle $\cL = \cO_S(D)$; that is, if and only if there exists a line bundle $\widetilde \cL$ on $\widetilde S$ such that $\cL|_S \cong \cO_S(D)$, and likewise for $C$. But the existence of a line bundle $\widetilde \cL$ on $\widetilde S$ extending $\cO_S(D)$ is equivalent to the existence of a line bundle $\widetilde \cM$ on $\widetilde S$ extending $\cO_S(C)$, since they're related by $\widetilde \cM = \cO_{\widetilde S}(4) \otimes \widetilde \cL$.
\end{proof}

Going back to the exact sequence~(\ref{normal bundle sequence}), we have shown that there exist first-order deformations of $S$ that contain no first-order deformations of $C$; thus the sequence~(\ref{normal bundle sequence}) is not exact on global sections, and hence the dimension $h^0(\cN_{C/\PP^3}) = 56$, showing that $\cH_1$ is generically smooth.

This concludes the proof of Proposition~\ref{mumford example H1}.
\end{proof}

\subsection{The Noether-Lefschetz theorem}

Here is some background for the  argument above. The analysis is built on a striking fact about curves on surfaces in $\PP^3$, called the \emph{Noether-Lefschetz theorem}.


\begin{theorem}[Noether-Lefschetz]\label{Noether Lefschetz}
Let $\Sigma_{d,e}$ be the set of surfaces 
$S \subset \PP^3$ of degree $d \geq 4$ in $\PP^3$ that contain a curve $C$ of degree $e$ that is not a complete intersection
of another surface with $S$. If $d\geq 4$, then $\Sigma_{d,e}$ is a proper algebraic subset of the projective space of surfaces of
degree $d$.
\end{theorem}

Thus, for example, a very general quartic surface contains no lines, conics or twisted cubics---facts you can readily establish for yourself via a standard dimension count (Exercises~\ref{lines on quartic} and \ref{conics on quartic}).
In fact, calculations like the one suggested in these exercises were how Noether first came to propose Theorem~\ref{Noether Lefschetz}; it was not until Lefschetz that a complete proof was given.

The crucial ingredient in the proof of the generic smoothness part of Proposition~\ref{mumford example H1} as a strengthened form of the Noether-Lefschetz Theorem:

\begin{theorem}[Deformation Noether-Lefschetz]
If $S \subset \PP^3$ is a smooth surface of degree $d \geq 4$, and $C \subset S$ is any curve that is not a complete intersection with $S$, then there exists a first-order deformation $\widetilde S$ of $S$ that does not contain a first-order deformation of $C$.
\end{theorem}

In Proposition~\ref{special NL} we proved this by ad-hoc methods in the case of interest to us here; the proof of the general case, given for example in~\cite[Theorem 6.1]{Huizenga}, uses Hodge theory.

\subsection{Case 2: $C$ lies on a cubic surface $S$}

We suppose henceforward that $C$ is a smooth irreducible curve of genus 14 and degree 24 in $\PP^3$ that lies on a cubic surface. We will see that $C$ is residual in the complete
intersection of the cubic and a degree 6 surface to either the disjoint union of two conics or the disjoint union of a line and a conic. Each of these
types corresponds to a 56-dimensional component of $\cH^\circ_{14,3,24}$. One of these componentsis reduced, while the other is everywhere
non-reduced.

The first of these components is treated in Proposition~\ref{non-mumford} and the second in Proposition~\ref{mumford component}.

B\'ezout's Theorem tells us that  $S$  is unique, and we will restrict ourselves to the open subset $\cH_2 \subset \cH^\circ_{14,3,24} \setminus \cH_1$ where the surface $S$ is smooth, which in fact is dense---see~\cite{Nasu2008}.

B\'ezout's Theorem tells us that $C$ cannot lie on a quartic surface not containing $S$. If $C$ lay on a quintic surface not containing $S$ then $C$ would be residual to a line in the complete intersection of $S$ and the quintic, and the Theorem~\ref{liaison genus formula-first version} would tell us that 
$$
g(C) = (14-1)\frac{3+5-4}{2} = 26,
$$
a contradiction, so $C$ lies on no quintic surface.

On the other hand, Table~\ref{postulation table} tells us that there is at least a $84-61 = 23$-dimensional vector space of sextic polynomials vanishing on  $C$, only a 20-dimensional subspace of which can vanish on $S$. Thus there is a $\PP^2$ of sextic surfaces containing $C$ but not containing $S$, and, choosing one of them we can write
$$
S \cap T = C \cup D
$$
with $T$ a sextic surface and $D$ a curve of degree 4. The liaison formula  tells us that
$$
g(C) - g(D) = (14 - 4)\frac{3+6-4}{2} = 25,
$$
so the arithmetic genus of $D$ is $-1$. We will henceforth take $T$ to be general among sextics containing $C$, so that $D$ will be a general member of the (at least) 2-dimensional linear system cut on $S$ by sextics containing $C$.

\begin{proposition}\label {2a,b}
$D$ must either be (a) the disjoint union of a line and a twisted cubic on $S$; or (b) a union of two disjoint conics on $S$. 
\end{proposition}

The reader is invited to prove this following the steps in Exercise~\ref{character of D}.

Since neither of the cases described in Proposition~\ref{2a,b} is a specialization of the other, we conclude that the locus $\cH_2$ is the union of two disjoint loci $\cH_{2}'$ and $\cH_{2}''$ corresponding to these two cases. We consider these in turn.


\subsubsection{Case $2a$: $D$ is the disjoint union of a twisted cubic and a line}

\begin{proposition}\label{non-mumford}
The locus $\cH_{2}' \subset \cH^\circ_{14,3,24}$ parameterizing curves $C$ residual to the disjoint union of a line and a twisted cubic  in the complete intersection of a sextic and a smooth cubic surface is an irreducible component of  $\cH^\circ_{14,3,24}$. It has dimension 56, and is generically smooth.
\end{proposition} 
 
\begin{proof}
Let $\cH$ be the locus in the Hilbert scheme $\cH_{-1, 3, 4}$ corresponding to disjoint unions of twisted cubics and lines, and consider the correspondence
$$
\Phi = \{(C,D,S,T) \in \cH_{2}' \times \cH \times \PP^{19} \times \PP^{83} \mid S \cap T = C \cup D \}.
$$
We have $\dim \cH = 16$, and  by Proposition~\ref{quartic curve postulation} the fiber of $\Phi$ over a point $[D] \in \cH$ is an open subset of the product $\PP^5 \times \PP^{37}$;  so we see that $\Phi$ is irreducible of dimension 58. The fibers of $\Phi$ over $ \cH_{2}'$ are 2-dimensional, and we conclude that $\cH_{2}'$ is irreducible of dimension 56.

Finally, we calculate the dimension of the Zariski tangent space $H^0(\cN_{C/\PP^3})$ to $\cH_{2}'$ at a general point $[C]$. We do this, as before, by considering the exact sequence associated to the inclusion of $C$ in $S$:
$$
0 \to \cN_{C/S} \to \cN_{C/\PP^3} \to \cN_{S/\PP^3}|_C \to 0
$$ 
Here there is no ambiguity about the first term: by adjunction, the degree of the normal bundle of $C$ in $S$
%---that is, the self-intersection of $C$ in $S$---
is 60, which is greater than $2g(C) - 2 = 46$; so $h^1(\cN_{C/S}) = 0$ and $h^0(\cN_{C/S}) = 37$.

On the other hand, $\cN_{S/\PP^3}|_C \cong \cO_C(3)$, and from Table~\ref{postulation table}, we see that $h^0( \cO_C(3))$ can a priori be 19, 20 or 21. We will use the explicit description of $C$ to show that,  in this case, $h^0(\cO_C(3))=19$.

For this purpose, let $L$ and $T$ denote the line component and the twisted cubic component of $D$ respectively; and let $H$ denote the hyperplane class on $S$. From the adjunction formula we can compute the self-intersection numbers of these curves on $S$ as $(L \cdot L) = -1$ and $(T \cdot T) = 1$. Since $C \sim 6H - D$ on $S$, we have
$$
(C\cdot L) = ((6H - L - T) \cdot L) = 7; \quad \text{and} \quad (C\cdot T) = ((6H - L - T) \cdot L) = 17
$$
In other words, the curves $L$ and $T$ intersect $C$ in divisors $E_L$ and $E_T$ of degrees $7$ and $17$ respectively. By Serre duality, 
$$
h^1(\cO_C(3)) = h^0(K_C(-3)) 
$$
and by adjunction,
$$
K_C(-3) = K_S(C)(-3)|_C = \cO_S(-H + 6H - D - 3H)|_C = \cO_C(2)(-E_L-E_T).
$$
Now, the quadrics in $\PP^3$ cut out on $C$ the complete linear series $|\cO_C(2)|$, 
%\fix{Could be proven by using the representation of a cubic surface as a blowup of the plane.-- another use of our discussion of del Pezzo} 
so $h^1(\cO_C(3))$ is the dimension of the space of quadratic polynomials vanishing on $E_L$ and $E_T$. But $E_L$ consists of seven points on the line $L$, so any quadric containing $E_L$ contains $L$; and likewise since $E_T$ has degree $17 > 2\cdot 3$, any quadric containing $E_T$ contains $T$. Since no quadric contains the disjoint union of a line and a twisted cubic, we conclude that $h^1(\cO_C(3))=0$ and $h^0(\cO_C(3)) = 19$.

Putting this all together, we conclude that $h^0(\cN_{C/\PP^3}) = 56$; so the component $\cH_{2}'$ of the Hilbert scheme $\cH^\circ_{14,3,24}$ is generically smooth of dimension 56.
\end{proof}

\subsubsection{Case $2b$: $D$ is the disjoint union of two conics}

\begin{proposition}\label{mumford component}
The locus $\cH_{2}'' \subset \cH^\circ_{14,3,24}$ parameterizing curves $C$ residual to the disjoint union of two conics in the complete intersection of a sextic and a smooth cubic surface is an irreducible component of  $\cH^\circ_{14,3,24}$. It has dimension 56, but is non-reduced: its tangent space at
a generic point has dimension 57.
\end{proposition} 

\begin{proof}

The analysis of this case follows the same path as the preceding until the very last step, where the residual curve $D$ is the disjoint union of two conic curves rather than the disjoint union of a line and a twisted cubic. What difference does this make? Both the disjoint union of two conic curves and the disjoint union of a line and a twisted cubic are curves of degree 4 and arithmetic genus $-1$, so they both have Hilbert polynomial $p(m) = 4m+2$. The difference is that they do not have the same Hilbert function, according to the following proposition:

\begin{proposition}\label{quartic curve postulation}
Let $E$ be the  disjoint union of two conic curves in $\PP^3$ and $E'$ the disjoint union of a line and a twisted cubic. Let $h(m)$ and $h'(m)$ be their respective Hilbert functions, and $p(m) = 4m+2$ their common Hilbert polynomial.
\begin{enumerate}
\item For all $m \neq 3$, we have $h(m) = h'(m)$; and both are equal to $p(m) = 4m+2$ for $m\geq 3$; but
\item $h(2) = 9$, while $h'(2) = 10$ (in other words, $E$ lies on a unique quadric surface, while $E'$ is not contained in any quadric surface).
\end{enumerate}
\end{proposition}

\begin{proof} 
Let $S$ be the homogeneous coordinate ring of $\PP^3$, and let
$I_E = I_{Q_1}\cap I_{Q_2}$ be the homogeneous ideal of $E$, where the $I_{Q_i}$ are the homogeneous
ideals of the two disjoint conics. Similarly, let $I_{E'} = I_L\cap I_T$ be the
homogeneous ideal of $E'$, where $I_L$ is the homogeneous ideal of a line and $I_T$ is the homogeneous ideal
of a disjoint twisted cubic. We have exact sequences
\begin{align*}
0\to &S/I_E \to S/I_{Q_1} \oplus S/I_{Q_2} \to S/(I_{Q_1}+I_{Q_2})\to 0\\
0\to &S/I_{E'} \to S/I_L \oplus S/I_T \to S/(I_L+I_T)\to 0.
\end{align*}
Writing $h_Q, h_L,h_T$ for the Hilbert functions of $Q,L$ and $T$ respectively, we have
\begin{align*}
h_Q(m) &= 2m+1\\
h_L(m) &= m+1\\
h_T(m) &=3m+1 
\end{align*}
for all $m\geq 0$. 

Because each of $E,E'$ is a disjoint union, the rings $U := S/(I_{Q_1}+I_{Q_2})$ and $V := S/(I_L+I_T)$
have finite length. We claim that $U \cong k[x,y]/(q_1,q_2)$ is a complete intersection of 2 quadrics while
$V \cong k[x,y]/(x^2,xy,y^2)$. It follows that the dimensions of the homogeneous components of 
$U$ in degrees $0,1,2,3\dots$ are $1,2,1,0\dots$ while those of $V$ are $1,2,0,0\dots$. Together
with the computation above, this will prove the Proposition.

To analyze $U$, let write $I_{Q_i} = (\ell_i, q_i)$ where the $\ell_i$ are linear forms and the $q_i$ are 
quadratic forms. Since $I_{Q_1} +I_{Q_2}$ has finite length, the four forms
$\ell_1,\ell_2,q_1,q_2$ must be a regular sequence. Working modulo $(\ell_1,\ell_2)$ we see that 
$U$ is isomorphic to a complete intersection of 2 quadrics in 2 variables, as claimed.

To prove that $V$ has the given Hilbert function, it suffices to show that the degree 2 part of $V$ is 0. Since the Hilbert function of $S/I_L \oplus S/I_T$ is $4m+2$, this is equivalent to showing that the degree 2 part of
$S/I_{L\cup T}$ is 10-dimensional; that is, that no quadric vanishes on
both $L$ and $T$. Since $T$ spans $\PP^3$ and is irreducible, the quadric must be irreducible. By Proposition~\ref{link unmixed},
the residual $L\cup T$ to $C$ is unmixed, and it follows that $T$ is unmixed and spans $\PP^{3}$. 

We claim that if a line and a curve of degree 3 and genus 0 lie on any quadric, then they meet: If the quadric is smooth then $T$ would have class $(1,2)$ and the line would have to have class $(1,0)$ or
$(0,1)$ both of which meet $T$. If the quadric is an irreducible cone, then we note that  every curve meets every line on the cone. If $T$ lies on the union of two planes then $T$ has components in both planes and thus meets any line in one of them; and finally if $T$ lies on a double plane, then the line would meet $T_{\rm red}$. Thus
$T\cup L$ cannot lie on a quadric, and we are done.
\end{proof}


To return to the proof of Proposition~\ref{mumford component}, let $\cH$ now be the locus in the Hilbert scheme $\cH_{-1,3,4}$ corresponding to disjoint unions of two conics, and consider the correspondence
$$
\Phi = \{(C,D,S,T) \in \cH_{2}'' \times \cH \times \PP^{19} \times \PP^{83} \mid S \cap T = C \cup D \}.
$$
Once more we have $\dim \cH = 16$, and the fiber of $\Phi$ over a point $[D] \in \cH$ is again an open subset of the product $\PP^5 \times \PP^{37}$ (unions of two disjoint conics imposes the same number of conditions on cubics and sextics as the disjoint union of a line and a twisted cubic); so we see that $\Phi$ is irreducible of dimension 58. The fibers of $\Phi$ over $ \cH^{''}$ are 2-dimensional, and we conclude that $ \cH^{''}$ is irreducible of dimension 56.

The calculation of the dimension of the Zariski tangent space $H^0(\cN_{C/\PP^3})$ to $ \cH^{''}$ at a general point $[C]$ also proceeds as in the last case: we start with the exact sequence
$$
0 \to \cN_{C/S} \to \cN_{C/\PP^3} \to \cN_{S/\PP^3}|_C \to 0.
$$ 
Again, the line bundle $\cN_{C/S}$ has degree 60 and so is nonspecial with $h^1(\cN_{C/S}) = 0$ and $h^0(\cN_{C/S}) = 37$.

However, the determination of the cohomology of the third term, $\cN_{S/\PP^3}|_C \cong \cO_C(3)$ is different. Let $Q$ and $Q'$ be the  two conics comprising the residual curve $D$; and let $H$ denote the hyperplane class on $S$. The planes $P,P'$ spanned by $Q$ and $Q'$ respectively meet in a line $L$. Since $L$ contains the scheme of length 4 of intersection with $Q\cup Q'$, it is contained in $S$. Thus the curves $Q$ and $Q'$ are linearly equivalent on $S$, so we can write the class of $C$ on $S$ as $6H-2Q \sim 4H+2L$.

Since $Q\cap Q' = \emptyset$ we have  $Q \cdot Q = 0$; and since $C \sim 6H - 2Q$ on $S$, we have
$$
(C\cdot Q) = ((6H - 2Q) \cdot Q) = 12.
$$
In other words, the curves $Q$ and $Q'$ intersect $C$ in divisors $E_Q$ and $E_{Q'}$ of degree $12$. As before, we can write
$$
h^1(\cO_C(3)) = h^0(K_C(-3)) = h^0(\cO_C(2)(-E_Q-E_{Q'}))
$$
and using again the completeness of the linear series cut out on C by quadrics, we see that \emph{$h^1(\cO_C(3))$ is the dimension of the space of quadratic polynomials vanishing on $E_Q$ and $E_{Q'}$}; again, since $12 > 2\cdot 2$, this is the same as the space of quadrics containing the two curves $Q$ and $Q'$. 

Here is where the stories diverge: we saw in Proposition~\ref{quartic curve postulation} that whereas there is no quadric containing the disjoint union of a line and a twisted cubic, there is indeed a unique quadric containing the union of two given disjoint conics, namely, the union of the planes of the conics.  
Thus $h^1(\cO_C(3))=1$ so  $h^0(\cO_C(3)) = 20$ and correspondingly $h^0(\cN_{C/\PP^3}) = 57$.
\end{proof}



\subsubsection{What's going on here?}

What accounts for the different behaviors of curves in cases $2a$ and $2b$? Here is one explanation:

To start, let $C$ be a curve corresponding to a general point of $\cH_{2}'$. As we've seen, we have
$$
h^1(\cO_C(3)) = 0 \quad \text{and} \quad h^0(\cO_C(3)) = 19,
$$
so we see already from Table~\ref{postulation table} that $C$ must lie on a cubic surface. Moreover, by upper-semicontinuity, the same is true of any deformation of $C$, and so in an \'etale neighborhood of $[C]$ the Hilbert scheme looks like a projective bundle over the space of cubic surfaces.

By contrast, if $C$ is the curve corresponding to a general point of $\cH_{2}''$, we have
$$
h^1(\cO_C(3)) = 1 \quad \text{and} \quad h^0(\cO_C(3)) = 20.
$$
In other words, $C$ is not forced to lie on a cubic surface, it just chooses to do so! The ``extra'' section of the normal bundle corresponds to a first-order deformation of $C$ that is not contained in any deformation of $S$. 
If we could extend these deformations to arbitrary order, we would arrive at a family of curves whose general member lay in the first component $\cH_1$; but we know that a general point of $ \cH^{''}$ is not in the closure of $\cH_1$, and so \emph{these deformations of $C$ must be obstructed}.

One note: it may seem that the phenomenon described in this last example---a component of the Hilbert scheme that is everywhere nonreduced, even though the objects parametrized are perfectly nice smooth, irreducible curves in $\PP^3$---represents a pathology, and indeed, it was first described by David Mumford, in a paper entitled ``Pathologies"! But, as Ravi Vakil has shown, it is to be expected: \cite{MR2227692} shows that \emph{every} complete local ring over an algebraically closed field, up to adding power series variables, occurs as the completion of the local ring of a Hilbert scheme of smooth curves---that is, in effect, every singularity is possible. 

\section{Exercises}
 The next three exercises refer to the discussion of the two types of curves of degree 9 and genus 10 in Section~\ref{deg9 section}:
\begin{exercise}\label{degree 9 type 2 is irreducible}
In Section~\ref{deg9 section} we identified a locus $\cH^\circ_2$ of degree 9, genus 10 curves lying on a quadric.
One can show that $\cH^\circ_2$  is irreducible by a monodromy argument using Example~\ref{monodromy of rulings}, but one can also prove it via a liaison:  a curve in $\cH^\circ_2$ is residual to a union of three skew lines in the intersection of a quadric and a sextic curve. Use this to establish that $\cH^\circ_2$ is irreducible.
\end{exercise}

\begin{exercise}
 we used a dimension count to conclude that a general curve of type 1 could not be a specialization of a curve of type 2, and vice versa. Prove these assertions directly: specifically, argue that
\begin{enumerate}
\item by upper-semicontinuity of $h^0(\cI_{C/\PP^3}(2))$, argue that a curve $C$ not lying on a quadric cannot be the specialization of curves $C_t$ lying on quadrics; and
\item show that for a general curve of type $(3,6)$ on a quadric, $K_C \not\cong \cO_C(2)$, and deduce that a general curve of type 2 is not a specialization of curves of type 1.
\end{enumerate}
\end{exercise}

\begin{exercise}
Let $\Sigma_1$ and $\Sigma_2 \subset \cH^\circ_{10,3,9}$ be the loci of curves of types 1 and 2 respectively. 
\begin{enumerate}
\item What is the intersection of the closures of $\Sigma_1$ and $\Sigma_2$ in $ \cH^\circ_{10,3,9}$?
\item What is the intersection of the closures of $\Sigma_1$ and $\Sigma_2$ in the whole Hilbert scheme $\cH_{9m-9}(\PP^3)$?
\end{enumerate}
\end{exercise}

\begin{exercise}\label{smooth quartic surfaces}
The general members of the family of quartic surfaces containing a smooth conic are themselves smooth. %\fix{give a hint}
\end{exercise}

\begin{exercise}\label{lines on quartic}
Let $\GG(1,3)$ be the Grassmannian of lines in $\PP^3$, let $\PP^{19}$ denote the space of quartic surfaces $S \subset \PP^3$, and consider the incidence correspondence
$$
\Gamma = \{ (S, L) \in \PP^{19} \times \GG(1,3) \mid L \subset S \}
$$
Calculate the dimension of $\Gamma$, and deduce in particular that the projection map $\Gamma \to \PP^{19}$ cannot be dominant.
\end{exercise} 


\begin{exercise}\label{conics on quartic} In the preceding exercise, replace the Grassmannian $\GG(1,3)$ with the restricted Hilbert schemes $\cH^\circ$ parametrizing conics and twisted cubics, and carry out the analogous calculation to deduce that a general quartic surface $S \subset \PP^3$ contains no conics or twisted cubics. What goes wrong when we replace $\cH^\circ$ with the restricted Hilbert scheme of curves of higher degree?
\end{exercise} 

The next two exercises refer to the discussion following Proposition~\ref{2a,b}
\begin{exercise}\label{character of D}
(Guided exercise to prove this proposition: first, $D$ cannot have multiple components; then, must be disconnected.)
\end{exercise}

\begin{exercise}
(Guided exercise to prove this AND deduce that $\cH_{2}'$ and $\cH_{2}''$ are irreducible, either by the incidence correspondences or by monodromy.)
\end{exercise}

\begin{exercise}\label{many large components}
Let $\cH^\circ$ be a component of the Hilbert scheme parametrizing curves of degree $d$ and genus $g$ in $\PP^3$ that dominates the moduli space $M_g$. For $s, t \gg d$, let $\cK^\circ$ be the family of smooth curves residual to a curve $C \in  \cH^\circ$ in a complete intersection of surfaces of degrees $s$ and $t$.
\begin{enumerate}
\item Show that $\cK^\circ$ is open and dense in a component of the Hilbert scheme of curves of degree $st-d$ and the appropriate genus.
\item Calculate the dimension of $\cK^\circ$, and in particular show that it is strictly greater than $h(g,r,d)$.
\end{enumerate}
\end{exercise}

\begin{exercise}\label{moduli of plane curves}
\begin{enumerate}
\item Let $C \subset \PP^2$ be a smooth plane curve of degree $d$. Show that the $g^2_d$ cut by lines on $C$ is unique; that is, $W^2_d(C)$ consists of one point.
\item Using this, find the dimension of the locus of smooth plane curves in $M_g$.
\end{enumerate}
\end{exercise}

\begin{exercise}\label{balanced CI}
Consider the locus of curves $C \subset \PP^3$ that are complete intersections of a quadric surface and a surface of degree $m$. Show that these comprise components of the restricted Hilbert scheme, and that their images in moduli have dimension asymptotically approaching $g$ as $m \to \infty$.
\end{exercise}

\begin{exercise}\label{balanced CI in higher codim}
Consider the locus $\cH^\circ_{ci}$, in the Hilbert scheme $\cH^\circ$, of smooth, irreducible, nondegenerate curves $C \subset \PP^r$ that are complete intersections of $r-1$ hypersurfaces of degree $m$. 
\begin{enumerate}
\item Show that $\cH^\circ_{ci}$ is open in $\cH^\circ$;
\item Calculate the dimension of $\cH^\circ_{ci}$ (and observe that it is irreducible); and
\item Show that the dimension of the image of $\cH^\circ_{ci}$ in $M_g$ is asymptotically $2g/r!$ as $m \to \infty$
\end{enumerate}
\end{exercise}


%footer for separate chapter files

\ifx\whole\undefined
%\makeatletter\def\@biblabel#1{#1]}\makeatother
\makeatletter \def\@biblabel#1{\ignorespaces} \makeatother
\bibliographystyle{msribib}
\bibliography{slag}

%%%% EXPLANATIONS:

% f and n
% some authors have all works collected at the end

\begingroup
%\catcode`\^\active
%if ^ is followed by 
% 1:  print f, gobble the following ^ and the next character
% 0:  print n, gobble the following ^
% any other letter: normal subscript
%\makeatletter
%\def^#1{\ifx1#1f\expandafter\@gobbletwo\else
%        \ifx0#1n\expandafter\expandafter\expandafter\@gobble
%        \else\sp{#1}\fi\fi}
%\makeatother
\let\moreadhoc\relax
\def\indexintro{%An author's cited works appear at the end of the
%author's entry; for conventions
%see the List of Citations on page~\pageref{loc}.  
%\smallbreak\noindent
%The letter `f' after a page number indicates a figure, `n' a footnote.
}
\printindex[gen]
\endgroup % end of \catcode
%requires makeindex
\end{document}
\else
\fi
