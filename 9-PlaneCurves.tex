%header and footer for separate chapter files

\ifx\whole\undefined
\documentclass[12pt, leqno]{book}
\usepackage{graphicx}
\input style-for-curves.sty
\usepackage{hyperref}
\usepackage{showkeys} %This shows the labels.
%\usepackage{SLAG,msribib,local}
%\usepackage{amsmath,amscd,amsthm,amssymb,amsxtra,latexsym,epsfig,epic,graphics}
%\usepackage[matrix,arrow,curve]{xy}
%\usepackage{graphicx}
%\usepackage{diagrams}
%
%%\usepackage{amsrefs}
%%%%%%%%%%%%%%%%%%%%%%%%%%%%%%%%%%%%%%%%%%
%%\textwidth16cm
%%\textheight20cm
%%\topmargin-2cm
%\oddsidemargin.8cm
%\evensidemargin1cm
%
%%%%%%Definitions
%\input preamble.tex
%\input style-for-curves.sty
%\def\TU{{\bf U}}
%\def\AA{{\mathbb A}}
%\def\BB{{\mathbb B}}
%\def\CC{{\mathbb C}}
%\def\QQ{{\mathbb Q}}
%\def\RR{{\mathbb R}}
%\def\facet{{\bf facet}}
%\def\image{{\rm image}}
%\def\cE{{\cal E}}
%\def\cF{{\cal F}}
%\def\cG{{\cal G}}
%\def\cH{{\cal H}}
%\def\cHom{{{\cal H}om}}
%\def\h{{\rm h}}
% \def\bs{{Boij-S\"oderberg{} }}
%
%\makeatletter
%\def\Ddots{\mathinner{\mkern1mu\raise\p@
%\vbox{\kern7\p@\hbox{.}}\mkern2mu
%\raise4\p@\hbox{.}\mkern2mu\raise7\p@\hbox{.}\mkern1mu}}
%\makeatother

%%
%\pagestyle{myheadings}

%\input style-for-curves.tex
%\documentclass{cambridge7A}
%\usepackage{hatcher_revised} 
%\usepackage{3264}
   
\errorcontextlines=1000
%\usepackage{makeidx}
\let\see\relax
\usepackage{makeidx}
\makeindex
% \index{word} in the doc; \index{variety!algebraic} gives variety, algebraic
% PUT a % after each \index{***}

\overfullrule=5pt
\catcode`\@\active
\def@{\mskip1.5mu} %produce a small space in math with an @

\title{Personalities of Curves}
\author{\copyright David Eisenbud and Joe Harris}
%%\includeonly{%
%0-intro,01-ChowRingDogma,02-FirstExamples,03-Grassmannians,04-GeneralGrassmannians
%,05-VectorBundlesAndChernClasses,06-LinesOnHypersurfaces,07-SingularElementsOfLinearSeries,
%08-ParameterSpaces,
%bib
%}

\date{\today}
%%\date{}
%\title{Curves}
%%{\normalsize ***Preliminary Version***}} 
%\author{David Eisenbud and Joe Harris }
%
%\begin{document}

\begin{document}
\maketitle

\pagenumbering{roman}
\setcounter{page}{5}
%\begin{5}
%\end{5}
\pagenumbering{arabic}
\tableofcontents
\fi


\chapter{Plane Curves}
\label{PlaneCurvesChapter}

**** David: This started out as a chapter in which we extend the constructions/theorems of the rest of the book to the case of singular curves. We since abandoned that goal as too difficult and open-ended, and my understanding is that we were going to replace it with a chapter on plane curves. So here's the chapter on plane curves, but I haven't changed the name of the file (and I commented out, rather than deleted, the stuff that was already written). ****

\

As anyone who has read the last chapter knows, describing curves in projective space $\PP^3$  is difficult: the ideal of such a curve typically has three or more generators, which in turn have to satisfy certain syzygies; in consequence, even basic facts about them---for example, the dimension of the family of all curves of given degree and genus---remain very unclear.

The case of plane curves makes a striking contrast: a curve $C \subset \PP^2$ is necessarily the zero locus of a single homogeneous polynomial, and conversely any homogeneous polynomial $F(X,Y,Z)$ defines a plane curve. There is a downside, however: while any smooth projective curve can be embedded in $\PP^r$ for any $r \geq 3$, most curves cannot be embedded in the plane. 

On the other hand, it is true that every curve can be birationally embedded in $\PP^2$: we can embed $C$ as a curve $\tilde C \subset \PP^r$ in a higher-dimensional projective space and find a projection $\PP^r \to \PP^2$ that carries $\tilde C$ birationally onto its image. This is indeed how 19th century geometers typically described a curve, in the days before abstract varieties: as the normalization of a plane curve. (The points on the normalization were realized as valuations on the function field of $C$, essentially taking advantage of the fact that for smooth curves birational and biregular isomorphism are the same thing.) Much of the analysis of the geometry of the curve---for example, the description of the linear systems on the curve---was carried out in this setting.

Plane curves, in other words, occupied a central role in the development of the theory of algebraic curves; and there are still many aspects of the geometry of a curve that are best approached in this way. In this chapter, we'll describe some of the tools used to study plane curves.

To describe our approach at the outset: we will start, in Section~\ref{smooth plane curves}, with the case of smooth plane curves. As noted, this covers only a small fraction of all curves, but it will allow us to establish our basic constructions in a relatively simple setting. Then, in Section~\ref{nodal plane curves}, we'll extend this analysis to nodal plane curves; this is a significant expansion by virtue of Theorem~\ref{****}, which says that every smooth projective curve is the normalization of a nodal plane curve. Finally, in Section~\ref{arbitrary plane curve} we'll describe the most general form of this analysis, applicable to any reduced  and irreducible plane curve.

\section{Smooth plane curves}\label{smooth plane curves}

Let's start by posing two ``keynote" problems, which we'll solve in this section for smooth plane curves. In both, we'll be given the equation $F(X,Y,Z)$ of a smooth plane curve $C$; in the second, we'll be given in addition a divisor $D = \sum m_ip_i$ on $C$. We ask:
\begin{enumerate}
\item Find all regular differentials/sections of $K_C$; that is, write down a basis for $H^0(K_C)$; and
\item Describe the complete linear system $|D|$; that is, find all effective divisors $E$ on $C$ with $E \sim D$.
\end{enumerate}

Note that this subsumes questions like, are two given divisors $D$ and $E$ linearly equivalent? And, is a given divisor $D$ linearly equivalent to an effective divisor? As we said above, in this section we'll solve these problems for a smooth plane curve; in the following section we'll do it for (the normalization of) a plane curve with nodes; this will allow us to answer these questions algorithmically for an arbitrary smooth curve.

 So: let's say $C \subset \PP^2$ is the plane curve given as the zero locus of a homogeneous polynomial $F(X,Y,Z)$ of degree $d$. We'll introduce affine coordinates $x = X/Z$ and $y = Y/Z$ on the affine open subset $U \cong \AA^2$ given by $Z \neq 0$, and let $f(x,y) = F(X,Y,1)$ be the inhomogeneous form of $F$, so that $\tilde C = C \cap U$ is given as the zero locus $V(f) \subset U$. Finally, just for simplicity, we'll assume that $C$ is transverse to the line $L = V(Z)$ at infinity, meeting it at $d$ points $p_1,\dots,p_d$, none of which is the point $[0,1,0]$ (equivalently, $\tilde C$ has no vertical asymptotes).
 
 We start by drastically scaling back our ambitions: instead of asking for a basis of $H^0(K)$, let's just see if we can write down a single rational 1-form on $C$. This is manageable: we can just take a regular 1-form on $\AA^2$, such as $dx$, and restrict/pull back to $C$. In fact, this will be regular on the open subset $\tilde C$, but a change of coordinate calculation shows that it will have double poles at the points $p_1,\dots,p_d$ of $C \cap L$.
 
 How do we get rid of the poles of $dx$? The natural thing to do would be simply to divide $dx$ by a polynomial $h(x,y)$ of degree 2 or more, but there's a problem: $h(x,y)$ will vanish at points of $C \cap U$, potentially creating poles of the quotient $dx/h$. There is a solution to this: \emph{choose a polynomial $h$ that vanishes only at those points of $C \cap U$ where $dx$ already has a zero}; hopefully, the zeroes of $h$ will just cancel the zeroes of $dx$ rather than creating new poles.
 
 In fact, we have just the polynomial: we take
 $$
 h(x,y) = \frac{\partial f}{\partial y}(x,y).
 $$
 To see that this works, note that on $C$,
 $$
 df = \frac{\partial f}{\partial x}dx + \frac{\partial f}{\partial y}dy \equiv 0.
 $$
Since $\frac{\partial f}{\partial x}$ and $\frac{\partial f}{\partial y}$ have no common zeroes on $\tilde C$ by the hypothesis of smoothness, we see that at any point $p \in \tilde C$,
$$
\ord_p(dx) = \ord_p(\frac{\partial f}{\partial y})
$$ 
so in fact the quotient 
$$
\omega_0 = \frac{dx}{\partial f/\partial y}
$$
is everywhere regular and nowhere 0 in $\tilde C$.

What about the points $p_i$? Well, the differential $dx$ had poles of order $2$ at the points $p_i$; and $\partial f/\partial y$, being a polynomial of degree $d-1$, will have poles of order $d-1$. We conclude that $\omega_0$ has zeroes of order $d-3$ at the points $p_i$; in other words, the divisor
$$
(\omega_0) = (d-3)D.
$$
In particular, if $d \geq 3$ then $\omega_0$ is a global regular differential on $C$.

This also says that we can afford to multiply $\omega_0$ by any polynomial $g(x,y)$ of degree $d-3$ or less without introducing poles, so that 
$$
g\omega_0 = \frac{g(x,y)dx}{\partial f/\partial y}
$$ 
is likewise a global regular differential, for $g$  any polynomial of degree $\leq d-3$.

We have thus found a vector space of regular differentials, of dimension $\binom{d-1}{2}$. But at the same time, the degree of a differential like $\omega_0$ is
$$
\deg((\omega_0)) = (d-3)\deg(D) = d(d-3),
$$
so that the genus of $C$ is 
$$
\frac{d(d-3)}{2} + 1 = \binom{d-1}{2}.
$$
In other words, we have found all the global regular differentials on $C$! We have
$$
H^0(K_C) = \left\{ \frac{g(x,y)dx}{\partial f/\partial y} \mid \deg g \leq d-3\right\};
$$
or, equivalently, the space of regular differentials on $C$ has basis $\{\omega_{i,j} \}_{i+j \leq d-3}$, where
$$
\omega_{i,j} =  \frac{x^iy^jdx}{\partial f/\partial y}
$$

In fact, we could have done this all abstractly, without coordinates: by adjunction, we have
$$
K_C = (K_{\PP^2} \otimes \cO_{\PP^2}(d))|_C = \cO_C(d-3),
$$
and from the exact sequence
$$
0 \to \cO_{\PP^2}(-3) \to \cO_{\PP^2}(d-3) \to \cO_C(d-3)=K_C \to 0
$$
and the vanishing of $H^1(\cO_{\PP^2}(-3))$, we see that the map on global sections
$$
H^0(\cO_{\PP^2}(d-3)) \to H^0(K_C)
$$
is surjective. 

\begin{exercise}\label{gonality of smooth plane curve}
Let $C$ be a smooth plane curve of degree $d$. Show that $C$ admits a map $C \to \PP^1$ of degree $d-1$, but does not admit a map $C \to \PP^1$ of degree $d-2$ or less.
\end{exercise}


\subsection{Finding complete linear systems on smooth plane curves}

What about our second problem, finding all effective divisors linearly equivalent to a given divisor $D$? To answer this, start by expressing $D$ as the difference 
$$
D = E - F
$$
of two effective divisors on $C$. Next, let $G(X,Y,Z)$ be a polynomial in the plane of any degree $m$ vanishing on the divisor $E$, but not vanishing identically on $C$ (taking $m$ sufficiently large ensures the existence of such a polynomial), and let $A$ be the divisor cut on $C$ residually; that is, we write
$$
(G) = E + A
$$
as divisors on $C$. Now, let $H$ be another polynomial of the same degree $m$ as $G$, vanishing on $A + F$ but again not vanishing identically on $C$. Let $D'$ be the divisor cut on $C$ by $H$ residual to $A+F$; that is, write
$$
(H) = A + F + D'.
$$
Now, the divisor of the rational function $H/G$ is principal, so we have
$$
0 \sim (H) - (G) = D' + F - E
$$
or in other words, \emph{$D'$ is an effective divisor linearly equivalent to $D$.}

Note that in the simplest nontrivial case $d=3$, we have reproduced the classic description of the group law on a plane cubic curve $C$. If we choose as origin on the curve $C$ a point $o$, then to add two points $p$ and $q \in C$ means to find the (unique) effective divisor of degree 1 linearly equivalent to $p + q - o$. In this situation, we can carry out the process described above with $m=1$: draw the line $L$ through the points $p$ and $q$, and let $r \in C$ be the remaining point of intersection of $L$ with $C$; then draw the line $M$ though the points $r$ and $o$, and let $s \in C$ be the remaining point of intersection of $L$ with $C$. This is the classical construction of the group law.

We claim now that in fact we have found in this way \emph{all} effective divisors $D' \sim D$. To see this, suppose $D'$ is any effective divisor with $D' \sim D$. Carrying out the first step of the process as before, we arrive at a divisor $A$ with 
$$
\cO_C(A+F+D') = \cO_C(A+F+D)  = \cO_C(m).
$$
But from the exact sequence 
$$
0 \to \cO_{\PP^2}(m-d) \to \cO_{\PP^2}(m)  \to \cO_C(m) ]\to 0
$$
and the vanishing of $H^1(\cO_{\PP^2}(m-d))$, we have that every global section of $ \cO_C(m)$ is the restriction to $C$ of a homogeneous polynomial of degree $m$ on $\PP^2$. Thus there is a polynomial $H$ cutting out the divisor $A+F+D'$ on $C$, as claimed.

Note that if, in the process described, it turns out there is no polynomial $H$ vanishing on  $A + F + D$ but not vanishing identically on $C$, that simply means that $|D| = \emptyset$; that is, \emph{$D$ is not linearly equivalent to any effective divisor}. (It may not be obvious that the existence of such an $H$ is independent of the choice of $m$ or $G$, but the argument here shows it is.)

\section{Nodal plane curves}\label{nodal plane curves}

As noted, smooth plane curves are very special among all curves. We now want to carry out the analyses above for a larger class of plane curves, curves with at most nodes as singularities. These are still special among all plane curves, but as we'll see in Section~\ref{projection section} below, every smooth curve is the normalization of a nodal plane curve, so that this will in theory allow us to answer the ``keynote" questions above for an arbitrary smooth curve. (In the final section of this chapter, we'll indicate how the constructions here may be extended to an arbitrary plane curve.)

The set-up, in any event, is as follows: we have a nodal plane curve $C_0 \subset \PP^2$, with normalization $\nu : C \to C_0$; or, equivalently, a smooth projective curve $C$ and a birational embedding of $C$ in $\PP^2$ with image a nodal curve $C_0$. Our goals will be as before: 
\begin{enumerate}
\item to write out explicitly all global regular 1-forms on $C$; and
\item given a divisor $D$ on $C$, to  determine $|D|$; that is, find all effective divisors linearly equivalent to $D$.
\end{enumerate}

As before, we'll choose homogeneous coordinates  $[X,Y,Z]$ on $\PP^2$ so that the curve $C_0$ intersects the line $L = V(Z)$ at infinity transversely at points $p_1,\dots,p_d$ other than $[0,1,0]$; in addition, we can assume that all the nodes of $C_0$ lie in the affine plane $U = \PP^2 \setminus L$, and that neither branch of $C_0$ at a node has vertical tangent. (As before, these conditions are not logically necessary; they serve only to keep the notation reasonably simple, and in any case are satisfied by a general choice of coordinates.) Let the nodes of $C_0$ be $q_1,\dots,q_\delta$, with $r_i, s_i \in C$ lying over $q_i$; we'll denote by $\Delta$ the divisor $\sum r_i + \sum s_i$ on $C$.

Let $F(X,Y,Z)$ be the homogeneous polynomial of degree $d$ defining the curve $C_0$, and let $f(x,y) = F(x,y,1)$ be the defining equation of the affine part $C_0 \cap U$ of $C_0$. We start by considering the rational differential $\nu^*(dx)$ on $C$. In the smooth case, we saw that this differential was regular and nonzero in the finite plane, but had poles of order 2 at the point of $C \cap L$; this followed from the equation
$$
 df = \frac{\partial f}{\partial x}dx + \frac{\partial f}{\partial y}dy \equiv 0.
 $$
and the fact that $\frac{\partial f}{\partial x}$ and $\frac{\partial f}{\partial y}$ have no common zeroes on $C_0$. But now $\frac{\partial f}{\partial x}$ and $\frac{\partial f}{\partial y}$ \emph{do} have common zeroes; specifically, the pullbacks $\nu^*(\frac{\partial f}{\partial x})$ and $\nu^*(\frac{\partial f}{\partial y})$ have simple zeroes at the points $r_i$ and $s_i$. We conclude, accordingly, that the differential $\nu^*dx$ has double poles at the points $p_i$, and simple poles at the points $r_i$ and $s_i$; proceeding as before, we see that for a polynomial $g(x,y)$ of degree $\leq d-3$, the differential
$$
\nu^*( \frac{g(x,y)dx}{\partial f/\partial y})
$$
will be regular except for simple poles at the points $r_i$ and $s_i$.

So, how do we get rid of these poles? There is one simple way: we require that $g$ vanishes at the points $q_i$. We say in this case that $g$ (and the curve defined by $g$) \emph{satisfies the adjoint conditions}. In any event, we see that
$$
 \left\{ \nu^* \frac{g(x,y)dx}{\partial f/\partial y} \mid \deg g \leq d-3 \text{ and } g(q_i) = 0 \; \forall i \right\} \subset H^0(K_C).
$$
Now, in the smooth case, we were able to compare dimensions to conclude that this inclusion was indeed an equality. We can do the same thing here: to begin with, we have seen that the  rational 1-form $\omega = \nu^*(\frac{dx}{\partial f/\partial y})$ has zeroes of order $d-3$ at the points $p_1,\dots,p_d$ and simple poles at the points $r_i$ and $s_i$ and is otherwise regular and nonzero; in other words, if we set $H = p_1+\dots + p_d$, the divisor
$$
(\omega) = (d-3)H - \Delta.
$$
In particular, we see that
$$
\deg((\omega)) = d(d-3) - 2\delta
$$
and correspondingly
$$
g(C) = \binom{d-1}{2} - \delta;
$$
this is called the genus formula for plane curves.

On the other hand, the space of polynomials $g$ of degree $\leq d-3$ vanishing at the points $q_i$ has dimension at least $ \binom{d-1}{2} - \delta$; we conclude from this that indeed
$$
H^0(K_C) =  \left\{ \nu^* \frac{g(x,y)dx}{\partial f/\partial y} \mid \deg g \leq d-3 \text{ and } g(q_i) = 0 \; \forall i \right\},
$$
and as lagniappe we see also that \emph{the nodes $q_i$ of an irreducible nodal plane curve of degree $d$ impose independent conditions on curves of degree $d-3$}.

In Exercise~\ref{gonality of smooth plane curve}, we saw how to use the description of the canonical series on a smooth plane curve to determine its gonality. Now that we have an analogous description of the canonical series on (the normalization of) a nodal plane curve, we can deduce a similar statement about the gonality of such a curve. Here are the first two cases  

\begin{exercise}
Let $C_0$ be a plane curve of degree $d$ with one node and no other singularities, and let $C$ be its normalization. Show that $C$ admits a unique map $C \to \PP^1$ of degree $d-2$, but does not admit a map $C \to \PP^1$ of degree $d-3$ or less.
\end{exercise}

\begin{exercise}
Let $C_0$ be a plane curve of degree $d$ with two nodes and no other singularities, and let $C$ be its normalization. Show that $C$ admits two maps $C \to \PP^1$ of degree $d-2$, but does not admit a map $C \to \PP^1$ of degree $d-3$ or less.
\end{exercise}


\subsection{Linear series on a nodal curve}

Next, we take up the second of our keynote problems in this setting: with $C \to C_0 \subset \PP^2$ as above, given a divisor $D$ on $C$, can we find the complete linear series $|D|$?

In fact we can, by a process analogous to what we did in the smooth case. We'll do this first in the case where $D = E-F$ is the difference of two effective divisors whose support is disjoint from the support $\{r_i, s_i\}$ of $\Delta$; the general case is only notationally more complicated. To start, we find an integer $m$ and a polynomial $G$ vanishing on the divisor $E$ \emph{and at the nodes $r_1,\dots,r_\delta$ of $C_0$}, but not vanishing identically on $C_0$. We can then write the zero locus of $G$ pulled back to $C$ as
$$
(\nu^*G) = E + \Delta + A,
$$
as before. Once more, just for simplicity, let's assume that the support of $A$ is disjoint from the support of $\Delta$; this means just that the curve $V(G)$ is smooth at the points $q_i$ and is not tangent to either of the branches of $C_0$ there.

Next, we find polynomials $H$ of the same degree $m$, vanishing at $A+F$ and at the points $q_i$ but not on all of $C_0$. Let $D'$ be the divisor cut on $C$ by $H$ residual to $E + \Delta + A$; that is, we write
$$
(\nu^*H) = E + \Delta + A + D'.
$$
Finally, since $\nu^*(G/H)$ is a rational function on $C$, we see that 
$$
E + \Delta + A = (\nu^*H) \sim (\nu^*G) = E + \Delta + A + D',
$$
and we conclude that $D'$ is an effective divisor linearly equivalent to $D$ on $C$.

But, do we get in this way \emph{all} effective divisors linearly equivalent to $D$ on $C$? The answer is yes, but it's not immediate; it follows from the following proposition, known classically as \emph{completeness of the adjoint series}.

\begin{proposition}\label{adjoint completeness}
If $C_0 \subset \PP^2$ is a nodal plane curve and $\nu : C \to C_0$ its normalization, then the linear series cut on $C$ by plane curves of degree $m$ passing through the nodes is complete.
\end{proposition}

Note  that the solution to our problem follows from this proposition exactly as in the smooth case; that is, \emph{every} effective divisor $D' \sim D$ on $C$ is obtained in this way.

\begin{proof}
To prove Proposition~\ref{adjoint completeness}, it will be helpful to introduce another surface: the blow-up $\pi : S \to \PP^2$ of $\PP^2$ at the points $r_i$. The proper transform on $C_0 \subset \PP^2$ in $S$ is the normalization of $C_0$, which we will again call $C$.

There are two divisor classes on $S$ that will come up in our analysis: the pullback of the class of line in $\PP^2$, which we'll denote $H$; and the sum of the exceptional divisors, which we'll call $E$. In these terms, we have
$$
C \sim dH - 2E \quad \text{and} \quad K_S \sim -3H + E
$$
(the first follows from the fact that $C_0$ has multiplicity 2 at each of the points $q_i$, the second from considering the pullback to $S$ of a rational 2-form on $\PP^2$). If $A$ is a curve in $\PP^2$ of degree $m$ passing through the points $q_i$, we can associate to it the effective divisor $\pi^*A - E$; this gives us an isomorphism
$$
H^0(\cI_{\{q_1,\dots,q_\delta\}/\PP^2}(m)) \cong H^0(\cO_S(mH-E)).
$$
In these terms we can describe the linear series cut on $C$ by plane curves of degree $m$ passing through the nodes of $C_0$ as the image of the map
$$
H^0(\cO_S(mH-E)) \to H^0(\cO_C(mH-E)),
$$
and the proposition amounts to the assertion that this map is surjective.

The obvious way to prove this is to view this map as part of the long exact cohomology sequence associated to the exact sequence of sheaves
$$
0 \to \cO_S(mH-E-C) = \cO_S((m-d)H + E)  \to \cO_S(mH-E) \to \cO_C(mH-E) \to 0,
$$
from which we see that it will suffice to establish that $H^1(\cO_S((m-d)H + E)) = 0$. To do this, we apply Serre duality, which says that $H^1(\cL) \cong H^1(K_S\otimes \cL^{-1})^*$; in this instance it 
tells us that
$$
H^1(\cO_S((m-d)H + E)) \cong H^1(\cO_S((d-m-3)H))^*
$$
Now, the line bundle $\cO_S((d-m-3)H)$ is just the pullback to $S$ of the bundle $\cO_{\PP^2}(d-m-3)$, which has vanishing $H^1$; thus the Proposition will follow from the 
\begin{lemma}
Let $X$ be a smooth projective surface, and $\pi : Y \to X$ a blow-up. If $\cL$ is any line bundle on $X$, then
$$
H^1(Y, \pi^*\cL) = H^1(X, \cL).
$$
\end{lemma}
The lemma follows by applying the Leray spectral sequence, which relates the cohomology of $\cL$ on $Y$ to the cohomology of the direct image $\pi_*\pi^*\cL$ (Leray is particularly simple in this setting, since all higher direct images are 0), and the observation that $\pi_*\pi^*\cL \cong \cL$.
\end{proof}


\subsection{Existence of good projections}\label{projection section}

In this section, we want to verify the assertion made above that every smooth curve is birational to a nodal plane curve. In characteristic 0, this is straightforward, and we will give the argument here.

\begin{proposition}\label{nodal projection}
If $C \subset \PP^n$ is a smooth curve in projective space, and $\Lambda \cong \PP^{n-3} \subset \PP^n$ a general $(n-3)$-plane, then the projection $\pi_\Lambda : C \to \PP^2$ is birational onto its image, which will be a nodal curve.
\end{proposition}

\begin{proof}
To start with, we can reduce to the case $n=3$: a general $(n-4)$-plane $\Gamma \subset \PP^n$ will be disjoint from the secant variety to $C$, so that the projection $\pi_\Gamma : C \to \PP^3$ will embed $C$ as a smooth curve in $\PP^3$; since the composition of $\pi_\Gamma$ with projection from a general point in $\PP^3$ is the projection $\pi_\Lambda : C \to \PP^2$ from a general $(n-3)$-plane, we need only prove the proposition for the projection of a smooth space curve $C \subset \PP^3$ from a general point.

Now, when we project $C \subset \PP^3$ from a general point, we do expect to introduce singularities: a general $p \in \PP^3$ will lie on a finite number of secant lines, which will correspond to singular points of the image $C_0$. (Note that since a general $p$ cannot lie on infinitely many secants, the projection will be birational onto its image; thus we just have to make sure the singularities introduced are just nodes. This requires that we verify three things: that if $p \in \PP^3$ is a general point, then
\begin{enumerate}
\item $p$ does not lie on any tangent line to $C$;
\item $p$ does not lie on any trisecant line; and
\item for each of the secant lines $\overline{q,r}$ to $C$ containing $p$, the tangent lines $\TT_qC$ and $\TT_rC$ are skew.
\end{enumerate}
These three conditions ensure that the image $C_0 = \pi_p(C)$ does not have any cusps, triple points or tacnodes, respectively.

The first is immediate: there is only a 1-dimensional family of tangent lines to $C$, which sweep out a surface in $\PP^3$; a general $p \in \PP^3$ will not lie on that surface. Similarly, we claim that there is at most a 1-dimensional family of trisecant lines to $C$; this follows from the Castelnuovo Lemma (\ref{}), which says that a general plane in $\PP^3$ cannot contain a trisecant line to $C$. Finally, it can't be the case that for a general pair of points $q, r \in C$ we have $\TT_qC \cap \TT_rC \neq \emptyset$; otherwise the projection $\pi_{\TT_qC} : C \to \PP^1$ would have derivative everywhere 0, but be nonconstant. Thus the space of what were classically called \emph{stationary secants}---secant lines $\overline{p,q}$ with $\TT_pC \cap \TT_qC \neq \emptyset$---is at most 1-dimensional, and a general $p \in \PP^3$ will not lie on one.
\end{proof}

Note that the the second and third cases in this argument rely on the hypothesis of characteristic 0. Indeed, in characteristic $p > 0$ it may not be the case that a general projection of a smooth curve to $\PP^2$ is nodal---but it's still true that every curve is the normalization of a nodal plane curve: we have the slightly weaker

\begin{proposition}[Proposition~\ref{nodal projection} in characteristic $p$]\label{positive characteristic nodes}
Let $C\subset \PP^n$ be a smooth curve. If we re-embed $C$ by a Veronese map of sufficiently high degree---that is, we let $\nu_m : \PP^n \to \PP^N$ be the $m$th Veronese map, and let $\tilde C = \nu_m(C)$ be image of $C$---then the projection of $\tilde C$ from a general $\PP^{N-3}$ will be nodal.
\end{proposition}

\begin{exercise}
In the setting of Proposition~\ref{positive characteristic nodes}, for a general $\Lambda \cong \PP^{N-3}$ show that
\begin{enumerate}
\item there do not exist three points $p,q,r \in \tilde C$ such that $p, q, r$ and $\Lambda$ are all contained in a $\PP^{N-2}$; and
\item there does not exist a pair of points $p, q \in \tilde C$ such that 
$p, q$ and $\Lambda$ are contained in a $\PP^{N-2}$ and $\TT_p(\tilde C), \TT_q(\tilde C)$ and $\Lambda$ are contained in a $\PP^{N-1}$.
\end{enumerate}
\end{exercise}

From this one can deduce Proposition~\ref{positive characteristic nodes}.

\begin{exercise}
Let $C_0$ be a plane quartic curve with two nodes $q_1, q_2$; let $\nu : C \to C_0$ be its normalization, and let $o \in C$ be any point not lying over a node of $C_0$.
By the genus formula, $C$ has genus 1. Using the construction above, describe the group law on $C$ with $o$ as origin.
\end{exercise}

\section{Arbitrary plane curves}

Well, not exactly arbitrary: in this section, we'll deal with a plane curve $C_0$ that is assumed to be reduced and irreducible, with normalization $\nu : C \to C_0$. 

The geometry of singular curves is a fascinating topic, from the local analysis of the singularities to the global questions involving linear series on singular curves. Indeed, it's remarkable how many of the constructions and theorems we've discussed in the realm of smooth curves can be extended to the world of singular curves, given the right definitions (and some restrictions on the type of singularities, such as the Gorenstein condition). But this is a topic beyond our ken, at least in this book; for us, the questions are about smooth curves, with singular curves appearing as a useful adjunct. The description, in the last section, of complete linear series on a smooth curve $C$, using a nodal plane model $C_0$ of $C$, is a perfect example.

There is, however, one fundamental invariant of a singular curve $C$ that is both readily calculated and highly relevant in relating it to smooth curves: the \emph{arithmetic genus} of $C$. This will come up in what we're going to do next, which is to describe linear series on a smooth curve $C$ via a birational model as a plane curve with general singularities, and so we'll take a moment out here and introduce this notion.

\subsection{Arithmetic genus and geometric genus}

To start with, the arithmetic genus is a very broadly applicable: it is defined for an arbitrary one-dimensional scheme over a field. Recall that among the characterizations of the genus $g$ of a smooth projective curve $C$ there was one in terms of the Euler characteristic of the structure sheaf: we have $g = 1 - \chi(\cO_C)$. This is directly equivalent to the characterization in terms of the constant term of the Hilbert polynomial of any projective embedding.

And that is how we extend the notion of genus to arbitrary singular curves $C_0$: for any 1-dimensional scheme $C_0$ over a field, we define the \emph{arithmetic genus} of $C_0$ to be $1-\chi(\cO_{C_0})$. Very often (as, for example, right now!) we want also to deal at the same time with the genus of the normalization $\nu : C \to C_0$; to distinguish between these two notions of the genus of a singular curve, we call $1-\chi(\cO_{C_0})$ the arithmetic genus of $C_0$ and denote it $p_a(C_0)$; the genus of the normalization is called the \emph{geometric genus} and often denoted $g(C_0)$ or $p_g(C_0)$.

What's the difference? Well, we can relate the two notions via the map of sheaves
$$
\cO_{C_0} \to \nu_*\cO_C.
$$
This is injective; the cokernel sheaf will be a skyscraper sheaf supported exactly on the singular points of $C_0$. Denoting this sheaf by $\cF$, we have an exact sequence
$$
0 \to \cO_{C_0} \to \nu_*\cO_C \to \cF \to 0.
$$

Now, the normalization map $\nu: C \to C_0$ is finite, so that the higher direct images $R^i\nu_*\cO_C = 0$ for $i > 0$; it follows from the Leray spectral sequence that $\chi(\nu_*\cO_C) = \chi(\cO_C)$. We have, accordingly,
$$
p_a(C_0) - g(C) =  \chi(\cO_{C}) -   \chi(\cO_{C_0}) = \chi(\cF) = h^0(\cF);
$$ 
in other words, the difference between the arithmetic and geometric genera of $C_0$ is the sum of the vector space dimensions of the stalks on $\cF$; colloquially, it's the number of linear conditions a function $f$ on $C$ has to satisfy to be the pullback of a function from $C_0$. The length of the stalk of $\cF$ at a particular singular point $p \in C_0$ is called the \emph{$\delta$-invariant of the singularity}; to rephrase the statement above in these terms, we have
$$
p_a(C_0) - g(C) = \sum_{p \in (C_0)_{sing}} \delta_p
$$ 

Happily, the $\delta$ invariant of a singularity is readily calculated. Here are some examples:
\begin{enumerate}

\item (nodes) If $p \in C_0$ is a node, with points $r,s \in C$ lying over it, the condition for a function $f$ on $C$ to descend is simply that $f(r)=f(s)$; this is one linear condition and accordingly $\delta_p = 1$.

\item (cusps) If $p \in C_0$ is a cusp, with  $r \in C$ lying over it, the condition for a function $f$ on $C$ to descend is simply that the derivative $f'(r)=0$; again, this is one linear condition and accordingly $\delta_p = 1$.

\item (tacnodes) Suppose now that $p \in C_0$ is a \emph{tacnode}, that is, $C_0$ has two smooth branches at $p$ simply tangent to one another. There will be two points $r, s \in C$ lying over it, and the condition for a function $f$ on $C$ to descend is that in terms of suitable local coordinates both $f(r)=f(s)$ and $f'(r)=f'(s)$.  This represents two linear conditions and accordingly $\delta_p = 2$.

\item (planar triple points) Next up, consider an ordinary triple point $p \in C_0$ of a plane curve: that is, a singularity consisting of three smooth branches meeting pairwise transversely, such as the zero locus of $y^3-x^3$. There will be three points $r,s,t \in C$ lying over $p$, and certainly a necessary condition for a function $f$ on $C$ to descend is that $f(r)=f(s)=f(t)$---two linear conditions. But there's a third, less obvious linear condition: in order for $f$ to descend, the derivatives $f'(r), f'(s), f'(t)$ have to satisfy a linear condition---a reflection of the fact that a function on $C_0$ cannot vanish to order 2 on each of two branches without vanishing to order 2 along the third as well. Thus $\delta_p = 3$

\item (spatial triple points) We will be concerned in what follows only with planar singularities, but spatial triple points provide a useful contrast to the last example. A spatial triple point is a singularity consisting of three smooth branches, with linearly independent tangent lines, so that its Zariski tangent space is 3-dimensional. An example would be the union of the three coordinate axes in $\AA^3$.

In this case, in contrast to the last one, the condition that $f(r)=f(s)=f(t)$ is both necessary and sufficient for $f$ to descend, and accordingly we have $\delta_p=2$.

\end{enumerate}

\begin{exercise}
Let $p \in C$ be a singular point of a reduced curve $C$. Show that if $\delta_p = 1$, then $p$ must be either a node or a cusp.
\end{exercise}



\subsection{Linear series on (the normalization of) a plane curve} 

We return now to our basic setting: we have a reduced and irreducible curve $C_0 \subset \PP^2$ with normalization $\nu : C \to C_0$, and we want to extend our solution to the keynote problems---finding all regular differentials on $C$, and finding all divisors on $C$ linearly equivalent to a given $D \in \Div(C)$---to this more general setting.

In this setting, if we simply try to mimic the analysis above in the nodal case we're led to introduce the \emph{adjoint ideal} of each singularity (which is simply the maximal ideal of the point in the case of a node), in terms of which we have theorems analogous to the results obtained above in the nodal case.

So: let $C_0 \subset \PP^2$ be a reduced and irreducible plane curve, with normalization $\nu : C \to C_0$. We focus for now on one singular point $q \in C_0$, with points $r_1,\dots,r_k \in C$ lying over $q$. If our goal is to describe the canonical series on $C$, we can start as we did in the previous two sections: by considering differentials of  the form $g(x,y)\omega_0$, where
$$
\omega_0 = \nu^* \frac{dx}{\partial f/\partial y},
$$
and $f$ is the defining equation of $C_0$ in an affine open containing $q$. As we saw in the nodal case, $\omega_0$ will have poles at the points $r_i$; let $m_i$ be the order of the pole of $\omega_0$ at $r_i$. We can define the \emph{adjoint ideal} of $C_0$ at $q$ to be the ideal
$$
A_q = \left\{ g \in \cO_{\PP^2, q} \mid \ord_{r_i}(\nu^*g) \geq m_i \; \forall i \right\}
$$

In other words, $A$ is the ideal of functions $g$ such that $\nu^* \frac{gdx}{\partial f/\partial y}$ is regular at all the points $r_i$. We accordingly define the \emph{adjoint ideal $\cI_A$ of $C_0$} to be the product of $A_q$ over all singular points  $q \in C_0$; the \emph{adjoint series of degree $m$} is then the linear series $H^0(\cI_A(m))$. 

In these terms, we can give the solution to our keynote problems much as we did in the case of plane curves with nodes. Specifically:

First, we can say that every global regular 1-form on the curve $C$ is of the form 
$$
\frac{g(x,y) dx}{\partial f/\partial y},
$$
with $g$ in the adjoint ideal $A$, and of degree $d-3$ or less.

Second, if we are given a divisor $D = E - F$ on the curve $C$, we can find all effective divisors $D'$ on $C$ linearly equivalent to $D$ exactly as we did in the previous case. We start by choosing a polynomial $G$ of any degree $m$ in the adjoint ideal and vanishing on $E$ but not vanishing identically on $C_0$; we write
$$
(G) = E + \Delta + A.
$$
We then find all polynomials $H$ of degree $m$ in the adjoint ideal, vanishing on $A + F$ but not vanishing identically on $C_0$; writing
$$
(H) = F + A + \Delta + D'
$$
we arrive at an effective divisor $D'$ on $C$ linearly equivalent to $D$. Indeed,  the analog of the theorem of completeness of the adjoint series tells us that we arrive at \emph{every} effective divisor $D'$ on $C$ linearly equivalent to $D$ in this way.



At this point it may seem that without an explicit description of the adjoint ideal we have merely slapped a label on our ignorance. But in fact, the adjoint ideal is relatively straightforward to find. To begin with, let's do some simple examples:

\begin{example}[nodes and cusps]
We have already seen that in case $q$ is a node of $C_0$, there are two points of $C$ lying over it, and $m_1=m_2=1$; the adjoint ideal is thus just the maximal ideal $\cI_q$ at $q$. In the case of a cusp, for example the zero locus of $y^2-x^3$, there is only one point $r=r_1$ of $C$ lying over $q$, and the differential $\omega_0$ vanishes to order $m_1=2$; since the pullback to $C$ of any polynomial $g$ vanishing at $q$ will vanish to order at least two at $r$, and so again the adjoint ideal is again just the maximal ideal at $q$.
\end{example}


\begin{example}[tacnodes]
Next, consider the case of a \emph{tacnode}; that is, a singularity with two smooth branches simply tangent to one another, such as the zero locus of $y^2-x^4$. In this case there are again two points of $C$ lying over $q$, and a simple calculation shows that $m_1=m_2=2$. The adjoint ideal is thus the ideal of functions vanishing at $q$ and having derivative 0 in the direction of the common tangent line to the branches.
\end{example}


\begin{example}[ordinary triple points]
In the case of an ordinary triple point---three smooth branches simply tangent to one another pairwise---there are three points of $C$ lying over $q$, and we have $m_1=m_2=m_3= 2$; the adjoint ideal is correspondingly just the square of the maximal ideal at $q$
\end{example}

\begin{exercise}
Find the adjoint ideals of the following plane curve singularities:
\begin{enumerate}
\item a \emph{triple tacnode}: three smooth branches, pairwise simply tangent
\item a triple point with an infinitely near double point: three smooth branches, two of which are simply tangent, with the third transverse to both
\item a unibranch triple point, such as the zero locus of $y^3-x^4$
\end{enumerate}
\end{exercise}

In general, the adjoint ideal of an isolated plane curve singularity is something we can determine in practice; for example, here is a simply general description in case the individual branches of $C_0$ at $p$ are each smooth:

\begin{proposition}
Let $\nu : C \to C_0$ be the normalization of a plane curve $C_0$ and $p \in C_0$ a singular point. Denote the branches of $C_0$ at $p$ by $B_1,\dots,B_k$, and let $r_i$ be the point in $B_i$ lying over $p$. If the individual branches $B_i$ of $C_0$ at $p$ are each smooth, and we set
$$
m_i = \sum_{j \neq i} mult_p(B_i \cdot B_j
$$
then the adjoint ideal of $C_0$ at $p$ is simply the ideal of functions $g$ such that $ord_{r_i}(\nu*g) \geq m_i$.
\end{proposition}

\subsection{The conductor ideal}

There is another ideal we can associate to an isolated curve singularity, called the \emph{conductor ideal}. It's simple to define: if $C_0$ is a reduced curve and $\nu : C \to C_0$ its normalization, we can think of the direct image $\nu_*\cO_C$ as a module over the structure sheaf $\cO_{C_0}$; the conductor ideal is simply the annihilator of the quotient $\nu_*\cO_C/\cO_{C_0}$. In concrete terms, on any affine open $U \subset C_0$ this is the ideal of functions $g \in \cO_{C_0}(U)$ such that for any function $h \in \cO_C(\nu^{-1}(U))$ the product $hg$ will be the pullback of a function on $C_0$.

For example, in the case of a node $q \in C_0$, with $r_1,r_2$ the points of $C$ lying over $q$, we see that any function on $C$ vanishing at both $r_1$ and $r_2$ is the pullback of a function on $C_0$; thus the conductor is simply the maximal ideal at $q$. Similarly, if $q \in C_0$ is a cusp, with $r \in C$ lying over it, a function $f$ on $C$ will descend to $C_0$---that is, be the pullback of a function on $C_0$---if it vanishes to order 2 at $r$; since the pullback to $C$ of any function on $C_0$ vanishing at $q$ vanishes to order at least 2 at $r$, the conductor ideal is just the maximal ideal at $q$.

The sharp-eyed reader will have noticed a coincidence here, and will not be completely surprised by the

\begin{theorem}
For any plane curve singularity, the adjoint ideal and the conductor ideal coincide!
\end{theorem}

This is a reflection of the fact that plane curve singularities are necessarily local complete intersections, and hence \emph{Gorenstein}, about which we will hear more in the following chapter.

%Throughout this book thus far, we have developed techniques for dealing with smooth, projective curves; to the extent that we have considered singular curves we have studied them by applying the ideas and constructions we've developed to their normalizations. But singular curves have a fascinating geometry in their own right, not only for the singularities themselves but the effect singularities have on associated objects such as the Jacobian. In this chapter, we will undertake a brief survey of the geometry of singular curves and what we can say about linear series on them.
%
%To say what classes of curves we'll be dealing with here: for the most part, we will confine ourselves to working with reduced, projective curves. Many of the results we will derive will in fact will be applicable to a larger class of curves, namely those that are \emph{Cohen-Macaulay}; these may well be non-reduced, but cannot have embedded points. In Section~\ref{} below, we'll mention some examples of nonreduced curves to which we can apply our ideas (such as, for example, \emph{ribbons}), and we'll develop these ideas further in the following chapter, where we introduce more algebraic techniques. But for the rest of this chapter, we will take the objects of our study to be reduced, projective curves over $\CC$.
%
%
%\section{The arithmetic genus and singularities}
%
%To start at the beginning, when we first defined the notion of the \emph{genus} of a smooth projective curve, we gave several different but equivalent characterizations of the genus. As you might expect, these diverge in the presence of singularities, so we adopt the following universal definition.
%
%\begin{definition}
%Let $C$ be an arbitrary one-dimensional projective scheme over a field $\CC$. By the \emph{arithmetic genus} $p_a(C)$ of $C$ we mean 1 minus the Euler characteristic of the structure sheaf of $C$:
%$$
%p_a(C) \; = \; 1 - \chi(\cO_{C}).
%$$
%In contrast, if $C$ is reduced, we define the \emph{geometric genus} to be the genus of the normalization $C^\nu$ of $C$.
%\end{definition}
%
%Note that the arithmetic genus satisfies many of the formulas derived above in the smooth case: for example, if $C \subset S$ is a divisor on a smooth surface, the adjunction formula holds:
%$$
%p_a(C) \; = \; \frac{C\cdot C + K_S\cdot C}{2} + 1.
%$$
%Thus, for example, a double conic curve $C = V((XY-Z^2)^2) \subset \PP^2$, like every other plane quartic curve, has arithmetic genus 3. (If you're curious, we'll see what the Jacobian of $C$ looks like in Section\ref{} below.)
%
%\subsection{Relation between the arithmetic and geometric genus}
%
%Our first order of business is to understand the relationship between the arithmetic and geometric genera of a reduced projective curve $C$. To this end, let $\nu : C^\nu \to C$ be the normalization of $C$, and consider the exact sequence of sheaves on $C$:
%$$
%0 \to \cO_C \to \nu_*\cO_{C^\nu} \to \cF \to 0
%$$
%where $\cF$ is simply defined to be the quotient $\nu_*\cO_{C^\nu}/\cO_C$; note that $\cF$ is supported exactly at the singular points of $C$.
%
%The point here is that, because the map $\nu$ is finite, there are no higher direct images of $\nu_*\cO_{C^\nu}$; so the Leray-Serre spectral sequence tells us that
%$$
%\chi(\nu_*\cO_{C^\nu}) \; = \chi(\cO_C).
%$$
%Thus, the difference between the arithmetic and geometric genera of $C$ is
%$$
%g(C) \; = \; p_a(C) - h^0(\cF).
%$$
%We can refine this a little: for each singular point $p \in C$, denote by $\delta_p$ the dimension (as vector space over $\CC$) of the stalk $\cF_p$ of $\cF$ at $p$. This is called the \emph{delta-invariant} of $p$, and is the most fundamental numerical invariant of a curve singularity. In these terms, we can write
%$$
%g(C) \; = \; p_a(C) - \sum_{p \in C} \delta_p.
%$$
%
%\subsection{The delta invariant}
%
%To see how this works in practice, let's calculate the delta invariant of some relatively simple singularities.
%
%\subsubsection{the $\delta$-invariant of a node}
%
%To start with the simplest singularity, suppose that $p$ is a node of $C$: that is, a neighborhood of $p \in C$\footnote{we can take this to be either a complex analytic neighborhood or an \'etale neighborhood} is the union of two smooth curves intersecting transversely at $p$. In this case, there will be two points $q, r \in C^\nu$ in the normalization lying over $p$, and we see that if $U$ is a suitably small neighborhood of $p \in C$ then a function $\tilde f \in \cO_{C^\nu}(\nu^{-1}(U)$ on the preimage of $U$ in $C^\nu$  is the pullback of a function $f \in \cO_C(U)$ if and only if the values $\tilde f(q) = \tilde f(r)$ agree. It follows that the stalk of the sheaf $\cF$ at $p$ is one-dimensional, so $\delta_p = 1$.
%
%\subsubsection{the $\delta$-invariant of a cusp}
%
%By definition, a \emph{cusp} of a curve $C$ is a point $p \in C$ such that the normalization map is given in terms of suitable local coordinates in a neighborhood $U$ of $p$ as
%$$
%\nu : t \mapsto (t^2, t^3).
%$$
%We see from this that a function  is the pullback of a function $f \in \cO_C(U)$ if and only if the derivative $f'(0) = 0$; this being again one linear condition, we see that $\delta_p = 1$.
%
%\subsubsection{the $\delta$-invariant of a tacnode}
%$\tilde f \in \cO_{C^\nu}(\nu^{-1}(U))$
%As with a node, a suitably small neighborhood $U$ of a tacnode $p \in C$ is a union of two smooth curves; but this time the two branches are simply tangent to one another rather than transverse. It is no longer the case, accordingly, that a function $\tilde f \in \cO_{C^\nu}(\nu^{-1}(U))$ is a pullback if and only if the values of $\tilde f$ at the points  $q, r \in C^\nu$ lying over $p$ agree: for example, a function that has a simple zero at $q$ but vanishes to order 2 or more at $r$ cannot be a pullback. The correct statement is that $\tilde f$ will descend to $C$ if and only if $f(q) = f(r)$ and the derivatives $f'(q)$ and $f'(r)$ (with respect to suitably chosen local coordinates) agree. Thus we see that $\delta_p = 2$.
%
%\subsubsection{the $\delta$-invariant of an ordinary triple point}
%
%An ordinary triple point is again a \emph{planar} singularity, meaning its Zariski tangent space is 2-dimensional, or, equivalently, it is embeddable in a smooth surface. A small neighborhood of such a point $p \in C$ consists of a union of three smooth branches, intersecting pairwise transversely at $p$. In order for a function on the preimage of $U$ to be a pullback, naturally, its values at the three points $q,r,s \in \nu^{-1}(p)$ lying over $p$ have to agree, so that $\delta_p \geq 2$. But this is not a sufficient condition: a function  $\tilde f$ with a simple zero at $q$ but vanishing to order at least 2 at $r$ and $s$ cannot be a pullback. Rather, the fact that the tangent lines to the three branches are linearly dependent means that in order for $\tilde f$ to descend its derivatives at $q, r$ and $s$ must satisfy a linear relation as well, and so we have $\delta_p = 3$. 
%
%\subsubsection{the $\delta$-invariant of a spatial triple point}
%
%By way of contrast with the last example, suppose now that $p$ is a \emph{spatial triple point}: in other words, a neighborhood of $p \in C$ is a union of three smooth branches with linearly independent tangent lines (so that $\dim T_p(C) = 3$). In this case, it is a necessary and sufficient condition for a function $\tilde f$ to descend is simply that its values at the points lying over $p$ agree, from which we see that $\delta_p = 2$.
%
%\begin{exercise}
%Consider the following curves $C \subset \PP^3$ of degree 3 in $\PP^3$. In each case, determine the arithmetic genus; and, in those cases where $p_a(C) = 0$ show that the curve $C$ is indeed the flat limit of a family of twisted cubics.
%\begin{enumerate}
%\item the union of a line and a conic curve meeting transversely at one point;
%\item the union of a conic and a tangent line;
%\item the union of three concurrent, coplanar lines;
%\item the union of three concurrent but not coplanar lines.
%\end{enumerate}
%\end{exercise}
%
%\section{The dualizing sheaf and Riemann-Roch for singular curves}
%
%Without question, the most fundamental object we deal with in analyzing the geometry of a smooth curve $C$ is its canonical bundle/divisor class $K_C$, defined simply to be the cotangent bundle of $C$ or equivalently the sheaf of regular 1-forms. Is there an analog of this in the case of singular curves?
%
%The answer is an emphatic ``yes:" if $C$ is Cohen-Macaulay (and in particular, if $C$ is reduced), we can introduce the \emph{dualizing sheaf} $\omega_C$, which is a more than adequate understudy for the role of canonical bundle. (Indeed, in a large range of cases, those of \emph{Gorenstein} curves, it is locally free, as we'll see in the subsection below.) We will give a relatively concrete description of the dualizing sheaf here, and a more abstract, algebraic definition in the next chapter.
%
%To understand the definition of the dualizing sheaf, it is useful (and amusing) to recall a bogus proof of the Riemann-Roch formula, which highlights the key property of the canonical bundle. To set this up, suppose now that $C$ is a smooth projective curve of genus $g$, and imagine we have a divisor $D = p_1 + \dots + p_d$ consisting of $d$ distinct points on $C$. We choose a local coordinate $z_i$ on $C$ around $p_i$, and ask: given a $d$-tuple of scalars $a_1,\dots,a_d$, when is there a rational function on $C$, regular away from the points $p_i$, with polar part $a_i/z_i$ at $p_i$? 
%
%Since any rational function regular away from the $p_i$ is determined, up to the addition of a scalar, by its polar parts at the $p_i$, this is tantamount to asking for the dimension of the vector space $\cL(D)$ of rational functions with at most simple poles along $D$. In particular, we see that the dimension
%$$
%\dim \c(D) \; \leq \; d + 1,
%$$
%with equality holding iff every $d$-tuple of scalars $a_1,\dots,a_d$ represents the polar part of some function $f \in \cL(D)$. But there is an obstruction to this being the case: if $\phi$ is any global regular 1-form on $C$, then for any $f \in \cL(D)$ we have
%$$
%\sum_i Res_{p_i} (f \cdot \phi) \; = \; 0,
%$$
%which (potentially) imposes a linear condition on the polar parts of $f$. Of course, if $\omega$ vanishes at all the points $p_i$ of $D$, this condition is vacuous; the actual number of conditions imposed is the difference $g - h^0(K_C -D)$. Altogether, then, we have established the inequality
%$$
%h^0(D) \; \leq \; d + 1 - g + h^0(K_C - D).
%$$
%
%Now we apply this inequality to the divisor $K-D$, which has degree $2g-2-d$; we arrive at
%$$
%h^0(K-D) \; \leq \; 2g - 2 + 1 - g + h^0(D).
%$$
%Finally, we add the last two inequalities, and almost all the terms cancel, leaving us with
%$$
%h^0(D) + h^0(K-D) \; \leq \; h^0(K-D) + h^0(D);
%$$
%since equality holds here, it most hold in both the inequalities above, and we deduce the statement of the Riemann-Roch formula.
%
%The point of this derivation of Riemann-Roch is that is emphasizes the crucial fact underlying the formula: that \emph{the sum of the residues of a rational 1-form on a smooth projective curve is zero}. Now, suppose that $C$ is a possibly singular projective curve, and $D = p_1+\dots + p_d$ a divisor $D$ (whose support we will for simplicity assume is disjoint from the singular locus of $C$). Say we want to derive an analogous formula for the dimension $h^0(D)$ of the space of rational functions on $C$ with at most poles along a divisor $D$, with the role of a global regular 1-form $\phi \in H^0(K)$ played by a global section of the dualizing sheaf $\omega_C$. We need to define the dualizing sheaf $\omega_C$ so that its sections may be viewed as rational differentials on $C$, with three properties:
%
%\begin{enumerate}
%\item we should have $\deg(\omega_C) = 2p_a(C)-2$;
%\item we should have $h^0(\omega_C) = p_a(C)$; and, crucially,
%\item for any rational function $f \in h^0(D)$ and every section $\phi \in H^0(\omega_C)$, the sum of the residues of the rational differential $f\phi$ at the points of $D$ is 0.
%\end{enumerate}
%
%
%
%\subsection{the Gorenstein condition}
%
%\section{Picard groups of singular curves}
%
%Let's start with the simplest case: suppose $C$ is an irreducible curve with a node $r$ and no other singularities. We'll denote by $\nu : C^\nu \to C$ the normalization of $C$, and let $p, q \in C^\nu$ be the two points of $C^\nu$ lying over $r$.
%
%Let $\Pic^0(C)$ denote the group of line bundles of degree 0 on $C$, and $\Pic^d(C)$ the set of line bundles of degree $d$ (which is again a principal homogeneous space for $\Pic^0(C)$). To describe these, consider the map
%$$
%\nu^* : \Pic^0(C) \to \Pic^0(C^\nu)
%$$
%given simply by associating to a line bundle $\cL \in \Pic^0(C)$ its pullback $\nu^*\cL$ to $C^\nu$. If $\cL \in \Pic^0(C)$ is any line bundle on $C$, the fibers $(\nu^*\cL)_p$ and $(\nu^*\cL)_q$ of the pullback are each identified with the fiber $\cL_r$, and so with each other; conversely, if $\cM$ is any line bundle on $C^\nu$ and $\phi : \cM_p \cong \cM_q$ any isomorphism between the fibers of $\cM$ at $p$ and $q$, we can identify the fibers to arrive at a line bundle $\cL$ on $C$ whose pullback to $C^\nu$ is $\cM$ with the specified identification of fibers.
%
%We see thus that the map $\nu^*$ above is surjective, with fibers isomorphic to $\CC^*$; that is, we have an exact sequence
%$$
%0 \to \CC^* \to \Pic^0(C) \to \Pic^0(C^\nu) \to 0.
%$$
%In other words, $\Pic^0(C)$ is a $\CC^*$-bundle over $\Pic^0(C^\nu)$ (though not a product). \fix{to describe the extension, we need the Poincar\'e bundle of $C^\nu$---is this something we're planning to introduce?}
%
%A similar description applies to an irreducible curve $C$ with an ordinary cusp $r$ and no other singularities. Again,  denote by $\nu : C^\nu \to C$ the normalization map, with $p \in C^\nu$ the point lying over the cusp of $C$. Just as in the nodal case, the pullback $\nu^*\cL$ of a line bundle $\cL$ on $C$ comes equipped with a trivialization over the scheme-theoretic preimage $\nu^{-1}(p)$; the difference is that now $\nu^{-1}(p) = 2r$ is a double point. The space of such trivializations is now a copy of $\CC$ rather than $\CC^*$, so we have instead a sequence
%$$
%0 \to \CC \to \Pic^0(C) \to \Pic^0(C^\nu) \to 0.
%$$
%
%We note in passing that these descriptions of the Picard variety $\Pic^0(C)$ can also be arrived at via a variant of Abel's theorem. For example, in the nodal case, if we let $H^0(\omega_C)^*$ be the space of linear functions on the space of sections of the dualizing sheaf $\omega_C$, we can define an inclusion of the first homology group $H_1(C,\ZZ)$ (the condition that a section of $\omega$ correspond to a rational differential on $C^\nu$ with opposite residues at $p$ and $q$ allows us to define the integral of such a form along a loop passing through the node).  \fix{need a picture here}
%We can then define the Jacobian to be the quotient $H^0(\omega_C)^*/H_1(C,\ZZ)$, and a variant of the Abel-Clebsch theorem tells us that this is naturally identified with $\Pic^0(C)$.
%
%The difference here is that, while $H^0(\omega_C)^*$ is still a complex vector space of dimension $g$, the lattice $H_1(C,\ZZ)$ has rank only $2g-1$: it contains the homology of $C^\nu$, with one additional generator corresponding to a loop on $C$ passing through the node. The quotient is therefore not compact, and we arrive at the same picture of $\Pic^0(C)$ as a $\CC^*$-bundle over $\Pic^0(C^\nu)$. 
%
%We conclude this discussion with some terminology and one important fact. To start with, the object $\Pic^0(C)$ described here may be called simply the Jacobian of $C$; in some sources, however, it is called a \emph{generalized Jacobian}; the term ``Jacobian" is reserved for Jacobians of smooth curves, which are abelian varieties. Similarly, the sort of algebraic group arising here---an extension of an abelian variety by a product of $\CC^*$s and $\CC$s---is often called a \emph{semi-abelian variety}, to distinguish it from abelian varieties.
%
%\begin{fact}
%Picard varieties fit in families: if $\pi : \cC \to B$ is a flat, projective morphism whose fibers  are irreducible  curves having at worst nodes and cusps, the Picard varieties $\{\Pic^0(C_b)\}_{b \in B}$ likewise form a flat family: that is, there exists an associated morphism $\cP \to B$ whose fiber over each point $b \in B$ is the Picard variety of the corresponding curve $C_b$.
%\end{fact}
%
%\subsection{Compactifying the Jacobian}  
%
%Many of the applications of the Jacobian we gave in Chapter~\ref{} depended only on the fact that the Jacobian of a smooth curve is irreducible of dimension $g$, and those continue to hold in the case of nodal and/or cuspidal curves. For example, we have the
%
%\begin{exercise}
%Let $C$ be a projective curve of genus $g$ having only nodes and cusps as singularities. Show that a general line bundle of degree $g+3$ on $C$ is very ample.
%\end{exercise}
%
%On the other hand, many of the deeper applications of the Jacobian rely essentially on the fact that it is a complete variety, and here we need to modify our construction if we are to port over these results. We need, in other words, to \emph{compactify} the Jacobians of singular curves, if we can, and this is what we'll describe below.

%footer for separate chapter files

\ifx\whole\undefined
%\makeatletter\def\@biblabel#1{#1]}\makeatother
\makeatletter \def\@biblabel#1{\ignorespaces} \makeatother
\bibliographystyle{msribib}
\bibliography{slag}

%%%% EXPLANATIONS:

% f and n
% some authors have all works collected at the end

\begingroup
%\catcode`\^\active
%if ^ is followed by 
% 1:  print f, gobble the following ^ and the next character
% 0:  print n, gobble the following ^
% any other letter: normal subscript
%\makeatletter
%\def^#1{\ifx1#1f\expandafter\@gobbletwo\else
%        \ifx0#1n\expandafter\expandafter\expandafter\@gobble
%        \else\sp{#1}\fi\fi}
%\makeatother
\let\moreadhoc\relax
\def\indexintro{%An author's cited works appear at the end of the
%author's entry; for conventions
%see the List of Citations on page~\pageref{loc}.  
%\smallbreak\noindent
%The letter `f' after a page number indicates a figure, `n' a footnote.
}
\printindex[gen]
\endgroup % end of \catcode
%requires makeindex
\end{document}
\else
\fi
