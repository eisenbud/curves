%header and footer for separate chapter files

\ifx\whole\undefined
\documentclass[12pt, leqno]{book}
\usepackage{graphicx}
\usepackage{eps-to-pdf}
\input style-for-curves.sty
%\input sl-macros.sty
\usepackage{hyperref}
\usepackage{showkeys} %This shows the labels.
\usepackage{msribib}
\usepackage{pdfpages}
\usepackage{draftwatermark}
\SetWatermarkText{DRAFT:\ \today}
\SetWatermarkScale{2}
\SetWatermarkColor[gray]{0.9}

%\usepackage{SLAG,msribib,local}
%\usepackage{amsmath,amscd,amsthm,amssymb,amsxtra,latexsym,epsfig,epic,graphics}
%\usepackage[matrix,arrow,curve]{xy}
%\usepackage{graphicx}
%\usepackage{diagrams}
%
%%\usepackage{amsrefs}
%%%%%%%%%%%%%%%%%%%%%%%%%%%%%%%%%%%%%%%%%%
%%\textwidth16cm
%%\textheight20cm
%%\topmargin-2cm
%\oddsidemargin.8cm
%\evensidemargin1cm
%
%%%%%%Definitions
%\input preamble.tex
%\input style-for-curves.sty
%\def\TU{{\bf U}}
%\def\AA{{\mathbb A}}
%\def\BB{{\mathbb B}}
%\def\CC{{\mathbb C}}
%\def\QQ{{\mathbb Q}}
%\def\RR{{\mathbb R}}
%\def\facet{{\bf facet}}
%\def\image{{\rm image}}
%\def\cE{{\cal E}}
%\def\cF{{\cal F}}
%\def\cG{{\cal G}}
%\def\cH{{\cal H}}
%\def\cHom{{{\cal H}om}}
%\def\h{{\rm h}}
% \def\bs{{Boij-S\"oderberg{} }}
%
%\makeatletter
%\def\Ddots{\mathinner{\mkern1mu\raise\p@
%\vbox{\kern7\p@\hbox{.}}\mkern2mu
%\raise4\p@\hbox{.}\mkern2mu\raise7\p@\hbox{.}\mkern1mu}}
%\makeatother

%%
%\pagestyle{myheadings}

%\input style-for-curves.tex
%\documentclass{cambridge7A}
%\usepackage{hatcher_revised} 
%\usepackage{3264}
   
\errorcontextlines=1000
%\usepackage{makeidx}
\let\see\relax
\usepackage{makeidx}
\makeindex
% \index{word} in the doc; \index{variety!algebraic} gives variety, algebraic
% PUT a % after each \index{***}

\overfullrule=5pt
\catcode`\@\active
\def@{\mskip1.5mu} %produce a small space in math with an @

\title{A Chapter from ``The Practice of Algebraic Curves"}
\author{\copyright David Eisenbud and Joe Harris}
%%\includeonly{%
%0-intro,01-ChowRingDogma,02-FirstExamples,03-Grassmannians,04-GeneralGrassmannians
%,05-VectorBundlesAndChernClasses,06-LinesOnHypersurfaces,07-SingularElementsOfLinearSeries,
%08-ParameterSpaces,
%bib
%}

\date{\today}
%%\date{}
%\title{Curves}
%%{\normalsize ***Preliminary Version***}} 
%\author{David Eisenbud and Joe Harris }
%
%\begin{document}

\begin{document}
\maketitle

\pagenumbering{roman}
\setcounter{page}{5}
%\begin{5}
%\end{5}
\pagenumbering{arabic}
\tableofcontents
\fi


\def\adj{{\mathfrak F}}
\chapter{Using a singular plane model}
\label{PlaneCurvesChapter}
\label{PlaneCurveChapter}

In the first part of Chapter~\ref{genus 1 chapter} we showed how to use an embedding of a smooth curve $C$
in $\PP^2$ to understand differentials and linear series on $C$. But most smooth curves cannot be embedded in
$\PP^2$ so, as such, the technique is not generally applicable. However, by Proposition~\ref{nodal projection},
 any smooth curve $C$ can be projected 
birationally to a curve $C_0\subset\PP^2$ with only nodes. In the first part of this chapter we will show
how to use such a nodal model $C_0$ to describe differentials and linear series on $C$, a theory well-understood
by Brill and Noether;  and then explain
what is necessary to adapt the technique to birational images of $C$ with arbitrary singularities.

\section{Nodal plane curves}\label{nodal curves section}

The methods of Chapter~\ref{3b}
 can be applied, with one change, when $C_{0}\subset \PP^{2}$
is a nodal curve. 

Let $\nu: C\to C_{0}\subset \PP^{2}$ be the normalization morphism from a smooth curve,
and let $B\subset C_{0}$ be a subscheme.
It will be convenient to speak of \emph{the linear series cut out on $C$} by curves of degree $m$
containing $B$: though $B$ is only a subscheme, $\nu^{-1}(B)$ may be considered as a Cartier divisor because
$C$ is smooth, and we define
the linear series cut out on $C$ by curves of degree $m$
containing $B$ to be
the linear series  $\sV$ of divisors in $C$ that are residual to $\nu^{-1}(B)$ in the pullbacks 
of the intersections of $C_{0}$ with curves of degree $m$. More formally, since $B' := \nu^{-1}(B)$
is a Cartier divisor on $C$, we can say that if $L$ is the divisor of a line in $\PP^{2}$ and $L'$
its pullback to $C$ then
$\sV$ is a space of divisors corresponding to
sections of $\sO_{C}(mL'-B')$; more precisely, it is the space of divisors corresponding to
sections of $\sO_{C}(mL'-B')$ in the image of the restriction/pullback map
$$
H^0(\cI_{B/\PP^2}(m)) \to H^0(\sO_{C}(mL'-B')).
$$

\subsection{Differentials on a nodal plane curve}\label{canonical series on nodal plane curves}

Let $C_{0} \subset \PP^2$  be a curve of degree $d$ with $\delta$ nodes and no other singularities. By the adjunction
formula (Proposition~\ref{adjunction}), Proposition~\ref{pa and delta}, and the first example that follows it, 
the genus $g$ of the normalization $C$ of $C_{0}$ is
the arithmetic genus $p_{a}(C_{0}) = \binom{d-1}{2}$ of $C_{0}$ minus $\delta$, that is,
$$
g = \binom{d-1}{2} -\delta.
$$
We will make this explicit by exhibiting a vector space of $g$ regular differential forms on $C$.

Choose homogeneous coordinates  $[X,Y,Z]$ on $\PP^2$ so that the curve $C_0$ intersects the line $L = V(Z)$ in a divisor $D$ consisting only of smooth points of $C_{0}$  other than $[0,1,0]$. In particular,  all the nodes of $C_0$ will lie in the affine plane $U = \PP^2 \setminus L$.
We may harmlessly assume that  neither branch of $C_0$ at a node has vertical tangent. (These conditions are satisfied by a general choice of coordinates.) Let the nodes of $C_0$ be $q_1,\dots,q_\delta$, with $r_i, s_i \in C$ lying over $q_i$; we'll denote by $\Delta$ the divisor $\sum r_i + \sum s_i$ on $C$.

Let $F(X,Y,Z)$ be the homogeneous polynomial of degree $d$ defining the curve $C_0$, and let $f(x,y) = F(x,y,1)$ be the defining equation of the affine part $C_{0}^{\circ}:= C_0 \cap U$ of $C_0$. Let $\nu: C\to C_0$ be the normalization map. We start by considering the rational differential 
$\nu^*(dx)$ on 
$C^{\circ}:= \nu^{-1}(C_{0}^{\circ})$. 

In the smooth case where $C_{0}=C$ we saw that this differential was regular and nonzero on $C^{\circ}$; this followed from the fact that 
that $f_{x}$ and $f_{y}$ had no common zeroes on $C_0$. But now $f_{x}$ and $f_{y}$ have common zeroes: they both vanish to order 1 at the points $q_{i}$ and thus $\nu^*(f_{x})$ and $\nu^*(f_{y})$ have simple zeroes at the points $r_i$ and $s_i$. 

As before, the differential $\nu^*dx$ has  double poles along the divisor $D$ on $C_{0}$ lying over the point at infinity in $\PP^{1}$
and we see that for a polynomial $e(x,y)$ of degree $\leq d-3$, the differential
$$
\nu^*\left( \frac{e(x,y)dx}{f_{y}}\right)
$$
is regular except for simple poles at the points $r_i$ and $s_i$.

We can get rid of these poles by requiring that $e$ vanishes at the points $q_i$. We say in this case that $e$ (and the curve defined by $e$) \emph{satisfies the conditions of adjunction}. 

\begin{theorem}\label{canonical from adjoint 1}
If $C_{0}$ is a nodal plane curve of degree $d$ with normalization $\nu: C\to C_{0}$
then the  regular differentials on  $C$, in terms of the notation above, 
 are precisely those of the form
 $$
\nu^{*}\biggl( \frac{e(x,y)dx}{f_{y}}\biggr)
$$
where 
$e(x,y)$ ranges over the polynomials of degree $\leq d-3$
vanishing at the nodes of $C_{0}.$

Thus if $\adj(C_{0})\subset C_{0}$ denotes the union
of the reduced points at the nodes of $C$, then $|\omega_{C}|$ is the linear series cut out on $C$ by 
forms of degree $d-3$ containing $\adj(C_{0})$.
\end{theorem}
See Figure~\ref{canonical on normalization} for a picture in the case where $C_0$ has a single node.
\begin{proof}
The dimension of the space of polynomials $e(x,y)$ of degree at most $d-3$ is $\binom{d-1}{2}$,
and vanishing at $\delta$ nodes imposes at most $\delta$ linear conditions on $e$. The linear map sending
$e\mapsto \nu^{*}(edx/f_{y})$ is injective, and the target has dimension 
$\binom{d-1}{2}-\delta$, so this must be an isomorphism.
\end{proof}
We will give a more conceptual proof of this theorem in Section~\ref{arbitrary plane curves}.

In particular, Theorem~\ref{canonical from adjoint 1}
shows that the linear series cut out on $C$ by 
forms of degree $d-3$ containing $\adj(C_{0})$ is complete. (We will soon see that
 the linear series cut out on $C$ by 
forms of degree $m$ containing $\adj(C_{0})$ is complete for every $m$.)


This gives another proof of Lemma~\ref{adjoint independent}.

\begin{corollary}
If $C$ is a nodal plane curve of degree $d$, then the nodes of $C_{0}$ impose independent
conditions on forms of degree $d-3$.
\end{corollary}
\begin{proof}
 Otherwise the space of differential forms on the normalization of $C_{0}$ would be too large.
\end{proof}

One can use Theorem~\ref{canonical from adjoint 1} to re-embed a plane curve of 
geometric genus $g$ as a canonical
curve in $\PP^{g-1}$:

\begin{corollary}
 The canonical ideal of the normalization of a nodal plane curve of degree $d$ is the ideal of polynomial relations
 among the forms of degree $d-3$ that vanish at the nodes of the curve. \qed
\end{corollary}

\begin{figure}
\centerline {\includegraphics[width=3in]{"main/Fig14-2"}}
\caption{A curve $C$ of geometric genus 2 represented as the normalization of a plane curve $C_{0}$ of degree 4 with a node, and a canonical divisor, represented as a conic containing the node.{Silvio: Nice picture. Would be clearer if the plane of $C_{0}$ were shown. Maybe $C,C_{0}$ different colors?}}
\label{canonical on normalization}
\end{figure}

\subsection{Linear series on a nodal plane curve}\label{linear series on nodal plane curves}

Since $C_{0}$ is singular, not every effective divisor on $C$ is the preimage of an
effective Cartier divisor on $C_{0}$. As an example, one may take a single point lying over a node
as in Figure~\ref{Fig14.2}. 

However,
we can still represent every divisor on $C$ as the preimage of a divisor on $C_{0}$ up to linear
equivalence, and the same goes for any reduced curve:

\begin{lemma}
Let $\nu: C\to C_{0}$ be the normalization of any reduced projective curve. If $D$ is any divisor
on $C$, then $D$ is linearly equivalent to the pullback of a divisor supported on the smooth locus of  $C_{0}$. More precisely,  every effective divisor on $C$ containing $\Delta$ can be written as the pullback
of a Cartier divisor on $C_{0}$\end{lemma}

We will see a more general version in Theorem \ref{Cartier on C}.

\begin{proof}
It suffices to prove the result locally on $C_{0}$, where it is geometrically obvious:  if the node
$p\in C_{0}$ has preimages $q,r$ corresponding to the branches of $C_{0}$ at $p$, then
a divisor $aq+br$ with both $a,b$ strictly positive, is locally the pullback of the intersection of $C_{0}$
with a curve $C'$ passing through $p$  that is $(a-1)$-times tangent to one branch and
$(b-1)$-times tangent to the other.
\end{proof}


Returning to the case of a nodal plane curve $C_{0}$ and its normalization $\nu: C\to C_{0}$,
suppose that $D$ is a divisor on $C$ that is the pullback of a difference of Cartier
divisors $D_{+}-D_{-}$ on $C_{0}$. We will compute the complete linear series $|D|$.
Let $\adj(C_{0})$ be the set of nodes in $C$ and let $\Delta$ be the preimage of $\adj(C_{0})$ in $C$.

\begin{theorem}\label{linear series on nodal curves}
Let $D = D_{+}-D_{-}$ be a divisor on $C$, and let $G$ be a form on $\PP^{2}$ 
that vanishes on  $D_{+}+\adj(C_{0})$ but not identically on $C_{0}$.

If $G$ has degree $m$ and $A = (\nu^{*}G)-D_{+}-\Delta$, then every effective divisor on
$C$ linearly equivalent to $D$ (if any) has the form $(\nu^*H)-D_{-}-A-\Delta$ for some $H$ of degree $m$
that vanishes on $\adj(C_{0})+A$ but not identically on $C_{0}$.
\end{theorem}
 
 The reason for including $\adj(C_{0})$ is explained in Section~\ref{why add Delta}.
\begin{corollary}
With notation as in Theorem~\ref{linear series on nodal curves}, the ideal of the image of
$C$ with respect to the linear series $|D|$ is the ideal of polynomial relations among the forms
of degree $m$ in $\ff_{C/C_{0}}$ that vanish on $D_{-}+ A$.\qed
\end{corollary}

\begin{proof}[Proof of Theorem~\ref{linear series on nodal curves}]
 Choose an integer $m$ sufficiently big that there is a form $G$ vanishing on  $D_{+}$ and $\adj(C_{0})$ so that
 $\nu^{*}(G)$ vanishes on $D_{+}+\Delta$, but not everywhere on $C$.
 (If $D_{+}$ contains some positive multiple of a point $p$ of $\Delta$ this means that $G$ defines
 a curve sufficiently tangent to the corresponding branch of $C_0$.) 
 
 We can then write the zero locus of $G$ pulled back to $C$ as
$$
(\nu^*G) = D_{+} + \Delta + A,
$$
as before. 

\begin{figure}
\centerline {\includegraphics[width=3.0in]{"main/Fig14-3"}}
\caption{If $G$ is tangent to a branch of $C_{0}$ at the node, then $D_{+}$ contains
a point of $\Delta$ on $C$. {Silvio: C should be labeled. G and $D_{+}$ could be the same color}}
\label{Fig14.2}
\end{figure}

Next, we look for forms $H$ of the same degree $m$, vanishing at $A+D_{-}$ and on $\adj(C_{0})$
 but not on all of $C_0$. If there are no such polynomials $H$ then, as we shall show,
there are no effective divisors equivalent to $D$. Supposing that there is such a form $H$, let $D'$ be the divisor 
$$
D' = (\nu^*H) -( D_{+} + \Delta),
$$
that is, $D'$ is residual to $( D_{+} + \Delta)$ in $(\nu^*H)$. 

Since $\nu^*(G/H)$ is a rational function on $C$ we have
$$
D_{-} +\Delta + A+ D' = (\nu^*H) \sim (\nu^*G) = D_{+} + \Delta + A,
$$
and thus $D'$ is an effective divisor linearly equivalent to $D = D_{+}-D_{-}$ on $C$.

To complete the argument we must show that we get \emph{all} divisors $D'$ in this way.
In this case the curve $C$ can be desingularized by blowing up the plane once at each node,
and we can give a proof based on the resulting surface $S$. The same technique would work for any curve with only
ordinary multiple points, in which case the total transform of $C_{0}$ on $S$ has normal crossings. We will give a different proof, extending this theorem to curves with arbitrary singularities, in Section~\ref{arbitrary plane curves}.

\begin{proposition}\label{adjoint completeness1}
If $C_{0}$ is a reduced irreducible plane curve all of whose singularities are ordinary nodes, then for each
integer $m$,
the linear series cut out on the normalization $C$ of $C_{0}$ by forms of degree $m$ containing the nodes
is complete.
\end{proposition}

\begin{proof}
To prove Proposition~\ref{adjoint completeness1}, we work on the blow-up $\pi : S \to \PP^2$ of $\PP^2$ at the nodes $q_i$ of $C_0$. The proper transform of $C_0 \subset \PP^2$ in $S$ is the normalization of $C_0$, which we will again call $C$.

Let $L$ be the class on $S$ of the pullback of a line in $\PP^2$  and let $E$ be the sum of the exceptional divisors, the preimage of $\adj(C_{0})$. We write $h= L\cap C$ and $e = E\cap C= \sum (p_i+q_i)$ for the corresponding divisors on $C$. 
Because $C$ has double points at each $q_{i}$ we have
$
C \sim dL - 2E 
$
and   by
Theorem~\ref{divisor classes on blowup} we have $K_S \sim -3L + E$.

The proper transform of a degree $m$ curve $A\subset \PP^2$  passing simply through the points $q_i$
is $\pi^*A - E$; this gives an isomorphism
$$
H^0(\cI_{\{q_1,\dots,q_\delta\}/\PP^2}(m)) \cong H^0(\cO_S(mL-E)).
$$
In these terms we can describe the linear series cut on $C$ by plane curves of degree $m$ passing through the nodes of $C_0$ as the image of the map
$$
H^0(\cO_S(mL-E)) \to H^0(\cO_C(mL-E)),
$$
and we must show that this map is surjective.

From the long exact cohomology sequence associated to the exact sequence of sheaves
$$
0 \to \cO_S((m-d)L + E)  \to \cO_S(mL-E) \to \cO_C(mL-E) \to 0,
$$
 we see that it will suffice to prove that $H^1(\cO_S((m-d)L + E)) = 0$. 
 
By Serre duality on $S$,
%we have  $H^1(\cL) \cong H^1(K_S\otimes \cL^{-1})^*$. In this instance it 
%tells us that
$$
H^1(\cO_S((m-d)L + E)) \cong H^1(\cO_S((d-m-3)L))^*.
$$
The line bundle $\cO_S((d-m-3)L)$ is 
 the pullback to $S$ of the bundle $\cO_{\PP^2}(d-m-3)$, which has vanishing $H^1$. Lemma~\ref{H1 on pullback} completes the proof.
\end{proof}

\begin{lemma}\label{H1 on pullback}
Let $X$ be a smooth projective surface, and $\pi : S \to X$ the blow-up of a finite set of reduced points. If $\cL$ is any line bundle on $X$, then
$$
H^1(S, \pi^*\cL) = H^1(X, \cL).
$$
\end{lemma}

\begin{proof} Because $\PP^2$ is normal, and $\pi_*(\sO_S)$ is a finite birational algebra over $\sO_{\PP^2}$, we have $\pi_*(\sO_S) = \sO_{\PP^2}$.
Since any invertible sheaf $\sL$ on $\PP^{2}$ is locally isomorphic to $\sO_{\PP^2}$,   is also an isomorphism. 

The Leray spectral sequence (Theorem~\ref{Leray}) gives an exact sequence
$$
0\to H^{1}(\pi_{*}(\sL)) \to H^{1}(\sL) \to  H^{0}(R^{1}(\pi_{*}(\sL))\to 0
$$

 The restriction of $\pi^{*}(\sL)$ to any fiber of $\pi$ is trivial and has vanishing $H^{1}$,
so
$H^{0}(R^{1}(\pi_{*}(\sL))) = 0$, and $\pi_{*}\pi^{*}\sO_{\PP^{2}}(1)$ is an invertible sheaf.
The natural map
$\pi_{*}\pi^{*}(\sL) \to \sL$ is an isomorphism away from the codimension
2 set of points blown up. Thus these two sheaves are isomorphic, and
$$
H^1(S, \pi^*\cL) = H^{1}(\pi_{*}\pi^{*}\sL) = H^{1}(\sL),
$$
completing the proof.
\end{proof}

This concludes the proof of Theorem~\ref{linear series on nodal curves}
\end{proof}



\begin{proposition}\label{effect of blowup on genus}
 Let $C$ be a curve on a smooth surface $S$, and let $\nu : S' \to S$ be the blowup of $S$ at $p$. If $C'$ is the strict transform of $C$, then
 $$
 p_a(C') = p_a(C) -{m\choose 2},
 $$
 where $m$ is the multiplicity of $p\in C$.
\end{proposition}
\begin{proof}
This follows from comparing the adjunction formulas on $S$ and $S'$. To start, we have
$$
p_a(C) = \frac{C^2 + K_S\cdot C}{2} + 1.
$$
On $S'$ let $E$ be the divisor class of the exceptional divisor. As we've seen,
$$
K_{S'} = \nu^*K_S + E,
$$
while the class of $C'$ is given by
$$
C' \sim \nu^*C - mE.
$$
It follows that
$$
(C')^2 = C^2 + m^2E^2 = C^2 - m^2 \quad \text{and} \quad K_{S'}\cdot C' = K_S\cdot C + m.
$$
Thus, applying adjunction on $S'$, we find
$$
p_a(C') = \frac{{C'}^2 + K_{S'}\cdot C'}{2} + 1 = \frac{C^2 + K_S\cdot C - m(m-1)}{2} + 1 = p_a(C) -{m\choose 2}
$$
as stated.
\end{proof}

\begin{fact}
Any plane curve can be desingularized by
iteratively blowing up of singular points of $C$, then of the strict transform, and so on. See for example
\cite{Fulton1989} or \cite{Brieskorn1986}.  This is in fact the same
sequence of transforms as the one given in Exercise~\ref{Mumford resolution argument}. 
The points on the various blowups that
map to the original singular point are called \emph{infinitely near points}.
\end{fact}

This gives a nice formula for the delta-invariant of any singularity:

\begin{corollary}
\label{computing delta}

Thus the $\delta$ invariant of any singularity of a plane curve $C_{0}$ at a point $p$ can be computed as the sum of the numbers $\binom{m_{q}}{2}$
where the sum runs over all \emph{infinitely near} singular points $q$ and $m_{q}$
denotes the multiplicity of the pullback of $C_{0}$ at $q$.\qed
\end{corollary}

\section{Arbitrary plane curves} \label{arbitrary plane curves}

Throughout this section $C_0 \subset \PP^2$ denotes a reduced and irreducible plane curve with arbitrary singularities. Let $\nu : C \to C_0$ be its normalization, and write $L'$ for the pullback to $C$ of the class $L$ of a line in $\PP^{2}$.

It is possible to carry out an analysis of linear series on the normalization of an arbitrary plane curve in a manner  analogous to what we did in the preceding section for nodal curves, replacing the the set of nodes by the scheme $\adj(C_{0})\subset C_{0}$ called the \emph{adjoint scheme}, which in this case is the scheme
defined by the \emph{conductor ideal} $\ff_{C/C_0}$, the annihilator in $\sO_{C_{0}}$ of $\nu_{*}(\sO_{C})/\sO_{C_{0}}$---see Theorem~\ref{general adjoint}. Let $\Delta\subset C$ be the Cartier divisor
defined by the pullback of $\ff_{C/C_0}$ to $C$.

\subsection{The conductor ideal and linear series on the normalization}\label{why add Delta}

The following result gives a simple way of expressing any divisor on $C$ in terms of $\Delta$ and a
Cartier divisor on $C_{0}$:

\begin{theorem}\label{Cartier on C}
If $D$ is an effective (Cartier) divisor on $C$ that contains $\Delta$, then $D$ is the pullback of a Cartier divisor on $C_{0}$. 
\end{theorem}

\begin{proof}
The result is local, so it suffices to treat the affine case of an affine curve $C_{0}$ with coordinate
ring $\sO'$ contained
in its integral closure $\sO$, the coordinate ring of its normalization $C$.
The ideal of $\Delta$ in $\sO$ is the conductor $\ff_{\sO/{\sO'}}$ (which is contained in $\sO'$, but stable under
multiplcation by elements of $\sO$, and thus also an ideal of $\sO$). 
Thus $\sI_{D}$ may be regarded as an ideal---though not necessarily a principal ideal---of $\sO'$. Since the  ground field $\CC$ is infinite, $\sI_{D}\subset \sO'$ is the integral closure of a principal ideal $(x)$
of $\sO'$.
(see for example \cite[Chapter 8]{Swanson-Huneke}).
This means that for sufficiently large $n$ we have $x\sI_{D}^{n} = \sI_{D}^{n+1}$.

Let $D_{0}$ be the Cartier divisor on $C_{0}$ corresponding to $x$. Pulling everything back to $C$
we have 
$$
\nu^*(D_{0})+nD = (n+1)D
$$
whence $\nu^*(D_{0}) = D$.
 \end{proof}

In view of the previous theorem, we can specify an arbitrary linear series on $C$
as the difference $\cE - \Delta$, where $\cE$ is a linear series with $\Delta$ in its base locus, by specifying a
linear series on $C_0$ containing  $\adj(C_{0})\subset C_{0}$. The following theorem makes
this algorithmic.

\begin{theorem}\label{linear series on arbitrary curves}
Let $D = D_{+}-D_{-}$ be a divisor on $C$, and let $G$ be a form on $\PP^{2}$,
vanishing on $\adj(C_{0})$, whose pullback to $C$
vanishes on $D_{+}$ but not on all of $C$. 

If $G$ has degree $m$ and $(\nu^{*}G) = D_{+}+A+\Delta$, then every effective divisor on
$C$ linearly equivalent to $D$ (if any) has the form $(\nu^*H)-D_{-}-A-\Delta$ for some $H$ of degree $m$
 vanishing on $\adj(C_{0})+A$. Thus, the global sections of $H^{0}(\sO_{C}(D))$ can be written
as divisors cut by forms of degree $m$ in $\ff_{C/C_{0}}$ with basepoints at $D_{-}+A+\Delta$.
\end{theorem}

The following consequence allows one to embed $C$ via the linear series $|D|$:

\begin{corollary}
With notation as in Theorem~\ref{linear series on arbitrary curves}, the ideal of the image of
$C$ with respect to the linear series $|D|$ is the ideal of polynomial relations among the forms
of degree $m$ in $\ff_{C/C_{0}}$ that vanish on $D_{-}+A$.\qed
\end{corollary}

\begin{proof}[Proof of Theorem~\ref{linear series on arbitrary curves}]
If $H\in \ff(C_{0})$ vanishes on $A$ but not on all of $C_{0}$, then as before
$$
 D_{-} +\Delta + A+ D' = (\nu^*H) \sim (\nu^*G) = D_{+} + \Delta + A,
$$
So $D' = (\nu^*{H})-D_{-}-A-\Delta$ is linearly equivalent to $D$. The proof that every
divisor $D'$ linearly equivalent to $D$ has this form is the content of
Theorem~\ref{conductor completeness} below.
\end{proof}


The following result was known classically as the \emph{completeness of the adjoint series}.

\begin{theorem}\label{conductor completeness}
For every integer $m\geq 0$ the series cut out on $C$ by forms of degree $m$
on $\PP^{2}$ containing $\adj(C_{0})$ is complete.
\end{theorem}

\begin{proof}
If $R_{0}\subset R$ is an inclusion of commutative rings, then the set
$\ff_{R/R_{0}} := \ann_{R_{0}}(R/R_{0})\subset R_{0}$ is by definition an ideal of $R_{0}$, but is also an ideal of $R$; this follows
because if $f\in \ff_{R/R_{0}}$ and $r\in R$, then $fR\subset R_{0}$ so
 $(rf)R = frR \subset fR \subset R_{0}$. 

If $R_{0}$ is a domain and $R$ is a subring of the quotient field $Q(R)$ of $R$, then
 $\ff_{R/R_{0}} \cong \Hom_{R_{0}}(R, R_{0})$. To see this, note that $R_{0}$ and $R$ become
 equal after tensoring with $Q(R_{0})$ and thus 
 $\Hom_{R_{0}}(R,R_{0}) \subset \Hom_{Q}(Q,Q) = Q$ 
 may be identified
 with the set of elements $\{\alpha\in Q\mid \alpha R \subset R_{0}\}$. If $\alpha$ is in this set, then
  $\alpha\cdot 1 = \alpha \in R_{0}$, as required.
  
Returning to the case of the curve $C_{0}$, it follows that the global sections of $(\nu^{*}(\sO_{C_{0}}(m))( -\Delta)$ on $C$
are, on each affine open set $U$, represented by the elements of $\sO_{C_{0}}(U)$ that  are restrictions to $U$
of forms of degree $m$ contained in 
the ideal $\ff_{C/C_{0}}$. Thus
the global sections of the sheaf $\widetilde{\ff_{C/C_{0}}}(m)$ cut out a complete linear series on $C$.

Write $S =\CC[x_{0}, x_{1}, x_{2}]$ for the homogeneous coordinate ring of $\PP^{2}$.
It remains to prove that the homogeneous ideal $\ff_{C/C_{0}}$ maps
surjectively to $H^{0}_{*}(\widetilde{\ff_{C/C_{0}}})$, and this amounts to the
statement that the depth of $\ff_{C/C_{0}}$ as an $S$-module is (at least) 2.
%There is a natural  map
%$$
%\ff_{C/C_{0}} \to \sHom_{\sO_{C_{0}}(\nu_{*}\sO_{C}, \sO_{C_{0}}},
%$$ and the argument
%above shows that this map is an isomorphism on each affine open set, proving that as sheaves
%$$
%\ff_{C/C_{0}} \cong \sHom_{\sO_{C_{0}}}(\nu_{*}\sO_{C}, \sO_{C_{0}}.
%$$
Set $R_{0} = H^{0}_{*}(\sO_{C_{0}})$ and $R = H^{0}_{*}(\nu_{*} (\sO_{C}))$.
We see from the general considerations above that
$$
\ff_{R/R_{0}} = \Hom_{R_{0}}(R, R_{0}).
$$ 
Any non-zerodivisor on a module $M$
is a non-zerodivisor on $\Hom(P, M)$ for any module $P$ since $(a\phi)(p) = a(\phi(p))$ by definition. 
Since $R_{0} = S/(F)$, it is a module of depth 2, and we may choose a regular sequence
$a,b$ of elements in $R_{0}$. From the short exact sequence
$$
0\rTo R_{0}\rTo^{a} R_{0}\rTo R_{0}/(a) \rTo 0
$$
we get a left exact sequence
$$
0\rTo \Hom_{R_{0}}(R,R_{0})\rTo^{a} \Hom_{R_{0}}(R,R_{0})\rTo \Hom_{R_{0}}(R,R_{0}/(a)).
$$
Thus 
$$
\Hom_{R_{0}}(R,R_{0})/aHom_{R_{0}}(R,R_{0}) \subset \Hom_{R_{0}}(R,R_{0}/(a))
$$
and since $b$ is a non-zerodivisor on $\Hom_{R_{0}}(R,R_{0}/(a))$, it is a non-zerodivisor
on $\Hom_{R_{0}}(R,R_{0})/aHom_{R_{0}}(R,R_{0})$ as well.
\end{proof}

Since we saw directly that the adjoint ideal was equal to the conductor ideal in the case of
a nodal curve, this result gives another, less ad hoc, proof that the effective divisors equivalent to $D$
are all defined by
pullbacks of forms of degree $m$ that contain $\Delta$
as constructed in Proposition~\ref{adjoint completeness1}.

\subsection{Differentials}

Let $C^\circ_0$ be the intersection of $C_0$ with the open set $\AA^{2}\cong U\subset \PP^{2}$ where $Z \neq 0$, 
and let $C^\circ \subset C$ be the preimage of $C^\circ_0$.

\begin{theorem}\label{general differentials}
If $C_{0}$ meets the line $L$ at infinity only in smooth points of $C_{0}$ other than $(0,1,0)$, then the complete canonical series on the normalization $\nu: C \to C_{0}$ is cut out by differentials of the form
$$
 \frac{e(x,y)dx}{f_{y}}
$$
where $e(x,y)$ is a polynomial of degree $\leq d-3$ contained in the 
conductor ideal $\ff_{C^{\circ}/C_{0}^{\circ}}$
\end{theorem}

As in the case of nodal curves, one can useTheorem~\ref{general differentials} to re-embed a plane curve of 
geometric genus $g$ as a canonical
curve in $\PP^{g-1}$:

\begin{corollary}
 The canonical ideal of the normalization $C$ of a plane curve $C_{0}$ of degree $d$ 
 is the ideal of polynomial relations
 among the forms of degree $d-3$ in the conductor ideal $\ff_{C/C_{0}}$. \qed
\end{corollary}

\begin{proof}
The proof of Theorem~\ref{general differentials} consists of four parts. 

First, because we have assumed that $(0,1,0)$ does not lie on $C$, the function $x$ defines a
ramified $d$-sheeted cover of $C$ to $\PP^{1}$. Because $C_{0}$ meets $L$ only in smooth
points and the differential $dx$ has a pole of order 2 at the point at infinity in $\PP^{1}$,
$dx$
has polar locus twice the divisor $ \nu^{-1}(C_{0}\cap L)$. It follows that
the differential
$\varphi_0 := dx/f_{y}$ is regular, with a zero of order $d-3$,
along the divisor of $C$ lying over $C_0\cap L$.

Second, the function on $C^\circ_0$ defined by $x$  
is a finite map to $\AA^1$, and thus the field of rational functions $\kappa(C) = \kappa(C^\circ_0)$ is a finite
separable extension of $\CC(x)$. By~\cite[Section 16.5]{Eisenbud1995}, the module of differentials 
$\omega_{\kappa(C)/\CC}$ is generated over $\kappa(C^\circ_0)$ by $dx$. Thus every rational
differential form on $C$ can be expressed as a rational function
times $dx$. Since $\varphi_{0} := dx/f_{y}$ vanishes to order $d-3$ along $C_{0}\cap L$,
the regular differential forms on $C$ must be of the form $e(x,y)\varphi_{0}$ where
$e(x,y)$ is a rational function of degree $\leq d-3$. (The set of rational forms that occur in this
way is called the \emph{Dedekind complementary module}.)
 
A sophisticated form of Hurwitz's theorem that will be explained in Chapter~\ref{LinkageChapter}
shows that the sheaf $\omega_{C}$ of regular differential forms on $C$ can be expressed as
$$
\sHom_{\PP^{1}}(\nu_{*}(\sO_{C}), \omega_{\PP^{1}}),
$$
or more correctly $\nu^{!}\pi^{!}\sHom_{\PP^{1}}(\pi_{*}\nu_{*}(\sO_{C}), \omega_{\PP^{1}})$
where $\pi$ is the map $C_{0}\to \PP^{1}$ defined by $x$---see Chapter~\ref{LinkageChapter}. Since the maps involved are finite,
we will ignore this refinement, and treat $\sO_{C_{0}}$ and $\nu_{*}(\sO_{C})$ as sheaves on $\PP^{1}$.

Since $\sO_{C_{0}}\subset \nu_{*}(\sO_{C})$, this sheaf is naturally contained
in 
$$
\sHom_{\PP^{1}}(\sO_{C_{0}}, \omega_{\PP^{1}}).
$$
We will show in Theorem~\ref{general adjoint} that 
$$
\ff_{C/C_{0}} = 
\frac{\sHom_{\PP^{1}}(\sO_{C_{0}}, \omega_{\PP^{1}})}
{\sHom_{\PP^{1}}(\nu_{*}(\sO_{C}), \omega_{\PP^{1}})}.
$$
As will be explained in Chapter~\ref{LinkageChapter}, the numerator of this quotient
$$
\omega_{C_{0}}:= \sHom_{\PP^{1}}(\sO_{C_{0}}, \omega_{\PP^{1}}) = 
\sHom_{\PP^{1}}(\sO_{C_{0}}, \sO_{\PP^{1}})(-2)
$$
 is properly called the dualizing module  of the singular curve $C_{0}$.
Thus every regular differential form on $C$ can be expressed as an element of the conductor
times some element of $\omega_{C_{0}}$. To prove the Theorem, we must show that
$\varphi_{0}$ generates $\omega_{C_{0}}$ as a module over $\sO_{C_{0}}$.

Passing to the field of rational functions $\kappa := \kappa(C)$, and noting that
$\kappa(C) = \kappa(C_{0}) $ we use the well-known result from Galois theory that 
$$
\Hom_{\kappa(\PP^{1})}(\kappa, \kappa(\PP^{1}))
$$
is a 1-dimensional vector space over $\kappa$, generated by the trace map $T$. Moreover, 
because $\sO_{C_{0}}$ is integral over $\sO_{\PP^{1}}$, which is normal,
$$
T(\sO_{C_{0}})\subset \sO_{\PP^{1}}.
$$

Less well-known is the result from commutative
algebra:

\begin{theorem}\label{Kunz}
If $C_{0}^{\circ}$ is an affine plane curve defined by the
equation $f(x,y)=0$ and such that $\CC[x,y]/(f)$ is finite over $\CC[x]$,
then $\Hom_{\CC[x]}(\CC[x,y]/(f), \CC[x])$ is generated by $(1/f_{y})T$.
\end{theorem}

See \cite[Theorem 15.1]{Kunz} for a proof using valuations, and \cite[Theorem A.1]{MR4026452} for a proof in
a more general context.

Given Theorem~\ref{Kunz}, we can complete the proof of Theorem~\ref{general differentials}. Via the isomorphism $\CC(x) \cong \CC(x)dx$ sending 1 to $dx$, the trace map is identified (up to scalar)
with the map $(1\mapsto dx) \in \Hom_{\CC(x)}(\kappa, \CC(x)dx)$. Thus by Theorem~\ref{Kunz}
the canonical module of $C_{0}^{\circ}$ is identified with $\sO_{C_{0}^\circ} \phi_{0}$ as
required.
\end{proof}

%Theorem~\ref{general differentials} can be generalized to Gorenstein singularities. However the example
%of spatial triple point below shows that the general situation can look rather different: there the (local) conductor
%is the maximal ideal at the singular point, but there are two linear conditions for a function on $C$ to descend
%to $C_{0}$



\begin{example}[nodes and cusps]
We have already seen that in case $q$ is a node of $C_0$, there are two points of $C$ lying over it, and the multiplicities of $\varphi_0$ at these two points are $m_1=m_2=1$; the adjoint ideal is thus 
 the maximal ideal $\cI_q$ at $q$. In the case of a cusp, analytically isomorphic to the zero locus of $y^2-x^3$, there is only one point $r$ of $C$ lying over the cusp point $q$. The cusp can be parametrized, locally analytically,
 by $x = t^{2}, y = t^{3})$ and it follows that the differential 
 $$
 \varphi_0 = \frac{dx}{f_{y}} =  \frac{2tdt}{2t^{3}} =  \frac{dt}{t^{2}}
 $$ 
 has a pole of of order 2. since the pullback to $C$ of any polynomial $g$ vanishing at the cusp $q$ vanishes to order at least two at $r$, the adjoint ideal is again the maximal ideal at $q$. We can also see this by computing the
 conductor ideal as the annihilator of $\CC\[t\]/\CC\[t^{2}, t^{3}\]$.
\end{example}

\begin{example}[tacnodes]
Next, consider the case of a \emph{tacnode} (Figure~\ref{Fig14.4}); that is, a plane-curve singularity with two smooth branches simply tangent to one another, analytically isomorphic to the zero locus of $y^2-x^4$, parametrized locally analytically with two branches $x = t, y =  t^{2}$ and $x=t, y = -t^{2}$.

\begin{figure}
\centerline {\includegraphics[width=2in]{"main/Fig14-4"}}
\caption{A tacnode: two smooth branches tangent to one another.}
\label{Fig14.4}
\end{figure}

At the two points of $C$ lying over $q$, we have
  $$
 \varphi_0 =  \frac{dt}{ 2t} \hbox{ and } \frac{-dt}{ 2t}
 $$ 
which has a simple pole at each. 
The adjoint ideal is thus the ideal of functions vanishing at $q$ and having derivative 0 in the direction of the common tangent line to the branches.
\end{example}

\begin{example}[ordinary $n$-fold points]
In the case of an ordinary $n$-fold point of a plane curve---$n$ smooth branches pairwise transverse to one another---there are $n$ points
$r_i$ of $C$ lying over $q$. The polynomial $f_y$ vanishes to order $n-1$ at $q$, so $dx/f_y$ has a pole of order $n-1$ at
each $r_i$. It follows that for $e(x,y)dx/f_y$ to be regular, $e$ must vanish to order $n-1$ at each $r_i$. 

We can see that the conductor ideal is the full $(n-1)$-rst power of $(x,y)$ by using the
normalization map
$$
\nu^{*}: \frac{\CC[x,y]}{\prod_{i=1}^{n} (x-\alpha_{i}y)} \to
 \prod_{i=1}^{n} \frac{\CC[x,y]e_{i}}{(x-\alpha_{i}y)}
$$
where the $\alpha_{i}$ are distinct elements of $\CC$ and the $e_{i}$ are orthogonal idempotents.
The element $1 = \sum_{i}e_{i}$ goes to 0 in the quotient, and $e_{i}$ is annihilated by $x-\alpha_{i}y$,
so the quotient is annihilated by each of the $n$ elements $g_{j} := \prod_{i\neq j} (x-\alpha_{i}y)$.
As forms on $\PP^{1}$, all the $g_{i}$ except $g_{j}$ vanish at the 
point $(\alpha_{j}, 1)$, so the $g_{i}$ are linearly independent . Since $(x,y)^{n-1}$ is minimally generated by $n$ elements, $(x,y)^{n-1} = (g_{1}, \dots, g_{n})$.

A consequence of this computation is that the $\delta$-invariant of the ordinary multiple point---that is, the difference in arithmetic genus between a plane curve that has just one such singular point and its normalization---is $\binom{n}{2}$, the dimension of $k[x,y]/(x,y)^{n-1}$, as we proved before in Proposition~\ref{effect of blowup on genus} \end{example}

\begin{example} (spatial triple points) Spatial triple points provide a contrast to the last example. A spatial triple point is a singularity consisting of three smooth branches, with linearly independent tangent lines, meeting in a point $p$ so that its Zariski tangent space is 3-dimensional; the simplest example is the origin as a point on the union of the three coordinate axes in $\AA^3$.

In this case the conductor is the annihilator of the cokernel of
$$
\nu^{*}: R := \frac{\CC[x,y,z]}{(xy, xz, yz)} \to \frac{\CC[x,y,z]}{(x,y)} \times \frac{\CC[x,y,z]}{(x,z)} \times \frac{\CC[x,y,z]}{(y,z)} =: \overline R.
$$
Since $x\overline {R} = x \CC[x,y,z]/(y,z)$ is in the image of $\CC[x,y,z]/(xy, xz, yz)$, and similarly with $y$ and $z$,
we see that the conductor is the maximal ideal $(x,y,z)$. However, the $\delta$-invariant, the length
of the quotient $(\overline R)/R$, is 2: for a function $f$ in $\overline R$ to be in $R$, it
is necessary and sufficient that $f$ take the same value at the three points above the singular point, 
and this is
2 linear conditions.
\end{example}

\section{Exercises}

In Exercise~\ref{gonality of smooth plane curve}, we saw how to use the description of the canonical series on a smooth plane curve to determine its gonality. Now that we have an analogous description of the canonical series on (the normalization of) a nodal plane curve, we can deduce a similar statement about the gonality of such a curve. Here are the first two cases: 

\begin{exercise}
Let $C_0$ be a plane curve of degree $d\geq 4$ with one node $p$ and no other singularities, and let $C$ be its normalization. Show that $C$ admits a unique map $C \to \PP^1$ of degree $d-2$, but does not admit a map $C \to \PP^1$ of degree $d-3$ or less.

Hint: If $D = q_1 + \dots + q_{d-2}$ had $r(D) \geq 1$, the points $q_1 + \dots + q_{d-2}$ and $p$ would fail to impose independent conditions on plane curves of degree $d-3$ and hence lie on a line.
\end{exercise}

\begin{exercise}
Let $C_0$ be a plane curve of degree $d\geq 5$ with two nodes $p$ and $p'$ and no other singularities, and let $C$ be its normalization. Show that $C$ admits two maps $C \to \PP^1$ of degree $d-2$, but does not admit a map $C \to \PP^1$ of degree $d-3$ or less.

Hint: If $D = q_1 + \dots + q_{d-2}$ had $r(D) \geq 1$, the points $q_1 + \dots + q_{d-2}$ and $p, p'$ would fail to impose independent conditions on plane curves of degree $d-3$ and hence by Proposition~\ref{independent conditions} below $d-1$ of them would lie on a line.
\end{exercise}

\begin{exercise}
Generalizing the examples above, show that if a nodal plane curve of degree $d$ has $\delta\leq d+3$ nodes,
then its gonality is $d-2$, and moreover every $g^1_d$ on the curve is given by projection from one of the nodes.
You may use the following result:
\begin{proposition}\label{independent conditions}
 A set of $n \leq 2d+ 2$ distinct
points in the plane fails to impose independent conditions on curves of degree
$d$ if and only if either $d + 2$ of the points  are collinear or $n = 2d + 2$ and all the points lie
on a conic.
\end{proposition} 
See~\cite[p. 302]{MR1376653} for a proof of this.

Hint: a $g^1_{d-2}$ is a set of points that, together with the nodes, impose dependent conditions on forms of degree $d-3$.
\end{exercise}

\begin{exercise}\label{general case of divisors on nodal curves}
Suppose that $C$ is a smooth curve and $\nu: C \to C_0$ is a map to a plane curve with
only nodes as singularities. Let $D = D_{+}-D_{-}$ be a divisor on $C$. Modify the 
technique of Section~\ref{linear series on nodal plane curves} to compute the complete
linear series $|D|$ without assuming that $D$ is disjoint from the preimages of the singular
points. 
Hint: Be careful to subtract the right multiples of the points that are preimages of the singular
points.
\end{exercise}

\begin{exercise}
Let $C_0$ be a plane quartic curve with two nodes $q_1, q_2$
and 
let $\nu :\nobreak C \to C_0$ be its normalization. 
By the adjunction formula, $C$ has genus~1.
For an arbitrary point
$o \in C$ not
lying over a node of $C_0$,
 give a geometric description
of the group law on $C$ with $o$ as origin.

Hint: To add two points $s$ and $t \in C$, choose a conic curve $D$
passing though $s, t, q_1$ and $q_2$, and let $u$ and $v$ be the remaining
points of $C_0 \cap D$; then take the conic 
$D'$ passing though $u, v,
q_1, q_2$ and $o$. The sum $s+t$ will then be the remaining point of $D'
\cap C_0$.
\end{exercise}

\begin{exercise}
Find the adjoint ideals of the following plane curve singularities:
\begin{enumerate}
\item a \emph{triple tacnode}: three smooth branches, pairwise simply tangent (in classical language, this is a \emph{triple point with an infinitely near double point})
\item a triple point with an infinitely near double point: three smooth branches, two of which are simply tangent, with the third transverse to both
\item a unibranch triple point, such as the zero locus of $y^3-x^4$
\end{enumerate}

Hint: For the first, the adjoint ideal is the ideal of functions vanishing to order 4 on each branch (so that the general member of the ideal with have zero locus consisting of two smooth branches simply tangent to the branches of the triple point). For the last, the adjoint is simply the square of the maximal ideal.
\end{exercise}

Here is a  general description in case the individual branches of $C_0$ at $p$ are each smooth:

\begin{exercise}
Let $\nu : C \to C_0$ be the normalization of a plane curve $C_0$ and $p \in C_0$ a singular point. Denote the branches of $C_0$ at $p$ by $B_1,\dots,B_k$, and let $r_i$ be the point in $B_i$ lying over $p$. If the individual branches $B_i$ of $C_0$ at $p$ are each smooth, and we set
$$
m_i = \sum_{j \neq i} \mathrm{mult}_p(B_i \cdot B_j)
\marginparhere{does mult here have the meaning defined in chapter 10?}
$$
then the adjoint ideal of $C_0$ at $p$ is the ideal of functions $g$ such that $\ord_{r_i}(\nu^*g) \geq m_i$.

Hint: The computation can be done locally analytically. Let $R = \widehat{\sO_{C_0, p}}$ be the completion of the local ring
of $C_0$ at $r_i$. The integral closure is then the product of rings $R_i = \widehat{\sO_{B_i}} \cong k\[t_i\]$,
with $R_i = R/P_i$ as $P_i$ runs over the minimal primes of $R$. The multiplicity
$m_i$ is the colength of the ideal $\sum_{j\neq i}P_j \subset R$.
\end{exercise}


%footer for separate chapter files

\ifx\whole\undefined
\makeatletter\def\@biblabel#1{#1]}\makeatother
\gdef\urlhook{\url}
\bibliography{slag}
\bibliographystyle{msribib}


%%%% EXPLANATIONS:

% f and n
% some authors have all works collected at the end

\catcode`\^\active
%if ^ is followed by 
% 1:  print f, gobble the following ^ and the next character
% 0:  print n, gobble the following ^
% any other letter: print letter
\makeatletter
\def^#1{\ifx1#1f\expandafter\@gobbletwo\else
        \ifx0#1n\expandafter\expandafter\expandafter\@gobble\else#1\fi\fi}
\makeatother
\let\moreadhoc\relax
\def\indexintro{%An author's cited works appear at the end of the
%author's entry; for conventions
%see the List of Citations on page~\pageref{loc}.  
%\smallbreak\noindent
The letter `f' after a page number indicates a figure, `n' a footnote.}
\printindex[gen]
%requires makeindex
\end{document}
\else
\fi
