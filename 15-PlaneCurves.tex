
\chapter{Using a singular plane model}
\label{PlaneCurvesChapter}
\label{PlaneCurveChapter}


In the first part of Chapter~\ref{genus 1 chapter} we showed how to
use an embedding of a smooth curve $C$
in $\PP^2$ to understand differentials and linear series on $C$.
In that form,
the technique
has limited applicability,
since most smooth curves cannot be embedded in
$\PP^2$.
However, by Proposition~\ref{nodal projection},
 any smooth curve $C$ can be projected
birationally to a curve $C_0\subset\PP^2$ with only nodes.
We open this chapter by showing
how to use such a nodal model $C_0$ to describe differentials and
linear series on $C$, a theory well-understood
by Brill and Noether;  and then explain
what is necessary to adapt the technique to birational images of $C$
with arbitrary singularities.

\section{Nodal plane curves}\label{nodal curves section}

The methods of Chapter~\ref{3b}
 can be applied, with one change, when $C_{0}\subset \PP^{2}$
\index{nodal curve}%
is a nodal curve.

Let $\nu: C\to C_{0}\subset \PP^{2}$ be the
\index{normalization}%
normalization morphism from a smooth curve,
and let $B\subset C_{0}$ be a subscheme.
It will be convenient to speak of
\index{linear series!cut out by curves}%
\emph{the linear series cut out on $C$} by curves of degree $m$
containing $B$: though $B$ is only a subscheme, $\nu^{-1}(B)$ may be
considered as a
Cartier divisor
\index{Cartier divisor}%
 because
$C$ is smooth, and we define
the linear series cut out on $C$ by curves of degree $m$
containing $B$ to be
the linear series  $\sV$ of divisors in $C$ that are residual to $\nu^{-1}(B)$ in the pullbacks
of the intersections of $C_{0}$ with curves of degree $m$.
Formally, since $B' \colonequals  \nu^{-1}(B)$
is a Cartier divisor on $C$, we can say that if $L$ is the divisor of a line in $\PP^{2}$ and $L'$
its pullback to $C$ then
$\sV$ is a space of divisors corresponding to
sections of $\sO_{C}(mL'-B')$; more precisely, it is the space of divisors corresponding to
sections of $\sO_{C}(mL'-B')$ in the image of the restriction/pullback map
$$
H^0(\cI_{B/\PP^2}(m)) \to H^0(\sO_{C}(mL'-B')).
\unif
$$

\subsection{Differentials on a nodal plane curve}\label{canonical series on nodal plane curves}

Let $C_{0} \subset \PP^2$  be a curve of degree $d$ with $\delta$
nodes and no other singularities. By the
adjunction formula
\index{differential form!on nodal plane curve}%
\index{nodal curve!differential on}%
\index{adjunction formula}%
 (Proposition~\ref{adjunction}),
Proposition~\ref{pa and delta}, and the first example that follows it,
the genus $g$ of the normalization $C$ of $C_{0}$ is
the arithmetic genus $p_{a}(C_{0}) = \tbinom{d-1}{2}$
of $C_{0}$ minus $\delta$, that is,
$$
g = \mbinom{d-1}{2} -\delta.
$$
We will make this explicit by
\index{regular!differential}%
\index{differential!regular}%
exhibiting a vector space of $g$ regular differential forms on $C$.

Choose homogeneous coordinates  $[X,Y,Z]$ on $\PP^2$ so that
$C_0$ intersects the line $L = V(Z)$ in a divisor $D$ consisting only
of smooth points of $C_{0}$  other than $[0,1,0]$, and
so that at each node of $C_0$ (necessarily contained in the affine plane $U = \PP^2 \setminus L$) the tangents to $C_0$ have finite slope.
Let the nodes of $C_0$ be $q_1,\dots,q_\delta$, with $r_i, s_i \in C$ lying
over $q_i$; we'll denote by $\Delta$ the divisor $\sum r_i + \sum s_i$ on $C$.

Let $F(X,Y,Z)$ be the homogeneous polynomial of degree $d$ defining
the curve $C_0$, and let $f(x,y) = F(x,y,1)$ be the defining equation
of the affine part $C_{0}^{\circ}\colonequals  C_0 \cap U$ of $C_0$.
Let $\nu: C\to C_0$ be the normalization map. We start by considering
the rational differential
\index{rational!differential}%
$\nu^*(dx)$ on
$C^{\circ}\colonequals  \nu^{-1}(C_{0}^{\circ})$.

In the smooth case where $C_{0}=C$ we saw that this differential was
regular and nonzero on $C^{\circ}$; this followed from the fact
that $f_{x}$ and $f_{y}$ had no common zeroes on $C_0$. But now
$f_{x}$ and $f_{y}$ have common zeroes: they both vanish to order~1 at
the points $q_{i}$ and thus $\nu^*(f_{x})$ and $\nu^*(f_{y})$ have
simple zeroes at the points $r_i$ and $s_i$.

As before, the differential $\nu^*(dx)$
has  double poles along the divisor
$D$ on $C_{0}$ lying over the point at infinity in $\PP^{1}$
and we see that for a polynomial $e(x,y)$ of degree $\leq d-3$, the
differential
$$
\nu^*\left( \frac{e(x,y)\,dx}{f_{y}}\right)
$$
is regular except for simple poles at the points $r_i$ and $s_i$.

We can get rid of these poles by requiring that $e$ vanishes at the
points $q_i$. We say in this case that $e$ (and the curve defined by $e$)
 \index{conditions of adjunction}.
\emph{satisfies the conditions of adjunction}.

\begin{theorem}\label{canonical from adjoint 1}
If $C_{0}$ is a nodal plane curve of degree $d$ with normalization $\nu:
C\to C_{0}$
then the  regular differentials on  $C$, in terms of the notation above,
 are precisely those of the form
 $$
\nu^{*}\biggl( \frac{e(x,y)\,dx}{f_{y}}\biggr)
,
$$
where
$e(x,y)$ ranges over the polynomials of degree $\leq d-3$
vanishing at the nodes of $C_{0}$.

Thus if $\adj(C_{0})\subset C_{0}$ denotes the union
\index{F@$\adj(C_{0})$}%
of the reduced points at the nodes of $C$, then $|\omega_{C}|$ is the
linear series cut out on $C$ by
forms of degree $d-3$ containing $\adj(C_{0})$.
\unif
\end{theorem}

See Figure~\ref{canonical on normalization} for a picture in a case
where $C_0$ has a single node.
\begin{proof}
The dimension of the space of polynomials $e(x,y)$ of degree at most
$d-3$ is $\tbinom{d-1}{2}$,
and vanishing at $\delta$ nodes imposes at most $\delta$ linear conditions
on $e$. The linear map sending
$e\mapsto \nu^{*}(e\,dx/f_{y})$ is injective, and the target has dimension
$\tbinom{d-1}{2}-\delta$, so this must be an isomorphism.
\end{proof}
We will give a more conceptual proof of this theorem in
Section~\ref{arbitrary plane curves}.

In particular, Theorem~\ref{canonical from adjoint 1}
shows that the linear series cut out on $C$ by
forms of degree $d-3$ containing $\adj(C_{0})$ is complete. (We will
soon see that
 the linear series cut out on $C$ by
forms of degree $m$ containing $\adj(C_{0})$ is complete for every $m$.)


This gives another proof of Lemma~\ref{adjoint independent}.

\begin{corollary}
If $C$ is a nodal plane curve of degree $d$, then the nodes of $C_{0}$
impose independent
\index{independent conditions}%
conditions on forms of degree $d-3$.
\unif
\end{corollary}

\begin{proof}
 Otherwise the space of differential forms on the normalization of $C_{0}$
 would be too large.
\end{proof}

One can use Theorem~\ref{canonical from adjoint 1} to re-embed a plane
curve of
geometric genus $g$ as a
canonical curve
\index{canonical curve}%
in $\PP^{g-1}$:

\begin{corollary}
 The canonical ideal of the normalization of a nodal plane curve of
 degree $d$ is the ideal of polynomial relations
 among the forms of degree $d-3$ that vanish at the nodes of the
 curve. \qed
\end{corollary}

\begin{figure}
\centerline {\includegraphics[width=3in]{"main/Fig14-2"}}
\caption{A curve $C$ of geometric genus 2 represented as the normalization
of a plane curve $C_{0}$ of degree 4 with a node, and a canonical divisor,
represented by a conic containing the node.}
\label{canonical on normalization}
\end{figure}

\subsection{Linear series on a nodal plane curve}
\label{linear series on nodal plane curves}

Since $C_{0}$ is singular, not every
effective divisor
\index{effective divisor}%
on $C$ is the
\index{linear series!on nodal plane curves}%
preimage of an
effective Cartier divisor on $C_{0}$. As an example, one may take a
single point lying over a node
as in Figure~\ref{Fig14.2}.

However,
we can still represent every divisor on $C$ as the preimage of a divisor
on $C_{0}$ up to linear
equivalence, and the same goes for any reduced curve:

\begin{lemma}
Let $\nu: C\to C_{0}$ be the normalization of any reduced projective
\index{normalization}%
curve. If $D$ is any divisor
on $C$, then $D$ is linearly equivalent to the pullback of a divisor
supported on the smooth locus of  $C_{0}$. More precisely,  every
effective divisor on $C$ containing $\Delta$ can be written as the
pullback
of a
Cartier divisor
\index{Cartier divisor}%
on $C_{0}$.
\unif
\end{lemma}

We will see a more general version in Theorem \ref{Cartier on C}.

\begin{proof}
It suffices to prove the result locally on $C_{0}$, where it is
geometrically obvious:  if the node
$p\in C_{0}$ has preimages $q,r$ corresponding to  branches $Q$ and $R$ of $C_{0}$
at $p$, then
a divisor $aq+br$ with both $a,b$ strictly positive, is locally the
pullback of the intersection of $C_{0}$
with a curve $C'$ meeting $Q$ with multiplicity $a$  at $p$ and meeting $R$
with multiplicity $b$ at $p$. For example, assuming that $a\leq b$, we could take
$C'$ to be the union of $a-1$ general lines through $p$ and a smooth plane curve
meeting $R$ with multiplicity $b-a+1$ at $p$.
\end{proof}


Returning to the case of a nodal plane curve $C_{0}$ and its normalization
$\nu: C\to C_{0}$,
suppose that $D$ is a divisor on $C$ that is the pullback of a difference
of Cartier
divisors $D_{+}-D_{-}$ on $C_{0}$. We will compute the complete linear
series $|D|$.
Let $\adj(C_{0})$ be the set of nodes in $C$ and let $\Delta$ be the
preimage of $\adj(C_{0})$ in $C$.

\begin{theorem}\label{linear series on nodal curves}
Let $D = D_{+}-D_{-}$ be a divisor on $C$, and let $G$ be a form on
$\PP^{2}$
that vanishes on  $D_{+}+\adj(C_{0})$ but not identically on $C_{0}$.

If $G$ has degree $m$ and $A = (\nu^{*}G)-D_{+}-\Delta$, then every
effective divisor on
$C$ linearly equivalent to $D$ (if any) has the form
$(\nu^*H)-D_{-}-A-\Delta$ for some $H$ of degree $m$
that vanishes on $\adj(C_{0})+A$ but not identically on $C_{0}$.
\unif
\end{theorem}

The reason for including $\adj(C_{0})$ is that an arbitrary linear series on $C$
cannot be represented as the pullback of a linear series on $C_{0}$ without this.
We will see a generalization in Theorem \ref{linear series on arbitrary curves} and the surrounding discussion.

\begin{corollary}
With notation as in Theorem~\ref{linear series on nodal curves}, the
ideal of the image of
$C$ with respect to the linear series $|D|$ is the ideal of polynomial
relations among the forms
of degree $m$ in $\ff_{C/C_{0}}$ that vanish on $D_{-}+ A$.\qed
\unif
\end{corollary}

\begin{proof}[Proof of Theorem~\ref{linear series on nodal curves}]
 Choose an integer $m$
big enough
that there is a form $G$ vanishing
 on  $D_{+}$ and $\adj(C_{0})$ so that
 $\nu^{*}(G)$ vanishes on $D_{+}+\Delta$, but not everywhere on $C$.
 (If $D_{+}$ contains some positive multiple of a point $p$ of $\Delta$
 this means that $G$ defines
 a curve sufficiently tangent to the corresponding branch of $C_0$.)
Then, as before,
we can write the zero locus of $G$ pulled back to $C$ as
$$
(\nu^*G) = D_{+} + \Delta + A.
$$

\begin{figure}
\centerline {\includegraphics[width=3.0in]{"main/Fig14-3"}}
\caption{If $G$ is tangent to a branch of $C_{0}$ at the node, then
$D_{+}$ contains
a point of $\Delta$ on $C$.}
\label{Fig14.2}
\end{figure}

Next, we look for forms $H$ of the same degree $m$, vanishing at $A+D_{-}$
and on $\adj(C_{0})$
 but not on all of $C_0$. If there are no such polynomials $H$ then,
 as we shall show,
there are no effective divisors equivalent to $D$. Supposing that there
is such a form $H$, let $D'$ be the divisor
$$
D' = (\nu^*H) -( D_{+} + \Delta),
$$
that is, $D'$ is residual to $( D_{+} + \Delta)$ in $(\nu^*H)$.

Since $\nu^*(G/H)$ is a rational function on $C$ we have
$$
D_{-} +\Delta + A+ D' = (\nu^*H) \sim (\nu^*G) = D_{+} + \Delta + A,
$$
and thus $D'$ is an effective divisor linearly equivalent to $D =
D_{+}-D_{-}$ on $C$.

To complete the argument we must show that we get \emph{all} divisors $D'$
in this way.
In this case the curve $C$ can be desingularized by blowing up the plane
once at each node,
and we can give a proof based on the resulting surface $S$. The same
technique would work for any curve with only
\index{ordinary multiple points}%
\index{total transform}%
\index{normal!crossing}%
ordinary multiple points, in which case the total transform of $C_{0}$ on
$S$ has normal crossings. We will give a different proof, extending this
theorem to curves with arbitrary singularities, in Section~\ref{arbitrary
plane curves}.

\begin{proposition}\label{adjoint completeness1}
If $C_{0}$ is a reduced irreducible plane curve all of whose singularities
are ordinary nodes, then for each
integer $m$,
the linear series cut out on the normalization $C$ of $C_{0}$ by forms
of degree $m$ containing the nodes
is complete.
\end{proposition}

\begin{proof}
To prove Proposition~\ref{adjoint completeness1}, we work on the blowup
$\pi : S \to \PP^2$ of $\PP^2$ at the nodes $q_i$ of $C_0$. The proper
transform of $C_0 \subset \PP^2$ in $S$ is the normalization of $C_0$,
\index{normalization}%
which we will again call $C$.

Let $L$ be the class on $S$ of the pullback of a line in $\PP^2$
and let $E$ be the sum of the exceptional divisors, the preimage of
$\adj(C_{0})$. We write $h= L\cap C$ and $e = E\cap C= \sum (p_i+q_i)$
for the corresponding divisors on $C$.
Because $C$ has double points at each $q_{i}$ we have
$
C \sim dL - 2E
$
and   by
Theorem~\ref{divisor classes on blowup} we have $K_S \sim -3L + E$.

The proper transform of a degree $m$ curve $A\subset \PP^2$  passing
simply through the points $q_i$
is $\pi^*A - E$; this gives an isomorphism
$$
H^0(\cI_{\{q_1,\dots,q_\delta\}/\PP^2}(m)) \cong H^0(\cO_S(mL-E)).
$$
In these terms we can describe the linear series cut on $C$ by plane
curves of degree $m$ passing through the nodes of $C_0$ as the image of
the map
$$
H^0(\cO_S(mL-E)) \to H^0(\cO_C(mL-E)),
$$
and we must show that this map is surjective.

From the long exact cohomology sequence associated to the exact sequence
of sheaves
$$
0 \to \cO_S((m-d)L + E)  \to \cO_S(mL-E) \to \cO_C(mL-E) \to 0,
$$
 we see that it will suffice to prove that $H^1(\cO_S((m-d)L + E)) = 0$.

By Serre duality on $S$,
$$
H^1(\cO_S((m-d)L + E)) \cong H^1(\cO_S((d-m-3)L))^*.
$$
The line bundle $\cO_S((d-m-3)L)$ is
 the pullback to $S$ of the bundle $\cO_{\PP^2}(d-m-3)$, which has
 vanishing $H^1` `$. Lemma~\ref{H1 on pullback} completes the proof.
\end{proof}

\begin{lemma}\label{H1 on pullback}
Let $X$ be a smooth projective surface, and $\pi : S \to X$ the
\index{blowup}%
blowup of
a finite set of reduced points. If $\cL$ is any line bundle on $X$, then
$$
H^1(S, \pi^*\cL) = H^1(X, \cL).
$$
\end{lemma}

\begin{proof} Because $\PP^2$ is normal, and $\pi_*(\sO_S)$ is a
finite birational algebra over $\sO_{\PP^2}$, we have $\pi_*(\sO_S)
= \sO_{\PP^2}$.
Since any invertible sheaf $\sL$ on $\PP^{2}$ is locally isomorphic to
$\sO_{\PP^2}$,   is also an isomorphism.

The
\index{Leray spectral sequence}%
Leray spectral sequence
(Theorem~\ref{Leray}) gives an exact sequence
$$
0\to H^{1}(\pi_{*}(\sL)) \to H^{1}(\sL) \to  H^{0}(R^{1}(\pi_{*}(\sL))\to
0
$$

 The restriction of $\pi^{*}(\sL)$ to any fiber of $\pi$ is trivial and
 has vanishing $H^{1}$,
so
$H^{0}(R^{1}(\pi_{*}(\sL))) = 0$, and $\pi_{*}\pi^{*}\sO_{\PP^{2}}(1)$
is an invertible sheaf.
The natural map
$\pi_{*}\pi^{*}(\sL) \to \sL$ is an isomorphism away from the codimension
2 set of points blown up. Thus these two sheaves are isomorphic, and
$$
H^1(S, \pi^*\cL) = H^{1}(\pi_{*}\pi^{*}\sL) = H^{1}(\sL),
$$
completing the proof.
\end{proof}

This concludes the proof of Theorem~\ref{linear series on nodal curves}
\end{proof}



\begin{proposition}\label{effect of blowup on genus}
 Let $C$ be a curve on a smooth surface $S$, and let $\nu : S' \to S$
 be the blowup of $S$ at $p$. If $C'$ is the strict transform of $C$, then
 $$
 p_a(C') = p_a(C) -{m\choose 2},
 $$
 where $m$ is the multiplicity of $p\in C$.
\end{proposition}
\begin{proof}
This follows from comparing the adjunction formulas on $S$ and $S'$. To
start, we have
$$
p_a(C) = \frac{C^2 + K_S\cdot C}{2} + 1.
$$
On $S'$ let $E$ be the divisor class of the exceptional divisor. As
we've seen,
$$
K_{S'} = \nu^*K_S + E,
$$
while the class of $C'$ is given by
$$
C' \sim \nu^*C - mE.
$$
It follows that
$$
(C')^2 = C^2 + m^2E^2 = C^2 - m^2 \quad \text{and} \quad K_{S'}\cdot C'
= K_S\cdot C + m.
$$
Thus, applying adjunction on $S'$, we find
the desired equality:
$$
\medmuskip2mu
p_a(C') = \frac{C'{}^2 + K_{S'}\cdot C'}{2} + 1 = \frac{C^2 + K_S\cdot
C - m(m-1)}{2} + 1 = p_a(C) -\tbinom{m}{2}.
\eqno\qed
$$
\let\qed\relax
\end{proof}

\begin{fact}
Any plane curve can be desingularized by
iteratively blowing up of singular points of $C$, then of the strict
transform, and so on. See for example
\cite{Fulton1989} or \cite{Brieskorn1986}.  This is in fact the same
sequence of transforms as the one given in Exercise~\ref{Mumford
resolution argument}.
The points on the various blowups that
map to the original singular point are called \emph{infinitely near
points}.
\end{fact}

This gives a nice formula for the $\delta$ invariant of any singularity:

\begin{corollary}
\label{computing delta}
The $\delta$ invariant of any singularity of a plane curve $C_{0}$ at
a point $p$ can be computed as the sum of the numbers $\tbinom{m_{q}}{2}$
over all infinitely near singular points $q$,
\index{infinitely near point}%
where
$m_{q}$
denotes the multiplicity of the pullback of $C_{0}$ at $q$.\qed
\end{corollary}

\section{Arbitrary plane curves} \label{arbitrary plane curves}

Throughout this section $C_0 \subset \PP^2$ denotes a reduced and
irreducible plane curve with arbitrary singularities. Let $\nu : C
\to C_0$ be its normalization, and write $L'$ for the pullback to $C$
of the class $L$ of a line in $\PP^{2}$.

It is possible to carry out an analysis of linear series on the
normalization of an arbitrary plane curve in a manner  analogous to
\index{normalization}%
what we did in the preceding section for nodal curves, replacing
the set of nodes by the
\emph{adjoint scheme}
\index{adjoint scheme}%
\index{F@$\adj(C_{0})$}%
$\adj(C_{0})\subset C_{0}$, which in the case of a plane curve is the scheme
\index{conductor|defi}%
\index{f@$\ff_{C/C_0}$}%
defined by the \emph{conductor ideal} $\ff_{C/C_0}$, the annihilator
in $\sO_{C_{0}}$ of $\nu_{*}(\sO_{C})/\sO_{C_{0}}$
(see Theorem~\ref{general adjoint}).

\subsection*{The conductor ideal and linear series on the normalization}
%\label{why add Delta}

Let
$\Delta\subset C$ be the
divisor
defined by the pullback of $\ff_{C/C_0}$ to $C$.
The following result gives a simple way of expressing any divisor on $C$
\index{delta@$\Delta$ (Cartier divisor)}%
in terms of $\Delta$ and a
Cartier divisor on $C_{0}$:

\begin{theorem}\label{Cartier on C}
If $D$ is an effective divisor on $C$ that contains $\Delta$,
then $D$ is the pullback of a Cartier divisor on $C_{0}$.
\unif
\end{theorem}

\begin{proof}
The result is local, so it suffices to treat the affine case of an affine
curve $C_{0}$ with coordinate
ring $\sO'$ contained
in its integral closure $\sO$, the coordinate ring of its normalization
\index{normalization}%
$C$.
The ideal of $\Delta$ in $\sO$ is the conductor $\ff_{\sO/{\sO'}}$
(which is contained in $\sO'$, but stable under
multiplication by elements of $\sO$, and thus also an ideal of $\sO$).
Thus $\sI_{D}$ may be regarded as an ideal\emdash though not necessarily
a principal ideal\emdash of $\sO'$. Since the  ground field $\CC$ is
infinite, $\sI_{D}\subset \sO'$ is the integral closure of a principal
ideal $(x)$
of $\sO'$.
(See \cite[Chapter 8]{Swanson-Huneke}, for example.)
This means that for sufficiently large $n$ we have $x\sI_{D}^{n} =
\sI_{D}^{n+1}$.

Let $D_{0}$ be the Cartier divisor on $C_{0}$ corresponding to
$x$. Pulling everything back to $C$
we have
$
\nu^*(D_{0})+nD = (n+1)D
$,
whence $\nu^*(D_{0}) = D$.
 \end{proof}

In view Theorem~\ref{Cartier on C} we can specify an arbitrary linear
series on $C$
as the difference $\cE - \Delta$, where $\cE$ is a linear series with
$\Delta$ in its base locus, by specifying a
linear series on $C_0$ containing  $\adj(C_{0})\subset C_{0}$. The
following theorem makes
this algorithmic.

\begin{theorem}\label{linear series on arbitrary curves}
Let $D = D_{+}-D_{-}$ be a divisor on $C$, and let $G$ be a form on
$\PP^{2}$,
vanishing on $\adj(C_{0})$, whose pullback to $C$
vanishes on $D_{+}$ but not on all of $C$.

If $G$ has degree $m$ and $(\nu^{*}G) = D_{+}+A+\Delta$, then every
effective divisor on
$C$ linearly equivalent to $D$ (if any) has the form
$(\nu^*H)-D_{-}-A-\Delta$ for some $H$ of degree $m$
 vanishing on $\adj(C_{0})+A$. Thus, the global sections of
 $H^{0}(\sO_{C}(D))$ can be written
as divisors cut by forms of degree $m$ in $\ff_{C/C_{0}}$ with basepoints
at $D_{-}+A+\Delta$.
\unif
\end{theorem}

It follows that one can compute the image of $C$ under the
map given by $|D|$ directly from $C_{0}$ and the conductor:

\begin{corollary}
With notation as in Theorem~\ref{linear series on arbitrary curves},
the ideal of the image of
$C$ with respect to the linear series $|D|$ is the ideal of polynomial
relations among the forms
of degree $m$ in $\ff_{C/C_{0}}$ that vanish on $D_{-}+A$.\qed
\unif
\end{corollary}

\begin{proof}[Proof of Theorem~\ref{linear series on arbitrary curves}]
If $H\in \ff(C_{0})$ vanishes on $A$ but not on all of $C_{0}$, then
as before
$$
 D_{-} +\Delta + A+ D' = (\nu^*H) \sim (\nu^*G) = D_{+} + \Delta + A,
$$
So $D' = (\nu^*{H})-D_{-}-A-\Delta$ is linearly equivalent to $D$. The
proof that every
divisor $D'$ linearly equivalent to $D$ has this form is the content of
Theorem~\ref{conductor completeness} below,
which was known classically as the \emph{completeness of the adjoint series}.
\index{adjoint series, completeness of}%
\index{completeness of the adjoint series}%
\end{proof}

\begin{theorem}\label{conductor completeness}
For every integer $m\geq 0$ the series cut out on $C$ by forms of
degree $m$
on $\PP^{2}$ containing $\adj(C_{0})$ is complete.
\end{theorem}

\begin{proof}
If $R_{0}\subset R$ is an inclusion of commutative rings, then the ideal
$$\ff_{R/R_{0}} \colonequals  \ann_{R_{0}}(R/R_{0})\subset R_{0}$$
is called the \emph{conductor} of $R_{0}\subset R$.
\index{conductor}%
It is by
definition an ideal of $R_{0}$, but is also an ideal of $R$; this follows
because if $f\in \ff_{R/R_{0}}$ and $r\in R$, then $fR\subset R_{0}$ so
 $(rf)R = frR \subset fR \subset R_{0}$.

If $R_{0}$ is a domain and $R$ is a subring of the quotient field $Q(R)$
of $R$, then
 $\ff_{R/R_{0}} \cong \Hom_{R_{0}}(R, R_{0})$. To see this, note that
 $R_{0}$ and $R$ become
 equal after tensoring with $Q(R_{0})$ and thus
 $\Hom_{R_{0}}(R,R_{0}) \subset \Hom_{Q}(Q,Q) = Q$
 may be identified
 with the set of elements $\{\alpha\in Q\mid \alpha R \subset R_{0}\}$. If
 $\alpha$ is in this set, then
  $\alpha\cdot 1 = \alpha \in R_{0}$, as required.

Returning to the case of the curve $C_{0}$, it follows that the global
sections of $(\nu^{*}(\sO_{C_{0}}(m))( -\Delta)$ on $C$
are, on each affine open set $U$, represented by the elements of
$\sO_{C_{0}}(U)$ that  are restrictions to $U$
of forms of degree $m$ contained in
the ideal $\ff_{C/C_{0}}$. Thus
the global sections of the sheaf $\widetilde{\ff_{C/C_{0}}}(m)$ cut out
a complete linear series on $C$.

Write $S =\CC[x_{0}, x_{1}, x_{2}]$ for the homogeneous coordinate ring
of $\PP^{2}$.
It remains to prove that the homogeneous ideal $\ff_{C/C_{0}}$ maps
surjectively to $H^{0}_{*}(\widetilde{\ff_{C/C_{0}}})$, and this amounts
to the
statement that the depth of $\ff_{C/C_{0}}$ as an $S$-module is
at least 2.
Set $R_{0} = H^{0}_{*}(\sO_{C_{0}})$ and $R = H^{0}_{*}(\nu_{*} (\sO_{C}))$.
We see from the general considerations above that
$$
\ff_{R/R_{0}} = \Hom_{R_{0}}(R, R_{0}).
$$
Any nonzerodivisor on a module $M$
is a nonzerodivisor on $\Hom(P, M)$ for any module $P$ since $(a\phi)(p)
= a(\phi(p))$ by definition.
Since $R_{0} = S/(F)$, it is a module of depth 2, and we may choose a
regular sequence
$a,b$ of elements in~$R_{0}$. From the short exact sequence
$$
0\to R_{0}\ruto {\ a} R_{0}\to R_{0}/(a) \to 0
$$
we get a left exact sequence
$$
0\to \Hom_{R_{0}}(R,R_{0})\ruto {\ a} \Hom_{R_{0}}(R,R_{0})\to
\Hom_{R_{0}}(R,R_{0}/(a)).
$$
Thus
$$
\Hom_{R_{0}}(R,R_{0})/(a\Hom_{R_{0}}(R,R_{0})) \subset
\Hom_{R_{0}}(R,R_{0}/(a))
$$
and since $b$ is a nonzerodivisor on $\Hom_{R_{0}}(R,R_{0}/(a))$, it is
a nonzerodivisor
on $\Hom_{R_{0}}(R,R_{0})/(a\Hom_{R_{0}}(R,R_{0}))$ as well.
\end{proof}

Since we saw directly that the adjoint ideal was equal to the conductor
ideal in the case of
a nodal curve, this result gives another, less ad hoc, proof that the
effective divisors equivalent to $D$
are all defined by
pullbacks of forms of degree $m$ that contain $\Delta$
as constructed in Proposition~\ref{adjoint completeness1}.

\subsection*{Differentials}

Let $C^\circ_0$ be the intersection of $C_0$ with the open set
$\AA^{2}\cong U\subset \PP^{2}$ where $Z \neq 0$,
and let $C^\circ \subset C$ be the preimage of $C^\circ_0$.

\begin{theorem}\label{general differentials}
If $C_{0}$ meets the line $L$ at infinity only in smooth points of
$C_{0}$ other than $(0,1,0)$, then the complete canonical series on the
normalization $\nu: C \to C_{0}$ is cut out by differentials of the form
$$
 \frac{e(x,y)\,dx}{f_{y}}
$$
where $e(x,y)$ is a polynomial of degree $\leq d-3$ contained in the
conductor ideal $\ff_{C^{\circ}/C_{0}^{\circ}}$.
\unif
\end{theorem}

As in the case of nodal curves, one can use
Theorem~\ref{general differentials} to re-embed a plane curve of
geometric genus $g$ as a canonical curve in $\PP^{g-1}$:

\begin{corollary}
 The canonical ideal of the normalization $C$ of a plane curve $C_{0}$
\index{normalization}%
\index{canonical ideal}%
\index{conductor}%
 of degree $d$
 is the ideal of polynomial relations
 among the forms of degree $d-3$ in the
conductor ideal
 $\ff_{C/C_{0}}$. \qed
\end{corollary}

\begin{proof}[Proof of Theorem~\ref{general differentials}]
We proceed in four steps.
First, because  $(0,1,0)$ does not lie on $C$,
the function $x$ defines a
ramified $d$-sheeted cover of $C$ to $\PP^{1}$. Because $C_{0}$ meets $L$
only in smooth
points and the differential $dx$ has a pole of order 2 at the point at
infinity in $\PP^{1}$,
$dx$
has polar locus twice the divisor $ \nu^{-1}(C_{0}\cap L)$. It follows
that
the differential
$\varphi_0 \colonequals  dx/f_{y}$ is regular, with a zero of order $d-3$,
along the divisor of $C$ lying over $C_0\cap L$.

Second, the function on $C^\circ_0$ defined by $x$
is a finite map to $\AA^1` `$, and thus the field of rational functions
$\kappa(C) = \kappa(C^\circ_0)$ is a finite
separable extension of $\CC(x)$. By \cite[Section 16.5]{Eisenbud1995},
the module of differentials
$\omega_{\kappa(C)/\CC}$ is generated over $\kappa(C^\circ_0)$ by
$dx$. Thus every rational
differential form on $C$ can be expressed as a rational function
times $dx$. Since $\varphi_{0} \colonequals  dx/f_{y}$ vanishes to order
$d-3$ along $C_{0}\cap L$,
the regular differential forms on $C$ must be of the form
$e(x,y)\varphi_{0}$ where
$e(x,y)$ is a rational function of degree $\leq d-3$. (The set of rational
forms that occur in this
\index{Dedekind complementary module}%
way is called the
\emph{Dedekind complementary module}.)

Third,
a sophisticated form of Hurwitz's theorem,
to be
explained in
Chapter~\ref{LinkageChapter},
shows that the sheaf $\omega_{C}$ of regular differential forms on $C$
can be expressed as $\nu^{!}\pi^{!}\sHom_{\PP^{1}}(\pi_{*}\nu_{*}(\sO_{C})
\omega_{\PP^{1}})$, where $\pi$ is the map $C_{0}\to \PP^{1}$ defined by $x$.
Locally, this is expressed more simply as
$$
\sHom_{\PP^{1}}(\nu_{*}(\sO_{C}), \omega_{\PP^{1}}),
$$
where we use the action of $\sO_{C}$ on $\nu_{*}(\sO_{C})$.
 Since the maps involved are finite,
we will  identify $\sO_{C}$  with $\nu_{*}(\sO_{C})$
 and write
$$
\omega_{C} = \sHom_{\PP^{1}}(\sO_{C}, \omega_{\PP^{1}}),
$$

Since $\sO_{C_{0}}\subset \sO_{C}$, this sheaf is naturally
contained
in
$$
 \sHom_{\PP^{1}}(\sO_{C_{0}}, \omega_{\PP^{1}}).
$$
As will be explained in Chapter~\ref{LinkageChapter},
$$
\omega_{C_{0}}\colonequals  \sHom_{\PP^{1}}(\sO_{C_{0}}, \omega_{\PP^{1}})
=
\sHom_{\PP^{1}}(\sO_{C_{0}}, \sO_{\PP^{1}})(-2)
$$
 is properly called the
dualizing module
\index{dualizing module}%
of the singular curve $C_{0}$.
We will show in Theorem~\ref{general adjoint} that
$\ff_{C/C_{0}} $ is the annihilator of the quotient of two $\sO_{c_{0}}$ modules
\vspace*{3pt}
$$
\vspace*{3pt}
\ff_{C/C_{0}} = \ann_{C_{0}}
\frac{\sHom_{\PP^{1}}(\sO_{C_{0}}, \omega_{\PP^{1}})}
{\sHom_{\PP^{1}}(\sO_{C}, \omega_{\PP^{1}})}.
$$
We will also see that $\sHom_{\PP^{1}}(\sO_{C_{0}}, \omega_{\PP^{1}})$
is an invertible sheaf on $C_{0}$, and since the quotient is supported only
at the finite set of singularities of $C_{0}$, it follows that every regular differential form on $C$ can be expressed as an
element of $\ff_{C/C_{0}} $
times some element of $\omega_{C_{0}}$.

 To prove the theorem, it now suffices to
show that
$\varphi_{0}=dx/f_{y}$ generates $\omega_{C_{0}}$ as a module over $\sO_{C_{0}}$.

Passing to the field of rational functions $\kappa \colonequals
\kappa(C)$, and noting that
\index{kappa@$\kappa$}%
$\kappa(C) = \kappa(C_{0}) $ we use the well-known result from
Galois theory
\index{Galois theory}%
that
$$
\Hom_{\kappa(\PP^{1})}(\kappa, \kappa(\PP^{1}))
$$
is a 1-dimensional vector space over $\kappa$, generated by the trace map.
\index{trace map}%
$\Tr$. Moreover,
because $\sO_{C_{0}}$ is integral over $\sO_{\PP^{1}}$, which is normal,
$$
\Tr(\sO_{C_{0}})\subset \sO_{\PP^{1}}.
$$

For the fourth and last step we need a
result from commutative
algebra:

\begin{theorem}\label{Kunz}
If $C_{0}^{\circ}$ is an affine plane curve defined by the
equation $f(x,y)=0$ and such that $\CC[x,y]/(f)$ is finite over $\CC[x]$,
then $\Hom_{\CC[x]}(\CC[x,y]/(f), \CC[x])$ is generated by $(1/f_{y})T$.
\unif
\end{theorem}

See \cite[Theorem 15.1]{Kunz} for a proof using valuations, and
\cite[Theorem A.1]{MR4026452} for a proof in
a more general context.

Given Theorem~\ref{Kunz}, we can complete the proof of
Theorem~\ref{general differentials}. Via the isomorphism $\CC(x) \cong
\CC(x)\,dx$ sending 1 to $dx$, the trace map is identified (up to scalar)
with the map $(1\mapsto dx) \in \Hom_{\CC(x)}(\kappa, \CC(x)\,dx)$. Thus
by Theorem~\ref{Kunz}
the canonical module of $C_{0}^{\circ}$ is identified with
$\sO_{C_{0}^\circ} \phi_{0}$ as
required.
\end{proof}

In Examples~\ref{nodes and cusps}--\ref{ord n-fold}, we will consider a singular
 curve defined in the affine plane by an equation $f(x,y) = 0$
 such that $\CC[x,y]/(f)$ is finite over $\CC[x]$, and write $\varphi_{0} = dx/f_{y}$
 for a local generator of the module of differentials.

\begin{example}[nodes and cusps]
\label{nodes and cusps}
We have already seen that in case $C_{0}$ has a node at a point $q$ there are
two points of $C$ lying over $q$, and the multiplicities of $\varphi_0$
at these two points are $m_1=m_2=1$; the adjoint ideal is thus
\index{cusp}%
 the maximal ideal $\cI_q$ at $q$.
When $q$ is a
cusp
of $C_{0}$ \emdash that is, $C_{0}$ is
analytically
 isomorphic to the zero locus of $y^2-x^3` `$ in a neighborhood of the point $q = (0,0)$ \emdash there is only one point
 $r\in C$ lying over
$q$. In an analytic neighborhood of the cusp, $C_{0}$ can be parametrized
 by $x = t^{2}$, $y = t^{3}$
and it follows that the differential
 $$
 \varphi_0 = \frac{dx}{f_{y}} =  \frac{2t\,dt}{2t^{3}} =  \frac{dt}{t^{2}}
 $$
 has a pole
of order 2 at $r$. Since the pullback to $C$ of any polynomial
 $g$ vanishing at the cusp $q$ vanishes to order at least two at $r$,
 the adjoint ideal is again the maximal ideal at $q$. We can also see
 this by computing the
 conductor ideal as the annihilator of $\CC\[t\]/\CC\[t^{2}, t^{3}\]$.
\end{example}

\begin{example}[tacnodes]
Next, consider the case where $C_{0}$ has a tacnode at  $q$ (Figure~\ref{Fig14.4}); that
\index{tacnode}%
is, a plane-curve singularity with two smooth branches simply tangent
to one another, analytically isomorphic to the zero locus of $y^2-x^4`
`$ in a neighborhood of $q=(0,0)$. The two branches are parametrized locally analytically by $x = t$, $y =
t^{2}$ and $x=t$, $y = -t^{2}$.

\begin{figure}
\centerline {\includegraphics[width=2in]{"main/Fig14-4"}}
\caption{A tacnode: two smooth branches tangent to one another.}
\label{Fig14.4}
\end{figure}

At the two points $r_{1}, r_{2} \in C$ lying over $q$, we have
  $$
 \varphi_0 =  \pm\frac{dt}{ 2t^{2}}
 $$
and each of these differentials has a double pole at each of $r_{1}, r_{2}$.
\index{adjoint ideal}%
The adjoint ideal is thus the ideal of functions vanishing at $q$
and having derivative 0 in the direction of the common tangent line to
the branches.
\end{example}

\begin{example}[ordinary $n$-fold points]
\label{ord n-fold}
Consider the case where $C_{0}$ has an ordinary $n$-fold point at  $q$,
where
$n$
smooth branches intersect pairwise transversely.
There are
$n$ points
\index{ordinary $n$-fold point}%
$r_i$ of $C$ lying over $q$. The polynomial $f_y$ vanishes to order $n-1$
at $q$, so $dx/f_y$ has a pole of order $n-1$ at
each $r_i$. It follows that for $e(x,y)\,dx/f_y$ to be regular, $e$ must
vanish to order $n-1$ at each $r_i$.

We can see that the conductor ideal is the full $(n-1)$-rst power of
$(x,y)$ by using the
normalization map
$$
\nu^{*}: \frac{\CC[x,y]}{\prod_{i=1}^{n} (x-\alpha_{i}y)} \to
 \prod_{i=1}^{n} \frac{\CC[x,y]e_{i}}{(x-\alpha_{i}y)}
$$
where the $\alpha_{i}$ are distinct elements of $\CC$ and the $e_{i}$
are orthogonal idempotents.
The element $1 = \sum_{i}e_{i}$ goes to 0 in the cokernel of $\nu^{*}$
and $e_{i}$
is annihilated by $x-\alpha_{i}y$,
so the quotient is annihilated by each of the $n$ elements $g_{j}
\colonequals  \prod_{i\neq j} (x-\alpha_{i}y)$.
As forms on $\PP^{1}$, all the $g_{i}$ except $g_{j}$ vanish at the
point $(\alpha_{j}, 1)$, so the $g_{i}$ are linearly independent. Since
$(x,y)^{n-1}$ is minimally generated by $n$ elements, $(x,y)^{n-1} =
(g_{1}, \dots, g_{n})$.

A consequence of this computation is that the $\delta$ invariant of the
\index{delta invariant@$\delta$ invariant!of ordinary multiple point}%
ordinary multiple point is $\tbinom{n}{2}$, the dimension of
\index{normalization}%
$k[x,y]/(x,y)^{n-1}$, as we proved just after
Theorem~\ref{divisor classes on blowup} and in
Proposition~\ref{effect of blowup on genus}.
\end{example}

\begin{example}[spatial triple points]\label{spatial triple points}
Spatial triple points provide a contrast to the last example. A
spatial triple point is a singularity consisting of three smooth
branches, with linearly independent tangent lines, meeting
in a point~$p$ so that its Zariski tangent space is 3-dimensional; the simplest
\index{triple point!in space}%
example is the origin as a point on the union of the three coordinate
axes in $\AA^3` `$.

In this case the conductor is the annihilator of the cokernel of
\vspace{3pt}
$$
\nu^{*}: R \colonequals  \frac{\CC[x,y,z]}{(xy, xz, yz)} \to
\frac{\CC[x,y,z]}{(x,y)} \times \frac{\CC[x,y,z]}{(x,z)} \times
\frac{\CC[x,y,z]}{(y,z)} =: \ovR.
$$
Since $x\ovR = x \CC[x,y,z]/(y,z)$ is in the image of
$\CC[x,y,z]/(xy, xz, yz)$, and similarly with $y$ and $z$,
we see that the conductor is the maximal ideal $(x,y,z)$. However,
the $\delta$ invariant, the length
\index{delta invariant@$\delta$ invariant}%
of the quotient $(\ovR)/R$, is 2: for a function $f$ in $\ovR$ to be in $R$, it
is necessary and sufficient that $f$ take the same value at the three
points above the singular point,
and this
amounts to two
linear conditions on $f$.
\end{example}

\section{Exercises}

In Exercise~\ref{gonality of smooth plane curve}, we saw how to use the
\index{gonality!of nodal plane curve}%
description of the canonical series on a smooth plane curve to determine its
gonality.
Now that we have an analogous description of the canonical
series on (the normalization of) a nodal plane curve, we can deduce a
similar statement about the gonality of such a curve. Here are the first
two cases:

\begin{exercise}
Let $C_0$ be a plane curve of degree $d\geq 4$ with one node $p$ and no
other singularities, and let $C$ be its normalization. Show that $C$
admits a unique map $C \to \PP^1$ of degree $d-2$, but does not admit
\index{nodal curve!gonality}%
a map $C \to \PP^1$ of degree $d-3$ or less.
\tohint{15.1}
\end{exercise}

\begin{exercise}
Let $C_0$ be a plane curve of degree $d\geq 5$ with two nodes $p$ and $p'$
and no other singularities, and let $C$ be its normalization. Show that
\index{normalization}%
$C$ admits two maps $C \to \PP^1$ of degree $d-2$, but does not admit
a map $C \to \PP^1$ of degree $d-3$ or less.
\tohint{15.2}
\end{exercise}

\begin{exercise}
Generalizing the examples above, show that if a nodal plane curve of
degree $d$ has $\delta\leq d+3$ nodes,
then its gonality is $d-2$, and moreover every
$g^1_d$
\index{g@$g^1_d$}%
on the curve is
given by projection from one of the nodes.
You may use the following result
(see \cite[p.~302]{MR1376653} for a proof):

\begin{unproposition}\label{independent conditions}
 A set of $n \leq 2d+ 2$ distinct
points in the plane fails to impose independent conditions on curves
of degree
$d$ if and only~if either $d + 2$ of the points  are collinear or $n =
2d + 2$ and all the points lie
on a conic.
\tohint{15.3}
\end{unproposition}
\end{exercise}

\begin{exercise}\label{general case of divisors on nodal curves}
Suppose that $C$ is a smooth curve and $\nu: C \to C_0$ is a map to a
plane curve with
only nodes as singularities. Let $D = D_{+}-D_{-}$ be a divisor on
$C$. Modify the
technique of Section~\ref{linear series on nodal plane curves} to compute
the
complete linear series
$|D|$ without assuming that $D$ is disjoint from the
\index{nodal curve}%
preimages of the singular
\index{complete linear series}%
points.
\tohint{15.4}
\end{exercise}

\begin{exercise}
Let $C_0$ be a
plane quartic curve
\index{plane quartic curve}%
\index{quartic curve}%
with two nodes $q_1, q_2$
and
let $\nu :\nobreak C \to C_0$ be its normalization.
\index{normalization}%
By the adjunction formula, $C$ has genus~1.
For an arbitrary point
$o \in C$ not
lying over a node of $C_0$,
 give a geometric description
of the
group law
\index{group law}%
on $C$ with $o$ as origin.
\tohint{15.5}
\end{exercise}

\begin{exercise}
Let $p$ be a point in $\PP^{2}$ and let $C_{0}$ be a curve that, in
a neighborhood of $p$, consists of 3 smooth branches that are
pairwise simply tangent ($C_{0}$~could be given locally analytically
by the equation $y(y-x^{2})(y+x^{2})=0$, for example.) Use
Corollary~\ref{computing delta} to show that the $\delta$ invariant
of $C_{0}$ at $p$ is 6.
\tohint{15.6}
\end{exercise}

\begin{exercise}
Find the
adjoint ideals
\index{adjoint ideal}%
of these plane curve singularities:
\begin{enumerate}
\item a
\emph{triple tacnode},
also known in classical language as
a \emph{triple point with an}
\emph{infinitely near triple point}:
\index{infinitely near point!double, triple}%
three smooth branches, pairwise simply
\index{tacnode!triple}%
\index{triple tacnode}%
tangent;
\item a triple point with an infinitely near double point: three smooth
branches, two of which are simply tangent, with the third transverse;
\item a
unibranch triple point,
such as the zero locus of $y^3-x^4$.
\index{unibranch}%
\index{triple point!types of}%
\tohin{15.7}
\end{enumerate}\label{tnih15.7}
\end{exercise}

Here is a  general description in case the individual branches of $C_0$
at $p$ are each smooth:

\begin{exercise}
Let $\nu : C \to C_0$ be the normalization of a plane curve $C_0$ and
\index{normalization}%
$p \in C_0$ a singular point. Denote the branches of $C_0$ at $p$ by
$B_1,\dots,B_k$, and let $r_i$ be the point in $B_i$ lying over $p$. If
the individual branches $B_i$ of $C_0$ at $p$ are each smooth, and we set
$$
m_i = \sum_{j \neq i} \mathrm{mult}_p(B_i,  B_j),
$$
where $\mathrm{mult}_{p}$ is defined as in Section~\ref{surface basics}, then the adjoint ideal of $C_0$ at $p$ is the ideal of functions $g$
such that $\ord_{r_i}` `\nu^*g \geq m_i$.
\tohint{15.8}
\end{exercise}


