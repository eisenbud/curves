%header and footer for separate chapter files

\ifx\whole\undefined
\documentclass[12pt, leqno]{book}
\usepackage{graphicx}
\input style-for-curves.sty
\usepackage{hyperref}
\usepackage{showkeys} %This shows the labels.
%\usepackage{SLAG,msribib,local}
%\usepackage{amsmath,amscd,amsthm,amssymb,amsxtra,latexsym,epsfig,epic,graphics}
%\usepackage[matrix,arrow,curve]{xy}
%\usepackage{graphicx}
%\usepackage{diagrams}
%
%%\usepackage{amsrefs}
%%%%%%%%%%%%%%%%%%%%%%%%%%%%%%%%%%%%%%%%%%
%%\textwidth16cm
%%\textheight20cm
%%\topmargin-2cm
%\oddsidemargin.8cm
%\evensidemargin1cm
%
%%%%%%Definitions
%\input preamble.tex
%\input style-for-curves.sty
%\def\TU{{\bf U}}
%\def\AA{{\mathbb A}}
%\def\BB{{\mathbb B}}
%\def\CC{{\mathbb C}}
%\def\QQ{{\mathbb Q}}
%\def\RR{{\mathbb R}}
%\def\facet{{\bf facet}}
%\def\image{{\rm image}}
%\def\cE{{\cal E}}
%\def\cF{{\cal F}}
%\def\cG{{\cal G}}
%\def\cH{{\cal H}}
%\def\cHom{{{\cal H}om}}
%\def\h{{\rm h}}
% \def\bs{{Boij-S\"oderberg{} }}
%
%\makeatletter
%\def\Ddots{\mathinner{\mkern1mu\raise\p@
%\vbox{\kern7\p@\hbox{.}}\mkern2mu
%\raise4\p@\hbox{.}\mkern2mu\raise7\p@\hbox{.}\mkern1mu}}
%\makeatother

%%
%\pagestyle{myheadings}

%\input style-for-curves.tex
%\documentclass{cambridge7A}
%\usepackage{hatcher_revised} 
%\usepackage{3264}
   
\errorcontextlines=1000
%\usepackage{makeidx}
\let\see\relax
\usepackage{makeidx}
\makeindex
% \index{word} in the doc; \index{variety!algebraic} gives variety, algebraic
% PUT a % after each \index{***}

\overfullrule=5pt
\catcode`\@\active
\def@{\mskip1.5mu} %produce a small space in math with an @

\title{Personalities of Curves}
\author{\copyright David Eisenbud and Joe Harris}
%%\includeonly{%
%0-intro,01-ChowRingDogma,02-FirstExamples,03-Grassmannians,04-GeneralGrassmannians
%,05-VectorBundlesAndChernClasses,06-LinesOnHypersurfaces,07-SingularElementsOfLinearSeries,
%08-ParameterSpaces,
%bib
%}

\date{\today}
%%\date{}
%\title{Curves}
%%{\normalsize ***Preliminary Version***}} 
%\author{David Eisenbud and Joe Harris }
%
%\begin{document}

\begin{document}
\maketitle

\pagenumbering{roman}
\setcounter{page}{5}
%\begin{5}
%\end{5}
\pagenumbering{arabic}
\tableofcontents
\fi


\chapter{Curves of genus 0 and 1}\label{genus 0 and 1 chapter}

The subject of algebraic curves abounds with examples amenable to explicit construction and analysis. In this chapter, we will illustrate  the basic geometry and embeddings of curves of genus 0 and 1; as we introduce new techniques, we'll have analogous chapters dealing with curves of genus 2 through 6. Our knowledge of the geometry of curves becomes increasingly less complete as the genus increases, and 6, as we shall see, is a natural turning point. 
%At the end of this chapter, the reader will be able to say with some confidence that he or she has seen every curve of genus $g \leq 6$, and understands its geometry.
\section{pre-requisites and conventions}


Basic results used in this section: B\'ezout, Riemann-Roch, Lasker (aka AF+BG), Clifford, Adjunction.
To write: an appendix on cohomology covering RR, exact sequences.
Section on Families: define family, define Hilbert Scheme and Chow variety;  but say we're not going to treat them formally very much. Flatness referred to ``Geom Schemes''; our families are smooth.



Would it be more confusing or less to use the same letter for a polynomial vanishing on $C$ and the surface it defines?

\




\section{Curves of genus 0} 

rational curves as projections of rational normal curves. Rational quartic in $\PP^3$ as curve of type 1,3 on quadric do dimension count. Branch points can be chosen. $g^3_4$ is sum of $g^1_1$ and a $g^1_3$. Cheerful fact: Set theor comp int problem.. Maximal rank for forms of degree d. Open questions: Hilbert functions? generators of the ideal? mention "secant conjecture"?

\subsection{Rational normal curves}

The first thing to observe about curves of genus 0 is that \emph{there is only one}: any curve $C$ of genus 0 is isomorphic to $\PP^1$. This follows immediately from the statement (\ref{degree 2g+1 embedding}) that any line bundle of degree $2g+1$ or greater on a curve of genus $g$ is very ample: if $p \in C$ is any point, by Riemann-Roch we have $h^0(\cO_C(p)) = 2$, and so the linear series $|\cO_C(p)|$ gives an ``embedding" of $C$ in $\PP^1$. Note that this works only because we are working over an algebraically closed field $K$: without that assumption, $C$ may not have any $K$-rational points at all, and indeed the classification of curves of genus 0 over non-algebraically closed fields is a subject that goes back to Gauss.

Another key fact about $\PP^1$ is that \emph{there is only one line bundle of degree $d$ on $\PP^1$} for any $d$; this is the bundle $\cO_{\PP^1}(d)$. This follows from direct observation: if $D = z_1+z_2+\dots+z_d$ and $E = w_1+\dots+w_d$ are two divisors of degree $d$, the rational function
$$
f(z) \; = \; \frac{(z-z_1)(z-z_2)\cdots(z-z_d)}{(z-w_1)(z-w_2)\cdots(z-w_d)}
$$
gives a rational equivalence between $D$ and $E$. Note that $h^0(\cO_{\PP^1}(d)) = d+1$; this follows from Riemann-Roch, or we can see it directly either by writing out explicitly the vector space of rational functions with poles along a divisor $D = z_1+z_2+\dots+z_d$:
$$
L(D) \; = \; \left\{ \frac{g(z)}{(z-z_1)(z-z_2)\cdots(z-z_d)} \mid \deg(g) \leq d \right\}
$$

The map $\phi_d : \PP^1 \to \PP^d$ associated to the complete linear series $|\cO_{\PP^1}(d)|$ is called the $d$th \emph{Veronese map} on $\PP^1$; the 
image $C \subset \PP^d$ of $\PP^1$ under this map  is called the \emph{rational normal curve} of degree $d$. In case $d=2$, this is simply a plane conic (as we'll see, it is the zero locus of a single quadratic polynomial on $\PP^2$); in case $d=3$ it's called the \emph{twisted cubic}.

It's easy to write down the equations that define a rational normal curve. First, in coordinates, we can realize the map $\phi_d$ as
$$
\phi_d : z \mapsto [1, z, z^2,\dots,z^d],
$$
from which we see that $C$ lies in the zero locus of the homogeneous quadratic polynomial $Q_{ijkl}(W) = W_iW_j - W_kW_l$ for every $i+j=k+l$. As a convenient way to package these, we can realize their span as the span of the $2\times 2$ minors of the matrix
$$
M \; = \; \begin{pmatrix}
W_0 & W_1 & \dots & W_{d-1} \\
W_1 & W_2 & \dots & W_d
\end{pmatrix}.
$$

In fact, these are all the quadratic polynomials on $\PP^d$ vanishing on $C$. To see this, consider the restriction map
$$
H^0(\cO_{\PP^d}(2)) \; \to \; H^0(\cO_{C}(2)) = H^0(\cO_{\PP^1}(2d)).
$$
This map is surjective (every monomial of degree $2d$ on $\PP^1$ is a product of two monomials of degree $d$); comparing dimensions, we see that the dimension of the kernel---that is, the space of quadratic polynomials on $\PP^d$ vanishing on $C$---has dimension
$$
\binom{d+2}{2} - (2d+1) \; = \; \binom{d}{2},
$$
which is exactly the dimension of the span of the minors of $M$. 

We can also see readily that $C$ \emph{is exactly the zero locus of these polynomials}. Explicitly, suppose that $p = [W_0,\dots,W_d] \in \PP^d$ is any point, and all the polynomials $Q_{ijkl}$ above vanish at $p$. If $W_0 = 0$, then from the vanishing of $Q_{1322}$ we see that $W_1 = 0$, and similarly we have $W_2 = \dots = W_{d-1}=0$; this the point $p = [0,\dots,0,1]$, which is a point on the rational normal curve. On the other hand, if $W_0 \neq 0$, set $\lambda = W_1/W_0$; we see in turn that $W_2/W_1 = \dots = W_d/W_{d-1} = \lambda$; thus $p = [1, \lambda, \dots,\lambda^d]$, again a point of the rational normal curve.

It's also true, though less easily shown, that in fact the polynomials $Q_{ijkl}$  generate the homogeneous ideal of the curve $C \subset \PP^d$.

There are two other properties of rational normal curves we should mention. The first is basically just terminology: in general, we say that a smooth curve $C \subset \PP^d$ is \emph{projectively normal} if the restriction map
$$
H^0(\cO_{\PP^d}(m)) \; \to \; H^0(\cO_{C}(m)) 
$$
is surjective for every $m$. By the same logic as above (every monomial of degree $md$ on $\PP^1$ is a product of $m$ monomials of degree $d$), we see that the rational normal curve is projectively normal. 

Projective normality is a significant geometric property of a curve. We'll see it in many settings, in particular the discussion of \emph{liaison} in Chapter~\ref{**}.

The other property of rational normal curves $C \subset \PP^r$ we should state is that they are \emph{projectively homogeneous}: the subgroup $G$ of the automorphism group $PGL_{r+1}$ of automorphisms of $\PP^r$ that carries $C$ to itself acts transitively on $C$. In fact, every automorphism $\sigma$ of $\PP^1$ is induced by an automorphism $\phi$ of $\PP^r$: since $\cO_{\PP^1}(r)$ is the unique line bundle of degree $r$ on $\PP^1$, we have $\sigma^*\cO_{\PP^1}(r) = \cO_{\PP^1}(r)$; thus $\sigma$ induces an action on $H^0(\cO_{\PP^1}(r))^*$, which gives the automorphism $\phi$ of $\PP^r = \PP H^0(\cO_{\PP^1}(r))^*$

In fact, the rational normal curve $C \subset \PP^r$ can also be characterized among irreducible, nondegenerate curves as the unique projectively homogeneous curve in $\PP^r$.  This is also not too hard to prove, but it requires the fact that a curve of genus $g \geq 2$ has only finitely many automorphisms, which we'll establish in Chapter~\ref{InflectionsChapter}.


\subsubsection{Other characterizations of rational normal curves}

Rational normal curves are special in many senses (except, of course, the literal one), several of which characterize the rational normal curve among all irreducible, nondegenerate curves in projective space. Before going on, we'll mention some of these here.


The first, and most significant, is that  rational normal curve of degree $d$ can also be characterized as the unique irreducible, nondegenerate curve of \emph{minimal degree} $d$ in $\PP^d$. To see this, suppose that $C \subset \PP^d$ is any irreducible, nondegenerate curve. If $p_1,p_2,\dots,p_{d}$ are any $d$ points of $C$, they lie in a hyperplane which meets $C$ in at least those points, whence $\deg(C) \geq d$; and if we have equality then the projection $\pi_\Lambda : C \to \PP^1$ from the plane $\Lambda = \overline{p_1,p_2,\dots,p_{d-1}}$ has degree 1, from which we see that $C \cong \PP^1$.

In fact, the same logic yields another highly significant fact about rational normal curves: \emph{any $m \leq d+1$ points on a rational normal curve $C \subset \PP^d$ are linearly independent}. This follows readily: in case $m=d+1$, Bezout tells us that the points cannot lie in a hyperplane. Alternatively, if the points are the images of points $\lambda_1,\dots,\lambda_{d+1} \in \PP^1$, this is tantamount to the nonvanishing of the Vandermonde determinant
$$
\begin{vmatrix}
1 & \lambda_1 & \lambda_1^2 & \dots & \lambda_1^d \\
1 & \lambda_2 & \lambda_2^2 & \dots & \lambda_2^d \\
\vdots & & & & \vdots \\
1 & \lambda_{d+1} & \lambda_{d+1}^2 & \dots & \lambda_{d+1}^d \\
\end{vmatrix}.
$$
The case $m < d+1$ follows. More generally, by the same argument \emph{any zero-dimensional subscheme $\Gamma \subset C$ of degree $m \leq d+1$ spans an $(m-1)$-plane in $\PP^d$}. Indeed, the rational normal curve is the unique curve with this property, as we'll see in Chapter~\ref{InflectionsChapter}.

(One sidelight: having established that the smallest possible degree of an irreducible, nondegenerate curve in $\PP^d$ is $d$, it's natural to ask what is the minimal degree of an irreducible, nondegenerate variety $X \subset \PP^d$ of dimension $k$. We'll see the answer ($\deg(X) \geq d-k+1$) and describe varieties of minimal degree in Section~\ref{**}.)

Finally, one last special property of rational curves $C \subset \PP^r$: they lie on more hypersurfaces than any other irreducible, nondegenerate curve in $\PP^r$. For example, we've seen that there is a  $\binom{d}{2}$-dimensional vector space of quadrics vanishing on the rational normal curve $C \subset \PP^r$; that is,
$$
h^0(\cI_{C/\PP^r}(2)) = \binom{d}{2}
$$
We now have the

\begin{proposition}
If $C \subset \PP^r$ is any irreducible, nondegenerate curve, then
$$
h^0(\cI_{C/\PP^r}(2)) \leq \binom{d}{2};
$$
and if equality holds then $C$ is a rational normal curve
\end{proposition}

\begin{proof}
The key to the analysis here is to look at the restriction of the quadrics containing $C$ to a general hyperplane $H \cong \PP^{r-1} \subset \PP^r$. If we let $\Gamma = H \cap C$, then on the sheaf level we have an exact sequence
$$
0 \to \cI_{C/\PP^r}(1) \to \cI_{C/\PP^r}(2) \to \cI_{\Gamma/\PP^{r-1}}(2) \to 0.
$$ 
Now, the nondegeneracy hypothesis says that $h^0(\cI_{C/\PP^r}(1)) = 0$, and since the hyperplane section $\Gamma$ of $C$ must contain at least $r$ linearly independent points, which impose independent conditions on quadrics, we have
$$
h^0(\cI_{\Gamma/\PP^{r-1}}(2)) \leq h^0(\cO_{\PP^{r-1}}(2)) - r = \binom{r}{2},
$$
establishing the desired inequality.
\end{proof}

\begin{exercise}
Establish the analogous statement for hypersurfaces of any degree $d$; that is, no irreducible, nondegenerate curve in $\PP^r$ lies on more hypersurfaces of degree $d$ than the rational normal curve.
\end{exercise}

\begin{exercise}
Prove directly the special case $r=3$: that the twisted cubic is the unique irreducible, nondegenerate space curve lying on three quadrics.
\end{exercise}


\subsection{Other rational curves}

What about other rational curves in projective space? Since any linear series $\cD$ of degree $d$ on $\PP^1$ is a subseries of the complete series $|\cO_{\PP^1}(d)|$, we see that \emph{any rational curve $C \subset \PP^r$ of degree $d$ is a projection of a rational normal curve in $\PP^d$}. Slightly more generally, any map $\phi : \PP^1 \to \PP^r$ of degree $d$ is given as
$$
z \; \mapsto \; [f_0(z), \dots, f_r(z)]
$$
for some $(r+1)$-tuple of polynomials $f_\alpha$ of degree $d$ on $\PP^1$, which is to say it's the composition of the embedding $\phi_d : \PP^1 \to \PP^d$ of $\PP^1$ as a rational normal curve with a linear projection $\pi : \PP^d \to \PP^r$. 

Given how easy it is to describe rational curves in projective space in this way, it is in some ways surprising how many open questions there are about such curves. We'll talk more about some of these questions in the following section; for now, we'll try to give a sense of what we can say about such curves by considering one of the first and simplest cases: smooth rational curves of degree $4$ in $\PP^3$.

So: let $C \subset \PP^3$ be a smooth, nondegenerate curve of degree 4 and genus 0 in $\PP^3$. To describe the geometry of $C$, the first thing to determine is what surfaces it lies on---that is, what degree polynomials on $\PP^3$ vanish on $C$. To start with, we can ask: does $C$ lie on a quadric surface? To answer this, we consider again the restriction map
$$
H^0(\cO_{\PP^3}(2)) \; \to \; H^0(\cO_{C}(2)) = H^0(\cO_{\PP^1}(8)).
$$
Here the vector space on the left---homogeneous quadratic polynomials on $\PP^3$---has dimension 10, while the one on the right, either by Riemann-Roch or by direct examination, has dimension 9. We conclude that \emph{the curve $C$ must lie on at least one quadric surface $Q \subset \PP^3$}.

Since $C$ is irreducible and nondegenerate, it can't lie on a union of planes, so the quadric $Q$ must either be smooth or a cone over a conic curve. We'll see in a moment that the latter case can't occur, so let's assume for now that $Q$ is smooth. 

The natural follow-up question is, what is the class of $C$ in the Picard group of $Q$? We know that $Q \cong \PP^1 \times \PP^1$, with the fibers of the two projections appearing as lines of the two rulings of $Q$. Lines $L$ and $M$ of the two rulings generate the Picard group, so that we must have $C \sim aL + bM$ for some $a, b$ (in other words, in terms of the isomorphism $Q \cong \PP^1 \times \PP^1$, $C$ is the zero locus of a bihomogeneous polynomial of bidegree $(a,b)$), and we ask what $a$ and $b$ are. The choices are limited: since $C$ is a quartic curve, we must have $a+b = 4$. Adjunction tells us which must be the case: the genus formula for curves on $Q$ tells us that the genus of a smooth curve of class $(a,b)$ on $Q$ has genus $(a-1)(b-1)$, whence the class of our curve $C$ must be $(1,3)$ (for a suitable ordering of the two rulings).

It follows in particular that \emph{$Q$ is the unique quadric containing $C$}. One way to see this is that since $C$ has class $(1,3)$ it meets the lines of the first ruling three times; if $Q'$ is any quadric containing $C$, then, it must contain all these lines and hence must equal $Q$. Alternatively, we may consider the exact sequence
$$
0 \to \cI_{C/Q}(2) \to \cO_Q(2)  \to \cO_C(2) \to 0.
$$
If $C$ has class $L+3M$, we have $\cI_{C/Q}(2) = \cO_{Q}(L-M)$. Since this bundle has negative degree of every line of the first ruling, it has no sections; hence the restriction map $H^0(\cO_Q(2))  \to H^0(\cO_C(2))$ is injective and so there are no  quadrics in $\PP^3$ containing $C$ other than $Q$.

(It is interesting to compare the two arguments above: they are exactly the same argument, expressed first in 19th century language and then in the language of the 20th century.)

We can also describe the rest of the ideal of $C$ similarly. For example, to find the cubic polynomials vanishing on $C$ we consider the restriction map
$$
H^0(\cO_{\PP^3}(3)) \; \to \; H^0(\cO_{C}(3)) = H^0(\cO_{\PP^1}(12)).
$$
The dimensions of these two vector spaces being 20 and 13 respectively, we see that $C$ must lie on at least 7 cubics; four of these are simply products of $Q$ with linear forms, and so we see that $C$ must lie on at least three cubics modulo those containing $Q$. Indeed, these are easy to spot: if $L$ and $L'$ are any two lines of the first ruling, the divisor $C + L + L'$ has class $(3,3)$ on $Q$ and hence is the intersection of $Q$ with a cubic surface. As $L+L'$ varies in a two-dimensional linear series, we get three cubics containing $C$ modulo those containing $Q$. Conversely, any cubic containing $C$ (but not containing $Q$) will intersect $Q$ in the union of $C$ with a curve of type $(2,0)$ on $Q$, which is to say the sum of two lines of the first ruling, so these are all the cubics containing $C$.

Finally, we have to show that the quadric containing the curve $C$ cannot be a cone over a conic plane curve. The key question here is whether or not $C$ contains the vertex $p$ of the cone: if not, the same adjunction-based calculation shows that $C$ must have genus 1; while a parity argument (how many times does $C$ meet a line of the ruling of $Q$?) shows that if a curve $C \subset Q$ of even degree contains $p$ it must be singular there.


\begin{exercise}
Find all possible Hilbert functions of smooth rational quintic  curves $C \subset \PP^3$. (There are only two, depending on whether or not $C$ lies on a quadric, so this isn't so bad.)
\end{exercise}

\begin{exercise}
Every $g^3_4$ on $\PP^1$ is uniquely expressible as a sum of the $g_1^1$ and a $g^1_3$
\end{exercise}

\begin{exercise}
There is a 1-parameter family of rational quartic curves in $\PP^3$ up to projective equivalence. (Finding the invariants is a nice problem, which we should talk about.)
\end{exercise}

\subsection{Further problems (open and otherwise) concerning rational curves in projective space}

To begin with, we should remark that this one example of a non-linearly normal rational curve in projective space is misleading in that we can give such a complete description. For general $d$ and $r$, we have no idea what may be the Hilbert function of a rational curve of degree $d$ in $\PP^r$. Indeed, even in the limited case of $r=3$, our knowledge gives out around $d=9$.

We can, however, say some things about a \emph{general} rational curve $C \subset \PP^r$ of given degree $d$. To make sense of this, let $C_0 \subset \PP^d$ be a rational normal curve of degree $d$. As we've said, any rational curve of degree $d$ in $\PP^r$ is the projection $\pi_\Lambda(C_0)$ of $C_0$ from a $(d-r-1)$-plane $\Lambda \subset \PP^d$. If we let $\GG = \GG(d-r-1, d)$ be the Grassmannian of $(d-r-1)$-planes in $\PP^d$, and we let $U \subset \GG$ be the open subset of planes disjoint from the secant variety of $C_0$, we have a family of rational curves in $\PP^r$ parametrized by $U$ and including every smooth rational curve $C \subset \PP^r$ of degree $d$. Thus in particular we can talk about a \emph{general rational curve} of degree $d$ and genus $g$ in $\PP^r$, and ask about its geometry.

This is, in fact, still largely uncharted waters. Consider, for example, one of the most basic questions we might ask: what is the Hilbert function of a general rational curve $C \subset \PP^r$ of degree $d$? As in the example, this is tantamount to looking at the restriction map
$$
\rho_m : H^0(\cO_{\PP^r}(m) \to H^0(\cO_C(m)) = H^0(\cO_{\PP^1}(md)).
$$
Equivalently, we're asking: if $V$ is a general  $(r+1)$-dimensional vector space of homogeneous polynomials of degree $d$, what is the dimension of the space of polynomials spanned by $m$-fold products of polynomials in $V$? We might naively guess that the answer is, ``as large as possible," meaning that the rank of $\rho_m$ is $\binom{m+r}{r}$ when that number is less than $md+1$, and equal to $md+1$ when it's greater---in other words, the map $\rho_m$ is either injective or surjective for each $m$.

This, it turns out, is true, but it is only relatively recently known: the case $g=0$, as here, was done by Ballico in **** (??), and the analogous statement for curves of arbitrary genus, which we'll describe in Chapter~\ref{Brill-Noether}, was proved in 2019 by Eric Larson.

\subsubsection{The secant plane conjecture}

Another question we may ask about a curve in projective space is what secant planes it has. To frame the question, let's start with some language: given a smooth curve $C \subset \PP^r$, we say that an $e$-secant $s$-plane to $C$ is an $s$-plane $\Lambda \cong \PP^s \subset \PP^r$ such that the intersection $\Lambda \cap C$ has degree $\geq e$; if we exclude degenerate cases (for example, where $\Lambda \cap C$ fails to span $\Lambda$), this is the same as saying we have a divisor $D \subset C$ of degree $e$ whose span is contained in an $s$-plane.

Do we expect a curve $C \subset \PP^r$ to have any $e$-secant $s$-planes? To answer this, let's get old school and do a 19th century style dimension count. We start by observing that the set of $s$-planes in $\PP^r$ is parametrized by the Grassmannian $\GG = \GG(s,r)$, which had dimension $(s+1)(r-s)$. Inside $\GG$, the locus of planes that meet $C$ has codimension $r-s-1$ (the locus of planes containing a given point of $C$ has codimension $r-s$); so our very naive expectation might be that the locus of $e$-secant $s$-planes would have codimension $e(r-s-1)$ in $\GG$. If this were the case, we would expect a curve $C \subset \PP^r$ to have $e$-secant $s$-planes when 
$$
e \; \leq \; (s+1)\frac{r-s}{r-s-1}.
$$
Is this true of a general rational curve? The fact is, for most $e$, $r$ and $s$, we don't know!

\section{Curves of genus 1}

%Wonderful subject; refer to somewhere else. Double cover of $\PP^1$, leading to $y^2 - f(x)$. Plane cubic, quartic in $\PP^3$. Cheerful fact:  elliptic quintic is Pfaffian. Cheerful fact: any $g^5_6$ is the product of two $g^2_3$s. Get a $3\times 3$ matrix of linear forms. The image of the matrix and its transpose are $g^2_3$'s. Prove this by going to the Segre embedding $\PP^2\times \PP^2 \subset\PP^8$.

The subject of curves of genus 1 has played a huge role in the development of the theory of algebraic curves, and of algebraic geometry in general, for a very simple reason. We said at the outset of our discussion of curves of genus 0 that two key facts held: that all curves of genus 0 are isomorphic to $\PP^1$; and that on a given curve of genus 0 all divisors of a given degree are linearly equivalent. Neither of the analogous statements holds true for curves of genus 1; and the ways in which 19th century geometers dealt with this fact ultimately shaped all of algebraic geometry.

Specifically, classical geometers observed that there was a one-parameter family of curves of genus 1 up to isomorphism, and that on a given curve of genus 1 there was a one-dimensional family of divisors up to linear equivalence. These were, in many ways, the earliest examples of \emph{moduli spaces}, and they were ultimately generalized to the moduli space $M_g$ of curves of genus $g$, and the Picard variety $\Pic^d(C)$ parametrizing divisors of degree $d$ on a given curve $C$ up to linear equivalence.

What are we going to do in this chapter? The first thing to say is that we cannot begin to describe everything that has been said or done with curves of genus 1,
 a.k.a. elliptic curves.\footnote{Technically, an elliptic curve is a smooth curve of genus 1 with a distinguished point, called the \emph{origin}.}  They appeared, in the second half of the 19th century, as key objects in the developing subjects of geometry, number theory and complex analysis, and the literature is correspondingly rich---the total number of journal and book pages devoted to the topic probably exceeds a million. 
 
 Here we'll focus on the geometric side, and try to describe maps of genus 1 curves to projective space. As a sort of through-line for our discussion, we'll try to indicate in each case how the given projective model of a curve of genus 1 gives rise to the expectation that there is a one-parameter family of curves of genus 1 up to isomorphism. As for the fact that there are many different line bundles of a given degree $d$ on a curve $E$ of genus 1, we will adopt for now a workaround: for any $d$, \emph{the automorphism group of $E$ acts transitively on them}. In other words, if $\phi, \phi' : E \to \PP^r$ are two maps given by complete linear series $|L|$ and $|L'|$ of degree $d$ on $E$, then there exists  automorphisms $\alpha : \PP^r \to \PP^r$ and $\beta : E \to E$ such that $\phi' \circ \beta= \alpha \circ \phi$. In particular, if $\phi$ and $\phi'$ are embeddings---as will be the case when $d \geq 3$---then their images are projectively equivalent. As for the business of parametrizing line bundles on a given curve $C$, we will take that up in the next chapter, and see it applied in the case of curves of genus $g \geq 2$ in Chapter~\ref{genus 2 and 3 chapter}.

\subsection{Double covers of $\PP^1$}

Let $E$ be a smooth projective curve of genus 1. If $L$ is any line bundle of degree 1 on $E$, Riemann-Roch says that $h^0(L) = 1$, so if we're looking for nonconstant maps to projective space we have to go to degree 2 and higher.

To start with, suppose $L$ is a line bundle of degree 2 on $E$. By Riemann-Roch, $h^0(L) = 2$ and the linear series $|L|$ is base point free, so we get a map $\phi : E \to \PP^1$ of degree 2. By Riemann-Hurwitz, the map $\phi$ will have 4 branch points; by the remark above, these four points are determined, up to automorphisms of $\PP^1$ by the curve $E$, and are independent of the choice of $L$.
After composing with an automorphism of $\PP^1$ we can take these four points to be $0, 1, \infty$ and $\lambda$ for some $\lambda \neq 0, 1 \in \CC$. Since there is a unique double cover of $\PP^1$ with given branch divisor (see~\ref{**}) it follows that $E \cong E_\lambda$, where $E_\lambda$ is the curve given by the affine equation
$$
y^2 = x(x-1)(x-\lambda).
$$

When are two curves $E_\lambda$ and $E_{\lambda'}$ isomorphic? By what we've said, this will be the case if and only if there is an automorphism of $\PP^1$ carrying the points $\{0,1,\infty,\lambda\}$ to $\{0,1,\infty,\lambda'\}$, in any order. This will be the case if and only if $\lambda$ and $\lambda'$ belong to the same orbit under the action of the group $G \cong S_3$ of automorphisms of $\PP^1$ permuting the three points $0, 1$ and $\infty$; that is, if
$$
\lambda' \in \{\lambda, \; 1-\lambda, \; \frac{1}{\lambda},\;  \frac{1}{1-\lambda}, \; \frac{\lambda - 1}{\lambda}, \; \frac{\lambda}{\lambda - 1} \}.
$$
Now, the quotient of $\PP^1$ by the action of $G$ is again isomorphic to $\PP^1$ by Luroth's theorem, which means that the field of rational functions on $\PP^1$ invariant under $G$ is again a purely transcendental extension $K(j)$; explicitly, we can take
$$
j \; = \; 256\cdot \frac{\lambda^2 - \lambda + 1}{\lambda^2(\lambda - 1)^2}.
$$
(the factor of 256 is there for arithmetic reasons). In any case, we see explicitly that there is a unique smooth projective curve of genus 1 for each value of $j$; in particular, the family of all such curves is parametrized by a curve.

\subsection{Plane cubics}

Moving from degree 2 to degree 3, let $L$ be a line bundle of degree 3 on $E$. We see from Corollary~\ref{degree 2g+1 embedding} that the sections of $L$ give an embedding of $E$ as a smooth plane cubic curve; conversely, the genus formula tells us that a smooth plane cubic curve indeed has genus 1. 

We won't delve into the geometry of plane cubics, except to point out that once more we can use this representation to argue that the isomorphism classes of elliptic curves form a 1-dimensional family. To see this, observe that the space of homogeneous polynomials of degree 3 in three variables is 10-dimensional, and the space of plane cubic curves is correspondingly parametrized by  $\PP^9$; the locus of smooth curves is a Zariski open subset of this $\PP^9$. On the other hand, by what we've said, two plane cubics are isomorphic iff they are congruent under the group $PGL_3$ of automorphisms of $\PP^2$. Since the group $PGL_3$ has dimension 8, we would expect that the family of such curves up to isomorphism has dimension 1.

\subsection{Quartics in $\PP^3$} 

Onwards! Let $E$ again be a smooth projective curve of genus 1, and consider now the embedding of $E$ into $\PP^3$ given by the sections of a line bundle $L$ of degree 4. The first question we might ask is what polynomial equations in $\PP^3$ cut out the image, and as before we'll do this by looking at the restriction map
$$
\rho_2 \;  : \; H^0(\cO_{\PP^3}(2)) \; \to \; H^0(\cO_{E}(2)) = H^0(L^2).
$$
The space on the right---the space of homogeneous polynomials of degree 2 in four variables---has dimension 10, while by Riemann-Roch the space $H^0(L^2)$ has dimension 8. It follows that $E$ lies on at least two linearly independent quadrics $Q$ and $Q'$. Since $E$ does not lie in any plane, neither $Q$ nor $Q'$ can be reducible; thus by Bezout we see that
$$
E = Q \cap Q'
$$
is the complete intersection of two quadrics in $\PP^3$. Moreover, we also see from the Lasker-Noether ``AF+BG" theorem that the kernel of $\rho_2$ is exactly the span of $Q$ and $Q'$. Thus $E$ determines a point in the Grassmannian $G(2, H^0(\cO_{\PP^3}(2))) = G(2, 10)$ of pencils of quadrics; and by Bertini a Zariski open subset of that Grassmannian correspond to smooth quartic curves of genus 1. We can use this to once more calculate the dimension of the family of curves of genus 1: the Grassmannian $G(2,10)$ has dimension 16, while the group $PGL_4$ of automorphisms of $\PP^3$ has dimension 15, so we may conclude that the family of curves of genus 1 up to isomorphism has dimension 1.


\subsubsection{Projective normality II}
\fix{maybe this should be part of the homological algebra development much later}
Observe that last two cases (cubic and quartic genus 1 curves) are projectively normal; extend this to arbitrary smooth complete intersections.

Exercise: $C \subset Q \subset \PP^3$ of class $(a,b)$ is projectively normal iff $|a-b| \leq 1$.


\input footer.tex


