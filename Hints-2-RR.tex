%header and footer for separate chapter files

\ifx\whole\undefined
\documentclass[12pt, leqno]{book}
\usepackage{graphicx}
\input style-for-curves.sty
\usepackage{hyperref}
\usepackage{showkeys} %This shows the labels.
%\usepackage{SLAG,msribib,local}
%\usepackage{amsmath,amscd,amsthm,amssymb,amsxtra,latexsym,epsfig,epic,graphics}
%\usepackage[matrix,arrow,curve]{xy}
%\usepackage{graphicx}
%\usepackage{diagrams}
%
%%\usepackage{amsrefs}
%%%%%%%%%%%%%%%%%%%%%%%%%%%%%%%%%%%%%%%%%%
%%\textwidth16cm
%%\textheight20cm
%%\topmargin-2cm
%\oddsidemargin.8cm
%\evensidemargin1cm
%
%%%%%%Definitions
%\input preamble.tex
%\input style-for-curves.sty
%\def\TU{{\bf U}}
%\def\AA{{\mathbb A}}
%\def\BB{{\mathbb B}}
%\def\CC{{\mathbb C}}
%\def\QQ{{\mathbb Q}}
%\def\RR{{\mathbb R}}
%\def\facet{{\bf facet}}
%\def\image{{\rm image}}
%\def\cE{{\cal E}}
%\def\cF{{\cal F}}
%\def\cG{{\cal G}}
%\def\cH{{\cal H}}
%\def\cHom{{{\cal H}om}}
%\def\h{{\rm h}}
% \def\bs{{Boij-S\"oderberg{} }}
%
%\makeatletter
%\def\Ddots{\mathinner{\mkern1mu\raise\p@
%\vbox{\kern7\p@\hbox{.}}\mkern2mu
%\raise4\p@\hbox{.}\mkern2mu\raise7\p@\hbox{.}\mkern1mu}}
%\makeatother

%%
%\pagestyle{myheadings}

%\input style-for-curves.tex
%\documentclass{cambridge7A}
%\usepackage{hatcher_revised} 
%\usepackage{3264}
   
\errorcontextlines=1000
%\usepackage{makeidx}
\let\see\relax
\usepackage{makeidx}
\makeindex
% \index{word} in the doc; \index{variety!algebraic} gives variety, algebraic
% PUT a % after each \index{***}

\overfullrule=5pt
\catcode`\@\active
\def@{\mskip1.5mu} %produce a small space in math with an @

\title{Personalities of Curves}
\author{\copyright David Eisenbud and Joe Harris}
%%\includeonly{%
%0-intro,01-ChowRingDogma,02-FirstExamples,03-Grassmannians,04-GeneralGrassmannians
%,05-VectorBundlesAndChernClasses,06-LinesOnHypersurfaces,07-SingularElementsOfLinearSeries,
%08-ParameterSpaces,
%bib
%}

\date{\today}
%%\date{}
%\title{Curves}
%%{\normalsize ***Preliminary Version***}} 
%\author{David Eisenbud and Joe Harris }
%
%\begin{document}

\begin{document}
\maketitle

\pagenumbering{roman}
\setcounter{page}{5}
%\begin{5}
%\end{5}
\pagenumbering{arabic}
\tableofcontents
\fi


\chapter{Riemann-Roch hints}\label{Inflections hints}

 \begin{exercise}\label{characterization of degree}
\begin{enumerate}
\item The degree of a zero-dimensional subscheme $X\subset \PP^r$ is by definition the value of the Hilbert polynomial of $X$, which is a constant. Show that
this is the sum of the lengths of the components of $X$. (Hint: use Serre's vanishing theorem.)

\item The degree of a projective subscheme $X\subset \PP^r$ of dimension $n$ is defined to be the degree of the $0$-dimensional scheme
that is the intersection of $X$ with a general plane of degree $r-n$. Prove that this is the leading coefficient of the Hilbert polynomial, multiplied
by $(dim X)!$.

\item If $C\subset \PP^r$ is a smooth curve of genus $g$ and degree $d$, show that the Hilbert polynomial of $C$ is $H_C(m) = dm-g+1$. Hint: Use Riemann-Roch.

\end{enumerate}
\end{exercise}

In the following series of exercises, we are going to be working with smooth projective curves associated to a given affine curve $C^\circ := V(f(x,y)) \subset \AA^2$; this is the unique smooth projective curve containing the normalization of $C^\circ$ as a Zariski dense open subset.

\begin{exercise}
Let $C$ be the smooth projective curve associated to the affine plane curve $y^3 +x^3 = 1$, and let $\pi : C \to \PP^1$ be the map given by the rational function $x$.
\begin{enumerate}
\item Find the branch points and ramification points of $\pi$, and deduce that the genus of $C$ is 1.
\item For any two points $p, q \in C$ find the complete linear series $|p+q|$.
\item Find the (unique) map $\eta : C \to \PP^1$ of degree 2 such that $\eta((1,0)) = \eta((0,1))$, and determine the ramification points of $\eta$.
\item Show that $C$ is isomorphic to the smooth projective curve associated to the affine plane curve $y^2 +x^3 = 1$.
\end{enumerate}
\end{exercise}

For the next three exercises, let $C^\circ$ be the affine plane curve given as the zero locus of $y^2 - x^6 +1$, and let $C$ be the corresponding smooth projective curve. Note that the map $C^\circ \to \AA^1$ given by the projection $(x,y) \mapsto x$ extends to a map $\pi : C \to \PP^1$, expressing $C$ as a 2-sheeted cover of $\PP^1$ branched over the points $1, \zeta, \dots, \zeta^5$, where $\zeta$ is any primitive 6th root of unity. In addition, let $p$ and $q \in C$ be the two points lying over the point $\infty \in \PP^1$.

\begin{exercise}\label{hyperelliptic curve 1}
Let $C^\circ$ be the affine plane curve given as the zero locus of $y^2 - x^6 +1$, and let $C$ be the corresponding smooth projective curve. 

Show  that the map $C^\circ \to \AA^1$ given by the projection $(x,y) \mapsto x$ extends to a map $\pi : C \to \PP^1$, expressing $C$ as a 2-sheeted cover of $\PP^1$ branched over the points $1, \zeta, \dots, \zeta^5$, where $\zeta$ is any primitive 6th root of unity. Show that there are two distinct points $p$ and $q \in C$  lying over the point $\infty \in \PP^1$,
so that $C$ is unramified over $\infty$.

What is the genus of $C$?

Hint: The projection $(x,y) \mapsto x$ is the 2-1 map to $\PP^1$. As $x\to \infty$, the equation looks like $y^2-x^6$, ie $y= \pm x^3$ -- this gives the 2 points of the normalization over $\infty\in \PP^1$. Thus the map $C\to \PP^1: (x,y) \mapsto x$
is unramified at infinity, and has just 6 branch points, so Hurwitz' theorem gives the genus.

\end{exercise}

\begin{exercise} With $C$ as in Exercise~\ref{hyperelliptic curve 1}:
\begin{enumerate}

\item Let $r_\alpha$ be the branch point over $\zeta^\alpha$. Show that
$$
p+q \sim 2r_\alpha \quad \text{and} \quad \sum_{\alpha = 0}^5 r_\alpha \sim 3p+3q.
$$

\item Find the vector space $H^0(\cO_C(D))$ where $D = r_0 + r_2 + r_4$, and find the (unique) divisor $E$ on $C$ such that $E + r_1 \sim r_0 + r_2 + r_4$.

Hint: The rational function $x-\zeta^i$ has a double zero at $r_i$. Locally at the point $r_i$ we have $y^2 = (x-r_i)u$,
with $u$ a local unit, so $y$ has a simple zero at each of these points. The Riemann-Roch theorem shows that $h^0(\cO_C(D) = 2$. Since $y/(x^3-1)$ has simple poles at the points $r_0,r_2,r_4,$ we see that $H^0(\cO_C(D)) = \langle1, y/(x^3-1)\rangle.$ 

Since $y/(x^3-1)$ has simple poles at $r_0,r_2,r_4$ and simple zeros at $r_1,r_3,r_5$, we see that $E = r_3+r_5$.
\end{enumerate}

\end{exercise}

\begin{exercise}
With $C$ as in Exercise~\ref{hyperelliptic curve 1}:
Let $D$ be the divisor $D = p + q + r_0 + r_3$
\begin{enumerate}
\item Find the vector space $H^0(\cO_C(D))$. 

Hint: $x$ has simple zeros at $p,q$ and $y$ has triple zeros there ($y^2/x^6 \to 1$ as $x\to \infty$.)
Also, $y$ has simple zeros at the $r_i$ while $x$ has double zeros there. % Answer: $\langle 1,\ x,\ y/(x^2-1)\rangle$
\item Describe the map $\phi_{|D|} : C \to \PP^2$. 
\item Find the equation of the image curve $\phi_{|D|}(C) \subset \PP^2$, and describe its singularities.
%\fix{in $\CC[x,y]/(y^2-x^6+1)$ the elements$ a = x, b = y/(x^2-1)$ satisfy $b^2(a^2-1)-(a^6-1)/(a^2-1),
%or 
%$$b^2(a^2-1) = a-\zeta)(a-\zeta^2)(a-\zeta^4)(a-zeta^5)$$

Hint: 
Since $\h^0(\cO_C(p+q) = 2$, the divisor $r_0+r_3$ imposes only 1 condition, so $\phi_{|D|}$ maps $r_0,r_3$ to the same
point $q$
The image is a quartic, and thus has arithmetic genus 3 by the adjunction formula, so $q$ must be
an ordinary double point.
\end{enumerate}
\end{exercise}


\begin{exercise}
Let $C$ be the smooth projective curve associated to the affine curve 
$y^3 = x^5 -1$. 
Note that the map $\pi : C \to \PP^1$ given by the function $x$ expresses $C$ as a cyclic, 3-sheeted cover of $\PP^1$, branched over the 5th roots of unity and the point at $\infty$. By way of notation, if we take $\eta = e^{2\pi i/5}$ a primitive 5th root of unity, we'll denote by $r_\alpha$ the point $(\eta^\alpha, 0) \in C$ lying over $\eta^\alpha$, and by $p$ the point lying over $\infty \in \PP^1$.

\begin{enumerate}
\item Verify that there is indeed a unique point $p \in C$ lying over $\infty \in \PP^1$, and the map has ramification index 2 at $p$. 

Hint: On the closure  $Y^3Z^2-X^5+Z^5$ in $\PP^2$ the point at infinity is $x=z=0, y=1$. In local analytic coordinates this is
$z^2+z^5-x^5 = z'^2-x^5$, a cusp with parameterization $z'=t^5, x= t^2$. This shows that there is a unique point of $C$
over infinity, and since the map to $\PP^1$ is a triple cover it must have ramification index 2. 

\item Show that the genus of $C$ is 4.

\item Establish the linear equivalences
$$
3p \sim 3r_\alpha   
\quad \text{and} 
\quad r_1+ \dots + r_5 \sim 5p. 
$$

Hint; $x-\eta^\alpha$ and $y$.

\item Find a basis for the space $H^0(K_C)$ of regular differentials on $C$.

Hint: Since
$dx$ has a double pole at $\infty \in \PP^1$, it has a 6-fold pole at $p$ and defines the pull-back of the canonical
class on $\PP^1$. The function $y$ vanishes simply
at each $r_i$, while $dx$ vanishes to order 2 there. Thus $dx, dx/y, dx/y^2$ are all holomorphic sections of $\omega_C$.
Also, $1/y$ has a 5-fold 0 at $p$, so $x/y$ is holomorphic there, and $x dx/y$. \fix{what's wrong with
$x^2dx/y^2, x^3 dx/y^2$?}. .

\item Show that $C$ is not hyperelliptic.
\item Describe the canonical map $\phi_K : C \to \PP^3$ and find the equations of the image.
The description of the sections of $\omega$  shows that the canonical image lies on the cone over a conic,
and projects 3 to one onto the conic.
\item Let $D$ be the divisor $D = r_1+\dots+r_5$. Show that $h^0(K_C(-D)) = 1$; deduce that $r(D) = 2$, and find a basis for $H^0(\cO_C(D))$
Hint:  Since $dx$ vanishes doubly at $r_\alpha$ the sheaf $h^0(K_C(-D)) has the sections $dx, dx/y$
\item If $E = 3p$, show that $r(E) = 1$; that $|E|$ is the unique $g^1_3$ on $C$ and that $2E \sim K$.
Hint: the sections are $1,x$. If $|D|$ is a  $g^1_3$ then $h^0(K_C-D) = 2$
\end{enumerate}
\end{exercise}


\begin{exercise}\label{pa example}
\begin{enumerate}
 \item Show that the arithmetic genus of the disjoint union of two lines is $-1$.
\item Let $L\subset Q\subset \PP^3$ be a line on a smooth quadric surface in $\PP^3$. Show that the 
divisor $3L$, regarded as a 1-dimensional scheme, has $p_a(3L) = -2$.
\item Compare these results with the result of simply applying the adjunction formula.
\end{enumerate}
Hints:
\begin{enumerate}
 \item Use the definition $p_a(X) = 1-\chi \sO_X$ .
 \item Filter $\sO_{3L} \twoheadrightarrow \sO_{2L} \twoheadrightarrow \sO_L$. The ideals
 defining the successive quotients are the normal bundle and its square. 
\end{enumerate}
\end{exercise}

\begin{exercise}\label{gonality exclusion}
Show that a curve of genus $g \geq 3$ cannot be simultaneously hyperelliptic and a three-sheeted cover of $\PP^1$.
\end{exercise}

\begin{exercise}\label{planar triple pt}
Show that normalization of the affine planar triple point $xy(x-y) = 0$ in $\AA^2$ is the disjoint union of three
affine lines, $\Spec(\CC[x] \times \CC[y]\times \CC[z]$. Compute the linear conditions on the values and derivatives of three polynomial functions f,g,h defined on
these three lines that they ``descend'' to give a well-defined function on the planar triple point.
\end{exercise}

\begin{exercise} Generalize the results on planar triple points, above, to the case of a plane curve with $n$ pairwise
transverse branches, called an \emph{ordinary multiple point}.
\end{exercise}

\begin{exercise}\label{delta=1 characterization}
Let $p \in C$ be a singular point of a reduced curve $C$. Show that if $\delta_p = 1$, then $p$ must be either a node or a cusp.
\end{exercise}

%footer for separate chapter files

\ifx\whole\undefined
%\makeatletter\def\@biblabel#1{#1]}\makeatother
\makeatletter \def\@biblabel#1{\ignorespaces} \makeatother
\bibliographystyle{msribib}
\bibliography{slag}

%%%% EXPLANATIONS:

% f and n
% some authors have all works collected at the end

\begingroup
%\catcode`\^\active
%if ^ is followed by 
% 1:  print f, gobble the following ^ and the next character
% 0:  print n, gobble the following ^
% any other letter: normal subscript
%\makeatletter
%\def^#1{\ifx1#1f\expandafter\@gobbletwo\else
%        \ifx0#1n\expandafter\expandafter\expandafter\@gobble
%        \else\sp{#1}\fi\fi}
%\makeatother
\let\moreadhoc\relax
\def\indexintro{%An author's cited works appear at the end of the
%author's entry; for conventions
%see the List of Citations on page~\pageref{loc}.  
%\smallbreak\noindent
%The letter `f' after a page number indicates a figure, `n' a footnote.
}
\printindex[gen]
\endgroup % end of \catcode
%requires makeindex
\end{document}
\else
\fi
