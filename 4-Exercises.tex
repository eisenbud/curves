\input header.tex

\chapter{Exercises for Chapter 3-Jacobians}


\begin{exercise}
 Let $G$ be a finite group acting on a quasi-projective scheme $X$. Show that there is a finite covering of $X$ by invariant open affine sets. (Hint: consider the sum of the $G$-translates of a very ample divisor.)
\end{exercise}


\begin{exercise}\label{free actions}
We say that a group $G$ acts freely on $X$ if $gx = gy$ only when $g =1$ or $x=y$. Show that
 if $G$ is a finite group acting freely on a smooth affine variety $X$ then the quotient $X/G$ is smooth.
\end{exercise}


\begin{exercise}
 \label{sym2A2} 
 \begin{enumerate}
 \item Let $X = (\AA^{2})^{2}$ and let $G := \ZZ/2$ act on $X$ by permuting the two copies of  $\AA^{2}$; algebraically,
$(\AA^{2})^{2} = \Spec S$, with $S = k[x_{1},x_{2}, y_{1}, y_{2}]$ and the nontrivial element $\sigma\in G$ acts by
$\sigma(x_{i}) = y_{i}$. 
\item Show that $G$ acts freely on the complement of the diagonal, but fixes the diagonal pointwise.
\item Show that the algebra $S^{G}$ has dimension 4 and is generated by the 5 elements
$$ 
f_{1} = x_{1}+y_{1}, f_{2} = x_{2}+y_{2}, g_{1} = x_{1}y_{1}, g_{2} = x_{2}y_{2}, h = x_{1}y_{2}+x_{2}y_{1},
$$
perhaps by appropriately modifying the steps given in \cite[Exercise 1.6]{Eisenbud1995}. 
\item Show that $h^2$ lies in the subring generated by $f_1,\dots, f_4$, and thus $S^{(2)}$ is a hypersurface, singular
along the  codimension 2 subset $f_{1} = f_{2} = 0$, which is the image of the diagonal subset of the 
cartesian product $(\AA^{2})^{2}$.
\end{enumerate}
\end{exercise}


 \begin{exercise}[The universal divisor of degree $d$]\label{universal divisor}
Let $C$ be a smooth projective curve, and $C^{(d)}$ its $d$th symmetric power. Show that the locus
$$
\cD := \{ (D, p) \in C^{(d)} \times C \mid p \in D \}
$$
is a closed subvariety of the product $C^{(d)} \times C$, whose fiber over any point $D \in C^{(d)}$ is the divisor $D \subset C$.

(Hint: consider the $d$th Cartesian product $C^d$, and let
$$
\Delta_i = \{ \left( (x_1,\dots,x_d), x \right) \in C^d \times C \mid x_i = x \}
$$
be the $i$th diagonal. We have a diagram

\begin{diagram}
\bigcup \Delta_i & \rTo & C^d \times C \\
 \dTo & & \dTo \\
 \cD & \rTo & C^{(d)} \times C
\end{diagram}
and since the union $\cup \Delta_i$ is a projective variety, its image $ \cD \subset C^{(d)} \times C$ is closed.)
\end{exercise}

\begin{exercise}
For any smooth curve $C$, show that a general invertible sheaf of degree $g+2$ defines a birational map to $\\P^2$, and the image of this map has only nodes as singularities. Compute the number of singularities
of such a plane curve.

Hint: Given a general divisor class $D$ of degree $g+2$, we have to show three things: that the image $C_0 = \phi_D(C)$ does not have cusps; that it does not have triple points, and that it does not have tacnodes. (The facts that $|D|$ is base point free and birationally very ample will follow from these arguments, as noted below.)

Cusps: to say that a point $p \in C$ maps to a cusp of $C_0$ (that is, the differential $d\phi_D$ is zero at $p$) amounts to saying that $h^0(D-2p) \geq 2$; that is, $D-2p$ is a $g^1_g$. But by Riemann-Roch, $W^1_g = K_C - W_{g-2}$; so to say $\phi_D$ has a cusp means that
$$
\mu(D) \in 2W_1 + K_C - W_{g-2},
$$
and the locus on the right has dimension at most $g-1$, a general point of $J(C)$ will not lie in it. Note that this subsumes the fact that $|D|$ has no base points.

Triple points: to say that $C_0$ has a triple point means that for some divisor $E = p+q+r$ of degree 3, $h^0(D-E) \geq 1$; thus we must have 
$$
\mu(D) \in W_3 + W^1_{g-1}
$$
Now, to argue that this is not the case, we need to know that $\dim W^1_{g-1} \leq g-4$. In fact, that's not always true: the correct statement is that $\dim W^1_{g-1} = g-4$ if $C$ is non-hyperelliptic, and $\dim W^1_{g-1} = g-3$ if $C$ is hyperelliptic. (This is Marten's theorem.) In fact, the hyperelliptic case violates the statement of our theorem/exercise: if $C$ is hyperelliptic, then a general divisor of degree $g+2$ is of the form $g^1_2 + p_1+ \dots + p_g$, from which we see that $\phi_D$ maps $C$ birationally onto a plane curve of degree $g+2$ having a point of multiplicity $g$! So it seems we have to add ``$C$ nonhyperelliptic" to the hypotheses.

Tacnodes: To say that a pair of points $p, q \in C$ map to a tacnode of $C_0$ means two things: that $h^0(D-E) \geq 2$ (where $E = p+q$); and that $h^0(D-2E) \geq 1$. This means
$$
\mu(D) \in W_2 + (K_C - W_{g-2}) \quad \text{and} \quad \mu(D) \in 2W_2 + W_{g-2}.
$$
Now, each of these conditions is satisfied by a general divisor $D$ of degree $g+2$; the point is, they can't be satisfied simultaneously with the same divisor $E$; and to see this we would need a description of the tangent spaces to subvarieties of $J(C)$.

\end{exercise}


\begin{exercise}
Show that if $r \geq d-g$, then $W^r_d(C) \setminus W^{r+1}_d(C)$ is dense in $W^r_d(C)$ (that is, $W^{r+1}_d(C)$ does not contain any irreducible component of $W^r_d(C)$).
\end{exercise}

\begin{exercise}
Let $C \subset J(C)$ be the image of the Abel-Jacobi map $\mu_1$. Show that the self-intersection of the curve $C$ is 2,
\begin{enumerate}
\item by applying the adjunction formula to $C \subset J(C)$; and
\item by calculating the self-intersection of its preimage $C + p \subset C_2$ and using the geometry of the map $\mu_2$.
\end{enumerate}
\end{exercise}

\begin{exercise}
Consider the map $C \times C \to \Pic_0(C)$ defined by sending $(p, q)\in C \times C$ to the invertible sheaf $\cO_C(p-q)$. What is the degree of this map?
\end{exercise}

\begin{exercise}\label{blow-up of $J(C)$ at a point}
Let $C$ be a curve of genus 2. The canonical map $\phi_K : C \to \PP^1$ expresses $C$ as a 2-sheeted cover of $\PP^1$, and we have correspondingly an involution $\tau : C \to C$ exchanging points in the fibers of $\phi_K$ (equivalently, for any $p \in C$, we have $h^0(K_C(-p)) = 1$; $\tau$ will send $p$ to the unique zero of the unique section $\sigma \in H^0(K_C(-p))$). Let $\Gamma \subset C \times C$ be the graph of $\tau$.
\begin{enumerate}
\item Find the self-intersection of $\Gamma$ in $C \times C$
\item Show the self-intersection of the image of $\Gamma$ in $C_2$ is $-1$.
\end{enumerate}
\end{exercise}

\input footer.tex