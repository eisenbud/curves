%header and footer for separate chapter files

\ifx\whole\undefined
\documentclass[12pt, leqno]{book}
\usepackage{graphicx}
\input style-for-curves.sty
\usepackage{hyperref}
\usepackage{showkeys} %This shows the labels.
%\usepackage{SLAG,msribib,local}
%\usepackage{amsmath,amscd,amsthm,amssymb,amsxtra,latexsym,epsfig,epic,graphics}
%\usepackage[matrix,arrow,curve]{xy}
%\usepackage{graphicx}
%\usepackage{diagrams}
%
%%\usepackage{amsrefs}
%%%%%%%%%%%%%%%%%%%%%%%%%%%%%%%%%%%%%%%%%%
%%\textwidth16cm
%%\textheight20cm
%%\topmargin-2cm
%\oddsidemargin.8cm
%\evensidemargin1cm
%
%%%%%%Definitions
%\input preamble.tex
%\input style-for-curves.sty
%\def\TU{{\bf U}}
%\def\AA{{\mathbb A}}
%\def\BB{{\mathbb B}}
%\def\CC{{\mathbb C}}
%\def\QQ{{\mathbb Q}}
%\def\RR{{\mathbb R}}
%\def\facet{{\bf facet}}
%\def\image{{\rm image}}
%\def\cE{{\cal E}}
%\def\cF{{\cal F}}
%\def\cG{{\cal G}}
%\def\cH{{\cal H}}
%\def\cHom{{{\cal H}om}}
%\def\h{{\rm h}}
% \def\bs{{Boij-S\"oderberg{} }}
%
%\makeatletter
%\def\Ddots{\mathinner{\mkern1mu\raise\p@
%\vbox{\kern7\p@\hbox{.}}\mkern2mu
%\raise4\p@\hbox{.}\mkern2mu\raise7\p@\hbox{.}\mkern1mu}}
%\makeatother

%%
%\pagestyle{myheadings}

%\input style-for-curves.tex
%\documentclass{cambridge7A}
%\usepackage{hatcher_revised} 
%\usepackage{3264}
   
\errorcontextlines=1000
%\usepackage{makeidx}
\let\see\relax
\usepackage{makeidx}
\makeindex
% \index{word} in the doc; \index{variety!algebraic} gives variety, algebraic
% PUT a % after each \index{***}

\overfullrule=5pt
\catcode`\@\active
\def@{\mskip1.5mu} %produce a small space in math with an @

\title{Personalities of Curves}
\author{\copyright David Eisenbud and Joe Harris}
%%\includeonly{%
%0-intro,01-ChowRingDogma,02-FirstExamples,03-Grassmannians,04-GeneralGrassmannians
%,05-VectorBundlesAndChernClasses,06-LinesOnHypersurfaces,07-SingularElementsOfLinearSeries,
%08-ParameterSpaces,
%bib
%}

\date{\today}
%%\date{}
%\title{Curves}
%%{\normalsize ***Preliminary Version***}} 
%\author{David Eisenbud and Joe Harris }
%
%\begin{document}

\begin{document}
\maketitle

\pagenumbering{roman}
\setcounter{page}{5}
%\begin{5}
%\end{5}
\pagenumbering{arabic}
\tableofcontents
\fi

%\documentclass[12pt, leqno]{book}
%\usepackage{amsmath,amscd,amsthm,amssymb,amsxtra,latexsym,epsfig,epic,graphics}
%\usepackage[matrix,arrow,curve]{xy}
%\usepackage{graphicx}
%\usepackage{diagrams}
%%\usepackage{amsrefs}
%%%%%%%%%%%%%%%%%%%%%%%%%%%%%%%%%%%%%%%%%%
%%\textwidth16cm
%%\textheight20cm
%%\topmargin-2cm
%\oddsidemargin.8cm
%\evensidemargin1cm
%
%%%%%%Definitions
%\input preamble.tex
%\def\TU{{\bf U}}
%\def\AA{{\mathbb A}}
%\def\BB{{\mathbb B}}
%\def\CC{{\mathbb C}}
%\def\QQ{{\mathbb Q}}
%\def\RR{{\mathbb R}}
%\def\facet{{\bf facet}}
%\def\image{{\rm image}}
%\def\cE{{\cal E}}
%\def\cF{{\cal F}}
%\def\cG{{\cal G}}
%\def\cH{{\cal H}}
%\def\cHom{{{\cal H}om}}
%\def\h{{\rm h}}
% \def\bs{{Boij-S\"oderberg{} }}
%
%\makeatletter
%\def\Ddots{\mathinner{\mkern1mu\raise\p@
%\vbox{\kern7\p@\hbox{.}}\mkern2mu
%\raise4\p@\hbox{.}\mkern2mu\raise7\p@\hbox{.}\mkern1mu}}
%\makeatother
%
%%%
%%\pagestyle{myheadings}
%\date{April 30, 2018}
%%\date{}
%\title{Curves}
%%{\normalsize ***Preliminary Version***}} 
%\author{David Eisenbud and Joe Harris }
%
%\begin{document}

\chapter{What linear series exist?}\label{Brill-Noether}

\section{What linear series exist?}

In the last few chapters, we have alternated between setting up a general description of linear series on curves, and showing how this plays out in examples. It's time to return to the general theory, and the next question to ask, naturally, is ``What linear systems exist?"

There are various ways to interpret this question. Let's start by taking the question in its crudest form---for which $g, r$ and $d$ does there exist a curve $C$ of genus $g$ and a linear system $(\cL,V)$ on $C$ of degree $d$ and dimension $r$? In this form, the answer is given for line bundles of large degree $d \geq 2g-1$ by the Riemann-Roch theorem: on any curve, there exists a linear series of degree $d \geq 2g-1$ and dimension $r$ iff $r \leq d-g$. As for line bundles of degree $d \leq 2g-2$, we have Clifford's theorem; and combining Clifford with Riemann-Roch, we arrive at the answer to the crude form of our question:

\begin{theorem}\label{arbitrary linear series}
There exists a curve $C$ of genus $g$ and line bundle $\cL$ of degree $d$ on $C$ with $h^0(\cL) \geq r+1$ if and only if
$$
r \leq
\begin{cases}
d-g, \quad \text{if } d \geq 2g-1; \text{ and} \\
d/2,  \quad \text{if } 0 \leq d \leq 2g-2.
\end{cases}
$$
\end{theorem}

\begin{exercise}
Prove a slightly stronger version of Theorem~\ref{arbitrary linear series} in the range $d \leq g-1$: that under the hypotheses of Theorem~\ref{arbitrary linear series} there exists a \emph{complete} linear series of degree $d$ and dimension $r$ for any $r \leq d/2$.
\end{exercise}

\subsection{Castelnuovo's theorem}

Theorem~\ref{arbitrary linear series} gives a complete and sharp answer to the question originally posed: for which $d,r$ and $g$ does there exists a triple $(C,\cL,V)$ with $C$ a curve of genus $g$, $\cL$ a line bundle of degree $d$ on $C$ and $V \subset H^0(\cL)$ of dimension $r+1$. 

But maybe that wasn't the question we meant to ask! After all, we're interested in describing curves in projective space as images of abstract curves $C$ under maps given by linear systems on $C$. Observing that the linear series that achieve equality in Clifford's theorem give maps to $\PP^r$ that are 2 to 1 onto a rational curve, we might hope that we would have a different---and more meaningful---answer if we  restrict our attention to linear series $\cD = (\cL,V)$ for which the associated map $\phi_\cD$ is at least a birational embedding. 

Now, we'll see that the maximal dimension of a birationally very ample linear series of degree $d$ on a curve of genus $g$ is a decreasing function of $g$; so with this restriction, the question is tantamount to the

\begin{question}
What is the largest possible genus of an irreducible, nondegenerate curve $C \subset \PP^r$ of degree $d$?
\end{question}

A sharp bound, answering  this question, is given by a theorem of Castelnuovo, and is quite different from the inequality in Theorem~\ref{arbitrary linear series}.  We'll sketch the derivation of the inequality here; we'll prove that it is in fact sharp and describe in detail  the curves that achieve it in Chapter~\ref{}. \fix{*if* we keep the scrolls chaper}

The proof of Castelnuovo's bound if $C$ is a curve of degree $d$ and genus $g$ in $\PP^r$, the idea is to prove successive lower bounds for the dimensions $h^0(\cO_C(m))$ of multiples of the $g^r_d$ cut on $C$ by hyperplanes. For large values of $m$ the line bundle $\cO_C(m)$ is non-special, and so a lower bound on the dimension of its space of sections translates, via Riemann-Roch, into an upper bound on the genus $g$.

\begin{definition}
Let $\cL$ be any line bundle on a smooth projective variety $X$, and $D = \{p_1,\dots,p_d\}$ a collection of points of $X$. By the \emph{number of conditions imposed by $D$ on sections of $\cL$} we  mean  the difference
$$
h^0(\cL) - h^0(\cL \otimes \cI_{D/X});
$$
that is, the codimension in $H^0(\cL)$ of the subspace of sections vanishing on $D$. More generally, if $V \subset H^0(\cL)$ is any linear system, by the number of conditions imposed by $D$ on $V$ we will mean the difference
$$
\dim(V) - \dim \left(V \cap H^0(\cL\otimes \cI_{D/X}) \right).
$$
\end{definition}
Thus, for example, if $X = \PP^r$, the number of conditions imposed by $D$ on $H^0(\cO_{\PP^r}(m))$ is the value $h_D(m)$ of the Hilbert function of $D$.
Note that the number of conditions imposed by $D$ on a linear system $V$ is necessarily less than or equal to the degree $d$ of $D$; if it is equal we say that $D$ \emph{imposes independent conditions on $V$}.

To apply this notion, suppose $C \subset \PP^r$ is an irreducible, nondegenerate curve. Let $\Gamma = C \cap H$ be a general hyperplane section of $C$. Let $V_m \subset H^0(\cO_C(m))$ be the linear series cut on $C$ by hypersurfaces of degree $m$ in $\PP^r$, that is, the image of the restriction map
$$
H^0(\cO_{\PP^r}(m)) \to H^0(\cO_C(m)).
$$
We have:
\begin{align*}
h^0(\cO_C(m)) - h^0(\cO_C(m-1)) & \geq \text{\# of conditions imposed by $\Gamma$ on $H^0(\cO_C(m))$} \\
&\geq \text{\# of conditions imposed by $\Gamma$ on $V_m$} \\
&\geq \text{\# of conditions imposed by $\Gamma$ on $H^0(\cO_{\PP^r}(m))$}.
\end{align*}
Thus the dimension $h^0(\cO_C(m))$ is bounded below by the sum
$$
h^0(\cO_C(m)) \geq \sum_{k=0}^m h_\Gamma(k).
$$

To apply this we need a lower bound on the Hilbert function of a general hyperplane section $\Gamma$ of our curve $C$. This is turn requires that we have some knowledge of the geometry of $\Gamma$:

\begin{lemma}[general position lemma]\label{general position lemma}
If $C \subset \PP^r$ is an irreducible, nondegenerate curve and $\Gamma = C \cap H$ a general hyperplane section of $C$, then the points of $\Gamma$ are in linearly general position in $H \cong \PP^{r-1}$; that is, no $r$ points of $\Gamma$ lie in a hyperplane $\PP^{r-2} \subset H$.
\end{lemma}

Thus, for example, if $C \subset \PP^3$ is a space curve, the general position lemma says that no three points of a general plane section $\Gamma = H \cap C$ of $C$ are: collinear \fix{here with 2 ells; below with one. TeX seems to prefer 2, but 1 would be more logical}. Even this case is surprisingly tricky to prove (and it's false in characteristic $p$!); the following exercise sketches a proof.

\begin{exercise}
Let $C \subset \PP^3$ be an irreducible, nondegenerate space curve. Assuming characteristic 0,
\begin{enumerate}
\item Show that if $p, q \in C$ are general, then the tangent lines $\TT_pC$ and $\TT_qC \subset \PP^3$ do not intersect.
\item Using this, show that for general $p, q \in C$ the line $\overline{p,q}$ is not a trisecant; that is, it does not intersect $C$ a third time.
\item Using \fix{the subject of a different exercise?} the irreducibility of the variety of chords to $C$, show that $C$ can have at most a 1-parameter family of trisecant lines.
\item Deduce that a general plane $H \subset \PP^3$ does not contain three colinear points of $C$.
\end{enumerate}
\end{exercise}

The general position lemma was originally asserted by Castelnuovo. In modern language, it can be deduced as a special case of the more general \emph{uniform position lemma}; since this is a useful (and beautiful) theorem in its own right we'll take a few pages out and 
derive it. \fix{I changed this from ``describe the derivation}. 

\subsubsection{Uniform position} We start by introducing the \emph{monodromy group} of a generically finite cover. Let $f : Y \to X$ be a dominant map between varieties of the same dimension over $\CC$, and suppose that $X$ is irreducible. There is then an open subset $U \subset X$ such that $U$ and 
its preimage $V = f^{-1}(U)$ are smooth, and the restriction of $f$ to $V$ is a covering space in the classical topology; we'll denote by $d$ the number of sheets. 

Choose a base point $p_0 \in U \subset X$, and suppose $\Gamma := f^{-1}(p_0)  = \{q_1,\dots,q_d\}$. If $\gamma$ is any loop in $U$ with base point $p_0$, for any $i = 1, \dots, d$ there is a unique lifting of $\gamma$ to an arc $\tilde \gamma_i$ in $V$ with initial point $\tilde \gamma_i(0) = q_i$ and end point $\tilde \gamma_i(1) = q_j$ for some $j \in \{1,2,\dots,d\}$; in this way, we can associate to $\gamma$ a permutation of $\{1,2,\dots,d\}$. 
Since the set $\Gamma$ is discreet, the permutation depends only on the class of $\gamma$ in $\pi_1(U,p_0)$ so we have defined a homomorphism to the symmetric group:
$$
\pi_1(U,p_0)  \to {\rm Perm}(\Gamma) \cong S_d.
$$
The image $M$ of this map is called the \emph{monodromy group} of the map $f$. It depends on the labeling of the points of $\Gamma$, but is well-defined  up to conjugation. Moreover, it is independent of the choice of open set $U$: if $U' \subset U$ is a Zariski open subset, the map $\pi_1(U', p_0) \to \pi_1(U,p_0)$ is surjective,  so the image of $\pi_1(U', p_0)$ in $S_d$ is the same.

\begin{fact}
There is another characterization of the monodromy group $M$ that will not be used here but that is worth knowing. In the situation described above, the pullback map $f^*$ expresses the function field $K(Y)$ as a finite algebraic extension of $K(X)$; the degree $d$ is the degree of this extension, and $M$ is equal to the Galois group of the Galois normalization of $K(Y)$ over $K(X)$. (For a proof of this equality, see \cite{Harris}.)
\end{fact}

One note: we have assumed here that both $X$ and $Y$ are irreducible. In fact, we need only have assumed that $X$ is irreducible; we can apply the same construction in case $Y$ is reducible (note that any irreducible components of $Y$ that fail to dominate $X$ simply won't appear in the construction). 

Since $V$ is assumed smooth, it is connected, or path-connected, if and only if it is irreducible, and if $Y$ is equidimensional then this is the case if and only if $Y$ is irreducible. In this case we see that \emph{$Y$ is irreducible if and only if the monodromy group $M \subset S_d$ is transitive}. We can extend this result in a useful way as follows:

\begin{lemma}\label{transitivity lemma}
Let $f : Y \to X$ be a generically finite cover of degree $d$, with  monodromy group $M \subset S_d$; let $U \subset X$ and $V = f^{-1}(U) \subset Y$ be open sets as above. For any $k = 1,2,\dots,d$, let $V_k^*$ be the complement of the large diagonal in the $k$th fiber power of $V \to U$; that is,
$$
V_k^* = \{ (x; y_1,\dots, y_k) \in U \times V^k \mid f(y_i) = x \text{ and } y_i \neq y_j \; \forall i \neq j\}.
$$
Then $V_k^*$ is irreducible if and only if $M$ is $k$ times transitive. \qed
\end{lemma}

\begin{lemma}\label{transposition lemma}
Let $f : Y \to X$ be a generically finite cover of degree $d$ over an irreducible variety $X$, with  monodromy group $M \subset S_d$.  
If, that for some smooth point $p \in X$ the fiber $f^{-1}(p)\subset V$ consists of $d-2$ reduced points $p_1,\dots, p_{d-2}$ and one point $q$ of multiplicity 2, where $q$ is also a smooth point of $Y$, then $M$ contains a transposition.
\end{lemma}

\begin{proof} To begin with, let $U \subset X$ be a Zariski open subset of the smooth locus in $X$, as in the definition of the monodromy group, so that  $V := f^{-1}(U)$ is also smooth and the restriction $f|_V : V \to U$ expresses $V$ as a $d$-sheeted covering space of $U$, with $U,V$ smooth and $p\in X$

Now, let $A \subset X$ be a small neighborhood of $p$ in the classical topology. The preimage $f^{-1}(A)$ will have $d-1$ connected components, one containing each of the points $p_1,\dots,p_{d-2}$ and $q$; let
 $B \subset f^{-1}(A) \subset Y$ be the connected component of $f^{-1}(A)$ containing the point $q$.

Let $p' \in A \cap U$. Two of the $d$ points of $f^{-1}(p')$ will lie in the component $B$ of $f^{-1}(A)$ containing $q$; call these $q'$ and $q''$. Since $B \cap V$ is connected (it's the complement of a proper subvariety in the neighborhood $B$ of the smooth point $q \in Y$), we can draw a real arc $\gamma : [0,1] \to B \cap V$ joining $q'$ to $q''$; by construction, the permutation of $f^{-1}(p')$ associated to the loop $f \circ \gamma$ will exchange $q'$ and $q''$ and fix each of the remaining $d-2$ points of $f^{-1}(p')$.
\end{proof}

To apply these lemmas, let $C \subset \PP^r$ be a smooth irreducible, nondegenerate curve of degree $d$, let $X = {\PP^r}^*$ be the space of hyperplanes in $\PP^r$, and let
$$
Y = \{ (H, p) \in {\PP^r}^* \times C \mid p \in H \}
$$
The projection of $Y$ to ${\PP^r}^*$, restricted to the the open set $U\subset {\PP^r}^*$
of hyperplanes that are transverse to $C$, is a topological covering space of degree $d$ over   $U$. The monodromy group of this covering is called the monodromy group of the general hyperplane section of $C$.

\begin{theorem}[uniform position lemma]\label{uniform position lemma}
The monodromy group $M$ of the general hyperplane section of an irreducible curve
in $\PP^r_\CC$ is the full symmetric group.
\end{theorem}

Informally, this Theorem says that two subsets of the same cardinality in the general hyperplane section of $C$
are indistinguishable from the point of view of any discrete invariant that is constant in
a connected, smooth family. We shall see examples below.

This result fails over fields of finite characteristic; see \cite{Rathmann and ****} for examples and details of what is known.

\fix{de revised to here on March 27, 2022}

\begin{proof}
We show first that $M$ is twice transitive, and then that it contains a transposition. It follows that $M$ contains all the transpositions so that  $M = S_d$.

For the double transitivity, we introduce a related cover: set
$$
\Phi = \{ (H, p, q) \in {\PP^r}^* \times C \times C \mid p + q \subset H \}
$$
(Here $p+q$ is the divisor $p+q$ on $C$, viewed as a subscheme of $C$.) The projection $\pi_{2,3} : \Phi \to C \times C$ is a $\PP^{r-2}$-bundle, and so irreducible; applying Lemma~\ref{transitivity lemma}, we deduce that $M$ is twice transitive.

Finally, for the existence of a transposition in $M$, we have to use the hypothesis of characteristic zero to say that \emph{not every point of $C$ is a flex}; that is, if $p\in C$ is a general point and $H$ a general hyperplane containing the tangent line $\TT_pC$ then $H$ intersects $C$ with multiplicity 2 at $p$. (The notion of inflectionary behavior of curves in projective space will be taken up in Chapter~\ref{InflectionsChapter}; in particular, we will prove that in characteristic 0 not every point of  a curve $C\subset \PP^n$ is a flex.) Given this, let $p \in C$ be a non-flex point, and let $H \subset \PP^r$ be a general hyperplane containing the tangent line $T_pC$. Under these hypotheses, the fiber of $\Phi$ over the point $H \in {\PP^r}^*$ consists of the point $p$ with multiplicity 2, and $d-2$ reduced points; applying Lemma~\ref{transposition lemma}, we deduce that $M$ contains a transposition.
\end{proof}

Note that this proof works as well without the hypothesis of smoothness; we just restrict to the open subset of hyperplanes not passing through any of the singular points of $C$. It does, however, require in addition the fact that \emph{a general tangent line to $C$ is not bitangent}; we leave it as an exercise for the reader the reader to supply this argument.

Finally, as a consequence of Proposition~\ref{uniform position lemma} (and Lemma~\ref{transitivity lemma}) we can deduce the

\begin{lemma}[numerical uniform position lemma]
With $C \subset \PP^r$ and $\Gamma = C \cap H$ as above, any two subsets $\Gamma', \Gamma'' \subset \Gamma$ of the same cardinality $k$ have the same Hilbert function, i.e., impose the same number of conditions on $\cO_{\PP^{r-1}}(m)$ for all $m$.
\end{lemma}

\begin{proof}
In this situation, we restrict to the open set $U = {\PP^r}^* \setminus C^*$ of hyperplanes transverse to $C$, and introduce the fiber power
$$
V_k^* = \{ (x; y_1,\dots, y_k) \in U \times V^k \mid f(y_i) = x \text{ and } y_i \neq y_j \; \forall i \neq j\}.
$$
as above; $V_k^*$ parametrizes subsets $\Gamma$ of cardinality $k$ in hyperplane sections $H \cap C$ of $C$. Applying Lemma~\ref{transitivity lemma}, we see that $V_k^*$ is irreducible of dimension $r$. 

Now, the Hilbert function $h_\Gamma(m)$ is lower semicontinuous, so it achieves its maximum on a Zariski open subset of $V_k^*$. Since $V_k^*$ is irreducible, the complement of this open will have dimension strictly less than $r$;  a general hyperplane $H \in {\PP^r}^*$ will lie outside this image, meaning that $h_\Gamma(m)$ is the same for all $\Gamma \subset C \cap H$ of cardinality $k$.
\end{proof}

As an immediate consequence of Proposition~\ref{uniform position lemma}, we can complete the proof of Clifford's theorem as stated above. Recall that Clifford's theorem was deduced from the general inequality 
$$
\dim(\cD + \cE) \geq \dim \cD + \dim \cE.
$$
for any pair of nonempty linear series on a curve $C$; applying this to a linear series $\cD = |D|$ and the residual series $\cE = |K-D|$ we arrived at Clifford's inequality $r(D) \leq d/2$.

The rest of Clifford's theorem describes when we can have equality in Clifford's inequality. Now, in that case, we have
$$
r(D) + r(K-D) = g-1;
$$
which means that \emph{every canonical divisor is expressible as a sum of a divisor in $|D|$ and a divisor in $|K-D|$}. In other words, assuming the curve $C$ is non-hyperelliptic, a general hyperplane section $H \cap C$ of the canonical curve contains a subset of $d$ points whose sum is linearly equivalent to $D$. But by the monodromy statement above, this means \emph{every} subset of $d$ points in $H \cap C$ is linearly equivalent to $D$. But this is clearly false: for example, for any $d+1$ points $p_1,\dots,p_{d+1} \in H \cap C$ we would have
$$
p_1 + \dots + p_d \sim D \sim p_1,\dots, p_{d-1} + p_{d+1},
$$
meaning $p_d \sim p_{d+1}$. Thus $C$ must be hyperelliptic, and the remainder of Clifford's theorem follows.

The argument here gives rise to another application of Theorem~\ref{uniform position lemma}. A basic fact about linear series on curves---used for example in the derivation of Clifford's inequality---is that if $\cD$ and $\cE$ are any two linear series on a curve $C$, then
$$
\dim (\cD + \cE) \geq dim (\cD) + \dim (\cE).
$$
We can now observe that \emph{under the two hypothesis that the sum $\cD + \cE$ is birationally very ample and $g(C) > 0$, we have strict inequality: $\dim (\cD + \cE) > dim (\cD) + \dim (\cE)$}.

We return now to Castelnuovo's analysis.

The general position lemma is just the special case $m=1$ of the numerical uniform position lemma. This may not seem like much information about $\Gamma$, but in fact it's all we need to prove a sharp bound! The basic (and completely elementary) statement is

\begin{proposition}\label{min hilb}
If $\Gamma \subset \PP^n$ is a collection of $d$ points in linearly general position and spanning $\PP^n$, then 
$$
h_\Gamma(m) \geq \min\{d, mn+1\}
$$
\end{proposition}

\begin{proof}
Suppose first that $d \geq mn+1$, and let $p_1,\dots,p_{mn+1} \in \Gamma$ be any subset of $mn+1$ points. We want to show that $\Gamma' = \{p_1,\dots,p_{mn+1}\}$ imposes independent conditions of $H^0(\cO_{\PP^n}(m))$, that is, for any $p_i \in \Gamma'$ we can find a hypersurface $X \subset \PP^n$ of degree $m$ containing all the points $p_1,\dots, \hat{p_i},\dots,p_{mn+1}$ but not containing $p_i$.

This is easy: simply group the $mn$ points of $\Gamma' \setminus \{p_i\}$ into $m$ subsets $\Gamma_k$ of cardinality $n$; each set $\Gamma_k$ will span a hyperplane $H_k \subset \PP^n$, and we can take $X = H_1 \cup \dots \cup H_m$. 
\end{proof}

This may seem like a crude argument, but the bound derived is sharp, as the following exercise asserts

\begin{exercise}
Let $D \subset \PP^n$ be a rational normal curve. If $\Gamma \subset D$ is any collection of $d$ points on $D$ (or for that matter any subscheme of $D$ of degree $d$) then the Hilbert function of $\Gamma$ is
$$
h_\Gamma(m) = \min\{d, mn+1\}
$$
\end{exercise} 

\begin{fact}
Castelnuovo in fact proved a converse to this observation: he showed that if $d \geq 2n+3$, then \emph{any collection of $d$ points in $\PP^n$ in linearly general position with Hilbert function $h_\Gamma(m) = \min\{d, mn+1\}$ must lie on a rational normal curve}. This was a key ingredient in Castelnuovo's theorem characterizing curves of maximal genus.
\end{fact}

At this point, all that remains is to add up the lower bounds in the proposition. To this end, let $C \subset \PP^r$ be as above an irreducible, nondegenerate curve of degree $d$, and set $M = \lfloor{\frac{d-1}{r-1}}\rfloor$, so that we can write
$$
d = M(r-1) + 1 + \epsilon \quad \text{ with } \quad 0 \leq \epsilon \leq r-2.
$$
We have then
\begin{align*}
h^0(\cO_C(M)) &\geq \sum_{k=0}^M h^0(\cO_C(k)) - h^0(\cO_C(k-1)) \\
&\geq  \sum_{k=0}^M k(r-1)+1 \\
&= \frac{M(M+1)}{2}(r-1) + M + 1
\end{align*}
and similarly
$$
h^0(\cO_C(M+m)) \geq \frac{M(M+1)}{2}(r-1) + M + 1 + md.
$$
For sufficiently large $m$, the line bundle $\cO_C(M+m)$ will be nonspecial, so we can plug this in to Riemann-Roch to arrive at
\begin{align*}
g &= (M+m)d - h^0(\cO_C(M+m)) + 1 \\
&\leq (M+m)d - \bigl(  \frac{M(M+1)}{2}(r-1) + M + 1 + md \bigr) \\
& = M\bigl( M(r-1) + 1 + \epsilon \bigr) - \bigl(  \frac{M(M+1)}{2}(r-1) + M + 1 \bigr) \\
&= \frac{M(M-1)}{2}(r-1) + M\epsilon.
\end{align*}

To summarize our discussion: for positive integers $d$ and $r$, we write
$$
 d = M(r-1) + 1 + \epsilon \quad \text{ with } \quad 0 \leq \epsilon \leq r-2
$$
and set
$$
\pi(d,r) = \frac{M(M-1)}{2}(r-1) + M\epsilon.
$$
In these terms, we have proved the

\begin{theorem}[Castelnuovo's bound]
If $C \subset \PP^r$ is an irreducible, nondegenerate curve of degree $d$ and genus $g$, then
$$
g \leq \pi(d,r).
$$
\end{theorem}

We will see in Chapter~\ref{} that this is in fact sharp: for every $r$ and $d \geq r$, there do exist such curves with genus exactly $\pi(d,r)$. For now, we make a few observations:

\begin{enumerate}
\item In case $r=2$, all the inequalities used in the derivation of Castenuovo's bound are in fact equalities, and indeed we see that in this case $\pi(d,2) = \binom{d-1}{2}$ is the genus of a smooth plane curve of degree $d$.

\item In case $r=3$, we have
$$
\pi(d,3) =
\begin{cases}
\left( k - 1 \right)^2 &\text{ if $d=2k$ is even; and} \\
k(k-1) &\text{ if $d=2k+1$ is odd.}
\end{cases}
$$
In this case again, it's not hard to see the bound is sharp: these are exactly the genera of curves of bidegree $(k,k)$ and $(k+1,k)$ respectively on a quadric surface $Q \cong \PP^1 \times \PP^1 \subset \PP^3$.
\item In general, we see that for fixed $r$ asymptotically
$$
\pi(d,r) \sim \frac{d^2}{2(r-1)}.
$$
\end{enumerate}


\begin{exercise}
Show that with $C$ as above, the line bundle $\cO_C(M)$ is nonspecial. (We will see in Section~\ref{} that this is sharp; that is, there exist such curves $C$ with $\cO_C(M-1)$ special).
\end{exercise}

\section{Brill-Noether theory}

\subsection{Basic questions addressed by Brill-Noether theory}

In the last section, we restricted our attention to the linear series most of interest to us: those corresponding to embeddings of our curve in projective space (or at any rate birational embeddings) and their limits. But there is one other respect in which Castelnuovo theory fails to address a basic concern: the curves with linear systems achieving Castelnuovo's bound are, like hyperelliptic curves, very special. (In fact, we'll see in Section~\ref{**} that in general they are even rarer than hyperelliptic curves.) That is, if we were to pick a curve $C$ of genus $g$ ``at random," we would still have no idea what linear systems existed on $C$ or how they behaved.

Brill-Noether theory addresses exactly this issue: it asks, ``what linear series exist on a \emph{general} curve of a given genus?" When it comes to non-special linear series, of course, Riemann-Roch answers the question simply and completely, so we'll restrict our attention to special linear series (i.e., with $r > d - g$). That said, we'll start with the crudest form of the theorem:

\begin{theorem}\label{basic BN}
Fix non-negative integers $g, r$ and $d$ with $d \leq g+r$, and let $C$ be a general curve of genus $g$. If we denote by $W^r_d(C) \subset \Pic^d(C)$ the locus of invertible sheaves $\cL$ with $h^0(\cL) \geq r+1$, then
$$
\dim W^r_d(C) = \rho(g,r,d) := g - (r+1)(g-d+r).
$$
In particular, it is the case that a general curve of genus $g$ possesses a linear series of degree $d$ and dimension $r$ if and only if $\rho(g,r,d) \geq 0$.
\end{theorem}

In the following sections, we'll see why we might naively expect this to be the case, and we'll also list some of the myriad refinements and strengthenings of the theorem.  A proof of the existence half of the theorem (the ``if" part of the statement) may be found in \cite{3264};  we will give in Chapter~\ref{InflectionsChapter} of this book a relatively simple proof of the nonexistence part (the ``only if"). In the meantime, we'll mention here the special case $r=1$:

\begin{corollary}
If $C$ is any curve of genus $g$, then $C$ admits a rational function of degree $d$ for some positive $d \leq \lceil \frac{g+2}{2}\rceil$.
\end{corollary}

Thus, for example, any curve of genus 2 is hyperelliptic, any curve of genus 3 or 4 is either hyperelliptic or trigonal, and so on.

\subsection{Heuristic argument leading to the statement of Brill-Noether}

The Brill-Noether theorem, broadly construed as in Theorem~\ref{BN omnibus}, is a far-reaching description of the linear series to be found on a general curve. It starts, though, with a relatively simple dimension count---one that was first carried out almost a century and a half ago.

To set this up, let $C$ be a smooth projective curve of genus $g$, and $D = p_1 + \dots + p_d$ a divisor on $C$. We'll assume here the points $p_i$ are distinct; the same argument (albeit with much more complicated notation) can be carried out in general.

When does the divisor $D$ move in an $r$-dimensional linear series? Riemann-Roch gives an answer: it says that $h^0(D) \geq r+1$ if and only if the vector space $H^0(K-D)$ of 1-forms vanishing on $D$ has dimension at least $g-d+r$---that is, if and only if the  evaluation map
$$
H^0(K) \to H^0(K|_D) = \oplus K_{p_i}
$$
has rank at most $d-r$. 

We can represent this map by a $g \times d$ matrix. Choose a basis $\omega_1,\dots,\omega_g$ for the space $H^0(K)$ of 1-forms on $C$; choose an analytic open neighborhood $U_j$ of each point $p_j \in D$ and choose a local coordinate $z_j$ in $U_j$ around each point $p_j$, and write
$$
\omega_i = f_{i,j}(z_j)dz_j
$$
in $U_j$. We will have $r(D) \geq r$ if and only if the  matrix-valued function
$$
A(z_1,\dots,z_d) = 
\begin{pmatrix}
f_{1,1}(z_1) & f_{2,1}(z_1) & \dots & f_{g,1}(z_1) \\
f_{1,2}(z_2) & f_{2,2}(z_2) & \dots & f_{g,2}(z_2) \\
\vdots & \vdots &  & \vdots \\
f_{1,d}(z_d) & f_{2,d}(z_d) & \dots & f_{g,d} (z_d)
\end{pmatrix}
$$
has rank $d-r$ or less at $(z_1,\dots,z_d) = (0,\dots,0)$.

The point is, we can think of $A$ as a matrix valued function in the open set $U = U_1 \times U_2 \times \dots \times U_d \subset C_d$; and for divisors $D \in U$, we have $r(D) \geq r$ if and only if $\rank(A(D)) \leq d-r$. Now, in the space $M_{d,g}$ of $d \times g$ matrices, the subset of matrices of rank $d-r$ or less has codimension $r(g-d+r)$, and so we might naively expect that the locus of divisors with $r(D) \geq r$ would have dimension $d - r(g-d+r)$. At the same time, if any divisor of degree $d$ with $h^0(D) \geq r+1$ exists, then there must be at least an $r$-dimensional family of them; so we'd suspect that such divisors exist only if
$$
d - r(g-d+r) \; \geq \; r,
$$
which is exactly the Brill-Noether statement.

\subsection{Refinements of the Brill-Noether theorem}

As we indicated, Theorem~\ref{basic BN} represents only the most bare-bones version of Brill-Noether. The full statement  adds more information both about the geometry of the scheme $W^r_d(C)$ and about the linear series parametrized by $W^r_d(C)$.  We collect the basic facts into the 

\begin{theorem}[Brill-Noether theorem, omnibus version]\label{BN omnibus}
Let $C$ be a general curve of genus $g$. If we set $\rho = g - (r+1)(g-d+r)$, then
\begin{enumerate}
\item $\dim(W^r_d(C)) = \rho$;
\item\label{sing wrd} the singular locus of $W^r_d(C)$ is exactly $W^{r+1}_d(C)$;
\item\label{irr wrd} if $\rho > 0$ then $W^r_d(C)$ is irreducible;
\item\label{rho=0} if $\rho = 0$ then $W^r_d$ consists of a finite set of  points of cardinality
$$
\#W^r_d = g! \prod_{\alpha=0}^r \frac{\alpha!}{(g-d+r+\alpha)!};
$$
\item\label{Petri} if $L$ is any invertible sheaf on $C$, the map
$$
\mu : H^0(L) \otimes H^0(\omega_CL^{-1}) \rTo H^0(\omega_C)
$$
is injective;
\item\label{general va} if $|D|$ is a general $g^r_d$ on $C$, then
\begin{enumerate}
\item if $r \geq 3$ then $D$ is very ample; that is, the map $\phi_D : C \to \PP^r$   embeds $C$ in $\PP^r$;
\item if $r=2$ the map $\phi_D : C \to \PP^2$ gives a birational embedding of $C$ as a nodal plane curve; and 
\item if $r=1$, the map $\phi_D : C \to \PP^2$ expresses $C$ as a simply branched cover of $\PP^1$.
\end{enumerate}

\item\label{maximal rank} If $L \in W^r_d(C)$ is a general point (or any point, if $\rho = 0$), then for each $m > 0$ the multiplication map
$$
\rho_m : \Sym^m H^0(L) \to H^0(L^m)
$$
has maximal rank; that is, it is either injective or surjective.
\end{enumerate}
\end{theorem}

A few remarks are in order here:
\begin{enumerate}

\item As a special case of Part~\ref{rho=0}, we see that the number of $g^1_{k+1}$s on a general curve of genus $g = 2k$ is the $k$th Catalan number 
$$
c_k = \frac{2k!}{k!(k+1)!}.
$$
We have already seen this in the first three cases: in genus 2, it says the canonical series $|K|$ is the unique $g^1_2$ on a curve of genus 2; in genus 4, it says that there are exactly two $g^1_3$s on a general curve of genus 4 (note that this is true only for a general curve: we've seen that there are curves of genus 4 with only one $g^1_3$); and in genus 6 it says that a general curve of genus 6 has 5 $g^1_4$s, as we've seen in Chapter~\ref{genus 4, 5 and 6 chapter}.  In genus 8, it says that a general curve of genus 8 has 14 $g^1_5$s, and this is new, in the sense that we don't know of any way of seeing this directly from the geometry of a general curve of genus 8.

\item Part\ref{Petri} implies Part\ref{sing wrd}. In fact, a fairly elementary argument shows that at a point $L \in W^r_d(C) \setminus W^{r+1}_d(C)$, the tangent space to $W^r_d$ at the point $L$ is the annihilator of the image of $\mu$; given that $\mu$ is injective, we can compare dimensions and deduce that $W^r_d$ is smooth at $L$.

\item Part~\ref{maximal rank} is the celebrated \emph{maximal rank theorem} of Eric Larson. It answers in general a question that has come up multiple times so far in this book: every time we've asked what hypersurfaces contain a curve $C \subset \PP^r$ embedded by a linear system $|L|$, we've looked at the maps $\rho_m$. Each time, we knew the dimensions of the spaces $\Sym^m H^0(L)$ and $H^0(L^m)$, and the question was the rank of $\rho_m$; now we know the answer for a general linear system of any degree and dimension. In sum, \emph{the maximal rank theorem tells us the Hilbert function of a general curve $C \subset \PP^r$ embedded by a general linear system}.

\item There is a natural extension of the maximal rank theorem of Part~\ref{maximal rank}. Specifically, if $C \subset \PP^r$ is a general curve embedded by a general linear series, the maximal rank theorem tells us the dimension of the $m$th graded piece of the ideal of $C$, for any $m$: this is just the dimension of the kernel of $\rho_m$. But it doesn't tell us what a minimal set of generators fo the homogeneous ideal of $C$ might look like. For example, if $m_0$ is the smallest $m$ for which $I(C)_m \neq 0$, or numerically the smallest $m$ such that $\binom{m+r}{r} > md-g+1$, we can ask: is the homogeneous ideal $I(C)$ generated by $I(C)_{m_0}$?

To answer this question---given that we know the dimensions of $I(C)_m$ for every $m$---we simply need to know the ranks of the multiplication maps
$$
\sigma_m : I(C)_m \otimes H^0(\cO_{\PP^r}(1)) \to I(C)_{m+1}
$$
for each $m$. In particular, we may conjecture that \emph{the maps $\sigma$ have maximal rank}; if this is proved, then we will indeed know the degrees of a minimal set of generators for the homogeneous ideal $I(C)$.

\item There is another object worth mentioning: for any curve $C$, there exists a scheme $G^r_d(C)$ parametrizing linear series of degree $d$ and dimension $r$; that is, in set-theoretic terms,
$$
G^r_d = \left\{ (L, V) \mid L \in Pic^d(C), \text{ and } V \subset H^0(L) \text{ with } \dim V = r+1 \right\}.
$$

(This requires a little more technical machinery to construct, which is why we haven't introduced it.) $G^r_d(C)$ maps to $W^r_d(C)$; the map is a birational isomorphism, being an isomorphism over the open subset $W^r_d(C) \setminus W^{r+1}_d(C)$ and having positive-dimensional fibers over $W^{r+1}_d(C)$. In fact, another version of Part~\ref{sing wrd} is the statement that \emph{for a general curve $C$, the scheme $G^r_d(C)$ is smooth for any $d$ and $r$}.

\item Parts~\ref{general va} and~\ref{maximal rank} describe the geometry of a general curve $C$ as embedded in projective space by a general linear series. But we should add that there are many more questions remaining in this regard. One is the question of secant planes: for example, a naive dimension count would suggest that an irreducible, nondegenerate curve $C \subset \PP^3$ will have a finite number of 4-secant lines, but no 5-secant lines. Is this true for a general curve embedded in $\PP^3$ by a general linear series?
\end{enumerate}

\input footer.tex