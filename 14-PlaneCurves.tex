%header and footer for separate chapter files

\ifx\whole\undefined
\documentclass[12pt, leqno]{book}
\usepackage{graphicx}
\input style-for-curves.sty
\usepackage{hyperref}
\usepackage{showkeys} %This shows the labels.
%\usepackage{SLAG,msribib,local}
%\usepackage{amsmath,amscd,amsthm,amssymb,amsxtra,latexsym,epsfig,epic,graphics}
%\usepackage[matrix,arrow,curve]{xy}
%\usepackage{graphicx}
%\usepackage{diagrams}
%
%%\usepackage{amsrefs}
%%%%%%%%%%%%%%%%%%%%%%%%%%%%%%%%%%%%%%%%%%
%%\textwidth16cm
%%\textheight20cm
%%\topmargin-2cm
%\oddsidemargin.8cm
%\evensidemargin1cm
%
%%%%%%Definitions
%\input preamble.tex
%\input style-for-curves.sty
%\def\TU{{\bf U}}
%\def\AA{{\mathbb A}}
%\def\BB{{\mathbb B}}
%\def\CC{{\mathbb C}}
%\def\QQ{{\mathbb Q}}
%\def\RR{{\mathbb R}}
%\def\facet{{\bf facet}}
%\def\image{{\rm image}}
%\def\cE{{\cal E}}
%\def\cF{{\cal F}}
%\def\cG{{\cal G}}
%\def\cH{{\cal H}}
%\def\cHom{{{\cal H}om}}
%\def\h{{\rm h}}
% \def\bs{{Boij-S\"oderberg{} }}
%
%\makeatletter
%\def\Ddots{\mathinner{\mkern1mu\raise\p@
%\vbox{\kern7\p@\hbox{.}}\mkern2mu
%\raise4\p@\hbox{.}\mkern2mu\raise7\p@\hbox{.}\mkern1mu}}
%\makeatother

%%
%\pagestyle{myheadings}

%\input style-for-curves.tex
%\documentclass{cambridge7A}
%\usepackage{hatcher_revised} 
%\usepackage{3264}
   
\errorcontextlines=1000
%\usepackage{makeidx}
\let\see\relax
\usepackage{makeidx}
\makeindex
% \index{word} in the doc; \index{variety!algebraic} gives variety, algebraic
% PUT a % after each \index{***}

\overfullrule=5pt
\catcode`\@\active
\def@{\mskip1.5mu} %produce a small space in math with an @

\title{Personalities of Curves}
\author{\copyright David Eisenbud and Joe Harris}
%%\includeonly{%
%0-intro,01-ChowRingDogma,02-FirstExamples,03-Grassmannians,04-GeneralGrassmannians
%,05-VectorBundlesAndChernClasses,06-LinesOnHypersurfaces,07-SingularElementsOfLinearSeries,
%08-ParameterSpaces,
%bib
%}

\date{\today}
%%\date{}
%\title{Curves}
%%{\normalsize ***Preliminary Version***}} 
%\author{David Eisenbud and Joe Harris }
%
%\begin{document}

\begin{document}
\maketitle

\pagenumbering{roman}
\setcounter{page}{5}
%\begin{5}
%\end{5}
\pagenumbering{arabic}
\tableofcontents
\fi


\chapter{Plane Curves}
\label{PlaneCurvesChapter}


For a long time, plane curves were the only algebraic curves that were studied. Originally these were curves in the affine plane over the real numbers, but by about 1850 the complex projective plane was well understood, and curves in $\PP^2 = \PP^2_\CC$, corresponding to irreducible forms in 3 variables, were recognized as the natural objects of study. 

The work of Bernard Riemann dramatically changed the focus of the theory to branched coverings of   the ``Riemann Sphere'' ($\PP^1_\CC$). The Riemann-Roch theorem, in particular, gave information about the existence of meromorphic functions on such coverings, well beyond what could be done in the earlier theory. However, Riemann's work, depending as it did on the then-obscure ``Dirichlet principle'', was not universally accepted. In the 1860s Alfred Clebsch and, after the death of Clebsch  in 1872, Alexander Brill and Max Noether (Emmy Noether's father), undertook the ambitious program of redoing the Riemann-Roch theorem entirely in terms of plane curves. They went beyond Riemann in certain directions, too: the Brill-Noether Theorem treated in our Chapter~\ref{Brill-Noether chapter} was formulated by Brill and Noether, and ``proved'' by them through an unsupported general position assumption. 

A central difficulty in the Brill-Noether attempt on the Riemann-Roch theorem was that,
although any smooth curve can be embedded in $\PP^r$ for any $r \geq 3$, most curves cannot be embedded in the plane. 
However, as we saw in Section~\ref{good projections}, we can embed $C$ as a curve $ C \subset \PP^r$ in a higher-dimensional projective space and find a projection $\PP^r \to \PP^2$ that carries $C$ birationally onto its image $C_0$, called a plane model of $C$. The curve $C_0$ typically has singularities, and $C$ is the normalization of $C_0$. Brill and Noether wanted to prove the Riemann-Roch theorem for $C$ by formulating and proving a related theorem for $C_0$. In particular, they tried to characterize linear equivalence of divisors on $C$ in terms of certain ``clusters'' of points---we would say 0-dimensional subschemes---of $C_0$. 

To carry out this program, a key result is that if $D\subset C_0$ is contained in the intersection
$D'$ of $C_0$ and some other plane curve $C_0'$, then the difference 
$E := D'-D$ can be defined with properties such as that $D'-E = D$; Brill and Noether seem simply to have assumed that this is so. A bit later Frances Sowerby Macaulay proved that this is in fact possible  and also understood that it would not generally be possible if $D'$ were the intersection of three or more curves. 

Macaulay exploited this theory of residuation  to prove what he called the ``Generalized Riemann-Roch Theorem'' (now erroneously called the ``Cayley Bacharach Theorem''.) This early work of Macaulay led directly to his definitions of  ``perfection'' (a homogeneous ideal
$I  \subset S:= k[x_0, \dots, x_n]$ is perfect if $S/I$ is Cohen-Macaulay) and ``super-perfection'' (the case when $S/I$ is
Gorenstein). For all this, see~\cite{Eisenbud-Gray}.

In the first part of this Chapter we will use ideas that go back to Brill and Noether to find all effective divisors on $C$ equivalent on $C$ to a given (possibly) non-effective divisor. In Section~\ref{linear series on smooth plane curves} we treat the simple case of smooth plane curves, and in subsequent sections we treat the case of plane curves with nodes, and then plane curves with arbitrary singularities.

 
\section{Using plane models to compute linear series} \label{computing linear series}

In Proposition~\ref{nodal projection} we showed that any smooth projective curve $C$ can be linearly and birationally projected to a $C_0$ that has only nodes as singularities. In the first sections of this chapter we will show how to use such a \emph{plane model} to algorithmically determine the complete the linear series associated to any divisor on $C$.

In particular:
\begin{enumerate}
\item Given the equation $F(X,Y,Z)$ of a plane curve $C_0$, with normalization $C$, we will find a basis for $H^0(K_C)$; and
\item  Given, in addition, a divisor $D = \sum m_ip_i$ on $C$, we will describe the complete linear system $|D|$; that is, we will find all effective divisors $E$ on $C$ with $E \sim D$, or, equivalently, a basis for $H^0(\cO_C(D))$.
\end{enumerate}

Along the way we will see how to test when two given divisors $D$ and $E$ on $C$ are linearly equivalent; and whether a given divisor $D$ linearly equivalent to an effective divisor.

For clarity, we will pursue these goals in three stages: first when $C = C_0$ is a smooth plane curve (Section~ \ref{linear series on smooth plane curves}); next for nodal curves (Section~~\ref{nodal plane curves}), which in principle this covers all smooth curves; and finally for arbitrary reduced, irreducible plane curves (Section~\ref{arbitrary plane curves}): many times a curve $C$ is given to us as (the normalization of) a plane curve with singularities other than nodes, and while Proposition~\ref{nodal projection} assures us in principal that we can also realize $C$ as the normalization of a nodal plane curve, it is often easier to work with the given plane model. 

\section{Differentials on a smooth plane curve}\label{canonical series on smooth plane curves}

Let $C \subset \PP^2$  be a smooth plane curve, given as the zero locus of a homogeneous polynomial $F(X,Y,Z)$ of degree $d$. We'll introduce coordinates $x = X/Z$ and $y = Y/Z$ on the affine open subset $U \cong \AA^2$ given by $Z \neq 0$, and let $f(x,y) = F(x, y,1)$ be the inhomogeneous form of $F$, so that $\widetilde C = C \cap U$ is given as the zero locus $V(f) \subset  \AA^2$. 

Since an automorphism of $\PP^2$ can carry any line in the plane to the line at infinity, and any point on that line to the point $(0,1,0)$, we may assume:
\begin{enumerate}
\item The point $[0,1,0]$ (that is, the point at infinity in the vertical direction) does not lie on $C$; equivalently,  the projection $C \to \PP^1$ from $(0,1,0)$, which is given by $[X,Y,Z] \mapsto [X,Z]$ (or, in affine coordinates, $(x,y) \mapsto x$)  has degree $d$; and
\item The line $L$ at infinity given by $Z = 0$ intersects $C$ transversely in $d$ distinct points $p_1, \dots, p_d$; We write $H = p_1+ \dots +p_d$ for the corresponding divisor, the ``hyperplane section at infinity''.
\end{enumerate}

These conditions are not necessary: in Exercise~\ref{****} we will see how to do  without them.
 
We start toward our first goal by writing down a single rational 1-form on $C$: 
 take a regular 1-form on $\AA^2$, such as $dx$, and restrict/pull back to $C$. 

Though $dx$ is regular on $\widetilde C$, writing
$$
dx = d\frac{X}{Z} = \frac{X dZ+ Z dX}{Z^2}
$$
 shows that $dx$ has double poles at the points $p_1,\dots,p_d$ of the divisor $D := C \cap L$.
 
How do we get rid of the poles of $dx$? The extension of $\PP^2$ of a polynomial $h(x,y)$ of degree $d$ on
$\AA^2$ has a pole of order $d$ along the line $L$ at infinity. Thus if $h$ has degree at least 2 then $dx/h$ will have no poles at infinity. However, $h(x,y)$ will vanish at points of $C \cap U$, and this may create new poles of $dx/h$. Of course if $h$ vanishes only at  points of $C \cap U$ where $dx$ already has a zero, the zeroes of $h$ may cancel the zeroes of $dx$ rather than creating new poles.
 
 To avoid producing new poles in this way we may take
 $$
 h(x,y) = \frac{\partial f}{\partial y}(x,y).
 $$
 Note that on $\widetilde C$,
 $$
 df = \frac{\partial f}{\partial x}dx + \frac{\partial f}{\partial y}dy \equiv 0.
 $$
Because $C$ is smooth, the functions $\frac{\partial f}{\partial x}$ and $\frac{\partial f}{\partial y}$ have no common zeroes on $\widetilde C$. Moreover, the differentials $dx$ and $dy$ have no common zeros on $\widetilde C$. It follows that at for all $p \in \widetilde C$ we  have
$$
\ord_p(dx) = \ord_p(\frac{\partial f}{\partial y}), 
$$ 
and thus the quotient 
$$
\varphi_0 = \frac{dx}{\partial f/\partial y}
$$
is everywhere regular and nowhere 0 in $\widetilde C$.

The differential $dx$ has poles of order $2$ at the points $p_i$. The polynomial $\partial f/\partial y$, having degree $d-1$, has poles of order $d-1$. Thus $\varphi_0$ has zeroes of order $d-3$ at the points $p_i$; in other words, as divisors,
$$
(\varphi_0) = (d-3)D.
$$
In particular, if $d \geq 3$ then $\varphi_0$ is a global regular differential on $C$.

Moreover, we can  multiply $\varphi_0$ by any polynomial $e(x,y)$ 
  of degree $d-3$ or less without introducing poles, so that 
$$
e\varphi_0 = \frac{e(x,y)dx}{\partial f/\partial y}
$$ 
is likewise a global regular differential, for $g$  any polynomial of degree $\leq d-3$.

We have thus found a vector space of regular differentials, of dimension $\binom{d-1}{2}$. But at the same time, the degree of a differential like $\varphi_0$ is
$$
\deg((\varphi_0)) = (d-3)\deg(D) = d(d-3),
$$
so that the genus of $g(C)$ satisfies
$2g(C)-2 = d(d-3)$, whence
$$
\frac{d(d-3)}{2} + 1 = \binom{d-1}{2}.
$$
In other words, we have found all the global regular differentials on $C$! We have
$$
H^0(K_C) = \left\{ \frac{e(x,y)dx}{\partial f/\partial y} \mid \deg e \leq d-3\right\};
$$
or, equivalently, the space of regular differentials on $C$ has basis $\{\varphi_{i,j} \}_{i+j \leq d-3}$, where
$$
\varphi_{i,j} =  \frac{x^iy^jdx}{\partial f/\partial y}
$$

We could have achieved the same result by using the adjunction formula (see Section~\ref{Adjunction Formula}: we have
$$
K_C = (K_{\PP^2} \otimes \cO_{\PP^2}(d))|_C = \cO_C(d-3),
$$
and from the exact sequence
$$
0 \to \cO_{\PP^2}((d-3)-d) \rTo^f \cO_{\PP^2}(d-3) \to \cO_C(d-3)=K_C \to 0
$$
and the vanishing of $H^1(\cO_{\PP^2}(-3))$, we see that the map on global sections
$$
H^0(\cO_{\PP^2}(d-3)) \to H^0(K_C)
$$
is surjective. 


\section{Linear series on a smooth plane curves}\label{linear series on smooth plane curves}

Suppose that $D$ is any divisor on a smooth plane curve $C$, expressed as the difference of
two effective divisors, $D= D_0-D_\infty$. We would like to find all the \emph{effective} divisors linearly equivalent to $D$, that is, of the form
$D + (H/G)$, where $G, H$ are forms of the same degree $m$. Choose $m$ large enough so that
there is
 a form $G$ of degree $m$ that vanishes on $D_0$ plus some divisor $A$ (but not on all of $C$). If we can find a form $H$ of degree $m$ that vanishes on both $D_\infty$ and $A$, as well as a further divisor $D'$, but does not vanish identically on $C$, then
$$
D' = D + (H/G) = D_0- D_\infty - (D_0+A)+(D_\infty+A+D')
$$
as required. Since $m$ is already chosen, there may not exist any such polynomial; we shall see that in this case there is no effective divisor linearly equivalent to $D$. 

\begin{example}
Suppose that $C$ has degree 3, so that $C$ has genus 1. If we choose as origin on the curve $C$ a point $o$, then to add two points $p$ and $q \in C$ means to find the (unique) effective divisor of degree 1 linearly equivalent to $p + q - o$. In this situation, we can carry out the process described above with $m=1$: draw the line $L$ through the points $p$ and $q$, and let $r \in C$ be the remaining point of intersection of $L$ with $C$; then draw the line $M$ though the points $r$ and $o$, and let $s \in C$ be the remaining point of intersection of $L$ with $C$. This is the classical construction of the group law.
\end{example}

We claim that we find in this way all effective divisors $D' \sim D$. (This is a special case of the Completeness of the Adjoint series, Proposition~\ref{adjoint completeness}). In particular, this will show that if there is no polynomial $H$ as above, then there are no effective divisors equivalent to $D$.

Indeed, suppose $D'$ is any effective divisor with $D' \sim D$. Carrying out the first step of the process as before, we arrive at a divisor $A$ with 
$$
\cO_C(A+D_\infty+D') = \cO_C(A+D_\infty+D)  = \cO_C(m),
$$
that is, $A+D_\infty+D$ is linearly equivalent on $C$ to the divisor (H) defined by the vanishing of some form $H$ of degree $m$. This means that
$A+D_\infty+D  = (H') +(P/Q)$ for some forms $P,Q$ of equal degree, with $Q$ not vanishing identically on $C$. Since $A+D_\infty+D$ is effective,
the zeros of $Q$ on $C$ must be contained in the zeros of $H'P$ on $C$. Writing
$G$ for the generator of the ideal of $C$, and using the fact that
$G,Q$ is a regular sequence, and thus is unmixed (that is, has no embedded primary component), we see that $H'P = LG+HQ$ for some
forms $L,H$ of degree $\deg HP - \deg Q$. In particular, $H = (H'P- LG)/Q$ is a form
of degree $m$ defining $A+D_\infty+D$, as desired.

Alternatively, the fact that any regular sequence $G,Q$ defines an unmixed ideal is equivalent to the vanishing of the intermediate cohomology of all twists of the structure sheaf of $\PP^2$; and we can use this to achieve the same conclusion with fewer symbols, though more sophistication: from the exact sequence 
$$
0 \to \cO_{\PP^2}(m-d) \to \cO_{\PP^2}(m)  \to \cO_C(m) \to 0
$$
and the vanishing of $H^1(\cO_{\PP^2}(m-d))$, we again see that every global section of $ \cO_C(m)$ is the restriction to $C$ of a homogeneous polynomial of degree $m$ on $\PP^2$. 

Note that if, in the process described, it turns out there is no polynomial $H$ vanishing on  $A + D_\infty + D$ but not vanishing identically on $C$, that simply means that $|D| = \emptyset$; that is, $D$ is not linearly equivalent to any effective divisor. (It may not be obvious that the existence of such an $H$ is independent of the choice of $m$ or $G$, but the argument here shows that it is.)



\section{Differentials on a nodal plane curve}\label{canonical series on nodal plane curves}

As noted, smooth plane curves are very special among all curves. We now want to carry out the analyses above for curves with at most nodes as singularities. As we have seen in Section~\ref{nodal curves}, every smooth curve is the normalization of a nodal plane curve.


Given a smooth projective curve $C$ and a birational map of $C$ to $\PP^2$ with image a nodal curve $C_0$, our goal is as before:
\begin{enumerate}
\item to write out explicitly all global regular 1-forms on $C$; and
\item given a divisor $D$ on $C$, to  determine $|D|$; that is, find all effective divisors linearly equivalent to $D$.
\end{enumerate}

As before, we may choose homogeneous coordinates  $[X,Y,Z]$ on $\PP^2$ so that the curve $C_0$ intersects the line $L = V(Z)$ at infinity transversely at points $p_1,\dots,p_d$ other than $[0,1,0]$. In particular,  all the nodes of $C_0$ will lie in the affine plane $U = \PP^2 \setminus L$.
In addition, we can assume that  neither branch of $C_0$ at a node has vertical tangent. (These conditions serve only to keep the notation reasonably simple, and are satisfied by a general choice of coordinates.) Let the nodes of $C_0$ be $q_1,\dots,q_\delta$, with $r_i, s_i \in C$ lying over $q_i$; we'll denote by $\Delta$ the divisor $\sum r_i + \sum s_i$ on $C$.

Let $F(X,Y,Z)$ be the homogeneous polynomial of degree $d$ defining the curve $C_0$, and let $f(x,y) = F(x,y,1)$ be the defining equation of the affine part $C_0 \cap U$ of $C_0$. Let $\nu: C\to C_0$ be the normalization map. We start by considering the rational differential $\nu^*(dx)$ on $C$. In the smooth case, we saw that this differential was regular and nonzero in the finite plane, but had poles of order 2 at the points of $C \cap L$; this followed from the equation
$$
 df = \frac{\partial f}{\partial x}dx + \frac{\partial f}{\partial y}dy \equiv 0.
 $$
and the fact that $\frac{\partial f}{\partial x}$ and $\frac{\partial f}{\partial y}$ have no common zeroes on $C_0$. But now $\frac{\partial f}{\partial x}$ and $\frac{\partial f}{\partial y}$ \emph{do} have common zeroes; specifically, the pullbacks $\nu^*(\frac{\partial f}{\partial x})$ and $\nu^*(\frac{\partial f}{\partial y})$ have simple zeroes at the points $r_i$ and $s_i$. We conclude, accordingly, that the differential $\nu^*dx$ has double poles at the points $p_i$, 
and, proceeding as before, we see that for a polynomial $e(x,y)$ of degree $\leq d-3$, the differential
$$
\nu^*( \frac{e(x,y)dx}{\partial f/\partial y})
$$
will be regular except for simple poles at the points $r_i$ and $s_i$.

We can get rid of these poles by requiring that $e$ vanishes at the points $q_i$. We say in this case that $e$ (and the curve defined by $e$) \emph{satisfies the conditions of adjunction}. (In Section~\ref{adjoint ideal}, we'll describe the  conditions of adjunction associated to an arbitrary singularity.) In any event, we see that
$$
 \left\{ \nu^* \frac{e(x,y)dx}{\partial f/\partial y} \mid \deg e \leq d-3 \text{ and } e(q_i) = 0 \; \forall i \right\} \subset H^0(K_C).
$$
Now, in the smooth case, we were able to compare dimensions to conclude that this inclusion was indeed an equality. We can do the same thing here: to begin with, we have seen that the  rational 1-form $\varphi = \nu^*(\frac{dx}{\partial f/\partial y})$ has zeroes of order $d-3$ at the points $p_1,\dots,p_d$ and simple poles at the points $r_i$ and $s_i$ and is otherwise regular and nonzero; in other words, if we set $H = p_1+\dots + p_d$, the divisor
$$
(\varphi) = (d-3)H - \Delta.
$$
In particular, we see that
$$
\deg((\varphi)) = d(d-3) - 2\delta
$$
and correspondingly
$$
g(C) = \binom{d-1}{2} - \delta;
$$
this is called the genus formula for plane curves with nodes.

On the other hand, the space of polynomials $e$ of degree $\leq d-3$ vanishing at the points $q_i$ has dimension at least $ \binom{d-1}{2} - \delta$; we conclude from this that indeed
$$
H^0(K_C) =  \left\{ \nu^* \frac{e(x,y)dx}{\partial f/\partial y} \mid \deg e \leq d-3 \text{ and } e(q_i) = 0 \; \forall i \right\}.
$$
This argument would give another proof of Lemma~\ref{adjoint independent}.


\section{Linear series on a nodal plane curve}\label{linear series on nodal plane curves}

Given a divisor $D$ on $C$, we next compute the complete linear series $|D|$, using the birational map $\nu$ from 
$C$ to the nodal curve $C_0$.

We'll do this first in the case where $D = D_0-D_\infty$ is the difference of two effective divisors whose support is disjoint from the support $\{r_i, s_i\}$ of $\Delta$; the general case is only notationally more complicated. To start, we find an integer $m$ and a polynomial $G$ vanishing on the divisor $D_0$ \emph{and at the nodes $q_1,\dots,q_\delta$ of $C_0$}, but not vanishing identically on $C_0$. We can then write the zero locus of $G$ pulled back to $C$ as
$$
(\nu^*G) = D_0 + \Delta + A,
$$
as before. Once more, 
 for simplicity, let's assume that the support of $A$ is disjoint from the support of $\Delta$; this means 
  that the curve $V(G)$ is smooth at the points $q_i$ and is not tangent to either of the branches of $C_0$ there (this can certainly be done if we take $m$ large).

Next, we find polynomials $H$ of the same degree $m$, vanishing at $A+D_\infty$ and at the points $q_i$  but not on all of $C_0$. Let $D'$ be the divisor cut on $C$ by $H$ residual to $D_0 + \Delta + A$; that is, we write
$$
(\nu^*H) = D_0 + \Delta + A + D'.
$$
Finally, since $\nu^*(G/H)$ is a rational function on $C$, we see that 
$$
D_0 + \Delta + A = (\nu^*H) \sim (\nu^*G) = D_0 + \Delta + A + D',
$$
and we conclude that $D'$ is an effective divisor linearly equivalent to $D$ on $C$.

To prove that we get \emph{all} divisors $D'$ in this way we will work on the surface $S$ obtained by blowing up
$\PP^2$ at the points $q_i$. 

The following result, known classically as \emph{completeness of the adjoint series}, is what we need:

\begin{proposition}\label{adjoint completeness}
If $C_0 \subset \PP^2$ is a nodal plane curve and $\nu : C \to C_0$ its normalization $C$ then for any $m$ the linear series cut on $C$ by plane curves of degree $m$ passing through the nodes is complete.
\end{proposition}

The  meaning of this statement is that we get a complete linear series on $C$ by taking pullbacks of forms of degree $m$ on $\PP^2$
that vanish on the nodes, and then subtracting the divisor $\Delta$ of preimages of the nodes from the pullbacks.

\begin{proof}
To prove Proposition~\ref{adjoint completeness}, it will be helpful to introduce another surface: the blow-up $\pi : S \to \PP^2$ of $\PP^2$ at the points $q_i$. The proper transform on $C_0 \subset \PP^2$ in $S$ is the normalization of $C_0$, which we will again call $C$.

There are two divisor classes on $S$ that will come up in our analysis: the pullback of the class of a line in $\PP^2$, which we'll denote $H$; and the sum of the exceptional divisors, which we'll call $E$. We write $h= H\cap C$ and $e = \sum (p_i+q_i) = E\cap C$ for the corresponding divisors on $C$. In these terms, we have
$$
C \sim dH - 2E \quad \text{and} \quad K_S \sim -3H + E
$$
(the first follows from the fact that $C_0$ has multiplicity 2 at each of the points $q_i$, the second from considering the pullback to $S$ of a rational 2-form on $\PP^2$; see Theorem~\ref{divisor classes on blowup} for more details.)

 If $A$ is a curve in $\PP^2$ of degree $m$ passing through the points $q_i$, we can associate to it the effective divisor $\pi^*A - E$; this gives us an isomorphism
$$
H^0(\cI_{\{q_1,\dots,q_\delta\}/\PP^2}(m)) \cong H^0(\cO_S(mH-E)).
$$
In these terms we can describe the linear series cut on $C$ by plane curves of degree $m$ passing through the nodes of $C_0$ as the image of the map
$$
H^0(\cO_S(mH-E)) \to H^0(\cO_C(mH-E)),
$$
and the proposition amounts to the assertion that this map is surjective.

From the long exact cohomology sequence associated to the exact sequence of sheaves
$$
0 \to \cO_S(mH-E-C) = \cO_S((m-d)H + E)  \to \cO_S(mH-E) \to \cO_C(mH-E) \to 0,
$$
 we see that it will suffice to establish $H^1(\cO_S((m-d)H + E)) = 0$. To do this, we apply Serre duality on $S$, which says that $H^1(\cL) \cong H^1(K_S\otimes \cL^{-1})^*$. In this instance it 
tells us that
$$
H^1(\cO_S((m-d)H + E)) \cong H^1(\cO_S((d-m-3)H))^*.
$$
Now, the line bundle $\cO_S((d-m-3)H)$ is 
 the pullback to $S$ of the bundle $\cO_{\PP^2}(d-m-3)$, which has vanishing $H^1$. The following lemma completes the proof of the proposition:
\begin{lemma}
Let $X$ be a smooth projective surface, and $\pi : Y \to X$ a blow-up. If $\cL$ is any line bundle on $X$, then
$$
H^1(Y, \pi^*\cL) = H^1(X, \cL).
$$
\end{lemma}
The lemma follows by applying the Leray spectral sequence, which relates the cohomology of $\cL$ on $Y$ to the cohomology of the direct image $\pi_*\pi^*\cL$ (Leray is particularly simple in this setting, since all higher direct images are 0), and the observation that $\pi_*\pi^*\cL \cong \cL$.
\end{proof}

\section{Linear series on the normalization of an arbitrary plane curve} \label{arbitrary plane curves}

\begin{proposition}\label{effect of blowup on genus}
 Let $C$ be a curve on a smooth surface $S$, let $\nu : S' \to S$ be the blowup of $S$ at $p$. If $C'$ is the strict transform of $C$, then
 $$
 p_a(C') = p_a(C) -{m\choose 2},
 $$
 where $m$ is the multiplicity of $p\in C$.
\end{proposition}
\begin{proof}
This follows from comparing the adjunction formulas on $S$ and $S'$. To start, we have
$$
p_a(C) = \frac{C^2 + K_S\cdot C}{2} + 1.
$$
Now, on $S'$ let $E$ be the divisor class of the exceptional divisor. As we've seen,
$$
K_{S'} = \nu^*K_S + E,
$$
while the class of $C'$ is given by
$$
C' \sim \nu^*C - mE.
$$
It follows that
$$
(C')^2 = C^2 + m^2E^2 = C^2 - m^2 \quad \text{and} \quad K_{S'}\cdot C' = K_S\cdot C + m.
$$
Thus, applying adjunction on $S'$, we find
$$
p_a(C') = \frac{{C'}^2 + K_{S'}\cdot C'}{2} + 1 = \frac{C^2 + K_S\cdot C - m(m-1)}{2} + 1 = p_a(C) -{m\choose 2}
$$
as stated.
\end{proof}


Suppose again that $C_0 \subset \PP^2$ is reduced and irreducible, but with arbitrary singularities, and again let $\nu : C \to C_0$ be its normalization. Again we want to find the regular differentials on $C$ and the effective divisors on $C$ linearly equivalent to a given $D \in \Div(C)$. Again we take $\widetilde C_0$ to be the intersection of $C_0$ with the open set $z \neq 0$, that is, with $\AA^2\subset \PP^2$, and assume that
all the singular points of $C_0$ are contained in $\widetilde C_0$.

To mimic the analysis in the nodal case we need to generalize what we called the ``conditions of adjunction." In the nodal case, these were the conditions that a curve $e(x,y)=0 \subset \AA^2$ pass through the nodes of $C_0$, that is, locally, $e$ was contained in the maximal ideal
at each node. For an arbitrary singular point, the ideal that will play the role of the maximal ideal is called the \emph{adjoint ideal} of the singularity. It is easy to compute in simple cases, and has a nice description in terms of $\nu$ for arbitrary plane curves that will be given in Theorem~\ref{general adjoint}.

We first turn to the canonical series; that is, we wish to find expressions for all the regular differential forms on $C$. In the case of smooth and nodal curves we found we could express every regular differential form
on $C$ as a rational function times the pullback of the differential form written as $dx$ on $\AA^2$. We justified this
by counting---that is, by showing that we could get a $g(C)$-dimensional space of regular differentials in this way. The justification for 
this procedure for a reduced, irreducible curve with arbitrary singularities relies on the fact that $x$, as a function on $\widetilde C_0$
is a \emph{finite} map to $\AA^1$, and thus the field of rational functions $\kappa(C) = \kappa(\widetilde C_0)$ is a finite
separable extension of $\CC(x)$. By Theorem~\cite[****]{Eisenbud1995} that the module of differentials 
$\omega_{\kappa(C)/\CC}$ is generated over $\kappa(\widetilde C_0)$ by $dx$; that is, every differential can be expressed as a rational function
times $dx$. 

We focus for now on one singular point $q \in C_0$, with points $r_1,\dots,r_k \in C$ lying over $q$, and we wish to describe the conditions for a rational differential $e\varphi_0$ on $C$ to be regular, where
$$
\varphi_0 = \nu^* \frac{dx}{\partial f/\partial y},
$$
and $f$ is the defining equation of $C_0$ in $\AA^2$. As in the case of a smooth curve $C$, the differential
form $\varphi_0$ has a zero of order $d-3$ at each of the points of $C$ lying over the intersection of $C_0$ with the line  $L = \PP^2\setminus \AA^2$
at infinity. Thus we must must choose $e$ to be a function with poles on $C_0$ of order $\leq d-3$ on $L\cap C_0$ and no poles in $\widetilde C$. Any such function is represented (modulo $(f)$) by 
a polynomial $e$ in $x,y$ which must have degree $\leq d-3$, since otherwise $edx$ would have poles at infinity. In addition, $\varphi_0$ will have poles at the points $r_i$,
so $e$ must satisfy some ``conditions of adjunction," represented by belonging to an ideal---the adjoint ideal---at each singular point.

Let $m_i$ be the order of the pole of $\varphi_0$ at $r_i$. We can define the \emph{adjoint ideal} of $C_0$ at $q$ to be the ideal
$$
\mathfrak A_{q} = \left\{ e \in \cO_{\PP^2, q} \mid \ord_{r_i}(\nu^*e) \geq m_i \; \forall i \right\}
$$

In other words, $\mathfrak A_q$ is the ideal of functions $e$ such that $\nu^* \frac{gdx}{\partial f/\partial y}$ is regular at all the points $r_i$. We define the \emph{adjoint ideal $\mathfrak A_{C/C_0}$ of $C_0$} to be the intersection of $A_q$ over all singular points  $q \in C_0$; the \emph{adjoint series of degree $m$} is then the linear series $H^0(\mathfrak A_{C/C_0}(m))$. 

We can see, by an extension of the arguments above,  that every global regular 1-form on the curve $C$ is of the form 
$$
e(x,y) \varphi_0,
$$
with $e$ a polynomial of degree $\leq d-3$ in the adjoint ideal $\fA_{C/C_0}$; and more generally, the adjoint series of degree $m$ is complete for every $m$. Thus  we can compute complete linear series as in the nodal case, simply replacing
the divisor $\Delta = \sum q_i$ with the divisor determined by the adjoint ideal. We omit the details, which involve replacing
the surface $S$ with a repeated blowup of the plane on which the strict transform of $C_0$ is the normalization $C$.

In simple cases, the adjoint ideal is not difficult to compute. 

\subsection{Adjoint and deficiency}\label{adjoint ideal}

We begin as usual with examples:


\begin{example}[nodes and cusps]
We have already seen that in case $q$ is a node of $C_0$, there are two points of $C$ lying over it, and the multiplicities of $\varphi_0$ at these two points are $m_1=m_2=1$; the adjoint ideal is thus 
 the maximal ideal $\cI_q$ at $q$. In the case of a cusp, for example the zero locus of $y^2-x^3$, there is only one point $r=r_1$ of $C$ lying over $q$, and the differential $\varphi_0$ vanishes to order $m_1=2$; since the pullback to $C$ of any polynomial $g$ vanishing at $q$ will vanish to order at least two at $r$, the adjoint ideal is again the maximal ideal at $q$.
\end{example}

\begin{example}[tacnodes]
Next, consider the case of a \emph{tacnode}; that is, a plane-curve singularity with two smooth branches simply tangent to one another, such as the zero locus of $y^2-x^4$. In this case there are again two points of $C$ lying over $q$, and a simple calculation shows that $m_1=m_2=2$. The adjoint ideal is thus the ideal of functions vanishing at $q$ and having derivative 0 in the direction of the common tangent line to the branches.
\end{example}

\begin{example}[ordinary triple points]
In the case of an ordinary triple point of a plane curve---three smooth branches pairwise transverse to one another---there are three points of $C$ lying over $q$, and we have $m_1=m_2=m_3= 2$; the adjoint ideal is correspondingly 
 the square of the maximal ideal at $q$. 
\end{example}


In the next chapter we shall define the canonical (or dualizing) sheaf of a singular curve $C_0$ in 
general and show that if $\nu: C\to C_0$ is the normalization of a reduced, connected
projective curve $C$, then $\nu_*(\omega_C)$ is contained in $\omega_{C_0}$. We can
then define the \emph{adjoint ideal} of $C_0$ to be 
$$
\fA_{C/C_0} :=\ann_{\sO_{C_0}}\frac{\omega_{C_0}}{\nu_* \omega_C}.
$$
We will show that this is always equal to the \emph{conductor ideal}
$$
\mathfrak f_{C/C_0}: = \ann_{\sO_{C_0}}\frac{\nu_* \sO_C}{\sO_{C_0}}	
$$
Furthermore, if $C_0$ is a plane curve, then 
$$
\delta(C_0) = \length \frac{\nu_* \sO_C}{\sO_{C_0}} = \length \frac {\sO_{C_0}}{\mathfrak f_{C/C_0}}.
$$

For an ordinary node the value of the $\delta$-invariant is 1, and in general it may be regarded as the ``number of nodes equivalent to the singularities of $C_0$'' as
explained in Chapter~\ref{RiemannRochChapter}. The notation $\mathfrak f$ for the conductor comes from the german term \emph{F\"uhrer}. The property in the last statement of the
Theorem  was first noted (for plane curves) in Daniel Gorenstein's thesis under Oscar Zariski\footnote{Gorenstein is better remembered for his work on the classification of finite simple groups.}. This is the
reason why Grothendieck gave the name ``Gorenstein" to Cohen-Macaulay rings that have cyclic canonical 
modules--see~\cite{Bass}. 
%
%In the previous chapter we used the adjoint ideal as a condition on a polynomial
%$e$ to multiply differential forms on $C$ for $e\phi$ to belong to $\omega_C$ is always sufficient, but may be necessary only
%when $\omega_{C_0}$ is cyclic---which is equivalent to the statement that $C_0$ is locally
%Gorenstein. This condition is always satisfied for singularities of curves in the plane.
%See Example~\ref{nongorenstein} for a singularity that is not locally planar, and behaves differently.
%
%
%In the Theorem~\ref{general adjoint} we shall prove that the adjoint ideal can be computed
%as the \emph{conductor} ideal, the annihilator in $\sO_{C_0}$ of the quotient
%$\nu_*\sO_C/\sO_{C_0}$.
%
%\subsection{The ``deficiency" of a curve}

Arthur Cayley observed the importance for a nodal plane curve of degree $n$ of the number
that we would now call the arithmetic genus. In \cite{Cayley2} He wrote:
\begin{quotation}
 Before going further, it will be convenient to to introduce the term ``Deficiency," viz., a curve of the order $n$ with
 $\frac{1}{2}(n-1)(n-2)-D$ dps [double points] is said to have a deficiency $=D$: the foregoing theorem is that for curves with a deficiency $=0$, the coordinates are expressible rationally in terms of a parameter $\theta$.
  \end{quotation}
 
Riemann-Roch theorem and the modern definitions of the genus of a curve, algebraic geometers had
defined an invariant called the \emph{deficiency} that is equal to the genus When Clebsch, Brill and Noether tried to 
compute what amounts to the geometric genus of a plane curve, this was the idea they used, and we can shed some light
on the meaning of the adjoint ideal by examining the old definition.

The starting point may have been the observation that, while any $2n$ points on a plane conic curve $C$ form the intersection of $C$ with a  curve of degree $n$, the analogous statement---that any $3n$ points on a smooth plane cubic comprise the intersection of $C$ with a plane curve of degree $n$---is not true. This is a reflection of the fact that there are fewer rational functions on the cubic than on the conic: if $D = p_1+\dots + p_d$ is a $d$-tuple of points on a conic curve $C$, we have
$$
h^0(\cO_C(D)) = d+1,
$$ 
and so there is a rational function with poles just along $D$ and vanishing at any given $d$-tuple of points; by contrast, a plane cubic $C$ has genus 1, and so by Riemann-Roch for such a divisor we have
$$
h^0(\cO_C(D)) = d.
$$
This lack of rational functions was viewed as a defect of the curve, and called its \emph{deficiency} (and typically denoted $p$, rather than $g$). .

In general, suppose that $C$ is a smooth plane curve $C \subset \PP^2$ of degree $n$, and that $D = p_1+\dots + p_d$ is a divisor of degree $d$. To find the complete linear system $|D|$ by the method of Section~\ref{smooth plane curves} we start by choosing a plane curve $B$ of degree $m$ containing $D$ but not containing $C$; again for simplicity of notation we'll assume $m \geq n$. The intersection $B \cap C$ will then consist of $D$ plus another divisor $E$ on $C$, of degree $mn-d$; and what we saw is that the complete linear series $|D|$ will be cut on $C$ by plane curves of degree $m$ containing $E$, modulo those containing $C$.

We can easily estimate the size of this linear series: If the $(mn-d)$ points of $E$ impose independent conditions on curves of degree $m$, which will be the case when $d$ is large compared with $n$, the space of curves of degree $m$ containing $E$ will have dimension
$$
h^0(\cI_{E/\PP^2}(m)) = \binom{m+2}{2} - (mn-d);
$$ 
and the subspace of curves of degree $m$ containing all of $C$ has dimension
$$
h^0(\cO_{\PP^2}(m-n)) = \binom{m-n+2}{2}
$$
so the dimension of $H^0(\cO_C(D))$ is
$$
h^0(\cO_C(D)) \geq \binom{m+2}{2} - (mn-d) -  \binom{m-n+2}{2} = d + 1 - \binom{n-1}{2}
$$
and the inequality will be an equality if $d$ is large enough compared to $n$ (in fact,  when $d> 2\binom{n-1}{2}-2 = n^2-3n$,
as we know from the Riemann-Roch theorem plus the genus formula).  That is, a smooth plane curve of degree $n$ has deficiency $\binom{n-1}{2}$.

Now suppose the curve $C$ maps birationally to a curve $C_0$ which has a node at a point $p$; let $D$ again be a divisor of degree $d$, whose support we assume does not contain $p$. If we take our curve $B$ of degree $m$ to contain $D$ and also pass through the point $p$, then the residual intersection $E$ to $D+\nu^{-1}(p)$ on $C$ will consist of $mn-d-2$ points rather than $mn-d$. To construct the complete linear system $|D|$, we take curves of degree $m$ passing through this residual divisor $E$ and passing through $p$.

Here we have a sort of ``two-for-one" deal: the curves of degree $m$ that cut out the linear series $|D|$ have to satisfy the one additional condition of passing through the node $p$; but the number of points of $E$ that these curves have to contain has been reduced by 2. The expected dimension of the linear series associated to $DS$ is thus
$$
h^0(\cO_C(D)) = \binom{m+2}{2} - (mn-d-2) -  \binom{m-n+2}{2} = d  - \binom{n-1}{2}.
$$
so the deficiency of the curve is $\binom{n-1}{2}-1$, or one less. More generally, if $C$ has $\delta$ nodes, we can get this two-for-one deal $\delta$ times, resulting in a deficiency of $\binom{n-1}{2} - \delta$, giving us an early version of the formula for the  geometric genus of a plane curves with nodes.

What if $C$ has an ordinary triple point $p$? In this case, we can get a ``three-for-one" deal by requiring the curve $B$ of degree $m$ to pass through $p$. In this case the residual divisor $E$ will have degree 3 less; again, the curves of degree $m$ that cut out the linear series $|D|$ only have to satisfy the one additional condition of passing through $p$, so the apparent deficiency will be 2 less than for a smooth plane curve of degree $n$.

But we can do even better! If we require our initial curve of degree $m$ to be double at $p$, that reduces the degree of the residual divisor $E$  by 6. Of course, now the curves $B$ of degree $m$ containing $E$ that cut our the linear system $|D|$  will have to be double at $p$ as well, but this is only 3 linear conditions, the vanishing at $p$ along with the vanishing of the first derivatives. In effect, we have gotten a ``six-for-three" deal, suggesting (correctly, as we know) that the presence of a triple point should reduce the deficiency by 3. 

In general---for singularities beyond nodes or triple points---the adjoint ideal can be characterized as the largest ideal that gives the best ``deal" in this sense. 


\section{Exercises}

\begin{exercise}\label{gonality of smooth plane curve}
Let $C$ be a smooth plane curve of degree $d$. Show that $C$ admits a one-parameter family of maps $C \to \PP^1$ of degree $d-1$. Using the Riemann-Roch Theorem, show that $C$ does not admit a map $C \to \PP^1$ of degree $d-2$ or less.
\end{exercise}

In Exercise~\ref{gonality of smooth plane curve}, we saw how to use the description of the canonical series on a smooth plane curve to determine its gonality. Now that we have an analogous description of the canonical series on (the normalization of) a nodal plane curve, we can deduce a similar statement about the gonality of such a curve. Here are the first two cases: 

\begin{exercise}
Let $C_0$ be a plane curve of degree $d\geq 4$ with one node and no other singularities, and let $C$ be its normalization. Show that $C$ admits a unique map $C \to \PP^1$ of degree $d-2$, but does not admit a map $C \to \PP^1$ of degree $d-3$ or less.
\end{exercise}

\begin{exercise}
Let $C_0$ be a plane curve of degree $d\geq 5$ with two nodes and no other singularities, and let $C$ be its normalization. Show that $C$ admits two maps $C \to \PP^1$ of degree $d-2$, but does not admit a map $C \to \PP^1$ of degree $d-3$ or less.
\end{exercise}

\begin{exercise}
Generalizing some of the examples above, show that if a nodal plane curve of degree $d$ has $\delta\leq d+3$ nodes,
then its gonality is $d-2$, and moreover every $g^1_d$ on the curve is given by projection from one of the nodes.
You should make use of the following result from~\cite[p. 302]{MR1376653}:
\begin{proposition}
 A set of $n \leq 2d+ 2$ distinct
points fails to impose independent conditions on curves of degree
$d$ if and only if either $d + 2$ of the points  are collinear or $n = 2d + 2$ and all the points lie
on a conic.
\end{proposition} 
Hint: a $g^1_{d-2}$ is a set of points that, together with the nodes, impose dependent conditions on forms of degree $d-3$.
\end{exercise}

\begin{exercise}
Find the adjoint ideals of the following plane curve singularities:
\begin{enumerate}
\item a \emph{triple tacnode}: three smooth branches, pairwise simply tangent
\item a triple point with an infinitely near double point: three smooth branches, two of which are simply tangent, with the third transverse to both
\item a unibranch triple point, such as the zero locus of $y^3-x^4$
\end{enumerate}
\end{exercise}

Here is a  general description in case the individual branches of $C_0$ at $p$ are each smooth:

\begin{exercise}
Let $\nu : C \to C_0$ be the normalization of a plane curve $C_0$ and $p \in C_0$ a singular point. Denote the branches of $C_0$ at $p$ by $B_1,\dots,B_k$, and let $r_i$ be the point in $B_i$ lying over $p$. If the individual branches $B_i$ of $C_0$ at $p$ are each smooth, and we set
$$
m_i = \sum_{j \neq i} mult_p(B_i \cdot B_j)
$$
then the adjoint ideal of $C_0$ at $p$ is the ideal of functions $g$ such that $ord_{r_i}(\nu^*g) \geq m_i$.
(Hint: The computation can be done locally analytically. Let $R = \widehat{\sO_{C_0, p}}$ be the completion of the local ring
of $C_0$ at $r_i$. The integral closure is then the product of rings $R_i = \widehat{\sO_{B_i}} \cong k[[t_i]]$,
with $R_i = R/P_i$ as $P_j$ runs over the minimal primes of $R$. The multiplicity
$m_i$ is the colength of the ideal $\sum_{j\neq i}P_j \subset R$.
)
\end{exercise}

\begin{exercise}
Continuing the example of a plane curve with a triple point $p$, show that requiring our curve of degree $m$ to be triple at $p$ gives us a ``nine-for-six" deal---the same net effect as when we required our curve of degree $m$ to be double. (In general, an ideal contained in the adjoint ideal will work as well; the adjoint ideal is characterized as the largest ideal that works.)
\end{exercise}

\begin{exercise}
Now let $C \subset \PP^2$ be a plane curve with a tacnode at $p$. Show that by taking our curve of degree $m$ to pass through $p$ and be tangent to the branches of $C$ at $p$ we get a four-for-two deal; and that this is the best we can do, so that the effect of a tacnode is to drop the deficiency by 2.
\end{exercise}

\fix{Joe may add a sequence of exercises on "infinitely near points"; culminate (?) with the theorem that $\delta$ can be computed in general as a sum of multiplicities of iterated blowups.}


%footer for separate chapter files

\ifx\whole\undefined
%\makeatletter\def\@biblabel#1{#1]}\makeatother
\makeatletter \def\@biblabel#1{\ignorespaces} \makeatother
\bibliographystyle{msribib}
\bibliography{slag}

%%%% EXPLANATIONS:

% f and n
% some authors have all works collected at the end

\begingroup
%\catcode`\^\active
%if ^ is followed by 
% 1:  print f, gobble the following ^ and the next character
% 0:  print n, gobble the following ^
% any other letter: normal subscript
%\makeatletter
%\def^#1{\ifx1#1f\expandafter\@gobbletwo\else
%        \ifx0#1n\expandafter\expandafter\expandafter\@gobble
%        \else\sp{#1}\fi\fi}
%\makeatother
\let\moreadhoc\relax
\def\indexintro{%An author's cited works appear at the end of the
%author's entry; for conventions
%see the List of Citations on page~\pageref{loc}.  
%\smallbreak\noindent
%The letter `f' after a page number indicates a figure, `n' a footnote.
}
\printindex[gen]
\endgroup % end of \catcode
%requires makeindex
\end{document}
\else
\fi
