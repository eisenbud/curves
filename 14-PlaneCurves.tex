%header and footer for separate chapter files

\ifx\whole\undefined
\documentclass[12pt, leqno]{book}
\usepackage{graphicx}
\input style-for-curves.sty
\usepackage{hyperref}
\usepackage{showkeys} %This shows the labels.
%\usepackage{SLAG,msribib,local}
%\usepackage{amsmath,amscd,amsthm,amssymb,amsxtra,latexsym,epsfig,epic,graphics}
%\usepackage[matrix,arrow,curve]{xy}
%\usepackage{graphicx}
%\usepackage{diagrams}
%
%%\usepackage{amsrefs}
%%%%%%%%%%%%%%%%%%%%%%%%%%%%%%%%%%%%%%%%%%
%%\textwidth16cm
%%\textheight20cm
%%\topmargin-2cm
%\oddsidemargin.8cm
%\evensidemargin1cm
%
%%%%%%Definitions
%\input preamble.tex
%\input style-for-curves.sty
%\def\TU{{\bf U}}
%\def\AA{{\mathbb A}}
%\def\BB{{\mathbb B}}
%\def\CC{{\mathbb C}}
%\def\QQ{{\mathbb Q}}
%\def\RR{{\mathbb R}}
%\def\facet{{\bf facet}}
%\def\image{{\rm image}}
%\def\cE{{\cal E}}
%\def\cF{{\cal F}}
%\def\cG{{\cal G}}
%\def\cH{{\cal H}}
%\def\cHom{{{\cal H}om}}
%\def\h{{\rm h}}
% \def\bs{{Boij-S\"oderberg{} }}
%
%\makeatletter
%\def\Ddots{\mathinner{\mkern1mu\raise\p@
%\vbox{\kern7\p@\hbox{.}}\mkern2mu
%\raise4\p@\hbox{.}\mkern2mu\raise7\p@\hbox{.}\mkern1mu}}
%\makeatother

%%
%\pagestyle{myheadings}

%\input style-for-curves.tex
%\documentclass{cambridge7A}
%\usepackage{hatcher_revised} 
%\usepackage{3264}
   
\errorcontextlines=1000
%\usepackage{makeidx}
\let\see\relax
\usepackage{makeidx}
\makeindex
% \index{word} in the doc; \index{variety!algebraic} gives variety, algebraic
% PUT a % after each \index{***}

\overfullrule=5pt
\catcode`\@\active
\def@{\mskip1.5mu} %produce a small space in math with an @

\title{Personalities of Curves}
\author{\copyright David Eisenbud and Joe Harris}
%%\includeonly{%
%0-intro,01-ChowRingDogma,02-FirstExamples,03-Grassmannians,04-GeneralGrassmannians
%,05-VectorBundlesAndChernClasses,06-LinesOnHypersurfaces,07-SingularElementsOfLinearSeries,
%08-ParameterSpaces,
%bib
%}

\date{\today}
%%\date{}
%\title{Curves}
%%{\normalsize ***Preliminary Version***}} 
%\author{David Eisenbud and Joe Harris }
%
%\begin{document}

\begin{document}
\maketitle

\pagenumbering{roman}
\setcounter{page}{5}
%\begin{5}
%\end{5}
\pagenumbering{arabic}
\tableofcontents
\fi


\def\adj{{\mathfrak F}}
\chapter{Plane Curves}
\label{PlaneCurvesChapter}


For a long time, plane curves were the only algebraic curves that were studied. Originally these were curves in the affine plane over the real numbers, but by the second half of the 19th century the complex projective plane was well understood, and curves in $\PP^2 = \PP^2_\CC$, corresponding to irreducible forms in 3 variables, were recognized as the natural objects of study---see the historical appendix to this book for more details.

The work of Bernhard Riemann dramatically changed the focus of the theory to branched coverings of   the ``Riemann Sphere'' ($\PP^1_\CC$). The Riemann-Roch theorem, in particular, gave information about the existence of meromorphic functions on such coverings, well beyond what could be done in the earlier theory. However, Riemann's work, depending as it did on the then-obscure ``Dirichlet principle'', was not universally accepted. In the 1860s Alfred Clebsch and, after the death of Clebsch  in 1872, Alexander Brill and Max Noether (Emmy Noether's father), undertook the ambitious program of redoing the Riemann-Roch theorem entirely in terms of plane curves. They went beyond Riemann in certain directions, too: the Brill-Noether Theorem treated in our Chapter~\ref{Brill-Noether} was formulated by Brill and Noether, and ``proved'' by them through an unsupported general position assumption. 


A central difficulty in the Brill-Noether attempt on the Riemann-Roch theorem was that,
although any smooth curve can be embedded in $\PP^r$ for any $r \geq 3$, most curves cannot be embedded in the plane. 
However, as is shown in Section~\ref{good projections}, we can embed $C$ as a curve $ C \subset \PP^r$ in a higher-dimensional projective space and find a projection $\PP^r \to \PP^2$ that carries $C$ birationally onto its image $C_0$, called a plane model of $C$. The curve $C_0$ typically has singularities, and $C$ is the normalization of $C_0$. Brill and Noether wanted to prove the Riemann-Roch theorem for $C$ by formulating and proving a related theorem for $C_0$. In particular, they tried to characterize linear equivalence of divisors on $C$ in terms of certain ``clusters'' of points---we would say 0-dimensional subschemes---of $C_0$. 

To carry out this program, a key step was to show that if $D\subset C_0$ is contained in the intersection
$D'$ of $C_0$ and some other plane curve $C_0'$, then the difference 
$E := D'-D$ can be defined with properties such as that $D'-E = D$; Brill and Noether seem simply to have assumed that this is so. A bit later Frances Sowerby Macaulay proved that this is in fact possible  and also understood that it would not generally be possible if $D'$ were the intersection of three or more curves.\footnote{In modern terms, the intersection of two curves is Gorenstein; the intersection of 3 is generally not}

Macaulay exploited this theory of residuation  to prove what he called the ``Generalized Riemann-Roch Theorem'' (now widely known as the ``Cayley Bacharach Theorem'', after some precursors). This early work of Macaulay led directly to his definitions of  ``perfection'' (a homogeneous ideal
$I  \subset S:= k[x_0, \dots, x_n]$ is perfect if $S/I$ is Cohen-Macaulay) and ``super-perfection'' (the case when $S/I$ is
Gorenstein). For all this, see~\cite{Eisenbud-Gray}.

In this chapter we will take the point of view of Clebsch, Brill and Noether, and explain how to understand 
the differential forms and, given a (possibly ineffective) divisor $D$ on $C$, how to find all the 
effective divisors on $C$ that are linearly equivalent to $D$, in terms of a plane model---that is, 
in terms of a possibly singular plane curve $C_{0}$ and the normalization map $\nu: C\to C_{0}$.
We will give algorithms for the following tasks:

\begin{enumerate}
\item Given the equation $F(X,Y,Z)$ of a degree $d$ plane curve $C_0$, with normalization $C$,
and a certain subscheme $\adj(C_{0})\subset C_{0}$ supported on the singular locus of $C_{0}$,
construct a basis for $H^0(K_C)$ from curves of degree $d-3$ with base locus in $\adj(C_{0}$; and

\item  Given effective Cartier divisors $D_{0}$ and $D_{\infty}$, we will  find all effective divisors $E$ on $C$ with $E \sim D = D_{0}- D_{\infty}$, or, equivalently, a basis for $H^0(\cO_C(D))$, expressed in terms of curves of high degree with  base locus $D_{\infty}+\adj(C_{0})$. In particular, we will test whether a divisor $D$ linearly equivalent to any effective divisor, and we will see how to test when two given divisors $D$ and $E$ are linearly equivalent. 
\end{enumerate}

To explain the constructions  we start with the simplest case, where $C = C_0$ is a smooth plane curve (Section~ \ref{smooth plane curves}). But since most curves cannot be embedded in the plane, we must
allow singularities if we want to treat all smooth curves. 

In Proposition~\ref{nodal projection}  we will show that every smooth curve has a nodal plane model---that is,
a birational map to a plane curve with only nodes. This is a particularly
important special case, and also a relatively simple one, since in this case the subscheme $\adj(C_{0})$
is simply the set of nodes,
and we treat it in Section~~\ref{nodal curves section}. 

Though it is possible to realize a plane curve with arbitrary singularities as a birationally
equivalent nodal plane curve, it is often easier to work with the given plane model, so we treat
plane curves with arbitrary singularities in Section~\ref{arbitrary plane curves}. 

\section{Smooth plane curves}\label{smooth plane curves}

\subsection{Differentials on a smooth plane curve}\label{canonical series on smooth plane curves}

Let $C \subset \PP^2$  be a smooth plane curve, given as the zero locus of a homogeneous polynomial $F(X,Y,Z)$ of degree $d$. By the adjunction formula (Proposition~\ref{adjunction}) the canonical  divisors on $C$
are the intersections of $C$ with curves of degree $d-3$. In the spirit of Brill and Noether we
will make this explicit by constructing all
 the regular differential forms on $C$ in terms of forms of degree $d-3$:


For this purpose we introduce coordinates $x = X/Z$ and $y = Y/Z$ on the affine open subset $U \cong \AA^2$ given by $Z \neq 0$, and let $f(x,y) = F(x, y,1)$ be the inhomogeneous form of $F$, so that $C^\circ = C \cap U$ is given as the zero locus $V(f) \subset  \AA^2$. 

Since an automorphism of $\PP^2$ can carry any  any poin in $\PP^{2}$ to any other point, we may assume
that 
 The point $[0,1,0]$ (that is, the point at infinity in the vertical direction) does not lie on $C$ so that the
 projection  $\pi: C \to \PP^1$ from $(0,1,0)$, which is given by $[X,Y,Z] \mapsto [X,Z]$ (or, in affine coordinates, $(x,y) \mapsto x$)  has degree $d$. Let $D$ be the divisor defined by the intersection of $C$ with the line $Z=0$ at infinity.

Consider the
regular 1-form $dx$ on $\AA^2$, which we may regard as the pull-back of the form $dx$ on
 $\AA^{1}$.
Since the form $dx$ on $\PP^{1}$ has a double pole at infinity the form $dx|_{C}$ has polar
locus $2D$.
 
 \def\Co{{C^{\circ}}}
How do we get rid of the poles of $dx$? The extension to $\PP^2$ of a polynomial $h(x,y)$ of degree $m$ on
$\AA^2$ has a pole of order $m$ along the line $L$ at infinity. Thus if $h$ has degree at least 2 then $dx/h$ will have no poles at infinity. However, $h(x,y)$ will vanish at points of $\Co :=C \cap U$, and this may create new poles of $dx/h$. Of course if $h$ vanishes only at  points of $\Co$ where $dx$ has a zero, the zeroes of $h$ may cancel the zeroes of $dx$ rather than creating new poles.
 
 To avoid producing new poles in this way we may take
 $$
 h(x,y) = f_{y} := \frac{\partial f}{\partial y}(x,y).
 $$
 We claim that 
 $$
\varphi_0 = \frac{dx}{f_{y}}
$$
is everywhere regular and nowhere 0 in $C^\circ$. 

Note that $df$ vanishes identically when restricted to $C^\circ$, so
 $$
 0 \equiv df|_{\Co} = f_{x}dx|_{\Co} + f_{y}dy|_{\Co} .
 $$
Clearly $\varphi_0$ is regular at points $p$ where
where $f_{y}(p) \neq 0$. At such a point, if $dx|_{\Co}$ were 0, then since $\Co$ is smooth we would have $dy|_{\Co} \neq 0$, contradicting the equation above. Thus $\varphi_{0}$ is both regular and nonzero at such points. On the other hand, if $f_{y}(p) = 0$, then since $\Co$ is smooth $f_{x}(p) \neq 0$, so $dx|_{\Co}$ and $f_{y}$ vanish to the same
order, whence, again, $\varphi_0 = (dx/f_{y})|_{\Co}$ is regular and nonzero, proving the claim.

Put differently, if $L$ is the line at infinity, so that $U = \PP^{2}\setminus L \cong \AA^{2}$,
then the cotangent bundle on $U$ is 
$\Omega_{U}=\sO_{U}dx \oplus \sO_{U}dy,$
so 
the cotangent bundle $\Omega_{\Co}$ on $C^{\circ}$, which is the canonical bundle $\omega_{\Co}$,
 is the cokernel of the map from the normal sheaf $\sO_{C}(-d)|_{U}$ to the restriction of 
$\Omega_{U}$ to $C^{\circ}$. This map sends the local generator $f$ of the normal sheaf to
$f_{x}dx+f_{y}dy \in \Omega_{U}$, and because $f_{x}$ and $f_{y}$ have no common zeros, the generator $d_{y} \in \Omega_{C^{\circ}}$ is a multiple of $dx/f_{y}$.
Thus the free $\sO_{C^{\circ}}$-module $\omega_{C}|_{C^{0}}$ is generated by $dx/f_{y}$.

Since $f_{y}$ has degree $d-1$, the rational function $1/f_{y}$ vanishes to order $d-1$ on the line
at infinity, and thus in particular on the divisor $D$. Thus $\varphi_0$ vanishes to order $d-3$ on $D$; in other words, as divisors,
$$
(\varphi_0) = (d-3)D.
$$
In particular, if $d \geq 3$ then $\varphi_0$ is a globally regular differential on $C$. The divisor of
zeros of this differential has degree $d(d-3)$. If $g$ is the genus of $C$, then we must
have $2g-2 = d(d-3)$, whence 
$$
g = \frac{d(d-3)}{2} + 1 = \binom{d-1}{2}.
$$
We can produce a vector space of $\binom{d-1}{2}$ regular differentials by multiplying $\varphi_0$ by 
polynomials $e(x,y)$ 
  of degree $d-3$, since this will not introduce any poles. This proves:

\begin{theorem}
The space of regular differentials on a smooth plane curve $C$
with affine equation $f=0$ is 
$$
\{ \frac{e(x,y)}{f_{y}} \mid \hbox{ e(x,y) is a polynomial degree $\leq d-3$}\}.\qed
 $$
\end{theorem}
%We have thus found a vector space of regular differentials, of dimension $\binom{d-1}{2}$. But at the same time, the degree of a differential like $\varphi_0$ is
%$$
%\deg((\varphi_0)) = (d-3)\deg(D) = d(d-3),
%$$
%so that the genus of $g(C)$ satisfies
%$2g(C)-2 = d(d-3)$, whence
%$$
%\frac{d(d-3)}{2} + 1 = \binom{d-1}{2}.
%$$
%In other words, we have found all the global regular differentials on $C$! We have
%$$
%H^0(K_C) = \left\{ \frac{e(x,y)dx}{f_y} \mid \deg e \leq d-3\right\};
%$$
%or, equivalently, the space of regular differentials on $C$ has basis $\{\varphi_{i,j} \}_{i+j \leq d-3}$, where
%$$
%\varphi_{i,j} =  \frac{x^iy^jdx}{f_y}
%$$
%
%We could have achieved the same result by using the adjunction formula (see Section~\ref{Adjunction Formula}: we have
%$$
%K_C = (K_{\PP^2} \otimes \cO_{\PP^2}(d))|_C = \cO_C(d-3),
%$$
%and from the exact sequence
%$$
%0 \to \cO_{\PP^2}((d-3)-d) \rTo^f \cO_{\PP^2}(d-3) \to \cO_C(d-3)=K_C \to 0
%$$
%and the vanishing of $H^1(\cO_{\PP^2}(-3))$, we see that the map on global sections
%$$
%H^0(\cO_{\PP^2}(d-3)) \to H^0(K_C)
%$$
%is surjective. 
%

\subsection{Linear series on a smooth plane curves}\label{linear series on smooth plane curves}

Any divisor on a smooth plane curve $C$ may be expressed as the difference of
two effective divisors, $D= D_0-D_\infty$. We would like to find all the \emph{effective} divisors linearly equivalent to $D$, that is, of the form
$D + (H/G)$, where $G, H$ are forms of the same degree $m$. We begin by choosing
an integer $m$, large enough so there is a form $G$ of degree $m$ that vanishes on $D_0$ plus some divisor $A$ (but not on all of $C$). 

\begin{theorem}\label{equiv on smooth plane curve}
Let $D= D_0-D_\infty$ be a divisor on the smooth plane curve $C$. If
there is a form $G$ of degree $m$ defining a divisor $(G) = D_{0}+A$ on $C$, with $A$ effective, then
the effective divisors equivalent to $D$, if any, are precisely those 
of the form $(H) -(D_\infty+A)$ where $H$
is a form of degree $m$ vanishing on $D_\infty+A$, but not on all of $C$.
\end{theorem}

In particular, if no homogeneous polynomial $H$ of degree $m$ vanishes on  $A + D_\infty$ but not on $C$, then $D$ is not linearly equivalent to any effective divisor. The existence of such an $H$ is thus independent of the choices of $m$ and $G$, as we shall see in the proof.

The simplest special case of the theorem introduces a condition of great importance in the whole theory: we say that a curve $C$ in $\PP^{n}$ is \emph{arithmetically Cohen-Macaulay} if the hyperplane series
are all complete; or in more modern language, if the restriction maps
$$
H^{0}(\sO_{\PP^{n}}(m) \to H^{0}(\sO_{C}(m))
$$
are surjective for all $m$. We will meet this property several more times in the course of this book.

\begin{proposition}\label {completeness of hyperplanes on plane curve}
If $C$ is a plane curve, then any Cartier divisor on $D$ that is linearly equivalent to the divisor of
a form of degree $m$ is itself the divisor of a form of degree $m$; that is, plane curves are
arithmetically Cohen-Macaulay.
\end{proposition}

\begin{proof}
Let $C$ be the plane curve defined by $F=0$, and let $D$ be the divisor on $C$ defined by a form $L$
of degree $m$, not vanishing on (any component of) the curve $C$. If $G$ and $H$ are forms of the same degree $t$, 
not vanishing on any component of $C$,
and $D+(G/H)$ is effective, then the divisor $(LG)$  on $C$ must contain the divisor $(H)$ on $C$.
This means that the subscheme of $\PP^{2}$ defined by $(LG,F)$ contains the scheme defined 
by $(H,F)$. Since $H$ and $F$ have no components in common Theorem~\ref{Lasker} implies
that $LG = AH+BF$  for some forms $A,B$ with $\deg A = \deg LG -\deg H = m$ whence $D+(G/H) = (A)$
as required.
\end{proof}

\begin{example}
Suppose that $C$ has degree 3 and thus genus 1. If we choose as origin on the curve $C$ a point $o$, then to add two points $p$ and $q \in C$ means to find the (unique) effective divisor of degree 1 linearly equivalent to $p + q - o$. In this situation, Theorem~\ref{equiv on smooth plane curve} applies with $m=1$: there is a line $L$ 
containing $p+q$ defined by a linear form $G$. If $r \in C$ be the remaining point of intersection of $L$ with $C$ we can choose a linear form $H$ vanishing on $o+r$, and the line it defines meets $C$
in one additional point $s$. This is the classical construction of the group law.
\pict{illustrate this}
\end{example}

\begin{proof}[Proof of Theorem~\ref{equiv on smooth plane curve}]
First, suppose that we can find a form $H$ of degree $m$ as in the Theorem.
Setting $D' = (H) -(D_\infty+A)$ we have
$$
D' = D + (H/G) = D_0- D_\infty - (D_0+A)+(D_\infty+A+D')
$$
so $D'$ is linearly equivalent to $D$. 
\pict {illustrate the configuration of all the divisors}

We claim that we find in this way all effective divisors $D' \sim D$. 
To see this, suppose $D'$ is any effective divisor with $D' \sim D$, so that
$$
\cO_C(A+D_\infty+D') = \cO_C(A+D_\infty+D)  = \cO_C(m),
$$
that is, $A+D_\infty+D' \equiv (G)$ for some form $G$ of degree $m$. By Proposition~\ref{completeness of hyperplanes on plane curve}
this implies that $A+D_{0} = (H)$ for some form of degree $m$ as required.
\end{proof}


The argument given in Proposition~\ref{equiv on smooth plane curve} can be stated more generally thus:  if curves $F(X,Y,Z)=0$ and $Q(X,Y,Z)=0$ 
meet only in a finite set of points $\Gamma$ in $\PP^{2}$, and $E(X,Y,Z) = 0$ is a curve containing the intersection in an appropriate sense,
then $E = QH +LF$ for some forms $H$ and $L$, and this statement applies to arbitrarily singular curves. Recognizing its importance for the argument above and the generalizations to come, Max Noether in~\cite{Noether1873} dubbed it the \emph{Fundamental Theorem}, 
noting that it had often been used by geometers but not proven. After successive attempts and 
criticisms involving many mathematicians, he and Brill gave a complete proof, essentially using
regular sequences as above, in~\cite{Brill-Noether}. For more of this story see the account in~\cite{Eisenbud-Gray}.



\section{Nodal plane curves}\label{nodal curves section}

The methods above can be applied, with one change, when $C_{0}\subset \PP^{2}$
is a nodal curve.

More generally, let $\nu: C\to C_{0}\subset \PP^{2}$ be the normalization morphism from a smooth curve,
and let $B\subset C_{0}$ be a subscheme.
It will be convenient to speak of \emph{the linear series cut out on $C$} by curves of degree $m$
containing $B$: though $B$ is only a subscheme, $\nu^{-1}(B)$ may be considered as a Cartier divisor because
$C$ is smooth, and we define
the linear series cut out on $C$ by curves of degree $m$
containing $B$ to be
the linear series  $\sV$ of divisors in $C$ that are residual to $\nu^{-1}(B)$ in the pullbacks 
of the intersections of $C_{0}$ with curves of degree $m$.
\fix{have we defined ``residual to'' somewhere? Do we need to?}

\subsection{Differentials on a nodal plane curve}\label{canonical series on nodal plane curves}

Let $C_{0} \subset \PP^2$  be a curve of degree $d$ with $\delta$ nodes and no other singularities. By the adjunction
formula (Proposition~\ref{adjunction}), Proposition~\ref{pa and delta}, and the first example that follows it, 
the genus $g$ of the normalization $C$ of $C_{0}$ is
the arithmetic genus $p_{a}(C_{0}) = \binom{d-1}{2}$ of $C_{0}$ minus $\delta$, that is,
$$
g = \binom{d-1}{2} -\delta.
$$
We will make this explicit by exhibiting a vector space of $g$ regular differential forms on $C$.

We may choose homogeneous coordinates  $[X,Y,Z]$ on $\PP^2$ so that the curve $C_0$ intersects the line $L = V(Z)$ in a divisor $D$ consisting only of smooth points of $C_{0}$  other than $[0,1,0]$. In particular,  all the nodes of $C_0$ will lie in the affine plane $U = \PP^2 \setminus L$.
In addition, we can assume for simplicity that  neither branch of $C_0$ at a node has vertical tangent. (These conditions are satisfied by a general choice of coordinates.) Let the nodes of $C_0$ be $q_1,\dots,q_\delta$, with $r_i, s_i \in C$ lying over $q_i$; we'll denote by $\Delta$ the divisor $\sum r_i + \sum s_i$ on $C$.

Let $F(X,Y,Z)$ be the homogeneous polynomial of degree $d$ defining the curve $C_0$, and let $f(x,y) = F(x,y,1)$ be the defining equation of the affine part $C_{0}^{\circ}:= C_0 \cap U$ of $C_0$. Let $\nu: C\to C_0$ be the normalization map. We start by considering the rational differential 
$\nu^*(dx)$ on 
$C^{\circ}:= \nu^{-1}(C_{0}^{\circ})$. 

In the smooth case, $C_{0}=C$, we saw that this differential was regular and nonzero on $C^{\circ}$; this followed from the fact that 
that $f_{x}$ and $f_{y}$ had no common zeroes on $C_0$. But now $f_{x}$ and $f_{y}$ have common zeroes: they both vanish to order 1 at the points $q_{i}$ and thus $\nu^*(f_{x})$ and $\nu^*(f_{y})$ have simple zeroes at the points $r_i$ and $s_i$. 

As before, the differential $\nu^*dx$ has a double poles at the divisor $D$ on $C_{0}$ lying over the point at infinity in $\PP^{1}$
and we see that for a polynomial $e(x,y)$ of degree $\leq d-3$, the differential
$$
\nu^*( \frac{e(x,y)dx}{f_{y}})
$$
is regular except for simple poles at the points $r_i$ and $s_i$.

We can get rid of these poles by requiring that $e$ vanishes at the points $q_i$. We say in this case that $e$ (and the curve defined by $e$) \emph{satisfies the conditions of adjunction}. 

\begin{theorem}\label{canonical from adjoint 1}
If $C_{0}$ is a nodal plane curve of degree $d$ with normalization $\nu: C\to C_{0}$
then the  regular differentials on  $C$, in suitable affine coordinates as $f(x,y) = 0$ 
 are precisely those of the form
 $$
\nu^{*}\biggl( \frac{e(x,y)}dx{f_{y}}\biggr)
$$
where 
$e(x,y)$ ranges over the polynomials of degree $\leq d-3$
vanishing at the nodes of $C_{0}.$

Thus if $\adj(C_{0})\subset C_{0}$ denotes the union
of the reduced points at the nodes of $C$, then  the linear series cut out on $C$ by 
forms of degree $d-3$ containing $\adj(C_{0})$ is $|\omega_{C}|$.
\end{theorem}

\begin{proof}
The dimension of the space of polynomials $e(x,y)$ of degree at most $d-3$ is $\binom{d-1}{2}$,
and vanishing at $\delta$ nodes imposses at most $\delta$ linear conditions on $e$. The linear map sending
$e\mapsto \nu^{*}(edx/f_{y})$ is injective, and the target has dimension 
$\binom{d-1}{2}-\delta$, so this must be an isomorphism.
\end{proof}
We ill sketch a more conceptual proof of this theorem in Section~\ref{arbitrary plane curves}.

In particular, Theorem~\ref{canonical from adjoint 1}
shows that the linear series cut out on $C$ by 
forms of degree $d-3$ containing $\adj(C_{0})$ is complete. (We will soon see that
 the linear series cut out on $C$ by 
forms of degree $m$ containing $\adj(C_{0})$ is complete for every $m$.)


This gives another proof of Lemma~\ref{adjoint independent}.

\begin{corollary}
If $C$ is a nodal plane curve of degree $d$, then the nodes of $C_{0}$ impose independent
conditions on forms of degree $d-3$.
\end{corollary}
\begin{proof}
 Otherwise the space of differential forms on the normalization of $C_{0}$ would be too large.
\end{proof}

\pict{Curve $C_{0}$ of genus 4 represented as a quintic with 2 nodes, showing a canonical divisor
representeed as the intersection of $C_{0}$
with a conic through 2 nodes}

\subsection{Linear series on a nodal plane curve}\label{linear series on nodal plane curves}

Given a divisor $D = D_{0}-D_{\infty}$ on $C$, we will compute the complete linear series $|D|$, using a birational map $\nu$ from 
$C$ to the nodal curve $C_0$. Let $\Delta$ be the preimage in $C$ of the set of nodes of $C_{0}$.

Choose an integer $m$ and a polynomial $G$ whose pullback to $C$
  vanishes on the divisor $D_0+\Delta$ 
 but not identically on $C$. (If $D_0$ contains some positive multiple of a point $p$ of $\Delta$ this means that $G$ defines
 a curve sufficiently tangent to the corresponding branch of $C_0$.) This can always be achieved for
 sufficiently large $m$. We can then write the zero locus of $G$ pulled back to $C$ as
$$
(\nu^*G) = D_0 + \Delta + A,
$$
as before. 

\pict{illustration of a case where $D_0$ contains one point of $\Delta$.}

Next, we look for forms $H$ of the same degree $m$, vanishing at $A+D_\infty$ and at all the nodes $q_i$,  but not on all of $C_0$. If there are no such polynomials $H$ then, as we shall show,
there are no effective divisors equivalent to $D$. Supposing that $H$ is such a form, let $D'$ be the divisor 
$$
D' = (\nu^*H) -( D_0 + \Delta),
$$
that is, $D'$ is residual to $( D_0 + \Delta)$ in $(\nu^*H)$. 

Since $\nu^*(G/H)$ is a rational function on $C$ we have
$$
D_\infty +\Delta + A+ D' = (\nu^*H) \sim (\nu^*G) = D_0 + \Delta + A,
$$
and thus $D'$ is an effective divisor linearly equivalent to $D = D_{0}-D_{\infty}$ on $C$.

To complete the argument we must show that we get \emph{all} divisors $D'$ in this way.
In this case the curve $C$ can be desingularized by blowing up the plane once at each node,
and we can give a proof based on the resulting surface $S$. The same technique would work for any curve with only
ordinary multiple points, in which case the total transform of $C_{0}$ on $S$ has normal crossings. We will give a different proof, extending this theorem to curves with arbitrary singularities, in Section~\ref{arbitrary plane curves}.

\begin{proposition}\label{adjoint completeness1}
If $C_{0}$ is a reduced irreducible plane curve all of whose singularities are ordinary nodes, then for each
integer $m$,
the linear series cut out on the normalization $C$ of $C_{0}$ by forms of degree $m$ containing the nodes
is complete.
\end{proposition}

\begin{proof}
To prove Proposition~\ref{adjoint completeness1}, we work on the blow-up $\pi : S \to \PP^2$ of $\PP^2$ at the points $q_i$. The proper transform of $C_0 \subset \PP^2$ in $S$ is the normalization of $C_0$, which we will again call $C$.

Let $H$ be the class on $S$ of the pullback of a line in $\PP^2$  and let $E$ be the sum of the exceptional divisors. We write $h= H\cap C$ and $e = E\cap C= \sum (p_i+q_i)$ for the corresponding divisors on $C$. 
Because $C$ has double points at each $q_{i}$ we have
$
C \sim dH - 2E 
$
and   by
Theorem~\ref{divisor classes on blowup} we have $K_S \sim -3H + E$.

The proper transform of a degree $m$ curve $A\subset \PP^2$  passing through the points $q_i$
is $\pi^*A - E$; this gives an isomorphism
$$
H^0(\cI_{\{q_1,\dots,q_\delta\}/\PP^2}(m)) \cong H^0(\cO_S(mH-E)).
$$
In these terms we can describe the linear series cut on $C$ by plane curves of degree $m$ passing through the nodes of $C_0$ as the image of the map
$$
H^0(\cO_S(mH-E)) \to H^0(\cO_C(mH-E)),
$$
and we must show that this map is surjective.

From the long exact cohomology sequence associated to the exact sequence of sheaves
$$
0 \to \cO_S((m-d)H + E)  \to \cO_S(mH-E) \to \cO_C(mH-E) \to 0,
$$
 we see that it will suffice to prove that $H^1(\cO_S(mH - C + E)) = H^1(\cO_S((m-d)H + E)) = 0$. 
 
By Serre duality on $S$ gives
%we have  $H^1(\cL) \cong H^1(K_S\otimes \cL^{-1})^*$. In this instance it 
%tells us that
$$
H^1(\cO_S((m-d)H + E)) \cong H^1(\cO_S((d-m-3)H))^*.
$$
The line bundle $\cO_S((d-m-3)H)$ is 
 the pullback to $S$ of the bundle $\cO_{\PP^2}(d-m-3)$, which has vanishing $H^1$. Lemma~\ref{H1 on pullback} completes the proof.
\end{proof}

\begin{lemma}\label{H1 on pullback}
Let $X$ be a smooth projective surface, and $\pi : Y \to X$ a blow-up. If $\cL$ is any line bundle on $X$, then
$$
H^1(Y, \pi^*\cL) = H^1(X, \cL).
$$
\end{lemma}
\begin{proof}
For any invertible sheaf $\sL$ on $\PP^{2}$ the natural map $\pi_{*}\pi^{*}(\sL) \to \sL$ is an isomorphism
away from the codimension 2 set of nodes, and is thus an isomorphism. The Leray spectral sequence (Theorem~\ref{Leray}) gives an exact sequence
$$
0\to H^{1}(\pi_{*}(\sL)) \to H^{1}(\sL) \to  H^{0}(R^{1}(\pi_{*}(\sL))\to 0
$$

Take $\sL = \pi^{*}(\sO_{\PP^{2}}(1))$. The restrictions of $\sL$ to the fibers of $\pi$ have vanishing $H^{1}$,
so
$H^{0}(R^{1}(\pi_{*}(\sL)) = 0$.
Also, the restriction of $\pi^{*}\sO_{\PP^{2}}(1)$ to any fiber of $\pi$ is trivial,
so $\pi_{*}\pi^{*}\sO_{\PP^{2}}(1)$ is an invertible sheaf, and the natural map
$\pi_{*}\pi^{*}\sO_{\PP^{2}}(1) \to \sO_{\PP^{2}}(1)$ is an isomorphism away from the codimension
2 set of nodes of $C_{0}$. Thus these two sheaves are isomorphic, and
$$
H^{1}(\pi_{*}\pi^{*}\sO_{\PP^{2}}(1) ) = H^{1}(\sO_{\PP^{2}}(1)) = 0,
$$
completing the proof.
\end{proof}



\begin{proposition}\label{effect of blowup on genus}
 Let $C$ be a curve on a smooth surface $S$, let $\nu : S' \to S$ be the blowup of $S$ at $p$. If $C'$ is the strict transform of $C$, then
 $$
 p_a(C') = p_a(C) -{m\choose 2},
 $$
 where $m$ is the multiplicity of $p\in C$.
\end{proposition}
\begin{proof}
This follows from comparing the adjunction formulas on $S$ and $S'$. To start, we have
$$
p_a(C) = \frac{C^2 + K_S\cdot C}{2} + 1.
$$
On $S'$ let $E$ be the divisor class of the exceptional divisor. As we've seen,
$$
K_{S'} = \nu^*K_S + E,
$$
while the class of $C'$ is given by
$$
C' \sim \nu^*C - mE.
$$
It follows that
$$
(C')^2 = C^2 + m^2E^2 = C^2 - m^2 \quad \text{and} \quad K_{S'}\cdot C' = K_S\cdot C + m.
$$
Thus, applying adjunction on $S'$, we find
$$
p_a(C') = \frac{{C'}^2 + K_{S'}\cdot C'}{2} + 1 = \frac{C^2 + K_S\cdot C - m(m-1)}{2} + 1 = p_a(C) -{m\choose 2}
$$
as stated.
\end{proof}

\begin{fact}
Any plane curve can be desingularized by
iteratively blowing up of singular points of $C$, then of the strict transform, and so on. See for example
\cite{Fulton1989} \fix{is it there??} or \cite{Brieskorn1986}. The points on the various blowups that
map to the original singular point are called \emph{infinitely near points}.
\end{fact}

This gives a nice formula for the delta-invariant of any singularity:

\begin{corollary}
\label{computing delta}

Thus the $\delta$ invariant of any singularity of a plane curve $C_{0}$ at a point $p$ can be computed as the sum of the numbers $\binom{m_{q}}{2}$
where the sum runs over all \emph{infinitely near} singular points $q$.\qed
\end{corollary}

\section{Arbitrary plane curves} \label{arbitrary plane curves}

Throughout this section $C_0 \subset \PP^2$ denotes a reduced and irreducible, but  this time with arbitrary singularities. Let $\nu : C \to C_0$ be its normalization, and write $H$ for the pullback to $C$ of the class of a line.

It is possible to carry out an analysis of linear series on the normalization of an arbitrary plane curve in a manner  analogous to what we did in the preceding section for nodal curves, replacing the the set of nodes by the scheme $\adj(C_{0})\subset C_{0}$ called the \emph{adjoint scheme}, which is in this case is the scheme
defined by the \emph{conductor ideal} $\ff_{C/C_0}$, the annihilator in $\sO_{C_{0}}$ of $\nu_{*}(\sO_{C})/\sO_{C_{0}}$---see Theorem~\ref{general adjoint}. Let $\Delta\subset C = \nu^{*}(\adj(C))$ be the Cartier divisor
defined by the pullback of $\ff_{C/C_0}$ to $C$.

\subsection{The conductor ideal and linear series on the normalization}

\begin{theorem}\label{linear series on arbitrary curves}
Let $D = D_{0}-D_{\infty}$ be a divisor on $C$, and let $G$ be a form on $\PP^{2}$ whose pullback to $C$
vanishes on $D_{0}+\Delta$. 

If $G$ has degree $m$ and $A = (\nu^{*}G)-D_{0}-\Delta$, then every effective divisor on
$C$ linearly equivalent to $D$ (if any) has the form $(\nu^*H)-D_{\infty}-A-\Delta$ for some $H$ of degree $m$
that vanishing on $\adj(C_{0})+A$.
\end{theorem}

\begin{proof}
If $H$ is such a form, then as before
$$
 D_\infty +\Delta + A+ D' = (\nu^*H) \sim (\nu^*G) = D_0 + \Delta + A,
$$
So $D' = (\nu^*{H})-D_{\infty}-A-\Delta$ is linearly equivalent to $D$. The proof that every
divisor $D'$ linearly equivalent to $D$ has this form is the content of
Theorem~\ref{conductor completeness} below.
\end{proof}
%\begin{theorem}
%If $C_{0}$ is a reduced, irreducible curve in $\PP^{2}$ and $\sV$ is a complete linear series
%on the normalization $\nu: C\to C_{0}$ then for all sufficiently large integers $m$ there is an effective Cartier divisor $D_{m}$  of $C_{0}$ 
%such that $\sV$ is the linear series cut out on $C$ by curves of degree $m$ containing $D_{m}+\adj(C)$.
%\end{theorem}


The following result was known classically as the \emph{completeness of the adjoint series}.

\begin{theorem}\label{conductor completeness}
For every integer $m\geq 0$ the series cut out on $C$ by forms of degree $m$
on $\PP^{2}$ containing $\adj(C_{0})$ is complete.
\end{theorem}

\begin{proof}
If $R\subset R'$ is an inclusion of commutative rings, then the set
$\ff_{R'/R} := \ann_{R}(R'/R)\subset R$, which is defined as an ideal of $R$, is also an ideal of $R'$; this follows
because if $f\in \ff_{R'/R} $ and $r, r'\in R'$, so that $fr \in R$, then $(r'f)r = f(r'r) \in R$ as well. 

If $R_{0}$ is a domain and $R$ is a subring of the quotient field $Q(R)$ of $R$, then
 $\ff_{R/R_{0}} \cong \Hom_{R_{0}}(R, R_{0})$. To see this, note that $R_{0}$ and $R$ become
 equal after tensoring with $Q(R_{0})$ and thus 
 $Hom_{R_{0}}(R,R_{0}) \subset Hom_{Q}(Q,Q) = Q$ 
 may be identified
 with the set of elements $\{\alpha\in Q\mid \alpha R \subset R_{0}\}$. If $\alpha$ is in this set, then
  $\alpha\cdot 1 = \alpha \in R_{0}$, as required.
  
Returning to the case of the curve $C_{0}$, it follows that the global sections of $(\nu^{*}(\sO_{C_{0}}(m))( -\Delta)$ on $C$
are, on each affine open set $U$, represented by the elements of $\sO_{C_{0}}(U)$ that  are restrictions to $U$
of forms of degree $m$ contained in 
the ideal $\ff_{C/C_{0}}$.Thus
the global sections of the sheaf $\widetilde{\ff_{C/C_{0}}(m)}$ cut out a complete linear series on $C$.

Write $S =\CC[x_{0}, x_{1}, x_{2}]$ for the homogeneous coordinate ring of $\PP^{2}$.
It remains to prove that the homogeneous ideal $\ff_{C/C_{0}}$ maps
surjectively to $H^{0}_{*}(\widetilde{\ff_{C/C_{0}}})$, and this amounts to the
statement that the depth of $\ff_{C/C_{0}}$ as an $S$-module is (at least) 2.
%There is a natural  map
%$$
%\ff_{C/C_{0}} \to \sHom_{\sO_{C_{0}}(\nu_{*}\sO_{C}, \sO_{C_{0}}},
%$$ and the argument
%above shows that this map is an isomorphism on each affine open set, proving that as sheaves
%$$
%\ff_{C/C_{0}} \cong \sHom_{\sO_{C_{0}}}(\nu_{*}\sO_{C}, \sO_{C_{0}}.
%$$
Set $R_{0} = H^{0}_{*}(\nu_{*} (\sO_{C_{0}})$ and $R = H^{0}_{*}(\nu_{*} (\sO_{C})$.
We see from the general considerations above that
$$
\ff_{C/C_{0}} = Hom_{R_{0}}(R, R_{0}).
$$ 
A nonzerodivisor on a module $M$
is automatically a nonzerodivisor on $Hom(P, M)$ for any $P$ since $(a\phi)(p) = a(\phi(p))$ by definition. 
Since $R_{0} = S/(F)$, it is a module of depth 2, and we may choose a regular sequence
$a,b$ of elements in $R_{0}$. From the short exact sequence
$$
0\rTo R_{0}\rTo^{a} R_{0}\rTo R_{0}/(a) \rTo 0
$$
we get a left exact sequence
$$
0\rTo Hom_{R_{0}}(R,R_{0})\rTo^{a} Hom_{R_{0}}(R,R_{0})\rTo Hom_{R_{0}}(R,R_{0}/(a)).
$$
Thus 
$$
Hom_{R_{0}}(R,R_{0})/aHom_{R_{0}}(R,R_{0}) \subset Hom_{R_{0}}(R,R_{0}/(a))
$$
and since $b$ is a nonzerodivisor on $Hom_{R_{0}}(R,R_{0}/(a))$, it is a nonzerodivisor
on $Hom_{R_{0}}(R,R_{0})/aHom_{R_{0}}(R,R_{0})$ as well.
\end{proof}

Since we saw directly that the adjoint ideal was equal to the conductor ideal in the case of
a nodal curve, this result gives another, less ad hoc proof, that the effective divisors equivalent to $D$
are all defined by
pullbacks of forms of degree $m$ that contain $\Delta$
as constructed in Proposition~\ref{adjoint completeness1}.

\subsection{Differentials}

Let $C^\circ_0$ to be the intersection of $C_0$ with the open set $\AA^{2}\cong U\subset \PP^{2}$ where $Z \neq 0$.

\begin{theorem}\label{general differentials}
If $C_{0}$ meets the line $L$ at infinity only in smooth points of $C_{0}$ other than $(0,1,0)$, then the complete canonical series on the normalization $\nu: C \to C_{0}$ is cut out by differentials of the form
$$
e(x,y) \frac{dx}{f_{y}}
$$
where $e(x,y)$ is a polynomial of degree $\leq d-3$ contained in the 
conductor ideal $\ff_{C^{\circ}/C_{0}^{\circ}}$
\end{theorem}

\begin{proof}
The proof of this result consists of four parts. 

First, because we have assumed that $(0,1,0)$ does not lie on $C$, the function $x$ defines a
ramified $d$-sheeted cover of $C$ to $\PP^{1}$. Because $C_{0}$ meets $L$ only in smooth
points and the the differential $dx$ has a pole of order 2 at the point on $\PP^{1}$ at infinity,
$dx$
has polar locus twice the divisor $C_{0}\cap L = \nu^{-1}(C_{0}\cap L$. It follows that
the differential
$\varphi_0 := dx/f_{y}$ is regular, with a zero of order $d-3$,
along the divisor of $C$ lying over $C_0\cap L$.

Second, the function on $C^\circ_0$ defined by $x$  
is a finite map to $\AA^1$, and thus the field of rational functions $\kappa(C) = \kappa(C^\circ_0)$ is a finite
separable extension of $\CC(x)$. By~\cite[Section 16.5]{Eisenbud1995} that the module of differentials 
$\omega_{\kappa(C)/\CC}$ is generated over $\kappa(C^\circ_0)$ by $dx$. Thus every rational
differential form on $C$ can be expressed as a rational function
times $dx$. Since $\varphi_{0} = dx/f_{y}$ vanishes to order $d-3$ along $C_{0}\cap L$,
the regular differential forms on $C$ must be of the form $e(x,y)\varphi_{0}$ where
$e(x,y)$ is a rational function of degree $\leq d-3$. (The set of rational forms that occur in this
way is called the \emph{Dedekind complementary module}.)
 
A sophisticated form of Hurwitz Theorem that will be explained in Chapter~\ref{LinkageChapter}
shows that the sheaf $\omega_{C}$ of regular differential forms on $C$ can be expressed as
$$
\sHom_{\PP^{1}}(\nu_{*}(\sO_{C}), \omega_{\PP^{1}}).
$$
(or more correctly $\nu^{!}\pi^{!}\sHom_{\PP^{1}}(\pi_{*}\nu_{*}(\sO_{C}), \omega_{\PP^{1}})$
where $\pi$ is the map $C_{0}\to \PP^{1}$ defined by $x$---see Chapter~\ref{LinkageChapter}. Since the maps involved are finite,
we will ignore this refinement.)

Since $\sO_{C_{0}}\subset \nu_{*}(\sO_{C})$, This sheaf is naturally contained
in 
$$
\sHom_{\PP^{1}}(\sO_{C_{0}}, \omega_{\PP^{1}}).
$$
We will show in Theorem~\ref{general adjoint} that 
$$
\ff_{C/C_{0}} = 
\frac{\sHom_{\PP^{1}}(\sO_{C_{0}}, \omega_{\PP^{1}})}
{\sHom_{\PP^{1}}(\nu_{*}(\sO_{C}), \omega_{\PP^{1}})}.
$$
As will be explained in Chapter~\ref{LinkageChapter}, the numerator of this quotient
$$
\omega_{C_{0}}:= \sHom_{\PP^{1}}(\sO_{C_{0}}, \omega_{\PP^{1}}) = 
\sHom_{\PP^{1}}(\sO_{C_{0}}, \sO_{\PP^{1}})(-2)
$$
 is properly called the dualizing module  of the singular curve $C_{0}$.
Thus every regular differential form on $C$ can be expressed as an element of the conductor
times some element of $\omega_{C_{0}}$. To prove the Theorem, we must show that
$\varphi_{0}$ generates $\omega_{C_{0}}$ as a module over $\sO_{C_{0}}$.

Passing to the field of rational functions $\kappa := \kappa(C)$, and noting that
$\kappa(C) = \kappa(C_{0}) $ we use the well-known result from Galois theory that 
$$
Hom_{\kappa(\PP^{1})}(\kappa, \kappa(\PP^{1})
$$
is generated over $\kappa$ by the trace map $T$. Moreover, 
because $\sO_{C_{0}}$ is integral over $\sO_{\PP^{1}}$, which is normal,
$$
T(\sO_{C_{0}})\subset \sO_{\PP^{1}}.
$$

Less well-known is the result from commutative
algebra:
\begin{fact}
\begin{theorem}\label{Kunz}
If $C_{0}^{\circ}$ is an affine plane curve defined by the
equation $f(x,y)=0$ and such that $\CC[x,y]/(f)$ is finite over $\CC[x]$,
then $Hom_{\CC[x]}(\CC[x,y]/(f), \CC[x])$ is generated by $(1/f_{y})T$.
\end{theorem}
See \cite[Theorem 15.1]{Kunz} for a proof using valuations, and \cite[Theorem A.1]{MR4026452} for a proof in
a more general context.
\end{fact}

Via the isomorphism $\CC(x) \cong \CC(x)dx$ sending 1 to $dx$ the trace map is identified (up to scalar)
with the map $(1\mapsto dx) \in Hom_{\CC(x)}(\kappa, \CC(x)dx)$. Thus by Theorem~\ref{Kunz}
the canonical module of $C_{0}^{\circ}$ is identified with $\sO_{C_{0}^{\circ}/\CC} dx/f_{y}$ as
required.
\end{proof}

%Theorem~\ref{general differentials} can be generalized to Gorenstein singularities. However the example
%of spatial triple point below shows that the general situation can look rather different: there the (local) conductor
%is the maximal ideal at the singular point, but there are two linear conditions for a function on $C$ to descend
%to $C_{0}$



\begin{example}[nodes and cusps]
We have already seen that in case $q$ is a node of $C_0$, there are two points of $C$ lying over it, and the multiplicities of $\varphi_0$ at these two points are $m_1=m_2=1$; the adjoint ideal is thus 
 the maximal ideal $\cI_q$ at $q$. In the case of a cusp, analytically isomorphic to the zero locus of $y^2-x^3$, there is only one point $r$ of $C$ lying over the cusp point $q$. The cusp can be parameterized, locally analytically,
 by $x = t^{2}, y = t^{3})$ and it follows that the differential 
 $$
 \varphi_0 = \frac{dx}{f_{y}} =  \frac{2tdt}{2t^{3}} =  \frac{dt}{t^{2}}
 $$ 
 has a pole of of order 2. since the pullback to $C$ of any polynomial $g$ vanishing at the cusp $q$ will vanish to order at least two at $r$, the adjoint ideal is the maximal ideal at $q$. We can also see this by computing the
 conductor ideal as the annihilator of $\CC[[t]]/\CC[[t^{2}, t^{3}]]$.
\end{example}

\begin{example}[tacnodes]
Next, consider the case of a \emph{tacnode}; that is, a plane-curve singularity with two smooth branches simply tangent to one another, analytically isomorphic to the zero locus of $y^2-x^4$, parameterized locally analytically with two branches $x = t, y =  t^{2}$ and $x=t, y = -t^{2}$.
\pict{the tacnode}
At the two points of $C$ lying over $q$, we have
  $$
 \varphi_0 =  \frac{dt}{ 2t} \hbox{ and } \frac{-dt}{ 2t}
 $$ 
whch has a simple poles at each. 
The adjoint ideal is thus the ideal of functions vanishing at $q$ and having derivative 0 in the direction of the common tangent line to the branches.
\end{example}

\begin{example}[ordinary $n$-fold points]
In the case of an ordinary $n$-fold point of a plane curve---$n$ smooth branches pairwise transverse to one another---there are $n$ points
$r_i$ of $C$ lying over $q$. The polynomial $f_y$ vanishes to order $n-1$ at $q$, so $dx/f_y$ has a pole of order $n-1$ at
each $r_i$. It follows that for $e(x,y)dx/f_y$ to be regular, $e$ must vanish to order $n-1$ at each $r_i$. 

We can see that the conductor ideal is the full $(n-1)$-rst power of $(x,y)$ by using the
normalization map
$$
\nu^{*}: \frac{\CC[x,y]}{\prod_{i=1}^{n} (x-\alpha_{i}y)} \to
 \prod_{i=1}^{n} \frac{\CC[x,y]e_{i}}{(x-\alpha_{i}y)}
$$
where the $\alpha_{i}$ are distinct elements of $\CC$ and the $e_{i}$ are orthogonal idempotents.
The element $1 = \sum_{i}e_{i}$ goes to 0 in the quotient, and $e_{i}$ is annihilated by $x-\alpha_{i}y$,
so the quotient is annihilated by each of the $n$ elements $g_{j} := \prod_{i\neq j} (x-\alpha_{i}y)$.
These elements are linearly independent: regarded as forms on $\PP^{1}$, all but $g_{j}$ vanish at the 
point $(\alpha_{j}, 1)$. Since $(x,y)^{n-1}$ is minimally generated by $n$ elements, these ideals must be equal.

A consequence of this computation is that the $\delta$-invariant of the ordinary multiple point---that is, the difference in arithmetic genus between a plane curve that has just one such singular point and its normalization---is $binom{n}{2}$, the dimension of $k[x,y]/(x,y)^{n-1}$, as we proved before in Proposition~\ref{effect of blowup on genus} \end{example}

\begin{example} (spatial triple points) Spatial triple points provide a contrast to the last example. A spatial triple point is a singularity consisting of three smooth branches, with linearly independent tangent lines, meeting in a point $p$ so that its Zariski tangent space is 3-dimensional; the simplest example is the origin as a point on the union of the three coordinate axes in $\AA^3$.

In this case the conductor is the annihilator of the cokernel of
$$
\nu^{*}: R := \frac{\CC[x,y,z]}{(xy, xz, yz)} \to \frac{\CC[x,y,z]}{(x,y)} \times \frac{\CC[x,y,z]}{(x,z)} \times \frac{\CC[x,y,z]}{(y,z)} =: \overline R.
$$
Since $x\overline {R} = x \CC[x,y,z]/(y,z)$ is in the image of $\CC[x,y,z]/(xy, xz, yz)$, and similarly with $y$ and $z$,
we see that the conductor is the maximal ideal $(x,y,z)$, but the $\delta$-invariant, the length
of the quotient $(\overline R)/R$, is 2 (for a function $f$ in $\overline R$ to descend to $R$, it
is necessary and sufficient that $f$ take the same value at the three points above the singular point:
2 linear conditions.)
\end{example}

\section{Exercises}

In Exercise~\ref{gonality of smooth plane curve}, we saw how to use the description of the canonical series on a smooth plane curve to determine its gonality. Now that we have an analogous description of the canonical series on (the normalization of) a nodal plane curve, we can deduce a similar statement about the gonality of such a curve. Here are the first two cases: 

\begin{exercise}
Let $C_0$ be a plane curve of degree $d\geq 4$ with one node and no other singularities, and let $C$ be its normalization. Show that $C$ admits a unique map $C \to \PP^1$ of degree $d-2$, but does not admit a map $C \to \PP^1$ of degree $d-3$ or less.
\end{exercise}

\begin{exercise}\label{general case of divisors on nodal curves}
Suppose that $C$ is a smooth curve and $\nu: C \to C_0$ is a map to a plane curve with
only nodes as singularities and let $D = D_0-D_\infty$ be a divisor on $C$. Modify the 
technique of Section~ref{linear series on nodal plane curves} to compute the complete
linear series $|D|$ without assuming that $D$ is disjoint from the preimages of the singular
points. 
Hint: Be careful to subtract the right multiples of the points that are preimages of the singular
points.
\end{exercise}

\begin{exercise}
Let $C_0$ be a plane curve of degree $d\geq 5$ with two nodes and no other singularities, and let $C$ be its normalization. Show that $C$ admits two maps $C \to \PP^1$ of degree $d-2$, but does not admit a map $C \to \PP^1$ of degree $d-3$ or less.
\end{exercise}

\begin{exercise}
Generalizing some of the examples above, show that if a nodal plane curve of degree $d$ has $\delta\leq d+3$ nodes,
then its gonality is $d-2$, and moreover every $g^1_d$ on the curve is given by projection from one of the nodes.
You should make use of the following result from~\cite[p. 302]{MR1376653}:
\begin{proposition}
 A set of $n \leq 2d+ 2$ distinct
points fails to impose independent conditions on curves of degree
$d$ if and only if either $d + 2$ of the points  are collinear or $n = 2d + 2$ and all the points lie
on a conic.
\end{proposition} 
Hint: a $g^1_{d-2}$ is a set of points that, together with the nodes, impose dependent conditions on forms of degree $d-3$.
\end{exercise}

\begin{exercise}
Find the adjoint ideals of the following plane curve singularities:
\begin{enumerate}
\item a \emph{triple tacnode}: three smooth branches, pairwise simply tangent
\item a triple point with an infinitely near double point: three smooth branches, two of which are simply tangent, with the third transverse to both
\item a unibranch triple point, such as the zero locus of $y^3-x^4$
\end{enumerate}
\end{exercise}

Here is a  general description in case the individual branches of $C_0$ at $p$ are each smooth:

\begin{exercise}
Let $\nu : C \to C_0$ be the normalization of a plane curve $C_0$ and $p \in C_0$ a singular point. Denote the branches of $C_0$ at $p$ by $B_1,\dots,B_k$, and let $r_i$ be the point in $B_i$ lying over $p$. If the individual branches $B_i$ of $C_0$ at $p$ are each smooth, and we set
$$
m_i = \sum_{j \neq i} \mult_p(B_i \cdot B_j)
$$
then the adjoint ideal of $C_0$ at $p$ is the ideal of functions $g$ such that $\ord_{r_i}(\nu^*g) \geq m_i$.
(Hint: The computation can be done locally analytically. Let $R = \widehat{\sO_{C_0, p}}$ be the completion of the local ring
of $C_0$ at $r_i$. The integral closure is then the product of rings $R_i = \widehat{\sO_{B_i}} \cong k[[t_i]]$,
with $R_i = R/P_i$ as $P_j$ runs over the minimal primes of $R$. The multiplicity
$m_i$ is the colength of the ideal $\sum_{j\neq i}P_j \subset R$.
)
\end{exercise}


%footer for separate chapter files

\ifx\whole\undefined
%\makeatletter\def\@biblabel#1{#1]}\makeatother
\makeatletter \def\@biblabel#1{\ignorespaces} \makeatother
\bibliographystyle{msribib}
\bibliography{slag}

%%%% EXPLANATIONS:

% f and n
% some authors have all works collected at the end

\begingroup
%\catcode`\^\active
%if ^ is followed by 
% 1:  print f, gobble the following ^ and the next character
% 0:  print n, gobble the following ^
% any other letter: normal subscript
%\makeatletter
%\def^#1{\ifx1#1f\expandafter\@gobbletwo\else
%        \ifx0#1n\expandafter\expandafter\expandafter\@gobble
%        \else\sp{#1}\fi\fi}
%\makeatother
\let\moreadhoc\relax
\def\indexintro{%An author's cited works appear at the end of the
%author's entry; for conventions
%see the List of Citations on page~\pageref{loc}.  
%\smallbreak\noindent
%The letter `f' after a page number indicates a figure, `n' a footnote.
}
\printindex[gen]
\endgroup % end of \catcode
%requires makeindex
\end{document}
\else
\fi
