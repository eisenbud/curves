%header and footer for separate chapter files

\ifx\whole\undefined
\documentclass[12pt, leqno]{book}
\usepackage{graphicx}
\input style-for-curves.sty
\usepackage{hyperref}
\usepackage{showkeys} %This shows the labels.
%\usepackage{SLAG,msribib,local}
%\usepackage{amsmath,amscd,amsthm,amssymb,amsxtra,latexsym,epsfig,epic,graphics}
%\usepackage[matrix,arrow,curve]{xy}
%\usepackage{graphicx}
%\usepackage{diagrams}
%
%%\usepackage{amsrefs}
%%%%%%%%%%%%%%%%%%%%%%%%%%%%%%%%%%%%%%%%%%
%%\textwidth16cm
%%\textheight20cm
%%\topmargin-2cm
%\oddsidemargin.8cm
%\evensidemargin1cm
%
%%%%%%Definitions
%\input preamble.tex
%\input style-for-curves.sty
%\def\TU{{\bf U}}
%\def\AA{{\mathbb A}}
%\def\BB{{\mathbb B}}
%\def\CC{{\mathbb C}}
%\def\QQ{{\mathbb Q}}
%\def\RR{{\mathbb R}}
%\def\facet{{\bf facet}}
%\def\image{{\rm image}}
%\def\cE{{\cal E}}
%\def\cF{{\cal F}}
%\def\cG{{\cal G}}
%\def\cH{{\cal H}}
%\def\cHom{{{\cal H}om}}
%\def\h{{\rm h}}
% \def\bs{{Boij-S\"oderberg{} }}
%
%\makeatletter
%\def\Ddots{\mathinner{\mkern1mu\raise\p@
%\vbox{\kern7\p@\hbox{.}}\mkern2mu
%\raise4\p@\hbox{.}\mkern2mu\raise7\p@\hbox{.}\mkern1mu}}
%\makeatother

%%
%\pagestyle{myheadings}

%\input style-for-curves.tex
%\documentclass{cambridge7A}
%\usepackage{hatcher_revised} 
%\usepackage{3264}
   
\errorcontextlines=1000
%\usepackage{makeidx}
\let\see\relax
\usepackage{makeidx}
\makeindex
% \index{word} in the doc; \index{variety!algebraic} gives variety, algebraic
% PUT a % after each \index{***}

\overfullrule=5pt
\catcode`\@\active
\def@{\mskip1.5mu} %produce a small space in math with an @

\title{Personalities of Curves}
\author{\copyright David Eisenbud and Joe Harris}
%%\includeonly{%
%0-intro,01-ChowRingDogma,02-FirstExamples,03-Grassmannians,04-GeneralGrassmannians
%,05-VectorBundlesAndChernClasses,06-LinesOnHypersurfaces,07-SingularElementsOfLinearSeries,
%08-ParameterSpaces,
%bib
%}

\date{\today}
%%\date{}
%\title{Curves}
%%{\normalsize ***Preliminary Version***}} 
%\author{David Eisenbud and Joe Harris }
%
%\begin{document}

\begin{document}
\maketitle

\pagenumbering{roman}
\setcounter{page}{5}
%\begin{5}
%\end{5}
\pagenumbering{arabic}
\tableofcontents
\fi


\chapter{Appendix: Homological commutative algebra}\label{CA appendix}

\section{Introduction: Homological commutative algebra} 

The groundwork for homological commutative algebra was laid by Arthur Cayley, David Hilbert, Frances Sowerby Macaulay, Wolfgang Gr\"obner and others, in the context of polynomial rings. The rings we encounter in studying projective geometry are mostly factor rings $S/I$ where $S$ is a polynomial ring, or localizations (and more rarely completions) of these. But after the work of Emmy Noether and her student Wolfgang Krull it was apparent that much of commutative algebra could be done axiomatically, without reference to a base polynomial ring, making the theory at once simpler and more powerful. With the work of Chevalley, Zariski, and Cohen it became clear that most of the basic properties of interest were best treated starting from the case of local rings.

With the work of Auslander, Buchsbaum and Serre, homological techniques became important. This means, roughly, focusing on modules over these rings (representation theory) and on complexes of modules, especially free resolutions. The groundwork for this extension had of course been laid by Cayley, Hilbert, Macaulay, Gr\"obner\dots, but always in the context of polynomial rings. 

In this tradition, we will begin by describing homological properties of local rings. 
However, every statement can be transposed to the setting of a standard graded algebra, graded modules, and homogeneous ideals, where by a \emph{standard graded algebra} we mean a positively graded algebra $R$ over a field $k$ such that $R_{0} = k$ and $R$ is generated as an algebra by the finite-dimensional vector space $R_{1}$. The analogue of the maximal ideal of a local ring is then the maximal homogeneous ideal (necessarily generated by $R_{1}$. We generally leave the translation to the reader. It is also possible to define a local ring to be a non-negatively graded ring $R$ whose degree zero component is a local (ungraded) ring, and to develop the whole theory in this style, in parallel with Grothendieck's idea that it is best always to work with varieties over some base scheme. But that adds enough weight to otherwise simple arguments that we have not taken this path. 

We return to the standard graded case in Chapter 18, and discuss the homological properties in terms of syzygies and Betti tables of $R$ a graded module over a polynomial ring.

All rings in this chapter will be assumed Noetherian. To indicate that $R$ is a local ring with maximal ideal $\gm$ and residue field $R/\gm = k$, we sometimes  say: ``Let $(R,\gm,k)$ be a local ring.'') We denote by $\dim R$ the \emph{Krull dimension} of $R$; that is, the maximum length of a chain of prime ideals in $R$.

\section{Modules and sheaves: local and global cohomology}\label{local coho section}

When we study projective varieties both graded modules and the sheaves associated to them play a role. As Serre explained in %cite{Serre1955}, 
the category of coherent sheaves is the category of finitely generated graded modules \emph{modulo} the subcategory of graded modules of finite length. One expression of this is the relationship between local and global cohomology, which we will explain in this section. For the sake of simplicity, we will stick with coherent sheaves on projective space.

Let $S= k[x_{0}, \dots, x_{n}]$, the homogeneous coordinate ring of $\PP^{n}$, and write. Recall that the cohomology of the sheaf $\cM$ associated to a graded $S$-module $M$ is the cohomology of the Cech complex:
$$
\cC(M): 0\to \bigoplus_{0\leq i \leq n}^{n} M_{x_{i}} \to \bigoplus_{0\leq i<j\leq n} M_{x_{i}x_{j}} \to \cdots
$$
where $M_{m}$ denotes the localization of $M$ at the powers of $m$, corresponding to the restriction of the sheaf to the open set where $m \neq 0$. The same formula works for a module 
over any finitely generated graded $k$-algebra $R$ and the corresponding sheaf on $\Proj R$.
The homology of this complex at the term
$$
\bigoplus_{0\leq j_{0}<j_{1}<\cdots<j_{i}\leq n} M_{x_{j_{0}}x_{j_{1}}\cdots x_{j_{i}}}
$$
is 
$$
H^{i}_{*}(\cM):= \bigoplus_{d\in \ZZ} H^{i}(\cM(d)),
$$
the sum of the $i$-th cohomology spaces of all twists of $\cM$.

It is easy to see that we can add $M$ itself to the left of this complex, making an augmented complex
$$
\cC'(M): 0\to M \to \bigoplus_{0\leq i \leq n}^{n} M_{x_{i}} \to \bigoplus_{0\leq i<j\leq n} M_{x_{i}x_{j}} \to \cdots
$$
We define the homology of this complex to be the \emph{local cohomology} $H_{\gm}(M)$ of $M$ with respect to the ideal $\gm := (x_{0},\dots,x_{n})$. It follows immediately that for $i\geq 1$ we have
$H^{i}_{*}(\cM) = H^{i+1}_{\gm}(M)$, and this holds degree by degree: the local cohomology module inherits a grading from $M$ and the fact that the $x_{i}$ are homogeneous and its homogeneous component of degree $d$ is
$H^{i}(\cM(d))$. Thus the modules $H_{\gm}^{i}$ have geometric meaning for $i\geq 2$. 

We can easily elucidate the meaning for $i=0,1$ as well: First, the elements of $M$ that map to 0 in $M_{x_{j}}$ are the elements annihilated by some power of $x_{j}$ so $H^{0}_{\gm}(M)$ is the set of elements annihilated by some power of every $x_{j}$, or equivalently annihilated by some power of the maximal ideal. This is precisely the set of elements that go to 0 under the natural map
$M\to H^{0}_{*}(\cM)$.  In particuar, if
$M$ is the homogeneous coordinate ring $R = S/I$ of an algebraic set $X$, then $I$ is the saturated ideal of $X$ if and only if $H^{0}_{\gm}(R) = 0$.

We can interpret $H^{0}_{\gm}(M)$ in module-theoretic terms too: Write 
$(0_{M}:\gm^{e})$ for the elements of $M$ annihilated by $\gm^{e}$, and 
$H^{0}_{\gm}(M) =(0_{M}:\gm^{\infty})$ for the union of all the $(0_{M}:\gm^{e})$. Since $M$ is Noetherian, we have
$H^{0}_{\gm}(M) =(0_{M}:\gm^{e})$ for some $e$.  Since M is graded, this can be interpreted as the maximal submodule of $M$ of finite length. 

The discussion above shows that the sequence
$$
0 \to H^{0}_{\gm} (M) \to M\to H^{0}_{*}(\cM)
$$
is exact. But from the definitions it follows at once that we can extend this to an exact sequence
$$
0 \to H^{0}_{\gm} (M) \to M\to H^{0}_{*}(\cM) \to H^{1}_{\gm} (M) \to 0.
$$

The module $H^{1}_{\gm}(M)$ also has an important interpretation: Consider again the case $M = R = S/I$, corresponding to a projective scheme $X$, and suppose for simplicity that 
$n\geq 2$, so that $H^{1}(\cO_{\PP^{n}}(d)) = 0$ for all $d$. In this case the long exact sequence in cohomology associated to 
$$
0\to \cI_{X} \to \cO_{\PP^{n}} \to \cO_{X}\to 0,
$$
begins
$$
0\to H^{0}_{*}(\cI_{X}) \to H^{0}_{*}(\cO_{\PP^{n}}) \to H^{0}_{*}(\cO_{X})\to H^{1}_{*}(\cI_{X}) \to 0.
$$
Since the image of $H^{0}_{*}(\cO_{\PP^{n}}) \to H^{0}_{*}(\cO_{X})$ is  the same as
the image of $S/I \to H^{0}_{*}(\cO_{X})$ (both are equal to $S_{X}$), 
this shows that $H^{1}_{\gm}(S/I) = H^{1}_{*}(\cI_{X})$. This module will play an important role in the treatment of linkage, below.

The local cohomology modules  are typically artinian, but not finitely generated. But there is an 
expression of the cohomology of a sheaf or module in terms of finitely generated modules. If
$M = \oplus_{i} M_{i}$ is a graded module over a standard graded polynomial ring $S = k[x_{0}..x_{n}]$, and $M$ has infinitely many nonzero components, then the dual 
$\Hom_{k}(M,) = \prod_{i} M_{i}$ is naturally a module over the power series ring 
$\hat S = k[[x_{0}..x_{n}]]$, and thus is never finitely generated over $R$. But we can also form the
\emph{graded dual}, $\Hom_{k,gr}(M,k) := \oplus_{i}\Hom_{k}(M_{i},)$, which is again an $S$-module.

\begin{theorem}[Local duality]\label{local duality}
Let $S = k[x_{0}..x_{n}]$ be a standard graded polynomial ring over $k$. If $M$ is a finitely generated graded $S$-module, then
$$
H^{i}_{\gm} M = \Hom_{gr,k}(\Ext_{S}^{n+1-i}(M,S(-n-1), k).
$$
\end{theorem}


\section{Regular local rings and syzygies}
Let $(R,\gm,k)$ be a local ring. By the Principal Ideal Theorem \cite[]{E}, the maximal ideal $\gm$ cannot be generated by  $<\dim R$ elements.

\begin{definition}
 We say that $R$ is \emph{regular} if $\gm$ can be generated by $\dim R$ elements.
\end{definition}

This deceptively simple property was first identified as important by Krull, and later recognized by Zariski as the appropriate algebraic expression of nonsingularity: A point $p$ on a scheme $X$ is called nonsingular if and only if the local ring $R = \sO_{X,p}$ is \emph{regular}. This is justified by the fact that if $R$ is the local ring of a point $p$ on a variety over an algebraically closed field, then the cotangent space to $p$ is naturally identified with the 
$k$-vector space $\gm/\gm^{2}$, whose dimension is, by Nakayama's Lemma, the minimal number of generators of $\gm$. 

The analogue of ``regular'' in the case of a standard graded algebra $R$ is that $R$ is isomorphic to the 
polynomial ring on a basis of $R_{1}$. Indeed, the localization of $R$ at the maximal homogeneous ideal is regular in the local sense if and only if this condition is satisfied.

For the regularity of $\sO_{X,p}$ to be a reasonable algebraic analogue of non-singularity, it should of course imply
that $X$ is reduced and irreducible at $p$; that is, a regular local ring should be a domain. This was proven by Krull, before the work of Zariski:

\begin{proposition}
 If $R$ is regular local ring is an integral domain; that is, 0 is a prime ideal.
\end{proposition}
\begin{proof}
 We do induction on the dimension. If $\dim R = 0$ then by definition $\gm$ is generated by 0 elements, so $R = k$,
 a field. If $\dim R>0$ then by the prime avoidance theorem \cite[]{E} there is an element $x$ not contained in trhe union of $\gm^{2}$ and the minimal primes of $R$. By the Principal Ideal Theorem, $R/(x)$ has dimension $\dim R -1$ and the maximal ideal $\gm/(x)$ has $\dim R-1$ generators, so $R/(x)$ is again regular.
 
 By induction, $(x)$ is a prime ideal of $R$ that is not a minimal prime. If $Q$ is a minimal prime contained in $(x)$,
 then $q\in Q$ implies $q = q'x$ for some $q'\in R$, and since $Q$ is prime, we have $q'\in Q$. Thus
 $Q = Qx$, and it follows from Nakayama's Lemma that $Q=0$, so $R$ is a domain.
\end{proof}

This result has a consequence that leads to an important definition:

\begin{corollary}
 Let $R$ be a regular local ring of dimension $d$. If $x_{1}, \dots, x_{d}$  generate $\gm$, then 
 $x_{i+1}$ is a nonzerodivisor modulo $(x_{1}, \dots, x_{i})$ for every $i = 1,\dots, n$
\end{corollary}
\begin{proof}
 Obvious, since $R/(x_{1},\dots, x_{i})$ is again regular, and thus a domain, and $x_{i+1}\notin (x_{1},\dots,x_{i})$.
\end{proof}

We say that a sequence of elements in the maximal ideal of $R$ that satisfies the condition of the Corollary is a \emph{regular sequence}. It is convenient to extend
the definition to modules:

\begin{definition}
 Let $R$ be a commutative ring, and let $M$ be an $R$-module. A sequence of
 elements $x_{1}, \dots, x_{n}\in R$ is called a \emph{regular sequence on $M$}, or an
 \emph{$M$-sequence}, if
 $x_{i}$ is a nonzerodivisor on $M/(x_{1}, \dots x_{i-1})M$  for all $i= 1,\dots, n$, and 
 $(x_{1}, \dots, x_{n})M \neq M$.
\end{definition}

Note that if $(R,\gm,k)$ is a local ring, $(x_{1},\dots,x_{n})\subset \gm$ and $M$ is finitely generated, then he last condition is auttomatic from Nakayama's Lemma.

 Recall
that if $M$ is a finitely generated $R$-module, then an \emph{$R$-free resolution} of $M$ is a sequence of free modules and maps
$$
\FF:\qquad F_{0} \lTo^{d_{1}} F_{1}\lTo^{d_{2}} F_{2}\cdots,
$$
an \emph{augmentation} map $F_{0} \rOnto^{d_{0}} M$ such that the kernel of $d_{i}$ is equal to the image of $d_{i+1}$ for every $i$. We say that the resolution is \emph{finite of length $n$} if $F_{n+1}= 0$ but $F_{n}\neq 0$.The resolution is called \emph{minimal} if the $d_{i}(F_{i}) \subset \gm F_{i-1}$ for all $i$; it follows from Nakayama's Lemma that this is the case if and only if the rank of $F_{i}$ is equal to the minimal
number of generates of $\ker d_{i-1}$ for all $i$. 

The minimal resolution of a module is a direct summand of any resolution; and it follows that any two minimal resolutions of a module are isomorphic~\cite[Theorem ***]{E}. 

\begin{example} The Koszul complex of a sequence $x_{1}, \dots, x_{n}$: 
Consider first a single element $x = x_{1}\in R$. We define the Koszul complex on $x$, denoted $\KK(x;R)$, to be the complex
$$
\KK(x; R): \quad R \lTo^{x} R \lTo 0.
$$
This complex is a minimal free resolution of $R/(x)$ if and only if $x$ is a nonzerodivisor contained in the maximal ideal of $R$. Observe that this is also the condition for the one element sequence $x$ to be a regular sequence.

Next consider 
a pair of elements $x_{1},x_{2}\in R$. The Koszul complex
on $x_{1},x_{2}$ is the $R$-free complex
$$
\KK(x_{1}, x_{2}; R): \quad R \lTo^{
\phi_{1}= \begin{pmatrix}
x_{1}&x_{2} 
\end{pmatrix}
} R^{2}\lTo ^{
\phi_{2}=\begin{pmatrix}
x_{2}\\-x_{1} 
\end{pmatrix}
}
R\lTo 0.
$$
It is obvious that $\coker \phi_{1} = R/(x_{1}, x_{2})$. Also $\ker \phi_{2}$ is the annihilator of the ideal $(x,y)$, and it follows from the theory of associated primes that this is 0 if and only if the ideal $(x_{1}, x_{2})$ contains a nonzerodivisor. For simplicity, let us assume that $x_{1}$ is a nonzerodivisor itself, although this is not actually necessary. The kernel of $\phi_{1}$ obviously consists of the elements $(y_{2},-y_{1})\in R^{2}$ such that $y_{2}x_{1} = y_{1}x_{2}$. Since we have assumed that $x_{1}$ is a nonzerodivisor,
the element $y_{2}$ is uniquely determined by $y_{1}$ such that $y_{1}x_{2} \in (x_{1})$, usually written 
$y_{1}\in ((x_{1}):x_{2})$. Thus, given that $x_{1}$ is a nonzerodivisor,
the kernel of $\phi_{1} $ is equal to the image of $\phi_{2}$ if and only if $x_{2}$ is a nonzerodivisor mod $x_{1}$; that is if and only if $x_{1},x_{2}$ is a regular sequence. 

Note that the right-hand term $R^{1}$ of $\KK(x_{1}, x_{2}; R)$ is somehow naturally indexed by the pair of elements $x_{1},x_{2}$; rather pedantically, we could write it as $\wedge^{2}(R^{2})$. This has the advantage that $\phi_{2}$ can be described as the result of extending $\phi_{1}$ to be a degree $-1$ derivation of the exterior algebra: if we denote the basis  elements of $R^{2}$ as $e_{1}, e_{2}$ so that $\phi_{1}(e_{i}) = x_{i}$, then
$\phi_{2}(e_{1}\wedge e_{2}) = \phi_{1}(e_{1})e_{2} - e_{1} \phi_{1}(e_{2})$. 
Here the negative sign comes because we have commuted the derivation, of degree $-1$, with an element of
odd degree, $e_{1}$. This leads us to rewrite the Koszul complex in the suggestive form:
$$
\KK(x_{1}, x_{2}; R): \quad \bigwedge^{0}R^{2} \lTo^{
\phi_{1}= \begin{pmatrix}
x_{1}&x_{2} 
\end{pmatrix}
} \bigwedge^{1}R^{2}\lTo ^{
\phi_{2}=\begin{pmatrix}
x_{2}\\-x_{1} 
\end{pmatrix}
}
\bigwedge^{2}R^{2}\lTo 0.
$$

In general the Koszul complex of a sequence of elements
$\KK(x_{1}, \dots, x_{n}; R)$ is defined to be the exterior algebra of $R^{n}= \oplus_{i=1}^{n} Re_{i}$, with first differential
$$
\bigwedge^{0}R^{n} = R \lTo ^{
\phi_{1 = }\begin{pmatrix}
 x_{1}&\cdots&x_{n} 
\end{pmatrix}
}
R^{n}=\bigwedge^{1}(R^{n}) 
$$
and the other differentials defined to extend $\phi_{1}$ to be a derivation of degree $-1$, so that 
$$
\phi_{m}(e_{i_1}\wedge \cdots \wedge e_{i_{m}})
= \sum_{j= 1}^{m} (-1)^{j-1}x_{i_{j}} e_{i_{1}}\wedge\cdots \wedge \widehat{e_{i_{j}}}\wedge \cdots \wedge e_{i_{m}}).
$$
It is easy to check that $\phi_{m-1}\phi_{m} = 0$ for all $m\geq 1$, so $\KK(x_{1}, \dots, x_{n}; R)$ is a complex.

There is a surprisingly simple necessary and sufficient condition for 
$\KK(x_{1}, \dots, x_{n};R)$
to be a minimal free resolution of $\coker \phi_{1} = R/(x_{1}, \dots, x_{n})$ \cite[]{E}:

\begin{theorem} If $(R,\gm)$ is a local ring, and $x_{1}, \dots x_{n} \in R$, then
 the Koszul complex $\KK(x_{1}, \dots, x_{n};R)$ is a minimal free resolution (of $R/(x_{1}, \dots, x_{n})$) if and only if 
$x_{1},\dots x_{n}$ is a regular sequence in $R$.\qed
\end{theorem}
This result also holds in the graded polynomial ring case, if we assume that the $x_{i}$ are all of
strictly positive degree. For a proof, see \cite[Theorem 17.6]{E}.
\end{example}


Here is the homological characterization of regularity:

%\begin{theorem}(Auslander, Buchsbaum, Serre \cite{}
% A local ring $R$ is regular if and only if the following equivalent statements hold:
% 
% every finitely generated $R$-module has an $R$-free resolution of finite length; and indeed of length $\leq \dim R$.
%\end{theorem}

\begin{theorem}[Auslander, Buchsbaum, Serre \cite{}]\label{regularity characterized}\label{ABS}
The following conditions on a $d$-dimensional local Noetherian ring $R$ with residue field $k$ are equivalent:
\begin{enumerate}
 \item $R$ is regular.
\item Every finitely generated $R$-module has a finite free resolution.
\item Every finitely generated $R$-module has a  free resolution of length at most $d$.
\item A minimal set of generators $x_{1},\dots, x_{d}$ of $\gm$ is a regular sequence; equivialently,
the Koszul complex $\KK(x_{1},\dots, x_{d};R)$ is the minimal  $R$-free resolution of  $k$.
\item $\Ext_R^{i}(k,M) = 0$ for all $i>d$ and all finitely generated modules $M$.
\item $\Ext_R^{d+1}(k,k) = 0$.
\end{enumerate}
\end{theorem}

Perhaps the most interesting part of this is the implication 1) $\to$ 3), a vast extension of Hilbert's Syzygy Theorem.
Given Theorem ***, and basic facts about the functor Tor, it is surprisingly easy to prove:

\begin{proof} [Proof that 1) $\to$ 3)] Suppose that $(R,\gm,k)$ is a regular local ring $M$ be a finitely generated
 $R$-module. Let $\FF$ be a minimal free resolution of $M$, so that the differentials of the complex of vector
 spaces $k\otimes_{R}\FF$ are all 0. It follows that the length of $\FF$ is the maximal $i$ such that
 $$
 H_{i}(k\otimes_{R}\FF) = \Tor_{i}^{R}(k,M) = 0.
 $$
 However, we can compute $\Tor_{i}^{R}(k,M)$ using a resolution of $k$. By Corollary\ref{} and 
 Theorem~\ref{},  the Koszul complex of a minimal sequence of generators of $\gm$  is the minimal free resolution of $k$, and it has length $d$, so $\Tor_{i}^{R}(k,M) = 0$ for $i>d$ as required.
\end{proof}

 The homological characterization of regularity enabled the proof of long-standing conjectures:
\begin{theorem} \cite{AB} If $R$ is a regular local ring then:
\begin{itemize}
 \item Every localization of $R$ at a prime ideal is again a regular local ring
  \item  $R$ is a unique factorization  domain
\end{itemize}
 \end{theorem}
 
 \section{Projective dimension}
The first new invariant that we can read from the minimal $S$-free resolution of a module $M$ is its length; that is, the number of nonzero maps, which is finite by the Syzygy Theorem. This is called the \emph{projective dimension} of $M$ as an $S$-module, written $pd_{S}M$. An older name, in some ways more suitable, was \emph{homological codimension}; this is justified by the following results:

\begin{proposition}\label{pd lower bound}
If $M$ is a graded $S$-module then $\pd(M)$ is at least the codimension of the support of $M$.
\end{proposition}

In case $\pd(M)$ is equal to the codimension of the support of $M$, we say that $M$ is a
Cohen-Macaulay $S$-module, or equivalently that the sheaf $\widetilde M$ is 
\emph{arithmetically Cohen-Macaulay}. When $M = S_{X}$, the homogeneous coordinate ring of a projective scheme $X$, we say that $X$ is itself is arithmetically Cohen-Macaulay. From the examples above we see that plane curves, and also the twisted cubic, are Cohen-Macaulay.

A famous result of Auslander and Buchsbaum clarifies the meaning of projective dimension. We define the 
\emph{depth} of $M$ to be the maximum length $\ell$ of a \emph{regular sequence on $M$}; that is, a sequence $G_{1},\dots,G_{\ell}$ of homogeneous forms of strictly positive degree such that 
\begin{align*}
G_{1} &\hbox{ is a nonzerodivisor on } M;\\
G_{2} &\hbox{ is a nonzerodivisor on } M/G_{1}M;\\
\vdots&\phantom{\hbox{ is a nonzerodivisor on } }\vdots\\
G_{\ell} &\hbox{ is a nonzerodivisor on } M/(G_{1},\dots,G_{\ell-1})M.
\end{align*}

\begin{theorem}
If $M$ is a finitely generated graded module over the polynomial ring $S := \CC[x_{0},\dots,x_{n}]$, and $M$ has depth $\ell$, then the projective dimension of $M$ is $n+1-\ell$.
\end{theorem}


Suppose again that $M$ is a finitely generated graded module. Every associated prime of $M$ must then be homogeneous, and, since the set of zerodivisors on $M$ is the union of all the associated primes,  there is form $G_{1}$ of positive degree that is a nonzerodivisor on $M$ if and only if the maximal ideal $\gm$ is not an associated prime of $M$, or equivalently $M$ contains no element annihilated by $\gm$ \fix{this uses prime avoidance too; probably should have a reference}. Since $H^{1}_{gm}$ is the submodule of all elements of $M$ annihilated by a power of $\gm$, we see that
the projective dimension of $M$ is $< n+1$ if and only $H^{0}_{\gm}(M) = 0$, or equivalently
$M$ is a submodule of $H^{0}_{*}(\widetilde M)$.

Though this is not obvious from the definition, all maximal regular sequences on $M$ have the same length, and if the depth of $M$ is $\ell$ then a sequence of general linear forms of length $\ell$ is a regular sequence. This makes the depth easier to compute. Even better, the depth has an interpretation in terms of  cohomology:

\begin{theorem}\label{lc char of depth}
Let $M$ be a finitely generated graded $S$-module. The depth of $M$ is the smallest integer $i$ such that $H^{i}_{\gm}(M) \neq 0$.
\end{theorem}
 
 
\begin{exercise}
 Prove Theorem~\ref{lc char of depth} by induction on the length of a maximal regular sequence.
\end{exercise}
We can easily translate this into global cohomology in the case of a module of twisted global sections:

\begin{theorem}\label{Auslander-Buchsbaum} Suppose that $X\subset \PP^{n}$ is a  subscheme without 0-dimensional (isolated or embedded) components. The module $M = \oplus_{t\in \ZZ}H^{0}(\cO_{X}(t))$
is finitely generated, and $\depth M$ is the smallest integer $\ell\geq 2$ such that 
$H^{\ell-1}(\cO_{X}(t)) \neq 0$  for some $t$. The homogeneous coordinate ring of $X$ is equal to $M$ if
$\oplus_{t\in \ZZ}H^{1}(\cI_{X}(t)) = 0$ and has depth exactly 1 otherwise.
\end{theorem}
\begin{proof}
For the first statement, note that $H^{\ell}_{\gm}M = \oplus_{t}H^{\ell-1}(\cO_{X}(t))$. For the second statement
use the exact sequence
\end{proof}


 \section{Cohen-Macaulay rings}

 It is quite possible for a local ring $(R,\gm, k)$ of dimension $d$ to contain a regular sequence of length $d$
 without being regular; an easy example is the 2-dimensional local ring
 $$
 R = k[[x,y, z]]/(y^{4}-x^{3}z) \cong k[[s^{4}, s^{3}t, t^{4}]]
 $$
In fact, we claim that $z,x$ is such a regular sequence. Since the ring $R$ is 2-dimensional, and the maximal ideal requires 3 generators $x,y,z$, the ring $R$ is not regular. 

\begin{definition}
A local ring $(R,\gm,k)$ of dimension $d$ is said to be \emph{Cohen-Macaulay}  if $\gm$ contains a regular sequence of length $d$.
\end{definition}
 Note that every 0-dimensional (that is, Artinian) local ring is automatically Cohen-Macaulay.
 
The Cohen-Macaulay condition is made easier to check by the following important homological interpretation:

\begin{theorem}\label{depth}
Let $(R,\gm, k)$ be a local ring, and let $I\subset \gm$ be an ideal. 
Let $M$ be a finitely generated 
 $R$-module. Every maximal $M$-sequence in $I$ has the same length, called the \emph{depth of $I$ on $R$}, and this number is the smallest integer $i$ such that
 $\Ext^{i}_{R}(R/I, M)\neq 0$. Moreover, if $x_{1}, \dots, x_{i}$ is a maximal $M$-sequence
 in $I$ then $\Hom_{R}(R/I, M/(x_{1}, \dots, x_{i}))= \Ext^{i}_{R}(R/I, M)$ is independent of the maximal regular sequence.
\end{theorem}
 
 
\begin{proof} Suppose that $x_{1}, \dots, x_{i}\in \gm$ is a maximal regular sequence
on $M$. We will show by induction on $i$ that $\Ext^{i}_{R}(R/I, M) = \Hom(R/I,  M/(x_{1}, \dots, x_{i}) \neq 0$ and
that $\Ext^{j}_{R}(R/I,M) = 0$ for $j<i$.

First suppose $i=0$; that is, every element of $I$ is
a zero-divisor on $M$. This means that $I$ is contained in the union of the finitely
many associated primes of $M$. By the Prime Avoidance Lemma \cite[****]{E} I is contained in a single associated prime of $M$, and thus $I$ annihilates a nonzero  element $m\in M$
of $M$, so that $\Hom(R/I,M)$ contains a nonzero homomorphism sending the class of 1 to $m$.

Next suppose that $i>0$. Since $x_{2}, \dots, x_{i}$ is a maximal regular sequence on $M/(x_{1})M$ we see by induction. that $\Ext_R^{i-1}(R/I, M/x_{1}M) \neq 0$ and $\Ext^{j}_{R}(R/I,M/x_{1}M) = 0$ for $j<i-1$. From the short exact sequence
$$
0\to M\rTo^{x_{1}}M \rTo M/x_{1}M \to 0
$$
we get a long exact sequence in $\Ext_R$ containing the terms
\begin{align*}
 &\Ext_R^{j}(R/I,M) \rTo^{0} \Ext_R^{j}(R/I,M) \rTo \Ext_R^{j}(R/I,M/x_{1}M) \rTo\\ 
 &\Ext_R^{j+1}(R/I,M)\rTo^{0} \Ext_R^{j+1}(R/I,M) \rTo \cdots,
\end{align*}
where the maps marked 0 vanish because $x_{1}$ annihilates $R/I$; that is, we have 
short exact sequences
 $$
0\to \Ext_R^{j}(R/I,M) \rTo\Ext_R^{j}(R/I,M/x_{1}M) \rTo \Ext_R^{j+1}(R/I,M)\to 0.
 $$
By induction, the middle term of this sequence vanishes for $j<i-1$, so 
$\Ext_R^{j}(R/I,M) = 0$ for $j<i$ and 
$$
\Ext_R^{i}(R/I,M) \cong \Ext_R^{i-1}(R/I,M/x_{1}M) \cong \Hom(R/I, M/(x_{1}, \dots x_{i}))\neq 0
$$
as required.
\end{proof}

\begin{exercise}
 Use Theorem~\ref{depth} to check that the ring
 $$
 R = k[[s^{4}, s^{3}t, st^{3}, t^{4}]]
 $$
is \emph{not} Cohen-Macaulay.
\end{exercise}

The Cohen-Macaulay property has a homological interpretation that we shall use:

\begin{theorem} \label{lower bound for pd}
 Let $(R,\gm,k)$ be a local ring, and suppose that $S\to R$ is a map of local  rings such that $S$ is a regular local ring and $R$ is a finitely generated $S$-module. The length of a minimal resolution of $R$ as an $S$ module
 is at least $\dim S - \dim R$; and it is equal to this value if and only if the 
 ring $R$ is Cohen-Macaulay.
 \end{theorem}
  
  
  By Proposition~\ref{pd lower bound}, if $C\subset \PP^{n}$ is 1-dimensional, then the projective dimension of $S_{C}$ is at least $n-1$. But we can be much more precise. Recall that a curve $C\subset \PP^{n}$ is said to be \emph{projectively normal} if the homogeneous coordinate ring of $C$ is integrally closed (which implies, in particular, that $C$ is smooth).

\begin{theorem}
 Let $C\subset \PP^{n}$ be a purely 1-dimensional subscheme. The projective dimension of the homogeneous coordinate ring $S_{C}$ of $C$  is
$$
 pd_{S}S_{C} = 
\begin{cases}
n-1 &\hbox{if $H^{1}(\sI_{C}(t)) = 0$ for all $t\in \ZZ$}\\
n &\hbox{otherwise}.
\end{cases}.
$$
Thus in the first case $S_{C}$ is Cohen-Macaulay. In particular, if $C$ is a smooth curve, the $pd_{S}(S_{C}) = n-1$ if and only if $C$ is projectively normal. \fix{we used Serre's Criterion. Ref?}
\end{theorem}

Here is a version that gives a measure of how far $S_C$ is  from being Cohen-Macaulay:

\begin{theorem}
Let $C\subset \PP^{n}$ be a purely 1-dimensional subscheme, and let  
$$
\FF: F_{0}\lTo^{d_{1}} F_{1}\lTo^{d_{2}} F_{2}\lTo \cdots \lTo F_{n-1}\lTo {d_{n}}F_{n}\lTo F_{n+1} \lTo 0
$$
be the minimal $S$-free resolution of the homogeneous coordinate ring of $C$. We have $F_{n+1}=0$, and 
$$
\oplus_{t\in \ZZ} (H^{1}\sI_{C}(t)) = \Hom_{\CC}(\Ext^{n}(S_{C}, S(-n-1)),\CC)
$$ 
which is sometimes called the \emph{Rao module} of $C$. Thus, up to a shift in grading,
the Rao module of $C$ is the vector space dual of the cokernel of the dual $d_{n}^{*}: F_{n-1}^{*}\to F_{n}$. This is a graded module of finite length.
\end{theorem}


\fix{Having introduced local coho, this is pretty much done.}


\begin{theorem}
 Let $C\subset \PP^{n}$ be a  curve (or more generally a purely 1-dimensional subscheme). The homogeneous coordinate ring $S_{C}= S/I_{C}$ of $C$ is Cohen-Macaulay if and only if
 $H^{1}(\cI_{C}(d)) = 0 $ for all $d$.
\end{theorem}

\begin{proof}
 Except in the trivial case $n=1$ we have $H^{1}(\cO_{\PP^{n}} (d))$ for all $d$, and since 
$C$ is supposed purely 1-dimensional we have $H^{0}(\cO_{C}(d) )= 0$ for $d<0$, so from the
exact sequence 
$$
0\to \cI_{c}\to \cO_{\PP^{n}} \to \cO_{C}\to 0
$$
we deduce that $H^{1}(\cO_{C}(d))= 0$ for all $d<0$ in any case.


By Theorem~\ref{lower bound for pd}, the projective dimension of $S_{C}$ is at least $\dim S -\dim S_{c}= n-1$.

 Let $\gm$ be the maximal homogeneous ideal $(x_{0},\dots, x_{n})$ of 
 the homogeneous coordinate ring $S$ of $\PP^{n}$.
 By the Auslander-Buchsbaum formula \fix{this should be first, and we should give the graded version too(x} \ref{AB} the projective dimension
 of $S_{C}$ is $n+1$ (the dimension of  $S$ minus the depth of $\gm S_{C}$, that is, the length of a maximal regular sequence in $S_{C}$, which can be taken to be homogeneous. 
 
 If a finitely generated graded $S$-module $M$ has depth $>0$ (that is, $\gm$ contains a homogeneous nonzerdivisor on $M$) then clearly no nonzero element of $M$ is annihilated by $\gm$. The converse of this statement is a consequence of the theory of primary decomposition. Further,
 writing $\tilde M$ for the associated coherent sheaf on $\PP^{n}$, we have a map
 $M \to \oplus_{d\in \ZZ}H^{0}(\tilde M(d))$ whose kernel is precisely the set of elements annihilated by
 some power of $\gm$.
 
 If $J$ is the homogeneous ideal of any scheme, then by definition $J$ is saturated; that is, no element is annihilated by $\gm$, or equivalently $\gm$ is not an associated prime ideal. Thus $S/I_{C}$ has projective dimension $\leq n = \dim S - 1$.
To simplify the notation, if $\cF$ is a coherent sheaf, then we write $H_{*}^{i}(\cF)$ for $\bigoplus_{i}H^{i}(\cF(i))$. 

Suppose now that $f\in S_{C}$ is a nonzerodivisor of degree $d$ and let $H$ be the hypersurface in $\PP^{n}$ that it defines. By Theorem *** we must decide whether $S_{C}/fS_{C}$ contains a non-zerodivisor, that is, whether 
$S_{C}/fS_{C}$ is saturated, or, equivalently, whether the map
$$
\alpha: S_{C}/fS_{C} \to H_{*}^{0}(\widetilde{S_{C}/fS_{C}})  = H_{*}^{0}(\cO_{H\cap C}). 
$$
is an injection.


The diagram below has exact rows and columns, 
\begin{diagram}[small]
&& 0&&0& \\
&&\dTo&&\dTo\\
0&\rTo& S_{C}&\rTo^{f}&S_{C}(d)& \rTo &S_{C}/(f)(d)&\rTo&0\\
&& \dTo&& \dTo&& \dTo^{\alpha}\\
0&\rTo&H_{*}^{0}(\cO_{C})&\rTo^{f}&H_{*}^{0}(\cO_{C}(d)) &\rTo& H_{*}^{0}(\cO_{H\cap C}(d))&\rTo& 0\\
&& \dTo&& \dTo&& \dTo\\
0&\rTo&H_{*}^{1}(\cI_{C})&\rTo^{f}&H_{*}^{1}(\cI_{C}(d)) &\rTo& H_{*}^{1}(\cI_{H\cap C}(d))\\
&& \dTo&& \dTo\\
&&0&&0
\end{diagram}
and it follows from a diagram chase (the ``snake lemma'') that the kernel of $\alpha$ is the same as the kernel of 
$$
H_{*}^{1}(\cI_{C}) \rTo^{f} H_{*}^{1}(\cI_{C}(d)).
$$
 By Serre's vanishing theorem, $H_{*}^{1}(\cI_{C})$ is zero in high degree, and since multiplication by $f$ raises the degree, its kernel is 0 if and only if $H_{*}^{1}(\cI_{C})$ is zero, completing the proof.
\end{proof}

\section{Gorenstein rings and duality}
Intermediate between the class of Cohen-Macaulay rings and the class of regular rings is the class of Gorenstein rings. Roughly speaking, they are the rings for which duality is the simplest. As we shall see, all complete intersections are Gorenstein, a fact that will be central to 
our study of linkage, below.

\begin{definition}
A local ring $(R,\gm,k)$ of dimension $d$  is said to be \emph{Gorenstein} if it is Cohen-Macaulay and 
$\Ext_R^{d}(k, R) =k$. A (not necessarily) local ring is Gorenstein if all its localizations
are Gorenstein.
\end{definition}

If $(R,\gm,k)$ is regular then the resolution of $k$ is the Koszul complex of any set of $d$ generators of $\gm$, and we see directly that
$\Ext_R^{d}(k, R) =k$, so $R$ is Gorenstein.

\begin{proposition}
 If $(R,\gm,k)$ is Cohen-Macaulay, and $x_{1}, \dots x_{s} \in R$ is a regular sequence, then $R$ is Gorenstein if and only if $R/(x_{1}, \dots x_{s})$ is Gorenstein.
\end{proposition}
 
\begin{proof}
Since every regular sequence in $R$ is part of a maximal regular sequence, it suffices to prove the result when $s = d$, the dimension of $R$ so that $\overline R = R/(x_{1}, \dots x_{d})$ is Artinian. Since $R$ is Cohen-Macaulay, the smallest $i$ such that 
$\Ext_{R}^{i}(k, R) \neq 0$ is $d$, so by Lemma~\ref{Ext and nzd}, we see that 
$$
\Ext_{R}^{d}(k, R) = \Ext_{R}^{0}(k, \overline R) = Hom_{\overline R}(k,\overline R)
$$
proving the Proposition.
\end{proof}
\begin{lemma}\label{Ext and nzd}
 If $(R,\gm,k)$ is a local ring, and $x\in \gm$ is a nonzerodivisor on $N$ that annihilates $M$, and $i$ is the smallest index such that $ \Ext_R^{i}(M, N) \neq 0$, then
 $$
 \Ext_R^{i}(M, N) = \Ext_{R}^{i-1}(M, N/xN).
 $$
for all $i$. 
 \end{lemma}
\begin{proof}
The element $x$ annihilates all the $Ext^{j}_{R}(M,N)$ because it annihilates $M$. 
The short exact sequence $0\to N\rTo^{x}N \rTo N/xN \to 0$ gives rise to a long exact sequence containing the terms
and
$$
0 =  Ext_{R}^{i-1}(M,N) \rTo Ext_{R}^{i-1}(M,N/xN) \rTo Ext_{R}^{i}(M,N) \rTo^{0} \cdots.
$$
\end{proof}

The algebraic version of the canonical module of a scheme is usually called the \emph{dualizing module}:
\begin{definition}
 Let $(R, \gm, k)$ be a local Cohen-Macaulay ring of dimension $d$. A dualizing module for $R$ is a Cohen-Macaulay
 $R$ module with $\dim M = d$ such that $\Ext^{d}_{R}(M,k)\cong k$.
\end{definition}

\begin{proposition}
 Let $R$ be a local Cohen-Macaulay ring. Any two canonical modules for $R$ are isomorphic. Moreover,
 if $R = S/I$, with $S$ regular, then $\Ext_{S}^{\codim R}(R,S)$ is a canonical module for $R$.
\end{proposition}

\begin{proof}
 ******
\end{proof}

We shall see that $(R,\gm,k)$ is Gorenstein if and only if it is Cohen-Macaulay and 
$R$ itself is a dualizing module. We begin with the 0-dimensional case, where we can identify the dualizing module with the injective hull of the residue field. Recall that $E$ can be characterized as a module $E$ containing a copy of $k$ such that 
 $k$ is \emph{essential} in $E$; that is, every nonzero submodule of $E$ meets $k$; and
$E$ maximal with this property in the sense that if $E\subsetneq E'$, then $k$ is not essential
in $E$. Such a module always exists, by Zorn's lemma, and it is not difficult to show that it it unique up to isomorphism. Except for rings of dimension 0, it is never finitely generated.

\begin{theorem}\label{duality for Gor}
If $(R,\gm,k)$ is a local ring of dimension $0$, and $E$ is the injective hull of $k$, then 
$(-)^{\vee} := \Hom_{R}(-,E)$ is a perfect duality on modules of finite length. That is,
 $(-)^{\vee}$ is a contravariant equivalence of categories. Moreover, for any module $M$ of finite length we have
\begin{enumerate}
\item $\length\ M^{\vee} = \length\ M$; in particular, $E$ is a module of finite length $= \length\ R$.
\item The natural map $\nu_{M}: M\to M^{\vee\vee}$ is an isomorphism.
\item The ring $R$ is Gorenstein if and only if $R$ has a unique minimal nonzero ideal.
\item The ring $R$ is Gorenstein if and only if $R\cong E$.
\end{enumerate}
\end{theorem}

\begin{proof}
First, since $E$ is injective the functor $Hom_{R}(-,E)$ is exact. Since $k$ is essential in $E$, 
the largest submodule of $E$ annihilated by $\gm$ must be $k$ itself.

We  prove both (1) and (2) by induction on the length of $M$. If $\length M = 1$, then $M = k$.
The previous remark shows that  
$k^{\vee} = k$, so $k^{\vee\vee}= k$, proving (1). Choosing a generator $\phi$ of
$k^{\vee}$ and a generator $\alpha$ of $k$, we see that 
$\nu_{k}$ takes $\alpha$ to the map sending $\phi$ to $\phi(\alpha) \neq 0$. Since $k$ is a simple module, $\nu_{k}: k\to k$ is a monomorphism, and thus an isomorphism, as requireed.

  Now suppose by induction that (1) and (2) are true for all modules $M'$ of length at most $j$, and that $M$ is a module of length $j+1$.
  
Any minimal nonzero submodule of $M$ is isomorphic to $k$, so we may choose an exact sequence
$0\to k\to M\to M'\to 0$. Applying $(-)^{\vee}$ we get an exact sequence
$0\to M'^{\vee}\to M^{\vee }\to k^{\vee} \to 0$, proving (1) for $M$, and a diagram
$$
\begin{diagram}[small]
 0&\rTo& k&\rTo& M&\rTo& M'&\rTo &0\\
 &&\dTo^{\nu_{k}}&&\dTo^{\nu_{M}} &&\dTo^{\nu_{M'}}\\
 0&\rTo& k^{\vee\vee}&\rTo& M^{\vee\vee}&\rTo& M'^{\vee\vee}&\rTo& 0\\
\end{diagram}
$$
with exact rows, proving (2) for $M$, and completing the induction.

To prove (3) we note that if $R$  has a unique minimal ideal $I \cong k$ if and only if
$\Hom_{R}(k,R) = k$. Since any 0-dimensional ring is Cohen-Macaulay, this is equivalent to the Gorenstein property. 

Finally we prove (4): Since $E$ has unique miminal ideal in any case, we see that
$R\cong E$ implies that $R$ is Gorenstein. Conversely, suppose that $I\cong k$ is the
unique minimal nonzero ideal of $R$. The unique map
$I\hookleftarrow E$ extends to a map $\phi: R\to E$. If $\ker \phi$ were nonzero it would contain $I$, so $\phi$ is a monomorphism. Moreover,
$\length R = \length R^{\vee} = E$, so $\phi$ is surjective as well.
\end{proof}

\begin{fact}
Conditions (2) and (4) of Theorem~\ref{duality for Gor} have extensions to the higher dimensional case, though we will not need to use them:

\begin{theorem}\label{Gorenstein characterized}
 Suppose that $(R, \gm, k)$ is a $d$-dimensional local Noetherian ring $R$. the following conditions are equivalent:
\begin{enumerate}
\item $R$ is Gorenstein.
\item $R$ has finite injective dimension (equivalently, injective dimension $d$) as an $R$-module.
\item $R$ is a Cohen-Macaulay ring and the functor $\Hom_{R}(-,R)$ is a perfect duality on the category of maximal Cohen-Macaulay modules.
  \end{enumerate}  
  \qed
\end{theorem}
\end{fact}

Note that the equivalence of 1) and s) in Theorem~\ref{Gorenstein characterized} implies that the localization of a Gorenstein ring is Gorenstein, something not obvious from the definition.


\fix{The following proof is incomplete}
\begin{proof} 
Suppose first that $d=0$ and $R$ is Gorenstein, so that $\Hom_{R}(k,R) = k$ ---that is, $R$ has a unique minimal submodule $N$, necessarily $\cong k$.

Let $E$ be the $R$-injective hull of $k$. The inclusion $N\subset R$ induces an inclusion $R\subset E$. Since
$\Hom_{R}(-, E)$ is an exact functor, and $\Hom_{R}(k, E)\cong k$, it follows by induction on the length of a finitely generated module $M$ that the length of $M$ is equal to the length of $\Hom_{R}(M, E)$.  Thus
$E = \Hom_{R}(R,E)$ has the same length as $R$, so $R$ and $E$ coincide. Thus $\Ext^{i}_{R}(M,R)$ vanishes for all $i>0$ and all $R$-modules $M$. This shows that 1) implies 2) and 3) in this case.s

Next suppose that $d=0$ and $\Ext^{1}_{R}(k,R) = 0$. So 2) implies 3) in this case.

To show that 3) implies 1), let $N$ be the largest submodule of $R$ that is annihilated by $\gm$, so that $N \cong k^{s}$ for some $s$. We must show that $s = 1$. 
The vanishing of $\Ext^{1}_{R}(R/N,R)$ shows that the map we get a short exact sequence
$$
\Hom_{R}(R,R) \to \Hom_{R}(N, R) \cong N = k^{s}
$$
is surjective. But this map factors through $R\to R/\gm = k$, so $s=1$, so $R$ is Gorenstein.

Now we do induction on $d$,  and we may suppose $d>0$. Suppose first that  there is a nonzerodivisor $x\in \gm$.

\fix{ the following para is almost right; the resolution over R/x is the mapping cone, ... -- the conditions 2,3 refer to Exts over different rings} From the exact sequences $(*)$ of Lemma~\ref{Ext and nzd} we see that each of the three conditions of the
Theorem for $R/(x)$ is equivalent to the corresponding condition for $R$. By induction, the three conditions are
equivalent for $R/(x)$, so they are equivalent for $R$.

If $R$ is Gorenstein then it is Cohen-Macaulay by definition, and since $d>0$, $\gm$ automatically contains a nonzerodivisor. Thus, to conclude the proof, it suffices to show that $\Ext_{R}^{d+1}(k,R) = 0$ implies that $R$ contains a nonzerodivisor.

In the contrary case, the prime ideal $\gm$ is an associated prime of 0; this means that there is a submodule $N$ of $R$ isomorphic to $k$.
\end{proof}

\fix{Need to add: somewhere: $S$ regular, $R = S/I$ is Gorenstein iff 
$$
\omega_{R} := \Ext^{\codim R}_{S}(R,S) \cong R
$$
up to shift.}
\section{What Makes a Complex Exact}

Given that the resolution of any module over a polynomial ring $S = k[x_1,\dots,x_n]$ (or regular local ring) has  length bounded by $n$, the resolution of an $S$-module that is a $k$-th syzygy---that is, the image of the $k$-th map in a resolution---must have length $\leq n-k$. What is special about the presentation matrix of such a module? -- put differently, what is it about the $k$-th map in a resolution that is different than the first map?

One answer to this question was given by David Hilbert in his orginal paper \cite{} and another can be given in terms of Gr\"obner bases and initial ideals \cite{}. Here we give a third answer, one that lends itself better to conceptual proofs, such as the proof of exactness of the Eagon-Northcott complex resolving the ideal of minors of a matrix given in Section~\ref{}. We give the result in a special case; in its general form in \cite{} it applies to finite free complexes over any ring.

\begin{theorem}\label{WMACE} Let 
$$
{\bf F}:  \quad 0\rTo F_m\rTo^{\phi_m} \cdots\rTo^{\phi_2} F_1\rTo^{\phi_1} F_0
$$
be a finite complex of finitely generated free modules over a polynomial ring or, more generally, a Cohen-Macaulay ring. The following conditions are equivalent:
\begin{enumerate}

\item The complex is exact (and thus a resolution of $\coker \phi_1$);

\item The ranks $r_i$ of the maps $\phi_i$ satisfy $r_i+r_{i+1}  = \rank F_i$ and the $r_i\times r_i$ minors of a matrix representation of
$\phi_i$ generate an ideal of codimension $\geq i$.

\item For every prime ideal $P$ of codimension $c$, the localized truncated complex 
$$
({\bf F}_{\geq c})_P: \quad 0\rTo (F_m)_P\rTo^{(\phi_m)_P} \cdots\rTo^{(\phi_{c+1})_P} (F_c)_P
$$
is split exact (and thus is a free resolution of a projective module.)
\end{enumerate}
\end{theorem}

The following easy Proposition plays a role in the proof:
\begin{proposition}
 Let $\phi: F \to G$ be a map of finitely generated free modules over a local ring $R$, and let $I_{r}(\phi)$ denote the ideal generated by the 
 $r\times r$ minors of a matrix representing $\phi$. The cokernel of $\phi$ is free if and only if, for some
 integer $r$ we have $I_{r} = R$ and $I_{r+1} = 0$.

\end{proposition}
The proof that exactness implies items 2,3 is easy to summarize: if ${\bf F}$ is a free resolution of an $S$-module $M$ and $P$ is a prime of codimension $c$
then $({\bf F})_P$ is a free resolution of
$M_P$ over $S_P$, a regular ring of dimension $c$.Thus the kernel of $(\phi_{c-1})_P$ must be projective (actually, free) over $S_P$, 
and $({\bf F}_{\geq c})_P$ is a free resolution of this module, and thus split exact. Taking $P$ to be a minimal prime (that is, if $S$ is a domain, the prime 0) gives the condition on ranks in
item 1), while the condition on codimension follows because the cokernel of each $(\phi_i)_P$ is projective for $i\geq c$. 

For a detailed treatment and the opposite implications, we refer to \cite{book}.



%footer for separate chapter files

\ifx\whole\undefined
%\makeatletter\def\@biblabel#1{#1]}\makeatother
\makeatletter \def\@biblabel#1{\ignorespaces} \makeatother
\bibliographystyle{msribib}
\bibliography{slag}

%%%% EXPLANATIONS:

% f and n
% some authors have all works collected at the end

\begingroup
%\catcode`\^\active
%if ^ is followed by 
% 1:  print f, gobble the following ^ and the next character
% 0:  print n, gobble the following ^
% any other letter: normal subscript
%\makeatletter
%\def^#1{\ifx1#1f\expandafter\@gobbletwo\else
%        \ifx0#1n\expandafter\expandafter\expandafter\@gobble
%        \else\sp{#1}\fi\fi}
%\makeatother
\let\moreadhoc\relax
\def\indexintro{%An author's cited works appear at the end of the
%author's entry; for conventions
%see the List of Citations on page~\pageref{loc}.  
%\smallbreak\noindent
%The letter `f' after a page number indicates a figure, `n' a footnote.
}
\printindex[gen]
\endgroup % end of \catcode
%requires makeindex
\end{document}
\else
\fi
