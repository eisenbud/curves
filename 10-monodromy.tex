
%header and footer for separate chapter files

\ifx\whole\undefined
\documentclass[12pt, leqno]{book}
\usepackage{graphicx}
\usepackage{eps-to-pdf}
\input style-for-curves.sty
%\input sl-macros.sty
\usepackage{hyperref}
\usepackage{showkeys} %This shows the labels.
\usepackage{msribib}
\usepackage{pdfpages}
\usepackage{draftwatermark}
\SetWatermarkText{DRAFT:\ \today}
\SetWatermarkScale{2}
\SetWatermarkColor[gray]{0.9}

%\usepackage{SLAG,msribib,local}
%\usepackage{amsmath,amscd,amsthm,amssymb,amsxtra,latexsym,epsfig,epic,graphics}
%\usepackage[matrix,arrow,curve]{xy}
%\usepackage{graphicx}
%\usepackage{diagrams}
%
%%\usepackage{amsrefs}
%%%%%%%%%%%%%%%%%%%%%%%%%%%%%%%%%%%%%%%%%%
%%\textwidth16cm
%%\textheight20cm
%%\topmargin-2cm
%\oddsidemargin.8cm
%\evensidemargin1cm
%
%%%%%%Definitions
%\input preamble.tex
%\input style-for-curves.sty
%\def\TU{{\bf U}}
%\def\AA{{\mathbb A}}
%\def\BB{{\mathbb B}}
%\def\CC{{\mathbb C}}
%\def\QQ{{\mathbb Q}}
%\def\RR{{\mathbb R}}
%\def\facet{{\bf facet}}
%\def\image{{\rm image}}
%\def\cE{{\cal E}}
%\def\cF{{\cal F}}
%\def\cG{{\cal G}}
%\def\cH{{\cal H}}
%\def\cHom{{{\cal H}om}}
%\def\h{{\rm h}}
% \def\bs{{Boij-S\"oderberg{} }}
%
%\makeatletter
%\def\Ddots{\mathinner{\mkern1mu\raise\p@
%\vbox{\kern7\p@\hbox{.}}\mkern2mu
%\raise4\p@\hbox{.}\mkern2mu\raise7\p@\hbox{.}\mkern1mu}}
%\makeatother

%%
%\pagestyle{myheadings}

%\input style-for-curves.tex
%\documentclass{cambridge7A}
%\usepackage{hatcher_revised} 
%\usepackage{3264}
   
\errorcontextlines=1000
%\usepackage{makeidx}
\let\see\relax
\usepackage{makeidx}
\makeindex
% \index{word} in the doc; \index{variety!algebraic} gives variety, algebraic
% PUT a % after each \index{***}

\overfullrule=5pt
\catcode`\@\active
\def@{\mskip1.5mu} %produce a small space in math with an @

\title{A Chapter from ``The Practice of Algebraic Curves"}
\author{\copyright David Eisenbud and Joe Harris}
%%\includeonly{%
%0-intro,01-ChowRingDogma,02-FirstExamples,03-Grassmannians,04-GeneralGrassmannians
%,05-VectorBundlesAndChernClasses,06-LinesOnHypersurfaces,07-SingularElementsOfLinearSeries,
%08-ParameterSpaces,
%bib
%}

\date{\today}
%%\date{}
%\title{Curves}
%%{\normalsize ***Preliminary Version***}} 
%\author{David Eisenbud and Joe Harris }
%
%\begin{document}

\begin{document}
\maketitle

\pagenumbering{roman}
\setcounter{page}{5}
%\begin{5}
%\end{5}
\pagenumbering{arabic}
\tableofcontents
\fi



\chapter{Monodromy of Hyperplane Sections}\label{uniform position}

\section{Uniform position and monodromy} \label{uniformSection}
We now return to the situation where the ground field $k$ is algebraically closed of characteristic 0.

The central result of this chapter is the  \emph{uniform position lemma}, which deals with the monodromy group of the points of a general hyperplane section of a curve $C \subset \PP^r$. To define the monodromy group  and prove the  uniform position lemma, we will use the classical topology. An equivalent algebraic definition is described in Cheerful Fact~\ref{Galois equals monodromy} below, but the strong form of uniform position can fail in positive characteristic,
as shown by the examples in Exercise~\ref{strange curves}.


We may describe the monodromy group informally as follows: Suppose that $C \subset \PP^r$ is an irreducible curve of degree $d$ over $\CC$, and $H_0 \subset \PP^r$ a hyperplane transverse to $C$; say $C \cap H_0 = \{p_1,\dots,p_d\}$. As we vary $H_0$ continuously along a real arc $\{H_t\}$, staying within the open subset $U \subset {\PP^r}^*$ of hyperplanes transverse to $C$, we can ``follow" each of the points $p_i(t)$ of intersection of $C$ with the hyperplane $H_t$.

Now imagine that the hyperplanes $H_t$ come back to the original $H_0$ at time $t=1$; that is, we have a continuous family $\{H_t\}_{0 \leq t \leq 1} \subset U$ with $H_1 = H_0$. Each of the points $p_i$ then traces out a continuous real arc 
$\{p_i(t) \in C \cap H_t\}_{0 \leq t \leq 1}$. Since $H_1 = H_0$, the end point $p_i(1)$ is one of the original points $p_j \in C \cap H_0$. In this way, we get a permutation of the set $C \cap H_0$; the group of all permutations arrived at in this way is called the \emph{monodromy group} of the points $C \cap H_0$. 

We will now give a precise definition of the monodromy group in a more general setting, and prove that the monodromy group of the points of a general hyperplane section of an irreducible curve
 is the full symmetric group; this is the uniform position lemma. The rest of the chapter will be a series of applications.

\subsection{The monodromy group of a generically finite morphism}

Let $f : Y \to X$ be a dominant map between varieties of the same dimension over $\CC$, and suppose that $X$ is irreducible. There is then a Zariski open subset $U \subset X$ such that $U$ and 
its preimage $V = f^{-1}(U)$ are smooth, and the restriction of $f$ to $V$ is a covering space in the classical topology. Let $d$ be the number of sheets. This is the degree of the extension $K(Y)/K(X)$. %\fix{should we put in the proof?}

Homotopy theory  associates a monodromy group to any finite topological covering map $f : V \to U$, defined as follows: Choose a base point $p_0 \in U$, and suppose $\Gamma := f^{-1}(p_0)  = \{q_1,\dots,q_d\}$. If $\gamma$ is any loop in $U$ with base point $p_0$, for any $i = 1, \dots, d$ there is a unique lifting of $\gamma$ to an arc $\tilde \gamma_i$ in $V$ with initial point $\tilde \gamma_i(0) = q_i$ and end point $\tilde \gamma_i(1) = q_j$ for some $j \in \{1,2,\dots,d\}$. Since we could traverse the loop in the opposite direction, the index $j$ determines $i$, and the map $i\mapsto j$ is a permutation of $\{1,2,\dots,d\}$. 
Since the set $\Gamma$ is discreet, the permutation depends only on the class of $\gamma$ in $\pi_1(U,p_0)$ so we have defined a homomorphism to the symmetric group:
$$
\pi_1(U,p_0)  \to {\rm Perm}(\Gamma) \cong S_d.
$$
The image $M$ of this map is called the \emph{monodromy group} of the map $f$. It depends on the labeling of the points of $\Gamma$, but a change in labeling
only changes the group by conjugation with the corresponding permutation, so the monodromy group is well defined up to 
conjugation in $S_{d}$.

\begin{fact}\label{Galois equals monodromy}
In our setting  the monodromy group is independent of the choice of open set $U$: if $U' \subset U$ is a Zariski open subset, the complement of $U\setminus U'$ has
real codimension 2 so the map $\pi_1(U', p_0) \to \pi_1(U,p_0)$ is surjective. Thus the image of $\pi_1(U', p_0)$ in $S_d$ is the same. 

The theory of finite coverings of algebraic varieties is not only analogous to Galois theory, it \emph{is} Galois theory: In the situation described above, if $Y$ is irreducible, then the pullback map $f^*$ expresses the function field $K(Y)$ as a finite algebraic extension of $K(X)$. The monodromy group of $f$  is the Galois group of the Galois normalization of $K(Y)$ over $K(X)$ (see \cite{Harris1979}). Indeed, in early treatments of Galois theory, such as the famous \emph{Trait\'e des Substitutions}~\cite{MR1188877}, originally published in 1870, function fields played as large a role as number fields.
%\fix{what's the embedding in the symmetric group? must be the full perm group of the roots of primitive poly.}
\end{fact}

Since we assumed that $X$ is irreducible, the space $U$ is (path) connected, and it follows that the monodromy group is transitive if and only if the space $V$ is (path) connected. But $V$ is a smooth
variety, so this is the case if and only if $V$ is irreducible. For example, the monodromy in the family
of smooth quadric surfaces in $\PP^3$ interchanges the two rulings:

\begin{example}\label{monodromy of rulings}
Consider the family 
$$
X := \{(p,Q) \mid p\in Q\hbox{ a point on a smooth quadric surface in $\PP^3$}\}
$$
of pointed smooth quadric surfaces in $\PP^3$ and the double covering by  
$$
V:= \{((p,Q), L)\mid (p,Q)\in X,\ L \hbox{ a line with $p \in L\subset Q$}\}.
$$
The variety $V$ is irreducible: if we project $V$ to the (irreducible) variety of lines in $\PP^3$, 
the fiber consists of a point on the line and quadric containing the line---that is, the product
of $\PP^1$ and the (dual of the) projective space
on the space of quadratic forms in the ideal of the line. Thus the monodromy of the family
is transitive, and exchanges the two lines through the point.
\end{example}

\subsection{Uniform position}


We will next compute the monodromy group of the  universal hyperplane section of a curve, constructed as follows:
Let $C \subset \PP^r$ be an irreducible, nondegenerate curve of degree $d$, and let $X = {\PP^r}^*$ be the space of hyperplanes in $\PP^r$. We define the \emph{universal hyperplane section of $C$} to be the projection  $f: Y\to {\PP^r}^*$ of the incidence variety
$$
Y = \{ (H, p) \in {\PP^r}^* \times C \mid p \in H \}.
$$
The fibers of $f$ are the hyperplane
sections of $C$, so $f$ is a finite surjective map. If we let $U\subset {\PP^r}^*$ be the open subset of hyperplanes
meeting $C$ transversely, then the restriction of $f$ to the preimage $V$ of $U$ is a covering space
whose fibers each consist of $d$ distinct points. The preimage in $Y$ of a point $p\in C$ is the set of hyperplanes containing
$p$, a copy of $\PP^{r-1}$, and thus $Y$ is irreducible. Thus the monodromy group of $f$ is transitive. But much more is true:

\begin{theorem}[Uniform Position Theorem]\label{uniform position lemma}
The monodromy group of the universal hyperplane section of an irreducible curve $C \subset \PP^r$ is the full symmetric group $S_d$.
\end{theorem}

We postpone the proof to develop some necessary tools.

Theorem~\ref{uniform position lemma} fails over fields of finite characteristic, though there is no known counterexample for smooth curves; see Exercise~\ref{strange curves} for singular examples, and \cite{Rathmann} and \cite{Kadets} for what is known. 

Theorem~\ref{uniform position lemma} implies that two subsets of the same cardinality in the general hyperplane section of $C$
are indistinguishable from the point of view of any discrete invariant that is semicontinuous in the Zariski topology. To make this precise, we introduce a definition:

\begin{definition}
Let $\phi : Y \to X$ be a finite morphism. By the \emph{restricted fiber power} $\tilde Y^n/X$ we will mean the complement of all diagonals in the ordinary fiber power; that is,
$$
\tilde Y^n/X := \{ (x, y_1,\dots, y_n) \in X \times Y^n \mid \phi(y_i) = x \text{ and } y_i \neq y_j \; \forall i \neq j \}
$$
\end{definition}

In down-to-earth terms, a point of $\tilde Y^n/X$ is a set of $n$ distinct points in a fiber of $\phi$ together with
the choice of a total order on these points. 

\begin{lemma}\label{transitivity lemma}
Let $f : Y \to X$ be a generically finite cover of degree $d$, with  monodromy group $M \subset S_d$.
$M$ is $n$-transitive if and only if the restricted fiber power $\tilde Y^n/X$ is irreducible.
\end{lemma}

\begin{proof}
If we restrict ourselves to open subsets $U \subset X$ and $V = f^{-1}(U) \subset Y$ such that $U$ and $V$ are smooth and the restriction $f|_V : V \to U$ is a covering in the classical topology, then the restricted fiber powers are unions of connected components of the usual fiber powers $Y^n/X$. The condition that the monodromy is $n$-transitive is equivalent to the condition that the restricted fiber power $\tilde Y^n/X$ is connected; since the fiber powers are all smooth, this is equivalent to $\tilde Y^n/X$ being irreducible.
\end{proof}



\section{Flexes and bitangents are isolated}\label{isolated flexes and bitangents}

There are  further general position results that we need for the proof of the uniform position lemma.  We have separated
them from the results on linear general position because these require characteristic 0; they are really local analytic
statements.

%\fix{We haven't talked about inflectionary points yet, so I rewrote this just for flex lines. The original is still there, just commented out.}

\subsection{Not every tangent line is a flex}\label{isolated tangents and bitangents}

Recall that a smooth point $p$ on a curve $C \subset \PP^r$ is called a \emph{flex point} if the tangent line to $C$ at  $p$ has contact of order 3 or more with $C$ at $p$.

\begin{lemma}\label{finite inflections}
If $r>1$ and $C \subset \PP^r$ is a smooth, irreducible and nondegenerate curve, then not every point of $C$ is a flex point.
\end{lemma}

The proof applies to any linear series on a curve.

\begin{proof}
We begin by lifting the inclusion $C \hookrightarrow \PP^r$ to an arc $v (t)$ in $\CC^{r+1}$, so that the tangent line at the point $t=0$ in projective space
is represented by the span of $v(0)$ and $v'(0) \in \CC^{r+1}$. To say that $v(t)$ is a flex point is to say that the  vectors $v(t), v'(t)$ and $v''(t)$ are linearly dependent. If this holds for all $t$ then
$$
v(t) \wedge  v'(t) \wedge v''(t) \; \equiv \; 0.
$$

When we take the derivative of the wedge product $v(t) \wedge v'(t) \wedge v''(t)$ by applying the product rule we see that the first two terms are zero because they contain a repeated factor; it follows that
$$
v(t) \wedge  v'(t) \wedge v'''(t) \equiv 0,
$$
so that $v'''(t)$ lies in the span of $v(t)$ and $v'(t)$ as well. Indeed, as we continue to take derivatives, we see in each case that all but one term is zero, and we deduce that $v^{(l)}(p)$ lies in the span of $v(t)$ and $v'(t)$ for all $l$. This being characteristic 0, it follows that $C$ lies in a line, contrary to our hypothesis
\end{proof}


 \subsection{Not every tangent is bitangent}
 
 This statement seems even more obvious than Lemma~\ref{finite inflections} above, but---like that lemma---is false in characteristic $p$! \fix{put an example in the exercises?}
 
 \begin{lemma}\label{tangent not bitangent}
 Let $C \subset \PP^r$ be a smooth, irreducible, nondegenerate curve, with $r > 1$. If $p \in C$ is a general point, then the tangent line $\TT_p(C) \subset \PP^r$ is not tangent to $C$ at any other point.
 \end{lemma}
 
 \begin{proof} Again the result is local, this time with a pair of germs $D\rTo^v \PP^r$ and $D\rTo^w \PP^r$. If every
 tangent line to the two curves is a bitangent, we will show that they are both contained in a line.
 
  Let $C_1, C_2$ be the images of $v,w$ respectively, and let $\tilde v, \tilde w$ be lifts to $\CC^{r+1}$.
 Let  
 $$
 \Sigma := \{ (p,q) \in D \times D \mid \text{ and }\TT_{v(p)}(C_1) = \TT_{w(q)}(C_2) \}
 $$
 be the variety parametrizing bitangents to the two germs. 
 
 
 
 The statement that the tangent lines to $C$ at the points $v(t)$ and $w(t)$ are equal then says that the vectors $\tilde v(t), \tilde v'(t),\tilde w(t)$ and $\tilde w'(t)$ all lie in a 2-dimensional subspace $\Lambda \subset \CC^{r+1}$; in particular,
 $$
 \tilde v(t) \wedge \tilde v'(t) \wedge \tilde w(t) \equiv 0 \quad \text{and likewise} \quad \tilde v(t) \wedge \tilde w(t) \wedge \tilde w'(t) \equiv 0
 $$
 
We proceed now exactly as in the proof of Lemma~\ref{finite inflections}: taking derivatives, we see that all derivatives of $\tilde v(t)$
and $\tilde w$ at $t=0$ lie in $\Lambda$
and hence $C_1$ and $C_2$ are both contained in the line in $\PP^r$ corresponding to $\Lambda$.
 \end{proof}

We do not know whether  a general tangent line to a smooth curve $C$ can intersect $C$ in a point other than $p$, except of course in case $r=2$. \fix{Joe will write to Noam to ask for a reference.}

\section{Proof of the uniform position theorem}

Let $C \subset \PP^r$ be an irreducible, nondegenerate curve of degree $d$ and $f : Y \subset {\PP^r}^* \times C \to  X = {\PP^r}^*$ its universal hyperplane section; let $U \subset {\PP^r}^*$ be the open subset of hyperplanes transverse to $C$ and $V = f^{-1}(U)$; let $M \subset S_d$ be the monodromy group of $V$ over $U$.
To show that  $M$ is the full symmetric group, it suffices to show that $M$ contains all transpositions, and for this it is enough to show that $M$ is doubly transitive and contains one transposition.

To prove that $M$ is doubly transitive, we can give a concrete description of the restricted fiber power $\tilde V^2/U$: let
$$
\Sigma := \left\{ (H, p, q) \in {\PP^r}^* \times C \times C \mid p, q \in H \text{ and } p \neq q \right\}.
$$
Projection on the second and third factors expresses $\Sigma$ as a $\PP^{r-2}$-bundle over the complement $C \times C \setminus \Delta$ of the diagonal in $C \times C$. Thus $\Sigma$ is irreducible, and it follows that the restricted fiber square $\tilde V^2/U$, which is a Zariski open subset of $\Sigma$, is as well. Note that this part of the argument does not rely on any assumption about the characteristic.

Next we give a criterion for a monodromy group to contain a transposition:

\begin{lemma}\label{transposition lemma}
Let $f : Y \to X$ be a generically finite cover of degree $d$ over an irreducible variety $X$, with  monodromy group $M \subset S_d$.  
If,  for some smooth point $p \in X$ the fiber $f^{-1}(p)\subset V$ consists of $d-2$ reduced points $p_1,\dots, p_{d-2}$ and one point $q$ of multiplicity 2, where $q$ is also a smooth point of $Y$, then $M$ contains a transposition.
\end{lemma}

\begin{figure}
\centerline {\includegraphics[height=1in]{"main/Fig10.1"}}
\caption{Monodromy action around the ramification point of a double cover}
\label{default}
\end{figure}

\begin{proof} Note that the hypothesis implies that $Y$ is smooth
locally near the fiber over $p$. Let $U \subset X$ be a Zariski open subset of the smooth locus in $X$, as in the definition of the monodromy group, so that  $V := f^{-1}(U)$ is also smooth and the restriction $f|_V : V \to U$ expresses $V$ as a finite $d$-sheeted covering space of $U$, with $U,V$ smooth and $p\in X$.

Let $B_i\subset Y$ be disjoint small connected closed neighborhoods of the points
in $f^{-1}(p)$. Because $f|_V$  is finite, the image $A := \cap f(B_i)$  is a locally
closed subset of the same dimension as $X$ and thus $A$
 contains a neighborhood
of $p$ in the classical topology.

Let $p' \in A \cap U$. Two of the $d$ points of $f^{-1}(p')$  lie in the component  of $B$ containing $q$; call these $q'$ and $q''$. Since $B \cap V$ is the complement of a proper subvariety in $B$ it is connected, and we can draw a real arc $\gamma : [0,1] \to B \cap V$ joining $q'$ to $q''$; by construction, the permutation of $f^{-1}(p')$ associated to the loop $f \circ \gamma$ will exchange $q'$ and $q''$ and fixes each of the remaining $d-2$ points of $f^{-1}(p')$.
\end{proof}

\begin{proof}[Completion of the proof of the uniform position lemma]
 It remains to show that a reduced irreducible curve $C$ of degree $d$ (in characteristic 0)
 has a hyperplane section consisting of $d-2$ smooth points and one double point; that is, a hyperplane simply tangent to the curve at one point and transverse everywhere else. By the results of 
 Section~\ref{isolated flexes and bitangents} there is a tangent line $L$ to $C$ at a smooth point that is not flex tangent, and is not tangent to $C$ at any other point. A general hyperplane $H$ containing $L$ meets $C$ doubly in the point of
 tangency. At any other point $p$ of $L\cap C$, if any, the intersection $C\cap H$ is transverse
 unless $H$ contains the tangent line at $p$. Containing such a line is a  proper codimension 1 condition on
 the hyperplanes containing $L$, and since there can only be finitely many such points, a
 general  $H$ will be transverse at all such $p$. On the other hand, at points not on $L$
 the intersection $H\cap C$ is transverse by Bertini's theorem. This completes the proof.
\end{proof}

\subsection{Uniform position for higher-dimensional varieties}

We mention here a generalization of Theorem~\ref{uniform position lemma} for irreducible varieties $X \subset \PP^r$ of any dimension $k$. To set this up, let $\GG(r-k,r)$ be the Grassmannian parametrizing $(r-k)$-planes $\Lambda \subset \PP^r$. We introduce the \emph{universal $(r-k)$-plane section of} $X$:
$$
Y = \{ (\Lambda, p) \in \GG(r-k,r) \times X \mid p \in \Lambda \}.
$$
Projection on the first factor expresses $Y$ as a generically finite cover of $\GG(r-k,r)$, and we can ask for its monodromy. The answer is the same as for curves: 


\begin{theorem}\label{higher dim uniform position lemma}
The monodromy group of the universal $(r-k)$-plane section of an irreducible $k$-dimensional variety $X \subset \PP^r$ is the full symmetric group $S_d$.
\end{theorem}

\begin{proof}
In fact, this follows from Theorem~\ref{uniform position lemma}. To prove it, fix a general $(r-k+1)$-plane $\Gamma \subset \PP^r$; since a general hyperplane section of an irreducible variety $X \subset \PP^r$ of dimension $k \geq 2$ is again irreducible, we see that $C := \Gamma \cap X$ is an irreducible curve. The restriction of the universal $(r-k)$-plane section of $X$ to the sub-Grassmannian $\GG(r-k, \Gamma) \cong {\PP^r}^*$ of $(r-k)$-planes contained in $\Gamma$ is just the universal hyperplane section of the curve $C$, which we know has monodromy $S_d$; since the monodromy group of a cover can only get smaller under restriction to a subvariety of the target, the result follows.
\end{proof}

We will see an application of this result in Proposition~\ref{plane curve nodes} below.

 \section{Applications of uniform position}
\subsection{Irreducibility of fiber powers}
If we apply the uniform position lemma to the universal hyperplane section of a curve $C \subset \PP^r$ we get an irreducibility result:

\begin{corollary}\label{hyperplane section monodromy} If $C\subset \PP^r$ is a smooth curve of degree $d$, then 
for $k\leq d$, the restricted fibered powers $\tilde Y^k/X$  of the universal hyperplane section 
of $C$ are irreducible.
\end{corollary}

\subsection{Numerical uniform position}
\begin{corollary}[numerical uniform position lemma]\label{numerical uniform position lemma}
Let $C\subset \PP^r$ be an irreducible curve, and let $\Gamma = H\cap C$ be a general hyperplane section. Any two subsets of $\Gamma$ with the same cardinality impose the same number of conditions on forms of any degree; that is, any two subsets of the same cardinality have the same Hilbert function.
\end{corollary}


\begin{figure}
\centerline {\includegraphics[height=1in]{"main/Fig10-2"}}
\caption{7 points in the plane, 6 on a conic, in linearly general but not uniform position}
\label{default}
\end{figure}

\begin{proof} Let $U = {\PP^r}^* \setminus C^*$ be the open subset of hyperplanes transverse to $C$, and let $Y\to U$ be the universal hyperplane section.
Corollary~\ref{hyperplane section monodromy} says that the restricted fiber powers $\tilde V^n/U$ are irreducible.

Now, for each $m$ the number of conditions that $\Gamma$ imposes on forms of degree $m$ is lower semicontinuous, so it achieves its maximum on a Zariski open subset of $\tilde V^n/U$. Since $\tilde V^n/U$ is irreducible, the complement $Z$ of this open set has dimension strictly less than $\dim \tilde V^n/U = \dim U$. Thus a general hyperplane $H \in {\PP^r}^*$  lies outside the image of $Z$, meaning that the number of conditions imposed by all the $k$-element subsets $\Gamma \subset C \cap H$ have this maximal value.
\end{proof}

This result may be seen as an important strengthening of Theorem~\ref{basic linear independence}, since if $C$ is a reduced, irreducible nondegenerate curve in $\PP^n$ then a general subset of $n$ points of $C$ is linearly independent and spans a hyperplane; Corollary~\ref{numerical uniform position lemma} says that this is true for every subset of $n$ points of every general hyperplane section, 
which reproves Theorem~\ref{basic linear independence}, though only in characteristic 0. 

\subsection{Sums of linear series}

Another consequence of the uniform position lemma is a result about sums of linear series.
Recall that if $D$ is a divisor on a curve $C$ we write $r(D) = \dim |D| = h^0(\cO_C(D))-1$.

\begin{corollary}\label{Clifford equality plus}
If $D,E$ are effective divisors on a curve $C$ then
$$
r(D+E) \geq r(D)+r(E).
$$
If the genus of $C$ is $>0$ and $|D+E|$ is birationally very ample, then the inequality is strict.
\end{corollary}.

On $\PP^1$, by contrast, any effective divisor $D$ has $r(D) = \deg D$, so the inequality above is
always an equality for $C = \PP^1$.

\begin{proof}
 The inequality follows in general because the sums of divisors in $|D|$ and divisors in $|E|$ already move in 
 a family of dimension $r(D)+r(E)$; the key point is the strict inequality in case $D+E$ is birationally very ample.
 
If $D+E$ is birationally very ample and $r(D+E) = r(D)+r(E)$ then restricting to an open set
we may identify $C$ with its image under the complete linear series $|D+E|$, and we see that a general hyperplane section $H\cap C$ contains a divisor equivalent to $D$.

Let $Y$ be the $\deg D +\deg E$ restricted fiber power of the universal hyperplane.
A point $y\in Y$ is a hyperplane section plus an ordering of its points.  Let $\phi: Y \to \Pic_d(C)$ be the Abel-Jacobi map taking $y$
 to the class of the divisor that is the sum of first $d$ points in this order. The preimage  $Y'$ of the point of $\Pic_d(C)$ corresponding to the class of $D$ is a closed subset, and
since every divisor in the class of a hyperplane section contains a divisor
linearly equivalent to  $D$, the subvariety $Y'$ dominates $\PP^{n*}$, and thus
has the same dimension as $Y$. Consequently $Y'=Y$, and the sum of the first $d$ points
in any ordering of the general hyperplane section---that is, the sum of any $d$
of the points---is equivalent to $D$.

Thus if $p\in D$ and $q\notin D$, then $D-p+q \equiv D$, whence $q\equiv p$. Thus
$r(p)\geq 1$, so $C\cong \PP^1.$
\end{proof}

\subsection{Nodes of plane curves}\label{plane curve nodes}

In Section~\ref{severi variety}, we introduced the \emph{Severi variety} $V_{d,g}$; this is the locally closed subset of the projective space $\PP^N$ of all plane curves of degree $d$ parametrizing irreducible plane curves of degree $d$ having $\delta := \binom{d-1}{2} - g$ nodes and no other singularities. We proved there that $V_{d,g}$ was smooth, and by Cheerful Fact~\ref{severi irreducible} it is irreducible for all $d$ and $g$. 

To compute the monodromy we introduce the incidence correspondence
$$
\Phi := \{(C, p) \in V_{d,g} \times \PP^2 \mid p \in C_{sing} \}.
$$ 
This is a $\delta$-sheeted covering space of $V_{d,g}$, and we compute its monodromy is. We start with the extremal case $g = 0$, where we can prove

\begin{proposition}
The monodromy group of $\Phi$ over $V_{d,0}$ is the full symmetric group $S_\delta$, with $\delta = \binom{d-1}{2}$.
\end{proposition}

\begin{proof}
Every rational nodal curve $C \subset \PP^2$ is the projection of a rational normal curve $\tilde C \subset \PP^d$ from a $(d-3)$-plane $\Lambda \subset \PP^d$. Moreover, if $\Lambda$ is general,  the nodes of the projection correspond to the points of intersection of $\Lambda$ with the  secant variety $X \subset \PP^d$ of the rational normal curve $\tilde C$. Applying Theorem~\ref{higher dim uniform position lemma}, the result follows.
\end{proof}

This result has an immediate consequence, which played a major role in the proof of Cheerful Fact~\ref{severi irreducible}: given the description in Section~\ref{severi variety} of $V_{d,g}$ in a neighborhood of a point in $V_{d,0}$, it follows that there is a unique irreducible component of $V_{d,g}$ containing $V_{d,0}$ in its closure. Thus, to prove the irreducibility of $V_{d,g}$, it is sufficient to show that every component of $V_{d,g}$ contains $V_{d,0}$ in its closure. 

Going in the other direction, note also that if we assume Cheerful Fact~\ref{severi irreducible}, then we can deduce the analogous result for all $g$:


\begin{proposition}
The monodromy group of $\Phi$ over $V_{d,g}$ is the full symmetric group $S_\delta$, with $\delta = \binom{d-1}{2} - g$.
\end{proposition}



\section{Exercises}

As a consequence of Theorem~\ref{higher dim uniform position lemma}, we can deduce the Bertini irreducibility theorem:

\begin{exercise}
Let $X \subset \PP^r$ be an irreducible variety of dimension $k \geq 2$. Show that a general hyperplane section of $X$ is irreducible.
\end{exercise}

\begin{exercise}
In Example~\ref{monodromy of rulings}, we gave a global argument to say that in the family of smooth quadric surfaces in $\PP^3$ the monodromy exchanges the two rulings of a quadric by lines. Prove this with a local calculation, analyzing the family
$$
Q_t := V(X^2+Y^2+Z^2 + tW^2)
$$
in a neighborhood of $t=0$.
\end{exercise}

\begin{exercise}
Let $C \subset \PP^r$ be a union of irreducible curves $C_i$ of degrees $d_i$. Prove that the monodromy group of the points of a general hyperplane section of $C$ is the product $\prod S_{d_i}$.
\end{exercise}

Hint: We need to know that the dual hypersurfaces $C_i^* \subset {\PP^r}^*$ are all distinct; given this, we can exhibit loops that induce a given permutation of the points of $H \cap C_i$ while fixing the points of $H \cap C_j$ for $j \neq i$.

\begin{exercise}
Let $d$ and $e$ be positive integers, $\PP^M$ the space of plane curves of degree $d$ and $\PP^N$ the space of plane curves of degree $e$, and
$$
\Phi := \{ (D, E, p) \in \PP^M \times \PP^N \times \PP^2 \mid p \in D \cap E \}.
$$
Prove that the monodromy group of the projection $\pi : \Phi \to \PP^M \times \PP^N$ is the symmetric group $S_{de}$ on $de$ letters
\begin{enumerate}
\item by applying Theorem~\ref{uniform position lemma}; and
\item from scratch, using the method used in the proof of Theorem~\ref{uniform position lemma}.
\end{enumerate}
\end{exercise}

\begin{exercise}
If $E \subset \PP^2$ is a smooth plane cubic curve, then a point $p \in E$ is  a \emph{flex} if $\cO_E(3p) \cong \cO_E(1)$. By Corollary~\ref{torsion points} in case $g=1$, there are nine of them. Now let $\PP^9$ be the space of plane cubics, and let
$$
\Phi := \{ (E, p) \in \PP^9 \times \PP^2 \mid p \text{ is a flex of } E \}.
$$
Show that the monodromy group of $\Phi \to \PP^9$ is a proper subgroup of $S_9$. (Hint: the line joining any two flexes of a plane cubic $E$ contains a third.)
\end{exercise}

\section{potential exercises}
\begin{enumerate}
 \item more generally: in a pencil of quadrics in $\PP^3$ specializing to a cone: when is the monodromy of the rulings trivial? (family is singular, line is not transverse)
 \item same for a suff general CI of 3 quadrics.
 \item in the fam of quadrics in the ideal of a twisted cubic, the monodromy is trivial. this implies that the disc locus is a double conic.
 \item: simply branched cover of prime degree: implies that monodromy is the symm group.
 
\end{enumerate}



\input footer.tex