%header and footer for separate chapter files

\ifx\whole\undefined
\documentclass[12pt, leqno]{book}
\usepackage{graphicx}
\input style-for-curves.sty
\usepackage{hyperref}
\usepackage{showkeys} %This shows the labels.
%\usepackage{SLAG,msribib,local}
%\usepackage{amsmath,amscd,amsthm,amssymb,amsxtra,latexsym,epsfig,epic,graphics}
%\usepackage[matrix,arrow,curve]{xy}
%\usepackage{graphicx}
%\usepackage{diagrams}
%
%%\usepackage{amsrefs}
%%%%%%%%%%%%%%%%%%%%%%%%%%%%%%%%%%%%%%%%%%
%%\textwidth16cm
%%\textheight20cm
%%\topmargin-2cm
%\oddsidemargin.8cm
%\evensidemargin1cm
%
%%%%%%Definitions
%\input preamble.tex
%\input style-for-curves.sty
%\def\TU{{\bf U}}
%\def\AA{{\mathbb A}}
%\def\BB{{\mathbb B}}
%\def\CC{{\mathbb C}}
%\def\QQ{{\mathbb Q}}
%\def\RR{{\mathbb R}}
%\def\facet{{\bf facet}}
%\def\image{{\rm image}}
%\def\cE{{\cal E}}
%\def\cF{{\cal F}}
%\def\cG{{\cal G}}
%\def\cH{{\cal H}}
%\def\cHom{{{\cal H}om}}
%\def\h{{\rm h}}
% \def\bs{{Boij-S\"oderberg{} }}
%
%\makeatletter
%\def\Ddots{\mathinner{\mkern1mu\raise\p@
%\vbox{\kern7\p@\hbox{.}}\mkern2mu
%\raise4\p@\hbox{.}\mkern2mu\raise7\p@\hbox{.}\mkern1mu}}
%\makeatother

%%
%\pagestyle{myheadings}

%\input style-for-curves.tex
%\documentclass{cambridge7A}
%\usepackage{hatcher_revised} 
%\usepackage{3264}
   
\errorcontextlines=1000
%\usepackage{makeidx}
\let\see\relax
\usepackage{makeidx}
\makeindex
% \index{word} in the doc; \index{variety!algebraic} gives variety, algebraic
% PUT a % after each \index{***}

\overfullrule=5pt
\catcode`\@\active
\def@{\mskip1.5mu} %produce a small space in math with an @

\title{Personalities of Curves}
\author{\copyright David Eisenbud and Joe Harris}
%%\includeonly{%
%0-intro,01-ChowRingDogma,02-FirstExamples,03-Grassmannians,04-GeneralGrassmannians
%,05-VectorBundlesAndChernClasses,06-LinesOnHypersurfaces,07-SingularElementsOfLinearSeries,
%08-ParameterSpaces,
%bib
%}

\date{\today}
%%\date{}
%\title{Curves}
%%{\normalsize ***Preliminary Version***}} 
%\author{David Eisenbud and Joe Harris }
%
%\begin{document}

\begin{document}
\maketitle

\pagenumbering{roman}
\setcounter{page}{5}
%\begin{5}
%\end{5}
\pagenumbering{arabic}
\tableofcontents
\fi


\chapter{History} 
\label{History}

In this Chapter we reproduce an excerpt of an unpublished essay by Jeremy Gray, sketching some of the landmarks in the history of algebraic geometry before the complex analytic work of
Abel, Jacobi and Riemann, and before Clebsch began the program of incorporating Riemann's insights into the theory of projective plane curves.

Surviving sources suggest that the study of conic sections, literally sections of a cone, may well have arisen with work of Menaechmus (380--320 BCE) on doubling the cube. This was a long-standing problem with a theological spin -- double the size of an altar (the Delian problem).  He expressed the solution in terms of the intersection
of a hyperbola and a parabola. 

About a century later, Diocles (290--180 BCE), and soon after him Apollonius of Perga (240--190 BCE), made significant progress with the conic sections.
Apollonius produced the first \emph{theory} of conic sections as sections of a cone. The names of the three types of section, the hyperbola, parabola, and ellipse,  are due to him.
We know this because four books of his conics survive in Greek and three more in Arabic; the eighth and final volume is lost. 

By the time of Descartes (1596--1650) there was already some sophisticated algebra expressed in a formalism that hadn't quite shaken off the Greek insistence on seeing everything as geometrical magnitudes: lengths, areas, volumes, and, well, what exactly? It was possible to write polynomial equations in this language. First Fermat (1607--1665) in 1636, and then much more boldly Descartes in 1637, realized that you could extend the language to two variables and so describe curves in the plane.

Descartes' s first achievement was to eliminate the dimensional aspect. A simple use of similar triangles allowed him to show that the product of two lengths could be seen as another length (not an area) so all geometrical quantities could be regarded as one-dimensional and the idea of dimension quietly dropped. Then came the real work. Almost all mathematical problems in his day were expressed in the language of geometry, except for some problems we would call diophantine that were implicitly about integers and rational numbers. Accordingly, the answer had to be expressed geometrically. Descartes's idea was to give letters to all the lengths involved in a problem, use the statement of the problem to express relationships between the letters, and reduce the equations to a single equation. Then solve the equation and express the answer again in geometrical terms.  

However, before Euler (1707--1783) in the late 1740s there was no concept of a function, everything was geometry, a relationship between two varying lengths.

The conic sections remained the primary object of study in algebraic geometry well into the 19th century, and
the transition to complex numbers happened only falteringly before 1850.\footnote{How to justify complex numbers? Euler's attitude was that there was nothing to explain. We have these expressions of the form $a+bi$ and they behave arithmetically like numbers, so let's call them numbers, and as for what the symbol $i$ means, well, to sure, you'll never meet a length of $i$ but we can imagine these expressions in our mind and control them, so what more do you want? Euler, and Cauchy after him, was not one for definitions of a philosophical nature, but what he was rejecting was the idea that ultimately mathematical quantities must be exhibited in nature: three sheep, a length of $\sqrt{2}$, and so on.} Several features promoted the use of synthetic methods (though algebraic methods were also often used.)  They can be elegant when algebraic methods are blunt; they correspond to the visual form of the conics; they provide a language for describing what is apparent or to be found in a problem. Against them is the obstinate fact that algebra is more general: it does not care if some quantities become negative, but what is a negative length? Once ways round that were found (by Poncelet and then Chasles, around 1820) the way was open to a truly systematic synthetic theory of conics. 

However, the synthetic methods did not extend to higher-order curves.  Although Euler and Cramer (1707--1752) had written on cubic and quartic curves, Pl\"ucker (1801--1868) saw that much needed to be done. His key idea was to study families of curves, using the symbolic notation he devised: If $S_1$ and $S_2$ stand for the equations of two curves of the same degree, then $S_1 + \lambda S_2$ is the equation of another affine curve of that kind---a linear series. But In his study of curves he discussed them first in  the plane, and then as they went off to infinity; he didn't say that the line at infinity could be mapped by a projective transformation into the finite part of the plane.

The way forward was indicated by M\"obius (1790--1868), who introduced barycentric coordinates: pick three points forming a triangle, and attach weights, positive, zero, or negative (not all zero) to these points to compute a point that is the barycenter.  The line at infinity, which is invisible in Cartesian coordinates, is a perfectly sensible line in barycentric coordinates. In this way M\"obius obtained a simple theory of conics and duality in the plane.  

However the first complete treatment of complex projective space is that of  von Staudt (1798--1867) in
his book Geometrie der Lage (Geometry of Position) (1847). Subsequent treatments were based on this. In 1922 H. F. Baker, the leading British algebraic geometer of his time, wrote: 
\begin{quotation}
 It was von Staudt to whom the elimination of the ideas of distance and congruence was a conscious aim, if, also, the recognition of the importance of this might have been much delayed save for the work of Cayley and Klein upon the projective theory of distance. Generalised, and combined with the subsequent Dissertation of Riemann, v. Staudt's volumes must be held to be the foundation of what, on its geometrical side, the Theory of Relativity, in Physics, may yet become \cite[p. 176]{MR2849917}.
\end{quotation}
%footer for separate chapter files

\ifx\whole\undefined
%\makeatletter\def\@biblabel#1{#1]}\makeatother
\makeatletter \def\@biblabel#1{\ignorespaces} \makeatother
\bibliographystyle{msribib}
\bibliography{slag}

%%%% EXPLANATIONS:

% f and n
% some authors have all works collected at the end

\begingroup
%\catcode`\^\active
%if ^ is followed by 
% 1:  print f, gobble the following ^ and the next character
% 0:  print n, gobble the following ^
% any other letter: normal subscript
%\makeatletter
%\def^#1{\ifx1#1f\expandafter\@gobbletwo\else
%        \ifx0#1n\expandafter\expandafter\expandafter\@gobble
%        \else\sp{#1}\fi\fi}
%\makeatother
\let\moreadhoc\relax
\def\indexintro{%An author's cited works appear at the end of the
%author's entry; for conventions
%see the List of Citations on page~\pageref{loc}.  
%\smallbreak\noindent
%The letter `f' after a page number indicates a figure, `n' a footnote.
}
\printindex[gen]
\endgroup % end of \catcode
%requires makeindex
\end{document}
\else
\fi
