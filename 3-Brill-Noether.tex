%header and footer for separate chapter files

\ifx\whole\undefined
\documentclass[12pt, leqno]{book}
\usepackage{graphicx}
\input style-for-curves.sty
\usepackage{hyperref}
\usepackage{showkeys} %This shows the labels.
%\usepackage{SLAG,msribib,local}
%\usepackage{amsmath,amscd,amsthm,amssymb,amsxtra,latexsym,epsfig,epic,graphics}
%\usepackage[matrix,arrow,curve]{xy}
%\usepackage{graphicx}
%\usepackage{diagrams}
%
%%\usepackage{amsrefs}
%%%%%%%%%%%%%%%%%%%%%%%%%%%%%%%%%%%%%%%%%%
%%\textwidth16cm
%%\textheight20cm
%%\topmargin-2cm
%\oddsidemargin.8cm
%\evensidemargin1cm
%
%%%%%%Definitions
%\input preamble.tex
%\input style-for-curves.sty
%\def\TU{{\bf U}}
%\def\AA{{\mathbb A}}
%\def\BB{{\mathbb B}}
%\def\CC{{\mathbb C}}
%\def\QQ{{\mathbb Q}}
%\def\RR{{\mathbb R}}
%\def\facet{{\bf facet}}
%\def\image{{\rm image}}
%\def\cE{{\cal E}}
%\def\cF{{\cal F}}
%\def\cG{{\cal G}}
%\def\cH{{\cal H}}
%\def\cHom{{{\cal H}om}}
%\def\h{{\rm h}}
% \def\bs{{Boij-S\"oderberg{} }}
%
%\makeatletter
%\def\Ddots{\mathinner{\mkern1mu\raise\p@
%\vbox{\kern7\p@\hbox{.}}\mkern2mu
%\raise4\p@\hbox{.}\mkern2mu\raise7\p@\hbox{.}\mkern1mu}}
%\makeatother

%%
%\pagestyle{myheadings}

%\input style-for-curves.tex
%\documentclass{cambridge7A}
%\usepackage{hatcher_revised} 
%\usepackage{3264}
   
\errorcontextlines=1000
%\usepackage{makeidx}
\let\see\relax
\usepackage{makeidx}
\makeindex
% \index{word} in the doc; \index{variety!algebraic} gives variety, algebraic
% PUT a % after each \index{***}

\overfullrule=5pt
\catcode`\@\active
\def@{\mskip1.5mu} %produce a small space in math with an @

\title{Personalities of Curves}
\author{\copyright David Eisenbud and Joe Harris}
%%\includeonly{%
%0-intro,01-ChowRingDogma,02-FirstExamples,03-Grassmannians,04-GeneralGrassmannians
%,05-VectorBundlesAndChernClasses,06-LinesOnHypersurfaces,07-SingularElementsOfLinearSeries,
%08-ParameterSpaces,
%bib
%}

\date{\today}
%%\date{}
%\title{Curves}
%%{\normalsize ***Preliminary Version***}} 
%\author{David Eisenbud and Joe Harris }
%
%\begin{document}

\begin{document}
\maketitle

\pagenumbering{roman}
\setcounter{page}{5}
%\begin{5}
%\end{5}
\pagenumbering{arabic}
\tableofcontents
\fi

%\documentclass[12pt, leqno]{book}
%\usepackage{amsmath,amscd,amsthm,amssymb,amsxtra,latexsym,epsfig,epic,graphics}
%\usepackage[matrix,arrow,curve]{xy}
%\usepackage{graphicx}
%\usepackage{diagrams}
%%\usepackage{amsrefs}
%%%%%%%%%%%%%%%%%%%%%%%%%%%%%%%%%%%%%%%%%%
%%\textwidth16cm
%%\textheight20cm
%%\topmargin-2cm
%\oddsidemargin.8cm
%\evensidemargin1cm
%
%%%%%%Definitions
%\input preamble.tex
%\def\TU{{\bf U}}
%\def\AA{{\mathbb A}}
%\def\BB{{\mathbb B}}
%\def\CC{{\mathbb C}}
%\def\QQ{{\mathbb Q}}
%\def\RR{{\mathbb R}}
%\def\facet{{\bf facet}}
%\def\image{{\rm image}}
%\def\cE{{\cal E}}
%\def\cF{{\cal F}}
%\def\cG{{\cal G}}
%\def\cH{{\cal H}}
%\def\cHom{{{\cal H}om}}
%\def\h{{\rm h}}
% \def\bs{{Boij-S\"oderberg{} }}
%
%\makeatletter
%\def\Ddots{\mathinner{\mkern1mu\raise\p@
%\vbox{\kern7\p@\hbox{.}}\mkern2mu
%\raise4\p@\hbox{.}\mkern2mu\raise7\p@\hbox{.}\mkern1mu}}
%\makeatother
%
%%%
%%\pagestyle{myheadings}
%\date{April 30, 2018}
%%\date{}
%\title{Curves}
%%{\normalsize ***Preliminary Version***}} 
%\author{David Eisenbud and Joe Harris }
%
%\begin{document}

\chapter{Brill-Noether Theory}
\section{What is a family of varieties}

\section{Discussion of $M_g$: dimension $3g-3$, irreducible}

\section{What linear series exist?}

In the last chapter, we established a basic correspondence between maps of curves to projective space and linear systems. The next question to ask, naturally, is ``What linear systems exist?"

There are various ways to interpret this question. Let's start by taking the question is its plain, unvarnished form---for which $g, r$ and $d$ does there exist a curve $C$ of genus $g$ and a linear system $(\cL,V)$ on $C$ of degree $d$ and dimension $r$? In this form, the answer is given for line bundles of large degree $d \geq 2g-1$ by the Riemann-Roch theorem: on any curve, there exists a linear series of degree $d \geq 2g-1$ and dimension $r$ iff $r \leq d-g$. 

\subsection{Clifford's theorem} 

Riemann-Roch still leaves open the question of what linear systems of degree $d \leq 2g-2$ may exist on a curve of genus $g$. The answer is given by the classical theorem of Clifford:

\begin{theorem}\label{Clifford}
Let $C$ be a curve of genus $g$ and $\cL$ a line bundle of degree $d \leq 2g-2$. Then
$$
r(\cL) \leq \frac{d}{2}.
$$
Moreover, if  equality holds then we must have either
\begin{enumerate}
\item $d=0$ and $\cL = \cO_C$;
\item $d = 2g-2$ and $\cL = K_C$; or
\item $C$ is hyperelliptic, and $|\cL|$ is a multiple of the $g^1_2$ on $C$.
\end{enumerate}
\end{theorem}

\begin{proof}

\end{proof}

Combining this with Riemann-Roch, we arrive at the

\begin{theorem}\label{arbitrary linear series}
There exists a curve $C$ of genus $g$ and line bundle $\cL$ of degree $d$ on $C$ with $h^0(\cL) \geq r+1$ if and only if
$$
r \leq
\begin{cases}
d-g, \quad \text{if } d \geq 2g-1; \text{ and} \\
d/2,  \quad \text{if } 0 \leq d \leq 2g-2.
\end{cases}
$$
\end{theorem}

\subsection{Castelnuovo's theorem}

Theorem~\ref{arbitrary linear series} gives a complete and sharp answer to the question originally posed: for which $d,r$ and $g$ does there exists a triple $(C,\cL,V)$ with $C$ a curve of genus $g$, $\cL$ a line bundle of degree $d$ on $C$ and $V \subset H^0(\cL)$ of dimension $r+1$. 

But maybe that wasn't the question we meant to ask! After all, we're interested in describing curves in projective space as images of abstract curves $C$ under maps given by linear systems on $C$. Observing that the linear series that achieve equality in Clifford's theorem give maps to $\PP^r$ that are 2 to 1 onto a rational curve, we might hope that we would have a different---and more meaningful---answer if we  restrict our attention to linear series $\cD = (\cL,V)$ for which the associated map $\phi_\cD$ is at least a birational embedding.  With this restriction, the question is tantamount to the

\begin{question}
What is the largest possible genus of an irreducible, nondegenerate curve $C \subset \PP^r$ of degree $d$?
\end{question}

The answer to this question is indeed quite different from the inequality provided by Theorem~\ref{arbitrary linear series}. It is the content of \emph{Castelnuovo's theorem}, which gives a sharp answer to this question, and which we'll describe in detail in Chapter~\ref{}.

\section{Brill-Noether theory}

\subsection{Basic questions addressed by Brill-Noether theory}

\subsection{Heuristic argument leading to the statement of BN}

The Brill-Noether theorem, as we'll see, is a far-reaching description of the linear series to be found on a general curve. It starts, though, with a relatively simple dimension count---one that was first carried out almost a century and a half ago.

To set this up, let $C$ be a smooth projective curve of genus $g$, and $D = p_1 + \dots + p_d$ a divisor on $C$. We'll assume here the points $p_i$ are distinct; the same argument (albeit with much more complicated notation) can be carried out in general.

When does the divisor $D$ move in an $r$-dimensional linear series? Riemann-Roch gives an answer: it says that $h^0(D) \geq r+1$ if and only if the vector space $H^0(K-D)$ of 1-forms vanishing on $D$ has dimension at least $g-d+r$---that is, if and only if the rank of the evaluation map
$$
H^0(K) \to H^0(K|_D) = \oplus K_{p_i}
$$
has rank at most $d-r$. 

We can represent this map by a $g \times d$ matrix. Choose a basis $\omega_1,\dots,\omega_g$ for the space $H^0(K)$ of 1-forms on $C$; choose an analytic open neighborhood $U_j$ of each point $p_j \in D$ and choose a local coordinate $z_j$ in $U_j$ around each point $p_j$, and write
$$
\omega_i = f_{i,j}(z_j)dz_j
$$
in $U_j$. We will have $r(D) \geq r$ if and only if the  matrix-valued function
$$
A(z_1,\dots,z_d) = 
\begin{pmatrix}
f_{1,1}(z_1) & f_{2,1}(z_1) & \dots & f_{g,1}(z_1) \\
f_{1,2}(z_2) & f_{2,2}(z_2) & \dots & f_{g,2}(z_2) \\
\vdots & \vdots &  & \vdots \\
f_{1,d}(z_d) & f_{2,d}(z_d) & \dots & f_{g,d} (z_d)
\end{pmatrix}
$$
has rank $d-r$ or less at $(z_1,\dots,z_d) = (0,\dots,0)$.

The point is, we can think of $A$ as a matrix valued function in the open set $U = U_1 \times U_2 \times \dots \times U_d \subset C_d$; and for divisors $D \in U$, we have $r(D) \geq r$ if and only if $\rank(A(D)) \leq d-r$. Now, in the space $M_d,g$ of $d \times g$ matrices, the subset of matrices of rank $d-r$ or less has codimension $r(g-d+r)$, and so we might naively expect that the locus of divisors with $r(D) \geq r$ would have dimension $d - r(g-d+r)$

\section{Statement of BN and add-ons}

 (very ampleness if $r \geq 3$, irreducibility of $W^r_d$ when $\rho > 0$, etc.)

\section{Special cases and consequences}


\input footer.tex