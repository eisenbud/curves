%header and footer for separate chapter files

\ifx\whole\undefined
\documentclass[12pt, leqno]{book}
\usepackage{graphicx}
\input style-for-curves.sty
\usepackage{hyperref}
\usepackage{showkeys} %This shows the labels.
%\usepackage{SLAG,msribib,local}
%\usepackage{amsmath,amscd,amsthm,amssymb,amsxtra,latexsym,epsfig,epic,graphics}
%\usepackage[matrix,arrow,curve]{xy}
%\usepackage{graphicx}
%\usepackage{diagrams}
%
%%\usepackage{amsrefs}
%%%%%%%%%%%%%%%%%%%%%%%%%%%%%%%%%%%%%%%%%%
%%\textwidth16cm
%%\textheight20cm
%%\topmargin-2cm
%\oddsidemargin.8cm
%\evensidemargin1cm
%
%%%%%%Definitions
%\input preamble.tex
%\input style-for-curves.sty
%\def\TU{{\bf U}}
%\def\AA{{\mathbb A}}
%\def\BB{{\mathbb B}}
%\def\CC{{\mathbb C}}
%\def\QQ{{\mathbb Q}}
%\def\RR{{\mathbb R}}
%\def\facet{{\bf facet}}
%\def\image{{\rm image}}
%\def\cE{{\cal E}}
%\def\cF{{\cal F}}
%\def\cG{{\cal G}}
%\def\cH{{\cal H}}
%\def\cHom{{{\cal H}om}}
%\def\h{{\rm h}}
% \def\bs{{Boij-S\"oderberg{} }}
%
%\makeatletter
%\def\Ddots{\mathinner{\mkern1mu\raise\p@
%\vbox{\kern7\p@\hbox{.}}\mkern2mu
%\raise4\p@\hbox{.}\mkern2mu\raise7\p@\hbox{.}\mkern1mu}}
%\makeatother

%%
%\pagestyle{myheadings}

%\input style-for-curves.tex
%\documentclass{cambridge7A}
%\usepackage{hatcher_revised} 
%\usepackage{3264}
   
\errorcontextlines=1000
%\usepackage{makeidx}
\let\see\relax
\usepackage{makeidx}
\makeindex
% \index{word} in the doc; \index{variety!algebraic} gives variety, algebraic
% PUT a % after each \index{***}

\overfullrule=5pt
\catcode`\@\active
\def@{\mskip1.5mu} %produce a small space in math with an @

\title{Personalities of Curves}
\author{\copyright David Eisenbud and Joe Harris}
%%\includeonly{%
%0-intro,01-ChowRingDogma,02-FirstExamples,03-Grassmannians,04-GeneralGrassmannians
%,05-VectorBundlesAndChernClasses,06-LinesOnHypersurfaces,07-SingularElementsOfLinearSeries,
%08-ParameterSpaces,
%bib
%}

\date{\today}
%%\date{}
%\title{Curves}
%%{\normalsize ***Preliminary Version***}} 
%\author{David Eisenbud and Joe Harris }
%
%\begin{document}

\begin{document}
\maketitle

\pagenumbering{roman}
\setcounter{page}{5}
%\begin{5}
%\end{5}
\pagenumbering{arabic}
\tableofcontents
\fi


\chapter{Curves of genus 4 and 5}\label{genus 4, 5 Chapter}

In this Chapter we focus on the linear systems that exist on curves of genus 4 and 5, and what this says about maps to $\PP^r$, focusing on the nonhyperelliptic case, which is quite different. Modulo a general position result that we will put off until Chapter~\ref{uniform position} we'll also prove the
important Theorem~\ref{canonical curves are ACM}, which allows us to identify the Hilbert functions of canonical curves of any genus.

We will say more about hyperelliptic curves in Chapter~\ref{ScrollsChapter}. 
In particular we will discuss the canonical embeddings of these curves. 
\section{Curves of genus 4}

In genus 4 we have a question that the elementary theory based on the Riemann-Roch formula cannot answer: are nonhyperelliptic curves of genus 4 \emph{trigonal}, that is, expressible as three-sheeted covers of $\PP^1$? The answer will emerge from our analysis in Proposition~\ref{genus 4 trigonal} below.

\subsection{The canonical model}\label{canonical genus 4}

Let $C$ be a non-hyperelliptic curve of genus 4. We start by considering the canonical map $\phi_K : C \hookrightarrow \PP^3$, which embeds $C$ as a curve of degree 6 in $\PP^3$. We identify $C$ with its image, and investigate the homogeneous ideal $I = I_C$ of equations it satisfies. As in previous cases we may try to describe $I$ by considering the restriction maps
$$
\rho_m : H^0(\cO_{\PP^3}(m)) \; \to \; H^0(\cO_{C}(m)) = H^0(mK_C).
$$

For $m=1$, this is by construction an isomorphism; that is, the image of $C$ is non-degenerate (not contained in any plane).

For $m=2$ we know that $\h^0(\cO_{\PP^3}(2)) = \binom{5}{3} = 10$, while by the Riemann-Roch
theorem we have
$$
h^0(\cO_C(2)) = 12 - 4 + 1 = 9.
$$
This shows that the curve $C \subset \PP^3$ must lie on at least one quadric surface $Q$. The quadric $Q$ must be irreducible, since any any reducible and/or non-reduced quadric must be a union of planes, and thus cannot contain an irreducible non-degenerate curve.
If $Q'\neq Q$ is any other quadric then, by B\'ezout's theorem, $Q\cap Q'$ is a curve of degree 4 and thus could not contain $C$. From this we see that $Q$ is unique, and it follows that $\rho_2$ is surjective.

What about cubics? Again we consider the restriction map
$$
\rho_3 : H^0(\cO_{\PP^3}(3)) \; \to \; H^0(\cO_{C}(3)) = H^0(3K_C).
$$
The space $H^0(\cO_{\PP^3}(3))$ has dimension $\binom{6}{3} = 20$, while  the Riemann-Roch formula shows that
$$
h^0(\cO_C(3)) = 18 - 4 + 1 = 15.
$$
It follows that the ideal of $C$ contains at least a 5-dimensional vector space of cubic polynomials. We can get a 4-dimensional subspace as products of the unique quadratic polynomial $F$ vanishing on $C$ with linear forms---these define the cubic surfaces containing $Q$. Since $5 > 4$ we  conclude that the curve $C$ lies on at least one cubic surface $S$  not containing $Q$. 
B\'ezout's theorem shows that the curve $Q \cap S$ has degree 6; thus it must be equal to $C$. 

Let $G=0$ be the cubic form defining the surface $S$. By Lasker's theorem the ideal $(F,G)$ is unmixed, and thus is equal to the homogeneous ideal of $C$. Putting this together, we have proven the first statement of the following result:

\begin{theorem}
The canonical model of any nonhyperelliptic curve of genus 4 is a complete intersection of a quadric $Q = V(F)$ and a cubic surface $S = V(G)$ meeting transversely along nonsingular points of each. Conversely, any smooth curve that is the complete intersection of a quadric and a cubic surface in $\PP^3$ is the canonical model of a nonhyperelliptic curve of genus 4.
\end{theorem}
 
\begin{proof}
Let $C = Q\cap S$ with $Q$ a quadric and $S$ a cubic. Because $C$ is nonsingular and a complete intersection, both $S$ and $Q$ must be nonsingular at every point of their intersection Applying the adjunction formula to $Q\subset \PP^3$ we get
$$
\omega_Q = (\omega_{\PP^3} \otimes \cO_{\PP^3}(2))|_Q = \cO_Q(-4+2) = \cO_Q(-2).
$$
Applying it again to $C$ on $Q$, and noting that $\cO_Q(C) = \cO_Q(3)$, we get
$$
\omega_C = ((\omega_{Q} \otimes \cO_{3}(3))|_C = \cO_C(-2+3) = \cO_C(1)
$$
as required. 
\end{proof}

Since the quadric surface $Q$ containing the canonical curve $C$  is unique, its rank is an invariant of $C$.
Since $C$ is irreducible and non-degenerate, the quadric cannot be a double plane or the union of two planes, but it can be singular (rank 3) or smooth (rank 4). On the other hand, the singularities of a cubic $F$ such that $F\cap Q = C$ play no role. Of course $F$ must be nonsingular along $C$, since else 
$C$ would be singular. We can vary $F$ by adding a multiple of the equation of $Q$ to the equation of $F$, and since this linear system of cubics has base locus only along $C$, Bertini's Theorem shows that the general such cubic is nonsingular everywhere.

\subsection{Maps to projective space}

\subsubsection{Maps to $\PP^1$}

We can now answer the question we asked at the outset, whether a nonhyperelliptic curve of genus 4 can be expressed as a three-sheeted cover of $\PP^1$. This amounts to asking if there are any divisors $D$ on $C$ of degree 3 with $r(D) \geq 1$; since we can take $D$ to be a general fiber of a map $\pi : C \to \PP^1$, we can for simplicity assume $D = p+q+r$ is the sum of three distinct points.

By the geometric Riemann-Roch theorem, a divisor $D = p+q+r$ on a canonical curve $C \subset \PP^{g-1}$ has $r(D) \geq 1$ if and only if the three points $p,q,r \in C$ are colinear. If three points $p,q,r \in C$ lie on a line $L \subset \PP^3$ then the quadric $Q$ would meet $L$ in at least three points, and hence would contain $L$. Conversely,  if $L$ is a line contained in $Q$, then the divisor $D = C \cap L = S \cap L$ on $C$ has degree  3. Thus we can answer our question in terms of the family of lines contained in $Q$.

Any smooth quadric is isomorphic to $\PP^1\times \PP^1$, and contains two families of lines, or \emph{rulings}. On the other hand, any  quadric of rank 3 is a cone over a smooth plane conic, and thus has just one ruling. By the argument above, the pencils of divisors on $C$ cut out by the lines of these rulings are the $g^1_3$s on $C$. This proves:

\begin{proposition}\label{genus 4 trigonal}
A nonhyperelliptic curve of genus 4 may be expressed as a 3-sheeted cover of $\PP^1$ in either one or two ways, depending on whether the unique quadric containing the canonical model of the curve is singular or smooth.
\end{proposition}

A curve expressible as a 3-sheeted cover of $\PP^1$ is called \emph{trigonal}; by the analyses of the preceding sections, we have shown that \emph{every curve of genus $g \leq 4$ is either hyperelliptic or trigonal}. 

\subsubsection{Maps to $\PP^2$}

We can also describe the lowest degree plane models of nonhyperelliptic curves $C$ of genus 4. 
We can always get a plane model of degree 5 by projecting $C$ from a point $p$ of the canonical model of $C$. Moreover, the Riemann-Roch theorem shows that if $D$ is a divisor of degree 5 with $r(D)=2$ then,  $\h^0(K-D) = 1$. Thus $D$ is of the form $K-p$ for some point $p \in C$, and the map to $\PP^2$ corresponding to $D$ is $\pi_p$. These  maps $\pi_p: C\to \PP^2$ have the lowest possible degree (except for those whose image is  contained in a line) because, by Clifford's theorem a nonhyperelliptic curve of genus 4 cannot have a $g^2_4$.

We now consider the singularities of the plane quintic $\pi_p(C)$. Suppose as above that $C = Q\cap S$, with $Q$ a quadric. If a line $L$ through $p$ meets $C$ in $p$ plus a divisor of degree $\geq 2$ then, as we have seen, $L$ must lie in $Q$.  All other lines through $p$ meet $C$ in at most a single reduced point,  whose image is thus a nonsingular point of $\pi(C)$. Moreover, a line that met $C$ in $>3$ points would have to lie in both the quadric and the cubic containing $C$, and therefore would be contained in $C$. Since $C$ is irreducible there can be no such line; thus the image $\pi_p(C)$ has at most double points.

We will distinguish two cases, depending on whether the quadric $Q$ is smooth or singular. We will make use of the Gauss map of the quadric, described by the next Lemma.

\begin{lemma}
Let $L \subset S \subset \PP^3$ be a line on a surface $S \subset \PP^3$ of degree $d$. The Gauss map $\cG : S \to {\PP^3}^*$ sending each point $p \in S$ to the tangent plane $\TT_p(S)$ maps $L$ into the dual line in $\PP^3$ (that is, the locus of planes containing $L$); if $S$ is smooth along $L$ then $\cG$  has degree $d-1$, and if $S$ is singular anywhere along $L$ it has strictly lower degree.
\end{lemma}

\begin{proof}
Suppose that in terms of homogeneous coordinates $[X,Y,Z,W]$ on $\PP^3$ the line $L$ is given by $X = Y = 0$. Then the defining equation $F$ of $S$ can be written
$$
F(X,Y,Z,W) = X\cdot G(Z,W) + Y\cdot H(Z,W) + J(X,Y,Z,W)
$$
where $J$ vanishes to order 2 along $L$; that is, $J \in (X,Y)^2$. The Gauss map $\cG|_L$ restricted to $L$ is then given by
$$
[0,0,Z,W] \mapsto [G(Z,W), H(Z,W), 0, 0].
$$
The polynomials $G$ and $H$ have degree $d-1$, and have a common zero if and only if $S$ is singular somewhere along $L$; the lemma follows.
\end{proof}

\begin{example} (Gauss map of a quadric)\label{Gauss of Quadric}
 Let $Q\subset \PP^3$ be a smooth quadric, and let $L\subset Q$ be the line $X=Y =0$. Since we may write the equation of $Q$ as $XZ+YW = 0$, the Gauss map of $Q$, restricted to $L$, maps $L$ one-to-one onto the dual line. Indeed, the Gauss map takes $Q$ isomorphically onto its dual, which is also a smooth quadric.
 
 We can also see this geometrically: if $H \subset \PP^3$ is any plane containing the line $L \subset Q$, then $H$ intersects $Q$ in the union of $L$ and a line $M$; the hyperplane section $Q \cap H = L \cup M$ is then singular at a unique point $p \in L$. Thus the Gauss map gives a bijection between points on $L$ and planes containing $L$. 
\end{example}

Given this, we can analyze the geometry of projections $\pi_p(C)$ of our canonical curve $C = Q \cap S$ as follows:

\begin{enumerate}
\item $Q$ is nonsingular:
In this case there are two lines $L_1, L_2$ on $Q$ that pass through $p$; they meet $C$ in $p$ plus divisors $E_1$ and $E_2$ of degree 2. If each $E_i$ consists of distinct points, then, since the tangent planes to the quadric along $L_i$ are all distinct by Example~\ref{Gauss of Quadric} the plane curve $\pi(C)$ has two nodes, one at the image of each $E_i$.

On the other hand, if $E_i$ consists of a double point $2q$ (that is, $L_i$ is tangent to $C$ at $q\neq p$, or meets $C$ three times at $q = p$), then $\pi(C)$ has a cusp at the corresponding image point. 
In either case, $\pi(C)$ has two distinct singular points, each either a node or a cusp. The two $g^1_3$s on $C$ correspond to the projections from these singular points.

\pict{illustrate this 8.1 (same as cover picture)}

\item $Q$ is a cone:
In this case, since the curve cannot pass through the singular point of $Q$ there is a unique line $L\subset Q$ that passes through $p$. Let $p+E$ be the divisor on $C$ in which this line meets $C$. The tangent planes to $Q$ along $L$ are all the same. Thus if $E = q_1+q_2$ consists of two distinct points, the image $\pi_p(C)$ has two smooth branches sharing a common tangent line at
$\pi_p(q_1) = \pi_p(q_2)$. Such a point is called a \emph{tacnode} of $\pi_p(C)$. On the other hand, if $E= 2q$, that is, if $L$ meets $C$ tangentially at one point $q\neq p$ (or meets $C$ 3 times at $p$) then the image curve has a higher order cusp, called a \emph{ramphoid cusp}. In either case, the one $g^1_3$ on $C$ is the projection from the unique singular point of $\pi(C)$.
\pict{8.2: illustrate this case too.}

\end{enumerate}

\pict{add pictures illustrating some of the possibilities above.}


\section{Curves of genus 5}

We next consider a nonhyperelliptic curve $C$ of genus 5. There are now two questions that cannot be answered by simple application of the Riemann-Roch theorem:

\begin{enumerate}
\item Is $C$ expressible as a 3-sheeted cover of $\PP^1$? In other words, does $C$ have a $g^1_3$?
\item Is $C$ expressible as a 4-sheeted cover of $\PP^1$? In other words, does $C$ have a base-point-free $g^1_4$?
\end{enumerate}

As in the preceding case, the answers can be found through an investigation of the geometry of the canonical model $C \subset \PP^4$ of $C$. This is an octic curve in $\PP^4$, and as before the first question to ask is what sort of polynomial equations define $C$. We start with quadrics, by considering the restriction map
$$
r_2 : H^0(\cO_{\PP^4}(2)) \; \to \; H^0(\cO_{C}(2)).
$$
On the left, we have the space of homogeneous quadratic polynomials on $\PP^4$, which has dimension $\binom{6}{4} = 15$, while by the Riemann-Roch theorem the target is a vector space of dimension
$$
2\cdot8 - 5 + 1 = 12.
$$
We deduce that $C$ lies on at least 3 independent quadrics. (We will see in the course of the following analysis that it is exactly 3; that is, $r_2$ is surjective.) Since $C$ is irreducible and, by construction, does not lie on a hyperplane, each of the quadrics containing $C$ is irreducible, and thus the intersection of any two is a surface of degree 4. There are now two possibilities:  The intersection of (some) three quadrics $Q_1 \cap Q_2 \cap Q_3$ containing the curve is 1-dimensional; or every such intersection is two dimensional. 

\subsection{First case: the intersection of the quadrics is a curve}\label{non-trigonal genus 5}

We first consider the case where $Q_1 \cap Q_2 \cap Q_3$ is 1-dimensional.  By B\'ezout's theorem the intersection is a curve of degree 8, and since $C$ also has degree 8 we must have $C=Q_1 \cap Q_2 \cap Q_3$; that is, the canonical curve is a complete intersection. Lasker's theorem then shows that the three quadrics $Q_i$ generate the whole homogeneous ideal of $C$; in particular, there are no additional quadrics containing $C$.

We can now answer the first of our two questions for curves of this type. As in the genus 4 case the geometric Riemann-Roch theorem implies that $C$ has a $g^1_3$ if and only if the canonical model of $C$ contains 3 colinear points or, more generally, meets a line $L$ in a divisor of 3 points. When $C$ is the intersection of quadrics, this cannot happen, since the line $L$ would have to be contained in all the quadrics that contain $C$. Thus, in this case, 
$C$ has no $g^1_3$.

What about $g^1_4$s? Again invoking the geometric Riemann-Roch theorem, a divisor of degree 4 moving in a pencil lies in a 2-plane; so the question is, does $C \subset \PP^4$ contain a divisor of degree 4, say $D = p_1+\dots +p_4 \subset C$, that lies in a plane $\Lambda$? Supposing this is so, we consider the restriction map
$$
H^0(\cI_{C/\PP^4}(2)) \; \to \; H^0(\cI_{D/\Lambda}(2)).
$$
By what we have said, the left hand space is 3-dimensional. We will show that the right-hand space
is 2-dimensional, so that one of the quadrics vanishes identically on $\Lambda$.

\begin{lemma}\label{4-tuples}
Let $\Gamma \subset \PP^2$ be any scheme of dimension 0 and degree 4. Either $\Gamma$ is contained in a line $L \subset \PP^2$, or $\Gamma$ imposes independent conditions on quadrics, that is, $h^0(\cI_{\Gamma /\PP^2}(2)) = 2$.
\end{lemma}

\begin{proof}
We will do this in case $\Gamma$ is reduced, that is, consists of four distinct points; the reader is asked to supply the analogous argument in the general case in Exercise~\ref{non-red 4-tuples}. Suppose to begin with that $\Gamma$ fails to impose independent conditions on quadrics, and let $q \in \PP^2$ be a general point. Since we are assuming that $h^0(\cI_{\Gamma /\PP^2}(2)) \geq 3$, we see that there are at least two conics $C', C'' \subset \PP^2$ containing $\Gamma \cup \{q\}$. By B\'ezout's Theorem, these two conics have a component in common, which can only be a line $L$; thus we can write $C' = L \cup L'$ and $C'' = L \cup L''$ for some pair of distinct lines $L', L'' \subset \PP^2$. The intersection $C' \cap C''$ thus consists of the line $L$ and the single point $L' \cap L"$. Since this must contain $\Gamma \cup \{q\}$, and $q$ does not lie on the line joining any two points of $\Gamma$, we conclude that $L' \cap L'' = \{q\}$ and hence $\Gamma \subset L$.
\end{proof}

\begin{fact}
Lemma~\ref{4-tuples} is the first case of a more general statement: If $n\leq 2d+1$ points in the plane fail to impose independent conditions on forms of degree $d$, then $d+2$ of the point lie on a line. See \cite[p. 302]{MR1376653} for a proof.
\end{fact}


 It follows that the 2-plane $\Lambda$ spanned by $D$ must be contained in one of the quadrics $Q \subset \PP^4$ containing $C$. This implies in particular that the quadric is singular: If $V = \CC^3\subset \CC^5$
is a  3-dimensional subspace of a 5-dimensional inner-product space, then $V$ meets its orthogonal space
in a line, which is a singular point of the corresponding quadric.
 
Thus $Q$ is a cone over a quadric in $\PP^3$, and it is ruled by the (one or two) families of 2-planes it contains, which are the cones over the (one or two) rulings of the quadric in $\PP^3$. The argument above shows that the existence of a $g_4^1$s on $C$ in this case implies the existence of a singular quadric containing $C$.

Conversely, suppose that $Q \subset \PP^4$ is a singular quadric containing $C = Q_1 \cap Q_2 \cap Q_3$. Now say $\Lambda \subset Q$ is  a 2-plane. If $Q'$ and $Q''$ are ``the other two quadrics" containing $C$, we can write
$$
\Lambda \cap C = \Lambda \cap Q' \cap Q'', 
$$ 
from which we see that $D = \Lambda \cap C$ is a divisor of degree 4 on $C$, and so has $r(D) = 1$ by the geometric Riemann-Roch theorem. Thus, the rulings of  singular quadrics containing $C$ cut out on $C$ pencils of degree 4; and every pencil of degree 4 on $C$ arises in this way.

Does $C$ lie on singular quadrics? There is a $\PP^2$ of quadrics containing $C$---a 2-plane in the space $\PP^{14}$ of quadrics in $\PP^4$---and the family of singular quadrics  consists of a  hypersurface of degree 5 in $\PP^{14}$, called the \emph{discriminant} hypersurface. By Bertini's theorem, not every quadric containing $C$ is singular. Thus the set of singular quadrics containing $C$ is a plane curve $B$ cut out by a quintic equation. So $C$ does indeed have a $g^1_4$, and is expressible as a 4-sheeted cover of $\PP^1$. Moreover, each singular quadric contributes either 1 or 2 $g^1_4$s, depending on whether it has rank 3 or 4. In sum, we have proven:

\begin{proposition}
Let $C \subset \PP^4$ be a canonical curve, and assume $C$ is the complete intersection of three quadrics in $\PP^4$. Then $C$ may be expressed as a 4-sheeted cover of $\PP^1$ in a one-dimensional family of ways, and there is a map from the set $W^1_4(C)$ of $g^1_4$s on $C$ to a plane quintic curve $B$, whose fibers have cardinality 1 or 2.
\end{proposition}

One can go further and ask about the geometry of the plane curve $B$ and how it relates to the geometry of $C$. The list of possibilities is given in \cite[p. 274]{ACGH}. %\fix{possibly include a photo here?}

\subsection{Second case: the intersection of the quadrics is a surface}\label{trigonal genus 5}

At the outset of our analysis of curves of genus 5, we saw that a canonical curve $C \subset \PP^4$ of genus 5 is contained in at least a three-dimensional vector space of quadrics. Assuming that the intersection of these quadrics was 1-dimensional---and hence $C$ was a complete intersection---led us to the analysis above. We'll now consider the alternative: what if the intersection of the quadrics containing our canonical curve is a surface?

The first thing to do in this case is to observe that the intersection must contain an irreducible, nondegenerate surface. This  follows from Fulton's Elementary B\'ezout  theorem:

\begin{theorem}\cite{Fulton}\label{Fulton Bezout}
Let $Z_1,\dots, Z_k \subset \PP^n$ be hypersurfaces of degrees $d_1,\dots,d_k$. If $\Gamma_1,\dots,\Gamma_m$ are the irreducible components of the intersection $\bigcap_1^kZ_j$, then
$$
\sum_{\alpha = 1}^m \deg(\Gamma_\alpha) \; \leq \; \prod_{i=1}^k d_i.
$$
\end{theorem}

\begin{proof}
We do induction on $k$, the result being trivial for $k=1$. Assuming that the the result
is true for $k-1$, we consider the irreducible components $V_i$ of $\bigcap_1^{k-1}Z_j$. If $Z_k$ contains
$V_i$, then $V_i$ is again a component of $\bigcap_1^kZ_j$. Otherwise,
$V_i\cap Z_k$ is a union of components whose degrees sum to $d_k\deg V_i$. Thus
the sum of the degrees of the components of $\bigcap_1^kZ_j$ is at most $d_i$ times the
sum of the degrees of components of $\bigcap_1^{k-1}Z_j$, as required.
\end{proof}

Returning to the canonical curve $C \subset \PP^4$, suppose that the intersection $X = Q_1 \cap Q_2 \cap Q_3$ of the three quadrics containing $C$ has dimension 2. If $C$ were a component of $X$, then the sum of the degrees of the irreducible components of $X$ would be strictly greater than 8, which Fulton's theorem doesn't allow. Thus $C$ must be contained in a 2-dimensional irreducible component  $S$ of $X$, and this surface $S$ is necessarily nondegenerate.

We will return to this surface in Section~\ref{syzy and geom}, and and again in Chapter~\ref{scrollsChapter}, once we develop the theory of rational normal scrolls; see Exercise~\ref{trigonal genus 5}.
Here is the result:

\begin{theorem}
Let $C \subset \PP^4$ be a canonical curve of genus 5. Then $C$ lies on exactly three quadrics, and either
\begin{enumerate}
\item $C$ is the intersection of these quadrics; in which case $C$ is not trigonal, and the variety $W^1_4(C)$ of expressions of $C$ as a 4-sheeted cover of $\PP^1$ is a two-sheeted cover of a plane quintic curve; or
\item The quadrics containing $C$ intersect in a cubic surface that is
a nonsingular rational normal scroll; in this case, $C$ has a unique $g^1_3$, and the variety $W^1_4(C)$ consists of the union of two copies of $C$ meeting at two points.
\end{enumerate}
\end{theorem}

%From Frank: In Cheerful Fact 7.2.6. the rational normal scroll of a trigonal genus 5 curve is a cone over a rational normal curve, while in general it is the blow up of PP^2 in a point.
%DE: clearly both cases occur. How are they distinguished in terms of the $W^r_d$? 

\begin{fact}
In the second case of the Theorem, the ideal of the curve has the form $I_C = (Q_1, Q_2, Q_3, F_1, F_2)$ 
where the $Q_i$ are quadrics and the $F_i$ are cubic forms. The quadrics $Q_i$ cut out a surface scroll, which
must be smooth by Corollary~\ref{curves on a singular scroll}).

Moreover, there are 2 linear relations among the
$Q_i$, and we shall see in Chapter~\ref{SyzygiesChapter} that this is a special case of the Eagon-Northcott
complex. TheThe generators of $I_C$ above may be written as the $4\times 4$ Pfaffians
of a skew-symmetric $5\times 5$ matrix of the form
%$$
%\begin{pmatrix}
%0&0&\ell_0 &\ell_1 &\ell_2\\
%0&0&\ell_1&\ell_2&\ell_3\\
%-\ell_0&-\ell_1& 0&Q_1&Q_2\\
%-\ell_1&-\ell_2&-Q_1&0&Q_3\\
%-\ell_2&-\ell_3&-Q_2&-Q_3&0
%\end{pmatrix}
%$$
$$
\begin{pmatrix}
0&g_1&g_2&\ell_0&\ell_1\\
-g_1&0&g_3&\ell_2&\ell_3\\
-g_2&-g_3&0 &\ell_3&\ell_4\\
-\ell_0&-\ell_2&-\ell_3&0&0\\
-\ell_1&-\ell_3&-\ell_4&0&0
\end{pmatrix}
$$
where $\ell_0,\dots,\ell_3$ are linear forms and $g_1, g_2, g_3$ are quadrics. The
 $2\times 2$
minors of the matrix of linear forms in the last two columns are the three quadrics $Q_i$ contained in the ideal
of $C$, and those two columns are the linear relations on the $Q_i$ mentioned above.
The columns of the whole $5\times 5$ matrix generate the syzygies of $I_C$. Moreover, the
$4\times 4$ Pfaffians of any sufficiently general matrix of this form define a trigonal canonical curve.
\end{fact}



\section{Exercises}

\begin{exercise} \label{ex7.1}
Let $C$ be a smooth projective non-hyperelliptic curve of genus 4, and $|D|$ a $g^1_3$ on $C$ (that is, a linear equivalence class $D$ of degree 3, with $r(D) = 1$). Show that the following are equivalent:
\begin{enumerate}
\item $D \sim K-D$. 
\item the multiplication map $\mu : H^0(D) \otimes H^0(K-D) \to H^0(K)$ fails to be surjective
\item the unique quadric $Q$ containing the canonical curve of $C$ is singular; and
\item $|D|$ is the unique $g^1_3$ on $C$.
\end{enumerate}
\end{exercise}

\begin{exercise}\label{ex7.2}
Let $C$ again be a smooth projective non-hyperelliptic curve of genus 4. We have seen that $C$ is birational to a quintic plane curve $C_0$ with two nodes $p, q \in C_0$. Show that the canonical series of $C$ is cut out by the system of plane conic curves passing through $p$ and $q$ in the sense
that if $D$ is a curve meeting C at each node and transverse to each branch at the nodes, then
the sum of the points of $D\cap C$ \emph{other than the nodes} is a canonical divisor.
\end{exercise}

\begin{exercise}\label{ex7.3}
Let $C$  be a smooth projective non-hyperelliptic curve of genus 4 and let $D$ be a general divisor of degree 7 on $C$. By the $g+3$ theorem (Theorem~\ref{g+3 theorem}), $h^0(D) = 4$ and the map $\phi_D : C \to \PP^3$ is an embedding. Show that the image $C \subset \PP^3$ does not lie on any quadric surfaces, but does lie on two cubic surfaces $S$ and $T$; describe the intersection $S \cap T$.
\end{exercise}

\begin{exercise}\label{ex7.4}
In the setting of the preceding exercise, suppose now that $C$ \emph{is} hyperelliptic. Show that in this case the image of $C$ under the map $\phi_D : C \to \PP^3$ does lie on a quadric surface $Q$, and in fact is a curve of type $(2,5)$ on $Q$. Show also that if $D$ is of the form $D \sim 2g^1_2 + p + q + r$ then the quadric surface $Q$ is singular, and the image curve $\phi_D(C)$ has a triple point at the vertex of $Q$.
 \end{exercise}

\begin{exercise}\label{ex7.5}
Consider a space $M^r_{d,g}$ parametrizing $g^r_d$s on curves of genus $g$; that is,
$$
M^r_{d,g} = \{ (C, L) \mid C \text{ a smooth curve of genus } g, L \in \pic^d(C) \text{ and } h^0(L) \geq r+1 \}.
$$
The analysis of this chapter shows that $M^1_{3,4}$ is a two-sheeted cover of $M_4$. Show that it is in fact irreducible.
\end{exercise}

\begin{exercise}\label{ex7.6}
The arguments in the chapter show that the canonical model of a non-hyperelliptic trigonal curve of genus 5 lies on an irreducible, nondegenerate cubic surface $S \subset \PP^4$. In Chapter~\ref{ScrollsChapter}, we'll see that such a surface is either smooth or a cone over a twisted cubic curve. Show that the latter case cannot occur. 
\end{exercise}

\begin{exercise}\label{ex7.7}
Let $C$ be a smooth projective curve of genus 5. The $g+3$ theorem (Theorem~\ref{g+3 theorem}) says that $C$ admits an embedding in $\PP^3$ as a curve of degree 8. Does it admit an embedding of degree 7?
\end{exercise}

\begin{exercise}\label{non-red 4-tuples}\label{ex7.8}
Complete the proof of Lemma~\ref{4-tuples}
\end{exercise}

\begin{exercise}\label{ex7.9}
Verify that if $C$ is a nonhyperelliptic curve of genus 5 then the variety $W^1_4(C)$ is a curve of arithmetic genus 11.
Hint\ref{ex7.9}: \end{exercise}

\section{Slight hints}
Hint\ref{ex7.1}: Consider the lines of the ruling of the quadric on which the canonical curve lies.

Hint\ref{ex7.2}: Try blowing up the plane at the nodes. Look at Chapter~\ref{PlaneCurveChapter} if you get stuck.
Hint\ref{ex7.3}: Use the adjunction formula for both parts.

Hint\ref{ex7.4}: Show that each of the divisors $E$ of the $g^1_2$ span a line in $\PP^3$. 

Hint\ref{ex7.5}: Consider the incidence correspondence
$$
\Gamma := \{ (Q, L) \in \PP^9 \times \GG(1,3) \mid L \subset Q \}
$$
where $\PP^9$ is the space of quadrics $Q\subset \PP^3$.

Hint\ref{ex7.6}: Consider the image of the divisors $D$ and $2D$, where $D$ is a $g^1_3$.

Hint\ref{ex7.7}: Consider the trigonal and non-trigonal cases separately. 

\input footer.tex


