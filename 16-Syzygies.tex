%header and footer for separate chapter files

\ifx\whole\undefined
\documentclass[12pt, leqno]{book}
\usepackage{graphicx}
\input style-for-curves.sty
\usepackage{hyperref}
\usepackage{showkeys} %This shows the labels.
%\usepackage{SLAG,msribib,local}
%\usepackage{amsmath,amscd,amsthm,amssymb,amsxtra,latexsym,epsfig,epic,graphics}
%\usepackage[matrix,arrow,curve]{xy}
%\usepackage{graphicx}
%\usepackage{diagrams}
%
%%\usepackage{amsrefs}
%%%%%%%%%%%%%%%%%%%%%%%%%%%%%%%%%%%%%%%%%%
%%\textwidth16cm
%%\textheight20cm
%%\topmargin-2cm
%\oddsidemargin.8cm
%\evensidemargin1cm
%
%%%%%%Definitions
%\input preamble.tex
%\input style-for-curves.sty
%\def\TU{{\bf U}}
%\def\AA{{\mathbb A}}
%\def\BB{{\mathbb B}}
%\def\CC{{\mathbb C}}
%\def\QQ{{\mathbb Q}}
%\def\RR{{\mathbb R}}
%\def\facet{{\bf facet}}
%\def\image{{\rm image}}
%\def\cE{{\cal E}}
%\def\cF{{\cal F}}
%\def\cG{{\cal G}}
%\def\cH{{\cal H}}
%\def\cHom{{{\cal H}om}}
%\def\h{{\rm h}}
% \def\bs{{Boij-S\"oderberg{} }}
%
%\makeatletter
%\def\Ddots{\mathinner{\mkern1mu\raise\p@
%\vbox{\kern7\p@\hbox{.}}\mkern2mu
%\raise4\p@\hbox{.}\mkern2mu\raise7\p@\hbox{.}\mkern1mu}}
%\makeatother

%%
%\pagestyle{myheadings}

%\input style-for-curves.tex
%\documentclass{cambridge7A}
%\usepackage{hatcher_revised} 
%\usepackage{3264}
   
\errorcontextlines=1000
%\usepackage{makeidx}
\let\see\relax
\usepackage{makeidx}
\makeindex
% \index{word} in the doc; \index{variety!algebraic} gives variety, algebraic
% PUT a % after each \index{***}

\overfullrule=5pt
\catcode`\@\active
\def@{\mskip1.5mu} %produce a small space in math with an @

\title{Personalities of Curves}
\author{\copyright David Eisenbud and Joe Harris}
%%\includeonly{%
%0-intro,01-ChowRingDogma,02-FirstExamples,03-Grassmannians,04-GeneralGrassmannians
%,05-VectorBundlesAndChernClasses,06-LinesOnHypersurfaces,07-SingularElementsOfLinearSeries,
%08-ParameterSpaces,
%bib
%}

\date{\today}
%%\date{}
%\title{Curves}
%%{\normalsize ***Preliminary Version***}} 
%\author{David Eisenbud and Joe Harris }
%
%\begin{document}

\begin{document}
\maketitle

\pagenumbering{roman}
\setcounter{page}{5}
%\begin{5}
%\end{5}
\pagenumbering{arabic}
\tableofcontents
\fi


\chapter{Syzygies}
\label{SyzygiesChapter}

%Useful for drawing betti tables:
%$$
%\vbox{\offinterlineskip %\baselineskip=15pt
%\halign{\strut\hfil# \ \vrule\quad&# \ &# \ &# \ &# \ &# \ &# \ 
%&# \ &# \ &# \ &# \ &# \ 
%\cr
%degree&\cr
%\noalign {\hrule}
%0&1&--&--&--&--&--&--\cr
%1&--&17&46&45&4&--&--\cr
%2&--&--&--&--&25&18&4\cr
%\noalign{\bigskip}
%\omit&\multispan{8}{\bf Conjectural shape of $F_\bullet$}\cr
%\noalign{\smallskip}
%}}
%$$
%
%
%\centerline{\scriptsize
%\begin{tabular}{r|ccc} 
%$j\backslash i$&0&1&2\\ 
%\hline 
%0&1&$-$&$-$\\ 
%1&$-$&3&2\\ 
%\end{tabular}}
%
%
%$$
%\begin{matrix} 
%j\backslash i&\vline&0   &  1    & \cdots & n    \cr\hline
%\vdots&\vline&\vdots&\vdots & \cdots    &\vdots     \cr 
%       0&\vline&\beta_{0,0}&\beta_{1,1}&\cdots&\beta_{n,n}\cr
%       1&\vline&\beta_{0,1}&\beta_{1,2}&\cdots&\beta_{n,n+1}\cr
%\vdots&\vline&\vdots&\vdots & \cdots    &\vdots     \cr 
%\end{matrix}
%$$         
%%
%
\def\length{{\rm length}}

%Motivation: canonical embedding turns intrinsic invariants into projective invariants. Hilbert Function. Projective Normality; Canonical Module.
%
%What are the projective invariants that correspond to Clifford index? Conjecturally, Green's conjecture. Inequality from Eagon-Northcott.
%
%Hilbert Syzygy theorem, Hilbert function derivation, Unique minimal resolution, Betti table, 
%\fix{ introduce tools as they are used}


In Chapter~\ref{linkageChapter} we studied the resolutions of curves in $\PP^3$ understanding them through their Hartshorne-Rao modules. The
simplest case was that in which the Hartshorne-Rao module vanishes, that is, the case of arithmetically Cohen-Macaulay curves.
In this chapter we will show that curves embedded by complete linear series of high degree, and also canonical curves, are arithmetically
Cohen-Macaulay. We will then introduce the Eagon-Northcott complexes, and use them to study the resolutions of canonical curves
more closely. We close by explaining Green's conjecture, which proposes a particular relationship between the intrinsic geometry
of a curve and the shape of its minimal free resolution in its canonical embedding.

\section{What is a syzygy?}
This section contains no proofs; all of the material may be found in~\cite{Eisenbud1995}.

Let $(R,\gm, k)$ be a local ring and let $M$ be a finitely generated module. A \emph{free resolution} of $M$ is an exact complex
$$
\FF: \cdots \rTo F_s \rTo^{\phi_s} F_{s-1} \rTo^{\phi_{s-1}} \cdots \rTo F_1 \rTo^{\phi_1}  F_0 \rTo M \rTo 0.
$$
The resolution is \emph{minimal} if a minimal set of generators of $F_i$ maps to a minimal set of generators of the kernel of the following map,
or equivalently (by Nakayama's Lemma) the maps in $\FF\otimes_R k$ are all 0.

The first examples of a minimal free resolution are the Koszul complexes. If $I = (f_1,\dots, f_t)$
is a complete intersection, then For $t = 2,3$ these look like
$$
\begin{aligned}
 0\rTo &R \rTo^{
\begin{pmatrix}
-f_2\\f_1
\end{pmatrix}}
 R^2\rTo^{
 \begin{pmatrix}
f_1&f_2
\end{pmatrix}}
 R \rTo R/I \rTo 0\\
%%
0\rTo R \rTo^{
\begin{pmatrix}
f_1\\f_2\\f_3
\end{pmatrix}} R^3&\rTo{
\begin{pmatrix}
 0&f_3&-f_2\\
 -f_3&0&f_1\\
 f_2&-f_1&0
\end{pmatrix}}
 R^3\rTo^{
 \begin{pmatrix}
f_1&f_2&f3
\end{pmatrix}}R \rTo R/I \rTo 0
\end{aligned}
$$

Minimal free resolutions of a given module $M$ are all isomorphic, and the \emph{projective dimension}  of $M$ is their length, the number of nonzero
maps (possibly infinity). Equivalently, 
$$
\pd\ M = \max\{s \mid \Ext^s(M,k) \neq 0.
$$
Fundamental
results of Auslander, Buchsbaum and Serre say that if $R$ is \emph{regular}---that is, the Krull dimension of $R$ is equal to the minimal number
of generators of $\gm$---if and only if the minimal free resolution of the residue field is finite, in which case
the minimal free resolution of every module is finite.

As the theory of linkage in $\PP^3$ shows, a curve $C\subset \PP^3$ is linked to a complete intersection
if and only if  that $H^1(\sI_C(m)) = 0$ for all $m$. This important condition is generalized to all dimensions and
codimensions as follows:
we say that a subscheme $X\subset \PP^r$ is \emph{arithmetically Cohen-Macaulay}
if $H^1(\sI_{X/\PP^r}(m) = 0$ for all $m$ and also $H^i(\sO_X(m)) = 0$ for all $m$ and all $0<i<\dim X$. The first of these
conditions is strongly dependent on the embedding; the second, which is in any case trivial for curves, is not. One can 
restate the conditions by saying that the homogeneous coordinate ring of $X$ is a Cohen-Macaulay ring.


\section{Curves of high degree}\label{high degree ACM}
Let $C$ be a smooth curve of  genus $g$. We have seen that every invertible sheaf of degree $d \geq 2g+1$ is very ample. We will now show that
such a curve,  embedded by the corresponding complete linear series, is arithmetically Cohen-Macaulay.

\begin{theorem}\label{high degree ACM}
If $C\subset \PP^r$ is a smooth projective curve, embedded by a complete linear series $|D|$ of degree $d\geq 2p_a(C)+1$, then
the restriction maps 
$$
H^0(\sO_{\PP^{r}}(m)) \to H^0(\sO_C(m))
$$
 are surjective for all $m$. Equivalently, $H^1(\sI_C(m)) = 0$ for all $m$.
\end{theorem}

\begin{proof} We do induction on $m$, noting that
For $m\leq 1$ the conclusion is immediate. 

The case $m=2$ requires a special argument. Note that the complete linear series
embeds $C$ in $\PP^{r}$.  Let $H$ be a general hyperlane in $\PP^{r}$.
 By Theorem~\ref{linear general position} the $d$ points $\Gamma = C\cap H$ are in linearly general position. By Proposition~\ref{arbitrary hyperplane} they span $H \cong \PP^{r}$.
 
By the Riemann-Roch theorem $r = d-p_a$. The inequality $d\geq 2p_a+1$ is equivalent to the statement that $d\leq 2(r-1)+1$ so by Lemma~\ref{min hilb} the points of $\Gamma$ impose $d$ conditions on quadrics.
This implies that
the restriction map $H^0(\sO_{\PP^{r}}(2))\to H^0(\sO_H(2)) \to H^0(\sO_{\Gamma/H}(2))$ is surjective, whence $H^1(\sI_\Gamma(2)) = 0.$

Applying $H^1$ to exact sequence 
$$
0\to \sI_C(1) \to \sI_C(2) \to \sI_{\Gamma/H}(2) \to 0
$$
of Proposition~\ref{arbitrary hyperplane}
we see that $H^1(\sI_C(2))=0$.

Now suppose that  $m>2$. By induction, it suffices to show that the multiplication map
$$
H^0(\sO_C(1) \otimes H^0(\sO_C(m-1)\to H^0(\sO_{C}(m))
$$
is surjective.

For this we use what David Mumford has called the \emph{base-point-free pencil trick}:
Let $V$ be a pair of sections of $H^0(\sO_{\PP^{d-g}}(1)) = H^0(\sO_C(1))$ that do not intersect anywhere on $C$ (and thus define a base-point-free pencil on $C$).
Since $V\otimes \sO_{C}(m-1)\to \sO_{C}(m)$ it follows that the kernel $\sL$ is an invertible sheaf,
and from the formula
$$
\sO_C(2m-2) \cong \wedge^2(V\otimes \sO_{C}(m-1))\cong \sL \otimes \sO_{C}(m)
$$
it follows that $\sL \cong \sO_C(m-2)$.
Though not necessary for the argument, we note that we have essentially described a Koszul complex
$$
0\to \wedge^2V \otimes \sO_C(m-2) \to V\otimes \sO_{C}(m-1)\to \sO_{C}(m)\to 0.
$$

Since $m\geq 3$ we have $\deg(\sO_C(m-2)) \geq d > 2p_a-2$, so $H^1(\sO_C(m-2)) =0$, and it follows that the map $V\otimes H^0(\sO_{C}(m-1))\to H^0(\sO_{C}(m))$
is surjective. Thus  
$H^0(\sO_C(1))\otimes H^0(\sO_{C}(m-1))\to H^0(\sO_{C}(m))$
is surjective, completing the proof.
\end{proof}


\section{Canonical Curves}\label{canonical ACM}
A slightly more subtle argument will show that
canonical curves are arithmetically Cohen-Macaulay, a result (in the smooth case) of~\cite{Noether-canonical} that we have been using ever since Section~\ref{Noether theorem section}.
The necessary modification is to replace the base-point-free pencil with one having many base points in a controlled way.
We follow the treatment in \cite{Schreyer}. 

If $C$ is a reduced and irreducible curve then by Theorem~\ref{linear general position} a general hyperplane $H\subset \PP^{g-1}$
meets $C$ transversely in a set of points in linearly general position. Any subset $\Gamma$ of $g-2$ of these points will span a $g-3$-plane $\Lambda$
meeting $C$ transversely in $\Gamma$, and we call $\Lambda$ a \emph{simple $(g-2)$-secant $(g-3)$-plane}. 

\begin{theorem}\label{canonical curves are ACM}
If $C\subset \PP^{g-1}$ is a canonically embedded curve of genus $g$, then 
$$
\HH^{0}(\sO_{\PP^{g-1}}(m)) \to \HH^{0}(\sO_{C}(m));
$$
is  surjective for all $m$; equivalently,
$\HH^{1}(\sI_{C/\PP^{g-1}}(m)) = 0$ for all $m$.
\end{theorem}
  
\begin{proof} 
%The Hilbert polynomial $\chi_{C}(t) = h^{0}\sO(t)-\h^{1}\sO(t)$ of $C$ has degree equal to
%$\dim C = 1$, so it is determined by two values.

For $m = 0,1$ the conclusion is immediate. Let $H$ be a general hyperplane in $\PP^{g-1}$, 
and let $\Lambda \subset H$ be a simple $(g-2)$-secant $(g-3)$-plane to $C$. Set $\Gamma = \Lambda \cap C$.

By induction on $m$ it suffices to prove the surjectivity of
$$
\HH^{0}(\sO_{C}(H)) \otimes \HH^{0}(\sO_{C}((m-1)H)) \to \HH^{0}(\sO_{C}(mH))
$$
for $m\geq 2$, where $H$ is the hyperplane divisor on $C$.

Since the points of $\Gamma$
are linearly independent, we may choose homogeneous coordinates $x_{i} \in \HH^{0}(\sO_{C}(1))$ so that
$x_{i}(p_{j}) \neq 0$ if and only if $i = j$. It follows that the sections
$x_{i}^{m}$ of $\sO_{C}(m)$ span $\HH^{0}(\sO_{C}(m)|_{\Gamma})$. 
Thus 
it suffices to prove the surjectivity of
$$
\HH^{0}(\sO_{C}(H-\Gamma)) \otimes \HH^{0}(\sO_{C}((m-1)H)) \to \HH^{0}(\sO_{C}(mH-\Gamma))
$$
for  $m\geq 2$.

Let
$V\subset \HH^{0}(\sO_{C}(H))$ be the two-dimensional subspace of linear forms vanishing on
$\Lambda$, and thus on the $p_{i}$. 
Since $H-\Gamma$ is base-point free away from $\Gamma$, the two sections of $V$
define a surjective map $V\otimes \sO_C \to  \sO_{C}(H-\Gamma)$ and thus a short exact sequence of sheaves
$$
(*) \quad 0\to \sL  \to V\otimes \sO_C((m-1)H) \to  \sO_{C}(mH-\Gamma)\to 0
$$
for some invertible sheaf $\sL$. We must show that the right-hand map is surjective on global sections for $m\geq 2$.

From the exact sequence it follows that
$$
\sL\otimes \sO_{C}(mH-\Gamma) \cong  \sO_C(2(m-1)H) 
$$
so $\sL \cong \sO_C((m-2)H+\Gamma).$ 


To deal with the case $m=2$ we compute the dimensions of the cohomology of the three sheaves in the exact squence (*).
By the Riemann-Roch theorem,
$$
h^0(\sO_C(2H-\Gamma)) = 3g-3 - (g-2) = 2g-1.
$$ 
On the other hand,
$$
h^0(\sO_C(\Gamma)))=\deg \Gamma -g+1+h^0(H-\Gamma) = -1 + h^0(\sO_C(H-\Gamma)) = 1
$$
and 
$$
\dim V \otimes H^0(\sO_C(H)) = 2g.
$$
Thus the left exact sequence
$$
0\to H^0(\sO_{C}(\Gamma)  \to V\otimes H^0(\sO_C(H)) \to H^0( \sO_{C}(2H-\Gamma))) \to 0
$$
is also right exact, as required for the case $m=2$.

For $m\geq 3$ the divisor $(m-1)H+\Gamma$ has degree $>2g-2$, so
$H^1(\sO_{C}((m-1)H+\Gamma)) = 0$, which implies that the sequence $(*)$ is
exact on global sections, as required.
\end{proof}

\begin{corollary}\label{canonical hilbert function}
If $C\subset \PP^{g-1}$ is a canonical curve with a simple $g-2$-secant $g-3$-plane, then the Hilbert function of the homogeneous coordinate ring $S_{C}$ of  $C$ depends only on $g$, and is given by:
$$
\dim({S_{C}})_{d} = h^{0}(\cO_{C}(d)) = 
\begin{cases}
 0 &\mbox {if } d<0\\
 1 & \mbox {if }  d=0\\
 g & \mbox {if }  d=1\\
 (2n-1)g+1 & \mbox {if }  d>1\\
\end{cases}
$$
\end{corollary}
\begin{proof}
By Theorem~\ref{canonical curves are ACM} implies, in particular, that the homogeneous coordinate ring of $C$ can be identified with $\oplus_{n\in \ZZ}\HH^0\sO_C(n)$.  
\end{proof}

\begin{fact}
Though we have stated and proved this theorem for smooth curves,
only a few additional arguments are needed to treat a much more  general case. In \cite{Schreyer} 
the same result is proven for
any purely one-dimensional, nondegenerate closed subscheme $C\subset \PP^{g-1}$ satisfying
$$
 h^{0}(\sO_{C}) = 1,\ h^{0}(\sO_{C}(1) = g, \ \omega_{C} = \sO_{C}(1),
$$ 
and having a simple $(g-2)$-secant $(g-3)$-plane.
\end{fact}
 
 \fix{perhaps add here enough assertions to compute the regularity of a canonical curve and the symmetry of its resolution:
CM imlplies that the dual of the resolution resolves $\omega$; Gorenstein means $\omega$ is invertible. This would be a quick fix
allowing the completion of section 1.6.}

\section{How syzygies can reflect geometry}\label{syzy and geom}

One of the main ways in which syzygies can be seen to reflect the geometry of a scheme $C\subset \PP^r$
depends on the possibility of factoring the line bundle $\sO_C(1)$ as the tensor product of two bundles on $C$
both of which have at least two independent global sections. Suppose for example that $C$ is nondegenerate, 
so that  $\sO_C(1) = \sL_1\otimes \sL_2$. Choose $m\geq 2$ independent global sections
$\sigma_1, \sigma_2$ of  $H^0(\sL_1)$ and $n\geq 2$ independent global sections $\tau_1,\dots, \tau_n$ of $H^0(\sL_2)$. Set
$$
l_{i,j}= \sigma_i\tensor \tau_j \in H^0(\sO_C(1)) = H^0(\sO_{\PP^r}(1))
$$ and consider the matrix 
$$
M = 
\begin{pmatrix}
 l_{1,1}&l_{1,2}&\dots&l_{1,n}\\
 \cdots&\cdots&\cdots&\cdots\\
  l_{m,1}&l_{m,2}&\dots&l_{m,n}
\end{pmatrix},
$$
which we think of as an $m\times n$ matrix of linear forms on $\PP^r$.

We claim that  the $2\times 2$ minors $l_{\ell,j} l_{\ell',j'}-l_{\ell,j'} l_{\ell',j}$ are in the homogeneous ideal of $I_C$ of $C$ in $\PP^r$. 
To see this,
let $\kappa$ be the ring of rational functions on $C$. Choosing identifications $\sL_i\otimes \kappa  \cong \kappa $ we see that the $\sigma_i$ and the $\tau_j$ commute with each other as elements of $\kappa  \otimes_{\sO_X} \kappa $, and thus 
$$
\bigl(l_{1,j} l_{2,j'}-l_{1,j'}l_{2,j}\bigr)|_C = \sigma_1\tau_j\sigma_2\tau_j' - \sigma_1\tau_j'\sigma_2\tau_j =0.
$$

\begin{example}
The most familiar example is that of the twisted cubic. In this case the global sections $x_0\dots x_3$ of $\sO_C(1)$ may be identified with the forms $s^3, s^2t, st^2, t^3 \in k[s,t]$, and if $p\in C \cong \PP^1$ then the multiplication of sections
in the factorization  $\sO_C(1) = \sO_C(p) \otimes \sO_C(2p)$ 
$$
\bordermatrix{
 &s^2&st&t^2\cr
 s& s^3&s^2t&st^2\cr
 t& s^2t&st^2&t^3
}
$$
 leads to the familiar matrix
$$
\begin{pmatrix}
x_0&x_1&x_2\\
x_1&x_2&x_3 
\end{pmatrix}.
$$
We have $I_2(M) = I_C$, and the same idea works for any Veronese or Segre variety.
\end{example}

In case $C$ is reduced and irreducible the matrix above has a special property: $\kappa $ is a domain, so no product of a nonzero
section of $\sL_1$ with a nonzero section of $\sL_2$ can be zero. We can state this without any reference to $C$:

\begin{definition}
Let $R$ be a commutative ring. A map $M:R^n\to R^m$ is 1-generic if the kernel of the corresponding
 map $R^{n}\otimes R^{m*} \to R$  contains no pure tensor $a\otimes b$. In more concrete terms, a matrix
$M$ is \emph{1-generic} if there are no invertible matrices $A,B$ such that  $AMB$ has some entry equal to 0.
\end{definition}

The case $m=2$, leading to a $2\times m$ matrix is the most interesting, because, as shown in Chapter~\ref{scrolls}, the ideal $I_2(M)$ of a 1-generic matrix of linear forms is the homogeneous ideal of a rational normal 
scroll a variety of of codimension $n-1$ and degree $n$. 

Moreover, the minimal free resolution of $I_2(M)$ has a special form called the 
Eagon-Northcott complex (Theorem~\ref{Eagon-Northcott}), which will appear as a subcomplex of the minimal free resolution of $I_C$. The presence of such a variety containing $C$ or
a subcomplex of this special form in the minimal free resolution of $C$ is thus necessary for the 
factorization of the line bundle $\sO_C(1)$ as above, and it is sometimes sufficient, as well.

\section{The Eagon-Northcott Complex of a $2\times n$ matrix of linear forms}

The Eagon-Northcott complex is a complex of free modules associated to any matrix over any commutative ring. The most familiar special case is the Koszul complex, which one may think of as the Eagon-Northcott complex of a $1\times n$ matrix, and  even in the general case the Eagon-Northcott complex is in a sense built out of the Koszul complexes. A full treatment of this and a whole family of related complexes can be found in 
\cite[Appendix A2]{Eisenbud1995}, and, from a more conceptual and general point of view, in \cite{Weyman-book}. Here we will only
make use of the case of a matrix such as the one above, we will present a simplified account in that case only. Here is the result we need:

\begin{theorem}\label{Eagon-Northcott}
 Let $S = k[x_0,\dots, x_r]$ be a polynomial ring,  and let $M: F\to G$ be a homomorphism with
 $F = S^n(-1), G= S^2$. If $M$ is 1-generic, then the minimal free resolution of $S/I_2(M)$ has the form:
\begin{align*}
EN(M) := 
S \lTo{\bigwedge^2 M} 
 \bigwedge^2 F&
 \lTo^{\delta_{2}}
 S^{2*}\otimes \bigwedge^3 F  \lTo^{\delta_{3}}
  (\Sym^2S^{2})^*\otimes\bigwedge^4F  \\
 &\lTo^{\delta_{4}}\cdots\lTo^{\delta_{n-1}} 
(\Sym^{n-2}S^{2})^*\otimes\bigwedge^nF 
 \lTo 0.
\end{align*}
\end{theorem}

\begin{proof} We first show that $r\geq n$; more precisely, we show that the span of the entries of $M$ has dimension $\geq n+1$. As noted above, to say that the $2\times n$ matrix of linear forms $M$ is 1-generic means that
the kernel of the corresponding map $ \phi: k^2\otimes k^n \to S_1$ contains no pure tensors. In the projective
space $\PP^{2n-1} = \PP(k^2\otimes k^n)$ the pure tensors form a variety isomorphic to $\PP^1\times \PP^{n-1}$, and thus of dimension $n$. Consequently the kernel of $\phi$ can have dimension at most $n-1$, whence the image of $\phi$ 
in $S_1 = k^{r+1}$ has dimension at least $2n-(n-1) = n+1$. 

We begin the discussion of $EN(M)$ by defining the maps $\delta_i$ and and proving that the given sequence is indeed a complex---that is, consecutive maps compose to 0. For simplicity of notation, we choose a generator of $\wedge^2 S^2$
 and identify it with $S$, which gives a sense to the map labeled $\bigwedge^2M$.
 
  Although it is not hard to do this directly, the dual maps
 $$
 \partial_i: \Sym^{i-2} G \otimes \bigwedge^i F^* \rTo \Sym^{i-1} G \otimes \bigwedge^{i+1} F^*
 $$
 have a more familar-looking description, so we define these instead. Indeed, the map $M$ corresponds to an
 element $\mu\in G\otimes F^*$. We may think of $ \Sym^{i-2} G \otimes \bigwedge^i  F^*$
 as a (bigraded) component of the exterior algebra over $ \Sym G$ of 
 $$
  \Sym G \otimes \bigwedge_S  F^*= \bigwedge_{ \Sym G} (\Sym G \otimes  F^*).
 $$
We define $\partial_i$ to be  multiplication by $\mu$ in the sense of this exterior algebra. Since $\mu$ has degree 1
in this sense, its square is 0. 

To show that $(\bigwedge^2 M)\circ \delta_2$ is zero, it is simplest to choose a matrix representing $M$.
Direct computation using only the usual expansion of a determinant
along a row shows that, up to sign,
pure basis vector $e\otimes f_i\wedge f_j\wedge f_k$ of $G^*\otimes \bigwedge^3 F$
maps under the composition $(\bigwedge^2) M\circ \delta_2$ to the determinant
of the $3\times 3$ matrix obtained from $M$ by repeating the row corresponding to $e$ and
the columns $i,j,k$. This determinant is 0 because it has a repeated row.

We next prove the split exactness of a complex of the form $EN(M')$ where $M'$ is surjective, so that we
may write $F = G\oplus F'$ and the map $M': G\oplus F' \to G$ as projection on the first factor. 
Of course
it suffices to prove the split exactness of the dual sequence, $EN(M')^*$:
\begin{align*}
EN(M')^* := 
S \rTo{\bigwedge^2 M'^*} 
 \bigwedge^2 F^*&
 \rTo^{\partial_{2}}
 G\otimes \bigwedge^3 F^*  
 \rTo^{\partial_{3}}
  \Sym^2G\otimes\bigwedge^4F^*  \\
 &\rTo^{\partial_{4}}\cdots\rTo^{\partial_{n-1}} 
\Sym^{n-2}G\otimes\bigwedge^nF^* 
 \rTo 0.
\end{align*}
In this case the proof
is an exercise in multilinear algebra. 
We begin by proving split exactness at the 
positions $\Sym^{i} G \otimes \bigwedge^{i+2}  F^*$ where $i\geq 1$.

The module
$ \Sym^{i} G \otimes \bigwedge^{i+2}  F^*$
decomposes as
\begin{align*}
&\Sym^{i} G \otimes \bigwedge^2 G^* \otimes \bigwedge^{i} F'^*\oplus \\
&\Sym^{i} G \otimes  G^*\otimes \bigwedge^{i+1} F'^* \oplus \\
&\Sym^{i} G \otimes  \bigwedge^{i+2} F'^* 
\end{align*}
Note that under our hypothesis, the element $\mu' \in G\otimes F^* = G\otimes G^* \oplus G\otimes F'^*$
has the form $(\mu_G, 0)$, where $\mu_G$ represents the identity map $G \to G$. Thus the complex
$EN(M')^*$ is a direct sum over $i$ of 3-term complexes of the form
%$$
%\Sym^{i+1} G \otimes \bigwedge^2 G^* 
%\lTo^{-\wedge \mu'} 
%\Sym^{i} G \otimes  G^*
%\lTo^{-\wedge \mu'} 
%\Sym^{i-1} G
%$$
$$
\Sym^{i-1} G 
\rTo^{-\wedge \mu'} 
\Sym^{i} G \otimes  G^*
\rTo^{-\wedge \mu'} 
\Sym^{i+1} G \otimes \bigwedge^2 G^* 
$$
tensored with various $\bigwedge^j F'^*$, and it suffices to show that the former are split exact when
$i\geq 0$. Now $\Sym G$ may be identified with $R:= S[x,y]$, where $x,y$ are a basis of $S^2$, and
as such the sequences above may be identified with components of the Koszul complex of $x,y$ over $R$,
%$$
%R\otimes \wedge^2 G\lTo R\otimes G \lTo R .
%$$
$$
0\to R \rTo R\otimes G \rTo R\otimes \wedge^2 G
$$
The only homology of this sequence is $R/(x,y)R$ at the right so if we replace  $R\otimes \wedge^2 G \cong R$ by the ideal $(x,y)R$, this sequence is a split exact sequence of free $S$-modules. This is the desired result.

It remains to treat the beginning of the complex $EN(M')^*$,
$$
S \rTo{\bigwedge^2 M'^*} 
 \bigwedge^2 F^*
 \rTo^{-\wedge \mu'}
G\otimes \bigwedge^3 F^*
$$
which, in our case, may be written:
%$$
%\wedge^2 F^* = \bigwedge^2 G^* \oplus (G^*\otimes F'^*) \oplus \bigwedge^2 F'*
%$$
%Consider the pair of maps
\begin{align*}
S \rTo{\bigwedge^2 M'^*} 
 &\bigwedge^2 G^* \ \oplus\ (G^*\otimes F'^*)\ \oplus\ \bigwedge^2 F'^*
 \rTo^{-\wedge \mu'} \\
 &G\otimes \bigwedge^2 G^*\otimes F'^*\ \oplus\ (G\otimes G^*\otimes \bigwedge^2 F'^*)\ \oplus\ G\otimes \bigwedge^3 F'^*
\end{align*}
The map $\bigwedge^2 M'$ is the projection to $\bigwedge^2 G$ composed with the chosen isomorphism
$\bigwedge^2 G \cong S$, and is thus a split monomorphism. To complete the argument, we must show that
 the map marked $-\wedge \mu'$ is a monomorphism on $(G^*\otimes F'^*) \oplus \bigwedge^2 F'^*$.
 But this map is the direct sum of the two maps
  $$
 (G^* \rTo^{-\wedge \mu'} G\otimes \bigwedge^2 G^*)  \otimes F'^*
 $$
 and
 $$
(S  \rTo^{\mu'} G\otimes G^*) \otimes \bigwedge^2 F'^*
 $$
 which are evidently split monomorphisms. 
completing the proof of split exactness of $EN(M')^*$ and thus of $EN(M')$.


To go further we use a basic result, proven in a more general form (and with a slightly different statement) in \cite[Theorem 20.9]{Eisenbud1995}. We make the convention
that the codimension of the empty set is infinity.

\begin{theorem}\label{WMACE}
 Let $S = k[x_0,\dots, x_r]$, and
 $$ 
\FF:  F_0\lTo^{\phi_1}F_1 \lTo \cdots \lTo F_{n-1}\lTo^{\phi_n} F_n\lTo 0
 $$
be a finite complex of free $S$-modules. Set
$$
X_i = \{p\in \AA^{n+1} \mid  H_i(\FF \otimes \kappa(p)) \neq 0\}
$$
The complex $\FF$ is \emph{acyclic} (that is, $H_i(\FF) = 0$ for all $i>0$) if and only if
$$
\codim X_i \geq i
$$
for all $i>0$. Moreover, $X_{0}\supseteq X_{1}\supseteq \cdots \supseteq X_{n}$
\qed
\end{theorem}

For example, Nakayama's Lemma implies that $X_{0}$ is the support of $\coker \phi_{1}$; thus $X_{0}$ is the set defined by the $\rank F_{0}$-sized minors of $\phi_{1}$. Similarly, 
and that $X_{n}$ is the support of the cokernel of the dual of $\phi_{n}$. 

Also, if $n=1$, the theorem simply says that a map $F_1\to F_0$ is a monomorphism iff it becomes a monomorphism after tensoring with the field of rational functions $K$, which follows from the flatness of
localization and the fact that $F_1$ is torsion-free, so that
$F_1 \subset F_1 \otimes K$. 

\begin{fact}
Theorem~\ref{WMACE} is true in this form over any Cohen-Macaulay ring; for more general
rings, ``codimension'' must be replaced by ``grade'', as in the given reference.
The Theorem can be generalized
to case where the $F_i$ are not free, but are sufficiently ``like'' free modules, too.
\end{fact}

Conclusion of the proof of Theorem~\ref{Eagon-Northcott}.
Let $X_{i}\subset \AA^{r+1}$ be the variety defined from the complex $EN(M)$ as in 
Theorem~\ref{WMACE}. Since $EN(M)$ becomes split exact after inverting any $2\times 2$ minor of $M$
$X_{i}$ is
contained in the closed set defined by $I_{2}(M)$, for all $i$. Thus if $I_{2}(M)$ has codimension $n-1$,
then $EN(M)$ is acyclic. 
\end {proof}

%\fix{START COMMENTED OUT MATERIAL}
%We will also use a special case of the Auslander-Buchsbaum formula connecting projective dimension and depth:
%
%\begin{theorem}\label{Auslander-Buchsbaum}
%If $R$ is a regular local ring of dimension $d$, and $M$ is a finitely generated $R$-module, then the projective dimension of $M$ is $\leq d$ with equality only if
%$M$ contains a submodule of finite length. 
%\end{theorem}
%
%\begin{corollary}\label{associated primes}
%If $R$ is a regular local ring of dimension $d$, and $M$ is a finitely generated $R$-module, then the codimension of an associated prime of $A$ is at most the projective dimension of $A$. 
%\end{corollary}
%\begin{proof}[Proof of  Corollary~\ref{associated primes}]
% Projective dimension can only decrease under localization, and
% the associated primes $P$ of $A$ are those for which $A_{P}$ contains a submodule
% of finite length.
%\end{proof}
%
%With this and the multi-linear algebra above we can  prove the basic acyclicity result for an Eagon-Northcott complex:
%
%\fix{the Theorem as now stated doesn't need the following. I've copied the short proof
%of acyclicity into the end of the proof above.}
%\begin{proposition}\label{acyclicity}
%Let $S = k[x_0,\dots, x_r]$ be a polynomial ring,  and let $M: F\to G$ be a homomorphism with
% $F = S^n(-1), G= S^2$.
% $$
% S^n \cong F \rTo^M G \cong S^2
% $$
% is a (not necessarily homogeneous) map of free $S$-modules.
% The Eagon-Northcott complex $EN(M)$ is acyclic if and only if $\codim I_2(M) \geq n-1$, in which case the dual complex is also acyclic and
% the associated primes of $I_2(M)$ are all minimal and of codimension $n-1$.
% \end{proposition}
%
%\begin{proof}[Proof of Proposition~\ref{acyclicity}]
%Let $X_{i}\subset \AA^{r+1}$ be the variety defined from the complex $EN(M)$ as in 
%Theorem~\ref{WMACE}. Since $EN(M)$ becomes split exact after inverting any $2\times 2$ minor of $M$
%$X_{i}$ is
%contained in the closed set defined by $I_{2}(M)$, for all $i$. Thus if $I_{2}(M)$ has codimension $n-1$,
%then $EN(M)$ is acyclic. 
%
%In this case the projective dimension
%of $S/I_{2}(M)$ is $n-1$, so all the associated primes of $I_{2}(M)$ have codimension
%exactly $n-1$.
%
%If $EN(M)$ is  acylic then, by Theorem~\ref{WMACE}, the codimension of $X_{n-1}$ is at least $n-1$. Thus to prove that the acyclicity of $EN(M)$ implies  $\codim I_{2}(M) \geq n-1$ (and thus
%$\codim I_{2}(M))  n-1$, it suffices to show that $X_{n-1} = X_{0}$ as algebraic sets.
%
%To see this, note that
%the ideal of $2\times 2$ minors of $M$. By definition, $X_{n-1}$ 
%is the set of points $p$ where $\kappa(p)\otimes \delta_{n-1}$ is not an inclusion, 
%or equivalently, that 
%that 
%$$
%\kappa(p)\otimes F\otimes \Sym^{n-3}G 
%\cong 
%\kappa(p)\otimes \bigwedge^{n-1} F^{*} \otimes  \Sym^{n-3}G 
%\rTo^{\partial_{n-1}}
%\kappa(p)\otimes \bigwedge^{n} F^{*} \otimes  \Sym^{n-2}G
%\cong
%\kappa(p)\otimes \Sym^{n-2}G
%$$
%is not a split surjection, and it is easy to see that the composite map takes
%$a\otimes b$ to $\kappa(p) \otimes M(a)\cdot b$, so the cokernel is the $(n-2)$-nd symmetric power
%of the cokernel of the map $\kappa(p) \otimes M$. Thus $X_{n}$ is equal to the support
%of the cokernel of $M$ itself. 
%
%By Nakayama's Lemma, $X_{0}$ is the support of $M$; furthermore, the localization of $\coker M$ at $p$ is 0 if and only if
%one of the 
%$2\times 2$ minors of $M$ is a unit locally at $p$ so $X_{0}$, so this is defined
%set-theoretically by $I_{2}(M)$.
%
%It now follows from Theorem~\ref{WMACE} that all the $X_{i}$ are equal, so $EN(M)$ is 
%acyclic if and only if $EN(M)^{*}$ is acyclic. 
%
%Since $M$ is 1-generic the entries of of the second row of $M$ are linearly independent, and since the dimension of the span of all the linear forms is at least $n+1$, some element in the first row is outside the span of the the elements in the second. After a permutation of columns we may assume that $l_{1,1}, l_{2,1}, l_{2,2},\dots l_{2,n}$ are linearly independent, and we may take them to be a subset of the variables, say $x_{0},\dots x_{n+1}$
%
%We next show by induction on $n$ that  $I_{2}(M)$ is prime. In the case $n=2$ we have $I_{2}(M) =  (x_{0}x_{2}-x_{1}l_{1,2})$ which obviously does not factor. 
%
%Now suppose that $n>2$, and let $M'$ be the matrix $M$ with the first column omitted. we know by induction that $I_{2}(M')$ is prime of codimension $n-2$. Since $I:=I_{2}(M)$ does not have the maximal ideal as an associated prime, it is saturated. The ideal $I_{2}(M)+x_{0}$ properly contains $I_{2}(M')$ and thus has codimension $\geq n-1$ in $S/x_{0}$, whence we see that every component of $I_{2}(M)$ meets the open set
%$x_{0} = 1$. Restricting to this open set \fix{complete the proof}
%\end{proof}
%
%The first non-trivial example of a finite free resolution is the Koszul complex on 3 variables, which is the minimal $S = k[x,y,z]$-free resolution of the module $S/(x,y,z)$:
%$$
%0\to S(-3) \rTo^{
%\begin{pmatrix}
%x\\y\\z 
%\end{pmatrix}}
% S^3(-2) \rTo^{\begin{pmatrix}
%0&-z&y\\
%z&0&-x\\
%-y&x &0
%\end{pmatrix}}
%S^3(-1) \rTo^{
%\begin{pmatrix}
%x&y&z
%\end{pmatrix}}
%S
%$$
%In fact this is the first example that 
%Hilbert
% presented in his famous paper \cite{Hilbert1890}. 																											
%\fix{END COMMENTED OUT MATERIAL}

\section{Betti tables of canonical curves}
\fix{We need to add the regularity of the canonical curve = 3; that plus Gorenstein, self-dual resolution,
gives the shape of the Betti table in which you could look for invariants. Gorenstein comes from the $H^{1}I$
computation. In the homol alg appendix we should explain the resolution duality, apply it here.}


\section{Syzygies and the Clifford index}


Corollary~\ref{canonical hilbert function} implies that the dimension of the vector space of forms of degree $d$
vanishing on a canonical curve is independent of the curve; for example, for $d=2$ we get
$
\dim ({I_{C}})_{2} = {g-2\choose 2}
$
The next question one might ask is whether or not these quadric generate the ideal $I_{C}$, and (much) more generally, what is the 
Betti table of the homogeneous coordinate ring of $C$.

 For example,
when $C$ is trigonal, with a $g^{1}_{3}$ defined by a line bundle $\sL$, the complementary linear series,
defined by $\omega_{C}\otimes \sL^{-1}$ has $g-2$ sections, and we see from Theorem ****
that $C$ lies on the ${g-2\choose 2}$ quadrics defined by the minors of a $2\times g-2$, 1-generic matrix of linear forms. The exactness of the Eagon-Northcott complex associated to this matrix shows that there are no relations of degree 0 on these minors -- that is, they are linearly independent over the ground field. It follows that they generate the vector space of all quadrics containing $C$. But by **** the locus defined by
the ideal of minors is a rational normal scroll of dimension 2, and thus the minors cannot generate $I_{C}$.

Furthermore, if $g = 6$ and $C$ is isomorphic to a plane quintic curve, then the canonical series of the plane quintic is $5-3 = 2$ times the hyperplane series, and it follows that the canonical image of $C$ lies on the Veronese surface in $\PP^{5}$. Using Theorem *** again, we see that the Veronese is contained in (in fact, equal to) the intersection of the quadrics defined by the $2\times 2$ minors of a generic symmetric matrix, coming from the 
multiplication map 
$$
\HH^{0}(\sO_{\PP^{1}}(1))\otimes \HH^{0}(\sO_{\PP^{1}}(1)) \to \HH^{0}(\sO_{\PP^{1}}(2)) = \HH^{0}(\sO_{\PP^{5}}(1))
$$
and there are $6 = {g-2\choose 2}$ independent quadrics in this ideal. Again in this case, they cannot generate the ideal of the curve.

One might fear that this is the beginning of some long series of examples, but in fact it is not: 

\begin{theorem} [Petri]
The ideal of a canonical curve of genus $\geq 5$ is generated by the $\g-2\choose 2$-dimensional space of quadrics it contains unless the curve is either trigonal or isomorphic to a plane quintic; in the latter cases, the ideal of the curve is generated by quadrics and cubics.
\end{theorem}

For a modern treatment of Petri's Theorem in this level of generality see \cite{Schreyer}; for a different treatment see \cite{Arbarello-Sernesi}.

The two exceptions can be described simultaneously by using the Clifford index:

\begin{definition}
 The Clifford index Cliff $\sL$ of a line bundle $\sL$ on a curve $C$ is $d-2r$, where $d := \deg \sL$ and $r :=  h^0(\sL)-1$. The Clifford index Cliff $C$ of
 a curve $C$ of genus $\geq 2$ is the minimum of the Clifford indices of special line bundles with at least 2 sections.
\end{definition}

Cliffords Theorem \ref{****} says that Cliff $C \geq 0$, and that Cliff $C = 0$ if and only if $C$ is hyperelliptic. If $C$ is not hyperelliptic, then it turns out that Cliff $C=1$ if and only if $C$ is either trigonal or isomorphic to a plane quintic. The Clifford index of any smooth curve of genus $g\geq 2$ is $\leq \lceil g/2\rceil+1$, with equality for a general curve, as one sees from the Brill-Noether Theorem~\ref{}, and for ``most'' curves the line bundle $\sL$ of maximal Clifford index has only 2 sections, though there is an infinite sequence of examples where this
``Clifford dimension'' is greater.

Moving to cubic forms, we see that $\dim ({I_C})_3 = {g+2\choose 3}-(5g-5)$. Comparing this number with the number of (possibly linearly dependent)
cubics obtained by multiplying $g$ linear forms and ${g-2\choose 2}$ quadrics, we see that the ideal of the curve has at least
$$
{g-2\choose 2} - {g+2\choose 3}-(5g-5) 
$$
independent syzygies of total degree 3 (that is, linear syzygies on the quadrics. For example when $g=4$ so that $C\subset \PP^3$ there is one quadric and 5 independent
cubics, at most 4 of which are multiples of the quadric. Since the curve has degree $6 = 2\times 3$, the ideal of the curve must be generated by
the quadric and one cubic. When $g=5$ there are genuinely two possibilities: the three quadrics in the ideal might be a complete intersection
(then they generate the ideal), so the Betti table would be

\centerline{\small %\scriptsize
\begin{tabular}{r|ccc} 
$j\backslash i$&0&1&2\\ 
\hline 
0&1&$-$&$-$\\ 
1&$-$&2&$-$\\ 
2&$-$&$-$&$1$\\ 
\end{tabular}}

\noindent or the curve could be trigonal, in which case the 3 quadrics generate the ideal of a surface scroll $F$. In the latter
case, the Eagon-Northcott complex resolves the homogeneous coordinate  ring  $S_F$ of the scroll,
$$
0\to S^2(-3) \to S^3(-2) \to S \to S_F \to 0
$$
which has Betti table

\centerline{\small %\scriptsize
\begin{tabular}{r|ccc} 
$j\backslash i$&0&1&2\\ 
\hline 
0&1&$-$&$-$\\ 
1&$-$&3&$2$\\ 
\end{tabular}}
\noindent and we see that there are 2 linear relations among the quadrics. Thus the minimal generators of $I_C$ must include exactly 2 cubics as well as the 3 quadrics. Since the homogeneous ring of a canonical curve is Gorenstein, its minimal free resolution is symmetric, and this is enough for us to fill in its Betti table:

\centerline{\small %\scriptsize
\begin{tabular}{r|cccc} 
$j\backslash i$&0&1&2&3\\ 
\hline 
0&1&$-$&$-$&$-$\\ 
1&$-$&3&$2$&$-$\\ 
2&$-$&$2$&$3$&$-$\\ 
3&$-$&$-$&$-$&$1$\\ 
\end{tabular}}
\noindent Note that we can ``see'' the scroll reflected in the top two lines of the table.

From the analogue of the Hilbert-Burch Theorem for Gorenstein rings of codimension 3 one can show that the 5 generators can be written as the 
pfaffians of a skew symmetric $5\times 5$ matrix whose entries are of degrees 1 and 2, in the following pattern (we give just the degrees, and put - in the places that are 0):
$$
\begin{pmatrix}
 -&-&1&1&1\\
-&-&1&1&1\\
1&1&-&2&2\\
1&1&2&-&2\\
1&1&2&2&-
\end{pmatrix}
$$
Here the$2\times 2$ minors of the upper $2\times 3$ block of linear forms generate the ideal of the scroll. 

Applying this logic more generally we get the following result about the canonical embedding of curves with low degree maps to $\PP^1$:

\begin{theorem}
 Let $C\subset \PP^{g-1}$ be a reduced, irreducible canonical curve. If $C$ has a line bundle $\sL$ of degree $d$ with $h^0(\sL) = 2$  then
 there is a $2\times g+1-d$ 1-generic matrix of linear forms whose minors define a scroll of codimension $g-d$ containing $C$; and thus an
 Eagon-Northcott complex of length $g-d$ is a quotient complex of the minimal free resolution of $S_C$. In particular, the Betti table of $S_C$ is
 termwise $\geq$ that of the homogeneous coordinate ring of the scroll.
 \end{theorem}
 
 Thus the existence of the $g^1_d$ on $C$, together with the symmetry of the resolution of the Gorenstein ring $S_C$,
  implies that the Betti table of $S_C$ has the form
 
 \centerline{\small %\scriptsize
\begin{tabular}{r|cccccccccccc} 
$j\backslash i$&0&1&2&\dots&d-3&d-2&\dots&g-d-1&g-d&\dots&g-3&g-2\\ 
\hline 
0&1&$-$&$-$&$\cdots$&$-$&$-$&$-$&$-$&$-$&$-$&$-$&$-$\\ 
1&$-$&*&*&$\cdots$&*&*&$\cdots$&*&$?$&$\cdots$&?&?\\ 
2&$-$&?&?&$\cdots$&?&*&$\cdots$&*&*&$\cdots$&*&*\\ 
3&$-$&$-$&$-$&$\cdots$&$-$&$-$&$-$&$-$&$-$&$-$&$-$&1
\end{tabular}}
\noindent where we have assumed for illustration that $d-2<g-d-1$. The places marked $-$ are definitely 0 and those marked * are definitely nonzero. The rows marked 0 and 1 contain the Betti table of the scroll.

We can summarize this by saying
that if the curve $C$ has a line bundle $\sL$ of degree $d$ with exactly 2 sections, and thus of Clifford index $c = d-2$ the row labeled  2 
in the Betti diagram definitely has nonzero entries starting in the $c$-th place. As with the case of the plane quintics, above, one can
make a similar argument for \emph{any} line bundle of Clifford index $c$

Starting from such examples, Mark Green made a bold conjecture that was still open at the time this book was written:

\begin{conjecture}[Green's Conjecture]
If $C$ is a smooth canonical curve of genus g and Clifford index d-2, then the entries marked with ? in the Betti table above are all 0. 
\end{conjecture}

The conjecture was made for curves over a field of characteristic 0, and is known in many cases, though it is also known to fail in small finite characteristics.
For example, it is true for generic curves of each Clifford index, and is true for \emph{every} curve of  Clifford index $c = \lceil g/2\rceil+1$, the maximal value.
It is also true for plane curves, and in a number of other special cases. See **** for a survey.

\subsection{Low genus canonical embeddings} Schreyer's table of syzygies of canonical curves. (Note that M2 notation is
different; and the hyperelliptic cases are included. Maybe retype in normal notation?)

\includepdf{"SyzygiesGupto8"}
%\section{Low degree}

%footer for separate chapter files

\ifx\whole\undefined
%\makeatletter\def\@biblabel#1{#1]}\makeatother
\makeatletter \def\@biblabel#1{\ignorespaces} \makeatother
\bibliographystyle{msribib}
\bibliography{slag}

%%%% EXPLANATIONS:

% f and n
% some authors have all works collected at the end

\begingroup
%\catcode`\^\active
%if ^ is followed by 
% 1:  print f, gobble the following ^ and the next character
% 0:  print n, gobble the following ^
% any other letter: normal subscript
%\makeatletter
%\def^#1{\ifx1#1f\expandafter\@gobbletwo\else
%        \ifx0#1n\expandafter\expandafter\expandafter\@gobble
%        \else\sp{#1}\fi\fi}
%\makeatother
\let\moreadhoc\relax
\def\indexintro{%An author's cited works appear at the end of the
%author's entry; for conventions
%see the List of Citations on page~\pageref{loc}.  
%\smallbreak\noindent
%The letter `f' after a page number indicates a figure, `n' a footnote.
}
\printindex[gen]
\endgroup % end of \catcode
%requires makeindex
\end{document}
\else
\fi
