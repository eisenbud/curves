%header and footer for separate chapter files

\ifx\whole\undefined
\documentclass[12pt, leqno]{book}
\usepackage{graphicx}
\input style-for-curves.sty
\usepackage{hyperref}
\usepackage{showkeys} %This shows the labels.
%\usepackage{SLAG,msribib,local}
%\usepackage{amsmath,amscd,amsthm,amssymb,amsxtra,latexsym,epsfig,epic,graphics}
%\usepackage[matrix,arrow,curve]{xy}
%\usepackage{graphicx}
%\usepackage{diagrams}
%
%%\usepackage{amsrefs}
%%%%%%%%%%%%%%%%%%%%%%%%%%%%%%%%%%%%%%%%%%
%%\textwidth16cm
%%\textheight20cm
%%\topmargin-2cm
%\oddsidemargin.8cm
%\evensidemargin1cm
%
%%%%%%Definitions
%\input preamble.tex
%\input style-for-curves.sty
%\def\TU{{\bf U}}
%\def\AA{{\mathbb A}}
%\def\BB{{\mathbb B}}
%\def\CC{{\mathbb C}}
%\def\QQ{{\mathbb Q}}
%\def\RR{{\mathbb R}}
%\def\facet{{\bf facet}}
%\def\image{{\rm image}}
%\def\cE{{\cal E}}
%\def\cF{{\cal F}}
%\def\cG{{\cal G}}
%\def\cH{{\cal H}}
%\def\cHom{{{\cal H}om}}
%\def\h{{\rm h}}
% \def\bs{{Boij-S\"oderberg{} }}
%
%\makeatletter
%\def\Ddots{\mathinner{\mkern1mu\raise\p@
%\vbox{\kern7\p@\hbox{.}}\mkern2mu
%\raise4\p@\hbox{.}\mkern2mu\raise7\p@\hbox{.}\mkern1mu}}
%\makeatother

%%
%\pagestyle{myheadings}

%\input style-for-curves.tex
%\documentclass{cambridge7A}
%\usepackage{hatcher_revised} 
%\usepackage{3264}
   
\errorcontextlines=1000
%\usepackage{makeidx}
\let\see\relax
\usepackage{makeidx}
\makeindex
% \index{word} in the doc; \index{variety!algebraic} gives variety, algebraic
% PUT a % after each \index{***}

\overfullrule=5pt
\catcode`\@\active
\def@{\mskip1.5mu} %produce a small space in math with an @

\title{Personalities of Curves}
\author{\copyright David Eisenbud and Joe Harris}
%%\includeonly{%
%0-intro,01-ChowRingDogma,02-FirstExamples,03-Grassmannians,04-GeneralGrassmannians
%,05-VectorBundlesAndChernClasses,06-LinesOnHypersurfaces,07-SingularElementsOfLinearSeries,
%08-ParameterSpaces,
%bib
%}

\date{\today}
%%\date{}
%\title{Curves}
%%{\normalsize ***Preliminary Version***}} 
%\author{David Eisenbud and Joe Harris }
%
%\begin{document}

\begin{document}
\maketitle

\pagenumbering{roman}
\setcounter{page}{5}
%\begin{5}
%\end{5}
\pagenumbering{arabic}
\tableofcontents
\fi


\chapter{Linkage}
\label{LiaisonChapter}\label{linkageChapter}


\section{Introduction} 
\fix{needs an introduction}
%
%In studying the embeddings of a curve $C\subset \PP^r$  we have again and again estimated the number of
%forms of degree $d$ in the ideal of the curve using the sequence
%$$
%0\to H^0(\sI_C(d)) \to H^0(\sO_{\PP^r}(d))\to  H^0(\sO_C(d)) \to H^1(\sI_C(d))\to 0
%$$
%which is exact because (since $r$ must be $\geq 2$ in any interesting case) $H^1(\sO_{\PP^r}(d))= 0$. In words: 
%$H^1(\sI_C(d))$ measures the failure of hypersurfaces of degree $d$ to cut out a complete linear series on $C$. Curves for which
%$H^1(\sI_C(d))=0$ for all $d$ are said to be \emph{arithmetically Cohen-Macaulay}; Theorem~\ref{canonical curves are ACM} shows that
%all canonical curves have this property, and Theorem~\ref{high degree ACM} shows that any curve of genus $g$ embedded by a
% complete linear series of degree $\geq 2g+1$ does too.
%  
%In the interesting case of curves in $\PP^3$ the spaces $H^1(\sI_C(d))$ play a particularly important role that is
%the subject of the theory of linkage (often called by its French name, liaison). First we look into the question of when the $H^1(\sI_C(d))$
%all vanish.
%
%\section{The Cohen-Macaulay property}
%A sequence of elements $f_1,\dots, f_t\in \gm$ is a \emph{regular sequence} on $M$ if $f_i$ is a nonzerodivisor on $M/(f_1, \dots, f_{i-1})M$
%for $i = 1,\dots,t$. 
%All maximal regular sequences on $M$ have the same length, called the \emph{depth}  $\depth M$, and 
%$\depth M + \pd M = \dim R$. The depth of $M$ is always $\leq \dim M$, and in the case of equality $M$ is said to be a Cohen-Macaulay module. Every
%localization of a Cohen-Macaulay module is also Cohen-Macaulay. If $M$ is and $R$-algebra, then the depth of $M$ is independent
%of $R$, and if $M$ is Cohen-Macaulay we say simply that it is a Cohen-Macaulay ring. For example, any 0-dimensional module is Cohen-Macaulay,
%and a 1-dimensional module is Cohen-Macaulay if and only if $\gm$ is not an associated prime of $0\subset M$; that is, if and only if $M$ is purely 1-dimensional. In particular, any purely 1-dimensional scheme is Cohen-Macaulay.
%
%Analogous results hold over a positively graded polynomial ring such us the homogeneous coordinate ring $S = \CC[x_0,\dots,x_r]$ of $\PP^r$ when
%only graded modules are considered and all elements are taken to be homogeneous. Most interesting for the purposes of this book are the
%homogeneous coordinate rings $S_X$ and local rings $\sO_{X,p}$ of projective schemes $X$. If all the $\sO_X$ are Cohen-Macaulay rings then we say that
%$X$ is a Cohen-Macaulay scheme. If $S_X$ is Cohen-Macaulay then we say that $X$ is \emph{arithmetically Cohen-Macaulay}. Since the $\sO_{X,p}$ can be obtained as localizations of $S_X$ the scheme, this is a stronger condition than $X$ being Cohen-Macaulay.
%
%\begin{proposition}
%If $X\subset \PP^r$ is a projective scheme then $X$ is Cohen-Macaulay if and only if 
%$X$ is pure-dimensional and
%$$
%H^i(\sO_X(m))= 0
%$$
%for all $0<i<dim X$ and all $m\in \ZZ$. The scheme
% $X$ is arithmetically Cohen-Macaulay if and only if $X$ is Cohen-Macaulay and either $X$ is 0-dimensional or
%$H^1(\sI_X(m))$ = 0 for all $m$\footnote{The arithmetically Cohen-Macaulay case of this Proposition may seem more natural when stated in terms of local cohomology:
%the module $H^1_*(\sI_X(m) := \oplus_{i\in Z} H^1(\sI_X(m))$ is actually the second local cohomology with supports in the maximal homogeneous ideal,
%and the 0-th and 1-rst local cohomologies vanish automatically. Also, $H^i_*(\sO_X)$ is the $i+1$ local cohomology, so the condition
%becomes the vanishing of all local cohomology up to the dimension of $S_X$. See, for example~\cite[Appendix 1]{MR2103875}.}.
%\end{proposition}
%
%
%
%For example, a purely 1-dimensional scheme $C$  is arithmetically Cohen-Macaulay if and only if $H^1(\sI_C(m) = 0$
%for all m.
%By Serre's Vanishing theorem $H^1(\sI_X(m)) = 0$ for $m\gg 0$ in any case; and if $X$ has
%no 0-dimensional component then, since $H^1(\sI_X(m))$ is the cokernel of the natural map $H^0(\sO_{\PP^r}(m) \to H^0(\sO_{X}(m)$,
%it vanishes for $m<0$. Thus, assuming that  $X$
%no 0-dimensional component, the sum $D(X) := \oplus_{m\in \ZZ} H^1(\sI_X(m))$ is a graded $S$-module that
%is a finite dimensional vector space called the \emph{Hartshorne-Rao invariant} or \emph{deficiency module} of $X$;
%it is the obstruction to $X$ being arithmetically Cohen-Macaulay.
%

\section{Linkage of smooth curves in $\PP^3$}\label{SLinkage}\label{linkage section}

If the union of two smooth curves in $\PP^3$ is a complete intersection of surfaces, then the degrees and genera
of the curves are related. The relation is true much more generally, as we shall see in the next section, but the special
case is already useful and attractive:

\begin{theorem}\label{liaison genus formula-first version} Let $C, {C'}\subset \PP^3$ smooth irreducible curves of  of degrees $c,d$ whose union is the complete intersections of two surfaces $S,T$, with $S$ smooth. If the degrees of $S,T$ are $s,t$ respectively, then $\deg C+\deg C' = st$ and the genera of $C,C'$ are
related by
 $$
 g(C) - g({C'}) = \frac{s+t-4}{2}(\deg C-\deg {C'}).
 $$
\end{theorem}
In words, the difference between the genera of $C$ and ${C'}$ is proportional to the difference in their degrees, with constant of proportionality $(s+t-4)/2$.

For example, the twisted cubic lies on a quadric in the divisor class $(1,2)$. Its union with a line of class $(1,0)$ is the complete intersection with
a second quadric. Since $2+2=4$, the formula successfully predicts that they have the same genus. More interesting is the case of
a curve $C$ of genus 3 and degree 6, treated in Chapter~\ref{genus 2 and 3 chapter}: we saw that the curve lies on the complete intersection of
two cubics, which must then contain a second curve $C'$, necessarily of degree $3$. Is this a plane cubic (genus 1) or a twisted cubic (genus 0)?
the formula in the theorem shows that it must have genus 0---a twisted cubic.

Here is an illustration of one way in which liinkage can be used.
In Chapter~\ref{ModuliChapter} we considered the Hilbert scheme of twisted cubics, and proved that the open set consisting of 
smooth irreducible nondegenerate cubic curves in $\PP^3$---that is, twisted cubics---is irreducible of
dimension 12. We illustrate the use of linkage with a second proof:

\begin{proof}[Second proof of Proposition~\ref{hilb of twisted cubics}]
Let $C\subset \PP^3$ be a twisted cubic. In Chapter~\ref{linear series chapter} we used the sequence
$$
0\to H^0(\sI_C(2)) 
\to H^0(\sO_{\PP^3}(2) 
\to H^0(\sO_C)
$$
To deduce that $C$ lies on at least 2 quadrics. But the intersection of any two distinct quadrics $Q, Q' \supset C$ containing a twisted cubic curve $C$ has degree 4 and is unmixed; therefore it is the union of $C$ and a line $L \subset \PP^3$.

Conversely, suppose that $L \subset \PP^3$ is any line and  $Q, Q'$ two general quadrics containing $L$; write the intersection $Q \cap Q'$ as a union $L \cup C$. Since smooth quadrics contain lines a general quadric containing $L$ is smooth. The quadric $Q'$ will intersect $Q$ in a curve of type $(2,2)$, so the curve $C$ will have class $(2,1)$ or $(1,2)$. Since $\sI_Q =\sO_{\PP^3}(-2)$, the quadrics $Q'$ containing $L$ cut out on $Q$ the complete linear system of curves of type $(2,1)$, 
 which has no base locus, so Bertini's theorem tells us that $C$ is smooth, so that the intersection $Q \cap Q' = L \cup C$ is the union of $L$ and a twisted cubic. This suggests that we set up an incidence correspondence: let $\PP^9$ denote the projective space of quadrics in $\PP^3$, and consider
$$
\Phi = \{ (C, L, Q, Q') \in \cH^\circ \times \GG(1,3) \times \PP^9 \times \PP^9 \; \mid \; Q \cap Q' = C \cup L \}.
$$

We'll analyze $\Phi$ by considering the projection maps to $\cH^\circ$ and $\GG(1,3)$; that is, by looking at the diagram

\begin{diagram}[small]
& &  \Phi & & \\
& \ldTo^{\pi_1} & & \rdTo^{\pi_2} & \\
\cH^\circ & & & & \GG(1,3)
\end{diagram}

Consider first the projection map $\pi_2 : \Phi \to \GG(1,3)$ on the second factor. By what we just said, the fiber over any point $L \in \GG(1,3)$ is an open subset of $\PP^6 \times \PP^6$, where $\PP^6$ is the space of quadrics containing $L$; it follows that $\Phi$ is irreducible of dimension $4 + 2\times 6 = 16$. Going down the other side, we see that the map $\pi_1 : \Phi \to \cH^\circ$ is surjective, with fiber over every curve $C$ an open subsets of $\PP^2 \times \PP^2$, where $\PP^{2}$ is the projective space of quadrics containing $C$; we conclude that $\cH^\circ$ is irreducible of dimension 12.
\end{proof}

As the second proof of Proposition~\ref{hilb of twisted cubics} suggests, when the union of two curves $C$ and $C'$ forms a complete intersection we can use this fact to relate the geometry of their respective Hilbert schemes. This is a technique we'll use repeatedly. We prove it first in a special case, in which little more
than the adjunction formula is needed, and then consider a natural
generalization to purely 1-dimensional subschemes.

\fix{Add prop that double linkage on $S$ is linear equivalence. MAYBE add computation of the canonical module
in terms of linkage in this special case -- or put that with the general case, later.}

\section{Linkage of more general curves in $\PP^3$}
To say that the union of distinct smooth irreducible curves $C, C'$ without common components is a complete intersection $X$ means that 
the ideal $I_X$ of the complete intersection is the intersection $I_C\cap I_{C'}$. Since this ideal contains $I_CI_{C'}$, the ideal quotient
$I_X:I_C:= \{F \mid FI_C\subset I_X$
contains $I_{C'}$. On the other hand, if $F \notin I_{C'}$ and we choose $G\in I_C\setminus I_{C'}$, then $FG\notin I_{C'}$, so in fact
$I_X:I_C = I_{C'}$. It turns out that this relationship is the key to the formulas connecting the degrees and genera of $C,C'$, which hold 
for arbitrary purely 1-dimensional subschemes of $\PP^3$, as we shall see in Theorem~\ref{direct linkage}. This suggests the definition
of direct linkage for such subschemes:

\begin{definition}
Let $C,C'$ be purely 1-dimensional subschemes of $\PP^3$. We say that $C'$ is \emph{directly linked} to $C$ if there is a complete
intersection $X$ containing $C,C'$ and $I_X:I_C = I_{C'}$. We say that $C'$ is \emph{linked} to $C$ if they are connected by a chain of such
direct linkages, and we say that $C'$ is \emph{evenly linked} to $C$ if the chain involves an even number of steps.
\end{definition}

It turns out that the relationship of linkage is symmetric: as in the case of smooth curves above, if 
$C'$ is directly linked to $C$, then $C$ is directly linked to $C'$ as well (Theorem~\ref{general linkage}).
Thus
linkage is an equivalence relation (often referred to by its French name, Liaison).

It may seem from this definition that it would be difficult to check whether two curves in $\PP^3$---even smooth ones---are linked, but
a remarkable result of Hartshorne and Rao makes it relatively easy. If $C$ is any subscheme of $\PP^3$ the multiplication map
induces a map
$H^0(\sO_{P^3}(1) \otimes H^1(I_C(m)) \to H^1(I_C(m+1))$, making the direct sum
$$
D(C) := H^1_*\sI_C := \oplus_{m\in \ZZ} H^1(\sI_C(m)
$$
into a module over the homogeneous coordinate ring of $\PP^3$. If $C$ has no zero-dimensional components then $H^1\sI_C(m))$ vanishes
for $m<0$, and by Theorem~\ref{Serre} it vanishes for $m>>0$, so, as a vector space, $D(C)$ a finite dimensional in this case.
In our case, when $C$ is a purely 1-dimensional subscheme of $\PP^3$ we call $D(C)$ the \emph{Hartshorne-Rao module} of $C$.

Theorem~\ref{HR} of Hartshorne, which we will prove below,
says that if $C$ is directly linked to $C'$ then $D(C')$ is, up to a shift in grading, the dual of $D(C)$. The Thesis of Rao completed the picture:

\begin{theorem}\cite{MR520926}\label{Rao}
Two curves $C,C'$ are linked by an even length chain of direct linkages if and only if 
the Hartshorne-Rao modules $D(C)$ and $D(C')$ are isomorphic up to a shift in grading.
\end{theorem}

For example, the Hartshorne-Rao module of the union $C$ of two skew lines in $\PP^3$ is easily seen to be the residue field in degree 0. There are
two quadrics containing $C$, and the link with respect to these two is the union of two other skew lines, again with Hartshorne-Rao module
$k$ in degree 0. 

Perhaps the simplest case of Theorem~\ref{Rao} occurs when $C$ is arithmetically Cohen-Macaulay, so that $D(C) = 0$: it says simply that all 
arithmetically Cohen-Macaulay curves are linked to one another, and thus to complete intersections\footnote{Curves in the {\underline l}inkage 
{\underline c}lass of a {\underline c}omplete {\underline i}intersection (in any projective space) are sometimes said to be \emph{licci}, usually
pronounced as if in Italian as ``leechy''}. In this special case Theorem~\ref{Rao} was first proven in~\cite{Gaeta}, and we sketch the
 proof as a good warmup for the general case.
The first step is to invoke the following result, a special case of the Hilbert-Burch Theorem~\cite[Section 20.4]{Eisenbud1995}:

\begin{theorem}\label{Hilbert-Burch Theorem}
If $C$ is an arithmetically Cohen-Macaulay curve in $\PP^3$ then the homogeneous coordinate ring $S_C$ of $C$ has a homogeneous
minimal free resolution of the form
$$
0\rTo S^{n-1} \rTo^A S^n \rTo^\phi S \rTo S_C \rTo 0;
$$
where $S$ is the homogeneous coordinate ring of $\PP^3$ and, for simplicity, we have suppressed the twists associated to the summands of the
free modules in the resolution. Moreover, $\phi = \wedge^{n-1}A^*$ times a unit of $S$; that is, $I_C$ is minimally generated by the 
$(n-1)\times (n-1)$ minors of $A$.
\end{theorem}

\begin{proof}[Proof sketch of Theorem~\ref{Rao} when $D(C)=0$] 
Possibly after a change of basis in $S^n$ we may assume that the first two generators $F_1, F_2$ of $I_C$, corresponding to the minors of $A$
omitting the first and second columns, form a regular sequence. The inclusion of ideals $(F_1, F_2) \subset I_C$ induces a map of free resolutions:
$$
\begin{diagram}[small]
0&\rTo& S^{n-1} &\rTo& S^n &\rTo&S&\rTo &S_C &\rTo& 0\\
&&\uTo&&\uTo&&\uTo&&\uTo\\
0&\rTo& S &\rTo& S^2&\rTo& S&\rTo& S/(F,G) &\rTo& 0\\
\end{diagram},
$$
where, as before, we have suppressed all the twists.
We dualize this diagram and form the mapping cone.
Since the vertical map $\sO_{\PP^3}\to \sO_{\PP^3}$ on the right
is an isomorphism we may cancel these terms, and since we have taken $F_1, F_2$ to be two of the minimal generators of $I$ the 
second vertical map is a split inclusion, so we may cancel the $S^{2*}$ against two of the generators of $S^n$, obtaining a map
of (very short) complexes:
$$
\begin{diagram}[small]
 0&\lTo&\omega_C&\lTo&S^{n*} &\lTo& S^{n-1*}&\lTo&  0\\
 &&\dTo^\phi&&\dTo&&\dTo\\
 0&\lTo&S/(F_1,F_2)&\lTo&S &\lTo&0\\
% &&\dTo\\
% &&\sO_{C'}\\
% &&\dTo\\
% && 0
\end{diagram},
$$
Where we have written $\omega_C$ for $Ext^2_S(S/I_C, S)$; we shall see later that this is the dualizing module of $S/I_C$ (again
we have suppressed the twists), and that $\omega_C = (I_C:(F_1, F_2))/(F_1,F_2).$
Thus the cokernel of $\phi$ is $S/(I_C:(F_1, F_2))$. Because $I_C$ has codimension 2, the modules $Ext_S^i(S/I,S)$ vanish for $i<2$,
and it follows that the right-hand square of the diagram provides a free resolution 
$$
0\lTo S/(I_C:(F_1, F_2)) \lTo S\lTo S^{n-1*} \lTo S^{n-2*}\lTo 0
$$
From this we see that $S/I_C$ is linked to an arithmetically Cohen-Macaulay curve whose ideal $I_C:(F_1, F_2)$
has just $n-1$ generators. Continuing in this way, we arrive at a curve whose ideal has just 2 generators---a complete intersection.
\end{proof} 

\section{Degree and genus of linked curves}

%
%\begin{theorem}\label{liaison genus formula-first version} Let $C_1,C_2\subset \PP^3$ smooth irreducible curves  whose union is the complete intersection of surfaces $S,T$ of degrees $s,t$ respectively. Assuming that $S$ is smooth,
%The degrees and genera of the curves $C,C'$ are related by:
%$$
%\begin{aligned}
% &\deg C_1 +\deg C_2 = st\\
%&g(C_1) - g(C_2) = \frac{s+t-4}{2}(\deg C_1-\deg {C_2}).
% \end{aligned}
%$$
%\end{theorem}
%In particular, the difference between the genera of $C_1$ and ${C_2}$ is proportional to the difference in their degrees, with constant of proportionality $(s+t-4)/2$.
%
%\begin{proof}[Proof of Theorem~\ref{liaison genus formula-first version}]
%The first formula is immediate from B\'ezout's Theorem.
%
%To relate the genera, we begin with the  adjunction formula for curves on $S$, which give the genus in terms of the 
%intersection product:
%$$
%g(C_i) = \frac{C_i^2+C_i\cdot K_S}{2}+1
%$$
% Subtracting the formula for $C_2$ from the one for $C_1$ we get
%$$
%g(C_1)-g(C_2) = \frac{C_1^2-C_2^2 + (C_1-C_2)\cdot K_S}{2}.
%$$
%Writing $H$ for the class of a hyperplane, and using the adjunction formula in $\PP^3$, we see
%that $K_S = (s-4)H$, so $C_i\cdot K_S = (s-4)(\deg C)$.
%Also
%$$
%C_1^2-C_2^2 = (C_1-C_2)\cdot (C_1+C_2) = (C_1-C_2)\cdot (tH) = t(\deg C_1 -\deg C_2).
%$$
%Thus
%$$
%g(C1)-g(C_2) = \frac{t(\deg C_1 -\deg C_2) + (s-4)(\deg C_1-\deg C_2)}{2}.
%$$
%as required
% \end{proof}
%
%%\section{Linkage of purely 1-dimensional schemes in $\PP^3$}
%
%%The first step in dealing with a more general case, where $C,{C'}$ may not be reduced or irreducible, and my share components, is to agree
%%on the analogue of the statement $C\cup {C'} = S\cap T$. Writing $F,G$ for the defining equations of $S,T$ respectively,
%%In the simple case above, the primary decomposition of the ideal $(F,G)$ was the intersection of two prime ideals $I_C+I_{C'}$. From this description it
%%follows that
%%$$
%%(F,G):I_C := \{H \mid HI_C \subset (F,G)\} = I_{C'}
%%$$
%%Since the associated primes of $(F,G):I_C$ are among the associated primes of $(F,G)$, 
%%${C'}$ is again a purely 1-dimensional scheme. 
%%
%%Constructions of this type were understood by Macaulay in his great paper~\cite{Macaulay1913} or \cite{Eisenbud-Gray} for ideals in polynomial rings. For the modern theory, as
%%well as some of the history, see~\cite{MR0364271}.
%%By analogy, we will say that two purely 1-dimensional subschemes of $\PP^3$ are \emph{directly linked} by surfaces $S: \{F = 0\}$ and $T:  \{F = 0\}$ if
%%the equation above is satisfied, and we say that they are \emph{linked} if they are connected by a chain of direct linkages; the equivalence relation
%%defined in this way is called \emph{liaison}. 
%%
%%Historically, the first case to be understood was that of arithmetically Cohen-Macaulay curves in $\PP^3$.
%
%

The degrees and (arithmetic) genera 
of directly linked schemes are related exactly as in the simple case above:

\begin{theorem}\label{direct linkage}
If $C\subset \PP^3$ is a purely 1-dimensional subscheme, and ${C'}$ is directly linked to $C$, then $C$ is directly linked to ${C'}$.
If $C,{C'}\subset \PP^3$ of degrees $c,d$ are directly linked by surfaces of degrees $s,t$, then 
$\deg C+\deg C' = st$ and 
 \begin{equation}\label{linked genus formula}
p_a(C) - p_a({C'}) = \frac{s+t-4}{2}(\deg C - \deg C');
\end{equation}
\end{theorem}

\begin{proof}
 
\subsubsection{\it Proof that direct linkage is symmetric:}

Finally, we will use an elementary part of duality theory: if $S$ is a regular local ring of dimension $d$
with residue field $k$ and $R = S/(f_1,\dots, f_d)$, 
so that $R$ is a 0-dimensional complete intersection, then $R$ is injective as an $R$-module (that is, $R$ is 0-dimensional  and Gorenstein).
It follows that for any ideal $J\subset R$ we have $\ann J = Hom(R/J, R)$ and every $R$-module is reflexive,
so $\ann(\ann J) = J$.

Let $X\subset \PP^3$ be the subscheme defined by the complete intersection $I_X := (F,G)$.
To prove that the relation of linkage is symmetric we work with
the homogeneous ideals $I_C$ and $I_{C'} = I_X: I_C$. 
Since $I_X$ is a complete intersection, it is unmixed of codimension 2, and it
follows that $I_{C'}$ is unmixed of codimension 2 as well.
It is clear from the definition that
$$
I_C \subset I_X:(I_X:I_C) = I_X:I_{C'}.
$$
Since $I_C$ is unmixed of codimension 2, it is enough to check the equality
after localizing at each prime $P\subset S$ of codimension 2.
Moreover, both sides contain $(I_X)_P$, so we may pass to the ring $R = S_P/(I_X)_P$.
However, this is a zero-dimensional complete intersection so $\ann_R\ann_R(J) = J$.

\subsection{Dualizing modules}\label{duality}
To go further, we will use some facts about the dualizing modules of Cohen-Macaulay rings that generalize this
property, which comes from the fact that a Gorenstein ring is its own dualizing module; we digress from the 
proof of Theorem{direct linkage} to establish the necessary facts.

Recall that a Noetherian local ring $(R,\gm)$ of (Krull) dimension $d$ is said to be Cohen-Macaulay if there are elements $f_1,\dots,f_d\in \gm$ such that
$f_i$ is a nonzerodivisor modulo $(f_1,\dots,f_{i-1})$ for $i =1,\dots,d$, and a Noetherian ring is Cohen-Macaulay if each of its localizations at maximal ideals
is Cohen-Macaulay. Thus, for example every unmixed ring of dimension 0 or 1 is Cohen-Macaulay, and every purely 1-dimensional scheme is locally Cohen-Macaulay,
making the condition relevant in the situation of Theorem~\ref{liaison-full version}. 

Recall that in Chapter~\ref{RR} we claimed that the canonical sheaf of a smooth curve---the sheaf of differential forms---was ``the most important invertible sheaf'' after the structure sheaf. In the general setting of Cohen-Macaulay schemes, the analogue of the canonical sheaf is called the dualizing sheaf.
The general definition of the dualizing sheaf is not very illuminating; what is useful is how it is constructed and is cohomological properties relating to duality.
However, having a definition may be comforting. 
%Note that for any coherent sheaves $\sF, \sG$ on a quasiprojective  scheme $X$ there is are natural maps
%$H^p(\sF) \times \Ext^q(\sF, \sG) \to H^{p+q)(\sG)$ (Construction: An element of $\Ext^q(\sF, \sG)$ may be represented by an exact sequence starting from
%$\sG$ and ending with $\sF$. Break this into short exact sequences and use the connecting homomorphisms in the long exact sequences for cohomology.)

\begin{definition}
Let $X$ be a projective scheme over of pure dimension $d$. The \emph{dualizing sheaf} for $X$ is a coherent sheaf $\omega_X$ 
with a map $\eta: H^d(\omega_X) \to \CC$ such that for every coherent sheaf  $\sF$ the induced map
$$
H^a(\sF) \times { \rm Ext}_X^{d-a}(\sF, \omega_X) \to H^d(\omega_X) \rTo^\eta \CC
$$
is a perfect pairing for $a=d$. 
\end{definition}


\begin{fact}
If $X$ is Cohen-Macaulay, then the duality holds for all $a$ (in the non-Cohen-Macaulay case a similar result is true if we replace $\omega_X$ by a dualizing complex
and work in the derived category.) In this case, 
$$
\sHom(\omega_X, \omega_X) = \sO \hbox{ and, if $t>0$ then } \Ext_X^{t}(\omega_X, \omega_X) = 0,
$$
(see for example~\cite[Theorem[Theorems 3.3.4 and 3.3.10d]{BrunsHerzog}) so the spectral sequence $H^p(\sExt^{d-a-p}(\omega_X, \omega_X) \Rightarrow {\rm Ext}^{d-a}(\omega_X, \omega_X)$ degenerates and shows that 
$$
{\rm Ext_X}^{d-a}(\omega_X(m), \omega_X) = H^{d-a}(Hom(\omega_X(m), \omega_X)) = H^{d-a}(\sO_X(-m))
$$
Thus, if $X$ is Cohen-Macaulay, then 
$$
\chi(\omega_X(m)) =(-1)^{\dim X}\chi(\sO_X(-m)).
$$

As the name suggests, any two canonical sheaves are isomorphic in a way compatible with the
residue maps. If $X$ is smooth then the top degree differential forms $\omega_X :=\wedge^d(\Omega_X)$,
together with the classical residue (see for example~\cite[p. 648, 708]{Griffiths-Harris1978}), is a dualizing sheaf, as implied by Serre duality. Moreover, one can construct the dualizing sheaf on a scheme
$X$ by comparing it with any scheme $Y$ whose dualizing sheaf is known, in the following sense:

\begin{theorem}\label{omega}\label{general adjunction}
Suppose that $f: X\to Y$ is a finite morphism between projective schemes $X,Y$ of pure dimensions $d,e$. If $Y$ has a dualizing sheaf $\omega_Y$,
then $\omega := \sExt_Y^{e-d}(f_*\sO_X,  \omega_Y)$, regarded as a sheaf on $X$ is a dualizing sheaf for $X$ in a way compatible with the residue maps.
Moreover, if $Y$ is smooth, then $X$ is Cohen-Macaulay if and only if $ \sExt_Y^{e-d}(f_*\sO_X,  \omega_Y)= 0$ for all $m\neq e-d$.\qed
\end{theorem}

For all this see for example \cite{AltmanKleiman}.
\end{fact}


\subsection{Continuation of the Proof of Theorem~\ref{direct linkage}}

%\subsubsection{\it Degrees of directly linked curves:}
%Let $H$ be a general hyperplane in $\PP^3$ with equation $h=0$. The degrees of $C$ and $C'$ are equal to the degrees of the finite schemes $\Gamma := C\cap H$ and
%$\Gamma' := C'\cap H$. We claim that $\Gamma$ and $\Gamma'$ are directly linked by $X\cap H$, that is,
%$$
%(\sI_X:\sI_C)+(h) = (\sI_X + (h)) : (\sI_C + (h)).
%$$
%It is enough to prove that this holds modulo $\sI_X+h = \sI_{X\cap H}$, that is,
%$$
%\frac{\omega_C+(h)}{(h)} = \omega_{C\cap H}.
%$$
%Here the denominator on the left should more properly be written as $(h)(-1)$, but for simplicity in this and what follows we ignore the twists.
%Since $h$ is a nonzerodivisor modulo $I_X$ is it a nonzerodivisor on $\omega_C \subset \sO_X$, and thus
%the left hand side is isomorphic to $\omega_C/h\omega_C$, and it suffices to show that $\omega_{C\cap H)} = \omega_C/x\omega_C$,
%which follows from Lemma~\ref{restricting omega}.
%Since $\Gamma$ is directly linked to $\Gamma'$ by forms of degree $s,t$ we have
%$\deg \sO_{\Gamma'}+\deg \omega_\Gamma  = st$. On the other hand, since $\omega_\Gamma = \Hom(\sO_\Gamma, \sO_{X\cap H}$,
%and $\sO_{X\cap H}$ is 0-dimensional and Gorenstein, we have $\deg \omega_\Gamma = \deg \sO_\Gamma$, whence
%$\deg \sO_{\Gamma'}+\deg \sO_\Gamma = st$ as required.
%
%We will use a fact about the dualizing
%\begin{lemma}\label{restricting omega} If $C$ is a purely 1-dimensional projective scheme and $H$ is a general hyperplane,
%then $\omega_{C\cap H} = \omega_C(1) \mid_H$.
%\end{lemma}
%
%\begin{proof}
%Suppose  that $C\subset \PP^n$ and write $h=0$ for the equation of $H$. Since $H$ is general there is an exact sequence
%$$
%0\rTo \sO_C(-1) \rTo^h\sO_C\rTo \sO_{C\cap H} \rTo 0
%$$
%from which we get the long exact sequence containing
%$$\
%\begin{aligned}
% \cdots &\rTo \Ext^{n-1}(\sO_{C\cap H}, \omega_{\PP^n}) \rTo \omega_C \rTo^h\omega_C(1)\rTo \omega_{C\cap H}\\ 
% &\rTo 
%\Ext^{n}(\sO_{C}, \omega_{\PP^n}) \rTo \cdots
%\end{aligned}
%$$
%Since  $\sO_{C\cap H}$ is Cohen-Macaulay of codimension $n$ and $\sO_C$ is Cohen-Macaulay of codimension $n-1$,
%both $\Ext^{n-1}(\sO_{C\cap H}, \omega_{\PP^n})$ and $\Ext^{n}(\sO_{C}, \omega_{\PP^n})$ vanish, yielding the desired relation.
%\end{proof}

\subsubsection{\it Degrees of directly linked curves:}
Let $X$ be the complete intersection of surfaces of degrees $s,t$ containing $C$, and let $S_X = S/(F,G)$ be its homogeneous coordinate ring, where
$S = \CC[x_0,\dots,x_3]$ is the homogeneous coordinate ring of $\PP^3$.
From the free resolution
$$
0\rTo S(-s-t) \rTo^{
\begin{pmatrix}
 G \\ -F
\end{pmatrix}}
 S(-s)\oplus S(-t) \rTo^{
\begin{pmatrix}
 F & G
\end{pmatrix}}
 S \rTo S/(F,G) \rTo 0
$$
 and Theorem~\ref{omega} we see that
 $$
\omega_X =  \sExt^2_C(\sO_X, \omega_{\PP^3}) =\sExt^2(\sO_X, \sO_{\PP^3}(-4)) = \sO_X(s+b-4).
 $$
Note that for any ideals $J\subset I$ in a ring $A$ we have $Hom_A(A/I, A/J) \cong (J:I)/J$, where the isomorphism
sends a homomorphism $\phi$ to the element $\phi(1)$. Again from Theorem~\ref{omega} we have 
$$
\omega_C = \Hom_X(\sO_C, \omega_X) = \Hom_X(\sO_C, \sO_X)(s+t-4) = \frac{\sI_X:\sI_C}{\sI_X}(s+t-4),
$$
where we have identified $\sO_C$ with its pushforward under the inclusion map $C\to X$. 

Since $C$ is purely 1-dimensional it is Cohen-Macaulay, so
$\chi(\omega_C(m)) = -\chi(\sO_C(-m))$. It follows that the degree of $\omega_C$, which is the leading coefficient of the Hilbert polynomial of $\omega_C$, is 
equal to $\deg C$, and 
$$
st = \deg \sO_X =\deg \sO_{C'}+\deg \omega_C = \deg \sO_{C'}+\deg \sO_C
$$
as required.

\subsubsection{\it Arithmetic genera of directly linked curves:}

From Theorem~\ref{omega} we see that $\chi(\sO_X) = st(4-s-t)/2$. Since $\sO_{C'} = \sO_{\PP^3}/(\sI_X : \sI_C)$ and
$(\sI_X : \sI_C)/(\sI_X) = \omega_C(4-s-t)$ we have
$$
\begin{aligned}
-\frac{(s+t-4)}{2} (\deg C +& \deg C') \\&= -\frac{(s+t-4)}{2}st \\
&= \chi(\sO_X) \\&=  \chi(\sO_{C'})+\chi(\omega_C(4-s-t)) \\&= \chi(\sO_{C'})-\chi(\sO_C(s+t-4)) \\&= \chi(\sO_{C'})-(s+t-4)\deg C-\chi(\sO_C)
\\&= (1-p_a(\sO_{C'})) - (1-p_a(\sO_C) -(s+t-4)\deg C,
\end{aligned}
$$
whence 
$$
p_a(\sO_C) -p_a(\sO_{C'} = \frac{(s+t-4)}{2} (\deg C - \deg C'). 
$$
 \end{proof}

Linkage behaves in a simple way with respect to deficiency modules:

\begin{theorem}\label{HR}
If $C,C'$ are purely 1-dimensional subschemes of $\PP^3$ that are directly linked by a complete intersection of degrees $s,t$ then
$$
D(C') = Hom_\CC(D(C), \CC) (-s-t+4).
$$ 
as graded modules over the homogeneous coordinate ring of $\PP^3$.
\end{theorem}

\begin{proof}

Suppose that the homogeneous ideal of $C$ is generated by forms of degree $a_i, i=1,\dots,s$. Since $C$ is locally Cohen-Macaulay,
the local rings $\sO_{C,p}$ have projective dimension 2 as modules over $\sO_{\PP^3, p}$, and $\sI_{C,p}$ has projective dimension 1.
Thus we have an exact sequence
$$
0\to \sE \to \oplus_i\sO_{\PP^3}(-a_i) \to \sI_C \to 0.
$$
Since the first and second cohomology groups of the twists of $\sO_{\PP^3}$ vanish, we deduce an isomorphism
$$
D(C) := \oplus_{m\in \ZZ} H^1(\sI_C(m)) \cong \oplus_{m\in \ZZ} H^2(\sE(m)).
$$

Let $X$ be the complete intersection of two hypersurfaces, of degrees $s,t$, containing $C$. From the inclusion we deduce a
map of resolutions
$$
\begin{diagram}[small]
0&\rTo& \sE &\rTo& \oplus_i\sO_{\PP^3}(-a_i)                                         &\rTo&\sO_{\PP^3}&\rTo &\sO_C &\rTo& 0\\
&&\uTo&&\uTo&&\uTo&&\uTo\\
0&\rTo& \sO_{\PP^3}(-s-t) &\rTo& \sO_{\PP^3}(-s)\oplus \sO_{\PP^3}(-t) &\rTo& \sO_{\PP^3}&\rTo& \sO_X &\rTo& 0\\
\end{diagram}
$$
We dualize this diagram, form the mapping cone, and twist by $-s-t$. Note that $\Hom_{\PP^3}(\sO_C, \sO_{\PP^3}) = 0$. 
Also, since the vertical map $\sO_{\PP^3}\to \sO_{\PP^3}$ on the right
is an isomorphism we may cancel these terms. Noting that $\omega_C = \Ext^2(\sO_C, \sO_{\PP^3}(-4))$ we get the diagram with 
exact rows:
$$
\begin{diagram}[small]
 0&\lTo&\omega_C(-s-t+4)&\lTo&\sE^*(-s-t) &\lTo&  \oplus_i\sO_{\PP^3}(a_i-s-t)&\lTo&  0\\
 &&\dTo^\phi&&\dTo&&\dTo\\
 0&\lTo&\sO_X&\lTo&\sO_{\PP^3} &\lTo& \sO_{\PP^3}(-t)\oplus \sO_{\PP^3}(-s) &\lTo&0\\
 &&\dTo\\
 &&\sO_{C'}\\
 &&\dTo\\
 && 0
\end{diagram}.
$$
Here the column on the left is an exact sequence because $(\sI_X:\sI_C)/\sI_X \cong \omega_C(-s-t+4)$, as explained above.
We can now write a resolution of $\sI_{C'}$ by taking the mapping cone:
$$
\begin{diagram}
0\leftarrow \sI_{C'} \leftarrow \sO_{\PP^3}(-t)\oplus \sO_{\PP^3}(-s) \oplus \sE^*(-s-t) \leftarrow \oplus_i\sO_{\PP^3}(a_i-s-t)\leftarrow  0
\end{diagram}
$$
From this we see that 
$$
H^1(\sI_{C'}(m) \cong H^1(\sE^*(-s-t+m) \cong Hom_\CC( H^2(\sE(s+t-m-4)), \CC)
$$
where the last equality is from Serre duality on $\PP^3$. Summing over $m$ we see that
$D(C') \cong Hom(D(C)(s+t-4), \CC)$,
and since Serre duality is functorial, the isomorphism holds not only as graded vector spaces, but as graded $S$-modules. 

To prove that the relation of direct linkage is symmetric, we can repeat this argument starting from the locally free
resolution of $C'$, above, using the same forms of degree $s,t$ and we see that  direct link of $C'$ is $C$.
\end{proof}

\section{The linkage equivalence relation}
As an immediate consequence of Theorem~\ref{HR} we have:
\begin{corollary}(Hartshorne)
 If two curves $C,C'$ are linked by an even length chain of direct linkages, then 
 $D(C)$ and $D(C')$ are isomorphic up to a shift in grading.
\end{corollary}

As we mentioned at the beginning of this Chapter, the converse is also true: the Hartshorne-Rao modules, up to shift in grading, provide a complete invariant of
linkage. Even more precise results are known (and the characteristic 0 hypothesis is largely unnecessary); here is a sample:

\begin{fact}
\begin{theorem}
Let $S = \CC[x_0, \dots, x_3]$ be the homogeneous coordinate ring of $\PP^3$, and let $M$ be a graded $S$-module of finite length.
\begin{enumerate}
\item There is a smooth curve $C$ with $D(C) = M(m)$ for some integer $m$.
\item There is a minimum value of $m$ such that $M(m) = D(C_0)$ for some purely one-dimensional scheme $C_0$.
\item Every curve that is evenly linked to $C_0$ is obtained from $C_0$ by deformation and a process called
\end{enumerate}
\end{theorem}

Moreover, each Liaison class has a relatively simple structure, known as the \emph{Lazarsfeld-Rao property}.
We say that $C'$ is obtained from $C$ by an \emph{ascending double link} if $I_{C'} = fI_C+(g)$ for some regular sequence
contained in $I_C$---see Exercise~\ref{Basic double links}. 

\begin{theorem}\cite{MR1087803}\label{LR property}
Let $M = D(C_0)$ the the Hartshorne Rao invariant of a purely 1-dimensional subscheme of $\PP^3$, and suppose that
$M$ is minimal in the sense that no $M(m)$ is the invariant of a purely 1-dimensional scheme. 
\begin{enumerate}
 \item Every curve $C$ in $\PP^3$ with $D(C) = M$ is a deformation of $C_0$ through curves with invariant $M$.
 \item Every curve in the even linkage class of $C_0$ is the result of a series of ascending double links followed by a deformation.
\end{enumerate}
\end{theorem}

In \cite{MR714753} it is shown that general curves of reasonably large degree in $\PP^3$ and many others have the property in the hypothesis
of  Theorem~\ref{LR property}.
\end{fact}


\section{Exercises}

\begin{exercise}
 Verify that the genus formula agrees with the usual calculation of degrees and genera for divisors on a quadric of
 classes $(a,b)$ and $(d-a, d-b)$.
\end{exercise}

\begin{exercise}
Let $C$ be the disjoint union of 3 skew lines. 
\begin{enumerate}
 \item prove that $C$ lies on a unique quadric, and that $H^2(\sI_C) = 0$
 \item compute the Hartshorne-Rao module $D(C)$.
 \item show that if $\Gamma$ is the union of 3 points in $\PP^3$ then
 $H^1\sI(\Gamma) = 0$ iff the three points are colinear.
 \item Using the exact sequence in cohomology coming from the short exact sequence
$$
0\to \sI_C \rTo^{\ell} \sI_C(1) \to \sI_\Gamma(1) \to 0
$$
where $\ell$ is a linear form, show that the map of vector spaces
$$
H^1(\sI_C) \rTo^{\ell} H^1\sI_C(1))
$$
has rank<2 if and only if $\ell$ vanishes on 3 collinear points on the three lines (including the case when $\ell$ vanishes identically on one of the lines).
Conclude that if a different union $C'$ of 3 skew lines is linked to $C$, then $C'$ lies on the same quadric as $C$.
\end{enumerate}
See~\cite{Migliore} for more examples of this type.
\end{exercise}

\begin{exercise} (Liaison addition)\label{Liaison addition}
(From the unpublished Brandeis thesis of Phillip Schwartau): Show that if $(f, g)\subset I\cap J$ is a regular sequence,
 then $f I \cap gJ = (fg)$, and conclude that if $I,J$ are purely codimension 2 ideals
 defining purely 1-dimensional schemes $C,C'$ in $\PP^3$
 then  $(fI+gJ)$ defines a scheme $C''$ with $D(C'') = D(C) \oplus D(C')$.
\end{exercise}

\begin{exercise}(Basic double links)\label{Basic double links}
The special case of the construction in Exercise~\ref{Liaison addition} in which $C'$ is trivial is already interesting. 

\begin{enumerate}
 \item Show that if $I$ is a purely codimension 2 ideal
 defining a purely 1-dimensional scheme $C$ in $\PP^3$
 and $(f, g)\subset I$ is a regular sequence, then
 then  $(fI+g)$ defines a scheme $C'$ with $D(C') = D(C)(-g)$.

 \item Show directly that, with notation as above, $C'$ is directly linked to $C$
 in two steps. Since the degrees of the generators of $D(C')$ are more positive, this
 is sometimes called an \emph{ascending double link}. Geometrically it amounts to taking the
 union of $C$ with some  components that are complete intersections.
 \end{enumerate}

\end{exercise}

%footer for separate chapter files

\ifx\whole\undefined
%\makeatletter\def\@biblabel#1{#1]}\makeatother
\makeatletter \def\@biblabel#1{\ignorespaces} \makeatother
\bibliographystyle{msribib}
\bibliography{slag}

%%%% EXPLANATIONS:

% f and n
% some authors have all works collected at the end

\begingroup
%\catcode`\^\active
%if ^ is followed by 
% 1:  print f, gobble the following ^ and the next character
% 0:  print n, gobble the following ^
% any other letter: normal subscript
%\makeatletter
%\def^#1{\ifx1#1f\expandafter\@gobbletwo\else
%        \ifx0#1n\expandafter\expandafter\expandafter\@gobble
%        \else\sp{#1}\fi\fi}
%\makeatother
\let\moreadhoc\relax
\def\indexintro{%An author's cited works appear at the end of the
%author's entry; for conventions
%see the List of Citations on page~\pageref{loc}.  
%\smallbreak\noindent
%The letter `f' after a page number indicates a figure, `n' a footnote.
}
\printindex[gen]
\endgroup % end of \catcode
%requires makeindex
\end{document}
\else
\fi

\begin{corollary}[Corollary of the proof of Theorem~\ref{HR}]
If $C$ is a purely 1-dimensional subscheme of $\PP^3$ with homogeneous ideal $I = I_C$ then 
$$
D(C) \cong Hom_\CC (Ext^3(S/I, S), \CC)(-4), \CC)
$$
as graded modules over the homogeneous coordinate ring $S$ of $\PP^3$.
\end{corollary}

This is a special case of the local duality isomorphism between local cohomology and the dual of Ext; see for example \cite[Theorem A.1.9]{MR2103875}.
\begin{proof}
We may choose a surjection  $\psi:  \oplus_iS(-a_i)\rTo I$, and choose the map
$\phi: \oplus_i\sO_{\PP^3}(-a_i)\rTo\sI_C$
in the proof of Theorem~\ref{HR}
to be the corresponding map of sheaves, so that
$\sE$ is the sheafification of the graded module $E = \ker \psi$.

Since $I$ is a saturated ideal,
 the depth of $S/I$ is at least 1, so $\pd\ S/I\leq 3$, and $I$ has a free resolution of the form
$$
0\rTo G \rTo F \rTo \oplus_iS(-a_i)  \rTo S\rTo S/I \rTo 0.
$$
where $G\to F$ is a free presentation of $E$. and there is an exact sequence
$$
0 \to E^* \to F^* \to G^* \to Ext^3_S(S/I, S) \to 0.
$$
Since $\sO_C$ is Cohen-Macaulay the sheafification of $Ext^3_S(S/I, S)$ is 0; that is,
$Ext^3_S(S/I, S)$ has finite length, and writing $\widetilde{(\phantom{-})}$ for the sheafification functor,
we have a short exact sequence of sheaves 
$$
0\to \sE^* \to \widetilde{F^*} \to \widetilde{G^*}\to 0.
$$
From this we see that 
$$
Ext^3_S(S/I,S) = H^1_*(\sE^*) = Hom_\CC( H^2_*(\sE(-4)),\CC) = H^1_*(\sI)(-4),
$$
proving the assertion.
\end{proof}
