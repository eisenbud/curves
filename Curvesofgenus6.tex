%header and footer for separate chapter files

\ifx\whole\undefined
\documentclass[12pt, leqno]{book}
\usepackage{graphicx}
\input style-for-curves.sty
\usepackage{hyperref}
\usepackage{showkeys} %This shows the labels.
%\usepackage{SLAG,msribib,local}
%\usepackage{amsmath,amscd,amsthm,amssymb,amsxtra,latexsym,epsfig,epic,graphics}
%\usepackage[matrix,arrow,curve]{xy}
%\usepackage{graphicx}
%\usepackage{diagrams}
%
%%\usepackage{amsrefs}
%%%%%%%%%%%%%%%%%%%%%%%%%%%%%%%%%%%%%%%%%%
%%\textwidth16cm
%%\textheight20cm
%%\topmargin-2cm
%\oddsidemargin.8cm
%\evensidemargin1cm
%
%%%%%%Definitions
%\input preamble.tex
%\input style-for-curves.sty
%\def\TU{{\bf U}}
%\def\AA{{\mathbb A}}
%\def\BB{{\mathbb B}}
%\def\CC{{\mathbb C}}
%\def\QQ{{\mathbb Q}}
%\def\RR{{\mathbb R}}
%\def\facet{{\bf facet}}
%\def\image{{\rm image}}
%\def\cE{{\cal E}}
%\def\cF{{\cal F}}
%\def\cG{{\cal G}}
%\def\cH{{\cal H}}
%\def\cHom{{{\cal H}om}}
%\def\h{{\rm h}}
% \def\bs{{Boij-S\"oderberg{} }}
%
%\makeatletter
%\def\Ddots{\mathinner{\mkern1mu\raise\p@
%\vbox{\kern7\p@\hbox{.}}\mkern2mu
%\raise4\p@\hbox{.}\mkern2mu\raise7\p@\hbox{.}\mkern1mu}}
%\makeatother

%%
%\pagestyle{myheadings}

%\input style-for-curves.tex
%\documentclass{cambridge7A}
%\usepackage{hatcher_revised} 
%\usepackage{3264}
   
\errorcontextlines=1000
%\usepackage{makeidx}
\let\see\relax
\usepackage{makeidx}
\makeindex
% \index{word} in the doc; \index{variety!algebraic} gives variety, algebraic
% PUT a % after each \index{***}

\overfullrule=5pt
\catcode`\@\active
\def@{\mskip1.5mu} %produce a small space in math with an @

\title{Personalities of Curves}
\author{\copyright David Eisenbud and Joe Harris}
%%\includeonly{%
%0-intro,01-ChowRingDogma,02-FirstExamples,03-Grassmannians,04-GeneralGrassmannians
%,05-VectorBundlesAndChernClasses,06-LinesOnHypersurfaces,07-SingularElementsOfLinearSeries,
%08-ParameterSpaces,
%bib
%}

\date{\today}
%%\date{}
%\title{Curves}
%%{\normalsize ***Preliminary Version***}} 
%\author{David Eisenbud and Joe Harris }
%
%\begin{document}

\begin{document}
\maketitle

\pagenumbering{roman}
\setcounter{page}{5}
%\begin{5}
%\end{5}
\pagenumbering{arabic}
\tableofcontents
\fi


\chapter{Curves of genus 6}\label{genus 6 chapter}
\fix{This will come after the plane curves: we get the 5-ic del Pezzo for free, given the "conditions of adjunction", and then
the special cases will give us exercise in what the conditions of adjunction are.}


Throughout our analyses of curves of genus $g = 3, 4$ and 5, we have been able to analyze the geometry of the canonical model to verify the statement of the Brill-Noether theorem in each case. In genus 6, by contrast, a fundamental shift takes place: we cannot deduce the Brill-Noether theorem from studying the geometry of the canonical curve; rather, we need to use Brill-Noether to describe the canonical curve. We deal with hyperelliptic and trigonal curves elsewhere, so in this Chapter we will consider the remaining cases. The basis for what we do is the following result from~\cite{MR744297}:

\begin{theorem}
If $C\subset \PP^5$ is a non-trigonal canonical curve of genus 6, then $C$ is the complete intersection of a quadric hypersurface
with a surface of degree 5 that is the image of $\PP^2$ under a rational map defined by a 6-dimensional linear series
$(V,\sO_{\PP^2}(3))$, that is, by 6 cubic forms in 3 variables. 

The canonical map $C\to \PP^5$ is the composition
of the $g^2_5$ mapping $C$ birationall to the plane with a rational map from $\PP^2$ to $\PP^5$ whose image is a 
``weak del Pezzo surface'' as defined below.
\end{theorem}

As we shall see, in the general case the 6 forms generate the ideal of 4 reduced points in the plane; the resulting rational
image of $\PP^2$ in $\PP^5$ is called a del Pezzo surface. 

\begin{fact}
If $S\subset \PP^n$ is a smooth surface embedded by the complete ant-canonical linear series $|-K_S|$, then
$3\leq n\leq 9$ and $S$ has degree $K_S^2 = n$. Such a surface is called a 
del Pezzo surface.

Any del Pezzo surface of degree $n<9$ is the blowup of $\PP^2$
in $9-n$ distinct points in linearly general position, 
and the surface is the image of $\PP^2$ by  the linear series of of cubic forms containing these points. If $n=9$, the surface
is either the triple Veronese embedding of $\PP^2$ (thus: $\PP^2$ blown up at 0 points), or the double Veronese embedding of
$\PP^1\times \PP^1$.  

There is a very rich theory of del Pezzo surfaces, starting with the 27 lines on a cubic surface. The
beautiful book \cite{Manin} is an excellent reference, which also goes into some of the arithmetic theory.

In the case $n=5$ of interest to us, the ideal of the del Pezzo surface 
is generated by the $4\times 4$ Pfaffians of a $5\times 5$ skew symmetric matrix of linear forms. 
\end{fact}

In the remainder of this chapter, accordingly, we'll lay out the various possibilities for canonical curves of genus 6. First of all,  Brill-Noether tells us that a general canonical curve $C \subset \PP^5$ is not trigonal, but certainly trigonal curves of genus 6 exist; first we'll consider the geometry of these. After that, we'll turn to the picture of a general curve $C$ of genus 6, and (using Brill-Noether) describe the various linear series on $C$ and how they arise. Finally, there are 6 cases of non-trigonal curves, the geometry of whose linear series differs from that of a general curve of genus 6, and we'll spend a little time investigating and describing those.

\subsection{General curves of genus 6} 

We now returnnon-trigonal curves of genus 6. As before, a canonical curve $C \subset \PP^5$ of genus 6 lies on a 6-dimensional vector space of quadrics, but this in itself tells us little about the curve. Most of the theory we have been discussing goes was known in the early 20th century but, as far as we know the structure of this set of
6 quadrics was discovered only in:

\fix{is the following true? Do we need to exclude double covers of a plane cubic too?}
\begin{theorem}\label{general genus 6}
Every non-trigonal canonically embedded curve of genus 6 in $\PP^5$ is the complete intersection of an anti-canonical surface of degree 5 with a quadric.
\end{theorem}

\begin{proof}
 Brill-Noether gives us a $g^2_6$. If it is birational, we get the surface by blowing up a subscheme of degree 4. Can't have a base If it had a base point I guess we'd get an elliptic-hyperelliptic. If not complete, then a double cover of a twisted cubic, hyperelliptic etc
\end{proof}

\subsubsection{Quintic del Pezzos and weak del Pezzos}

Note that any such configuration of points is congruent to any other under the automorphism group $PGL_3$ of $\PP^2$; accordingly, we see that up to projective equivalence \emph{there is a unique quintic del Pezzo surface $S \subset \PP^5$}.

Note that if $L \subset S$ is a line, then since $L \cdot K_S = -1$ we must have $L\cdot L = -1$; that is, $L$ is a $(-1)$-curve and can be blown down; conversely, a $(-1)$-curve on $S$ will be a line under the embedding $S \subset \PP^5$. In fact, there are 10 such lines/exceptional divisors on $S$: in addition to the four exceptional divisors $E_i \subset S$, there are the proper transforms $L_{i,j}$ of the lines in the plane joining the four blown-up points pairwise.

One consequence of this the expression of $S$ as a blow-up of $\PP^2$ is not unique: any time we have four pairwise disjoint lines on $S$, we can blow them down and the resulting surface will be $\PP^2$. In fact, there are five such configurations of four lines: in addition to the $E_i$, we have the line $E_i$ and the three lines $L_{j,k}, L_{j,l}$ and $L_{k,l}$. There are thus five different maps $S \to \PP^2$ expressing $S$ as a blow-up of the plane at four points.

What about weak del Pezzos of degree 5? These are again blow-ups of the plane at four points; but now three of the points may be colinear, and some can be `infinitely near" points, meaning that we have a sequence of four blow-ups of $\PP^2$ in which some of the points blown up lie on the exceptional divisors of previous blow-ups. In each case, this means the blown-up surface will contain rational curves of self-intersection $-2$, which will be collapsed under the anticanonical map $\phi_{-K_S} : S \to \PP^5$ and whose images will be rational double points.

To describe the simplest of these, suppose that we vary our four distinct points  $p_i \in \PP^2$ in the plane until three of them---say, $p_1, p_2$ and $p_3$---are colinear. The anticanonical series on the resulting surface $S$ will again be given by cubics passing through the four points, but now two things are different:

\begin{enumerate}

\item Since three of the points $p_i$ are colinear, the anticanonical map will collapse the proper transform $L$ of the line containing $p_1, p_2$ and $p_3 \in \PP^2$; the image point $P$ will be a singular point of the image surface $S_0$. This point will be an ordinary double point of $S_0$, or what is known as an $A_1$ singularity.

\item Since the points $p_1, p_2$ and $p_3$ are colinear, the proper transform $L$ of the line containing them is collapsed, the limit of the line $L_{1,2}$ will be the same as the limit of the line $E_3$; likewise in the limit $L_{1,3}$ will coincide with $E_2$ and $L_{2,3}$ with $E_1$. Thus the surface $S_0$, instead of having 10 lines, will have 7: three double lines (``double" here meaning that they are each the limit of two lines on the nearby smooth del Pezzo surfaces; this also reflects the multiplicities of the corresponding points on the Fano scheme of $S_0$) and four single lines, $E_4$ and $L_{i,4}$ for $i = 1, 2$ and $3$. Note that the double lines are exactly the ones passing through the singular point of $S_0$.

\end{enumerate}

Another simple way to construct a weak del Pezzo quintic would be to have one of the points $p_i$ be infinitely near another: in other words, we blow up three distinct, non-colinear points $p_1,p_2, p_3 \in \PP^2$, then blow up a point on one of the exceptional divisors, say $E_3$. Again, we should think of this dynamically: imagine that we start with four general points $p_i$ and vary them in a one-parameter family so that in the limit $p_3$ and $p_4$ coincide.

In this situation, the proper transform in $S$ of the exceptional divisor $E_3$ has self-intersection $-2$, and is blown down under the anticanonical map to form an ordinary double point $P \in S_0$ of the image surface. Again, we see that the surface $S_0$ has seven lines: three double (in addition to the limits of $E_3$ and $E_4$ being the same, the limits of $L_{1,3}$ and $L_{1,4}$ coincide, as do $L_{2,3}$ and $L_{2,4}$) and four simple. In fact, this is not a coincidence; the surface $S_0$ is actually the same as in the previous example, as the following exercise asks you to show:

\begin{exercise}
Show that the surfaces $S_0$ constructed in the last two examples are in fact the same; that is, we can express the surface $S$ in each case either as a blow-up of $\PP^2$ at four distinct points, three of which are colinear, or at three distinct points and one infinitely near point, depending on which four lines on $S$ we blow down.
\end{exercise}

We can combine these constructions to create more singular del Pezzo quintics. For example, we can blow up $\PP^2$ at three colinear points and then blow up the result at a point of the exceptional divisor $E_3$; this will yield a singular surface $S_0 \subset \PP^5$ with two ordinary double points. At the extreme, there is a maximally singular del Pezzo surface (meaning every other singular del Pezzo specializies to it: we start by choosing a point $p \in \PP^2$ and a line $L \subset \PP^2$ through it. We blow up four times:

\begin{enumerate}
\item First, we blow up $p$;
\item Second, we blow up the point of intersection of the exceptional divisor of the first blow-up with the proper transform of the line $L$;
\item We then blow-up the the point of intersection of the exceptional divisor of the second blow-up with the proper transform of the line $L$; and finally
\item We blow up any point in the exceptional divisor of the third blow-up \emph{other than} the point of intersection  with the proper transform of the line $L$ (so we're not taking four colinear points).
\end{enumerate}

\begin{exercise}
Show that the surface $S_0$ in the last construction has a unique singular point (of type $A_4$, if you're familiar with the classification of rational double points). How many lines does $S_0$ contain?
\end{exercise}

\subsection{General canonical curves of genus 6}

Consider now a general curve $C$ of genus 6. By the Brill-Noether theorem (Theorem~\ref{BN omnibus}), $C$ will possess a $g^2_6$ (that is, a linear system of degree 6 and dimension 2), and this linear system will give a birational embedding $C \to \PP^2$ as a plane sextic curve $C_0 \subset \PP^2$ with 4 nodes. (We'll see in the following section that if $C$ is general, no three of these nodes are collinear; alternatively, this can be deduced directly from the Brill-Noether theorem as sketched in Exercise~\ref{}.) \fix{Because such a curve is not trigonal, the singular points of the image are at most double points; but why nodes -- do we need the strong B-N for this?}

Now let $S$ be the blow up of the plane at the four nodes of $C_0$, with exceptional divisors $E_1, \dots, E_4$, and let $C$ be the proper transform of the curve $C_0$; let $\phi_{-K_S} : S \to \PP^5$ be the embedding of $S$ as a del Pezzo surface. Letting $L$ denote the divisor class of the preimage in $S$ of a line in $\PP^2$, we see that the class of $C$ in the Picard group of $S$ is
$$
C \sim 6L - 2\sum E_i,
$$ 
and since the canonical divisor class of $S$ is $-3L + \sum E_i$,  Proposition~\ref{Adjunction Formula} shows that
$$
K_C = (3L - \sum E_i)|_C = -K_S|_C;
$$ 
in other words, the restriction to $C$ of the embedding $\phi_{-K_S} : S \to \PP^5$ is the canonical embedding of $C$. Moreover, since $\cO_S(C) = \cO_S(2)$, and the map 
$$
H^0(\cO_{\PP^5}(2)) \to H^0(\cO_S(2))
$$
is surjective \fix{this should have been included in the del Pezzo discussion}, we arrive at the conclusion that the canonical curve $C \subset \PP^5$ is the complete intersection of the del Pezzo surface $S \subset \PP^5$ with a quadric.

Note that we started this discussion by invoking the Brill-Noether theorem to say that the curve $C$ possessed a $g^2_6$. But in fact we now see there are five of them! As we said, there are five different maps $S \to \PP^2$ expressing $S$ as the blow-up of $\PP^2$ at four points, and the restriction to $C$ of each of these is the map associated to a $g^2_6$. Alternatively, having used the $g^2_6$ to birationally embed $C$ as a plane sextic $C_0 \subset \PP^2$ with nodes at four points $p_1,\dots,p_4$, we get four additional $g^2_6$s by taking the linear system of conics passing through 3 of the four nodes of $C_0$.

Note also that in terms of this picture we can see as well that $C$ possesses five $g^1_4$s: we have one cut on $C_0$ by the lines through any one of the four nodes $p_i$ of $C_0$, and in addition the one cut on $C_0$ by conics passing through all four.

\subsection{Other curves of genus 6}

We have described the geometry of a general curve of genus 6, invoking the Brill-Noether theorem to realize such a curve as a plane sextic with four nodes. But there are other non-trigonal curves of genus 6 that do not behave like the general such curve as described above, and we'll take a moment now to describe them.

To begin with, we have described a general canonical curve of genus 6 as the intersection of a del Pezzo quintic surface $S \subset \PP^5$ with a quadric $Q \subset \PP^5$; and conversely if $C = S \cap Q$ is the smooth intersection of a del Pezzo quintic and a quadric, then $C$ will be a canonical curve. If we realize $S$ as the blow-up of $\PP^2$ at four points, this gives us the plane model of $C$ as (the normalization of) a plane sextic curve with four nodes: the four exceptional divisors of the blow-up $S \to \PP^2$ appear as lines on $S \subset \PP^5$, and the quadric $Q$ will meet each of these four lines transversely in two distinct points. When we blow down the four lines to arrive at $\PP^2$, the image curve $C_0 \subset \PP^2$ will accordingly have four nodes.

What if $Q$ is tangent to one or more of the lines being blown down? In that case, of course, the image curve $C_0 \subset \PP^2$ will have a cusp rather than a node. We see in this way that 

\begin{exercise}
Let $S \subset \PP^5$ be a quintic del Pezzo surface; let $L_1,\dots,L_4 \subset S$ be four pairwise skew lines on $S$ and $\pi : S \to \PP^2$ the map blowing down the $L_i$. Let $p_1,\dots,p_4 $ be the images of the $L_i$.
\begin{enumerate}
\item Show that 
$$
h^0(\cO_{\PP^2}(6) \otimes m_{p_1}^3 \otimes \dots \otimes m_{p_4}^3) = 4,
$$
or in other words the $4 \times 6$ conditions that a plane sextic be triple at the points $p_i$ are independent.
\item Deduce from this that for any subset $I \subset \{1,2,3,4\}$, there is a plane sextic curve $C_0$ with a node at $p_i$ for $i \in I$ and a cusp at $p_i$ for $i \notin I$.
\end{enumerate}
\end{exercise}

There are still other possibilities for the geometry of our canonical curve $C \subset \PP^5$, which are of the form $C = S \cap Q$ with $S$ a weak del Pezzo surface. There are seven possible cases here (including the case where $S$ is del Pezzo, and six others). To describe the simplest of these, start with a configuration of four points  $p_1,\dots,p_4 \in \PP^2$ of which exactly three---say, $p_1,p_2$ and $p_3$---are collinear, and suppose $C_0$ is a plane sextic curve with nodes at the four points  $p_1,\dots,p_4$. Again, we see that we have four $g^1_4$s on the normalization $C$ of $C_0$, cut out by the lines passing through each of the nodes. But what was the fifth $g^1_4$ on $C$ in the general case---the linear series cut on $C$ by conics passing through all four---now has a fixed component, and so coincides with the series cut by lines through the fourth point $p_4$. This $g^1_4$ $\cD$ is thus the flat limit of two distinct $g^1_4$s on a general curve of genus 6 specializing to $C$---in other words, a double point of the scheme $W^1_4(C)$.

Indeed, we can see directly that $\cD$ is a non-reduced point of $W^1_4(C)$ from the omnibus Brill-Noether theorem~\ref{}. This identifies the Zariski tangent space $T_DW^r_d$ as the annihilator of the image of the map
$$
\mu : H^0(D) \otimes H^0(K-D) \to H^0(K).
$$
Now, if $D$ is the divisor cut by lines through the point $p_4$, then $K-D$ is the divisor cut by conics through $p_1,p_2,p_3$---that is, conics containing the line $L$ through the points $p_1,p_2,p_3$. The map $\mu$ is thus in this case the multiplication map between lines through $p_4$ and all lines, and that clearly has a one-dimensional kernel: if $\sigma$ and $\tau$ are sections of $\cO_C(D)$ corresponding to two lines through $p_4$, the element $\sigma \otimes \tau - \tau \otimes \sigma \in H^0(D) \otimes H^0(K-D)$ generates the kernel.

\begin{exercise}
Show that if the nodes of the curve $C_0$ are in linear general position---that is, no three collinear---then indeed the map $\mu$ is an isomorphism for each of the five $g^1_4$s on $C$.
\end{exercise}

We can describe similarly curves of genus 6 with only three, two or even one $g^1_4$. The most special is the case where $C$ has only one $g^1_4$; this is the normalization of a plane sextic with a \emph{flexed hyperoscnode}---that is, a double point consisting of two smooth branches with contact of order 4 with each other, and such that both branches have contact of order 3 with their common tangent line.

In general, we see that if $C$ is a non-trigonal curve of genus 6, the variety $W^1_4(C)$ is finite, and curvilinear (Zariski tangent space of dimension at most 1 at each point). There are 7 such schemes, corresponding to the number of partitions of 5, and indeed all occur.

\begin{exercise}
Find an example of a non-trigonal curve of genus 6 whose scheme $W^1_4(C)$ is isomorphic to each of the curvilinear schemes of degree 5 and dimension 0.
\end{exercise}

Indeed, the seven possibilities here correspond exactly to the seven isomorphism classes of weak del Pezzo quintic surfaces. (Exercise? Cheerful fact?)



%To begin with,
%
%To prove projective quadratic normality,  use general position: the general hyperplane section is 10 points in $\PP^4$ 8 of them lie on the union of two hyperplanes -- which won't contain the rest -- so they impose exactly 9 conditions. 
%
%Prove monodromy of hyperplane sections is the symmetric group. Do this carefully. Explain the correspondence between monodromy and Galois theory. 
%
%Deduce projective normality from quadratic normality.
%
%At this point, we're stuck: we still don't know what linear series exist on our curve, or much about the geometry of the canonical model. But if we invoke Brill-Noether, we have both: the curve has a $g^2_6$, which gives us a plane model as a sextic (with only double points, since no $g^1_3$s); the canonical series on the curve is cut out by cubics passing through the double points, which embeds the (blow-up of the) plane as a del Pezzo surface in $\P^5$, of which the canonical curve is a quadric section. Also, use the count of $g^2_6$s on $C$ to deduce the uniqueness of the del Pezzo.

\input footer.tex


