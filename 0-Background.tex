%header and footer for separate chapter files

\ifx\whole\undefined
\documentclass[12pt, leqno]{book}
\usepackage{graphicx}
\usepackage{eps-to-pdf}
\input style-for-curves.sty
%\input sl-macros.sty
\usepackage{hyperref}
\usepackage{showkeys} %This shows the labels.
\usepackage{msribib}
\usepackage{pdfpages}
\usepackage{draftwatermark}
\SetWatermarkText{DRAFT:\ \today}
\SetWatermarkScale{2}
\SetWatermarkColor[gray]{0.9}

%\usepackage{SLAG,msribib,local}
%\usepackage{amsmath,amscd,amsthm,amssymb,amsxtra,latexsym,epsfig,epic,graphics}
%\usepackage[matrix,arrow,curve]{xy}
%\usepackage{graphicx}
%\usepackage{diagrams}
%
%%\usepackage{amsrefs}
%%%%%%%%%%%%%%%%%%%%%%%%%%%%%%%%%%%%%%%%%%
%%\textwidth16cm
%%\textheight20cm
%%\topmargin-2cm
%\oddsidemargin.8cm
%\evensidemargin1cm
%
%%%%%%Definitions
%\input preamble.tex
%\input style-for-curves.sty
%\def\TU{{\bf U}}
%\def\AA{{\mathbb A}}
%\def\BB{{\mathbb B}}
%\def\CC{{\mathbb C}}
%\def\QQ{{\mathbb Q}}
%\def\RR{{\mathbb R}}
%\def\facet{{\bf facet}}
%\def\image{{\rm image}}
%\def\cE{{\cal E}}
%\def\cF{{\cal F}}
%\def\cG{{\cal G}}
%\def\cH{{\cal H}}
%\def\cHom{{{\cal H}om}}
%\def\h{{\rm h}}
% \def\bs{{Boij-S\"oderberg{} }}
%
%\makeatletter
%\def\Ddots{\mathinner{\mkern1mu\raise\p@
%\vbox{\kern7\p@\hbox{.}}\mkern2mu
%\raise4\p@\hbox{.}\mkern2mu\raise7\p@\hbox{.}\mkern1mu}}
%\makeatother

%%
%\pagestyle{myheadings}

%\input style-for-curves.tex
%\documentclass{cambridge7A}
%\usepackage{hatcher_revised} 
%\usepackage{3264}
   
\errorcontextlines=1000
%\usepackage{makeidx}
\let\see\relax
\usepackage{makeidx}
\makeindex
% \index{word} in the doc; \index{variety!algebraic} gives variety, algebraic
% PUT a % after each \index{***}

\overfullrule=5pt
\catcode`\@\active
\def@{\mskip1.5mu} %produce a small space in math with an @

\title{A Chapter from ``The Practice of Algebraic Curves"}
\author{\copyright David Eisenbud and Joe Harris}
%%\includeonly{%
%0-intro,01-ChowRingDogma,02-FirstExamples,03-Grassmannians,04-GeneralGrassmannians
%,05-VectorBundlesAndChernClasses,06-LinesOnHypersurfaces,07-SingularElementsOfLinearSeries,
%08-ParameterSpaces,
%bib
%}

\date{\today}
%%\date{}
%\title{Curves}
%%{\normalsize ***Preliminary Version***}} 
%\author{David Eisenbud and Joe Harris }
%
%\begin{document}

\begin{document}
\maketitle

\pagenumbering{roman}
\setcounter{page}{5}
%\begin{5}
%\end{5}
\pagenumbering{arabic}
\tableofcontents
\fi


\chapter{Background}

\section{The Noether-Lasker-Macaulay Theorems}

A key element in the algebraic study of plane curves initiated by Brill and Noether following Riemann's discoveries was what Noether named the ``Fundamental Lemma on Holomorphic Functions.'' In the original language it said that if the homogeneous forms $F(x_{0},x_{1},x_{2})=0$ and $G(x_{0},x_{1},x_{2})=0$ are the equations of two curves in $\PP^{2}$ that have no component in common, then any form $H$ that locally represents a function (the ``holomorphic functions'' of the name) vanishing on all the points
of the intersection must have an expression $H = AF+BG$, where $A,B$ are also homogeneous forms. It was extended by Lasker to the case of many polynomials (homogeneous or not) in many variables. (This is sometimes called the ``$AF+BG$ Theorem''.) 

\begin{theorem}\label{Lasker}
Suppose that $I = (f_{1}, \dots, f_{c}) \subsetneq S:=\CC[x_{0},\dots,x_{n}]$ is an ideal generated by $c$ homogeneous forms in a polynomial ring. 
If $X:= V(I)$ has codimension $c$, then $I$ is saturated and $\HH^{i}(\cO_{X}(d)) = 0$ for all $0<i<\dim X$ and all $d\in \ZZ$; and if $\dim X\geq 1$
then the map
$\HH^{0}(\cO_{\PP^{n}}(d)) \to \HH^{0}(\cO_{X}(d))$ is surjective for every $d$.
\end{theorem}

By the Principal Ideal Theorem(\cite[Theorem ***]{Eisenbud95} the codimension of $V(f_{1},\dots, f_{c})$ is at most $c$;
the Theorem thus covers the case where the codimension is ``as large as possible'', that is, the $V(f_{i})$ meet eachother
in a dimensionally transverse way.

In modern language, the conclusion says that the ring $S/I$ is Cohen-Macaulay. See for example \cite[Chapter 18]{Eisenbud95}, where the result it proven in greater generality.

Theorem~\ref{Lasker} immediately implies the orginal version of the theorem because (by the Nullstellensatz) the set of forms vanishing on $V(f_{1}, \dots, f_{c})$ is the saturation 
of the ideal $(f_{1}, \dots, f_{c})$. 

We will make use of the following algebraic Lemma, proven in  in \cite[Theorem 18.***]{Eisenbud95}:

\begin{lemma}\label{Cohen-Macaulay}
 Under the hypothesis of Theorem~\ref{Lasker}, the polynomial $f_{i}$ is a nonzerodivisor in the ring 
$S/(f_{1}, \dots, f_{i-1})$ for $i = 1, \dots, c$.
\end{lemma}

The conclusion is usually stated by saying that   $f_{1}, \dots, f_{c}$ is a \emph{regular sequence}.

\begin{proof}[Partial Proof]
 The result is obvious for $c=1$. For $c=2$ the hypothesis implies that $f_{1}$ and $f_{2}$ have no common factor.
If $g f_{2} \equiv 0\ {\rm mod}\ f_{1}$ then, since $S$ has unique prime factorization, $g$ must be divisible by
$f_{1}$, that is, $g  \equiv 0\ {\rm mod}\ f_{1}$, proving the result for $c=2$. In general, the result is equivalent to the 
statement that $S$ is a Cohen-Macaulay ring,  \cite[Proposition 18.9]{Eisenbud95}.
\end{proof}


\begin{proof} [Proof of Theorem~\ref{Lasker}] We do induction on $c$. For $c=0$ the saturation is trivial, and the vanishing in the case $c=0$ is the usual computation of the cohomology of line bundles on $\PP^{n}$. 

Now suppose that the theorem is true for $X' := (f_{1}, \dots f_{c-1})$. Note that $\dim X' = n-c+1\geq 1$.
 The surjectivity of 
the maps $\HH^{0}(\cO_{\PP^{n}}(d)) \to \HH^{0}(\cO_{X'}(d))$ shows that the homogeneous coordinate ring
of $X'$ is $R':=\oplus_{d} \HH^{0}(\cO_{X'}(d))$.

Write $e$ for the degree of $f_{c}$ . By Lemma~\ref{Cohen-Macaulay} there is a short
exact sequence
$$
0\to \cO_{X'}(-e) \rTo^{f_{c}} \cO_{X'} \to \cO_{X} \to 0.
$$
Tensoring with $\cO_{\PP^{n}}(d)$, and passing to cohomology, we obtain a long exact sequence that begins
\begin{align*}
0\to &\HH^{0}(\cO_{X'}(d-e_{1})) \rTo^{f_{1}}  \HH^{0}(\cO_{X'}(d)) \to  \HH^{0}(\cO_{X}(d))\to\\
&\HH^{1}(\cO_{X'}(d-e_{1}))\to \cdots .
\end{align*}
From the exactness of the top row we see that every element of $R$ 
that vanishes on $X$ is a multiple of $f_{c}$, so $f_{c}$ generates the homogeneous ideal of $X$ in $R$.
Lifting this back to the polynomial ring and using
 the inductive hypothesis, we see that $f_{1},\dots, f_{c}$ generate the homogeneous ideal of
$X$ in $S$.

Moreover, if $\dim X\geq 1$ then $\dim X' \geq 2$, so $\HH^{1}(\cO_{X'}(d-e_{1})) = 0$ for all $d$ by the induction,
whence the surjectivity statement holds for $X$.

Finally, the induction hypothesis and the exact sequences
$$
\HH^{i}(\cO_{X'}(d)) \to  \HH^{i}(\cO_{X}(d))\to \HH^{i+1}(\cO_{X'}(d-e_{1}))
$$
together show that $\HH^{i}(\cO_{X}(d)) = 0$ for $i<\dim X = \dim X'-1$.

\end{proof}

\subsection{Determinantal ideals}
There is a natural generalization of Theorem~\ref{Cohen-Macaulay}. Consider a homomorphism
$$
\bigoplus_{j=1}^{q}\cO_{\PP^{n}}(d_{j}) \rTo^{M}
\bigoplus_{i=1}^{p}\cO_{\PP^{n}}(e_{i})
$$
given by a 
 $p\times q$ matrix of forms
$$
\begin{pmatrix}
 f_{1,1}&f_{1,2}&\dots&f_{1,q}\\
\vdots&&\ddots&\vdots\\
f_{p,1}&f_{p,2}&\dots&f_{p,q}\\
\end{pmatrix}
$$
with $\deg f_{i,j} = \delta_{i,j} := e_{i}-d_{j}$. When $p=1$, this is just a sequence of forms of the type considered
in Theorem~\ref{Cohen-Macaulay
\section{B\'ezout's Theorem}

The most basic invariants of a subvariety $X$ of $\PP^{n}$ are its dimension and degree; for example, they determine its cohomology class in the integral cohomology $\HH^{*}(\PP^{n}; \ZZ) \cong \ZZ[x]/(x^{n+1})$.  It is convenient to compute these invariants in the case of schemes using the Hilbert polynomial. It is convenient that this definition extends at once to coherent sheaves:

\begin{theorem}
 Let $X\subset \PP^{n}$ be a subscheme. The function
 $P_{X}(t): = \chi(\cO_{X}(t)$
 is a polynomial whose degree is equal to the dimension of $X$ and whose leading coefficient
is $\deg X/(\dim X)!$. 
\end{theorem}

\begin{proof}
 We do induction on the dimension. If the dimension of $X$ is zero, then $\cO_{X}$ has a composition series
 whose successive factors have the form $\cO_{p}$, for various points $p\in \PP^{n}$, and the number of these points
 is the degree of $X$. By the additivity of the Euler characteristic, it suffices to prove the formula for a single point $p$.
 But $\cO_{p}(t) \cong \cO_{p}$, while $\HH^{0}(\cO_{p}) = 1$ and $\HH^{i}\cO_{p} = 0$. Thus 
 $\chi(\cO_{p}(t)) = 1$ for all $t$, as required.
 
 Now suppose $\dim X\geq 1$. If
 If $x$ is a general linear form on $\PP^{n}$ then $x$ is a non-zerodivisor on 
 $\cO_{X}$. Let $H$ be the hyperplane defined by $x$. Then $X' = X\cap H$ has the same degree as $X$, and dimension
 1 less. Twisting the exact sequence
 $$
 0\to \cO_{X}(-1) \to \cO_{X} \to \cO_{X'} \to 0
 $$
 by $\cO_{\PP^{n}}(t)$ we see that 
 $$
\chi(\cO_{X'}(t)) =  \chi(\cO_{X}(t)) - \chi(\cO_{X}(t-1)).
 $$
so the degree of $\chi(\cO_{X'}(t)$ is one less than the degree of $\chi(\cO_{X'}(t)$, and the leading coefficient
of $\chi(\cO_{X'}(t)$ is $\dim X$  times the leading coefficient of $\chi(\cO_{X}(t)$
\end{proof}

For example, $\PP^{n}$ itself has degree 1 since 
$$
\chi(\cO_{\PP^{n}}(t)) = {n+t\choose n} = \frac{t^{n}}{n!} +\hbox{lower degree terms}.
$$
Also, a hypersurface defined by a form of degree $d$ does indeed have degree $d$---otherwise we would not have made
this definition! More generally:

\begin{lemma}\label{nzd Bezout}
Let $X\subset \PP^{n}$ be a subscheme, and suppose that $f \in \HH^{0}(\cO_{\PP^{n}}(d))$ is form of degree $d$. Let $H$ be the hypersurface $V(f)$.  If
the induced map $\cO_{X}(-d) \to \cO_{X}$ is a monomorphism, then 
$\deg (H \cap X) = d\cdot\deg X$. In particular $H = H\cap \PP^{n}$ has degree $d$.
\end{lemma}
\begin{proof}
Twisting the exact sequence 
 $$
 0\to \cO_{X}(-d) \to \cO_{X} \to \cO_{H\cap X} \to 0
 $$
 by $\cO_{\PP^{n}}(t)$ and using the additivity of $\chi$ we see that
$\chi(\cO_{H\cap X}(t)) = \chi(\cO_{X}(t)) - \chi(\cO_{X}(t-d))$. An immediate computation shows that if 
$\chi(\cO_{X}(t)) = at^{m} +\hbox{lower degree terms}$ then 
$\chi(\cO_{H\cap X}(t)) = dat^{m-1} +\hbox{lower degree terms}$.
\end{proof}

The classic version of B\'ezout's Theorem says that plane curves of degrees $e_{1}$ and $e_{2}$ that
have no common components meet in $e_{1}e_{2}$ points, counted with multiplicity; that is, the degree of the 
intersection is $e_{1}e_{2}$.
More generally:

\begin{corollary}\label{classic Bezout}
 If $H_{1}, \dots, H_{c}$ are hypersurfaces in $\PP^{n}$ whose intersection
 $X = \cap_{i=1}^{c}H_{i}$ has codimension $c$, then 
 $$
 \deg X = \prod_{i=1}^{c}\deg H_{i}.
 $$
\end{corollary}
\begin{proof}
We do induction on $c$, the case $c=1$ being covered by Lemma~\ref{nzd Bezout}. The induction step
follows at once from Lemma~\ref{Cohen-Macaulay}.
\end{proof}


Using primary decomposition (see for example \cite[Section II.3.3]{GeomSchemes}) we can write any
scheme $X\subset \PP^{n}$ as a union of primary components $X_{i}$. The dimension of $X$ is
by definition the maximum of the dimensions of these components. Let $X'$ be the union of those
components whose dimension is equal to the dimension of $X$. Since $\cO_{X}$ is a homomorphic image of
$\oplus_{i}\cO_{X_{i}}$, and the pairwise intersections $X_{i}\cap X_{j}$ all have dimension $<\dim X$, 
we see that 
$$
\deg X = \deg X' = \sum_{\{i\mid \dim X_{i} = \dim X\}}\deg X_{i}
$$

If $f \in \HH^{0}(\cO_{\PP^{n}}(d))$, and $f$ does not contain the support of $X_{i}$, then
$f$ is a non-zerodivisor on $\cO_{X_{i}}$ in the sense of
Lemma~\ref{nzd Bezout} so $\deg H\cap X_{i} = \deg X_{i}$. 
In particular, if $\dim H\cap X = \dim X -1$, then
$\deg(H\cap X') = (\deg H)\deg X$. 

On the other hand, if $H$ does contain the
support of $X_{i}$ then from the exact squence
$$
0\to (\cI_{X}:f)/\cI_{X}(-d) \to \cO_{X_{i}}(-d) \rTo^{f} \cO_{X_{i}} \to \cO_{H\cap X_{i}} \to 0
$$
we deduce that $\deg (\cO_{H\cap X_{i}}) \leq d\deg X_{i}$.

Putting these observations together we deduce a weak but surprisingly general form of B\'ezout's Theorem:

\begin{proposition}(\cite[Exericse 8.4.6]{Fulton})\label {weak Bezout}
Let $X\subset \PP^{n}$ be a scheme, and let $H_{1}, \dots, H_{c}$ be hypersurfaces of degrees $d_{1}, \dots, d_{c}$.
The sum of the degrees of the isolated components of 
$$
H_{1}\cap\cdots\cap H_{c}\cap X
$$
is at most $(\prod_{i}d_{i})$ times the sum of the degrees of the isolated components of $X$. \qed
\end{proposition}

\begin{fact}
B\'ezout's theorem is the beginning of Intersection Theory, as described in \cite{Fulton} or \cite{3264}. One useful statement that can be extended well beyond the classical case we are considering is that
subvariety of $\PP^{n}$ has a fundamental class in $\HH^{*}(\PP^{n}; \ZZ)$, and the cup product in cohomology can be realized algebraically in terms of linear equivalence classes in the Chow ring.
\end{fact}

%\subsection{Primary Decomposition}
%
%Let $R$ be a Noetherian ring, and let $P$ be a prime ideal of $R$,
%that is an ideal $P\neq R$ such that $fg\in P$ and  $f\notin P$ implies $g \in P$.
%An ideal $Q\subset R$ is called \emph{P-primary} if $fg\in Q$ and $f\notin P$ implies $g \in Q$;
%and $Q$ is \emph{primary} if it is $P$-primary for some (necessarily unique) prime $P$.
%
%Let $I\subsetneq R$ be any ideal. We define $Ass(R/I)$ to be the set of prime ideals $P$ such that $P$ is the annihilator of some element of $R/I$. For example, if $I \subsetneq S:=\CC[x_{0},\dots,x_{n}]$ is a homogeneous ideal, then $I$ is saturated if and only if
%the maximal ideal $(x_{0}, \dots, x_{n})$ is \emph{not} in $Ass(S/I)$. 
%
%The following result was first proven, in a basic form, by the Chess champion Emmanuel Lasker in 1905 for polynomial rings, refined by F.S. Macaulay, who was the first to consider the embedded components seriously, and generalized to Noetherian rings by Emmy Noether in ***; this is the reason they are called Noetherian rings!
%
%\begin{thmdef} Let $R$ be a Noetherian ring, and let $I\subset R$ be an ideal
%$Ass(R/I)$ is a finite set, 
%and $I$ can be written as  
%$$
%I = \bigcap_{P\in Ass(R/I)} Q_{P}.
%$$
%where $Q_{P}$ is $P$-primary. Moreover, $Ass(R/I)$ is the unique minimal set of primes for which this is possible.
% If $R$ is $\ZZ$-graded and $I$ is homogeneous, then the prime ideals $P\in Ass(R/I)$ are also homogeneous, and the primary components $Q_{P}$ may be chosen to be homogeneous.
%
%The $Q_{P}$ corresponding to the minimal elements of $Ass(R/I)$ are called \emph{isolated components} of
%$I$, and they are unique; the other $Q_{P}$ are called \emph{embedded components}.
%\end{thmdef}
%
%We transfer the terminology of
%primary decomposition to schemes, saying that
%the scheme $X := V(I)$ is the union of  subschemes $X_{i}$, where each $X_{i} = V(Q_{P_{i}})$ for some primary
%component $Q_{P_{i}}$ of $I$, so that $X_{i}$ is supported on the algebraic
%set $V(P_{i})$. The \emph{isolated components} of $X$ are the schemes $X_{i}$ for isolated compoents
%$Q_{P_{i}}$; that is, the isolated components are those whose support is not contained in the support of any other
%component; and the embedded components of $X$ are the $X_{i}$ whose supports are properly contained in the supports
%of other components. For example, if
%$$
%I = (x_{0})\cap 
%(x_{1},x_{2})\cap 
%(x_{0},x_{1},x_{3}^{2}) 
%\subset \CC[x_{0},\dots,x_{3}]
%$$
%then the the primary decomposition is the one shown. $X =V(I)$ has, as isolated components, a line and a plane,
%and the plane contains an embedded point supported at the point $V(x_{0},x_{1},x_{3})$.
%\fix{picture}
%
%The first statement of the following Theorem is \cite[Exericse 8.4.6]{Fulton}.
%
%\begin{theorem}
%Let $X\subset \PP^{n}$ be a scheme defined by an ideal $(f_{1}, \dots f_{c})$, where $\deg f_{i} = e_{i}$.
%If the the isolated components of $X$ are $X_{1}, \dots, X_{s}$, then 
%$$
%\sum_{i} \deg X_{i} \leq \prod_{i} e_{i}.
%$$
%Moreover, if $\codim X = c$, then equality holds and $X$ has no embedded components.
%\end{theorem}
%
%B\'ezout via Koszul complex, at least in codim 2.
%
\section{Families of schemes}
\fix{schemes or varieties??}

%footer for separate chapter files

\ifx\whole\undefined
\makeatletter\def\@biblabel#1{#1]}\makeatother
\gdef\urlhook{\url}
\bibliography{slag}
\bibliographystyle{msribib}


%%%% EXPLANATIONS:

% f and n
% some authors have all works collected at the end

\catcode`\^\active
%if ^ is followed by 
% 1:  print f, gobble the following ^ and the next character
% 0:  print n, gobble the following ^
% any other letter: print letter
\makeatletter
\def^#1{\ifx1#1f\expandafter\@gobbletwo\else
        \ifx0#1n\expandafter\expandafter\expandafter\@gobble\else#1\fi\fi}
\makeatother
\let\moreadhoc\relax
\def\indexintro{%An author's cited works appear at the end of the
%author's entry; for conventions
%see the List of Citations on page~\pageref{loc}.  
%\smallbreak\noindent
The letter `f' after a page number indicates a figure, `n' a footnote.}
\printindex[gen]
%requires makeindex
\end{document}
\else
\fi

%\end{document}