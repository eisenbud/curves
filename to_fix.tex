%\documentclass[12pt, leqno]{article}
%\usepackage{amsmath,amscd,amsthm,amssymb,amsxtra,latexsym,epsfig,epic,graphics}
%\usepackage[matrix,arrow,curve]{xy}
%\usepackage{graphicx}
%\usepackage{diagrams}
%%\usepackage{amsrefs}
%%%%%%%%%%%%%%%%%%%%%%%%%%%%%%%%%%%%%%%%%%
%%\textwidth16cm
%%\textheight20cm
%%\topmargin-2cm
%\oddsidemargin.8cm
%\evensidemargin1cm


\ifx\whole\undefined
\documentclass[12pt, leqno]{book}
\input style-for-curves.sty

%%%%%Definitions
%\input preamble.tex
%\input style-for-curves.sty
%\def\TU{{\bf U}}
%\def\AA{{\mathbb A}}
%\def\BB{{\mathbb B}}
%\def\CC{{\mathbb C}}
%\def\QQ{{\mathbb Q}}
%\def\RR{{\mathbb R}}
%\def\facet{{\bf facet}}
%\def\image{{\rm image}}
%\def\cE{{\cal E}}
%\def\cF{{\cal F}}
%\def\cG{{\cal G}}
%\def\cH{{\cal H}}
%\def\cHom{{{\cal H}om}}
%\def\h{{\rm h}}
% \def\bs{{Boij-S\"oderberg{} }}

\makeatletter
\def\Ddots{\mathinner{\mkern1mu\raise\p@
\vbox{\kern7\p@\hbox{.}}\mkern2mu
\raise4\p@\hbox{.}\mkern2mu\raise7\p@\hbox{.}\mkern1mu}}
\makeatother

%%
%\pagestyle{myheadings}
\date{August 4, 2017}
%\date{}
\title{Odds and Ends}
%{\normalsize ***Preliminary Version***}} 
\author{David Eisenbud and Joe Harris }

\begin{document}

\maketitle

\setlength{\parskip}{5pt}

\section{Principles}
\begin{enumerate}

\item Model book: Beauville Complex Surfaces

\item 10-12 Chapters, average 20pp, each of which can be covered in 1 week (3 hours lecture)

\item Work over $\cC$, singular curves are images of smooth ones (no multiple components or embedded points.)

\item Background: able to write $H^0, H^1$, and use the terms scheme and Hilbert Function  without blushing. Assume Ch 2 of Geometry of Schemes.

\item Unproven assertions segregated in subsections labeled ``Many Cheerful Facts''.
\end{enumerate}

\section{Abbreviated version organization}
\begin{enumerate}
\item Consider moving moduli problems to beginning of Jacobians chapter. However, maybe $M_g$ in "curves of genus 1" .

\item Clifford -- Do the easy half -- sum of linear series -- in ch 1. Do the other half as application of general position. add not every secant is tri in char 0 via projection from tan line (exercise)
\item If we include the appendix on scrolls, and assuming that we talk about the curves on them, we could easily add a few lines about castelnuovo curves etc.


\item Torelli: factify it, starting with the general case of the differential of Abel-Jacobi

\item Temptations: 
	
\begin{itemize}
 \item add theta char in Jacobians ch. 
\item do tan vectors to Div, use it to give an algebraic formula for the derivative of Abel-Jacobi.
\end{itemize}
 

\end{enumerate}
\section{Notation and terminology}

The smooth curve is always $C$; its image somewhere is $C_0$
\begin{enumerate}
\item ring means commutative, noetherian, with unit.
\item Convention: ``open set" without modifier should mean Zariski open set; if we want to work with open sets in the analytic/classical/complex topology we should say ``analytic open set" or something.

\item Use "morphism" for what it is, and use "map" for rational maps.

 \item form or homogeneous polynomial?
 \item nonzerodivisor or non-zerodivisor?

\item Ideal sheaf, normal bundle, tangent bundle, differentials, principal parts: we write things like 
$\cT_{p}X$ and NOT $\cT_{X,p}$.  (DE votes for the former. Joe points out that
this will leave many things to change.) However  we write $\cI_{X}$ and NOT $\cI(X)$ and, for the homogeneous
ideal, $I_{X}$ and not $I(X)$.

\item Check notation for the direct sum of a number of copies of a bundle. Replace $\cO_{X}^{\oplus k}$ with $\cO_{X}^{k}$ but use 
$\cL^{\oplus k}$.

\item Secant varieties: the locus in the Grassmannian in denoted $\Psi_m(X)$; the locus in projective space is $\Sec_m(X)$. (need to search for $Sec_{}$ (with and without curly braces) outside of Ch 10).

\item The notation for ideal sheaves varies. If $Y \subset X$,  the ideal sheaf should be $\cI_{Y/X}$, and $\cI_{Y}$ (or $\cI_{p}$ in unambiguous cases.

\item Symbols: Silvio: Would it be possible to get a symbol for a wedge power of a vector space that's a little larger than the one for the wedge product of two vectors? The symbol in ``$\wedge^k V$" just strikes me as too small. Also, is there a good alternative to using $\cup$ for both unions and cup products?

\item Silvio: ``base point free" is three words, no hyphens. (I'm happy with pretty much any convention, but we might as well be consistent.)


\item We should respect the convention that the sheaf Hom of sheaves is
$\cH om(\cF,\cG)$, whereas $\Hom(\cF,\cG)$ is the vector space.

\item Silvio: When a lemma/proposition/theorem gets promoted or demoted, references to it are not always updated. Please check to make sure references to numbered statements are accurate. 

%\item In Chapter 13 we introduce the notation $\{ \a \}_k$ to mean the component of dimension $k$ in the class $\a$, and $[\a]_{d}$ for codim.  Check to make sure this is the convention in other chapters!.

\item IN ch 10 we use $\rDashto$ for rational maps. grep for ``ational map'' and fix them (there aren't too many) (Also, it might be nice to come up with a better arrow than $\rDashto$.)

\item Is it ``local complete intersection" or ``locally complete intersection"? The latter is epsilonically more descriptive, the former more standard. Use ``locally complete intersection''

\item David or Silvio: the hooked arrow in diagrams.tex ($\rInto$) looks really crappy (and the arrowheads don't match the ones in $\rTo$). Can we do better?


\item $\Sym^{m}$ rather than $\Sym_{m}$ for symmetric products (same for $\Sym$).
\item Ideal sheaf, normal bundle, tangent bundle, differentials, principal parts: do we write things like 
$\cT_{X,p}$ or $T_{p}X$? Do we write $\cI_{X}$ or $\cI(X)$? (DE votes for the former. Joe points out that
this will leave many things to change.)

\item the residue field at $p$ is $\kappa(p)$.

\item non-zerodivisor, non-negative, etc have hyphens.

\end{enumerate}

\section{Odds and Ends}

Capitalize words in section names?

Basic results used in this section [does this refer to the ``personalities chapter?'' or chapter 1?]: B\'ezout, Riemann-Roch, Lasker (aka AF+BG), Clifford, Adjunction.

Let's explicitly allow things like $\HH^0(D)$ where $D$ is a divisor, as well as $\HH^0(\cO(D))$, but be careful not to mix the two too much.

Would it be more confusing or less to use the same letter for a polynomial vanishing on $C$ and the surface it defines?

\

When we first introduce/set notation for linear series, define \emph{sum} of two linear series 
(with special case $\cE + E$).

\

We never put the classification of double covers of a variety with given branch divisor in 3264; this is a second chance. It would be very useful in several places in Personalities to have it; I propose we put it in.

\

Check references to the two rulings of a smooth quadric $Q \subset \PP^3$. I keep getting confused about which is which---the convention should be that a line of the first ruling means a curve of type $(1,0)$, that is, a fiber of the first projection map.

\

Expand the discussion of $\deg \geq 2g+1$ implies very ample: for example, line bundles of degree $2g$ are very ample unless $L$ is of the form $K(p+q)$ which means 

a) in genus $g \geq 3$, a general such line bundle is very ample; 

b) if $C$ is non-hyperelliptic, then any such line bundle is birationally very ample;

c) and in genus 2, it's birationally very ample unless $L = K^2$.

Also, when $\deg L = 2g-1$, $|L|$ is base-point-free iff $L$ is not of the form $K(p)$

\

Convention: ``open set" without modifier should mean Zariski open set; if we want to work with open sets in the analytic/classical/complex topology we should say ``analytic open set" or something.

\

basepoint = one word, so that e.g. ``basepoint-free" has one hyphen.

\

rational maps are denoted DashTo (with initial letter r, l, d or u to indicate direction)

\

Notation: where do we introduce the notation $g^r_d$? How about geometric vs. arithmetic genus?

\

We should have a discussion of minimal degree of irreducible, nondegenerate varieties in $\PP^n$, possibly in Chapter 1

\

Chapter 2: as consequence of RR, there is only one curve of genus 0

\

We go on about how Abel proved only the easy part of what is now called Abel's theorem; shouldn't we be calling it the Abel-Clebsch theorem instead?

\

where do we define ``rational normal curve"? What about ``twisted cubic"?

\

refer to ``B\' ezout's theorem" rather than ``B\' ezout"

\

Attributions: ``Kodaira-Serre duality" should be Serre duality; ``Riemann-Hurwitz" should be Hurwitz; ``Hartshorne-Rao" should be Rao

\ 

Chapter 6 has a reference to ``the push-pull formula''; put this terminology into the pre-requisite section?

\

Nodes: when we first use the term, we should define it.

\

We should adopt the convention that ``hyperelliptic" applies only to curves of genus $g \geq 2$.


Chapter 8: (and everywhere) Cheerful facts (environment fact) should be numbered along with the theorems etc.

\

Look for occurrences of ``specialization"; replace with ``flat limit" where appropriate

\

conflations: polynomial vs hypersurface it defined; divisor/divisor class/line bundle, etc. In each case, make statement of equivalence clearly (and put references in index?)

\

Clifford index/Voisin's theorem -- where should this be introduced/discussed first?

\

In a number of places we use the same symbol to denote a polynomial and the hypersurface it defines. Is this OK? If so, we should say somewhere that we're doing it; and if not, we should stop doing it.

\

``Noether-Lasker" should be ``Lasker's theorem"

\

``parametrize" should be ``parameterize"

\section{Notes from DE visit 6/19/21}

Chapter 1: start with brief (\~2 p) discussion of algebraic curves vs. Riemann surfaces (one direction via normalization; the other via Riemann existence)

Ch. 1: exercise on proof of resolution by projection (maybe in later chapter, after we establish $d \geq r$ for an irreducible, nondegenerate  curve of degree $d$ in $\PP^r$, if indded that's necessary)

Ch. 4, p. 5: avoid invoking Hodge theory; just use the fact that there exist $g$ independent holomorphic differentials


Ch. 5 genus 3: do bitangents and flexes. Describe $W^1_3$?

Ch. 4: introduce $W^r_d$ (or in Ch. 6, maybe?)

\

Theta-characteristics in general/bitangents to a plane quartic

\

discussion of moduli: unirationality (can we write down a general curve?); codimension in $M_g$ of locus of curves with ``special" linear series

\

description of branched covers of $\PP^1$ with given branch divisor -- put in initial discussion of hyperelliptic curves (Curves of genus 2 and 3 chapter)

\

the term ``hyperelliptic" should be reserved for curves of genus $g \geq 2$. Likewise, we should say ``non-hyperelliptic" only when the curve has genus $\geq 2$; say this somewhere, either in Ch. 2 or 4

\

Seeing as how we're constantly going back and forth between the algebraic and the complex-analytic viewpoints,  we should have a discussion of the basic GAGA results somewhere.

\

Description of curves admitting a degree $d$ map $C \to \PP^1$ with given branch divisor by $b$-tuples of permutations, mod simultaneous conjugation --- in Chapter 4 (where we do it for hyperelliptic curves), or elsewhere?

\

there's a reference to the plane curves chapter in Chapter 2; if we're not going to have a plane curves chapter that should be fixed. (If we are going to have a plane curves chapter, we should include a brief description of the discriminant hypersurface.)

\

when we introduce the ``expected dimension" of the Hilbert scheme, we should mention the analogous statements for the Hurwitz space and Severi varieties, and say that they are always irreducible of the expected dimension. (Do we want to sketch the proof of the irreducibility of $M_g$ via Hurwitz spaces?)

\

We have two versions of the Brill-Noether theorem: a ``bare bones" version asserting existence and nonexistence, and a full-fledged one (possibly including maximal rank). We should settle on names for these two, and make sure references are accurate.

\

We have different versions of the Brill-Noether theorem floating around the manuscript. I propose we have just two: a ``bare bones" version, asserting only existence when $\rho \geq 0$ and nonexistence when $\rho < 0$; and a ``full-fledged" version with everything up to and possibly including the statement of maximal rank. We don't have to call them that, but we should make sure we have the right references. 


\ 

Notation for germ of locally free sheaf at a point? $\sO_{C,p}\otimes \sL$ works, but is clumsy. In the inflections chapter I used $\frak{m}_{C,p}\sL$ for the maximal ideal times $\sL$ -- but $mm$ isn't defined yet (May 22, 2022).

\

Somewhere: we need a (brief) section on the genus formula for smooth curves on a surface, plus the modification we need for singular curves (i.e., introduce the delta-invariant of a singularity)

\

Scrolls are ubiquitous throughout the book. I propose that we have an ``interlude" in which we collect (and establish) the basic facts we'll need. (I don't think our Centennial paper fits the bill here -- we need more about the Picard groups of surface scrolls.) I'll continue to prove what we need on an ad-hoc basis, and then when we're closer to being done we can go through the text, collect the various discussions and try to put them together.

\

\end{document}



