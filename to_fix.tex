\documentclass[12pt, leqno]{article}
\usepackage{amsmath,amscd,amsthm,amssymb,amsxtra,latexsym,epsfig,epic,graphics}
\usepackage[matrix,arrow,curve]{xy}
\usepackage{graphicx}
\usepackage{diagrams}
%\usepackage{amsrefs}
%%%%%%%%%%%%%%%%%%%%%%%%%%%%%%%%%%%%%%%%%
%\textwidth16cm
%\textheight20cm
%\topmargin-2cm
\oddsidemargin.8cm
\evensidemargin1cm

%%%%%Definitions
\input preamble.tex
\def\TU{{\bf U}}
\def\AA{{\mathbb A}}
\def\BB{{\mathbb B}}
\def\CC{{\mathbb C}}
\def\QQ{{\mathbb Q}}
\def\RR{{\mathbb R}}
\def\facet{{\bf facet}}
\def\image{{\rm image}}
\def\cE{{\cal E}}
\def\cF{{\cal F}}
\def\cG{{\cal G}}
\def\cH{{\cal H}}
\def\cHom{{{\cal H}om}}
\def\h{{\rm h}}
 \def\bs{{Boij-S\"oderberg{} }}

\makeatletter
\def\Ddots{\mathinner{\mkern1mu\raise\p@
\vbox{\kern7\p@\hbox{.}}\mkern2mu
\raise4\p@\hbox{.}\mkern2mu\raise7\p@\hbox{.}\mkern1mu}}
\makeatother

%%
%\pagestyle{myheadings}
\date{August 4, 2017}
%\date{}
\title{Odds and Ends}
%{\normalsize ***Preliminary Version***}} 
\author{David Eisenbud and Joe Harris }

\begin{document}

\maketitle

\setlength{\parskip}{5pt}

section{Notation and terminology}
\begin{itemize}
\item ring means commutative, noetherian, with unit.
\item Convention: ``open set" without modifier should mean Zariski open set; if we want to work with open sets in the analytic/classical/complex topology we should say ``analytic open set" or something.

 \item form or homogeneous polynomial?
 \item nonzerodivisor or non-zerodivisor?
\end{itemize}


\section{Odds and Ends}

Capitalize words in section names?

Basic results used in this section [does this refer to the ``personalities chapter?'' or chapter 1?]: B\'ezout, Riemann-Roch, Lasker (aka AF+BG), Clifford, Adjunction.

Let's explicitly allow things like $\HH^0(D)$ where $D$ is a divisor, as well as $\HH^0(\cO(D))$, but be careful not to mix the two too much.

Would it be more confusing or less to use the same letter for a polynomial vanishing on $C$ and the surface it defines?

\

When we first introduce/set notation for linear series, define \emph{sum} of two linear series 
(with special case $\cE + E$).

\

We never put the classification of double covers of a variety with given branch divisor in 3264; this is a second chance. It would be very useful in several places in Personalities to have it; I propose we put it in.

\

Check references to the two rulings of a smooth quadric $Q \subset \PP^3$. I keep getting confused about which is which---the convention should be that a line of the first ruling means a curve of type $(1,0)$, that is, a fiber of the first projection map.

\

Expand the discussion of $\deg \geq 2g+1$ implies very ample: for example, line bundles of degree $2g$ are very ample unless $L$ is of the form $K(p+q)$ which means 

a) in genus $g \geq 3$, a general such line bundle is very ample; 

b) if $C$ is non-hyperelliptic, then any such line bundle is birationally very ample;

c) and in genus 2, it's birationally very ample unless $L = K^2$.

Also, when $\deg L = 2g-1$, $|L|$ is base-point-free iff $L$ is not of the form $K(p)$

\

Convention: ``open set" without modifier should mean Zariski open set; if we want to work with open sets in the analytic/classical/complex topology we should say ``analytic open set" or something.

\
\end{document}


