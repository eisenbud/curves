%\documentclass[12pt, leqno]{article}
%\usepackage{amsmath,amscd,amsthm,amssymb,amsxtra,latexsym,epsfig,epic,graphics}
%\usepackage[matrix,arrow,curve]{xy}
%\usepackage{graphicx}
%\usepackage{diagrams}
%%\usepackage{amsrefs}
%%%%%%%%%%%%%%%%%%%%%%%%%%%%%%%%%%%%%%%%%%
%%\textwidth16cm
%%\textheight20cm
%%\topmargin-2cm
%\oddsidemargin.8cm
%\evensidemargin1cm


\ifx\whole\undefined
\documentclass[12pt, leqno]{book}
\input style-for-curves.sty

%%%%%Definitions
%\input preamble.tex
%\input style-for-curves.sty
%\def\TU{{\bf U}}
%\def\AA{{\mathbb A}}
%\def\BB{{\mathbb B}}
%\def\CC{{\mathbb C}}
%\def\QQ{{\mathbb Q}}
%\def\RR{{\mathbb R}}
%\def\facet{{\bf facet}}
%\def\image{{\rm image}}
%\def\cE{{\cal E}}
%\def\cF{{\cal F}}
%\def\cG{{\cal G}}
%\def\cH{{\cal H}}
%\def\cHom{{{\cal H}om}}
%\def\h{{\rm h}}
% \def\bs{{Boij-S\"oderberg{} }}

\makeatletter
\def\Ddots{\mathinner{\mkern1mu\raise\p@
\vbox{\kern7\p@\hbox{.}}\mkern2mu
\raise4\p@\hbox{.}\mkern2mu\raise7\p@\hbox{.}\mkern1mu}}
\makeatother

%%
%\pagestyle{myheadings}
\date{August 4, 2017}
%\date{}
\title{To Fix}
%{\normalsize ***Preliminary Version***}} 
\author{David Eisenbud and Joe Harris }

\begin{document}

\maketitle

\setlength{\parskip}{5pt}

 
 \section{Conventions on Notation and terminology}
\begin{enumerate}

\item A \emph{curve} or variety is assumed  reduced, irreducible unless otherwise stated; we say "smooth" and "projective" when we use it. -- We have not done this. Do we still want to?

\item We should respect the convention that the sheaf Hom of sheaves is
$\cH om(\cF,\cG)$, whereas $\Hom(\cF,\cG)$ is the vector space. (search for occurrences of Hom without the backslash.

\item When a lemma/proposition/theorem gets promoted or demoted, references to it are not always updated. Please check to make sure references to numbered statements are accurate. 

\item Use $\rDashto$ for rational maps. grep for ``ational map'' and fix them (there aren't too many) . regular birational map -> birational morphism

\item ``gonality $k$" should mean expressible as a $k$-sheeted cover of $\PP^1$, not ``$k$ or less". So ``trigonal," for example, should mean non-hyperelliptic if $g \geq 3$.


\item Is the projectivization of a vector space $V$ written $\PP V$ or $\PP(V)$? (I vote for the former)

\item We use ``cusp" and ``ordinary cusp" interchangeably. Should we be consistent?


\end{enumerate}

 

\section{Substantive additions, problems in the text proper}


\begin{enumerate}
 
 \item We sometimes invoke a tiny bit of GAGA: Rather than making a global point of this, let's say what we're doing when we do it. For example, we construct Riemann surfaces
 as covering spaces and claim that they "are" projective curves. Also we construct the Jacobian analytically, and assert that
 we are dealing with algebraic families of algebraic divisor classes...
   
\item Check references to the two rulings of a smooth quadric $Q \subset \PP^3$. I keep getting confused about which is which---the convention should be that a line of the first ruling means a curve of type $(1,0)$, that is, a fiber of the first projection map.


\end{enumerate}
\section{Exercises}

\begin{enumerate}
\item Each chapter should have at least 10. 
\item Hints chapters need to be resynchronized, completed. But this should be done after all the exercises are added, with
hints, in the main text.

 \item when we introduce the ``expected dimension" of the Hilbert scheme, we should mention the analogous statements for the Hurwitz space and Severi varieties, and say that they are always irreducible of the expected dimension.


\item Add exercise on proof of resolution by projection -- done, I think.

\end{enumerate}
\end{document}



