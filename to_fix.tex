%\documentclass[12pt, leqno]{article}
%\usepackage{amsmath,amscd,amsthm,amssymb,amsxtra,latexsym,epsfig,epic,graphics}
%\usepackage[matrix,arrow,curve]{xy}
%\usepackage{graphicx}
%\usepackage{diagrams}
%%\usepackage{amsrefs}
%%%%%%%%%%%%%%%%%%%%%%%%%%%%%%%%%%%%%%%%%%
%%\textwidth16cm
%%\textheight20cm
%%\topmargin-2cm
%\oddsidemargin.8cm
%\evensidemargin1cm


\ifx\whole\undefined
\documentclass[12pt, leqno]{book}
\input style-for-curves.sty

%%%%%Definitions
%\input preamble.tex
%\input style-for-curves.sty
%\def\TU{{\bf U}}
%\def\AA{{\mathbb A}}
%\def\BB{{\mathbb B}}
%\def\CC{{\mathbb C}}
%\def\QQ{{\mathbb Q}}
%\def\RR{{\mathbb R}}
%\def\facet{{\bf facet}}
%\def\image{{\rm image}}
%\def\cE{{\cal E}}
%\def\cF{{\cal F}}
%\def\cG{{\cal G}}
%\def\cH{{\cal H}}
%\def\cHom{{{\cal H}om}}
%\def\h{{\rm h}}
% \def\bs{{Boij-S\"oderberg{} }}

\makeatletter
\def\Ddots{\mathinner{\mkern1mu\raise\p@
\vbox{\kern7\p@\hbox{.}}\mkern2mu
\raise4\p@\hbox{.}\mkern2mu\raise7\p@\hbox{.}\mkern1mu}}
\makeatother

%%
%\pagestyle{myheadings}
\date{August 4, 2017}
%\date{}
\title{To Fix}
%{\normalsize ***Preliminary Version***}} 
\author{David Eisenbud and Joe Harris }

\begin{document}

\maketitle

\setlength{\parskip}{5pt}



 
 \section{Conventions on Notation and terminology}
\begin{enumerate}

 \item reduced and irreducible, NOT integral (needs a search).

\item non-zerodivisor, non-negative, etc have hyphens.

\item Ideal sheaf, normal bundle, tangent bundle, differentials, principal parts: we write things like 
$\cT_{p}X$ and NOT $\cT_{X,p}$.  

\item use ``base-point free" is two words, with hyphens. but: a base point. 

\item We should respect the convention that the sheaf Hom of sheaves is
$\cH om(\cF,\cG)$, whereas $\Hom(\cF,\cG)$ is the vector space. (search for occurrences of Hom without the backslash.

\item Silvio: When a lemma/proposition/theorem gets promoted or demoted, references to it are not always updated. Please check to make sure references to numbered statements are accurate. 


\item Use $\rDashto$ for rational maps. grep for ``ational map'' and fix them (there aren't too many) 

\item use ``locally a complete intersection", not  ``local complete intersection".

\item use $\Sym^{m}$ rather than $\Sym_{m}$ for symmetric products. $C_d$, $Pic_d$ -- not superscripts.

\item ``gonality $k$" should mean expressible as a $k$-sheeted cover of $\PP^1$, not ``$k$ or less". So ``trigonal," for example, should mean non-hyperelliptic if $g \geq 3$.

\item The term ``ordinary node" is redundant (``node" = ``ordinary double point")

\item refer to ``B\' ezout's theorem" rather than ``B\' ezout"

\item ``parametrize" should be ``parameterize"

\item Notation for stalk (not germ) of locally free sheaf at a point: $\sO_{C,p},  \sL_p$  $\gm_{C,p}\sL_p$ 

\end{enumerate}


\section {to add to the introduction:}
\begin{enumerate}
\item ring means commutative, noetherian, with unit.
\item Convention: ``open set" without modifier should mean Zariski open set; if we want to work with open sets in the analytic/classical/complex topology we should say ``analytic open set" or something.

 \item form $=$ homogeneous polynomial?
 
 \item List some basic results in the intro: B\'ezout,  Lasker (aka AF+BG) some form of GAGA.

 \end{enumerate}
 

\section{Substantive additions, problems in the text proper}


\begin{enumerate}
 
\item Check references to the two rulings of a smooth quadric $Q \subset \PP^3$. I keep getting confused about which is which---the convention should be that a line of the first ruling means a curve of type $(1,0)$, that is, a fiber of the first projection map.

\item discussion of minimal degree of irreducible, nondegenerate varieties in $\PP^n$, in scrolls Chapter?

\item Nodes: define in Ch 1 or 2 (with discussion of $\delta$ invariant.)


\item Normalization as resolution of singularities should go in Ch 2.

\item Add exercise on proof of resolution by projection 

\item when we introduce the ``expected dimension" of the Hilbert scheme, we should mention the analogous statements for the Hurwitz space and Severi varieties, and say that they are always irreducible of the expected dimension.


\item Make an exercise: There's a basic ``exclusion principle" that we should state early and invoke as needed (or just useful). The simplest example is the observation that a curve $C$ of genus $g > 2$ cannot be simultaneously hyperelliptic and trigonal: the $g^1_2$ and $g^1_3$ would give a map to $\PP^1 \times \PP^1$; since 2 and 3 are relatively prime, this map gives a birational isomorphism of $C$ with a curve of type $(2,3)$ on $\PP^1 \times \PP^1$, violating the genus formula. In general, if $\alpha$ and $\beta$ are relatively prime, a curve of genus $g > (\alpha - 1)(\beta - 1)$ cannot have both a base-point-free $g^1_\alpha$ and a base-point-free $g^1_\beta$


\item {check references, attributions, etc.} to the omnibus B-N theorem


\item add pictures illustrating some of the possibilities for curves of genus 5  plane models. Joe draw pictures?

\item{Scrolls}
Add characterization of Castelnuovo curves -- referenced in the BN chapter section on Castelnuovo.
Prove 2n+3, leading to characterization of Castelnuovo curves

\end{enumerate}


\end{document}



