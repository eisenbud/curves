\documentclass[12pt, leqno]{article}
\usepackage{amsmath,amscd,amsthm,amssymb,amsxtra,latexsym,epsfig,epic,graphics}
\usepackage[matrix,arrow,curve]{xy}
\usepackage{graphicx}
\usepackage{diagrams}
%\usepackage{amsrefs}
%%%%%%%%%%%%%%%%%%%%%%%%%%%%%%%%%%%%%%%%%
%\textwidth16cm
%\textheight20cm
%\topmargin-2cm
\oddsidemargin.8cm
\evensidemargin1cm

%%%%%Definitions
\input preamble.tex
\def\TU{{\bf U}}
\def\AA{{\mathbb A}}
\def\BB{{\mathbb B}}
\def\CC{{\mathbb C}}
\def\QQ{{\mathbb Q}}
\def\RR{{\mathbb R}}
\def\facet{{\bf facet}}
\def\image{{\rm image}}
\def\cE{{\cal E}}
\def\cF{{\cal F}}
\def\cG{{\cal G}}
\def\cH{{\cal H}}
\def\cHom{{{\cal H}om}}
\def\h{{\rm h}}
 \def\bs{{Boij-S\"oderberg{} }}

\makeatletter
\def\Ddots{\mathinner{\mkern1mu\raise\p@
\vbox{\kern7\p@\hbox{.}}\mkern2mu
\raise4\p@\hbox{.}\mkern2mu\raise7\p@\hbox{.}\mkern1mu}}
\makeatother

%%
%\pagestyle{myheadings}
\date{August 4, 2017}
%\date{}
\title{Odds and Ends}
%{\normalsize ***Preliminary Version***}} 
\author{David Eisenbud and Joe Harris }

\begin{document}

\maketitle

\setlength{\parskip}{5pt}

\section{Odds and Ends}

Basic results used in this section: B\'ezout, Riemann-Roch, Lasker (aka AF+BG), Clifford, Adjunction.

Let's explicitly allow things like $\HH^0(D)$ where $D$ is a divisor, as well as $\HH^0(\cO(D))$, but be careful not to mix the two too much.

Would it be more confusing or less to use the same letter for a polynomial vanishing on $C$ and the surface it defines?

\

When we first introduce/set notation for linear series, define \emph{sum} of two linear series 
(with special case $\cE + E$).

\

We never put the classification of double covers of a variety with given branch divisor in 3264; this is a second chance. It would be very useful in several places in Personalities to have it; I propose we put it in.



\end{document}


