\documentclass[12pt, leqno]{book}
\usepackage{amsmath,amscd,amsthm,amssymb,amsxtra,latexsym,epsfig,epic,graphics}
\usepackage[matrix,arrow,curve]{xy}
\usepackage{graphicx}
\usepackage{diagrams}
%\usepackage{amsrefs}
%%%%%%%%%%%%%%%%%%%%%%%%%%%%%%%%%%%%%%%%%
%\textwidth16cm
%\textheight20cm
%\topmargin-2cm
\oddsidemargin.8cm
\evensidemargin1cm

%%%%%Definitions
\input preamble.tex
\def\TU{{\bf U}}
\def\AA{{\mathbb A}}
\def\BB{{\mathbb B}}
\def\CC{{\mathbb C}}
\def\QQ{{\mathbb Q}}
\def\RR{{\mathbb R}}
\def\facet{{\bf facet}}
\def\image{{\rm image}}
\def\cE{{\cal E}}
\def\cF{{\cal F}}
\def\cG{{\cal G}}
\def\cH{{\cal H}}
\def\cHom{{{\cal H}om}}
\def\h{{\rm h}}
 \def\bs{{Boij-S\"oderberg{} }}

\makeatletter
\def\Ddots{\mathinner{\mkern1mu\raise\p@
\vbox{\kern7\p@\hbox{.}}\mkern2mu
\raise4\p@\hbox{.}\mkern2mu\raise7\p@\hbox{.}\mkern1mu}}
\makeatother

%%
%\pagestyle{myheadings}
\date{April 30, 2018}
%\date{}
\title{Curves}
%{\normalsize ***Preliminary Version***}} 
\author{David Eisenbud and Joe Harris }

\begin{document}

\chapter{Basics of Curve Theory}
\section*{Outline}
 Background (corresponds to material in Hartshorne Ch 4, sects 1,2,4(?)---all except elliptic curves and ``classification")
\begin{enumerate}


\item Discussion of genus of smooth curves. Riemann-Roch. Hilbert coefficient.

\item Divisors and maps; canonical divisor -- cotangent bundle; canonical map as example
Exercise: Projections from on and off a curve

\item canonical series is bpf, $2g+1$ is very ample. Canonical Curves and geometric RR

\item Adjunction formula for curves in a surface (Quote from Hartshorne)

\item Clifford, including strong form, canonical series is va except in hyperelliptic case.

\item Riemann-Hurwitz (pull back a differential form)



\end{enumerate}

\section{Divisors, line bundles and linear systems}

One of the basic facts about projective varieties is that \emph{any regular function on a connected projective variety $X$ is constant} (such a function gives a regular map $X \to \AA^1$; since the im

\section{The genus of a curve}

The basic objects of study in this book are, simply, smooth, connected projective algebraic curves over an algebraically closed field $K$ of characteristic 0. Dropping any of these hypotheses leads us to many fascinating questions, some of which we'll discuss in \ref{****}; but before we can get into those we have to start with the basic case.

And any discussion of smooth projective curves pretty much has to start with the notion of the \emph{genus} of a curve, the sole discrete continuous invariant of curves. There are many ways of characterizing the genus of a curve $C$, some of which we'll list here:

\begin{enumerate}

\item If we are working over the complex numbers $K = \CC$, we can give $C$ the topology induced from the usual topology on $\PP^n_\CC$ (called the \emph{classical}, or \emph{analytic} topology. With this topology, the points of $C$ form a compact, oriented 2-manifold, and we define the genus of $C$ to be the genus of this surface; in other words,
$$
g(C) = 1 - \frac{\chi_{top}(C)}{2}.
$$

\item For a characterization of the genus not dependent on the assumption $K = \CC$, let $\omega$ be a rational 1-form on $C$. If we count the number of zeroes of $\omega$ and subtract the number of poles (both counted with multiplicity), the resulting integer will be equal to $2g(C) - 2$; in other words,
$$
g(C) = \frac{\deg(\omega)}{2} + 1
$$

\item Another characterization of the genus of $C$ is simple to state, though the proof that it's equivalent to the preceding ones is non-trivial (it will be a special case of the \emph{Riemann-Roch formula}, described in the following section: the genus of a curve $C$ is simply the dimension of the vector space of regular 1-forms on $C$.

\item 

\item 

\end{enumerate}

 

\end{document}