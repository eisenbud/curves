\documentclass[12pt, leqno]{book}
\usepackage{amsmath,amscd,amsthm,amssymb,amsxtra,latexsym,epsfig,epic,graphics}
\usepackage[matrix,arrow,curve]{xy}
\usepackage{graphicx}
\usepackage{diagrams}
%\usepackage{amsrefs}
%%%%%%%%%%%%%%%%%%%%%%%%%%%%%%%%%%%%%%%%%
%\textwidth16cm
%\textheight20cm
%\topmargin-2cm
\oddsidemargin.8cm
\evensidemargin1cm

\setlength{\parskip}{5pt}

%%%%%Definitions
\input preamble.tex
\def\TU{{\bf U}}
\def\AA{{\mathbb A}}
\def\BB{{\mathbb B}}
\def\CC{{\mathbb C}}
\def\QQ{{\mathbb Q}}
\def\RR{{\mathbb R}}
\def\facet{{\bf facet}}
\def\image{{\rm image}}
\def\cE{{\cal E}}
\def\cF{{\cal F}}
\def\cG{{\cal G}}
\def\cH{{\cal H}}
\def\cL{{\cal L}}
\def\cHom{{{\cal H}om}}
\def\h{{\rm h}}
 \def\bs{{Boij-S\"oderberg{} }}

\makeatletter
\def\Ddots{\mathinner{\mkern1mu\raise\p@
\vbox{\kern7\p@\hbox{.}}\mkern2mu
\raise4\p@\hbox{.}\mkern2mu\raise7\p@\hbox{.}\mkern1mu}}
\makeatother

%%
%\pagestyle{myheadings}
\date{April 30, 2018}
%\date{}
\title{Curves}
%{\normalsize ***Preliminary Version***}} 
\author{David Eisenbud and Joe Harris }

\begin{document}

\chapter{Basics of Curve Theory}
\section*{Outline}
 Background (corresponds to material in Hartshorne Ch 4, sects 1,2,4(?)---all except elliptic curves and ``classification")
\begin{enumerate}


\item Discussion of genus of smooth curves. Riemann-Roch. Hilbert coefficient.

\item Divisors and maps; canonical divisor -- cotangent bundle; canonical map as example
Exercise: Projections from on and off a curve

\item canonical series is bpf, $2g+1$ is very ample. Canonical Curves and geometric RR

\item Adjunction formula for curves in a surface (Quote from Hartshorne)

\item Clifford, including strong form, canonical series is va except in hyperelliptic case.

\item Riemann-Hurwitz (pull back a differential form)



\end{enumerate}

\section{Divisors, line bundles and linear systems}


The basic objects of study in this book are, simply, smooth, connected projective algebraic curves over an algebraically closed field $K$ of characteristic 0. Dropping any of these hypotheses leads us to many fascinating questions, some of which we'll discuss in \ref{****}; but before we can get into those we have to start with the basic case.


\subsection{The history of linear series in one page}

One of the basic facts about projective varieties is that \emph{any regular function on a connected projective curve $C$ is constant} (such a function gives a regular map $C \to \AA^1$; since the image is again a projective variety, it can only be a point).  In order to have nonconstant functions, accordingly, we have to look at rational functions. But the field of all rational functions is too large for most purposes, and so from a very early stage geometers looked at rational functions with \emph{bounded singularities}: they specified a finite collection of points $p_1,\dots,p_n \in C$, a corresponding collection of integers $m_i$, and considered the space of all rational functions $f$ on $C$, regular on $C \setminus \{p_1,\dots,p_n\}$  and satisfying

\begin{equation}\label{divisor condition}
\ord_{p_i}(f) \geq -m_i.
\end{equation}

To formalize this, they defined a \emph{divisor} on $C$ to be a formal finite linear combination
$$
D = \sum m_i \cdot p_i
$$
of points, and denoted by $\cL(D)$ the vector space of rational functions satisfying~(\ref{divisor condition}); the dimension of $\cL(D)$ was denoted $\ell(D)$.

But this is somewhat inefficient: for example, if $f$ is any rational function and we define the divisor $E = (f)$ to be
$$
E = \sum_{p\in C} \ord_p(f)\cdot p
$$
then we have an isomorphism $\cL(D+E) = \cL(D)$ obtained by multiplying by $f$. Accordingly, geometers defined an equivalence relation on the group of divisors, calling two divisors $D$ and $D'$ \emph{linearly equivalent} if their difference $E = D - D'$ was the divisor of a rational function; the group of equivalence classes of divisors was called the \emph{Picard group} $\Pic(C)$ of $C$.

As we headed into the 20th century, some further tweaking was called for in order to avoid the evils of equivalence relations. To any divisor $D = \sum m_ip_i$ on $C$, geometers associated a coherent sheaf $\cO_C(D)$, defined by
$$
\cO_C(D)(U) = \{ f \in K(U) \mid \ord_{p_i}(f) \geq -m_i \quad \forall p_i \in U \}.
$$
This is a locally free sheaf of rank 1, also called an \emph{invertible sheaf} or \emph{line bundle}. Two divisors are linearly equivalent iff the associated sheaves are isomorphic, so that the Picard group $\Pic(C)$ could be thought of simply as the group of line bundles on $C$.

What this formalism is good for is describing maps of a curve to projective space. To start with the classical approach, suppose $f : C \to \PP^r$ is a map. We can choose a hyperplane $H \cong \PP^{r-1} \subset \PP^r$ with complement $\PP^r \setminus H \cong \AA^r$; the map 
$$
C \setminus f^{-1}(H) \to \AA^r
$$
is then given simply by an $r$-tuple $(f_1,\dots,f_r)$ of regular functions on $C \setminus f^{-1}(H)$. Moreover, if we consider $f^{-1}(H)$ as a divisor on $C$, we can think of the functions $f$ as elements of $\cL(D)$.

All this requires a choice of hyperplane $H$, as well as an identification of the complement with $\AA^r$---an unnecessary evil. We can do better using the language of line bundles: if we let $\cL = f^*\cO_{\PP^r}(1)$ be the pullback to $C$ of the line bundle on $\PP^r$ associated to the divisor $H$, we can think of the $f_i$ as sections of $\cL$, so that associated to the map $f : C \to \PP^r$ we have a line bundle $\cL$ on $C$ and an $(r+1)$-dimensional vector space of global sections of $\cL$ (including the constant function 1). Conversely, if we are given a line bundle $\cL$ on $C$ and an $(r+1)$-dimensional vector space $V \subset H^0(\cL)$ of global sections of $\cL$ without common zeroes, we can choose a basis $\sigma_0,\dots,\sigma_r$ for $V$; we then get a map 

\begin{align*}
C &\to \PP^r \\
p &\mapsto [\sigma_0(p),\dots,\sigma_r(p)].
\end{align*}
The point is,  the values $\sigma_i(p)$ are elements of a one-dimensional vector space---the fiber of $\cL$ at $p$---so that the vector $[\sigma_0(p),\dots,\sigma_r(p)]$ is well-defined up to scalars.

If we were even more choice-averse and wanted to describe the map $f$ associated to a pair $(\cL, V)$ without choosing coordinates on $\PP^r$, we could describe $f$ as the map

\begin{align*}
C &\to \PP(V^*) \\
p &\mapsto H_p
\end{align*}
where $H_p = \{\sigma \in V \mid \sigma(p) = 0\}$ is the hyperplane in $V$ of sections $\sigma \in V$ vanishing at $p$.

All of this leads us to the fundamental

\begin{definition}
A \emph{linear system} on a curve $C$ is a pair $(\cL, V)$ with $\cL \in \Pic(C)$ a line bundle on $C$ and $V \subset H^0(\cL)$ a vector space of sections of $\cL$.
\end{definition}

The degree of the line bundle $\cL$ is called the degree of the linear system; the \emph{dimension} of the linear system is the dimension of the projective space $\PP V$; that is, the dimension of the vector space $V$ minus 1. A linear system is called \emph{base-point-free} is the sections $\sigma \in V \subset H^0(\cL)$ have no common zeroes. With all this said, we have the 

\begin{proposition}
There is a natural bijection between the set of nondegenerate maps $\phi : C \to \PP^r$ modulo $PGL_{r+1}$, and base-point-free linear systems of dimension $r$ on $C$.
\end{proposition}

Here ``nondegenerate" means the image of the map $\phi$ is not contained in any hyperplane. In this correspondence, the degree of the linear system corresponds to the \emph{projective degree} of the map $\phi$; that is, the cardinality of the preimage $\phi^{-1}(H)$ of a general hyperplane $H \subset \PP^r$.


\section{The genus of a curve}

Any discussion of smooth projective curves pretty much has to start with the notion of the \emph{genus} of a curve, the sole discrete continuous invariant of curves. There are many ways of characterizing the genus of a curve $C$, some of which we'll list here:

\begin{enumerate}

\item If we are working over the complex numbers $K = \CC$, we can give $C$ the topology induced from the usual topology on $\PP^n_\CC$ (called the \emph{classical}, or \emph{analytic} topology. With this topology, the points of $C$ form a compact, oriented 2-manifold, and we define the genus of $C$ to be the genus of this surface; in other words,
$$
g(C) = 1 - \frac{\chi_{top}(C)}{2}.
$$

\item For a characterization of the genus not dependent on the assumption $K = \CC$, let $\omega$ be a rational 1-form on $C$. If we count the number of zeroes of $\omega$ and subtract the number of poles (both counted with multiplicity), the resulting integer will be equal to $2g(C) - 2$; in other words,
$$
g(C) = \frac{\deg(\omega)}{2} + 1
$$

\item Another characterization of the genus of $C$ is simple to state, though the proof that it's equivalent to the preceding ones is non-trivial (it will be a special case of the \emph{Riemann-Roch formula}, described in the following section: the genus of a curve $C$ is simply the dimension of the vector space of regular 1-forms on $C$.

\item 

\item 

\end{enumerate}

 

\end{document}