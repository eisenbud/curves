\documentclass[12pt, leqno]{book}
\usepackage{amsmath,amscd,amsthm,amssymb,amsxtra,latexsym,epsfig,epic,graphics}
\usepackage[matrix,arrow,curve]{xy}
\usepackage{graphicx}
\usepackage{diagrams}
%\usepackage{amsrefs}
%%%%%%%%%%%%%%%%%%%%%%%%%%%%%%%%%%%%%%%%%
%\textwidth16cm
%\textheight20cm
%\topmargin-2cm
\oddsidemargin.8cm
\evensidemargin1cm

%%%%%Definitions
\input preamble.tex
\def\TU{{\bf U}}
\def\AA{{\mathbb A}}
\def\BB{{\mathbb B}}
\def\CC{{\mathbb C}}
\def\QQ{{\mathbb Q}}
\def\RR{{\mathbb R}}
\def\facet{{\bf facet}}
\def\image{{\rm image}}
\def\cE{{\cal E}}
\def\cF{{\cal F}}
\def\cG{{\cal G}}
\def\cH{{\cal H}}
\def\cHom{{{\cal H}om}}
\def\h{{\rm h}}
 \def\bs{{Boij-S\"oderberg{} }}

\makeatletter
\def\Ddots{\mathinner{\mkern1mu\raise\p@
\vbox{\kern7\p@\hbox{.}}\mkern2mu
\raise4\p@\hbox{.}\mkern2mu\raise7\p@\hbox{.}\mkern1mu}}
\makeatother

%%
%\pagestyle{myheadings}
\date{April 30, 2018}
%\date{}
\title{Curves}
%{\normalsize ***Preliminary Version***}} 
\author{David Eisenbud and Joe Harris }

\begin{document}

\chapter{Basics of Curve Theory}
\section*{Outline}
 Background (corresponds to material in Hartshorne Ch 4, sects 1,2,4(?)---all except elliptic curves and ``classification")
\begin{enumerate}


\item Discussion of genus of smooth curves. Riemann-Roch. Hilbert coefficient.

\item Divisors and maps; canonical divisor -- cotangent bundle; canonical map as example
Exercise: Projections from on and off a curve

\item canonical series is bpf, $2g+1$ is very ample. Canonical Curves and geometric RR

\item Adjunction formula for curves in a surface (Quote from Hartshorne)

\item Clifford, including strong form, canonical series is va except in hyperelliptic case.

\item Riemann-Hurwitz (pull back a differential form)



\end{enumerate}

\end{document}