\input header.tex

\chapter{Exercises for Chapter 4-Jacobians}


\begin{exercise}
 Let $G$ be a finite group acting on a quasi-projective scheme $X$. Show that there is a finite covering of $X$ by invariant open affine sets. (Hint: consider the sum of the $G$-translates of a very ample divisor.)
\end{exercise}


\begin{exercise}\label{free actions}
We say that a group $G$ acts freely on $X$ if $gx = gy$ only when $g$ is the identity or $x=y$. Show that
 if $G$ is a finite group acting freely on a smooth affine variety $X$ then the quotient $X/G$ is smooth.
 
 Hint: We may assume that $X = \Spec R$ is affine; the quotient is then $\Spec R^G$. Since $R$ is smooth (and thus normal),
 $R^G$ is normal.
 
  After tensoring with the 
 completion of $R^G$ at a maximal ideal $\gm$, $R$ becomes a direct product of the complete local rings of  
 maximal ideals lying over $\gm$, which correspond to the points in an orbit (\cite[Corollary 7.6 and Proposition 13.10]{Eisenbud1995}.
 Since the degree of $R$ over $R^G$ is the number of points in the orbit, each of the direct factors is birational to $\hat R^G_\gm$,
 and since $R^G$ is normal, each factor must be isomorphic to $\hat R^G_\gm$.
 \end{exercise}


\begin{exercise}
 \label{sym2A2} 
 \begin{enumerate}
 \item Let $X = (\AA^{2})^{2}$ and let $G := \ZZ/2$ act on $X$ by permuting the two copies of  $\AA^{2}$; algebraically,
$(\AA^{2})^{2} = \Spec S$, with $S = k[x_{1},x_{2}, y_{1}, y_{2}]$ and the nontrivial element $\sigma\in G$ acts by
$\sigma(x_{i}) = y_{i}$. 
\item Show that $G$ acts freely on the complement of the diagonal, but fixes the diagonal pointwise.
\item Show that the algebra $S^{G}$ has dimension 4 and is generated by the 5 elements
$$ 
f_{1} = x_{1}+y_{1}, f_{2} = x_{2}+y_{2}, g_{1} = x_{1}y_{1}, g_{2} = x_{2}y_{2}, h = x_{1}y_{2}+x_{2}y_{1},
$$
perhaps by appropriately modifying the steps given in \cite[Exercise 1.6]{Eisenbud1995}. 
\item Show that $h^2$ lies in the subring generated by $f_1,\dots, f_4$, and thus $S^{(2)}$ is a hypersurface, singular
along the  codimension 2 subset $f_{1} = f_{2} = 0$, which is the image of the diagonal subset of the 
cartesian product $(\AA^{2})^{2}$.
\end{enumerate}
\end{exercise}

\begin{exercise}[$g+1$ theorem]\label{g+1 theorem} 
Let $C$ be a smooth projective curve of genus $g$. If $D \in C_{g+1}$ is a general effective divisor of degree $g+1$ on $C$, then 
$D$ is base-point free and defines a  map to $\PP^1$ that has only simple ramification, with $2g+2$ distinct  branch points.
\end{exercise}

Hint: 
A general divisor of degree $\geq g+1$ is nonspecial, so $h^0(\sO_C(D)) = (g+1)-g+1 = 2$. To say that the linear series $|D|$ has no base points means that $D$ is not equivalent to a divisor of the form
$D'+p$, where $h^0(D')$ has degree $g$ and two independent sections. By the Riemann-Roch theorem, $h^0(D') = 1+h^0(K-D')$, so the condition on $D'$ means that $K-D'$ is effective. But $K-D'$ is a general divisor of degree $g-2$, and
the set of classes of effective divisors, the image of $C_{g-2}$, has dimension only $g-2$. 
Thus the set of divisors of the form $D'+p$ with $h^0(D') \geq 2$ has dimension only $g -2 +1$
so a general divisor $D$ of degree $g+1$ is base-point-free and defines a map $C\to \PP^1$. 

By Hurwitz' theorem the total ramification of such a map is $2g+2$. The divisors with points of ramification
index 2 or more are linearly equivalent to divisors of the form $D'+3p$, and the divisors that have two ramification
points mapping to the same point are equivalent to divisors of the form $D''+2p+2q$. Each of these sets
of divisors fills only a $g-1$-dimensional family of equivalence classes, the images of 
$C_{g-2}\times C$ or $C_{g-3}\times C_2$ respectively; so not every divisor $D$ is of this form.




% \begin{exercise}[The universal divisor of degree $d$]\label{universal divisor}
%Let $C$ be a smooth projective curve, and $C^{(d)}$ its $d$th symmetric power. Show that the locus
%$$
%\cD := \{ (D, p) \in C^{(d)} \times C \mid p \in D \}
%$$
%is a closed subvariety of the product $C^{(d)} \times C$, whose fiber over any point $D \in C^{(d)}$ is the divisor $D \subset C$.
%\end{exercise}

%\begin{exercise}
%For any smooth curve $C$, show that a general invertible sheaf of degree $g+2$ defines a birational map to $\\P^2$, and the image of this map has only nodes as singularities. Compute the number of singularities
%of such a plane curve.
%
%Hint: Given a general divisor class $D$ of degree $g+2$, we have to show three things: that the image $C_0 = \phi_D(C)$ does not have cusps; that it does not have triple points, and that it does not have tacnodes. (The facts that $|D|$ is base point free and birationally very ample will follow from these arguments, as noted below.)
%
%Cusps: to say that a point $p \in C$ maps to a cusp of $C_0$ (that is, the differential $d\phi_D$ is zero at $p$) amounts to saying that $h^0(D-2p) \geq 2$; that is, $D-2p$ is a $g^1_g$. But by Riemann-Roch, $W^1_g = K_C - W_{g-2}$; so to say $\phi_D$ has a cusp means that
%$$
%\mu(D) \in 2W_1 + K_C - W_{g-2},
%$$
%and the locus on the right has dimension at most $g-1$, a general point of $J(C)$ will not lie in it. Note that this subsumes the fact that $|D|$ has no base points.
%
%Triple points: to say that $C_0$ has a triple point means that for some divisor $E = p+q+r$ of degree 3, $h^0(D-E) \geq 1$; thus we must have 
%$$
%\mu(D) \in W_3 + W^1_{g-1}
%$$
%Now, to argue that this is not the case, we need to know that $\dim W^1_{g-1} \leq g-4$. In fact, that's not always true: the correct statement is that $\dim W^1_{g-1} = g-4$ if $C$ is non-hyperelliptic, and $\dim W^1_{g-1} = g-3$ if $C$ is hyperelliptic. (This is Marten's theorem.) In fact, the hyperelliptic case violates the statement of our theorem/exercise: if $C$ is hyperelliptic, then a general divisor of degree $g+2$ is of the form $g^1_2 + p_1+ \dots + p_g$, from which we see that $\phi_D$ maps $C$ birationally onto a plane curve of degree $g+2$ having a point of multiplicity $g$! So it seems we have to add ``$C$ nonhyperelliptic" to the hypotheses.
%
%Tacnodes: To say that a pair of points $p, q \in C$ map to a tacnode of $C_0$ means two things: that $h^0(D-E) \geq 2$ (where $E = p+q$); and that $h^0(D-2E) \geq 1$. This means
%$$
%\mu(D) \in W_2 + (K_C - W_{g-2}) \quad \text{and} \quad \mu(D) \in 2W_2 + W_{g-2}.
%$$
%Now, each of these conditions is satisfied by a general divisor $D$ of degree $g+2$; the point is, they can't be satisfied simultaneously with the same divisor $E$; and to see this we would need a description of the tangent spaces to subvarieties of $J(C)$.
%
%\end{exercise}


\begin{exercise}
Show that if $r \geq d-g$, then $W^r_d(C) \setminus W^{r+1}_d(C)$ is dense in $W^r_d(C)$ (that is, $W^{r+1}_d(C)$ does not contain any irreducible component of $W^r_d(C)$).

\noindent {\bf Hint}: If $D \in W^{r+1}_d$, show that for $p, q \in C$ general points we have $h^0(D+p-q) = h^0(L) - 1$.

\noindent {\bf Extended Hint}:  If $D$ is any divisor, show that for any $p \in C$,
$$
h^0(D+p) = 
\begin{cases}
h^0(D) + 1, &\text{if $p$ is a base point of } |K_C-D|; \text{ and} \\
h^0(D), &\text{otherwise}
\end{cases}
$$

\end{exercise}

\begin{exercise}
Let $C$ be a curve of genus 2, and let $C \subset J(C)$ be the image of the Abel-Jacobi map $\mu_1$. Show that the self-intersection of the curve $C$ is 2,
\begin{enumerate}
\item by applying the adjunction formula to $C \subset J(C)$; and
\item by calculating the self-intersection of its preimage $C + p \subset C_2$ and using the geometry of the map $\mu_2$.
\end{enumerate}

\noindent {\bf Hint}:  for the first part, observe that since the tangent bundle of $J(C)$ is trivial, its canonical bundle $K_{J(C)} \cong \cO_{J(C)}$. For the second, observe first that the self-intersection of the curve $C+p = \{q+p \mid q \in C\}$ is equal to its intersection number with $C' = \{q+p' \mid q \in C\}$ for any $p' \in C$, which is visibly 1; then, since the map $\mu : C_2 \to J(C)$ is a blow-up map, and $C$ meets the exceptional divisor in one point, the self-intersection of $C$ in $J(C)$ is 1 greater than the self-intersection of $C+p$ on $C_2$.
\end{exercise}

\begin{exercise}
Let $C$ be a curve of genus 2, and consider the map $\nu : C \times C \to \Pic_0(C)$ defined by sending $(p, q)\in C \times C$ to the invertible sheaf $\cO_C(p-q)$. Show that this map is generically finite, and compute its degree.

\noindent {\bf Hint}: We are asking, given $D$ a general divisor  of degree 0, for how many pairs $p, q \in C$ is $p-q \sim D$. Now, to say that   $p-q \sim D$ means that $h^0(D+q) \geq 1$, which by Riemann-Roch means that $h^0(K_C-D-q) \geq 1$. But $D$ being general means $h^0(K_C - D) = 1$; that is, the divisor class $K_C - D$ is represented by a unique effective divisor $r+s$ of degree 2, meaning $q$ must be equal to either $r$ or $s$. Thus there are exactly two points in the preimage $\nu^{-1}(D)$, and hence $\deg(\nu) = 2$.
\end{exercise}

\begin{exercise} \label{comparison with geometric RR}
Show that the image of the differential of the Abel-Jacobi map $C^{(d)} \to J(C)$ at a point of $C^{(d)} \setminus C^1_d$  corresponding to a reduced divisor is  the plane in $\PP^{g-1}$ spanned by the divisor $D$ on the canonical curve.

Hint: imitate the case $d=1$, done in section~\ref{Abel-Jacobi differential}.
\end{exercise}
%\fix{can we do this for non-reduced divisors?}

\begin{exercise}\label{blow-up of $J(C)$ at a point}
Let $C$ be a curve of genus 2. The canonical map $\phi_K : C \to \PP^1$ expresses $C$ as a 2-sheeted cover of $\PP^1$, and we have correspondingly an involution $\tau : C \to C$ exchanging points in the fibers of $\phi_K$ (equivalently, for any $p \in C$, we have $h^0(K_C(-p)) = 1$; $\tau$ will send $p$ to the unique zero of the unique section $\sigma \in H^0(K_C(-p))$). Let $\Gamma \subset C \times C$ be the graph of $\tau$.
\begin{enumerate}
\item Using the fact that a birational morphism of smooth surfaces must be the inverse of a sequence of blow-ups of reduced points (\cite[V.??]{H}) show the self-intersection of the image $C^1_2$ of $\Gamma$ in $C_2$ is $-1$.
\item Find the self-intersection of $\Gamma$ in $C \times C$.
\end{enumerate}

\noindent {\bf Hint}: For the first part, the image $C^1_2$  of $\Gamma$ in $C_2$ is the exceptional divisor of the blow-up map $\mu : C_2 \to J(C)$, and so has self-intersection $-1$. Since $\tau$ is an involution, its graph is the preimage of $C^1_2$; since the map $\mu : C_2 \to J(C)$ has degree 2, we have $\Gamma^2 = 2(C^1_2)^2 = -2$.
\end{exercise}

\begin{exercise}
Let $C$ be any smooth projective curve of genus $g$, $D$ a general divisor of degree $g+3$ on $C$ and $\phi_D : C \hookrightarrow \PP^3$ the corresponding embedding of $C$ in $\PP^3$.
\begin{enumerate}
\item Show that $C$ does not have any bitangent lines.
\item If we assume that $C$ is non-hyperelliptic, show that no tangent line to $C$ meets the curve again.
\end{enumerate}
\end{exercise}

\noindent {\bf Hint}: For the first part, observe that if a line $L \subset \PP^3$ were tangent to $C$ at two points $p$ and $q \in C$, we could write the divisor $D$ as $D = 2p + 2q + E$, with $h^0(E) \geq 2$; use Martens' theorem to show that the family of such divisors has dimension strictly less than $g$. 
The second part follows similarly from the strong form of Marten's theorem.

%\begin{exercise}\label{g+1 theorem}($g+1$-theorem)
%Show that if $D$ is a general divisor of degree $g+1$ on a smooth projective curve of genus $g$,  then $|D|$ defines a morphism to $\PP^1$ such that
%all ramification points are simple (2 branches coming together) and no two ramification points have the same 
%image. 
%
%Hint: The linear divisors of degree $g+1$ that fail to satisfy this condition are either of the form
%$D'+3p$ or $D''+2p+2q$; and such divisors form only  $g-1$ dimensional families.
%\end{exercise}

\input footer.tex