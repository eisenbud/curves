%header and footer for separate chapter files

\ifx\whole\undefined
\documentclass[12pt, leqno]{book}
\usepackage{graphicx}
\input style-for-curves.sty
\usepackage{hyperref}
\usepackage{showkeys} %This shows the labels.
%\usepackage{SLAG,msribib,local}
%\usepackage{amsmath,amscd,amsthm,amssymb,amsxtra,latexsym,epsfig,epic,graphics}
%\usepackage[matrix,arrow,curve]{xy}
%\usepackage{graphicx}
%\usepackage{diagrams}
%
%%\usepackage{amsrefs}
%%%%%%%%%%%%%%%%%%%%%%%%%%%%%%%%%%%%%%%%%%
%%\textwidth16cm
%%\textheight20cm
%%\topmargin-2cm
%\oddsidemargin.8cm
%\evensidemargin1cm
%
%%%%%%Definitions
%\input preamble.tex
%\input style-for-curves.sty
%\def\TU{{\bf U}}
%\def\AA{{\mathbb A}}
%\def\BB{{\mathbb B}}
%\def\CC{{\mathbb C}}
%\def\QQ{{\mathbb Q}}
%\def\RR{{\mathbb R}}
%\def\facet{{\bf facet}}
%\def\image{{\rm image}}
%\def\cE{{\cal E}}
%\def\cF{{\cal F}}
%\def\cG{{\cal G}}
%\def\cH{{\cal H}}
%\def\cHom{{{\cal H}om}}
%\def\h{{\rm h}}
% \def\bs{{Boij-S\"oderberg{} }}
%
%\makeatletter
%\def\Ddots{\mathinner{\mkern1mu\raise\p@
%\vbox{\kern7\p@\hbox{.}}\mkern2mu
%\raise4\p@\hbox{.}\mkern2mu\raise7\p@\hbox{.}\mkern1mu}}
%\makeatother

%%
%\pagestyle{myheadings}

%\input style-for-curves.tex
%\documentclass{cambridge7A}
%\usepackage{hatcher_revised} 
%\usepackage{3264}
   
\errorcontextlines=1000
%\usepackage{makeidx}
\let\see\relax
\usepackage{makeidx}
\makeindex
% \index{word} in the doc; \index{variety!algebraic} gives variety, algebraic
% PUT a % after each \index{***}

\overfullrule=5pt
\catcode`\@\active
\def@{\mskip1.5mu} %produce a small space in math with an @

\title{Personalities of Curves}
\author{\copyright David Eisenbud and Joe Harris}
%%\includeonly{%
%0-intro,01-ChowRingDogma,02-FirstExamples,03-Grassmannians,04-GeneralGrassmannians
%,05-VectorBundlesAndChernClasses,06-LinesOnHypersurfaces,07-SingularElementsOfLinearSeries,
%08-ParameterSpaces,
%bib
%}

\date{\today}
%%\date{}
%\title{Curves}
%%{\normalsize ***Preliminary Version***}} 
%\author{David Eisenbud and Joe Harris }
%
%\begin{document}

\begin{document}
\maketitle

\pagenumbering{roman}
\setcounter{page}{5}
%\begin{5}
%\end{5}
\pagenumbering{arabic}
\tableofcontents
\fi



\chapter{Linear Systems exercises}\label{linear systems exercises}

In the following series of exercises, we are going to be working with smooth projective curves associated to a given affine curve $C^\circ := V(f(x,y)) \subset \AA^2$; this is the unique smooth projective curve containing the normalization of $C^\circ$ as a Zariski dense open subset.

\begin{exercise}
Let $C$ be the smooth projective curve associated to the affine plane curve $y^3 +x^3 = 1$, and let $\pi : C \to \PP^1$ be the map given by the rational function $x$.
\begin{enumerate}
\item Find the branch points and ramification points of $\pi$, and deduce that the genus of $C$ is 1.
\item For any two points $p, q \in C$ find the complete linear series $|p+q|$.
\item Find the (unique) map $\eta : C \to \PP^1$ of degree 2 such that $\eta((1,0)) = \eta((0,1))$, and determine the ramification points of $\eta$.
\item Show that $C$ is isomorphic to the smooth projective curve associated to the affine plane curve $y^2 +x^3 = 1$.
\end{enumerate}
\end{exercise}

For the next three exercises, let $C^\circ$ be the affine plane curve given as the zero locus of $y^2 - x^6 +1$, and let $C$ be the corresponding smooth projective curve. Note that the map $C^\circ \to \AA^1$ given by the projection $(x,y) \mapsto x$ extends to a map $\pi : C \to \PP^1$, expressing $C$ as a 2-sheeted cover of $\PP^1$ branched over the points $1, \zeta, \dots, \zeta^5$, where $\zeta$ is any primitive 6th root of unity. Let $r_\alpha$ be the point $(\zeta^\alpha, 0) \in C^\circ \subset C$, and  let $p$ and $q \in C$ be the two points lying over the point $\infty \in \PP^1$.

\begin{exercise}\label{hyperelliptic curve 1}
What is the genus of $C$?
\end{exercise}

\begin{exercise} With $C$ as in Exercise~\ref{hyperelliptic curve 1}:
\begin{enumerate}
\item Show that
$$
p+q \sim 2r_\alpha \quad \text{and} \quad \sum_{\alpha = 0}^5 r_\alpha \sim 3p+3q.
$$

\item Find the vector space $H^0(\cO_C(D))$ where $D = r_0 + r_2 + r_4$, and find the (unique) divisor $E$ on $C$ such that $E + r_1 \sim r_0 + r_2 + r_4$.

\end{enumerate}

Hint: for the first part, consider the divisors of the rational functions $x-\zeta^\alpha$ and $y$. For the second, observe that the rational function 
$$
f(x,y) := \frac{y}{x^3-1} \in \cO_C(D)
$$
 and use the Riemann-Roch theorem to conclude that $\{1,f\}$ is a basis for $H^0(\cO_C(D))$.
\end{exercise}


\begin{exercise}
With $C$ as in Exercise~\ref{hyperelliptic curve 1}:
Let $D$ be the divisor $D = p + q + r_0 + r_3$
\begin{enumerate}
\item Find the vector space $H^0(\cO_C(D))$.
\item Describe the map $\phi_{|D|} : C \to \PP^2$.
\item Find the equation of the image curve $\phi_{|D|}(C) \subset \PP^2$, and describe its singularities.
\end{enumerate}
\end{exercise}

The following exercises deal with the  
 smooth projective curve $C$ associated to the affine curve $y^3 = x^5 -1$. Note that the map $\pi : C \to \PP^1$ given by the function $x$ expresses $C$ as a cyclic, 3-sheeted cover of $\PP^1$, branched over the 5th roots of unity and the point at $\infty$. By way of notation, if we take $\eta = e^{2\pi i/5}$ a primitive 5th root of unity, we'll denote by $r_\alpha$ the point $(\eta^\alpha, 0) \in C$ lying over $\eta^\alpha$, and by $p$ the point lying over $\infty \in \PP^1$.


\begin{exercise}
\begin{enumerate}
\item Verify that there is indeed a unique point $p \in C$ lying over $\infty \in \PP^1$, and the map has ramification index 2 at $p$. 
\item Show that the genus of $C$ is 4.
\item Establish the linear equivalences
$$
3p \sim 3r_\alpha \quad \text{and} \quad r_1+ \dots + r_5 \sim 5p.
$$
\item Find a basis for the space $H^0(K_C)$ of regular differentials on $C$.
\item Show that $C$ is not hyperelliptic.
\item Describe the canonical map $\phi_K : C \to \PP^3$ and find the equations of the image.
\item Let $D$ be the divisor $D = r_1+\dots+r_5$. Show that $h^0(K_C(-D)) = 1$; deduce that $r(D) = 2$, and find a basis for $H^0(\cO_C(D))$.
\item If $E = 3p$, show that $r(E) = 1$; that $|E|$ is the unique $g^1_3$ on $C$ and that $2E \sim K$.
\end{enumerate}
\end{exercise}

Hint(s): For the first part, the easiest way is to describe the monodromy of the cover $C \to \PP^1$ around the point at $\infty$; given the first part, the second follows from Riemann-Hurwitz. For the third, consider the rational functions on $C$ given by $x - \eta^\alpha$ and $y$. For the fourth part, start with the rational differential $\omega = \pi^*dx$ and find rational functions $f$ on $C$ so that $f\omega$ is regular.

For the fifth part, you can use your basis for $H^0(K_C)$ to show that the canonical map is one-to-one; or you can use Exercise~\ref{gonality exclusion} to observe that a curve of genus 4 cannot be simultaneously hyperelliptic and a three-sheeted cover of $\PP^1$. For the sixth, the ideal of the canonical curve is generated by a quadric and a cubic.

\begin{exercise}\label{gonality exclusion}
Show that a curve of genus $g \geq 3$ cannot be simultaneously hyperelliptic and a three-sheeted cover of $\PP^1$.
\end{exercise}

Hint: If $\mu : C \to \PP^1$ and $\nu : C \to \PP^1$ have degrees 2 and 3, the product map $\mu \times \nu : C \to \PP^1 \times \PP^1$ is a birational embedding; now use the genus formula for curves on $\PP^1 \times \PP^1$.

%%%%%%%%

\begin{exercise}\label{here there be basepoints}
 Show that there is no non-constant morphism $\PP^r\to \PP^s$ when $s<r$ by showing that any nontrivial linear
 series of dimension $<r$ on $\PP^r$ has a non-empty base locus.
 
 Hint: Use the principal ideal theorem.
\end{exercise}
 
 
\begin{exercise}
Extend the statement of Proposition~\ref{very ample} to incomplete linear series; that is, prove that the morphism associated to a linear series $(\cL, V)$ is an embedding iff
$$
\dim\big( V \cap H^0(\cL(-p-q))\big) = \dim V - 2 \quad \forall p, q \in C.
$$
\end{exercise}

\begin{exercise}
An automorphism of $\PP^r$ takes hyperplanes to hyperplanes. Deduce that it is given by the linear series
$\sV = (\sO_{\PP^r}(1), H^0(\sO_{\PP^r}(1)))$, and use this to show that $\Aut \PP^r = PGL(r+1)$. 
\end{exercise}

\begin{exercise}\label{projective automorphism}
Suppose that $C\subset \PP^r$ is embedded by a complete linear series. Show that an automorphism $\phi$ of $C$ is induced by an automorphism of $\PP^r$ if and only if $\phi$ preserves the invertible sheaf $\cO_{C}(1)$  in the sense that $\phi^*(\cO_{C}(1)) = \cO_{C}(1)$.
\end{exercise}

\begin{exercise}\label{planar triple pt}
Show that normalization of the affine planar triple point $xy(x-y) = 0$ in $\AA^2$ is the disjoint union of three
affine lines, $\Spec(\CC[x] \times \CC[y]\times \CC[z])$. Compute the linear conditions on the values and derivatives of three polynomial functions $f,g,h$ defined on
these three lines that they ``descend'' to give a well-defined function on the planar triple point.

Hint: Show first that $\CC[x] \times \CC[y]\times \CC[z]$ is integrally closed, and then that it is a finitely generated module over $\CC[x,y,z]/(xy(x-y))$.
Express a general element of $\CC[x,y,z]/(xy(x-y))$ as a triple of (finite) power series.
\end{exercise}

\begin{exercise} Generalize the results of Exercise~\ref{planar triple pt} to the case of a plane curve with $n$ pairwise
transverse branches, called an \emph{ordinary $n$-fold point}.
\end{exercise}

\begin{exercise}
Let $p \in C$ be a singular point of a reduced curve $C$. Show that if $\delta_p = 1$, then $p$ must be either a node or a cusp.
\end{exercise}

\begin{exercise}\label{normality of RNC}
 Show that $\CC[s^d,s^{d-1}t,\dots, t^d]$ is normal (i.e., integrally closed).
 
 Hint: Its integral closure must be
 contained in $\CC[s,t]$ and if $f$ is any polynomial
 in the integral closure then the homogeneous components of $f$ are also in the integral closure.
\end{exercise}





\input footer.tex