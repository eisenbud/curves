%header and footer for separate chapter files

\ifx\whole\undefined
\documentclass[12pt, leqno]{book}
\usepackage{graphicx}
\input style-for-curves.sty
\usepackage{hyperref}
\usepackage{showkeys} %This shows the labels.
%\usepackage{SLAG,msribib,local}
%\usepackage{amsmath,amscd,amsthm,amssymb,amsxtra,latexsym,epsfig,epic,graphics}
%\usepackage[matrix,arrow,curve]{xy}
%\usepackage{graphicx}
%\usepackage{diagrams}
%
%%\usepackage{amsrefs}
%%%%%%%%%%%%%%%%%%%%%%%%%%%%%%%%%%%%%%%%%%
%%\textwidth16cm
%%\textheight20cm
%%\topmargin-2cm
%\oddsidemargin.8cm
%\evensidemargin1cm
%
%%%%%%Definitions
%\input preamble.tex
%\input style-for-curves.sty
%\def\TU{{\bf U}}
%\def\AA{{\mathbb A}}
%\def\BB{{\mathbb B}}
%\def\CC{{\mathbb C}}
%\def\QQ{{\mathbb Q}}
%\def\RR{{\mathbb R}}
%\def\facet{{\bf facet}}
%\def\image{{\rm image}}
%\def\cE{{\cal E}}
%\def\cF{{\cal F}}
%\def\cG{{\cal G}}
%\def\cH{{\cal H}}
%\def\cHom{{{\cal H}om}}
%\def\h{{\rm h}}
% \def\bs{{Boij-S\"oderberg{} }}
%
%\makeatletter
%\def\Ddots{\mathinner{\mkern1mu\raise\p@
%\vbox{\kern7\p@\hbox{.}}\mkern2mu
%\raise4\p@\hbox{.}\mkern2mu\raise7\p@\hbox{.}\mkern1mu}}
%\makeatother

%%
%\pagestyle{myheadings}

%\input style-for-curves.tex
%\documentclass{cambridge7A}
%\usepackage{hatcher_revised} 
%\usepackage{3264}
   
\errorcontextlines=1000
%\usepackage{makeidx}
\let\see\relax
\usepackage{makeidx}
\makeindex
% \index{word} in the doc; \index{variety!algebraic} gives variety, algebraic
% PUT a % after each \index{***}

\overfullrule=5pt
\catcode`\@\active
\def@{\mskip1.5mu} %produce a small space in math with an @

\title{Personalities of Curves}
\author{\copyright David Eisenbud and Joe Harris}
%%\includeonly{%
%0-intro,01-ChowRingDogma,02-FirstExamples,03-Grassmannians,04-GeneralGrassmannians
%,05-VectorBundlesAndChernClasses,06-LinesOnHypersurfaces,07-SingularElementsOfLinearSeries,
%08-ParameterSpaces,
%bib
%}

\date{\today}
%%\date{}
%\title{Curves}
%%{\normalsize ***Preliminary Version***}} 
%\author{David Eisenbud and Joe Harris }
%
%\begin{document}

\begin{document}
\maketitle

\pagenumbering{roman}
\setcounter{page}{5}
%\begin{5}
%\end{5}
\pagenumbering{arabic}
\tableofcontents
\fi


\setlength{\parskip}{5pt}

\addtocounter{chapter}{-1}
\chapter{Introduction}
\label{IntroChapter}

\begin{quote}\it{I'm very well acquainted, too, with matters mathematical,\\
I understand equations, both the simple and quadratical,\\
About binomial theorem I am teeming with a lot o' news,\\
With many cheerful facts about the square of the hypotenuse.}
\end{quote}

---Gilbert and Sullivan, Pirates of Penzance, Major General's Song


\begin{quote}
Be simple by being concrete. Listeners are prepared to
accept unstated (but hinted) generalizations much more than they are able, on the spur of the moment, to
decode a precisely stated abstraction and to re-invent the special cases that motivated it in the first place. 
\end{quote}

--Paul Halmos, How to Talk Mathematics
\section{Why you want to read this book}

Algebraic geometry is an old subject. Descartes' introduction of coordinates in the plane and in space made it possible to relate the algebra of polynomials to the geometry of their zero loci. Gauss' proof of the Fundamental Theorem of Algebra showed that the degree of a polynomial, an algebraic invariant, was equal to the number of roots counted with mulitplicity, a geometric invariant. Mathematicians have been doing what is recognizably algebraic geometry for more than two centuries.

Within algebraic geometry, the study of algebraic curves is the oldest topic. Newton already classified all the possible types of real affine cubics. By the middle of
the 19th century a rich theory of curves in the complex projective plane was a central topic, overturned by Riemann's work in mid-century---what became the theory of Riemann surfaces, which introduced techniques of complex analysis into the field. This was  taken up and made algebraic as a theory of plane curves by Alexander Brill, Max Noether, F. S. Macaulay and many others. By the end of the century Halphen and others were interested in the classification of space curves as well. 

The development continues unabated today: in the second half of the 20th century Grothendieck's foundations lead to the solution of many classical problems and, in particular, to a firm foundation for the theory of moduli, allowing mathematicians to exploit the fact that algebraic varieties generally come in families parameterized by other algebraic varieties---something that we will return to often in this book. Algebraic Geometry has merged more and more with number theory, and the moduli space of curves plays an important role in String Theory, born from the needs of physicists.

For these reasons, the subject of algebraic curves is one of the richest in algebraic geometry, if not in all of mathematics. Examples abound; if you want to know whether a conjecture is plausible, you can generally find well-understood special cases on which to test it. Some of the fundamental constructions of algebraic geometry, like the construction of moduli spaces and their description, can be carried out in the setting of algebraic curves with a degree of precision and detail far beyond what has been possible in higher dimensions. 

By way of one example, consider the relatively simple cases of the geometry of plane curves of degrees 3 and 4: 

Every smooth plane cubic has exactly 9 flexes---points where the tangent line has contact of order 3 with the curves---and these points form a remarkable configuration; they're the only known example of a finite, non-colinear set of points in the plane such that the line joining any two contains a third. Generalizing the notion of the flexes of a plane cubic leads us to the subject of inflectionary behavior of linear series on curves in general, which arose separately in the theory of Weierstrass points on Riemann surfaces, and  has become a powerful tool. 

Every smooth quartic curve has exactly 28 bitangents, also forming a beautiful and mysterious configuration;  in \cite{MR0115124}, first published in 1837, Salmon computed the number (315) of 4-tuples of bitangents whose eight points of tangency lie on a conic. The extension of this idea to curves of higher genus leads us to the rich theory of theta-characteristics, bringing together algebra (in the form of a bilinear pairing on the points of order 2 in the Jacobian of a curve), analysis (in the form of theta-functions) and of course projective geometry.

The richest subject in what is arguably the richest branch of mathematics\footnote{Number theorists may quibble}---of course you want to read this book! 

\section{Why we wrote this book}

The wealth of beauty, both in theory and in examples, certainly makes the study of algebraic curves an attractive prospect. But it comes at a price: to absorb in detail all the things we've learned over the centuries about algebraic curves would take years, if not decades. This is, in essence, the conundrum facing anyone who undertakes to write a book on the subject: how to convey the wealth of information  (and the many many ways in which our knowledge is incomplete) without writing an encyclopedia.

\begin{quote}
\small\sf
We are dealing here with a fundamental and almost paradoxical difficulty. Stated briefly, it is that learning is sequential but knowledge is not. A branch of mathematics... consists of an intricate network  of interrelated facts, each of which contributes to the understanding of those around it. When confronted with this network for the first time, we are forced to follow a particular path, which involves a somewhat arbitrary ordering of the facts.
--Robert Osserman.
\end{quote}


For better or worse, this is our attempt to do  that. Our intended audience is a graduate student considering working in the field of algebraic curves, or a researcher in a related field whose work has led them to questions about algebraic curves. Our goal is to equip the reader with the understanding of both the techniques and the state of our knowledge necessary to read the current literature and work on open problems.

\section{What's so Practical?}

We love T. S. Eliot's book, \emph{Old Possum's Book of Practical Cats}, a collection of light verse introducing a collection of cats with distinct personalities. In a way, our approach reflects Eliot's: different areas and questions under the general umbrella of algebraic curve theory have distinct aspects---personalities, if you will.

\section{What's in this book}

\begin{quote}
\small\sf
We are dealing here with a fundamental and almost paradoxical difficulty. Stated briefly, it is that learning is sequential but knowledge is not. A branch of mathematics... consists of an intricate network network of interrelated facts, each of which contributes to the understanding of those around it. When confronted with this network for the first time, we are forced to follow a particular path, which involves a somewhat arbitrary ordering of the facts.

--Robert Osserman.

\end{quote}

We start in Chapter~\ref{linear systems} by laying out the basic objects we'll be dealing with through the remainder of the book: invertible sheaves, and linear systems, the central object of algebraic curve theory, and proving or sketching ideas about them that we will use. This chapter is intended as a review for someone who has been exposed already to material roughly equivalent to Chapter IV of \cite{H}. Thereafter, we alternate between chapters focussed on special cases and chapters developing more of the theory. Thus, for example, Chapter~\ref{genus 0 and 1 chapter} describes the geometry of curves of genus 0 and 1 in projective space. When we get to genus 2, however, a fundamental shift occurs: it's no longer the case that all invertible sheaves of a given degree on a curve $C$ of genus $g \geq 2$ are congruent mod the automorphism group of the curve, and so we first introduce the Jacaobian and the Picard varieties $\Pic_d(C)$ parametrizing isomorphism classes of line bundles in Chapter~\ref{new Jacobians chapter}.

Equipped with (some) knowledge of the Picard varieties, we proceed in Chapter~\ref{genus 2 and 3 chapter} to study curves of genus 2 and 3, describing in particular how the geometry of maps of our curve to projective space given by sections of line bundles depends on the bundle. 

Families of similar algebro-geometric objects are often naturally parameterized by algebraic varieties. This idea goes back to the beginning of algebraic geometry and the family of plane curves of degree $d$, but was only fully clarified in the work of Grothendieck, Deligne and Mumford, and we spend Chapter~\ref{ModuliChapter} explaining this modern understanding through some central examples.

After this, it's back to examples: in Chapter~\ref{genus 4, 5 Chapter} we analyze in some detail the geometry of curves of genus 4 and 5. We are able to answer all our questions about the geometry of these curves by ad-hoc methods. But to do the same  in genus 6 we need to quote the Brill-Noether theorem to complete our analysis, so Chapter~\ref{Brill-Noether} takes up the general
question: ``What linear series exist on curves of genus $g$?" We give three possible interpretations of this question, and the answers to each: Clifford's theorem, the Castelnuovo bound, and the Brill-Noether theorem, the last of which we describe and state but do not prove.

With the Brill-Noether statements as a tool, we study curves of genus 6 in Chapter~\ref{genus 6 chapter}. 
Then in Chapter~\ref{inflections chapter}, we introduce the theory of \emph{inflection points} of linear series, and use this to give a proof of (half of) Brill-Noether.


Next, in Chapter~\ref{PlaneCurvesChapter} we take up two topics within the vast subject  of plane curves. the geometry of (possibly singular) plane curves, and how it can be used in the study of smooth curves. In the late 19th century, before the advent of abstract algebraic varieties, people thought of a smooth curve $C$ of genus $g$  as the normalization of a plane curve $C_0$, and the geometry of $C$ was worked out via this plane model $C_0$. Even today, that approach offers insight into the geometry of curves---we describe, for example, an explicit algorithm for finding a complete linear system on a curve, and we give a description of the 
(small) Severi Variety---the collection of plane curves with $g$ nodes and no other singular points.

In Chapter~\ref{ScrollsChapter} we systematically describe a construction that has already appeared in different guises several times in the book: rational normal scrolls and the curves that lie on them. Chapter~\ref{SyzygiesChapter} presents some aspects of the theory of syzygies of the homogeneous ideals of curves and the the famous---and as of this writing open---conjecture of Mark Green that connects these to the Clifford index and the theory of rational normal scrolls. A table reproduced from work of Frank-Olaf Schreyer shows the sensitivity of the numerical information in the free resolution of a canonical curve to questions of the existence of special linear series on the curve.

 Finally, Chapters~\ref{HilbertSchemesChapter} and~\ref{HilbertSchemesCounterexamplesChapter} represent in some ways a culmination of the book. Both chapters are concerned with \emph{Hilbert schemes}, schemes $\cH_{g,r,d}$ that parametrize curves of given degree $d$ and genus $g$ in projective space $\PP^r$. In Chapter~\ref{HilbertSchemesChapter} we work out a number of examples, on the basis of which we arrive at an ``expected behavior" of the Hilbert scheme, which indeed holds in a certain range of $(g,r,d)$. Then, in Chapter~\ref{HilbertSchemesCounterexamplesChapter}, we give a series of counterexamples showing that the expected behavior does \emph{not} hold for all $(g,r,d)$, and describe some of the open problems and conjectures surrounding this question.



\subsection{Relation of this book to Other texts} 
Chapter IV of~\cite{Hartshorne1977} has a similar flavor to that of this book, and indeed we will take a familiarity with the material of that text as effectively a pre-requisite. A beautiful (and brief) account of a number of topics in a style we particularly admire is found in \cite{MumfordCJ}.

A far more extensive treatment, partially overlapping that of the present book and containing many other topics, can be found in the beautiful and encyclopaedic~\cite{ACGH} and~\cite{ACG}, and yet even these works do not cover all the major topics in the field. 
One of the topics we do not cover, but which can be found in \cite{ACG}, is the construction of the moduli space of curves. Many  ideas  about it can be found in~\cite{HarrisMorrison1998}, though  that is not a complete account. 

There are more elementary accounts of some of our material, in~\cite{Fulton1989} and \cite{Walker1978} (who goes farther than we do into local resolution of singularities) as well as~\cite{Griffiths-curves}, and a comprehensive treatment of the local theory of plane curves and their singularities in \cite{Brieskorn1986}. The topological questions there are developed in different directions in \cite{MR0239612} %Milnor
 and \cite{MR817982}%Eisenbud-Neumann}. 
 An interesting collection of topics is presented in~\cite{Clemens-Scrapbook}.

 The Riemann surface point of view is well represented in the books \cite{Forster} \cite{Gunning}, \cite{Gunning-2} \cite{Kirwan}\cite{Miranda}. 


\section{Prerequisites, notation and conventions}

We will assume familiarity, though not much technical information, with coherent sheaves and their cohomology. We review some of this theory, and that of divisors on projective 
varieties, in Chapter~\ref{linear series}. It is probably enough if the reader can write down $H^1$ and $H^0$ and exact sequences without blushing, and in any case we redo some of what is in~\cite[Chapter IV]{Hartshorne1977}  Chapter IV in our language. We also use a few basic facts about surfaces from the first two sections of~\cite[Chapter IV]{Hartshorne1977}.

The reader should also be familiar with the (Krull) dimension of rings and varieties, and their primary decomposition at the level of \cite{Atiyah-MacDonald}, including localization and with Lasker's often mis-attributed theorem that complete intersections are unmixed (the version for forms in 3 variables was proven by
Max Noether, and is now often called the $AF+BG$ theorem.) The condition that the homogeneous coordinate
rings of (some) curves are arithmetically Cohen-Macaulay is represented primarily by the completeness of linear series of hypersurface sections; the connection is briefly explained in Chapter~\ref{linear series}.




%\begin{quote}
%\small\sf
%We are all familiar with the after-the-fact tone---weary, self-justificatory, aggrieved, apologetic---shared by ship captains appearing before boards of inquiry to explain how they came to run their vessels aground, and by authors composing forewords.
%
%--John Lanchester 
%\bigskip
%
%\end{quote}
%


%footer for separate chapter files

\ifx\whole\undefined
%\makeatletter\def\@biblabel#1{#1]}\makeatother
\makeatletter \def\@biblabel#1{\ignorespaces} \makeatother
\bibliographystyle{msribib}
\bibliography{slag}

%%%% EXPLANATIONS:

% f and n
% some authors have all works collected at the end

\begingroup
%\catcode`\^\active
%if ^ is followed by 
% 1:  print f, gobble the following ^ and the next character
% 0:  print n, gobble the following ^
% any other letter: normal subscript
%\makeatletter
%\def^#1{\ifx1#1f\expandafter\@gobbletwo\else
%        \ifx0#1n\expandafter\expandafter\expandafter\@gobble
%        \else\sp{#1}\fi\fi}
%\makeatother
\let\moreadhoc\relax
\def\indexintro{%An author's cited works appear at the end of the
%author's entry; for conventions
%see the List of Citations on page~\pageref{loc}.  
%\smallbreak\noindent
%The letter `f' after a page number indicates a figure, `n' a footnote.
}
\printindex[gen]
\endgroup % end of \catcode
%requires makeindex
\end{document}
\else
\fi


