%header and footer for separate chapter files

\ifx\whole\undefined
\documentclass[12pt, leqno]{book}
\usepackage{graphicx}
\input style-for-curves.sty
\usepackage{hyperref}
\usepackage{showkeys} %This shows the labels.
%\usepackage{SLAG,msribib,local}
%\usepackage{amsmath,amscd,amsthm,amssymb,amsxtra,latexsym,epsfig,epic,graphics}
%\usepackage[matrix,arrow,curve]{xy}
%\usepackage{graphicx}
%\usepackage{diagrams}
%
%%\usepackage{amsrefs}
%%%%%%%%%%%%%%%%%%%%%%%%%%%%%%%%%%%%%%%%%%
%%\textwidth16cm
%%\textheight20cm
%%\topmargin-2cm
%\oddsidemargin.8cm
%\evensidemargin1cm
%
%%%%%%Definitions
%\input preamble.tex
%\input style-for-curves.sty
%\def\TU{{\bf U}}
%\def\AA{{\mathbb A}}
%\def\BB{{\mathbb B}}
%\def\CC{{\mathbb C}}
%\def\QQ{{\mathbb Q}}
%\def\RR{{\mathbb R}}
%\def\facet{{\bf facet}}
%\def\image{{\rm image}}
%\def\cE{{\cal E}}
%\def\cF{{\cal F}}
%\def\cG{{\cal G}}
%\def\cH{{\cal H}}
%\def\cHom{{{\cal H}om}}
%\def\h{{\rm h}}
% \def\bs{{Boij-S\"oderberg{} }}
%
%\makeatletter
%\def\Ddots{\mathinner{\mkern1mu\raise\p@
%\vbox{\kern7\p@\hbox{.}}\mkern2mu
%\raise4\p@\hbox{.}\mkern2mu\raise7\p@\hbox{.}\mkern1mu}}
%\makeatother

%%
%\pagestyle{myheadings}

%\input style-for-curves.tex
%\documentclass{cambridge7A}
%\usepackage{hatcher_revised} 
%\usepackage{3264}
   
\errorcontextlines=1000
%\usepackage{makeidx}
\let\see\relax
\usepackage{makeidx}
\makeindex
% \index{word} in the doc; \index{variety!algebraic} gives variety, algebraic
% PUT a % after each \index{***}

\overfullrule=5pt
\catcode`\@\active
\def@{\mskip1.5mu} %produce a small space in math with an @

\title{Personalities of Curves}
\author{\copyright David Eisenbud and Joe Harris}
%%\includeonly{%
%0-intro,01-ChowRingDogma,02-FirstExamples,03-Grassmannians,04-GeneralGrassmannians
%,05-VectorBundlesAndChernClasses,06-LinesOnHypersurfaces,07-SingularElementsOfLinearSeries,
%08-ParameterSpaces,
%bib
%}

\date{\today}
%%\date{}
%\title{Curves}
%%{\normalsize ***Preliminary Version***}} 
%\author{David Eisenbud and Joe Harris }
%
%\begin{document}

\begin{document}
\maketitle

\pagenumbering{roman}
\setcounter{page}{5}
%\begin{5}
%\end{5}
\pagenumbering{arabic}
\tableofcontents
\fi


\addtocounter{chapter}{-1}
\chapter{Introduction}
\label{IntroChapter}

\section{Why you want to read this book}

Algebraic geometry is an old subject. 
\blue{Descartes'}
\index{Descartes, Ren\'e}%
introduction in 1637 of coordinates in the plane and in space made it
possible to relate the algebra of polynomials to the geometry of their
zero loci. 
\blue{Gauss'} 
\index{Gauss, Leonhard}%
proof of the 
\blue{Fundamental Theorem of Algebra}
\index{Fundamental Theorem of Algebra}\index{theorem!fundamental of algebra}%
showed
that the degree of a polynomial, an algebraic invariant, was equal to
the number of roots counted with multiplicity, a geometric invariant.
Mathematicians have been doing what is recognizably algebraic geometry
for more than two centuries.
\index{Fundamental Theorem of Algebra}

Within algebraic geometry, the study of algebraic curves is the oldest topic. Newton already classified all the possible types of real affine cubics. By the middle of
\index{Brill, Alexander}%
\index{Noether, Max}%
\index{Macaulay, F. S.}%
the 
\redden{nineteenth}
%\marginpar{Some changes are marked only by redness, without notes}%
century a rich theory of curves in the complex projective plane was a
central topic, overturned by Riemann's work in mid-century\emdash what
became the theory of Riemann surfaces, which introduced new techniques
of complex analysis into the field. This was  taken up and made
%\marginpar{\redden{Should full ``first'' names be given in the index even when you use initials in the text, as with Macaulay?}}
algebraic as a theory of plane curves by Alexander Brill, Max Noether,
F. S. Macaulay and many others. By the end of the century 
\blue{Halphen} 
and
others were interested in the classification of space curves as well.
The interested reader will find more about the 
\index{Halphen, Georges-Henri}%
early work on algebraic curves in the Appendix to this book, written by historian Jeremy Gray.

The development continues unabated today: in the second half of the
\redden{twentieth} 
century 
\index{Grothendieck, Alexander}
\blue{Grothendieck's}
foundations led to the solution of many
classical problems and, in particular, to a firm foundation for the
theory of moduli, allowing mathematicians to exploit the fact that
algebraic varieties generally come in families parametrized by other
algebraic varieties\emdash something that we will return to often in
this book. Algebraic geometry has merged more and more with number
theory, and the moduli space of curves plays an important role in
string theory, responding to the needs of physicists.

For these reasons, the subject of algebraic curves is one of the richest in algebraic geometry, if not in all of mathematics. If you want to know whether a conjecture is plausible, you can generally find well-understood special cases on which to test it. Some of the fundamental constructions of algebraic geometry, like the construction of moduli spaces and their description, can be carried out in the setting of algebraic curves with a degree of precision and detail far beyond what has been possible in higher dimensions. 

Already the geometry of plane curves of degrees 3 and 4 shows some of the promise of the subject:

Every smooth plane cubic has exactly 9 flexes\emdash points where the
tangent line has contact of order 3 with the curves\emdash and these
points form a remarkable configuration; they're the only known example
of a finite, noncolinear set of points in the plane such that the line
joining any two contains a third. Generalizing the notion of the
flexes of a plane cubic leads us to the subject of inflectionary
behavior of linear series on curves in general, which arose separately
in the theory of Weierstrass points on Riemann surfaces, and  has
become a powerful tool.
\index{Weierstrass points!on Riemann surfaces}%

Every smooth quartic curve has exactly 28 bitangents, also forming a beautiful and mysterious configuration; see 
\index{bitangent}%
\index{bitangent! of a smooth quartic curve}%
\index{Salmon, George}%
Figure~\ref{fig28bitangents}. 
\blue{Salmon}
\citeyear[p.~197]{Salmon1852} 
%\marginpar{Please check on Hathitrust (URL in the bib) if this page reference is the right one.  Also, I find no attestation for the year 1837 mentioned in this passage originally. Also, no point in citing Chelsea, I think}
computed the number (315) of 4-tuples of bitangents whose eight points of tangency lie on a conic. The extension of these ideas to curves of higher genus leads us to the rich theory of theta-characteristics, bringing together algebra (in the form of a bilinear pairing on the points of order 2 in the Jacobian of a curve), analysis (in the form of theta-functions) and of course projective geometry.

The richest subject in what is  the richest branch of
%\marginpar{\redden{I would drop the footnote\emdash ``arguably'' does the job}}%
mathematics\footnote{Number theorists may quibble\dots}\emdash of
course you want to read this book!  

\section{Why we wrote this book}

The wealth of beauty, both in theory and in examples, certainly makes the study of algebraic curves an attractive prospect. But it comes at a price: to absorb in detail all the things we've learned over the centuries about algebraic curves would take years, if not decades. This is, in essence, the conundrum facing anyone who undertakes to write a book on the subject: how to convey the wealth of information  (and the many many ways in which our knowledge is incomplete) without writing an encyclopedia. We have chosen to try to be useful and broad but not necessarily complete. 

When we introduce a technique or a construction without full proofs,
we do so as a ``cheerful fact:''%
\footnote{Hats off to the ``Many Cheerful Facts'' seminar run
by the
\redden{U.\,C.\,Berkeley} 
graduate students, which gave us this idea!}

\begin{quote}\it{I'm very well acquainted, too, with matters mathematical,\\
I understand equations, both the simple and quadratical,\\
About binomial theorem I am teeming with a lot o' news,\\
With many cheerful facts about the square of the hypotenuse.}
\index{Gilbert, W. S.}%
\index{Sullivan, Arthur}%
\index{{{\it Pirates of Penzance}}}
\end{quote}
\emdash Gilbert and Sullivan, {\it Pirates of Penzance}, Major General's Song

Our intended audience is a graduate student considering working in the field of algebraic curves, or a researcher in a related field whose work has led them to questions about algebraic curves. Our goal is to equip the reader with the understanding of both the techniques and the state of our knowledge necessary to read the current literature and work on open problems.

\section{What's with practice?}

\begin{quote}
Be simple by being concrete. Listeners are prepared to
accept unstated (but hinted) generalizations much more than they are able, on the spur of the moment, to
decode a precisely stated abstraction and to re-invent the special cases that motivated it in the first place. 
\end{quote}

--Paul Halmos, {\it How to Talk Mathematics}
\index{Halmos, Paul}%

This book aims to present those ideas and methods from the theory of  algebraic curves that  used \emph{in practice} by mathematicians working in a variety of fields of mathematical research.

Although mathematicians aspire to understand their subjects deeply, we feel that we learn in stages: in early stages we accept large and difficult results as black boxes and explore the rich examples that they yield. That is how we have tried to organize this book: 
We begin with two chapters that we hope will bridge the gap between first courses in algebraic geometry/commutative algebra at the level of Fulton's or  Reid's well-known books \cite{Fulton1989}, \cite{MR982494} and the professional language of invertible sheaves, cohomology and linear series. An ideal background would
be Hartshorne's book \cite{Hartshorne1977} or Vakil's notes \cite{Vakil-notes} but very much less will suffice if the reader is willing to accept some advanced ideas or look them up at leisure; we have tried to give precise references where this might be required. Subsequent chapters roughly alternate between expositions of basic techniques (partly without proofs) and families of examples, treated in detail. 

\section{What's in this book}
In organizing this book we faced a common problem of  mathematical exposition:
%\marginpar{\redden{This quote is felicitous but appeared in 3248 with the same effect. Is it worth repeating it here?}}
\begin{quote}
\small\sf
We are dealing here with a fundamental and almost paradoxical difficulty. Stated briefly, it is that learning is sequential but knowledge is not. A branch of mathematics\dots\ consists of an intricate network of interrelated facts, each of which contributes to the understanding of those around it. When confronted with this network for the first time, we are forced to follow a particular path, which involves a somewhat arbitrary ordering of the facts.

--Robert 
\blue{Osserman}
%\redden{{\it Poetry of the Universe}}
\citeyear{Poetry}
\index{Osserman, Robert}%
\end{quote}

In Chapters~\ref{linear series} and~\ref{RiemannRochChapter} we lay out the central objects of algebraic curve theory: invertible sheaves, linear systems, canonical sheaves and the Riemann--Roch theorem, as well as
Hurwitz's theorem,
the adjunction formula and  some elementary facts about the geometry of surfaces. We prove or sketch the
proofs of many of these basic results. These chapters may serve as review for someone who has been exposed already to material roughly equivalent to Chapter IV of \cite{Hartshorne1977}. 

Thereafter, we alternate between chapters focussed on special cases and chapters developing more of the theory. Chapter~\ref{genus 0 and 1 chapter}  describes the geometry of curves of genus 0  in projective space. We emphasize some of the things that make rational normal curves so special, and take the opportunity to introduce the conditions implying that a curve is \emph{arithmetically Cohen--Macaulay}. 

In Chapter~\ref{genus 1 chapter} We begin by explaining the
Riemann--Roch theorem and its consequences for smooth plane curves,  in
the style of the late 
nineteenth
century, computing the canonical series and showing algorithmically how to compute the complete linear series of effective divisors linearly equivalent to a given (not necessarily effective) divisor, and we use this 
to describe the low-degree linear series on curves of genus 1, and demonstrate how counts of parameters suggest the presence of a moduli space.

Because most curves cannot be represented as smooth plane curves, a general treatment requires dealing with
singular curves and their normalizations. This requires somewhat more algebraic technique, so we postpone the general case to Chapter~\ref{PlaneCurvesChapter}.  

In genus $\geq 2$ there is a fundamental shift: not all invertible sheaves of a given degree on a curve $C$ of genus $g \geq 2$ are congruent modulo the automorphism group of the curve.
It is a salient feature of algebraic geometry that families of similar algebro-geometric objects are often naturally parametrized by algebraic varieties. This idea goes back to the beginning of algebraic geometry and the family of plane curves of degree $d$, but was only fully clarified in the work of Grothendieck, Deligne and Mumford.
\index{Grothendieck, Alexander}%
\index{Deligne, Pierre}%
\index{Mumford, David}%

Chapter~\ref{Jacobians chapter} introduces our first examples of this phenomenon: the \emph{fine moduli spaces} of divisors on a curve, as well as the Jacobian, for which we give the classic analytic construction, and the Picard varieties $\Pic_d(C)$ parametrizing isomorphism classes of invertible sheaves. We also introduce the subvarieties $W^{r}_{d}(C)$ parametrizing invertible sheaves with many sections. 

 Since the Jacobian is irreducible, we can speak meaningfully of a \emph{general} invertible sheaf, and of the dimension of various families of special invertible sheaves. With this in hand we prove
that every curve of genus $g>1$ can be embedded in $\PP^{3}$ as a curve of degree $g+3$.

Equipped with information about the Jacobian, we proceed in Chapter~\ref{genus 2 and 3 chapter} to study curves of genus 2 and 3, describing in particular how the geometry of a map of the curve to projective space,  given by sections of an invertible sheaf, depends on the sheaf. We show how a hyperelliptic curve of genus $g$
can be described by a set of $2g+2$ points in $\PP^{1}$, and we describe the canonical maps. Again, this
suggests the existence of moduli spaces.

We spend Chapters~\ref{ModuliChapter} and~\ref{CurvesModuli chapter} on other moduli spaces as they appear in the world of curves, though we do not give complete proofs of all the assertions made. We start in Chapter~\ref{ModuliChapter} with some central examples of fine moduli spaces, primarily
the Hilbert scheme. Using properties of the Hilbert scheme we show that curves of genus $\geq 2$
can have only finitely many automorphisms, and more generally that there can only be finitely
many morphisms from one such curve to another.

The most interesting example of a moduli space, the moduli of smooth (or stable) curves of genus $g$, is not a fine moduli space, and we spend Chapter~\ref{CurvesModuliChapter} describing what it is and isn't. In that chapter we also describe the Hurwitz space of coverings and the Severi variety of nodal plane curves.

After this, it's back to examples: in Chapter~\ref{genus 4, 5 Chapter} we analyze aspects of the geometry of curves of genus 4 and 5.  

In the following two chapters we take up 
the properties of the points of a general hyperplane sections of a curve in projective space. 
In Chapter~\ref{linear general position chapter} we give Rathmann's proof that
the points of a general hyperplane section of a reduced irreducible curve are in linearly general position (independent of the characteristic).
Consequences include Castelnuovo's bound  on the maximal genus of curves of a given degree.
We show that the curves that achieve the maximum genus, called Castelnuovo curves,
are arithmetically Cohen--Macaulay. These include all plane curves, as well as
canonical curves and linearly normal curves of high degree compared to their genera. 

We also
use the linear general position result to prove the strong forms of Clifford's and Martens' theorems
on special linear series; to show that every curve has a nodal plane model; and to show that a 
general invertible sheaf  of degree $g+2$ on a curve of genus $g$  maps the curve to a nodal curve
in the plane\emdash unless the curve is hyperelliptic, in which case the image has a single multiple point of
multiplicity $g$.

In characteristic 0 an even stronger statement than linearly general position is true: the monodromy of the hyperplane divisors is the full symmetric group. We take up this
and some other monodromy questions in Chapter~\ref{uniform position} and prove consequences
for secant planes and for sums of linear series.

In Chapter~\ref{Brill-Noether} we return to the general question: ``What linear series exist on curves of genus $g$?" We give three possible interpretations of this question, and the answers to each: Clifford's theorem, the Castelnuovo bound, and the Brill--Noether theorem, postponing the proof of the latter. 
We apply the Brill--Noether result to explore  various special classes of curves of genus 6.

Chapter~\ref{inflections chapter} prepares for the proof of the Brill--Noether theorem 
with a discussion of inflection points of linear series, generalizing the flexes of plane curves. The Pl\"ucker formula
counts the number of inflection points, with appropriate weights. In this chapter we explain
the remarkable connection between the study of inflections on a rational curve and the Schubert calculus of cycles in the Grassmannian. In Chapter~\ref{BrillNoetherproofChapter} we use this material, together with a degeneration to cuspidal rational curves, to give a relatively short proof of the Brill--Noether theorem.

In Chapter~\ref{PlaneCurvesChapter} we take the classical point of view of Brill and Noether,
and describe an explicit algorithm for finding the complete linear series on a smooth curve, using a singular plane model. This involves the result classically known as the completeness of the adjoint linear series. For the general case we use a simple case of the theory of dualizing sheaves, introduced in more generality in the next chapter.

Returning to curves in $\PP^{3}$ in Chapter~\ref{LinkageChapter} we explain the theory of Hartshorne and Rao classifying curves up to linkage. This theory applies to all purely 1-dimensional subschemes of $\PP^{3}$, and to explain this
we detour to discuss dualizing sheaves and Grothendieck's $f^{!}$ operation.

Curves that lie on a quadric in $\PP^3$ are easy to understand. In Chapter~\ref{ScrollsChapter} we systematically describe a natural generalization: rational normal scrolls and the curves that lie on them. This includes all the Castelnuovo curves of high degree. 

Chapter~\ref{SyzygiesChapter} presents some aspects of the theory of syzygies of the homogeneous ideals of curves and the the famous\emdash and, as of this writing, open\emdash conjecture of Mark Green that connects syzygies of canonical curves to the Clifford index and the theory of rational normal scrolls. A table reproduced from work of Frank-Olaf Schreyer shows the sensitivity of the numerical information in the free resolution of a canonical curve to questions of the existence of special linear series on the curve.

Chapter~\ref{HilbertSchemesChapter} is in some ways the culmination of the book. It is concerned with \emph{Hilbert schemes}, schemes $\cH_{g,3,d}$ that parametrize smooth curves of genus $g$ and degree $d$  in $\PP^3` `$. We work out examples up to degree 7, define the ``principal component"\emdash the only one dominating the moduli space of curves\emdash and derive dimension estimates from deformation theory and from the Brill--Noether theorems
using most of the ideas we have introduced. 

The historical appendix written by Jeremy Gray, Chapter~\ref{Appendix-History},  is a survey of some of the work on
algebraic curves up to that of Brill and Noether, and we are grateful to him for allowing us to include it.

\subsection*{Exercises and hints}

There are many exercises, gathered at the end of the chapters. There is a separate document, available at ***.ams.org, that
collects all the exercises and includes hints to many of them.

\subsection*{Relation of this book to other texts} 
Chapter IV of \cite{Hartshorne1977} has a similar flavor to that of this book, and contains details of most of the 
results in our Chapters~\ref{linear series} and~\ref{RiemannRochChapter}. A beautiful (and brief) account of a number of topics in a style we particularly admire is found in \cite{MumfordCJ}.

A far more extensive treatment, partially overlapping that of the present book and containing many other topics, can be found in the important and encyclopaedic \cite{ACGH} and \cite{ACG}, and yet even these works do not cover all the major topics in the field. 
One of the topics we do not cover is the construction of the moduli space of curves, which can be found
in \cite{ACG}. Though not a complete account, \cite{HarrisMorrison1998} deals directly with this
topic. 

There are more elementary accounts of some of our material, in \cite{Fulton1989} and \cite{Walker1978} (who goes farther than we do into local resolution of singularities) as well as \cite{Griffiths-curves}. The book \cite{Kunz} treats plane curves and their normalizations via the theory of valuations, and contains a detailed  account of the ideas we treat in Chapter~\ref{PlaneCurvesChapter} from this more algebraic point of view.  There is a comprehensive treatment of the local topological theory of plane curves and their singularities in \cite{Brieskorn1986}. The topological questions there are developed in different directions in \cite{MR0239612} %Milnor
 and \cite{MR817982}. %Eisenbud-Neumann}. 
 An  idiosyncratic collection of interesting topics is presented in \cite{Clemens-Scrapbook}.

 The Riemann surface point of view is well represented in the books \cite{Forster}, \cite{Gunning}, \cite{Gunning-2}, \cite{Kirwan}, and \cite{Miranda}. 


\section{Prerequisites, notation and conventions}
The reader should be familiar with the 
\blue{(Krull)}
dimension of rings and
\index{Krull dimension}%
\index{primary decomposition}%
varieties, and their 
\blue{primary decomposition} 
at the level of
\cite{Atiyah-MacDonald}. Ideally the reader  will already have some
familiarity with the geometry of curves and surfaces 
at the level of 
Chapter IV and the beginning of Chapter V of
\cite{Hartshorne1977}, though our summary of the necessary material
in the first two chapters of this book may suffice for the intrepid.

Unless otherwise mentioned, we assume that the ground field is the field of complex numbers $\CC$, though much of what we do
could be done over any algebraically closed field.

\subsection*{Commutative algebra} 
All the rings we consider are commutative with unit and Noetherian.

Since we are working over  a field of characteristic 0, we use the terms smooth and nonsingular interchangeably when
referring to a point on a scheme.

Some results that we use:
 \begin{theorem}[Lasker's theorem]\label{Lasker}
If $f_1,\dots, f_c \subset \CC[x_0,\dots, x_n]$ generates an ideal of codimension $c$, then 
the ideal $(f_1,\dots, f_c)$ is unmixed (all its primary components have codimension $c$).
\index{Lasker's theorem}%
\end{theorem}

\begin{theorem}\label{finiteness of normalization}
 If $R$ is a domain that is a finitely generated algebra over a field or a localization of such an algebra, then the
normalization ($=$ integral closure) of $R$ is a finitely generated $R$-module.
If $R$ is 1-dimensional, then its normalization is nonsingular.
\end{theorem}

\subsection*{Projective geometry}
 
Schemes 
%\marginpar{dropped hyphen}
are assumed 
\blue{quasiprojective,}
\index{quasiprojective!variety}\index{variety!definition}%
\index{point!definition}%
and \emph{varieties} (including
curves) are 
\blue{reduced and irreducible schemes}
unless otherwise stated.
\index{scheme!reduced}%
\index{scheme!irreducible}%
\blue{``Points''}
will always be closed points unless we say otherwise. 
%\marginpar{dropped ``explicitly''\\\redden{About the bold text: I won't be able to tell when you're departing from these conditions.  Is there a smaller set of conditions that are always satisfied in your usage, and then you can explicit flag when additional conditions are needed}}
{\bf Note to the patient reader: We usually use the term curve to refer to a smooth irreducible projective purely 1-dimensional scheme, but in some of the later chapters we explicitly allow more general 1-dimensional schemes.}

Though we occasionally use the classical topology, 
the term ``open set'' refers to the 
\blue{Zariski topology}
unless otherwise stated.
\index{topology!Zariski}\index{Zariski topology}%

We adopt the 
\blue{Grothendieck convention}
that points of projective space 
\index{Grothendieck!convention (for projective space)}%
\index{projective!space!definition}%
\index{Grassmanian}%
\index{notation!$\PP(V)$, $\PP^r$}%
\index{notation!$\Sym(V)$}%
$\PP(V)$ are 1-dimensional quotients of $V` `$, or hyperplanes in $V` `$, so that
$\Sym(V)$ is the homogeneous coordinate ring of $\PP(V)$ and lines in $V$ correspond to points of $\PP(V^*)$.
\index{notation!$G(k,V)$, $G(1,V) = \PP(V^*)$, $\GG(k,r)$}%
%\marginpar{\redden{$V*\to V^*` `$, ok?}}

However we write $G(k,V)$ for the variety of $k$-dimensional linear subspaces of $V` `$, so that in particular $G(1,V) = \PP(V^*)$.
%\marginpar{\redden{When does one use $\GG$ and the other $G$?}}
We also write $\GG(k,r)$ for $k$-dimen\-sional
projective subspaces of $\PP^r` `$.

The results we assume are
 well represented by the following classical theorems:

\begin{theorem}[B\'ezout's theorem]
If $X,Y\subset \PP^r$ are subvarieties 
\redden{satisfying the condition}
%\marginpar{reworded to allow line break}%
$\codim(X\cap Y) = \codim X + \codim Y$,
then $\deg (X\cap Y) = \deg X\deg Y$.%
\index{B\'ezout's theorem}%
\end{theorem}

\begin{theorem}[Bertini's theorem]\label{Bertini}
If $X\subset \PP^r$  is a nonsingular quasiprojective variety, and $\{H_\lambda \mid \lambda\in \Lambda\}$ is a linear family of hyperplanes of $\PP^r` `$, then for an open subset of $\lambda\in \Lambda$ the scheme $H_\lambda \cap X$ is nonsingular away from the union of the \emph{base locus}
$
\cap_{\lambda \in \Lambda} H_\lambda
$
and the singular locus of $X$.
\index{Bertini's theorem}%
\end{theorem}

\begin{theorem}[Main theorem of elimination theory]
 Any morphism $\phi: X\to Y$ of projective varieties (or schemes) is closed: if $X'\subset X$ is a Zariski closed subset,
 then $\phi(X') \subset Y$ is also closed.
\index{elimination theory}%
\index{Main theorem of elimination theory}%
\end{theorem}

\begin{corollary}
If $\phi: C\to D$ is a nonconstant morphism of (projective) curves, then $\phi$ is finite and surjective. 
\end{corollary}

If $D$ is a 
smooth curve then the local ring of $D$ at any point is a discrete valuation ring, so any torsion free module is flat. 
Thus:

\begin{proposition}
If $\phi: C\to D$ is a nonconstant morphism of smooth curves, then $\phi$ is finite, surjective, and flat.
\end{proposition}

If $X\subset \PP^{r}$ is any scheme, we define the 
\blue{\emph{homogeneous coordinate ring}}
\index{homogeneous coordinate ring}%
of $X\subset \PP^{r}$
to be $R_{X} = R_{X/\PP^{r}}:= S/I(X)$, where $S =\CC[x_{0}, \dots, x_{r}]$ is the homogeneous coordinate ring of $\PP^{r}$. We emphasize
that, unlike the coordinate ring of an affine variety, this is not an intrinsic invariant of $X$, but depends on the 
embedding in $\PP^{r}$. 

When $X$ is reduced and irreducible we write $\kappa(X)$ for the field of rational functions on $X$. In particular, if
$p$ is a closed point  we write $\kappa(p)\cong \CC$ for the residue class field at $p$.

\subsection*{Sheaves and cohomology} 

Some familiarity with coherent sheaves is recommended; a possible source is
the first chapter of 
%\marginpar{dropped ``our book''}
\cite{GeomSchemes}. 
As for cohomology, it is probably enough if the reader can write down $H^i$ and exact sequences without blushing.
In any case we review some of the theory of coherent sheaves and their cohomology theory, and that of divisors on projective 
varieties, in the first two Chapters. 

We occasionally use the bijection between algebraic and analytic sheaves on smooth projective curves, which preserves
cohomology and exact sequences. This is a special case of the results in \cite{GAGA}. 

If $\sF$ is a sheaf on $X$ and $X\subset Y$ then we will identify $\sF$ with
the sheaf usually written $\iota_*\sF$, where $\iota:X\to Y$ is the inclusion map,
 and thus regard $\sF$ as a sheaf on $Y$ as well.
The cohomology  $H^i(X, \sF) = H^i(Y,\iota_*\sF)$ canonically, so we will
simply and unambiguously write $H^i(\sF)$ for either of these. 

We write $h^i(\sF)$ or (if $D$ is a divisor) $h^{i}(D)$ for 
\index{notation!$h^i(\sF)$, $h^{i}(D)$}%
$\dim_{\CC}H^i(\sF)$ or $\dim_{\CC}H^i(\cO_X(D))$. 

If $\sF$ is a sheaf on projective space (perhaps supported on a subvariety) we write $H^i_*(\sF)$ for
\index{notation!$H^i_*(\sF)$}%
$\oplus_{m\in \ZZ} H^i(\sF(m))$. 



%footer for separate chapter files

\ifx\whole\undefined
%\makeatletter\def\@biblabel#1{#1]}\makeatother
\makeatletter \def\@biblabel#1{\ignorespaces} \makeatother
\bibliographystyle{msribib}
\bibliography{slag}

%%%% EXPLANATIONS:

% f and n
% some authors have all works collected at the end

\begingroup
%\catcode`\^\active
%if ^ is followed by 
% 1:  print f, gobble the following ^ and the next character
% 0:  print n, gobble the following ^
% any other letter: normal subscript
%\makeatletter
%\def^#1{\ifx1#1f\expandafter\@gobbletwo\else
%        \ifx0#1n\expandafter\expandafter\expandafter\@gobble
%        \else\sp{#1}\fi\fi}
%\makeatother
\let\moreadhoc\relax
\def\indexintro{%An author's cited works appear at the end of the
%author's entry; for conventions
%see the List of Citations on page~\pageref{loc}.  
%\smallbreak\noindent
%The letter `f' after a page number indicates a figure, `n' a footnote.
}
\printindex[gen]
\endgroup % end of \catcode
%requires makeindex
\end{document}
\else
\fi


