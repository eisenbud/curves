%header and footer for separate chapter files

\ifx\whole\undefined
\documentclass[12pt, leqno]{book}
\usepackage{graphicx}
\usepackage{eps-to-pdf}
\input style-for-curves.sty
%\input sl-macros.sty
\usepackage{hyperref}
\usepackage{showkeys} %This shows the labels.
\usepackage{msribib}
\usepackage{pdfpages}
\usepackage{draftwatermark}
\SetWatermarkText{DRAFT:\ \today}
\SetWatermarkScale{2}
\SetWatermarkColor[gray]{0.9}

%\usepackage{SLAG,msribib,local}
%\usepackage{amsmath,amscd,amsthm,amssymb,amsxtra,latexsym,epsfig,epic,graphics}
%\usepackage[matrix,arrow,curve]{xy}
%\usepackage{graphicx}
%\usepackage{diagrams}
%
%%\usepackage{amsrefs}
%%%%%%%%%%%%%%%%%%%%%%%%%%%%%%%%%%%%%%%%%%
%%\textwidth16cm
%%\textheight20cm
%%\topmargin-2cm
%\oddsidemargin.8cm
%\evensidemargin1cm
%
%%%%%%Definitions
%\input preamble.tex
%\input style-for-curves.sty
%\def\TU{{\bf U}}
%\def\AA{{\mathbb A}}
%\def\BB{{\mathbb B}}
%\def\CC{{\mathbb C}}
%\def\QQ{{\mathbb Q}}
%\def\RR{{\mathbb R}}
%\def\facet{{\bf facet}}
%\def\image{{\rm image}}
%\def\cE{{\cal E}}
%\def\cF{{\cal F}}
%\def\cG{{\cal G}}
%\def\cH{{\cal H}}
%\def\cHom{{{\cal H}om}}
%\def\h{{\rm h}}
% \def\bs{{Boij-S\"oderberg{} }}
%
%\makeatletter
%\def\Ddots{\mathinner{\mkern1mu\raise\p@
%\vbox{\kern7\p@\hbox{.}}\mkern2mu
%\raise4\p@\hbox{.}\mkern2mu\raise7\p@\hbox{.}\mkern1mu}}
%\makeatother

%%
%\pagestyle{myheadings}

%\input style-for-curves.tex
%\documentclass{cambridge7A}
%\usepackage{hatcher_revised} 
%\usepackage{3264}
   
\errorcontextlines=1000
%\usepackage{makeidx}
\let\see\relax
\usepackage{makeidx}
\makeindex
% \index{word} in the doc; \index{variety!algebraic} gives variety, algebraic
% PUT a % after each \index{***}

\overfullrule=5pt
\catcode`\@\active
\def@{\mskip1.5mu} %produce a small space in math with an @

\title{A Chapter from ``The Practice of Algebraic Curves"}
\author{\copyright David Eisenbud and Joe Harris}
%%\includeonly{%
%0-intro,01-ChowRingDogma,02-FirstExamples,03-Grassmannians,04-GeneralGrassmannians
%,05-VectorBundlesAndChernClasses,06-LinesOnHypersurfaces,07-SingularElementsOfLinearSeries,
%08-ParameterSpaces,
%bib
%}

\date{\today}
%%\date{}
%\title{Curves}
%%{\normalsize ***Preliminary Version***}} 
%\author{David Eisenbud and Joe Harris }
%
%\begin{document}

\begin{document}
\maketitle

\pagenumbering{roman}
\setcounter{page}{5}
%\begin{5}
%\end{5}
\pagenumbering{arabic}
\tableofcontents
\fi


\setlength{\parskip}{5pt}

\addtocounter{chapter}{-1}
\chapter{Introduction}
\label{IntroChapter}

\begin{quote}
\small\sf
``Es gibt nach des Verf. Erfarhrung kein besseres Mittel, Geometrie zu lernen, als
das Studium des Schubertschen `Kalk\"uls der abz\"ahlenden Geometrie'.''

(There is, in the author's experience, no better means of learning geometry than
the study of Schubert's ``Calculus of Enumerative Geometry.")

--B. L. van der Waerden (in a Zentralblatt review of an introduction to enumerative geometry
by Hendrik de Vries).
\bigskip

\end{quote}

%\noindent
%{\bf 1066 \& All That} (\cite{1066}) is ``A memorable history of England, comprising all the parts you can remember, including one hundred and three \emph{good} things, five \emph{bad} kings, and two \emph{genuine} dates\dots. History is not what you thought. \emph{It is what you can remember.} 
%\dots
%
%``In the year 1066 occurred the other memorable date in English History, viz. \emph{William the Conquereor, Ten Sixty-six.} 
%This is also called \emph{The Battle of Hastings,} and was when William I (1066) conquered England at the Battle of Senlac (\emph{Ten Sixty-six})\dots 
%The Norman Conquest was a Good Thing, as from this time onwards England stopped being conquered and thus was able to become top nation.''


\section{Why you want to read this book}

Algebraic geometry is an old subject. You could make the case that it dates back to Descartes, whose introduction of coordinates in the plane and in space made it possible to relate the algebra of polynomials to the geometry of their zero loci; in any event, mathematicians have been doing what is recognizably algebraic geometry for more than two centuries.

And within algebraic geometry, the study of algebraic curves is naturally the oldest topic. After all, the study of polynomials in two variables via their zero sets is the first nontrivial case of the general paradigm of algebraic geometry. Moreover, the theory of algebraic curves received a tremendous boost in the 19th century from the work of mathematicians studying complex analysis, from the introduction of the notion of Riemann surfaces in **** on.

What all this has meant is that the subject of algebraic curves is one of the richest in algebraic geometry, if not in all of mathematics. Examples abound; if you want to know whether a given hypothesis holds, you can almost always find well-understood special cases that will allow you to test it. And some of the fundamental constructions of algebraic geometry, like the construction of moduli spaces and their description, can be carried out in the setting of algebraic curves with a degree of precision and detail far beyond what has been possible (so far!) in higher dimensions. 

By way of one example, consider the relatively simple case of plane curves of degrees 3 and 4. In degree 3, a smooth plane cubic will have 9 flexes---points where the tangent line has contact of order 3 with the curves---and these points form a remarkable configuration; they're the only known example of a finite, non-colinear set of points in the plane such that the line joining any two contains a third. And generalizing the notion of the flexes of a plane cubic leads us to the subject of inflectionary behavior of linear series on curves in general, a fascinating topic and one that has played a role as a tool in the proof of fundamental theorems. And in degree 4, we see that every smooth quartic curve has exactly 28 bitangents, forming a beautiful and mysterious configuration (for example, there are **** 4-tuples of bitangents such that their eight points of tangency with the curve lie on a conic!). Moreover, extending this to curves of higher genus leads us to the rich theory of theta-characteristics, bringing together algebra (in the form of a bilinear pairing on the points of order 2 in the Jacobian of a curve), analysis (in the form of theta-functions) and of course projective geometry.

The richest subject in what is arguably the richest branch of mathematics\footnote{Number theorists may quibble}---of course you want to read this book! 

\section{Why we wrote this book}

The wealth of beauty, both in theory and in examples, certainly makes the study of algebraic curves an attractive prospect. But it comes at a price: to absorb in detail all the things we've learned over the centuries about algebraic curves would take years, if not decades. This is, in essence, the conundrum facing anyone who undertakes to write a book on the subject: how to convey the wealth of our knowledge of the subject (and the many many ways in which our knowledge falls short of what we might wish) without writing an encyclopedia.

For better or worse, this is our attempt to do exactly that. Our intended audience is a graduate student considering working in the field of algebraic curves, or a researcher in a related field whose work has led them to questions about algebraic curves. Our goal is to equip the reader with the understanding of both the techniques and the state of our knowledge necessary to read the current literature and work on open problems.

\section{What's with the title?}

The title is a shout-out to T. S. Eliot's book, \emph{Old Possum's Book of Practical Cats}, a collection of light verse from Eliot in which he introduces the reader to a collection of cats with distinct personalities. In a way, our approach reflects Eliot's: different areas and questions under the general umbrella of algebraic curve theory have distinct aspects---personalities, if you will---and our hope is to introduce these to the reader.


\section{What's in this book}

We start in Chapter~\ref{linear systems} by laying out the basic objects we'll be dealing with through the remainder of the book: basic notions like invertible sheaves, and linear systems, the central object of algebraic curve theory. Thereafter, we alternate between chapters focussed on special cases and chapters developing more of the theory. Thus, for example, Chapter~\ref{genus 0 and 1 chapter} describes the geometry of curves of genus 0 and 1 in projective space. When we get to genus 2, however, a fundamental shift occurs: it's no longer the case that all invertible sheaves of a given degree on a curve $C$ of genus $g \geq 2$ are congruent mod the automorphism group of the curve, and so we need to introduce the Picard variety $\Pic_d(C)$ parametrizing isomorphism classes of line bundles, which we do in Chapter~\ref{new Jacobians chapter}.

Equipped with (some) knowledge of the Picard varieties, we proceed in Chapter~\ref{genus 2 and 3 chapter} to study curves of genus 2 and 3, describing in particular how the geometry of maps of our curve to projective space given by sections of line bundles depends on the bundle. This done, we go back in Chapter~\ref{Brill-Noether} to developing the general theory. Here we ask the fundamental question, ``What linear series exist on curves of genus $g$?"; we give three possible interpretations of this question, and the answers to each. The first two are answered by Clifford's theorem and the Castelnuovo bound, both of which we prove in the chapter; the last one is answered by the Brill-Noether theorem, which we describe and state but do not prove.

After this, it's back to examples: in Chapter~\ref{genus 4, 5 and 6 chapter}, where we analyze in some detail the geometry of curves of genus 4, 5 and 6. An interesting shift occurs here: in the cases of genus 4 and 5, we are able to answer all our questions about the geometry of curves by ad-hoc methods; but in genus 6 we need to quote the Brill-Noether theorem to complete our analysis. Then in Chapter~\ref{inflections chapter}, we introduce the theory of \emph{inflectionary points} of linear series, and use this to give a proof of (half of) Brill-Noether.

Next up comes Chapters~\ref{HilbertSchemesChapter} and~\ref{HilbertSchemesCounterexamplesChapter}, which represent in some ways a culmination of the work so far. Both chapters are concerned with \emph{Hilbert schemes}, schemes $\cH_{g,r,d}$ that parametrize curves of given degree $d$ and genus $g$ in projective space $\PP^r$. In Chapter~\ref{HilbertSchemesChapter} we work out a number of examples, on the basis of which we arrive at an ``expected behavior" of the Hilbert scheme, which indeed holds in a certain range of $(g,r,d)$. Then, in Chapter~\ref{HilbertSchemesCounterexamplesChapter}, we give a series of counterexamples showing that the expected behavior does \emph{not} hold for all $(g,r,d)$, and describe some of the open problems and conjectures surrounding this question.

Finally, in Chapter~\ref{PlaneCurvesChapter} we take up a specific topic that has been central to curve theory since its inception: the geometry of (possibly singular) plane curves. Historically, before the advent of abstract algebraic varieties, the way most people thought of a curve $C$ of genus $g$ was as the normalization of a plane curve $C_0$, and pretty much all the geometry of $C$ was worked out via this plane model $C_0$. Even today, that approach offers insight into the geometry of curves---we describe, for example, an explicit algorithm for finding a complete linear system on a curve---and we give here a description of this general theory.

Finally, we have two appendices. In one, we discuss the general topic of moduli spaces: what we mean by a moduli problem, and when there exists a solution. This is a recurrent theme throughout the book, and we collect here the basic facts about these spaces. In the other, we discuss \emph{scrolls}, varieties that are ubiquitous in the study of the geometry of curves in projective space.

\begin{quote}
\small\sf
We are dealing here with a fundamental and almost paradoxical difficulty. Stated briefly, it is that learning is sequential but knowledge is not. A branch of mathematics... consists of an intricate network  of interrelated facts, each of which contributes to the understanding of those around it. When confronted with this network for the first time, we are forced to follow a particular path, which involves a somewhat arbitrary ordering of the facts.

--Robert Osserman.

\end{quote}



Where to begin? To start with the technical underpinnings of a subject risks losing the reader before the point of all that preliminary work is made clear; but to defer the logical foundations carries its own dangers---as the unproved assertions mount up, the reader may well feel adrift.

Intersection theory poses a particular challenge in this regard, since the development of its foundations is so demanding. It is possible, however, to state fairly simply and precisely the main foundational results of the subject, at least in the limited context of intersections on smooth projective varieties. The reader who is willing to take these results on faith for a little while, and accept this restriction, can then be shown ``what the subject is good for," in the form of examples and applications. This is the path we've chosen in this book, as we'll now describe.

\subsection{Overture}



\subsection{Relation of this book to Other texts} 
Chapter IV of~\cite{Hartshorne1977} has a similar flavor to that of this book, and indeed we will take a familiarity with the material of that text as effectively a pre-requisite. A beautiful (and brief) account of a number of topics in a style we particularly admire is found in \cite{Mumford-C&J}.
There are more elementary accounts of some of our material, in~\cite{ Fulton 2008 } and \cite{Walker} (who goes farther than we do into local resolution of singularities) as well as the notes of~\cite{Griffiths***}, and a comprehensive treatment of the local theory of plane curves and their singularities in \cite{Brieskorn}. The topological questions there are developed in different directions in \cite{Milnor} and \cite{Eisenbud-Neumann}. An interesting collection of topics is presented in ~\cite{Clemens}.
There is also a far more comprehensive treatment than that of the present book in \cite{ACGH}.

 The Riemann surface point of view is well represented in the books \cite{Forster} \cite{Gunning} \cite{Kirwan}\cite{Miranda}. The existence of the moduli spaces of curves is in the background of this book whenever we speak of ``general'' curves of given genus, we say only a little about it; many more ideas and details about it can be found in~\cite{Harris-Morrison}, though even that is not a complete account.


\section{Prerequisites, notation and conventions}

We will assume familiarity, though not much technical information, with coherent sheaves and their cohomology though some of this theory, and that of divisors on projective 
varieties, is reviewed in Appendix 2. It is probably enough if the reader can write down $H^1$ and $H^0$ and exact sequences without blushing, and in any case we redo some of what is in~\cite[Chapter IV]{Hartshorne1977}  Chapter IV in our language. We also use a few basic facts about surfaces from the first two sections of ~\cite[Chapter IV]{Hartshorne1977}.

The reader should also be familiar with the (Krull) dimension of rings and varieties, and their primary decomposition at the level of \cite{Atiyah-MacDonald}, including localization and with Lasker's often mis-attributed theorem that complete intersections are unmixed (the version for forms in 3 variables was proven by
Max Noether, and is now often called the $AF+BG$ theorem.)




\begin{quote}
\small\sf
We are all familiar with the after-the-fact tone---weary, self-justificatory, aggrieved, apologetic---shared by ship captains appearing before boards of inquiry to explain how they came to run their vessels aground, and by authors composing forewords.

--John Lanchester 
\bigskip

\end{quote}



\

%footer for separate chapter files

\ifx\whole\undefined
\makeatletter\def\@biblabel#1{#1]}\makeatother
\gdef\urlhook{\url}
\bibliography{slag}
\bibliographystyle{msribib}


%%%% EXPLANATIONS:

% f and n
% some authors have all works collected at the end

\catcode`\^\active
%if ^ is followed by 
% 1:  print f, gobble the following ^ and the next character
% 0:  print n, gobble the following ^
% any other letter: print letter
\makeatletter
\def^#1{\ifx1#1f\expandafter\@gobbletwo\else
        \ifx0#1n\expandafter\expandafter\expandafter\@gobble\else#1\fi\fi}
\makeatother
\let\moreadhoc\relax
\def\indexintro{%An author's cited works appear at the end of the
%author's entry; for conventions
%see the List of Citations on page~\pageref{loc}.  
%\smallbreak\noindent
The letter `f' after a page number indicates a figure, `n' a footnote.}
\printindex[gen]
%requires makeindex
\end{document}
\else
\fi


