%header and footer for separate chapter files

\ifx\whole\undefined
\documentclass[12pt, leqno]{book}
\usepackage{graphicx}
\input style-for-curves.sty
\usepackage{hyperref}
\usepackage{showkeys} %This shows the labels.
%\usepackage{SLAG,msribib,local}
%\usepackage{amsmath,amscd,amsthm,amssymb,amsxtra,latexsym,epsfig,epic,graphics}
%\usepackage[matrix,arrow,curve]{xy}
%\usepackage{graphicx}
%\usepackage{diagrams}
%
%%\usepackage{amsrefs}
%%%%%%%%%%%%%%%%%%%%%%%%%%%%%%%%%%%%%%%%%%
%%\textwidth16cm
%%\textheight20cm
%%\topmargin-2cm
%\oddsidemargin.8cm
%\evensidemargin1cm
%
%%%%%%Definitions
%\input preamble.tex
%\input style-for-curves.sty
%\def\TU{{\bf U}}
%\def\AA{{\mathbb A}}
%\def\BB{{\mathbb B}}
%\def\CC{{\mathbb C}}
%\def\QQ{{\mathbb Q}}
%\def\RR{{\mathbb R}}
%\def\facet{{\bf facet}}
%\def\image{{\rm image}}
%\def\cE{{\cal E}}
%\def\cF{{\cal F}}
%\def\cG{{\cal G}}
%\def\cH{{\cal H}}
%\def\cHom{{{\cal H}om}}
%\def\h{{\rm h}}
% \def\bs{{Boij-S\"oderberg{} }}
%
%\makeatletter
%\def\Ddots{\mathinner{\mkern1mu\raise\p@
%\vbox{\kern7\p@\hbox{.}}\mkern2mu
%\raise4\p@\hbox{.}\mkern2mu\raise7\p@\hbox{.}\mkern1mu}}
%\makeatother

%%
%\pagestyle{myheadings}

%\input style-for-curves.tex
%\documentclass{cambridge7A}
%\usepackage{hatcher_revised} 
%\usepackage{3264}
   
\errorcontextlines=1000
%\usepackage{makeidx}
\let\see\relax
\usepackage{makeidx}
\makeindex
% \index{word} in the doc; \index{variety!algebraic} gives variety, algebraic
% PUT a % after each \index{***}

\overfullrule=5pt
\catcode`\@\active
\def@{\mskip1.5mu} %produce a small space in math with an @

\title{Personalities of Curves}
\author{\copyright David Eisenbud and Joe Harris}
%%\includeonly{%
%0-intro,01-ChowRingDogma,02-FirstExamples,03-Grassmannians,04-GeneralGrassmannians
%,05-VectorBundlesAndChernClasses,06-LinesOnHypersurfaces,07-SingularElementsOfLinearSeries,
%08-ParameterSpaces,
%bib
%}

\date{\today}
%%\date{}
%\title{Curves}
%%{\normalsize ***Preliminary Version***}} 
%\author{David Eisenbud and Joe Harris }
%
%\begin{document}

\begin{document}
\maketitle

\pagenumbering{roman}
\setcounter{page}{5}
%\begin{5}
%\end{5}
\pagenumbering{arabic}
\tableofcontents
\fi


\setlength{\parskip}{5pt}

\addtocounter{chapter}{-1}
\chapter{Introduction}
\label{IntroChapter}

\begin{quote}\it{I'm very well acquainted, too, with matters mathematical,\\
I understand equations, both the simple and quadratical,\\
About binomial theorem I am teeming with a lot o' news,\\
With many cheerful facts about the square of the hypotenuse.}
\end{quote}

---Gilbert and Sullivan, Pirates of Penzance, Major General's Song


\begin{quote}
Be simple by being concrete. Listeners are prepared to
accept unstated (but hinted) generalizations much more than they are able, on the spur of the moment, to
decode a precisely stated abstraction and to re-invent the special cases that motivated it in the first place. 
\end{quote}

--Paul Halmos, How to Talk Mathematics
\section{Why you want to read this book}

Algebraic geometry is an old subject. Descartes' introduction in 1637 of coordinates in the plane and in space made it possible to relate the algebra of polynomials to the geometry of their zero loci. Gauss' proof of the Fundamental Theorem of Algebra showed that the degree of a polynomial, an algebraic invariant, was equal to the number of roots counted with mulitplicity, a geometric invariant. Mathematicians have been doing what is recognizably algebraic geometry for more than two centuries.

Within algebraic geometry, the study of algebraic curves is the oldest topic. Newton already classified all the possible types of real affine cubics. By the middle of
the 19th century a rich theory of curves in the complex projective plane was a central topic, overturned by Riemann's work in mid-century---what became the theory of Riemann surfaces, which introduced techniques of complex analysis into the field. This was  taken up and made algebraic as a theory of plane curves by Alexander Brill, Max Noether, F. S. Macaulay and many others. By the end of the century Halphen and others were interested in the classification of space curves as well. 

The development continues unabated today: in the second half of the 20th century Grothendieck's foundations lead to the solution of many classical problems and, in particular, to a firm foundation for the theory of moduli, allowing mathematicians to exploit the fact that algebraic varieties generally come in families parameterized by other algebraic varieties---something that we will return to often in this book. Algebraic Geometry has merged more and more with number theory, and the moduli space of curves plays an important role in String Theory, responding to the needs of physicists.

For these reasons, the subject of algebraic curves is one of the richest in algebraic geometry, if not in all of mathematics. Examples abound; if you want to know whether a conjecture is plausible, you can generally find well-understood special cases on which to test it. Some of the fundamental constructions of algebraic geometry, like the construction of moduli spaces and their description, can be carried out in the setting of algebraic curves with a degree of precision and detail far beyond what has been possible in higher dimensions. 

By way of one example, consider the relatively simple cases of the geometry of plane curves of degrees 3 and 4: 

\pict{small pictures illustrating these two things could go here}
Every smooth plane cubic has exactly 9 flexes---points where the tangent line has contact of order 3 with the curves---and these points form a remarkable configuration; they're the only known example of a finite, non-colinear set of points in the plane such that the line joining any two contains a third. Generalizing the notion of the flexes of a plane cubic leads us to the subject of inflectionary behavior of linear series on curves in general, which arose separately in the theory of Weierstrass points on Riemann surfaces, and  has become a powerful tool. 

Every smooth quartic curve has exactly 28 bitangents, also forming a beautiful and mysterious configuration;  in \cite{MR0115124}, first published in 1837, Salmon computed the number (315) of 4-tuples of bitangents whose eight points of tangency lie on a conic. The extension of this idea to curves of higher genus leads us to the rich theory of theta-characteristics, bringing together algebra (in the form of a bilinear pairing on the points of order 2 in the Jacobian of a curve), analysis (in the form of theta-functions) and of course projective geometry.

The richest subject in what is arguably the richest branch of mathematics\footnote{Number theorists may quibble}---of course you want to read this book! 

\section{Why we wrote this book}

The wealth of beauty, both in theory and in examples, certainly makes the study of algebraic curves an attractive prospect. But it comes at a price: to absorb in detail all the things we've learned over the centuries about algebraic curves would take years, if not decades. This is, in essence, the conundrum facing anyone who undertakes to write a book on the subject: how to convey the wealth of information  (and the many many ways in which our knowledge is incomplete) without writing an encyclopedia. We have chosen to try to be useful, but not complete.

Our intended audience is a graduate student considering working in the field of algebraic curves, or a researcher in a related field whose work has led them to questions about algebraic curves. Our goal is to equip the reader with the understanding of both the techniques and the state of our knowledge necessary to read the current literature and work on open problems.

\section{What's Practical?}

Although mathematicians aspire to understand their subjects deeply, we feel that we learn in stages: in early stages we accept large and difficult results as black boxes and explore the rich examples that they yield. That is how we have tried to organize this book: 
We begin with two chapters that we hope will bridge the gap between first courses in algebraic geometry/commutative algebra at the level,  of Fulton's or  Reid's well-known books \cite{Fulton1989}, \cite{MR982494} and the professional language of invertible sheaves, cohomology and linear series. An ideal background would
be Hartshorne's book \cite{Hartshorne1977} or Vakil's notes \cite{Vakil-notes} but in \emph{practice} very much less will suffice, if the reader is willing to accept some advanced ideas or look them up at leisure---we have tried to give precise references where this might be required. Subsequent chapters roughly alternate between expositions of basic techniques (partly without proofs) and families of examples, treated in detail. 

We love T. S. Eliot's book, \emph{Old Possum's Book of Practical Cats}, a collection of light verse introducing a collection of cats with distinct personalities. In a way, our approach in the chapters of examples reflects Eliot's: different areas and questions under the general umbrella of algebraic curve theory have distinct aspects---personalities, if you will.


\section{What's in this book}
All mathematical exposition faces a fundamental problem, well described by Osserman:
\begin{quote}
\small\sf
We are dealing here with a fundamental and almost paradoxical difficulty. Stated briefly, it is that learning is sequential but knowledge is not. A branch of mathematics... consists of an intricate network network of interrelated facts, each of which contributes to the understanding of those around it. When confronted with this network for the first time, we are forced to follow a particular path, which involves a somewhat arbitrary ordering of the facts.

--Robert Osserman.

\end{quote}

In Chapters~\ref{linear series} and~\ref{RiemannRochChapter} we lay out the central objects of algebraic curve theory: invertible sheaves, and linear systems, and we prove or sketch ideas about them that we will use. These chapters may serve as review for someone who has been exposed already to material roughly equivalent to Chapter IV of \cite{Hartshorne1977}. Thereafter, we alternate between chapters focussed on special cases and chapters developing more of the theory. Thus, for example, Chapter~\ref{genus 0 and 1 chapter} describes the geometry of curves of genus 0 and 1 in projective space. 

In genus 2, there is a fundamental shift: not all invertible sheaves of a given degree on a curve $C$ of genus $g \geq 2$ are congruent mod the automorphism group of the curve, and so we first introduce the Jacobian and the Picard varieties $\Pic_d(C)$ parametrizing isomorphism classes of line bundles in Chapter~\ref{Jacobians chapter}, and the subvarieties parametrizing invertible sheaves with many sections.

Equipped with this information we proceed in Chapter~\ref{genus 2 and 3 chapter} to study curves of genus 2 and 3, describing in particular how the geometry of maps of our curve to projective space given by sections of line bundles depends on the bundle. 

Families of similar algebro-geometric objects are often naturally parameterized by algebraic varieties. This idea goes back to the beginning of algebraic geometry and the family of plane curves of degree $d$, but was only fully clarified in the work of Grothendieck, Deligne and Mumford, and we spend Chapter~\ref{ModuliChapter} explaining this modern understanding of \emph{fine moduli spaces} through some central examples, primarily that of
the Hilbert scheme. The most interesting example of a moduli space, the moduli of smooth (or stable) curves of genus $g$, is however not a fine moduli space, and we spend Chapter~\ref{CurvesModuliChapter}

After this, it's back to examples: in Chapter~\ref{genus 4, 5 Chapter} we analyze in some detail the geometry of curves of genus 4 and 5. We are able to answer all our questions about the geometry of these curves by ad-hoc methods. But to do the same  in genus 6 we need to quote the Brill-Noether theorem to complete our analysis, and this requires the notion of a \emph{general} curve. 

Thus in Chapter~\ref{ModuliChapter} we turn to the notion of moduli, and describe some of the important fine moduli spaces, focusing
on the Hilbert scheme. However the most important example, the moduli space of curves, is not a fine moduli space, and we spend
Chapter~\ref{CurvesModuliChapter} describing the problems and the necessary machinery to overcome them (though we do not give
proofs). In that chapter we also describe the Hurwitz space of coverings and the Severi variety of nodal plane curves.

We also need more information about hyperplane sections. In Chapter~\ref{linear general position chapter} we give Rathmann's proof that
the points of a general hyperplane section of an irreducible reduced curve are in linearly general position (independent of the characteristic)
and deduce a number of consequences, not least the theory of Castelnuovo curves, which has as consequence the 
fact that canonical curves and curves of high degree are arithmetically Cohen-Macaulay.

In characteristic 0 an even stronger statement is true: the monodromy of the hyperplane divisors is the full symmetric group. We take up this
and some other monodromy questions in Chapter~\ref{uniform position}.

Finally in Chapter~\ref{Brill-Noether} we return to the general question: ``What linear series exist on curves of genus $g$?" We give three possible interpretations of this question, and the answers to each: Clifford's theorem, the Castelnuovo bound, and the Brill-Noether theorem, postponing the proof of the latter. We apply the Brill-Noether result to explore  various special classes of curves of genus 6.

Preparing for the proof of the Brill-Noether theorem in Chapter~\ref{BrillNoetherproofChapter} we discuss flexes of plane curves and their 
generalization, the inflections of arbitrary linear series in Chapter~\ref{inflections chapter}. 

In Chapter~\ref{PlaneCurvesChapter} we take up two topics within the vast subject  of plane curves. the geometry of (possibly singular) plane curves, and how it can be used in the study of smooth curves. In particular, we describe an explicit algorithm for finding any complete linear series on a curve, using a nodal plane model, and we give a description of the 
(small) Severi Variety---the collection of plane curves with $g$ nodes and no other singular points.

Curves that lie on a quadric in $\PP^3$ are easy to understand, In Chapter~\ref{ScrollsChapter} we systematically describe a natural generalization: the frational normal scrolls and the curves that lie on them. Chapter~\ref{SyzygiesChapter} presents some aspects of the theory of syzygies of the homogeneous ideals of curves and the the famous---and as of this writing open---conjecture of Mark Green that connects these to the Clifford index and the theory of rational normal scrolls. A table reproduced from work of Frank-Olaf Schreyer shows the sensitivity of the numerical information in the free resolution of a canonical curve to questions of the existence of special linear series on the curve.

 Finally, Chapter~\ref{HilbertSchemesChapter} represents in some ways a culmination of the book. It is concerned with \emph{Hilbert schemes}, schemes $\cH_{g,3,d}$ that parametrize smooth curves of given degree $d$ and genus $g$ in projective space $\PP^3$. We work out many examples up to degree 7, define the "principal component"---the only one dominating the moduli space of curves-- and derive dimension estimates from deformation theory and from the Brill-Noether theorems.

\subsection{Relation of this book to other texts} 
Chapter IV of~\cite{Hartshorne1977} has a similar flavor to that of this book. A beautiful (and brief) account of a number of topics in a style we particularly admire is found in \cite{MumfordCJ}.

A far more extensive treatment, partially overlapping that of the present book and containing many other topics, can be found in the beautiful and encyclopaedic~\cite{ACGH} and~\cite{ACG}, and yet even these works do not cover all the major topics in the field. 
One of the topics we do not cover, but which can be found in {ACG}, is the construction of the moduli space of curves. Many  ideas  about it can be found in~\cite{HarrisMorrison1998}, though  that is not a complete account. 

There are more elementary accounts of some of our material, in~\cite{Fulton1989} and \cite{Walker1978} (who goes farther than we do into local resolution of singularities) as well as~\cite{Griffiths-curves}, and a comprehensive treatment of the local theory of plane curves and their singularities in \cite{Brieskorn1986}. The topological questions there are developed in different directions in \cite{MR0239612} %Milnor
 and \cite{MR817982}%Eisenbud-Neumann}. 
 An interesting collection of topics is presented in~\cite{Clemens-Scrapbook}.

 The Riemann surface point of view is well represented in the books \cite{Forster} \cite{Gunning}, \cite{Gunning-2} \cite{Kirwan}\cite{Miranda}. 


\section{Prerequisites, notation and conventions}

In this book we work over the field of complex numbers $\CC$, though much of what we do
could be done over any field. 

\subsection{Commutative Algebra} 
All the rings we consider are commutative with unit and Noetherian.
The reader should be familiar with the (Krull) dimension of rings and varieties, and their primary decomposition at the level of \cite{Atiyah-MacDonald}. Our ground field is $\CC$, though nearly everything we do could be done over more general fields. The homogeneous coordinate ring of the (complex) projective space $\PP^r$ is $\CC[x_0, \dots, x_r]$. 

Since we are working over  a field of characteristic 0, we use the terms smooth and nonsingular interchangeably when
referring to a point on a scheme.

Some results that we use freely are:
 \begin{theorem}[Lasker's Theorem]\label{Lasker}
If $f_1,\dots, f_c \subset \CC[x_0,\dots, x_n]$ generates an ideal of codimension $c$, then 
the ideal $(f_1,\dots, f_c)$ is unmixed (all it's primary components have codimension $c$).
\end{theorem}

\begin{theorem}\label{finiteness of normalization}
 If $R$ is a domain that is a finitely generated algebra over a field or a localization of such an algebra, then the
normalization ($=$ integral closure) of $R$ is a finitely generated $R$-module.
If $R$ is 1-dimensional, then the normalization is nonsingular.
\end{theorem}

\subsection{Projective geometry}
Schemes are assumed quasi-projective, and \emph{varieties} (including curves) are reduced and irreducible schemes. ``Points'' will always be closed points unless we explicitly say otherwise. A curve is a 1-dimensional variety that is isomorphic to a closed subset of a projective space, and thus compact in the classical ($=$ analytic) topology. Though we occasionally use the classical topology, 
the term ``open set'' refers to the Zariski topology unless otherwise stated.

We assume some familiarity with projective geometry, such as the notion of the degree of a 
subvariety or subscheme of a projective space. The sort of results we use are
 well-represented by the following classical theorems:

\begin{theorem}[B\'ezout's Theorem]
If $X,Y\subset \PP^r$ are subvarieties of degrees $d,e$, and if $\codim(X\cap Y) = \codim X + \codim Y$,
then $\deg (X\cap Y) = \deg(X)\deg(Y)$.
\end{theorem}

\begin{theorem}[Bertini's Theorem]\label{Bertini}
If $X\subset \PP^r$  is a nonsingular variety, and $\{H_\lambda \mid \lambda\in \Lambda\}$ is a linear family of hyperplanes of $\PP^r$, then for an open subset of $\lambda\in \Lambda$ the scheme $H\cap X$ is nonsingular away from 
$
\cap_{\lambda \in \Lambda} H_\lambda.
$
\end{theorem}

\begin{theorem}[Main theorem of elimination theory]
 Any morphism $\phi: X\to Y$ of projective varieties (or schemes) is closed: if $X'\subset X$ is a Zariski closed subset,
 then $\phi(X') \subset Y$ is also closed.
\end{theorem}

\begin{corollary}
If $\phi: C\to D$ is a non-constant morphism of projective curves, then $\phi$ is finite and surjective. 
\end{corollary}

If $D$ is a 
smooth curve then the local ring of $D$ at any point is a discrete valuation ring so any torsion free module is flat. 
Thus:

\begin{proposition}
If $\phi: C\to D$ is a non-constant morphism of smooth curves, then $\phi$ is flat.
\end{proposition}

\subsection {Sheaves and cohomology} 

Some familiarity with schemes and coherent sheaves is recommended; a possible source is
the first chapter of our book \cite{GeomSchemes}.
As for cohomology, it is probably enough if the reader can write down $H^i$ and exact sequences without blushing.

In any case we review some of the theory of coherent sheaves and their cohomology theory, and that of divisors on projective 
varieties, in the first two Chapters. There we redo some of what is in~\cite[Chapter IV]{Hartshorne1977}  in our language. 

We occasionally use the bijection between algebraic and analytic sheaves on smooth projective curves, which preserves
cohomology and exact sequences. This is a special case of the results in \cite{GAGA}. 

Notation: If $\sF$ is a sheaf on a scheme $X$ we write $H^i(\sF)$ in place of $H^i(X; \sF)$ since this is independent of $X$.
If $\sF$ is a sheaf on a given projective space (perhaps supported on a subvariety) we write $H^i_*(\sF)$ for
$\oplus_{m\in \ZZ} H^i(\sF(m))$. We also use the notation $h^i(\sF)$ for $\dim_\CC H^i(\sF)$.







%\begin{quote}
%\small\sf
%We are all familiar with the after-the-fact tone---weary, self-justificatory, aggrieved, apologetic---shared by ship captains appearing before boards of inquiry to explain how they came to run their vessels aground, and by authors composing forewords.
%
%--John Lanchester 
%\bigskip
%
%\end{quote}
%


%footer for separate chapter files

\ifx\whole\undefined
%\makeatletter\def\@biblabel#1{#1]}\makeatother
\makeatletter \def\@biblabel#1{\ignorespaces} \makeatother
\bibliographystyle{msribib}
\bibliography{slag}

%%%% EXPLANATIONS:

% f and n
% some authors have all works collected at the end

\begingroup
%\catcode`\^\active
%if ^ is followed by 
% 1:  print f, gobble the following ^ and the next character
% 0:  print n, gobble the following ^
% any other letter: normal subscript
%\makeatletter
%\def^#1{\ifx1#1f\expandafter\@gobbletwo\else
%        \ifx0#1n\expandafter\expandafter\expandafter\@gobble
%        \else\sp{#1}\fi\fi}
%\makeatother
\let\moreadhoc\relax
\def\indexintro{%An author's cited works appear at the end of the
%author's entry; for conventions
%see the List of Citations on page~\pageref{loc}.  
%\smallbreak\noindent
%The letter `f' after a page number indicates a figure, `n' a footnote.
}
\printindex[gen]
\endgroup % end of \catcode
%requires makeindex
\end{document}
\else
\fi


