%header and footer for separate chapter files

\ifx\whole\undefined
\documentclass[12pt, leqno]{book}
\usepackage{graphicx}
\input style-for-curves.sty
\usepackage{hyperref}
\usepackage{showkeys} %This shows the labels.
%\usepackage{SLAG,msribib,local}
%\usepackage{amsmath,amscd,amsthm,amssymb,amsxtra,latexsym,epsfig,epic,graphics}
%\usepackage[matrix,arrow,curve]{xy}
%\usepackage{graphicx}
%\usepackage{diagrams}
%
%%\usepackage{amsrefs}
%%%%%%%%%%%%%%%%%%%%%%%%%%%%%%%%%%%%%%%%%%
%%\textwidth16cm
%%\textheight20cm
%%\topmargin-2cm
%\oddsidemargin.8cm
%\evensidemargin1cm
%
%%%%%%Definitions
%\input preamble.tex
%\input style-for-curves.sty
%\def\TU{{\bf U}}
%\def\AA{{\mathbb A}}
%\def\BB{{\mathbb B}}
%\def\CC{{\mathbb C}}
%\def\QQ{{\mathbb Q}}
%\def\RR{{\mathbb R}}
%\def\facet{{\bf facet}}
%\def\image{{\rm image}}
%\def\cE{{\cal E}}
%\def\cF{{\cal F}}
%\def\cG{{\cal G}}
%\def\cH{{\cal H}}
%\def\cHom{{{\cal H}om}}
%\def\h{{\rm h}}
% \def\bs{{Boij-S\"oderberg{} }}
%
%\makeatletter
%\def\Ddots{\mathinner{\mkern1mu\raise\p@
%\vbox{\kern7\p@\hbox{.}}\mkern2mu
%\raise4\p@\hbox{.}\mkern2mu\raise7\p@\hbox{.}\mkern1mu}}
%\makeatother

%%
%\pagestyle{myheadings}

%\input style-for-curves.tex
%\documentclass{cambridge7A}
%\usepackage{hatcher_revised} 
%\usepackage{3264}
   
\errorcontextlines=1000
%\usepackage{makeidx}
\let\see\relax
\usepackage{makeidx}
\makeindex
% \index{word} in the doc; \index{variety!algebraic} gives variety, algebraic
% PUT a % after each \index{***}

\overfullrule=5pt
\catcode`\@\active
\def@{\mskip1.5mu} %produce a small space in math with an @

\title{Personalities of Curves}
\author{\copyright David Eisenbud and Joe Harris}
%%\includeonly{%
%0-intro,01-ChowRingDogma,02-FirstExamples,03-Grassmannians,04-GeneralGrassmannians
%,05-VectorBundlesAndChernClasses,06-LinesOnHypersurfaces,07-SingularElementsOfLinearSeries,
%08-ParameterSpaces,
%bib
%}

\date{\today}
%%\date{}
%\title{Curves}
%%{\normalsize ***Preliminary Version***}} 
%\author{David Eisenbud and Joe Harris }
%
%\begin{document}

\begin{document}
\maketitle

\pagenumbering{roman}
\setcounter{page}{5}
%\begin{5}
%\end{5}
\pagenumbering{arabic}
\tableofcontents
\fi


\chapter{Inflections and Brill Noether}
\label{InflectionsChapter}

In this concluding chapter, we want to introduce one more aspect of the geometry of linear series on curves, the \emph{inflectionary points} of a linear system, and use it to give a proof of at least half of the classical Brill-Noether theorem.

Inflectionary points in general are a direct generalization of the notion of flex point of a smooth plane curve to curves in higher-dimensional space. Just as a point $p \in C$ on a smooth plane curve $C \subset \PP^2$ is called a \emph{flex point} if there is a line $L \subset \PP^2$ having contact of order 3 or more with $C$ at $p$, a point on a smooth, nondegenerate curve $C \subset \PP^r$ will be called an inflectionary point if there is a hyperplane $H \subset \PP^r$ having contact of order $r+1$ or more with $C$ at $p$. This notion can be extended to arbitrary linear series on smooth curves (as opposed to very ample ones); we'll see below that every linear series has finitely many inflectionary points, and how to count them.


\section{Inflection points,  Pl\"ucker formulas and Weierstrass points}

\subsection{Definitions}

To start with the definition: let $C$ be a smooth projective curve of genus $g$, and $\cD = (\cL,V)$ a $g^r_d$ on $C$; that is, let $\cL \in \Pic^d(C)$ be a line bundle of degree $d$ on $C$ and $V \subset H^0(\cL)$ an $(r+1)$-dimensional vector space of sections.

For any point $p \in C$, we can find a basis $\sigma_0, \dots, \sigma_r$ of $V$ consisting of sections vanishing to different orders at $p$ (just start with any basis and if two elements vanish to the same order, replace one with a linear combination of the two vanishing to strictly higher order; since the order $\ord_p(\sigma)$ of any section at $p$ is bounded above by $d$, this process must terminate). The set
$$
\{ \ord_p(\sigma) \mid \sigma \neq 0 \in V \}
$$
thus has cardinality $r+1$, and we can write it as
$$
\{ \ord_p(\sigma) \mid \sigma \neq 0 \in V \} = \{a_0,\dots,a_r\} \; \text{ with } \; a_0 < a_1 < \dots < a_r;
$$
the sequence $a_i = a_i(\cD,p)$ is called the \emph{vanishing sequence} of $\cD$ at $p$.  Finally, since $a_i \geq i$, we can set $\alpha_i = \alpha_i(\cD,p) = a_i - i$; the sequence $0 \leq \alpha_0 \leq \alpha_1 \leq \dots \leq \alpha_r$ is called the \emph{ramification sequence} of $\cD$ at $p$, and we say that $p$ is an \emph{inflectionary point} of the linear series $\cD$ if $(\alpha_0,\dots,\alpha_r) \neq (0,\dots,0)$. (Note that if $\cD$ is very ample, so that it may be viewed at the linear series cut on $C$ by hyperplanes for some embedding $C \subset \PP^r$, then this coincides with the notion above: $p$ is an inflectionary point if there is a hyperplane $H \subset \PP^r$ having contact of order $r+1$ or more with $C$ at $p$.)

Finally, we define the \emph{weight} of an inflectionary point to be the sum
$$
w(\cD, p) = \sum_{i=0}^r \alpha_i(\cD, p).
$$

\subsection{The Pl\"ucker formula}

There are two essential facts about the inflectionary points of a linear series $\cD = (\cL,V)$ on a smooth curve $C$.

The first is simply that \emph{not every point of $C$ is an inflectionary point}. This may seem obvious---try to imagine a plane curve in which every point is a flex!---but in fact it's false in characteristic $p$: there exist what are called ``strange curves," smooth curves in $\PP^r$ such that every point is inflectionary. Luckily, we are in characteristic 0 here, so we don't have to worry about this phenomenon.

The second is the total number of inflectionary points, properly counted, is determined by $d, g$ and $r$. Here ``properly counted" means that we count an inflectionary point $p \in C$ $w$ times, where $w = w(\cD,p)$ is the weight of $p$; the actual formula, called the \emph{Pl\"ucker formula}, is
$$
\sum_{p \in C} w(\cD, p) \; = \; (r+1)d + r(r+1)(g-1).
$$
Proofs of these two statements can be found in a variety of sources; see for example~\cite{} (3264?). 

It should be emphasized that the Pl\"ucker formula, while extremely useful (as we'll see in the remainder of this chapter), leaves many questions unanswered: we don't know, for example, what combinations of inflectionary points are possible, or what the behavior of the inflectionary points on a suitably general curve may be.

\subsection{Weierstrass points}

As with any extrinsic invariant of a curve in projective space, we can derive an intrinsic invariant of an abstract curve by applying the notion of inflectionary point to the canonical linear series. 

We define a \emph{Weierstrass point} of a curve $C$ to be an inflectionary point of the canonical linear series $|K_C|$. This amounts to saying a point $p$ is a Weierstrass point if there exists a canonical differential on $C$ vanishing to order $g$ or more at $p$; by Riemann-Roch, this is tantamount to the condition that $h^0(\cO_C(gp)) \geq 2$, or in other words to saying that there exists a rational function on $C$, regular away from $p$ and having a pole of order $g$ or less at $p$.

We can similarly characterize all the inflectionary indices of the canonical series at a point. We see from Riemann-Roch that for any $k \geq 0$, there exists a rational function on $C$ with a pole of order exactly $k$ at $p$---that is,
$$
h^0(\cO_C(kp)) > h^0(\cO_C((k-1)p))\text{---}
$$
if and only if 
$$
h^0(K_C(-kp)) = h^0(K_C((-k+1)p)); 
$$
that is, if and only if there does \emph{not} exist a regular differential on $C$ with a zero of order exactly $k-1$ at $p$. To give the classical terminology, we see from the above that there will exist exactly $g$ values of $k$ such that there does \emph{not} exist a rational function on $C$ with a pole of order exactly $k$ at $p$; these are called the \emph{gap values} of the point $p \in C$, and they comprise exactly the vanishing sequence of the canonical series $|K_C|$ at $p$. Moreover, it is clear that the complement in $\NN$ of the gap values---that is, the set of $k$ such that there does exist a rational function on $C$ with a pole of order exactly $k$ at $p$---for a semigroup, called the \emph{Weierstrass semigroup} of $p \in C$. Finally, the \emph{weight} $w_p$ of a Weierstrass point $p \in C$  is defined to be the weight $w(|K_C|,p)$ of $p$ as an inflectionary point of the canonical series.

From the general theory of ramification above, we see that a general point $p$ on any curve $C$ has gap sequence $(1,2,\dots,g)$, and correspondingly the semigroup $W_p = (0, g+1, g+2, \dots)$. A Weierstrass point is called \emph{normal} if it has weight 1; this is tantamount to saying that the gap sequence is $(1,2,\dots,g-1,g+1)$, or semigroup $(0, g, g+2, g+3, \dots)$. (The full Brill-Noether theorem tells us that a general curve $C$ has only normal Weierstrass points.) Finally, the Pl\"ucker formula tells us  the total weight of the Weierstrass points on a given curve $C$:
$$
\sum_{p \in C} w_p \; = \; g^3-g.
$$

There is still much we don't know about Weierstrass points in general. Most notably, we don't know what semigroups of finite index in $\NN$ occur as Weierstrass semigroups; an example of Buchweitz shows that not all semigroups occur, but there are also positive results, such as the statement ([EH]) that every semigroup of weight $w \leq g/2$ occurs, and its refinement and strengthening by Pflueger (\cite{**}).

\section{Finiteness of the automorphism group}

As an application of just a rudimentary knowledge of Weierstrass points, we will deduce a fundamental fact: that the automorphism group of a curve of genus $g\geq 2$ is finite. The idea behind the argument is simple: because the Weierstrass points of a curve $C$ are intrinsically defined, \emph{any automorphism of $C$ must carry Weierstrass points to Weierstrass points}. Since there are only finitely many Weierstrass points, then, it will suffice to show that the subgroup of $Aut(C)$ of automorphisms of $C$ that fix all the Weierstrass points individually is finite. In fact, the following two lemmas establish a strong version of this:

\begin{lemma}
Let $C$ be a smooth projective curve of genus $g \geq 2$, and $f: C \to C$ an automorphism of $C$.
\begin{enumerate}
\item If $f$ has $2g+3$ distinct fixed points, then $f$ is the identity; and
\item If $f$ has $2g+2$ distinct fixed points, then either $f$ is the identity or $C$ is hyperelliptic and $f$ is the hyperelliptic involution.
\end{enumerate}
\end{lemma}

\begin{proof}
There are two possible arguments here, one invoking the classical topology and applying the Lefschetz fixed point formula and the other more algebrao-geometric.

For the first, we recall the definition of the \emph{Lefschetz number} of a map $f : M \to M$ of a compact oriented real $n$-manifold $M$. This is the alternating sum of the traces of the action of $f$ on $H^i(X,\CC)$:
$$
L(f) := \sum_{i=0}^n {\rm Trace}\left(f^* : H^i(X,\CC) \to H^i(X,\CC)\right).
$$
The Lefschetz fixed point formula then says that if $f$ has isolated fixed points, the number of those points, properly counted, is equal to $L(f)$.

In the situation of a smooth projective curve over $\CC$, any automorphism other than the identity has isolated fixed points, and since the map is orientation-preserving each fixed point contributes positively to the total; thus the number of distinct fixed points is at most $L(f)$.

Now suppose $f: C \to C$ is any automorphism. Of necessity, $f$ acts as the identity on $H^0(C, \CC)$ and $H^2(C, \CC)$, so if we want to bound $L(f)$ we just have to say something about the action of $f$ on $H^1(C,\CC)$. To do this, note that the action of $f$ on $H^1(C,\CC)$ respects the \emph{Hodge decomposition}
$$
H^1(C,\CC)  = H^0(K_C) \oplus H^1(\cO_C).
$$  
Moreover, the action of $f$ on $H^0(K_C)$ preserves the definite Hermitian inner product
$$
H(\eta, \phi) = \int_C \eta \wedge \overline \phi,
$$
and it follows that \emph{the eigenvalues of the action of $f$ on $H^0(K_C)$ are all complex numbers of absolute value 1}, and likewise for the action on $H^1(\cO_C)$. The absolute value of the trace of $f^*: H^1(C,\CC) \to H^1(C,\CC)$ is at most 2g, and hence
$$
L(f) \leq 2 + 2g,
$$
proving the stated inequality in general.

Finally, if we have equality then $f$ must act as $-1$ on $H^1(C,\CC)$, and it follows (again from Lefschetz) that $f^2$ is the identity; applying Riemann-Hurwitz to the map from $C$ to the quotient $B = C/\langle f \rangle$ we may deduce that $B = \PP^1$, so $C$ is hyperelliptic and $f$ the hyperelliptic involution.
\end{proof}

An alternative, more algebraic argument for the lemma may be given using the intersection pairing on the surface $S = C \times C$ and applying the index theorem for surfaces. To carry this out, let $\Delta$ and $\Gamma \subset S$ be the diagonal and the graph of $f$ respectively, and let $\Phi_1$ and $\Phi_2 \subset S$ be fibers of the two projection maps; let $\delta, \gamma, \varphi_1$ and $\varphi_2 \in N(S)$ be the classes of these curves in the Neron-Severi group of $S$. We are trying to estimate the intersection number $b = \delta \cdot \gamma$.

We know all the other pairwise intersection number of these classes: the ones involving $\varphi_1$ or $\varphi_2$ are obvious; we have
$$
\delta^2 = 2 - 2g
$$
and since the automorphism $id_C \times f : C\times C \to C \times C$ carries $\Delta$ to $\Gamma$, we see that $\gamma^2 = 2-2g$ as well.

We can now apply the index theorem for surfaces to deduce our inequality. To keep things relatively simple, let's introduce two new classes: set
$$
\delta' = \delta - \varphi_1 - \varphi_2 \quad \text{and} \quad \gamma' = \gamma - \varphi_1 - \varphi_2,
$$
so that $\delta'$ and $\gamma'$ are orthogonal to the class $\varphi_1 + \varphi_2$. Since $\varphi_1 + \varphi_2$ has positive self-intersection, the index theorem tells us that the intersection pairing must be negative definite on the span $\langle \delta',\gamma' \rangle \subset N(S)$. In particular, the determinant of the intersection matrix


\begin{center}
\begin{tabular}{c|c|c}
& $\delta'$ &  $\gamma'$  \\
\hline
$\delta'$ & $-2g$ & $b-2$ \\
\hline
$\gamma'$ & $b-2$ & $-2g$ 
\end{tabular}
\end{center}
(where again $b = \gamma \cdot \delta$) must be positive, from which our inequality follows.

Having established an upper bound on the number of fixed points an automorphism $f$ of $C$ (other than the identity) may have, it remains to find a lower bound on the number of distinct Weierstrass points; this is the content of the next lemma.


\begin{lemma}
If $C$ is a smooth projective curve of genus $g \geq 2$, then $C$ has at least $2g+2$ distinct Weierstrass points; and if it has exactly $2g+2$ Weierstrass points it is hyperelliptic.
\end{lemma}

\begin{proof}
Let $p \in C$ be any point, and $w_1=w_1(p),\dots,w_g = w_g(p)$ the ramification sequence of the canonical series $|K_C|$ at $p$. By definition, 
$$
h^0(K_C(-(w_i+i)p)) = g - i.
$$
Applying Clifford's theorem we have
$$
g-i \leq \frac{2g - 2 - w_i - i}{2} + 1;
$$
solving, we see that
$$
w_i \leq i
$$
and hence
$$
w_p \leq \binom{g}{2}
$$
where $w_p$ is the total weight of $p$ as a Weierstrass point. Since the total weight of the Weierstrass points on $C$ is $g^3-g$ by Pl\"ucker, we see that the number of distinct Weierstrass points must be at least
$$
\frac{g^3-g}{\binom{g}{2}} = 2g+2.
$$
Finally, by the strong form of Clifford, equality here implies that the curve is hyperelliptic.
\end{proof}




\section{Proof of (half of) the Brill-Noether theorem}

In its most basic form, the Brill-Noether theorem asserts for any $d, g$ and $r$ that
\begin{enumerate}
\item if $\rho(g,r,d) := g - (r+1)(g-d+r) \geq 0$, then every curve $C$ of genus $g$ possesses as $g^r_d$; and
\item if $\rho < 0$ then a general curve of genus $g$ does not possess a $g^r_d$.
\end{enumerate}

The first part, often called the ``existence half" of Brill-Noether, was originally proved by Kempf (\cite{}) and Kleiman-Laksov (\cite{}); both proofs relied on an application of the Thom-Porteous formula to a particular map of vector bundles on the Jacobian of $C$. An account of this argument may also be found in Appendix A of \cite{}. 

We will now apply the notion of inflectionary points, and the Pl\"ucker formula in particular, to deduce the second half of the statement above---the ``nonexistence half." We will then go back and deduce some of the other parts of the full Brill-Noether statement from the same set-up.

The basic approach here is to consider a family of curves $\pi : \cC \to B$, where
\begin{enumerate}
\item $B$ is a smooth curve, with distinguished point $0 \in B$;
\item for all $b \neq 0 \in B$, the fiber $C_b = \pi^{-1}(b)$ is a smooth, projective curve of genus $g$;  and
\item the fiber $C_0$ over $0$ is a rational curve with $g$ ordinary cusps.
\end{enumerate}

We will establish the

\begin{theorem}
If $\cC \to B$ is a family of curves as above, then for general $b \in B$ the fiber $C_b$ does not possess a $g^r_d$ with $\rho < 0$.
\end{theorem}


The basic outline of the argument is by contradiction, but straightforward: we assume that the general curve $C_b$ in the family does have a $g^r_d$, consider what the limit of those $g^r_d$s might look like and, using our knowledge of the relatively simple curve $C_0$, arrive at a contradiction. By way of notation, let $B^\circ = B \setminus \{0\}$ and let $\cC^\circ = \pi^{-1}(B^\circ)$ be the complement in $\cC$ of the special fiber. The proof proceeds essentially in four/five steps.

\

\noindent {\bf Step 0: Existence of such a family}

\

\noindent {\bf Step 1: Finding a family of $g^r_d$s over $B^\circ$}

Suppose, accordingly, that the general curve $C_b$ in the family does have a line bundle  of degree $d$ with $r+1$ sections. The first thing to observe is that, possibly after a base change, we can pick out one such line bundle $\cL_b$ for each $b \neq 0$, varying regularly with $b$; or, in other words \emph{there exists a  line bundle $\cL^\circ$ on the complement $\cC^\circ$ of the special fiber such that}
$$
\deg(\cL^\circ|_{C_b}) = d \quad \text{and} \quad h^0(\cL^\circ|_{C_b}) \geq r+1
$$
for all $b \neq 0 \in B$. \fix{need to give argument for this assertion}

\

\noindent {\bf Step 2: Extending the line bundle $\cL^\circ$ to a sheaf on all of $\cC$}

Next, we want to extend $\cL^\circ$ to a sheaf on all of $\cC$. We claim that \emph{there exists a torsion-free sheaf $\cL$ on all of $\cC$ such that $\cL|_{\cC^\circ} \cong \cL^\circ$}.

To see this, we choose an auxiliary line bundle $\cM$ on $\cC$ with relative degree $e > d + 2g$ (for example, embed $\cC$ is projective space and take $\cM = \cO_{\cC}(m)$ for large $m$); in keeping with our notational conventions, let $\cM^\circ$ be the restriction of $\cM$ to $\cL^\circ$. Consider the line bundle 
$$
\cN^\circ = (\cL^\circ)^* \otimes \cM^\circ.
$$
The bundle $\cN^\circ$ has lots of sections: the direct image is locally free of rank $e-g+1 > 0$, and after restricting to an open neighborhood of $0 \in B$ we can assume it's generated by them. Choose a section $\sigma$ of $\cN^\circ$; let $D^\circ \subset \cC^\circ$ be its zero divisor, and let $D \subset \cC$ be the closure of $D^\circ$ in $\cC$. Now, away from $C_0$ we can write
$$
\cL^\circ = (\cN^\circ)^* \otimes \cM^\circ = \cI_{D^\circ/\cC^\circ} \otimes \cM^\circ
$$
and accordingly the sheaf
$$
\cL := \cI_{D/\cC} \otimes \cM
$$
is the desired sheaf. Note that this need not be locally free: the total space $\cC$ of our family may not be smooth at the cusps of the special fiber $C_0$ (even if the family we originally started with had smooth total space, the base change called for in Step 1 would yield a family with  total space singular at the cusps of $C_0$), and if $D$ passes through any of these points it need not be Cartier.

In sum, if the general fiber $C_b$ of our family has a $g^r_d$, we can conclude that the special fiber $C_0$ has a torsion-free sheaf $\cL_0$ with 
$$
c_1(\cL_0) = d;
$$
and, by upper-semicontinuity of cohomolgy,
$$
h^0(\cL_0) \geq r+1.
$$

\

\noindent {\bf Step 3: Local description of the sheaf $\cL_0$}

The next question is, what does $\cL_0$ look like if it's not locally free? Here we have a basic lemma:

\begin{lemma}\label{torsion free at cusp}
Let $p$ be a  cusp of a curve $C$. If $\cF$ is a torsion-free sheaf on $C$, then in a neighborhood of $p$ in $C$ the sheaf $\cF$ is either locally free or isomorphic to the ideal sheaf $\cI_{p/C}$ of $p$ in $C$.
\end{lemma}


\begin{exercise}
\begin{enumerate}
\item Show  that the conclusion of Lemma~\ref{torsion free at cusp} holds in case $p$ is a node of $C$
\item Show by example that the conclusion of Lemma~\ref{torsion free at cusp} is false in case $p$ is either a tacnode or a triple point of $C$.
\end{enumerate}
\end{exercise}

\begin{proof}

\end{proof}

\

\noindent {\bf Step 4: Applying the Pl\"ucker formula}

Now, back to our family $\pi : \cC \to B$ of curves. We have assumed that for some $d$ and $r$ with $\rho(g,r,d) < 0$ the general curve $C_b$ has a $g^r_d$, and deduced that the special fiber $C_0$ has a rank 1 torsion-free sheaf $\cL_0$ of degree $d$ with at least $r+1$ sections; we now have to derive from this a contradiction.

To see most clearly where this contradiction comes from, let's start with the simplest case: where $\cL_0$ is indeed locally free. In this case, let $\nu :  C^\nu \cong \PP^1 \to C$ be the normalization of $C$ and let $p_1,\dots,p_g \in \PP^1$ be the points lying over the cusps of $C_0$. We have
$$
\nu^*(\cL) \cong \cO_{\PP^1}(d)
$$  
and 
$$
V = \nu^*(H^0(\cL_0)) \subset H^0(\cO_{\PP^1}(d))
$$
is an $(r+1)$-dimensional space of sections. (If $H^0(\cL_0) > r+1$, just choose any $(r+1)$-dimensional subspace.) 

Now, given that any section $\sigma \in V \subset H^0(\cO_{\PP^1}(d))$ is pulled back from the cuspidal curve $C$, we see that \emph{$\sigma$ cannot vanish to order exactly 1 at the point $p_i \in \PP^1$ lying over any of the cusps of $C_0$}. It follows that for each $k$ the ramification index 
$$
\alpha_1(V,p_k) \geq 1 
$$
and hence in  general $\alpha_1(p_k,V) \geq 1$ for all $i \geq 1$. In particular, the weight of the inflectionary point $p_i$ for the linear series $V$ satisfies
$$
w(V, p_k) \geq r
$$
and correspondingly
$$
\sum_{k=1}^g w(V, p_k) \geq rg
$$
But the Pl\"ucker formula~\ref{} tells us that the total weight of all inflectionary points for the series $V$ is
$$
\sum_{p \in \PP^1} w(V,p) = (r+1)(d-r)
$$
and there's our contradiction: by the hypothesis that 
$$
\rho(g,r,d) := g - (r+1)(g-d+r) < 0
$$
we have $rg > (r+1)(d-r)$.

Finally, the case where $\cL_0$ is not locally free is if anything even easier. Suppose now that the sheaf $\cL$ fails to be locally free at $l$ of the cusps of $C_0$, say $\nu(p_1),\dots,\nu(p_l)$. Again, we can pull $\cL$ back to $\PP^1$; again we have
$$
\nu^*(\cL_0) \cong \cO_{\PP^1}(d);
$$  
and again we pull back section of $\cL$ to arrive at a linear system
$$
V = \nu^*(H^0(\cL_0)) \subset H^0(\cO_{\PP^1}(d))
$$
of degree $d$ and genus $g$ on $\PP^1$. The only difference here is that sections of $V$ all vanish at $p_1,\dots,p_l$, so that we have
$$
w(V,p_k) \geq 
\begin{cases}
r+1 &\text{ if } k \leq l; \text{ and} \\
r  &\text{ if } k > l.
\end{cases}
$$
so that
$$
\sum_{k=1}^g w(V, p_k) \geq rg + l
$$
and our contradiction is even more of a contradiction!

\section{Corollaries and extensions of our proof}

\begin{enumerate}
\item strong form of BN: if $(C,p_1,\dots,p_n)$ is a general pointed curve, the variety of linear series with specified ramification at the $p_i$ has expected dimension.
\item actual dimension estimate for $W^r_d$ (not just existence or nonexistence)
\item generic $g^2_*$ is birational; $g^3_*$ is very ample (on a general curve)
\item general $g^r_d$ has only simple ramification
\end{enumerate}
%footer for separate chapter files

\ifx\whole\undefined
%\makeatletter\def\@biblabel#1{#1]}\makeatother
\makeatletter \def\@biblabel#1{\ignorespaces} \makeatother
\bibliographystyle{msribib}
\bibliography{slag}

%%%% EXPLANATIONS:

% f and n
% some authors have all works collected at the end

\begingroup
%\catcode`\^\active
%if ^ is followed by 
% 1:  print f, gobble the following ^ and the next character
% 0:  print n, gobble the following ^
% any other letter: normal subscript
%\makeatletter
%\def^#1{\ifx1#1f\expandafter\@gobbletwo\else
%        \ifx0#1n\expandafter\expandafter\expandafter\@gobble
%        \else\sp{#1}\fi\fi}
%\makeatother
\let\moreadhoc\relax
\def\indexintro{%An author's cited works appear at the end of the
%author's entry; for conventions
%see the List of Citations on page~\pageref{loc}.  
%\smallbreak\noindent
%The letter `f' after a page number indicates a figure, `n' a footnote.
}
\printindex[gen]
\endgroup % end of \catcode
%requires makeindex
\end{document}
\else
\fi
