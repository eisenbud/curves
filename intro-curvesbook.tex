%header and footer for separate chapter files

\ifx\whole\undefined
\documentclass[12pt, leqno]{book}
\usepackage{graphicx}
\input style-for-curves.sty
\usepackage{hyperref}
\usepackage{showkeys} %This shows the labels.
%\usepackage{SLAG,msribib,local}
%\usepackage{amsmath,amscd,amsthm,amssymb,amsxtra,latexsym,epsfig,epic,graphics}
%\usepackage[matrix,arrow,curve]{xy}
%\usepackage{graphicx}
%\usepackage{diagrams}
%
%%\usepackage{amsrefs}
%%%%%%%%%%%%%%%%%%%%%%%%%%%%%%%%%%%%%%%%%%
%%\textwidth16cm
%%\textheight20cm
%%\topmargin-2cm
%\oddsidemargin.8cm
%\evensidemargin1cm
%
%%%%%%Definitions
%\input preamble.tex
%\input style-for-curves.sty
%\def\TU{{\bf U}}
%\def\AA{{\mathbb A}}
%\def\BB{{\mathbb B}}
%\def\CC{{\mathbb C}}
%\def\QQ{{\mathbb Q}}
%\def\RR{{\mathbb R}}
%\def\facet{{\bf facet}}
%\def\image{{\rm image}}
%\def\cE{{\cal E}}
%\def\cF{{\cal F}}
%\def\cG{{\cal G}}
%\def\cH{{\cal H}}
%\def\cHom{{{\cal H}om}}
%\def\h{{\rm h}}
% \def\bs{{Boij-S\"oderberg{} }}
%
%\makeatletter
%\def\Ddots{\mathinner{\mkern1mu\raise\p@
%\vbox{\kern7\p@\hbox{.}}\mkern2mu
%\raise4\p@\hbox{.}\mkern2mu\raise7\p@\hbox{.}\mkern1mu}}
%\makeatother

%%
%\pagestyle{myheadings}

%\input style-for-curves.tex
%\documentclass{cambridge7A}
%\usepackage{hatcher_revised} 
%\usepackage{3264}
   
\errorcontextlines=1000
%\usepackage{makeidx}
\let\see\relax
\usepackage{makeidx}
\makeindex
% \index{word} in the doc; \index{variety!algebraic} gives variety, algebraic
% PUT a % after each \index{***}

\overfullrule=5pt
\catcode`\@\active
\def@{\mskip1.5mu} %produce a small space in math with an @

\title{Personalities of Curves}
\author{\copyright David Eisenbud and Joe Harris}
%%\includeonly{%
%0-intro,01-ChowRingDogma,02-FirstExamples,03-Grassmannians,04-GeneralGrassmannians
%,05-VectorBundlesAndChernClasses,06-LinesOnHypersurfaces,07-SingularElementsOfLinearSeries,
%08-ParameterSpaces,
%bib
%}

\date{\today}
%%\date{}
%\title{Curves}
%%{\normalsize ***Preliminary Version***}} 
%\author{David Eisenbud and Joe Harris }
%
%\begin{document}

\begin{document}
\maketitle

\pagenumbering{roman}
\setcounter{page}{5}
%\begin{5}
%\end{5}
\pagenumbering{arabic}
\tableofcontents
\fi


\setlength{\parskip}{5pt}

\addtocounter{chapter}{-1}
\chapter{Introduction}
\label{IntroChapter}

\begin{quote}
\small\sf
``Es gibt nach des Verf. Erfarhrung kein besseres Mittel, Geometrie zu lernen, als
das Studium des Schubertschen `Kalk\"uls der abz\"ahlenden Geometrie'.''

(There is, in the author's experience, no better means of learning geometry than
the study of Schubert's ``Calculus of Enumerative Geometry.")

--B. L. van der Waerden (in a Zentralblatt review of an introduction to enumerative geometry
by Hendrik de Vries).
\bigskip

\end{quote}

%\noindent
%{\bf 1066 \& All That} (\cite{1066}) is ``A memorable history of England, comprising all the parts you can remember, including one hundred and three \emph{good} things, five \emph{bad} kings, and two \emph{genuine} dates\dots. History is not what you thought. \emph{It is what you can remember.} 
%\dots
%
%``In the year 1066 occurred the other memorable date in English History, viz. \emph{William the Conquereor, Ten Sixty-six.} 
%This is also called \emph{The Battle of Hastings,} and was when William I (1066) conquered England at the Battle of Senlac (\emph{Ten Sixty-six})\dots 
%The Norman Conquest was a Good Thing, as from this time onwards England stopped being conquered and thus was able to become top nation.''


\section{Why you want to read this book}

Algebraic geometry is an old subject. You could make the case that it dates back to Descartes, whose introduction of coordinates in the plane and in space made it possible to relate the algebra of polynomials to the geometry of their zero loci; in any event, mathematicians have been doing what is recognizably algebraic geometry for more than two centuries.

And within algebraic geometry, the study of algebraic curves is naturally the oldest topic. After all, the study of polynomials in two variables via their zero sets is the first nontrivial case of the general paradigm of algebraic geometry. Moreover, the theory of algebraic curves received a tremendous boost in the 19th century from the work of mathematicians studying complex analysis, from the introduction of the notion of Riemann surfaces in **** on.

What all this has meant is that the subject of algebraic curves is one of the richest in algebraic geometry, if not in all of mathematics. Examples abound; if you want to know whether a given hypothesis holds, you can almost always find well-understood special cases that will allow you to test it. And some of the fundamental constructions of algebraic geometry, like the construction of moduli spaces and their description, can be carried out in the setting of algebraic curves with a degree of precision and detail far beyond what has been possible (so far!) in higher dimensions. 

The richest subject in what is arguably the richest branch of mathematics\footnote{Number theorists may quibble}---of course you want to read this book! 

\section{Why we wrote this book}

The wealth of beauty, both in theory and in examples, certainly makes the study of algebraic curves an attractive prospect. But it comes at a price: to absorb in detail all the things we've learned over the centuries about algebraic curves would take years, if not decades. This is, in essence, the conundrum facing anyone who undertakes to write a book on the subject: how to describe the wealth of our knowledge of the subject without writing an encyclopedia.

For better or worse, this is our attempt to do exactly that.

\section{What's with the title?}




\section{What's in this book}


\begin{quote}
\small\sf
We are dealing here with a fundamental and almost paradoxical difficulty. Stated briefly, it is that learning is sequential but knowledge is not. A branch of mathematics... consists of an intricate network network of interrelated facts, each of which contributes to the understanding of those around it. When confronted with this network for the first time, we are forced to follow a particular path, which involves a somewhat arbitrary ordering of the facts.

--Robert Osserman.

\end{quote}



Where to begin? To start with the technical underpinnings of a subject risks losing the reader before the point of all that preliminary work is made clear; but to defer the logical foundations carries its own dangers---as the unproved assertions mount up, the reader may well feel adrift.

Intersection theory poses a particular challenge in this regard, since the development of its foundations is so demanding. It is possible, however, to state fairly simply and precisely the main foundational results of the subject, at least in the limited context of intersections on smooth projective varieties. The reader who is willing to take these results on faith for a little while, and accept this restriction, can then be shown ``what the subject is good for," in the form of examples and applications. This is the path we've chosen in this book, as we'll now describe.

\subsection{Overture}



\subsection{Relation of this book to ``Intersection Theory''} 


\subsection{Keynote problems} To highlight the sort of problems we'll  learn to solve, and to motivate the material we present, we'll begin each chapter with some {\it keynote questions}. 
%We urge the reader to pause for a moment and think about each one before diving into the body of the chapter.


\section{Prerequisites, notation and conventions}

\subsection{What you need to know before starting}
When it comes to prerequisites, there are two distinct questions: what you should know to start reading this book; and what you should be prepared to learn along the way. 


\subsection{Language}
 

Throughout this book, a \emph{scheme} $X$ will be a scheme of finite type over an algebraically closed field
$K$. We use the term \emph{integral} to mean reduced and irreducible; by a \emph{variety} we will mean an integral scheme. (The terms ``curve" and ``surface," however, refer to one-dimensional and two-dimensional schemes; in particular, they are not presumed to be integral.) A subvariety $Y \subset X$ will be presumed closed unless otherwise specified.
 If $X$ is a variety we write $K(X)$ for the field of rational functions on $X$. A \emph{sheaf} on $X$ will be a coherent sheaf unless otherwise noted.
 
By a \emph{point}
we mean a closed point. 
Recall that a \emph{locally closed} subscheme $U$ of a scheme $X$ is 
a scheme that is an open subset of a closed subscheme of $X$. We generally use the term
``subscheme'' (without any modifier) to mean a closed subscheme, and similarly for ``subvariety."

A consequence of the finite type hypothesis
is that to any subscheme $Y$ of 
$X$ has a \emph{primary decomposition}: locally, we can write the ideal of $Y$ as an irredundant intersection of primary ideals with distinct associated primes. We can correspondingly write $Y$ globally as an irredundant union of closed subschemes $Y_i$ whose supports are distinct subvarieties of $X$. In this expression, the subschemes $Y_i$ whose supports are maximal---corresponding to the minimal primes in the primary decomposition---are uniquely determined by $Y$; they are called the \emph{irreducible components} of $Y$. The remaining subschemes are called \emph{embedded components}; they are not determined by $Y$, though their supports are.

If a family of objects is parametrized by a scheme $B$, we will say that a ``general" member of the family has a given property $P$ if the set $U(P) \subset B$ of members of the family with that property contains an open dense subset of $B$. When we say that a ``very general" member has this property  we will mean that $U(P)$ contains the complement of a countable union of proper subvarieties of $B$.


By the \emph{projectivization} $\PP V$ of a vector space $V$ we'll mean the scheme $\proj(\Sym^* V^*)$; this is the space whose closed points correspond to one-dimensional subspaces of $V$.

If $X$ and $Y \subset \PP^n$ are subvarieties of projective space, we define the \emph{join} of $X$ and $Y$, denoted $\overline{X\,Y}$, to be the closure of the union of lines meeting $X$ and $Y$ at distinct points. If $X = \Gamma \subset \PP^n$ is a linear space, this is just the cone over $Y$ with vertex $\Gamma$; if $X$ and $Y$ are both linear subspaces, this is simply their span.

There is a one-to-one correspondence between vector bundles on a scheme $X$ and locally free sheaves on $X$. We will use the terms interchangeably, generally preferring ``line bundle'' and ``vector bundle'' to ``invertible sheaf'' and ``locally free sheaf''.


By a \emph{linear system}, or \emph{linear series}, on a scheme $X$ we will mean a pair $(\cL, V)$ where $\cL$ is a line bundle on $X$ and $V \subset H^0(\cL)$ a vector space of sections. Associating to a section $\s \in H^0(\cL)$ its zero locus $V(\s)$, we can also think of a linear system as a family $\cD = \{ V(\s) \mid \s \in V\}$ of subschemes parametrized by the projective space $\P V$; in this setting, we will sometimes refer to the linear system $\cD$. By the \emph{dimension} of the linear series we mean the dimension of the projective space $\P V$ parametrizing it; that is, $\dim V - 1$. Specifically, a one-dimensional linear system is called a \emph{pencil}; a two-dimensional system is called a \emph{net} and a three-dimensional linear system is called a \emph{web}.

We write $\cO_{X,Y}$
for the local ring of $X$ along $Y$, and more generally, if $\cF$ is a sheaf of
$\cO_{X}$-modules then we write $\cF_Y$ for the 
corresponding $\cO_{X,Y}$-module.

We can identify the Zariski tangent space to affine space $\AA^n$ with $\AA^n$ itself. If $X \subset \AA^n$ is a subscheme, by the \emph{affine tangent space} to $X$ at a point $p$ we will mean the affine linear subspace $p + T_pX \subset \AA^n$. If $X \subset \P^n$ is a subscheme, by the \emph{projective tangent space} to $X$ at $p \in X$, denoted $\T_pX \subset \P^n$, we will mean the closure in $\P^n$ of the affine tangent space to $X \cap \AA^n$ for any open subset $\AA^n \subset \P^n$ containing $p$. Concretely, if $X$ is the zero locus of polynomials $F_\alpha$ (that is, $X = V(I) \subset \P^n$ is the subscheme defined by the ideal $I = (\{F_\alpha\}) \subset K[Z_0,\dots,Z_n]$), the projective tangent space is the common zero locus of the linear forms
$$
\alpha Z) = \frac{\partial F_\alpha}{\partial Z_0}(p)Z_0 + \dots + \frac{\partial F_\alpha}{\partial Z_n}(p)Z_n.
$$

By a ``one-parameter family" we will always mean a family $X \to B$ with $B$ smooth and one-dimensional (an open subset of a smooth curve, or spec of a DVR or power series ring in one variable), with marked point $0 \in B$. In this context, ``with parameter $t$" means $t$ is a local coordinate on the curve, or a generator of the maximal ideal of the DVR or power series ring.


\subsection{Basic results on dimension and smoothness}



\newpage

\

\begin{quote}
\small\sf
We are all familiar with the after-the-fact tone---weary, self-justificatory, aggrieved, apologetic---shared by ship captains appearing before boards of inquiry to explain how they came to run their vessels aground, and by authors composing forewords.

--John Lanchester 
\bigskip

\end{quote}



\

%footer for separate chapter files

\ifx\whole\undefined
%\makeatletter\def\@biblabel#1{#1]}\makeatother
\makeatletter \def\@biblabel#1{\ignorespaces} \makeatother
\bibliographystyle{msribib}
\bibliography{slag}

%%%% EXPLANATIONS:

% f and n
% some authors have all works collected at the end

\begingroup
%\catcode`\^\active
%if ^ is followed by 
% 1:  print f, gobble the following ^ and the next character
% 0:  print n, gobble the following ^
% any other letter: normal subscript
%\makeatletter
%\def^#1{\ifx1#1f\expandafter\@gobbletwo\else
%        \ifx0#1n\expandafter\expandafter\expandafter\@gobble
%        \else\sp{#1}\fi\fi}
%\makeatother
\let\moreadhoc\relax
\def\indexintro{%An author's cited works appear at the end of the
%author's entry; for conventions
%see the List of Citations on page~\pageref{loc}.  
%\smallbreak\noindent
%The letter `f' after a page number indicates a figure, `n' a footnote.
}
\printindex[gen]
\endgroup % end of \catcode
%requires makeindex
\end{document}
\else
\fi


