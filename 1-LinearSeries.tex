%header and footer for separate chapter files

\ifx\whole\undefined
\documentclass[12pt, leqno]{book}
\usepackage{graphicx}
\input style-for-curves.sty
\usepackage{hyperref}
\usepackage{showkeys} %This shows the labels.
%\usepackage{SLAG,msribib,local}
%\usepackage{amsmath,amscd,amsthm,amssymb,amsxtra,latexsym,epsfig,epic,graphics}
%\usepackage[matrix,arrow,curve]{xy}
%\usepackage{graphicx}
%\usepackage{diagrams}
%
%%\usepackage{amsrefs}
%%%%%%%%%%%%%%%%%%%%%%%%%%%%%%%%%%%%%%%%%%
%%\textwidth16cm
%%\textheight20cm
%%\topmargin-2cm
%\oddsidemargin.8cm
%\evensidemargin1cm
%
%%%%%%Definitions
%\input preamble.tex
%\input style-for-curves.sty
%\def\TU{{\bf U}}
%\def\AA{{\mathbb A}}
%\def\BB{{\mathbb B}}
%\def\CC{{\mathbb C}}
%\def\QQ{{\mathbb Q}}
%\def\RR{{\mathbb R}}
%\def\facet{{\bf facet}}
%\def\image{{\rm image}}
%\def\cE{{\cal E}}
%\def\cF{{\cal F}}
%\def\cG{{\cal G}}
%\def\cH{{\cal H}}
%\def\cHom{{{\cal H}om}}
%\def\h{{\rm h}}
% \def\bs{{Boij-S\"oderberg{} }}
%
%\makeatletter
%\def\Ddots{\mathinner{\mkern1mu\raise\p@
%\vbox{\kern7\p@\hbox{.}}\mkern2mu
%\raise4\p@\hbox{.}\mkern2mu\raise7\p@\hbox{.}\mkern1mu}}
%\makeatother

%%
%\pagestyle{myheadings}

%\input style-for-curves.tex
%\documentclass{cambridge7A}
%\usepackage{hatcher_revised} 
%\usepackage{3264}
   
\errorcontextlines=1000
%\usepackage{makeidx}
\let\see\relax
\usepackage{makeidx}
\makeindex
% \index{word} in the doc; \index{variety!algebraic} gives variety, algebraic
% PUT a % after each \index{***}

\overfullrule=5pt
\catcode`\@\active
\def@{\mskip1.5mu} %produce a small space in math with an @

\title{Personalities of Curves}
\author{\copyright David Eisenbud and Joe Harris}
%%\includeonly{%
%0-intro,01-ChowRingDogma,02-FirstExamples,03-Grassmannians,04-GeneralGrassmannians
%,05-VectorBundlesAndChernClasses,06-LinesOnHypersurfaces,07-SingularElementsOfLinearSeries,
%08-ParameterSpaces,
%bib
%}

\date{\today}
%%\date{}
%\title{Curves}
%%{\normalsize ***Preliminary Version***}} 
%\author{David Eisenbud and Joe Harris }
%
%\begin{document}

\begin{document}
\maketitle

\pagenumbering{roman}
\setcounter{page}{5}
%\begin{5}
%\end{5}
\pagenumbering{arabic}
\tableofcontents
\fi


\chapter{Linear series and morphisms to projective space}\label{linear series}

This Chapter and the next are intended largely as a review of material that should be familiar, couched in the language we will use throughout the book. Experienced readers 
may want to skip it, and return if and when it is needed. Nearly all of the material can be found in texts such as \cite{H}.

%\section{Morphisms to projective space and families of Cartier divisors}

In this book we work over the field of complex numbers $\CC$, though much of what we do
could be done over any field. 
Schemes are assumed quasi-projective, and \emph{varieties} are reduced and irreducible---that is, \emph{integral}---schemes. ``Points'' will always be closed points unless we explicitly say otherwise.

A complex projective smooth curve may  be regarded abstractly as a compact Riemann surface. A Riemann surface is a compact, orientable 2-manifold with a complex structure, but it's hard to see the complex structure. For example though there
are $\cC^\infty$ isotopies that take any point to any other, the algebraic local rings of the points may well all be non-isomorphic. This variation and much else is revealed in the projective geometry of embeddings. Throughout this book we will study curves through their embeddings and other maps to projective space, defined in terms of linear series. 

Though affine varieties are defined by the functions on them---the coordinate ring, which is defined by the variety---the globally defined functions on projective varieties are constant, and
the homogeneous coordinate ring of a variety $X\subset \PP^n$ is not characteristic of $X$, but of $X$ together with some auxiliary data, 
either a family of divisors of a simple kind, or an invertible sheaf and a collection of global sections. Such data, more generally, can be used to define a morphism or a birational map
to a projective space, and we begin by describing these relations.

Let $\phi: C\to \PP^{r}$ be a morphism from a smooth curve $C$. If $H\subset \PP^r$ is a hyperplane that does not contain $\phi(C)$, then the preimage of $H$ is a finite set of points on $C$, with multiplicities when $H$ is tangent to $\phi(C)$ or (possibly) when $H$ passes through a singular point of $\phi(C)$. Such a set of points with non-negative integer multiplicities is called an \emph{effective divisor} on $C$; more generally, a \emph{divisor} (sometimes called a \emph{Weil divisor}) on a scheme $X$ is an integral linear combination of codimension 1 subvarieties, and it is called \emph{effective} if the coefficients are all non-negative. The \emph{degree} of a divisor on a smooth curve
is the sum of the coefficients. 

The divisors that arise as the pullbacks of general hyperplanes are special: since a hyperplane is defined by just one equation, which is locally given by the vanishing of a function, the pullback of a hyperplane will be locally defined by the vanishing of a single function,
and for a general hyperplane, this will be a nonzerodivisor; that is, it is an  \emph{effective Cartier divisor}. See \cite[pp. 140-146]{H} for more information; on a smooth curve every divisor is Cartier, so the difference between Weil and Cartier divisors will not be an issue for us.
The  word ``local'' scattered through the previous paragraph is needed because, if $X$ is a projective variety, then the only algebraic functions $X\to \CC$ are constant functions. Equivalently, the only projective subvarieties of an affine variety are points.

If we are given the family of divisors on $C$ that are the preimages of the intersections of hyperplanes with  $\phi(C)$, we can recover the morphism $\phi$ set-theoretically: it takes a point $p\in C$ to the point of projective space that is the intersection of those
hyperplanes whose preimages contain $p$. 

The relationship of two divisors on $C$ that are preimages of intersections of $\phi(C)$ with hyperplanes is simple to describe: If hyperplanes
$H, H'\subset \PP^r$ are defined by the linear forms $h, h'$  then $h/h'$ has a simple pole along $H'$---we may say that it ``vanishes along $H$'' to degree $-1$.
In this sense the divisor $H-H'$ on $\PP^n$ is defined by the rational function $\lambda= h/h'$. If neither $H$ nor $H'$ contain $\phi(C)$ then the pullback of $\lambda$ is a well-defined, nonzero rational function on $C$, and the divisor 
$\phi^{-1}(\phi(C)\cap H') - \phi^{-1}(\phi(C)\cap H)$ is defined by the pullback  $\phi^*(\lambda) := \lambda \circ \phi$. Thus the divisors arising from a given morphism to $\PP^{r}$ differ by the divisors of zeros minus poles of rational functions on $C$. 

If $C$ is a smooth curve then the local ring $\sO_{C,p}$ of $C$ at a point $p$ is a discrete valuation ring, and if $\pi$ is a generator of the maximal ideal of $\sO_{C,p}$, then any rational
function $\lambda$ on $C$ can be expressed uniquely as $u\pi^k$ where $u\in \sO_{C,p}$ is a unit and $k\in \ZZ$. We say that $k$ is
the \emph{order} of $\lambda$ at $p$, and write $k = \ord_p \lambda$. We associate to $\lambda$  the divisor
$$
(\lambda) := \sum_{p\in C} (\ord_p\lambda)p.
$$
The \emph{class group} $\pic(C)$  of $C$ is defined to be the the group of divisors on $C$ modulo the divisors of rational functions.
Thus the divisors on $C$ that are preimages of intersections of $\phi(C)$ with different hyperplanes all belong to the same
\emph{divisor class}, and form a linear series in the sense of the following section.

It is important to note that the degree of the divisor associated to a rational function on a smooth curve is always 0: this is evident on $\PP^1$, where rational 
functions are the ratios of two forms of the same degree; and a rational function $\phi$ on a smooth curve may be regarded as the pullback of a rational function
on $\PP^1$ under the map given by $\phi$; since any non-constant map of smooth curves is flat, the pullback multiplies the degree of a divisor by the degree 
of the covering. 

\section{Linear series}

We want to understand morphisms to $\PP^r$ more than set-theoretically, and we want to be able to produce them from data on $C$. For this we use the notion of linear series (sometimes called linear system). Our goal in the next sections is to explain this connection:

\begin{definition}
 A \emph{linear series} on a scheme $C$ is a pair $\sV  = (\sL, V)$ where $\sL$ is an invertible sheaf (defined in Section ~\ref{Invertible sheaves}) on $C$ and
 $V$ is a vector space of global sections of $\sL$. We define the \emph{dimension} of the linear series to be 
 $$
 \dim \sV := \dim V -1,
 $$
If $\sV$ has degree $d$ and dimension $r$ we say that \emph{$\sV$ is a $g^r_d$}---the old language means that $\sV$ it represents a group of $d$ points ``moving'' up to linear equivalence with $r$ degrees of freedom. If $C$ is a smooth curve, then the divisor of zeros and the divisor of poles of a (rational) section of $\sV$ are subschemes, and the \emph{degree} of $\sV$ is by definition the degree of the divisor of poles minus the degree of the divisor of zeros of any of its sections; since the 
 ratio of two sections is a rational function, this is independent of the section chosen. \end{definition}

The apparently perverse definition of the dimension of the linear series is because morphisms of a curve to $\PP^r$ are connected to linear series that have dimension $r$ in the sense above.
\fix{need def of base point before this!}
\begin{theorem}\label{morphisms and linear series}
There is a natural bijection between the set of nondegenerate morphisms $\phi : C \to \PP^r$ modulo $PGL_{r+1}$, and basepoint-free linear series of dimension $r$ on $C$, up to isomorphism.\end{theorem}

Here ``nondegenerate" means the image of the morphism $\phi$ is not contained in any hyperplane; the dimension of the series
 $\sV  = (\sL, V)$ is $\dim_\CC V -1$; and basepoint-free means that there is no point of $C$ where all the sections in $V$
vanish.

The proof of Theorem~\ref{morphisms and linear series} is contained in sections~\ref{morphism from series}
and \ref{series from morphism}.


\subsection{Invertible sheaves}\label{Invertible sheaves}

Recall first that a \emph{coherent sheaf} $\sL$ on a scheme $X$ may be defined by
giving 
\begin{itemize}
 \item An open affine cover $\{U_{i}\}$ of $X$; 
 \item For each $i$, a finitely generated $\sO_{X}(U_{i})$-module $L_{i}$;
 \item For each $i,j$, an isomorphism $\sigma_{i,j}: L_{i}\mid_{U_{i}\cap U_{j}} \to L_{j}\mid_{U_{i}\cap U_{j}}$
 satisfying the compatibility conditions $\sigma_{j,k}\sigma_{i,j} = \sigma_{i,k}$. 
 \end{itemize}

A \emph{global section} of $\sL$ is a family of elements $t_{i}\in L_{i}$ such that 
$\sigma_{i,j} t_{i} = t_{j}$. Such a section may be realized as the image of the constant function 1 under
a homomorphism of sheaves $\sO_{X} \to \sL$. If $X$ is projective, then 
by Theorem \cite[Thm III.5.2]{H} the space $H^{0}(\sL)$  of global sections is
a finite-dimensional vector space. If, moreover, $X$ is reduced and irreducible, then $H^{0}(\sO_{X}) = \CC$ because the only globally defined
functions on $X$ are the constant functions.

The coherent sheaf $\sL$ is said to be an \emph{invertible sheaf} on $X$ if there is an open cover as above with the additional property
that $L_{i} \cong \sO_X(U_{i})$, the free module on one generator. 

If $\sigma \in H^0(\sL)$ is a global section of an invertible sheaf
on $X$, and $p\in X$ is a point, then $\sigma(p)$ denotes the image of $\sigma$ under the natural map $H^0(\sL)$ to the stalk $\sL_{p}$, an $\sO_{X,p}$-module isomorphic to $\sO_{X,p}$. Since the isomorphism is not canonical, $\sigma$ does not define a function on $X$ at $p$; but since any two isomorphisms
differ by a unit in $\sO_{X,p}$, the vanishing locus, denoted $(\sigma)_0$ of $\sigma$ \emph{is} a well-defined subscheme of $X$. Moreover, if $X$ is integral, then the ratio of two global sections is a well-defined rational function, so the divisor class of 
$(\sigma)_0$ is independent of the choice of $\sigma$.

The terminology of ``vanishing'' is slightly confusing: when we  say that $\sigma$ vanishes at $p$ we mean
that if we choose a local isomorphism $\sL(U) \cong \sO_X(U)$, then $\sigma$ vanishes as a rational function using this
identification. This means that the section $\sigma(p)\in \sO_{X,p}$ is in the maximal ideal of the local ring $ \sO_{X,p}$, not that $\sigma(p)$ is zero
as an element of the local ring. Similarly, we say that $\sigma$ vanishes to order $m$ at $p$ if $\sigma(p)$ lies in the
$m$-th power of the maximal ideal. This idea is perhaps more natural under the identification of invertible sheaves and line bundles, described below.

\begin{proposition}
 The invertible sheaves on $X$ form a group under $\otimes_{X}$, called the 
\emph{Picard group of $X$}, denoted $\Pic(X)$. 
\end{proposition}
\begin{proof}
 If $\sF, \sG$ are invertible sheaves then so are $\sF\otimes_{\cO_X}\sG$ and  $\Hom_{\cO_X}(\sF, \sG)$, as one sees immediately by
restricting to the open sets where $\sF$ and $\sG$ are isomorphic to $\sO_{X}$. Moreover the natural isomorphisms
$$
\sF(U) \otimes_{X} \Hom(\sF(U), \sO_{X}(U)) \to \sO_{X}(U)\quad s \otimes f \mapsto f(s)
$$ 
patch together to define a global isomorphism 
$$
\sF \otimes_{\cO_X} \Hom(\sF, \sO_{X}) \to \sO_{X}
$$
justifying the definition
$\sF^{-1} := \Hom(\sF, \sO_{X})$ and thus the name ``invertible sheaf''. 
\end{proof}
 
If $D\subset X$ is an effective divisor, then we define $\sO_{X}(-D)$ to be the ideal sheaf of $D$. If $D$ is locally defined by the vanishing of a (locally defined) nonzerodivisor in $\sO_{X}$, (that is, $D$ is a Cartier divisor), then
$\sO_{X}(-D)$ is an invertible
sheaf.
We write $\sO_{X}(D)$ for the inverse, $\sO_{X}(-D)^{-1}$. The dual of the inclusion
$\sO_{X}(-D)\subset \sO_{X}$ is a map $\sO_{X} \to \sO_{X}(D)$ sending the global section $1\in \sO_{X}$ to a section
$\sigma\in \sO_{X}(D)$ that vanishes precisely on $D$.

\begin{example} [Invertible sheaves on $\PP^{r}$]\label{linear series on Pr} If $H\subset \PP^{r}$ is a hyperplane defined by the vanishing of a linear form $\ell = \ell(x_{0}, \dots x_{r})$ then the ideal sheaf $\sO_{\PP^{r}}(-1) := \sI_{H/\PP^{r}}\subset \sO_{\PP^{r}}$ is generated on the open affine set 
$U_{i}:= \{x_{i}\neq 0\} \cong \AA^{r}$
by $\ell/x_{i}$, and is thus an invertible sheaf with $\sO_{\PP^{r}}(-1)\mid_{U_i} \cong \sO_{\PP^{r}}(U_i)$.
Moreover, if $H'$ is the hyperplane defined by another linear form $\ell'$, then 
$$
\frac{\ell'}{\ell}\cdot\sI_{H/\PP^{r}} = \sI_{H'/\PP^{r}} 
$$
so the sheaves $\sI_{H/\PP^{r}}$ and $\sI_{H'/\PP^{r}} $ are isomorphic, justifying the name $\sO_{\PP^{r}}(-1)$.

The $d$-th tensor power of $\sO_{\PP^{r}}(-1)$ is called $\sO_{\PP^{r}}(-d)$; it is isomorphic to the
ideal sheaf of any hypersurface of degree $d$. Because polynomials satisfy the unique factorization property,
every effective divisor $D\subset \PP^{r}$ is a hypersurface of some degree $d$, so
$\sO_{\PP^{r}}(-D) \cong \sO_{\PP^{r}}(-d)$. Note that if $d>0$ then $H^{0}(\sO_{\PP^{r}}(-D)) = 0$, since it may be realized
as the sheaf of locally defined functions vanishing on $D$, and there are no such
globally defined functions except 0.

We take $\sO_{\PP^{r}}(d)$ to be the inverse of $\sO_{\PP^{r}}(-d)$. If $D$ is the hypersurface defined by 
a form $F$ of degree $d$, then $\sO_{\PP^{r}}(-D)$ is generated on $U_{i}$ by $F/x_{i}^{d}$, so
$\sO_{\PP^{r}}(D)$ is generated on $U_{i}$ by $x_{i}^{d}/F$.
Starting from the inclusion 
$
\sO_{\PP^{r}}(-D) \subset \sO_{\PP^{r}}
$
and taking inverses, we see that 
$
\sO_{\PP^{r}} \subset \sO_{\PP^{r}}(D)
$
and the global section $1\in H^0(\sO_{\PP^{r}})\subset H^0(\sO_{\PP^{r}}(D))$, restricted to
$U_{i}$, is $F/x_{i}^{d}$ times the local generator of $\sO_{\PP^{r}}(D)$ and thus vanishes on $D$.
 
To compute $H^0(\cO_{\PP^1} (d))$ directly, let $D = z_1 +z_2 +\cdots+z_d$ be a divisor of degree d and suppose that the coordinates are chosen so that none of the $z_i$ are at infinity. The sections of $\cO_{\PP^1} (D)$ are the rational functions with poles in $\AA^1$ only at 
the $z_i$. In affine coordinates, identifying the $z_i$ with complex numbers, these can each be written
$$
\frac{g(z)}{(z-z_1)(z-z_2)\cdots(z-z_d)}
$$
where $g$ is a polynomial. The condition that the point at infinity is not a pole is the condition $\deg(g) \leq d$. With this condition, these rational functions form a vector space of dimension $d+1$.

More generally, because every
rational function on $\PP^{r}$ has degree 0, and any two global sections differ by a rational
function, it follows that every global section of $\sO_{\PP^{r}}(d)$ vanishes on a divisor of degree $d$. Thus
we may identify $H^{0}(\sO_{\PP^{r}}(d))$ with the ${r+d\choose r}$-dimensional vector space of forms of degree $d$ on $\PP^{r}$.
\end{example}

\subsection{Invertible sheaves and line bundles}

If $\sL$ is an invertible sheaf on a variety $X$, and $p\in X$ is a point then the \emph{stalk} $\sL_p$ of $\sL$ is isomorphic to the local
ring $\sO_{X,p}$. We write $\gm_{X,p}$ for the maximal ideal of $\sO_{X,p}$. By definition the \emph{fiber} of $\sL$ is 
$\sL_p/\gm_{X,p}\cong \CC$. It is not hard to prove that the collection of these fibers forms a line bundle on $X$. 
%\fix{insert def of line bundle? maybe in a footnote, "if the reader knows..."} 
Moreover,
given a line bundle $L$ on $X$, we can recover an invertible sheaf $\sL$ associated to $L$ by defining
$\sL(U)$ to be the set of sections of $L$ defined over $U$. These two processes are inverse to one another, and allow
us to think of invertible sheaves and line bundles interchangeably.

Though we will systematically prefer the invertible sheaf terminology, there are at least two points in which the line bundle approach is more natural. First,  the vanishing of a section of an invertible sheaf at a point $p$ is genuinely the vanishing of the 
section of the line bundle as a function. Second, and more serious, given a morphism $f: Y\to X$ of schemes, the 
pullback $f^*(\sL)$ of an invertible sheaf $\sL$ on $X$ is defined as the tensor product of $\sO_Y$ with a sort of naive pullback; whereas the pullback of a line bundle is a straightforward set-theoretic operation.


\subsection{The morphism to projective space coming from a linear series} \label{morphism from series}
For any $\CC$-vector space $V$ of dimension $r+1$ with basis $x_{0}, \dots, x_{r}$, we write $\Sym(V) \cong \CC[x_{0},\dots, x_{r}]$ for the symmetric algebra on $V$, and
$\PP(V)\cong \PP^{r}_{\CC}$ to be the projective space ${\rm Proj}(\Sym(V))$, which is naturally isomorphic to the
space of lines in $V^{*}$ or, equivalently,  the space of 1-dimensional quotients of $V$. Note that the isomorphism $\PP(V)\cong \PP^{r}_{\CC}$ is well-defined up to the action
of $\Aut(\PP^r) = PGL(r+1)$.

Given a linear series $\sV:=(\sL, V)$  of dimension $r$ on a scheme $X$
we define the \emph{base locus} of $\sV$ to be the closed subscheme 
$$
B_\sV := \bigcap_{\sigma\in V}\{\sigma = 0\}.
$$
Note that we could have restricted the intersection to a basis of $V$, with the same result.
Let $W:=X\setminus B_\sV$ be the open subscheme where not all sections $\sigma_{i}$ vanish.

For any point $q\in W$ we  may choose an open neighborhood $W'\subset W$ of $q$, and an identification 
$$
t: \sL\mid_{W'} \rTo^{\cong} \sO_{W'}
$$
and define $\phi_{\sV}: W' \to \PP(V)$ by 
$$
W'\ni p \mapsto \bigl(t(\sigma_{0}(p)),\dots, t(\sigma_{r}(p))\bigr) \in \PP(V).
$$
This  is a morphism on $W'$. A change of neighborhoods $W'$ or of identifications $t$ would multiply
each value $t(\sigma_{i}(p))$ by a unit, the same one for each $i$, and thus the construction would define the same morphism. It follows that the morphisms
defined on different $W'$ agree on overlaps, and thus define a morphism $W \to \PP(V) \cong \PP^r$. This is the reason
that the dimension of $\sV$ is defined to be $r=\dim V -1$ instead of $\dim V$.

The most useful linear series are those that define morphisms defined on all of $X$. This happens when $B_\sV = \emptyset$,
that is, for every point $q\in X$, there is a section $\sigma \in V$ such that $\sigma$ does not vanish at $q$. In this case we say that $(\sL, \sV)$ is \emph{basepoint-free}.

\begin{example}\label{Veronese definition}
The morphism from $\PP^r$ defined by the  linear series $(\cO_{\PP^r}(d), H^0(\cO_{\PP^r}(d))$ has target
$\PP^{{r+d\choose r}-1}$, and takes a point $x_0,\dots x_r$ to the point whose coordinates are all the monomials of
degree $d$ in $x_0,\dots x_r$. It is called the \emph{$d$-th Veronese morphism} from $\PP^r$. For example on $\PP^1$, this has the form
$$
(x_0,x_1) \mapsto (x_0^d,\ x_0^{d-1}x_1,\ \dots,x_1^d).
$$
The image of $\PP^1$ under this morphism is called the \emph{rational normal curve} of degree $d$; in the case $d=2$ is the
\emph{plane conic}, and in the case $d=3$ it is called the \emph{twisted cubic}. Veronese himself studied the image of $\PP^2$
by the Veronese morphism of degree 2 now simply called \emph{the Veronese surface}.
\end{example}


\subsection{The linear series coming from a morphism to projective space}\label{series from morphism}

Conversely, suppose that we are given a morphism $\phi: X\to \PP^{r}$. With notation as in Example~\ref{linear series on Pr} we may choose an open affine cover $W_{i,j}$ of $X$ such that $\phi(W_{i,j})\subset U_{j}$. Composing the regular
functions
$x_{0}/x_{j},\dots, x_{r}/x_{j}$ with $\phi$ we get functions $\sigma_{0},\dots,\sigma_{r}$ on $W_{i,j}$.  The function $\sigma_{j}$ is the image under $\phi^*: \sO_{U_j} \to \sO_{W_{i,j}}$ of the function $x_j/x_j = 1$ on $U_{j}$, so $\sigma_j = 1\in \sO_{W_{i,j}}$. 

In particular, the module $\sL\mid_{W_{i,j}}$ generated by the rational functions 
$$
\{\sigma_p\mid_{W_{i,j}}\}= 
\{\phi^*((x_p/x_j)\mid_{U_j})\}
_{0\leq p\leq r}
$$
 is a free $\sO_{W_{i,j}}$-module on 1 generator. On the preimage of $U_j\cap U_k$ these sections differ by the common unit $\phi^*(x_k/x_j)$, and thus the collection of these modules defines an invertible sheaf $\sL$ on $X$ together with an
$r+1$-dimensional space of global sections $\sV := \langle \sigma_0,\dots \sigma_r\rangle$ that forms a basepoint-free linear series. Note that the subscheme  $\{\sigma_p = 0\} \subset W_{i,j}$  is the scheme-theoretic preimage of the
the hyperplane $\{x_p = 0\}\subset \PP^r$. This association of a linear series to a morphism is inverse to the construction
of Section~\ref{morphism from series},  completing the explanation and proof of Theorem~\ref{morphisms and linear series}\qed

\subsection{An upper bound on $h^0(\sL)$}

We will develop sophisticated ways of estimating the dimensions of linear series. We begin with an elementary bound:

\begin{theorem}\label{characterization of P1}
Let $C$ be a reduced, irreducible projective curve and let $\cL$ be an invertible sheaf of degree $d\geq 1$ on $C$. If $h^0(\cL) \geq d+1$ then
$$C \cong \PP^1,\ \cL \cong \cO_{\PP^1}(d), \hbox{ and  } h^0(\cL) = d+1.
$$
\end{theorem}

\begin{proof}
First, suppose that $d=1$. The linear series $(\cL, H^0(\cL))$ cannot have any base points, since
otherwise after subtracting one, we would get an invertible sheaf of degree $0$ with two independent global sections. This is impossible, since some linear combination of the sections would vanish at any given point, showing that the degree would be
$\geq 1$.

Thus we see that the linear series $(\cL, H^0(\cL))$ defines a morphism $\phi: C\to \PP^1$ of degree 1 whose fibers are the divisors defined by
the vanishing of sections of $\cL$, and which are thus of degree 1. Thus if $p\in C$ is the preimage of $q\in \PP^1$, the induced map of local rings
$\phi^*:\cO_{\PP^1, q} \to \cO_{C, p}$ is a finite, birational map. Since $\cO_{\PP^1, q}$ is integrally closed, this is an isomorphism. Thus 
$\phi$ is an isomorphism, and the statements $\cL \cong \cO_{\PP^1}(d), \hbox{ and  } \h^0(\cL) = d+1$ follow from the classification of invertible sheaves on $\PP^1$. 

If $d>1$, let $p_1,\dots p_{d-1}$ be general points of $C$, and set $\cL':=\cL(-p_1-\cdots-p_{d-1})$, a sheaf of degree 1.
 Since $\cL$ is locally isomorphic to the sheaf of functions on $C$, the condition of vanishing at a point imposes at most 1 linear condition on 
the global sections of $\cL$, and thus $H^0(\cL') \geq 2$. From the case $d=1$ we see that $C\cong \PP^1$, and the statements
about $\cL$ follow as before.
 \end{proof}

From the correspondence between invertible sheaves and maps to projective space, we now get:
\begin{corollary}\label{minimal degree curves}
If $C\subset \PP^d$ is a reduced, irreducible, non-degenerate curve, then the degree of $C$ is $\geq d$ with equality only in the case
of the rational normal curve.\qed
\end{corollary}

Recall from Example~\ref{Veronese definition} that the image of the $d$-th \emph{Veronese map}  
$$
\phi_d: \PP^1 \to \PP(H^0(\cO_{\PP^1}(d)) \cong \PP^d; \quad (s,t) \mapsto (s^d, s^{d-1}t, \dots, t^d)
$$
corresponding to by the complete linear series $|\cO_{\PP^1}(d)|$ is called the \emph{rational normal curve} of degree $d$. Rational normal curves arise often in the literature because they have many extremal properties, such as those of Corollaries~\ref{minimal degree curves} and \ref{points on a RNC}.

For a related result see Corollary~\ref{uninflected}.

\begin{corollary}\label{independence of points on a RNC}
Let $C\subset \PP^d$ be the rational normal curve of degree $d$. If $E$ is an effective divisor on $C$ of degree $e\leq d+1$, then the
span of $E$ (that is the dimension of the smallest linear space containing the subscheme $E$) has dimension $e-1$.
\end{corollary}
Less formally: any finite set or subscheme of $C$ is as linearly independent as possible

\begin{proof}
Let $L$ be the span of $E$. Since $E$ can impose at most $e$ conditions on hyperplanes, it follows that the dimension of the span of $E$ is
at most $e-1$.

On the other hand, the hyperplanes containing $E$ meet $C$ in a divisor of the form $E+E'$, where
$\deg E' = d-e$. Thus the projection of $C$ from $E$ is a non-degenerate curve of degree $d-e$ in $\PP^{d-(\dim {\rm span}(E))-1}$
so from Corollary~\ref{minimal degree curves} we get $d-e \geq d-(\dim {\rm span}(E)-1)$, as required.
\end{proof}

In the case of distinct points on a rational normal curve
it is easy to make a direct argument why they are as independent as possible: In affine coordinates chosen so that none of the points are
at infinity we can identify the points $\lambda_1,\dots,\lambda_{d+1} \in C \cong \PP^1$ with distinct complex numbers, and the independence (for $\ell = d+1$) is equivalent to the the nonvanishing of the Vandermonde determinant
$$
\begin{vmatrix}
1 & \lambda_1 & \lambda_1^2 & \dots & \lambda_1^d \\
1 & \lambda_2 & \lambda_2^2 & \dots & \lambda_2^d \\
\vdots & & & & \vdots \\
1 & \lambda_{d+1} & \lambda_{d+1}^2 & \dots & \lambda_{d+1}^d \\
\end{vmatrix}
= \prod_{1 \leq i < j \leq d+1} (\lambda_j - \lambda_i)
$$


\subsection{Incomplete linear series}


Let $\sV = (\sL, V)$ be a linear series on $X$.  The linear series is said to be \emph{complete} if $V = H^0(\cL)$; in this case it is sometimes denoted $|\cL|$. If $\cL \cong \cO_C(D)$, we also write it as $|D|$. 
 If $D$ is any divisor on $C$ we write $r(D)$ for the dimension of the complete linear series $|D|$; that is, $r(D) = h^0(\cO_C(D)) - 1$. Finally, in classical algebraic geometry a linear series of dimension 1 is called a \emph{pencil}, a linear series of dimension 2 is called a \emph{net} and, less commonly, a three-dimensional linear series is called a \emph{web}.  Only the first of these will be
 used in this book.

We can relate the geometry of the morphism associated to an incomplete linear series $V \subset H^0(\cL)$ to the geometry of the morphism associated to the complete linear series $|\cL|$. In general, if $V \subset W \subset H^0(\cL)$ are a pair of nested linear series, then a 1-dimensional quotient of $W$ restricts to a 1-dimensional quotient of $V$ unless it vanishes on $V$.
Thus we have a partially defined linear morphism $\pi: \PP(W)  \to \PP(V)$. The \emph{indeterminacy locus} of the map
consists of the set of 1-quotients vanishing on $V$, that is, to $\PP(W/V) \subset \PP(W)$; we will call it the 
\emph{center of the projection $\pi$.} (It is sometimes useful to
think of the dual picture: lines in $W^*$ map to lines in $V^*$ except when they lie in the subspace $(W/V)^* = Ann(V)\subset W^*$.)
The relation between the maps $\phi_V$ and $\phi_W$ is a factorization:
$$
\phi_V = \pi \circ \phi_W;
$$
that is, we have the diagram 

\begin{diagram}
& & \PP W^* \\
& \ruTo^{\phi_W} & \dDashto_\pi \\
C & \rTo^{\phi_V} & \PP V^*.
\end{diagram}
Thus if $W$ is basepoint-free, then $V$ is basepoint-free if and only if the center of the projection $\pi$ is disjoint from $\phi_W(C)$, and in this case we say that $\pi$ is \emph{regular} on $C$.

By way of language, we will say that a curve $C \subset \PP^r$ embedded by a complete linear series $|\sL|$ is \emph{linearly normal}; this is equivalent to saying that the pullback map
$$
H^0(\cO_{\PP^r}(1)) \to H^0(\sL)
$$
is surjective. Since regular projections of a curve correspond to subseries, this is equivalent to saying that $C$ is \emph{not} the regular  projection of a nondegenerate curve $\tilde C \subset \PP^{r+1}$. We write $\sO_C(1)$ for the restriction of $\sO_{\PP^r}(1)$ to $C$.

\subsection{Sums of linear series}
If
$\cD = (\cL,V)$ and $\cE = (\cM, W)$ be two linear series on a curve $C$. By the \emph{sum} $\cD + \cE$ of $\cD$ and $\cE$, we will mean the pair 
$$
\cD + \cE = (\cL \otimes \cM, U) 
$$
where $U \subset H^0(\cL \otimes \cM)$ is the subspace generated by the image of $V \otimes W$, under the multiplication/cup product map $H^0(\cL) \otimes H^0(\cM) \to H^0(\cL \otimes \cM)$---in other words, it's the subspace of the complete linear series $|\cL\otimes \cM|$ spanned by divisors of the form $D+E$, with effective divisors $D \in \cD$ and $E \in \cE$.
 
 
\begin{proposition}\label{sum of linear series}
 If $\cD$ and $\cE$ are two  linear series that contain effective divisors on a curve $C$, then
$$
\dim(\cD + \cE) \geq \dim \cD + \dim \cE.
$$
\end{proposition}
\begin{proof}
Saying $\dim \cD \geq m$ is equivalent to saying that we can find a divisor $D \in \cD$ containing any given $m$ points of $C$; since $\cD + \cE$ contains all pairwise sums $D + E$ with $D \in \cD$ and $E \in \cE$, we can certainly find a divisor $F \in \cD + \cE$ containing any given $\dim \cD + \dim \cE$ points of $C$.
\end{proof}

\section{Which linear series define embeddings?}

A linear series $\sV = (\sL, V)$ is called  \emph{very ample}  if it is basepoint-free and defines an embedding. If $D$ is a Cartier divisor on $X$, then we say that $D$ is \emph{very ample} if the complete linear series $|D|$ is very ample, and we say that $D$ is \emph{ample} if $mD$ is very ample for some integer $m>0$.

Given a linear series $\sV = (\sL, V)$ and an effective divisor $D$ on $C$, we  set
$
\sV(-D) = (\sL(-D),V(-D))
$
where
$$
\sL(-D): = \sL \otimes \sO(-D)\hbox{ and } V(-D) := \{ \sigma \in V \mid \sigma(D) = 0 \}.
$$
The difference $\dim \sV - \dim \sV(-D)$ is called the \emph{number of conditions imposed by $D$ on the linear series $\sV$}; we say that $D$ \emph{imposes independent conditions} on $\sV$ if $\dim \sV - \dim \s V(-D) = \deg(D)$.

Via the correspondence of Theorem~\ref{morphisms and linear series}, the statements about the geometry of a morphism $\phi : C \to \PP^r$ can be formulated as statements about the relevant linear series. In the case of complete series, these are statements about the vector space $H^{0}(\sL)$ of global sections of $\sL$. As is customary, we write $h^{0}(\sL)$ for the dimension of this vector space. It is useful to have criteria
in these terms for when a linear series defines an embedding, or even to be basepoint-free so that it
defines a morphism:

\begin{proposition}\label{very ample}\cite[Thm. IV.3.1]{H}
Let $\cL$ be an invertible sheaf on a smooth curve $C$. The complete linear series $|\cL|$ is basepoint-free iff
$$
h^0(\cL(-p)) = h^0(\cL) - 1 \quad \forall p \in C;
$$
and $\sL$ is very ample, iff
$$
h^0(\cL(-p-q)) = h^0(\cL) - 2 \quad \forall p, q \in C.
$$
\end{proposition} 

\begin{proof}
First, since vanishing at a point imposes one linear condition on sections of $\sL$ we have $h^0(\sL(-D)) \geq h^0(\sL)-\deg D$ for any
effective divisor $D$.

To say that $|\cL|$ is basepoint-free means that for every point $p\in C$ there is a section of $\sL$ that does not vanish at $p$; thus vanishing
at $p$ is a nontrivial linear condition on $H^0(\sL)$. Conversely, if $h^0(\cL(-p)) = h^0(\cL) - 1$ then $p$ imposes a nontrivial condition, so
some section of $\sL$ does not vanish at $p$.

Since a divisor of degree $d$ cannot impose more than $d$ condtions on a linear series, the statement $h^0(\cL(-p-q)) = h^0(\cL) - 2$ for all $p, q$ implies the condition for basepoint freeness; and saying that $\phi_\cL(p) \neq \phi_\cL(q)$ implies that the linear series defines a set-theoretic injection. The tangent space of $C$ at $p$ is $(\sI_C(p)/\sI_C(p)^2)^*$, so the condition that there is a section of $\sL$ that vanishes at $p$, but does not vanish
to order 2, implies that the differential $d\phi_\cL$ is injective at $p$ as well.

Let $\phi:C \to \PP^r$ be the map defined by $\sL$. To say that $\phi$  is an embedding locally at a point $p$, we need to know that the map of local rings
$$
\phi^*: \sO_{\PP^r,\phi(p)} \to \sO_{C,p} 
$$
is surjective. Since $C$ is proper, the map $C\to \phi(C)$ is finite,
so $\phi_*(\sO_{C})$ is coherent.
 The constants  $\CC =\sO_{\PP^r,\phi(p)}/\gm_{\PP^r,\phi(p)}$ pull back to the constants in
$\sO_{C,p}/\gm_{C,p}$. 
By Nakayama's lemma, the surjectivity will follow if 
$$
\frac{\sO_{C,p}}
{\phi^*(\gm_{\PP^r,\phi(p)})  \sO_{C,p}}
$$
is 1-dimensional; that is, if  the pullback of $\gm_{\PP^r,\phi(p)}$ is not contained in the square of the
maximal ideal $\gm_{C,p}$. The inequality $h^0(\sL(-2p) \neq h^0(\sL(-p))$ shows that there is a 
section in $H^0(\sL(-p))$ that does not vanish to order 2, completing the proof.
\end{proof}


For another example of the relationship between linear series on curves and morphisms of curves to projective space, consider a smooth curve $C \subset \PP^r$ embedded in projective space, and assume that $C$ is linearly normal. If $\phi : C \to C$ is any automorphism, we can ask whether $\phi$ is induced by an automorphism of $\PP^r$; in other words, does there exist an automorphism $\Phi : \PP^r \to \PP^r$ such that $\Phi(C) = C$ and $\Phi|_C = \phi$? The answer is expressed in Exercise~\ref{projective automorphism}.


If $\phi:X \to \PP^r$ is a generically finite morphism, then the \emph{degree of $\phi$} is the number of points in the preimage of a general point of $\phi(X)$. It follows that if $D := \sum_{p\in C} n_pp$ is a divisor on a smooth curve, and the linear series $|D|$ is basepoint-free, then the degree of the morphism associated to $|D|$ is $\deg D := \sum_{p\in C} n_p$.

\section{Exercises}

\begin{exercise}\label{here there be basepoints}
 Show that there is no non-constant morphism $\PP^r\to \PP^s$ when $s<r$ by showing that any nontrivial linear
 series of dimension $<r$ has a non-empty base locus.
\end{exercise}

\begin{exercise}
Extend the statement of Proposition~\ref{very ample} to incomplete linear series; that is, prove that the morphism associated to a linear series $(\cL, V)$
on a smooth curve is an embedding iff
$$
\dim\big( V \cap H^0(\cL(-p-q))\big) = \dim V - 2 \quad \forall p, q \in C.
$$
\end{exercise}

\begin{exercise}
An automorphism of $\PP^r$ takes hyperplanes to hyperplanes. Deduce that it is given by the linear series
$\sV = (\sO_{\PP^r}(1), H^0(\sO_{\PP^r}(1)))$, and use this to show that $\Aut \PP^r = PGL(r+1)$. 
\end{exercise}

\begin{exercise}\label{projective automorphism}
In the circumstances above, the automorphism $\phi$ is induced by an automorphism of $\PP^r$ if and only if $\phi$ carries the invertible sheaf $\cO_{C}(1)$ to itself; that is, $\phi^*(\cO_{C}(1)) \cong \cO_{C}(1)$. In this case we say that the automorphism
is projective. Show that every automorphism of a rational normal curve $C \subset \PP^d$  extends to $\PP^d$. Since the
automorphism group $PSL_3$ acts transitively on $\PP^1$, we say that
$C$ is \emph{projectively homogeneous}.


\end{exercise}

\begin{exercise}\label{normality of RNC}
 Show that $\CC[s^d,s^{d-1}t,\dots, t^d]$ is normal (ie, integrally closed) by noting that its integral closure must be
 contained in $\CC[s,t]$ and then showing that if $f$ is any polynomial
 in the integral closure then the homogeneous components of $f$ are also in the integral closure.
\end{exercise}


\input footer.tex
