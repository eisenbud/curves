%header and footer for separate chapter files

\ifx\whole\undefined
\documentclass[12pt, leqno]{book}
\usepackage{graphicx}
\usepackage{eps-to-pdf}
\input style-for-curves.sty
%\input sl-macros.sty
\usepackage{hyperref}
\usepackage{showkeys} %This shows the labels.
\usepackage{msribib}
\usepackage{pdfpages}
\usepackage{draftwatermark}
\SetWatermarkText{DRAFT:\ \today}
\SetWatermarkScale{2}
\SetWatermarkColor[gray]{0.9}

%\usepackage{SLAG,msribib,local}
%\usepackage{amsmath,amscd,amsthm,amssymb,amsxtra,latexsym,epsfig,epic,graphics}
%\usepackage[matrix,arrow,curve]{xy}
%\usepackage{graphicx}
%\usepackage{diagrams}
%
%%\usepackage{amsrefs}
%%%%%%%%%%%%%%%%%%%%%%%%%%%%%%%%%%%%%%%%%%
%%\textwidth16cm
%%\textheight20cm
%%\topmargin-2cm
%\oddsidemargin.8cm
%\evensidemargin1cm
%
%%%%%%Definitions
%\input preamble.tex
%\input style-for-curves.sty
%\def\TU{{\bf U}}
%\def\AA{{\mathbb A}}
%\def\BB{{\mathbb B}}
%\def\CC{{\mathbb C}}
%\def\QQ{{\mathbb Q}}
%\def\RR{{\mathbb R}}
%\def\facet{{\bf facet}}
%\def\image{{\rm image}}
%\def\cE{{\cal E}}
%\def\cF{{\cal F}}
%\def\cG{{\cal G}}
%\def\cH{{\cal H}}
%\def\cHom{{{\cal H}om}}
%\def\h{{\rm h}}
% \def\bs{{Boij-S\"oderberg{} }}
%
%\makeatletter
%\def\Ddots{\mathinner{\mkern1mu\raise\p@
%\vbox{\kern7\p@\hbox{.}}\mkern2mu
%\raise4\p@\hbox{.}\mkern2mu\raise7\p@\hbox{.}\mkern1mu}}
%\makeatother

%%
%\pagestyle{myheadings}

%\input style-for-curves.tex
%\documentclass{cambridge7A}
%\usepackage{hatcher_revised} 
%\usepackage{3264}
   
\errorcontextlines=1000
%\usepackage{makeidx}
\let\see\relax
\usepackage{makeidx}
\makeindex
% \index{word} in the doc; \index{variety!algebraic} gives variety, algebraic
% PUT a % after each \index{***}

\overfullrule=5pt
\catcode`\@\active
\def@{\mskip1.5mu} %produce a small space in math with an @

\title{A Chapter from ``The Practice of Algebraic Curves"}
\author{\copyright David Eisenbud and Joe Harris}
%%\includeonly{%
%0-intro,01-ChowRingDogma,02-FirstExamples,03-Grassmannians,04-GeneralGrassmannians
%,05-VectorBundlesAndChernClasses,06-LinesOnHypersurfaces,07-SingularElementsOfLinearSeries,
%08-ParameterSpaces,
%bib
%}

\date{\today}
%%\date{}
%\title{Curves}
%%{\normalsize ***Preliminary Version***}} 
%\author{David Eisenbud and Joe Harris }
%
%\begin{document}

\begin{document}
\maketitle

\pagenumbering{roman}
\setcounter{page}{5}
%\begin{5}
%\end{5}
\pagenumbering{arabic}
\tableofcontents
\fi


\chapter{Linear Systems}
This Chapter is intended largely as a review of material that should be familiar, couched in the language we will use throughout the book. Experienced readers 
may want to skip it, and return if and when that is needed. Nearly all of the material can be found in texts such as \cite{H}, the main exception being the treatment of
Noether's Theorem on canonical curves in Section~\ref{Noether theorem section}.


\section{Projective varieties, Morphisms to projective space, and families of Cartier divisors}

Though affine varieties are defined by the functions on them---the coordinate ring, which is defined by the variety---projective varieties have very few functions, and
the homogeneous coordinate ring of a variety $X\subset \PP^n$ is not characteristic of $X$, but of $X$ together with some auxiliary data, an 
either a family of divisors of a simple kind or an invertible sheaf and a collection of global sections. Such data, more generally, can be used to define a morphism or a birational map
to a projective space, and we begin by describing these relations.

Let $\phi: C\to \PP^{r}$ be a morphism from a smooth curve $C$. If $H\subset \PP^r$ is a hyperplane that does not contain $\phi(C)$, then the preimage of $\phi(C)\cap H$ is a finite sets of points on $C$, with multiplicities when $H$ is tangent to $\phi(C)$ or passes through a singular point of $\phi(C)$. Such a set of points with non-negative integer multiplicities is called an \emph{effective divisor} on $C$; more generally, a \emph{divisor} (sometimes called a \emph{Weil divisor})on a scheme $X$ is an integral linear combination of codimension 1 subvarieties, and it is called \emph{effective} if the coefficients are all non-negative. he \emph{degree} of a divisor on a smooth curve
is the sum of the coefficients; more generally it is the sum of the constituent codimension 1 subvarieties, weighted with their coefficients. 

The divisors that arise as the pullbacks of general hyperplanes are special: since a hyperplane is defined by just one equation, which is locally given by the vanishing of a function, the pullback of a hyperplane will be locally defined by the vanishing of a single function,
and for a general hyperplane, this will be a nonzerodivisor; that is, it is an  \emph{effective Cartier divisor}. See \cite[pp. 140-146]{H} for more information; on a smooth curve every divisor is Cartier, so the difference between Weil and Cartier divisors will not be an issue for us.) T
The  word ``local'' scattered through the previous paragraph is needed because, if $X$ is a projective variety, then the only algebraic functions $X\to \CC$ are constant functions. (Proof: the image of a projective variety
is again projective, and the only projective subvarieties of an affine variety are points.)

If we are given the family of divisors on $C$ that are the preimages of the intersections of hyperplanes with  $\phi(C)$, we can recover the morphism $\phi$ set-theoretically: it takes a point $p\in C$ to the point of projective space that is the intersection of those
hyperplanes whose preimages contain $p$. 

The relationship of two divisors on $C$ that are preimages of intersections of $\phi(C)$ with hyperplanes is simple to describe: If hyperplanes
$H, H'\subset \PP^r$ are defined by the linear forms $h, h'$  then $1/h'$ has a simple pole along $H'$---we may say that it ``vanishes along $H$'' to degree $-1$.
In this sense the divisor $H-H'$ on $\PP^n$ is defined by the rational function $\lambda= h/h'$. If neither $H$ nor $H'$ contain $C$ then the pullback of $\lambda$ is a well-defined, nonzero rational function on $C$, and the divisor 
$\phi^{-1}(\phi(C)\cap H') - \phi^{-1}(\phi(C)\cap H)$ is defined by the pullback  $\phi^*(\lambda) := \lambda \circ f$. Thus the divisors arising from a given morphism to $\PP^{r}$ differ by the divisors of zeros minus poles of rational functions on $C$. 


If $C$ is a smooth curve then the local ring $\sO_{C,p}$ of $C$ at a point $p$ is a discrete valuation ring, and if $\pi$ is a generator of the maximal ideal of $\sO_{C,p}$, then any rational
function $\lambda$ on $C$ can be expressed uniquely as $u\pi^k$ where $u\in \sO_{C,p}$ is a unit and $k\in \ZZ$. We say that
the \emph{order} of $\lambda$ at $p$, and write $k = \ord_p \lambda$. We associate $\lambda$ to the divisor
$$
(\lambda) := \sum_{p\in C} (\ord_p\lambda)p.
$$
The \emph{class group} of $C$ is defined to be the the group of divisors on $C$ modulo the divisors of rational functions.
Thus the divisors on $C$ that are preimages of intersections of $\phi(C)$ with different hyperplanes all belong to the same
\emph{divisor class}, and form a linear system in the sense of the following section.

It is important to note that the degree of the divisor associated to a rational function on a smooth curve is always 0: this is evident on $\PP^1$, where rational 
functions are the ratios of two forms of the same degree; and a rational function $\phi$ on a smooth curve may be regarded as the pullback of a rational function
on $\PP^1$ under the map given by $\phi$; since any non-constant map of smooth curves is flat, the pullback multiplies the degree of a divisor by the degree 
of the covering. 

\section{Morphisms and linear systems}
We want to understand morphisms to $\PP^r$ more than set-theoretically, and we want to be able to produce them from data on $C$. For this we use the notion of linear system (sometimes called linear series). Our goal in the next sections is to explain this connection:

\begin{definition}
 A \emph{linear system} on a scheme $X$ is a pair $\sV  = (\sL, V)$ where $\sL$ is an invertible sheaf (defined in Section ~\ref{Invertible sheaves}) on $X$ and
 $V$ is a vector space of global sections of $\sL$. We define the \emph{dimension} of the linear series to be $\dim \sV$ is defined to be
 $$
 \dim \sV := \dim V -1,
 $$
 While the \emph{degree} of $\sV$ is by definition the degree of the divisor of zeros and poles of any of its sections; since the 
 ratio of two sections is a rational function, this is independent of the section chosen. A $g^r_d$ is by definition a linear system
 of dimension $r$ and degree $d$.
\end{definition}



\begin{theorem}\label{morphisms and linear systems}
There is a natural bijection between the set of nondegenerate morphisms $\phi : C \to \PP^r$ modulo $PGL_{r+1}$, and basepoint-free linear systems of dimension $r$ on $C$.\end{theorem}

Here ``nondegenerate" means the image of the morphism $\phi$ is not contained in any hyperplane; the dimension of the series
 $\sV  = (\sL, V)$ is $\dim_\CC V -1$; and basepoint-free means that there is no point of $C$ where all the sections in $V$
vanish.

\subsection{Invertible sheaves}\label{Invertible sheaves}
Recall first that a \emph{coherent sheaf} $\sL$ on a scheme $X$ may be defined by
giving 
\begin{itemize}
 \item An open affine cover $\{U_{i}\}$ of $X$; 
 \item For each $i$, a finitely generated $\sO_{X}(U_{i})$-module $L_{i}$;
 \item For each $i,j$, an isomorphism $\sigma_{i,j}: L_{i}\mid_{U_{i}\cap U_{j}} \to L_{j}\mid_{U_{i}\cap U_{j}}$
 satisfying the compatibility conditions $\sigma_{j,k}\sigma_{i,j} = \sigma_{i,k}$. 
 \end{itemize}

A \emph{global section} of $\sL$ is a family of elements $t_{i}\in L_{i}$ such that 
$\sigma_{i,j} t_{i} = t_{j}$. Such a section may be realized as the image of the constant function 1 under
a homomorphism of sheaves $\sO_{X} \to \sL$. If $X$ is projective, then 
by Theorem \cite[Thm III.5.2]{H} the space $H^{0}(\sL)$  of global sections is
a finite-dimensional vector space. If, moreover, $X$ is reduced and irreducible, then $H^{0}(\sO_{X}) = \CC$ because the only globally defined
functions on $X$ are the constant functions.

The coherent sheaf $\sL$ is said to be an \emph{invertible sheaf} on $X$ if there is an open cover as above with the additional property
that $F_{i} \cong \sO_X(U_{i})$, the free module on one generator. 

If $\sigma \in H^0\sL$ is a global section of an invertible sheaf
on $X$, and $p\in X$ is a point, then $\sigma(p)$ is in the stalk of $\sL$ at $p$, a module isomorphic to $\sO_{X,p}$. Since the isomorphism is not canonical, $\sigma$ does not define a function on $X$ at $p$; but since any two isomorphisms
differ by a unit in $\sO_{X,p}$, the vanishing locus, denoted $(\sigma)_0$ of $\sigma$ \emph{is} a well-defined subscheme of $X$. Moreover, if $X$ is integral, then the ratio of two global sections is a well-defined rational function, so the divisor class of 
$(\sigma)_0$ is independent of the choice of $\sigma$.

\begin{proposition}
 The invertible sheaves on $X$ form a group under $\otimes_{X}$, called the 
\emph{Picard group of $X$}, denoted $\Pic(X)$. 
\end{proposition}
\begin{proof}
 If $\sF, \sG$ are invertible sheaves then so are $\sF\otimes_{X}\sG$ and  $\Hom_{X}(\sF, \sG)$, as one sees immediately by
restricting to the open sets where $\sF$ and $\sG$ are isomorphic to $\sO_{X}$. Moreover the natural isomorphisms
$$
\sF(U) \otimes_{X} \Hom(\sF(U), \sO_{X}(U)) \to \sO_{X}(U)\quad s \otimes f \mapsto f(s)
$$ 
patch together to define a global isomorphism 
$$
\sF \otimes_{X} \Hom(\sF, \sO_{X}) \to \sO_{X}
$$
justifying the definition
$\sF^{-1} := \Hom(\sF, \sO_{X})$ and thus the name ``invertible sheaf''. 
\end{proof}
 
If $D\subset X$ is an effective divisor, then we define $\sO_{X}(-D)$ to be the ideal sheaf of $D$. If $D$ is locally defined by the vanishing of a (locally defined) nonzerodivisor in $\sO_{X}$, (that is, $D$ is a Cartier divisor), then
$\sO_{X}(-D)$ is an invertible
sheaf.
We write $\sO_{X}(D)$ for the inverse, $\sO_{X}(-D)^{-1}$. The dual of the inclusion
$\sO_{X}(-D)\subset \sO_{X}$ is a map $\sO_{X} \to \sO_{X}(D)$ sending the global section $1\in \sO_{X}$ to a section
$\sigma\in \sO_{X}(D)$ that vanishes precisely on $D$.

\begin{example} [Invertible sheaves on $\PP^{r}$]\label{linear systems on Pr} If $H\subset \PP^{r}$ is a hyperplane defined by the vanishing of a linear form $\ell = \ell(x_{0}, \dots x_{r})$ then the ideal sheaf $\sO_{\PP^{r}}(-1) := \sI_{H/\PP^{r}}\subset \sO_{\PP^{r}}$ is generated on the open affine set 
$U_{i}:= \{x_{i}\neq 0\} \cong \AA^{r}$
by $\ell/x_{i}$, and is thus an invertible sheaf. 
Moreover, if $H'$ is the hyperplane defined by another linear form $\ell'$, then 
$$
\frac{\ell'}{\ell}\cdot\sI_{H/\PP^{r}} = \sI_{H'/\PP^{r}} 
$$
\fix{check that this is out notation for ideal sheaf}
so the sheaves $\sI_{H/\PP^{r}}$ and $\sI_{H'/\PP^{r}} $ are isomorphic, justifying the name $\sO_{\PP^{r}}(-1)$.

The $p$-th tensor power of $\sO_{\PP^{r}}(-1)$ is called $\sO_{\PP^{r}}(-d)$; it is isomorphic to the
ideal sheaf of any hypersurface of degree $d$. Because polynomials satisfy the unique factorization property,
every effective divisor $D\subset \PP^{r}$ is a hypersurface of some degree $d$, so
$\sO_{\PP^{r}}(-D) \cong \sO_{\PP^{r}}(-d)$. Note that if $d>0$ then $H^{0}(\sO_{\PP^{r}}(-D)) = 0$, since it may be realized
as the sheaf of locally defined functions vanishing on $D$, and there are no such
globally defined functions except 0.

We take $\sO_{\PP^{r}}(d)$ to be the inverse of $\sO_{\PP^{r}}(-d)$. If $D$ is the hypersurface defined by 
a form $F$ of degree $d$, then $\sO_{\PP^{r}}(-D)$ is generated on $U_{i}$ by $F/(x_{i}^{d})$, so
$\sO_{\PP^{r}}(D)$ is generated on $U_{i}$ by $x_{i}^{d}/F$.
Starting from the inclusion 
$
\sO_{\PP^{r}}(-D) \subset \sO_{\PP^{r}}
$
and taking inverses, we see that 
$
\sO_{\PP^{r}} \subset \sO_{\PP^{r}}(D)
$
and the global section $1\in H^0(\sO_{\PP^{r}})\subset H^0(\sO_{\PP^{r}}(D))$, restricted to
$U_{i}$, is $F/(x_{0}^{d})$ times the local generator of $\sO_{\PP^{r}}(D)$ and thus vanishes on $D$.
 Because every
rational function on $\PP^{r}$ has degree 0, and any two global sections differ by a rational
function, it follows that every global section of $\sO_{\PP^{r}}(d)$ vanishes on a divisor of degree $d$. Thus
we may identify $H^{0}(\sO_{\PP^{r}}(d))$ with the ${n+d\choose n}$-dimensional vector space of forms of degree $d$ on $\PP^{r}$.
\end{example}

The proof of Theorem~\ref{morphisms and linear systems} is contained in the material of the next two subsections:

\subsection{The morphism to projective space coming from a linear system} 
For any $\CC$-vector space $V$ of dimension $r+1$ with basis $x_{0}, \dots, x_{r}$, we write $\Sym(V) \cong \CC[x_{0},\dots, x_{r}]$ for the symmetric algebra on $V$, and
$\PP(V)\cong \PP^{r}_{\CC}$ to be the projective space ${\rm Proj}(\Sym(V))$, which is naturally isomorphic to the
space of lines in $V^{*}$. Note that the isomorphism $\PP(V)\cong \PP^{r}_{\CC}$ is well-defined up to the action
of $\Aut(\PP^r) = PGL(r+1)$.


Given a linear system $\sV:=(\sL, V)$  of dimension $r$ on a scheme $X$, 
we define the \emph{base locus} of $\sV$ to be the closed subscheme 
$$
B_\sV := \bigcap_{i= 0}^{r}\{\sigma_{i} = 0\}.
$$
Let $W:=X\setminus B_\sV$ be the open subscheme where not all sections $\sigma_{i}$ vanish.

For any point $q\in W$ we  may choose an open neighborhood $W'\subset W$ of $q$, and an identification 
$$
t: \sL\mid_{W'} \rTo^{\cong} \sO_{W'}
$$
and define $\phi_{\sV}: W' \to \PP(V)$ by 
$$
W'\ni p \mapsto \bigl(t(\sigma_{0}(p)),\dots, t(\sigma_{r}(p))\bigr) \in \PP(V).
$$
This  is a morphism on $W'$. A change of neighborhoods $W'$ or of identifications $t$ would multiply
each value $t(\sigma_{i}(p))$ by a unit, the same one for each $i$, and thus the construction would define the same morphism. It follows that the morphisms
defined on different $W'$ agree on overlaps, and thus define a morphism $W \to \PP(V) \cong \PP^r$. This is the reason
that the dimension of $\sV$ is defined to be $r=\dim V -1$ instead of $\dim V$.

The most useful linear series are those that define morphisms defined on all of $X$. This happens when $B_\sV = \emptyset$,
that is, for every point $q\in X$, there is a section $\sigma \in V$ such that $\sigma$ does not vanish at $x$. In this case we say that $(\sL, \sV)$ is \emph{basepoint free}.

\begin{example}\label{Veronese definition}
The morphism from $\PP^r$ defined by the complete linear system $|\cO_{\PP^r}(d)|$ has target
$\PP^{{r+d\choose r}-1}$, and takes a point $x_0,\dots x_r$ to the point whose coordinates are all the monomials of
degree $d$ in $x_0,\dots x_r$. It is called the \emph{$d$-th Veronese morphism} from $\PP^r$. For example on $\PP^1$, this has the form
$$
(x_0,x_1) \mapsto (x_0^d,\ x_0^{d-1}x_1,\ \dots,x_1^d).
$$
The image of $\PP^1$ under this morphism is called the \emph{rational normal curve} of degree $d$; in the case $d=2$ is the
\emph{plane conic}, and in the case $d=3$ it is called the \emph{twisted cubic}. Veronese himself studied the image of $\PP^2$
by the Veronese morphism of degree 2 now simply called \emph{the Veronese surface}.
\end{example}

\begin{exercise}\label{here there be basepoints}
 Show that there is no non-constant morphism $\PP^r\to \PP^s$ when $s<r$ by showing that any nontrivial linear
 system of dimension $<r$ has a non-empty base locus.
\end{exercise}

\subsection{The linear system coming from a morphism to projective space}

Conversely, suppose that we are given a morphism $\phi: X\to \PP^{r}$. With notation as in Example~\ref{linear systems on Pr} we may choose an open affine cover $W_{i,j}$ of $X$ such that $\phi(W_{i,j})\subset U_{j}$. Composing the regular
functions
$x_{0}/x_{j},\dots, x_{r}/x_{j}$ with $\phi$ we get functions $\sigma_{0},\dots,\sigma_{r}$ on $W_{i,j}$.  The function $\sigma_{j}$, is the image under $\phi^*: \sO_{U_j} \to \sO_{W_{i,j}}$ of the function $x_j/x_j = 1$ on $U_{j}$, so it $\sigma_j = 1\in \sO_{W_{i,j}}$. In particular, the module $\sL_{\phi^{-1}(U_j)}$ generated by the rational functions 
$$
\{(\sigma_i)_{\phi^{-1}(U_j)} = \phi^*(x_i/x_j)\}_{0\leq i\leq n}
$$
 is a free $\sO_{W_{i,j}}$-module on 1 generator. On the preimage of $U_j\cap U_k$ these sections differ by the common unit $\phi^*(x_k/x_j)$, and thus the collection of these modules defines an invertible sheaf $\sL$ on $X$ together with an
$r+1$-dimensional space of global sections $\sV := \langle \sigma_0,\dots \sigma_r\rangle$ that forms a basepoint free linear system. Note that the subscheme  $\{\sigma_k = 0\} \subset W_{i,j}$  is the scheme-theoretic preimage of the
the hyperplane $\{x_k = 0\}\subset \PP^r$. This completes the explanation and proof of Theorem~\ref{morphisms and linear systems}


\subsection{More about linear systems}

Let $\sV = (\sL, V)$ be a linear system on $X$.  The linear system is said to be \emph{complete} if $V = H^0(\cL)$; in this case it is sometimes denoted $|\cL|$. If $\cL \cong \cO_C(D))$, we also write it as $|D|$. 
 If $D$ is any divisor on $C$ we write $r(D)$ for the dimension of the complete linear series $|D|$; that is, $r(D) = h^0(\cO_C(D)) - 1$. Finally, a linear system of dimension 1 is called a \emph{pencil}, a linear system of dimension 2 is called a \emph{net} and, less commonly, a three-dimensional linear system is called a \emph{web}. 

We can relate the geometry of the morphism associated to an incomplete linear system $V \subset H^0(\cL)$ to the geometry of the morphism associated to the complete linear system $|\cL|$. In general, if $V \subset W \subset H^0(\cL)$ are a pair of nested linear systems, we have a linear map $W^* \to V^*$ dual to the inclusion $V \hookrightarrow W$, and a corresponding linear projection $\pi : \PP W^* \dashrightarrow \PP V^*$, with indeterminacy locus the subspace $\PP(Ann(V)) \subset \PP W^*$. In this case, we have 
$$
\phi_V = \pi \circ \phi_W;
$$
that is, we have the diagram 

\begin{diagram}
& & \PP W^* \\
& \ruTo^{\phi_W} & \dDashto_\pi \\
C & \rTo^{\phi_V} & \PP V^*.
\end{diagram}
In this case, given that $W$ is base-point-free, the condition that $V$ be base-point-free is equivalent to saying that the center $\PP(Ann(V))$ of the projection $\pi$ is disjoint from $\phi_W(C)$.

By way of language, we will say that a curve $C \subset \PP^r$ embedded by a complete linear series is \emph{linearly normal}; this is equivalent to saying that the restriction map
$$
H^0(\cO_{\PP^r}(1)) \to H^0(\cO_{C}(1))
$$
is surjective, which is in turn equivalent to saying that $C$ is not the regular  projection of a nondegenerate curve $\tilde C \subset \PP^{r+1}$.

A linear system $\sV = (\sL, V)$ is called  \emph{very ample}  if it is basepoint-free and defines an embedding. If $D$ is a Cartier divisor on $X$, then we say that $D$ is \emph{very ample} if the complete linear system $|D|$ is versy ample, and we say that $D$ is \emph{ample} if $mD$ is very ample for some integer $m>0$.


Given a linear system $\sV = (\sL, V)$ and an effective divisor $D$ on $C$, we'll  set
$
\sV(-D) = (\sL(-D),V(-D))
$
where
$$
\sL(-D): = \sL \otimes \sO(-D)\hbox{ and } V(-D) := \{ \sigma \in V \mid \sigma(D) = 0 \}.
$$
The difference $\dim \sV - \dim \sV(-D)$ is called the \emph{number of conditions imposed by $D$ on the linear system $\sV$}; we say that $D$ \emph{imposes independent conditions} on $\sV$ if $\dim \sV - \dim \s V(-D) = \deg(D)$.

On the other hand,  the invertible sheaf $\cL(D) = \cL \otimes \cO(D)$ to be the sheaf of rational sections $\sigma$ of $\cL$ satisfying $\ord_{p_i}(\sigma) \geq -m_i$ for all $i$; as a line bundle, this is the same as $\cL \otimes \cO_C(D)$.

Via the correspondence of Theorem~\ref{morphisms and linear systems}, the statements about the geometry of a morphism $\phi : C \to \PP^r$ can be formulated as statements about the relevant linear systems. We will see this in many instances throughout this book. It will be most convenient to formulate this in terms of the vector space $H^{0}(\sL)$ of global sections of $\sL$, and we write $h^{0}(\sL)$ for the dimension of this vector space. Here is a first example:

\begin{proposition}\label{very ample}\cite[Thm. IV.3.1]{H}
Let $\cL$ be an invertible sheaf on a smooth curve $C$. The complete linear system $|\cL|$ is base-point-free iff
$$
h^0(\cL(-p)) = h^0(\cL) - 1 \quad \forall p \in C;
$$
and $\sL$ is very ample, iff
$$
h^0(\cL(-p-q)) = h^0(\cL) - 2 \quad \forall p, q \in C.
$$
\end{proposition} 

\begin{proof}
Since a divisor of degree $d$ cannot impose more than $d$ condtions on a linear system, the statement $h^0(\cL(-p-q)) = h^0(\cL) - 2$ for all $p, q$ implies the condition for base-point freeness; and saying that $\phi_\cL(p) \neq \phi_\cL(q)$ implies that the linear system defines a set-theoretic injection. The tangent space of $C$ at $p$ is $(\sI_C(p)/\sI_C(p)^2)^*$, so the condition that there is a section of $\sL$ that vanishes at $p$, but does not vanish
to order 2, implies that the differential $d\phi_\cL$ is injective at $p$ as well.

Let $\phi$ be the map defined by $\sL$, and suppose that the image of $\phi$ is the curve $D$. To say that $\phi$  is an isomorphism locally at a point $p$, we need to know that the map of local rings
$$
\phi^*: \sO_{D,\phi(p)} \to \sO_{C,p}
$$
is an isomorphism. What we have shown so far is that 
$$
\phi^*: \frac{\sO_{D,\phi(p)}}{\gm_{D,\phi(p)}^2} \to \frac{\sO_{C,p}}{\gm_{C,p}^2}
$$
is an isomorphism. This is not enough, as we can see from the fact that the completion of $\sO_{C,p}$ is not
isomorphic to $\sO_{C,p}.$

The key additional fact is that $\phi$ makes $\sO_{C,p}$  a \emph{finitely generated} module over 
$\sO_{D,\phi(p)}$;
given this, the conclusion follows from Nakayama's Lemma, which shows that
$\sO_{D,\phi(p)}$ is generated as a module by the element 1.
See the exposition in \cite[Proposition 7.3 and Lemma 7.4]{H} for details.
\end{proof}


\begin{exercise}
Extend the statement of Proposition~\ref{very ample} to incomplete linear systems; that is, prove that the morphism associated to a linear system $(\cL, V)$ is an embedding iff
$$
\dim\big( V \cap H^0(\cL(-p-q))\big) = \dim V - 2 \quad \forall p, q \in C.
$$
\end{exercise}

\begin{exercise}
An automorphism of $\PP^r$ takes hyperplanes to hyperplanes. Deduce that it is given by the linear system
$\sV = \sO_{\PP^r}(1), H^0(\sO_{\PP^r}(1))$, and use this to show that $\Aut \PP^r = PGL(r+1)$. 
\end{exercise}

For another example of the relationship between linear series on curves and morphisms of curves to projective space, consider a smooth curve $C \subset \PP^r$ embedded in projective space, and assume that $C$ is linearly normal. If $\phi : C \to C$ is any automorphism, we can ask whether $\phi$ is induced by an automorphism of $\PP^r$; in other words, does there exist an automorphism $\Phi : \PP^r \to \PP^r$ such that $\Phi(C) = C$ and $\Phi|_C = \phi$? The answer is expressed in the following exercise.

\begin{exercise}\label{projective automorphism}
In the circumstances above, the automorphism $\phi$ is induced by an automorphism of $\PP^r$ if and only if $\phi$ carries the invertible sheaf $\cO_{C}(1)$ to itself; that is, $\phi^*(\cO_{C}(1)) = \cO_{C}(1)$.
\end{exercise}

\begin{example}
Consider the rational normal curve of degree $d$, which is the image of the morphism $\PP^1\to \PP^d$ given by the complete linear system $|\cO_{\PP^1}(d)|$. Since there is a unique invertible sheaf of each degree $n$ on $C$, and the curve is linearly normal, we see that \emph{every automorphism of a rational normal curve $C \subset \PP^d$  is projective}, so ``the'' rational normal curve of degree $d$ is well-defined up to an automorphism of $\PP^d$.  A similar statement holds
for the image of any Veronese morphism.
\end{example}

%If $\sL, \sL'$ are linear systems on a smooth curve $C$ and $D = (\sigma)_0,D' = (\sigma')_0$ are the divisors of zeros of sections of $\sL$ and $\sL'$ respectively, then $D+D'$ is the divisor of zeros of the section $\sigma\otimes \sigma'$ of
%$\sL\otimes \sL'$.


If $\phi:X \to \PP^r$ is a generically finite morphism, then the \emph{degree of $\phi$} is the number of points in the preimage of a general point of $\phi(X)$. Thus, for example, if $D := \sum_{p\in C} n_pp$ is a divisor on a smooth curve, and the linear system $|D|$ is basepoint free, then the degree of the morphism associated to $|D|$ is $\deg D := \sum_{p\in C} n_p$.

\subsection{The most interesting linear system}

The most important invertible sheaf on a smooth variety $X$ is the sheaf of global sections of the top exterior power of the  the cotangent bundle of $X$, called the canonical sheaf $\omega_X$ of $X$. A section of 
$\omega_X$ is thus a differential form of degree equal to the dimension of $X$, and the divisor class
of such a form is usually denoted $K_C$. 
\begin{theorem}
 The canonical sheaf of $\PP^{r}$ is $\sO_{\PP^{r}}(-r-1)$. 
\end{theorem}
\begin{proof}
Let $x_{0}, \dots, x_{r}$ be the projective coordinates on $\PP^{r}$ and let  $U = \PP^{r}\setminus H$ be the affine open set where $x_{0} \neq 0$. Thus $U \cong \AA^{r}$ with coordinates $z_{1 := }x_{1}/x_{0}, \dots, z_{r}:=x_{r}/x_{0}$. The space of $r$-dimensional differential forms on $U$ is spanned by $d(x_{1}/x_{0})\wedge\cdots\wedge d(x_{r}/x_{0})$, which is regular everywhere in $U$. In view of the formula
$$
d\frac{x_{i}}{x_{0}} = \frac{x_{0}dx_{i}-x_{i}dx_{0}}{x_{0}^{2}}
$$
we get
$$
d(x_{1}/x_{0})\wedge\cdots\wedge d(x_{r}/x_{0}) = \frac{dx_{1}\wedge\cdots\wedge dx_{r}}{x_{0}^{r}}-
\sum_{i=1}^{r} x_{i} \frac{ dx_{1}\wedge\cdots \wedge \widehat{d_{x_{i}}}\wedge \cdots \wedge dx_{r}}{x_{0}^{r+1}}
$$
which has a pole of order $r+1$ along the locus $H$ defined by $x_{0}$. Thus the divisor of this differential form
is $-(r+1)H$, and this is the canonical class.
\end{proof}

\begin{fact}
A different derivation: there is a short exact sequence of sheaves of differentials, called the Euler sequence:
$$
0\to \Omega_{\PP^{r}} \to \sO_{\PP^{r}}^{r+1} (-1) \to \sO_{\PP^{r}} \to 0.
$$
Taking exterior powers, we see that
$$
\bigwedge^{r}\Omega_{\PP^{r}} \otimes \bigwedge^{1}\sO_{\PP^{r}} = \bigwedge^{r+1} (\sO_{\PP^{r}}^{r+1} (-1)) = \sO_{\PP^{r}}(-r-1).
$$
\end{fact}

Computations of the canonical sheaf on a variety usually involve comparing the variety to another variety, such as projective space, where the canonical sheaf is already known. The most useful results of this type are  the \emph{adjunction formula}
and the \emph{Hurwitz' Theorem}.

\begin{proposition}\label{adjunction}(Adjunction Formula)
 Let $X$ be a variety that is a Cartier divisor on a variety $Y$. If the canonical divisor of $S$ is $K_{Y}$, then
 $K_{X}$ is the restriction to $X$ of the divisor $K_{Y}+X$.
\end{proposition}
This is a special case of \cite[****]{H}.
\begin{proof}
 There is an exact sequence of sheaves
 $$
0\to  \sI_{X/Y}\mid_{X} \to \Omega_{Y}\mid _{X} \to \Omega_{X} \to 0
 $$
 where $\Omega_{X}$ is the sheaf of differential forms on $X$ (see \cite[Theorem ***]{Eisenbud95}), and
$ \sI_{X/Y}\mid_{X} = \sO_{Y}(-X)\mid_{X} = \sO_{X}(-X)$. The proposition follows by taking top exterior powers.
\end{proof}

\begin{corollary}\label{canonical of plane curve}
If $C\subset \PP^{2}$ is a smooth plane curve of degree $d$, then $\omega_{C} = \sO_{C}(d-3)$; more generally, if
$X\subset \PP^{r}$ is a complete intersection of hypersurfaces of degrees $d_{1},\dots, d_{c}$ then
$\omega_{X} = \sO_{X}(\sum_{i}d_{i }-r-1).$
\end{corollary}

 Given a (nonconstant) morphism $f : C \to X$ of smooth projective curves, the Riemann-Hurwitz formula computes the canonical sheaf  $C$ in terms of that of  $X$ and the local geometry $f$. To do this we define the
\emph{ramification index} of $f$ at $p$,  denoted $\ram(f,p)$, by the formula of divisors
$$
 f^{-1}(f(p)) = \sum_{p\in C \mid f(p)=q} (\ram(f,p)+1)\cdot p
 $$
In terms of a suitable choice of local coordinates $z$ on $C$ around $p$ and $w$ on $X$ around $f(p)$, we can write the morphism as $z \mapsto w = z^m$ for some integer $m > 0$, and $\ram(f,p) = m-1$.

It follows from complex analysis (or the separability of field extensions in characteristic 0) that there are only finitely many
points on $C$ where $\ram(f,p) \neq 0$ (this would be false in characteristic $>0$ in the case where the
induced extension of fraction fields was inseparable.) Thus we may define the \emph{ramification divisor} of $f$ to be the divisor
 $$
 R = \sum_{p \in C} \ram(f,p)\cdot p \; \in \;  \Div(C).
 $$
 and the \emph{branch divisor} to be
 $$
 B = \sum_{q \in X} \Big(\sum_{p \in f^{-1}(q)} \ram(f,p) \Big)\cdot q \; \in \; \Div(X).
 $$
 Note that $R$ and $B$ have the same degree $\sum_{p \in C} \ram(f,p)$. 

 
\begin{theorem}(Hurwitz' Theorem) \cite[****]{H} \label{Hurwitz}
If $f:C\to X$ is a non-constant morphism of smooth curves, with ramification divisor $R$, then 
$$
\omega_{C} = f^{*}\omega_{X}(-R).
$$
\end{theorem}
 
 
\begin{proof}
Choose a rational 1-form $\omega$ on $X$, and $\eta = f^*(\omega)$ be its pullback to $C$. For simplicity, we will assume that the zeroes and poles of $\omega$ lie outside the branch divisor $B$, so that $\omega$ will be regular and nonzero at each branch point. (Since we have the freedom to multiply by any rational function on $X$ we can certainly find such a form, and in any event the calculation goes through without this assumption, albeit with more complicated notation.) 

Since the zeroes of $\omega$ lie outside the branch divisor $B$, for every zero of $\omega$ of multiplicity $m$ we have exactly $d$ zeroes of $\eta$, each with multiplicity $m$; and likewise for the poles of $\omega$. Meanwhile, at every point of $B$, the form $\omega$ is regular and nonzero. At a point $p$ where (locally) $f$ has the form $z \mapsto w = z^{e}$
and $\omega = dw,\ \eta dz$ we have $\eta = z^{e-1}dz$; that is $\eta$ has a zero of multiplicity $\ram(f,p)$ at  $p$.
Thus the divisor $K_{C}$ of $\eta$ is
$K_{C} = df^{*}(K_{X})+R$.
\end{proof}

\begin{example}
 Let $V$ be the vector space of homogeneous polynomials of degree $d$ in two variables; that is, $V = H^0(\cO_{\PP^1}(d))$. In the projectivization $\PP(V^{*}) \cong \PP^d$, let $\Delta$ be the locus of polynomials with a repeated factor. Since $\Delta$ is defined by the vanishing of the discriminant, it is a hypersurface. What is its degree?
 
 To answer this, let $W^{*}\subset V^{*}$ be a general 2-dimensional linear subspace---that is, a general pencil of forms of degree $d$ on $\PP^1$. The linear system $\sW = (\sO_{\PP^{d}}, W^{*})$ defines a morphism $\phi_{\sW} : \PP^{1} \to \PP(W) \cong \PP^{1}$ and the fiber over the point of $\PP(W)$ corresponding to a form $f$ of degree $d$ is the divisor $f = 0\subset \PP^{1}$. Thus the locus of polynomials in $W$ with a multiple root is the branch locus of $\phi_{\sW}$, where we count an $m$-fold root $m-1$ times.
 By Hurwitz' formula, the degree of the branch locus $B$ of a degree $d$ morphism from $\PP^{1}$ to $\PP^{1}$ is
 $$
 \deg B = \deg \omega_{\PP^{1}} - d\deg \omega_{\PP^{1}} = 2d-2.
 $$
 \end{example}
 
\begin{fact}
A famous result asserted by Franchetta and proved by **** is that the canonical sheaf (and its powers) are the \emph{only} sheaves that can be chosen uniformly among all, or even almost all, smooth curves. For a more precise statement, see ****.
\end{fact}

\section{Genus, Riemann-Roch and Serre Duality}

We will henceforward assume that the reader is acquainted with sheaf cohomology, at least sufficiently to write
$H^i(X, \sF$ or  $H^i(\sF)$ without blushing. If $D$ is a divisor on a scheme $X$ we will often
abbreviate $H^i(\sO_X(D))$ to $H^i(D)$, and we write $h^i(\sF)$ or $h^{i}(D)$ for $\dim_{\CC}H^i(\sF)$ or $\dim_{\CC}H^i(D)$. For $i>0$, $h^{i}(\sF)$ often appears as a kind of ``error term'' in formulas when one would like to compute
$H^{0}(\sF)$, so vanishing theorems have an important place in all of algebraic and analytic geometry. We will use the simplest of these often.
Note that if $\sF$ is a sheaf on $X$ and $X\subset Y$ then the cohomology  $H^i(X, \sF) = H^i(Y,\sF)$ canonically, so we will
simply and unambiguously write $H^i(\sF)$ for either of these.

\begin{theorem}[Serre Vanishing Theorem]\label{Serre vanishing} If $\sF$ is a coherent sheaf on $\PP^n$, then
$H^{i}(\sF)= 0$ for all $i>\dim \supp \sF$; and  $H^{i}(\sF(d))= 0$ for all $i>0$ and $d\gg 0$  
\end{theorem}

The (Zariski) Euler characteristic of a coherent sheaf $\sF$ is by definition 
$$
\chi(\sF) = \sum_{i=0^\infty} (-1)^ih^i(\sF).
$$
Using the second part of Theorem~\ref{Serre vanishing}, we see that the Euler characteristic of a coherent sheaf $\sF$ on a curve
 is  $\chi(\sF) = h^0(\sF) - h^1(\sF)$.
 
 The other important cohomological result we will use is duality. We will use it only for invertible sheaves on curves, so we give it in
 that special case:
 
\begin{theorem}[Serre Duality]\label{sd}
If $C$ is a smooth curve and $D$ is a divisor on $C$, then
$$
H^1(D) =\Hom_\CC(H^0(K_C-D), \CC),
$$
and thus $h^1(D) = h^0(K_C-D)$.
\end{theorem}

\subsection{The genus of a curve}

The sole topological invariant of a smooth projective curve $C$ is its genus. We can think of $C$ as a submanifold of the complex projective space $\PP^r(\CC)$ with the classical topology; as such, it is a compact, oriented surface, and its genus is half the rank of its first integral homology, $H^{1}(C; \ZZ)$---informally, the ``number of holes'':

\includegraphics[scale = 1]{RiemannSurface}
**** Riemann Surface of genus 3, from Wikimedia ****

Of course this definition does not apply to curves over fields other than $\CC$, and doesn't relate the genus to the algebra of the curve. However, we can relate the topological genus of a curve directly to its topological Euler characteristic. Since $C$ is topologically compact, connected, and oriented, we have
$H^{0}(C; \ZZ)=H^{2}(C; \ZZ) = \ZZ$, so:
$
\chi_{top}(C) = 2-2g.
$
By the Hopf index theorem, the topological Euler characteristic is the degree of the tangent sheaf, or equivalently, minus the degree of the cotangent sheaf $\omega_{C}$; that is, $\deg K_{C} = 2g-2$, and thus
$$
g(C) = \frac{\deg(K_C)}{2} + 1.
$$
This formula serves to define the genus of a smooth projective curve over any field. It is useful to 

Other characterizations of the genus require more machinery to establish. We will state some here, and use  tools from the following section to prove equivalence.

\begin{enumerate}

\item 
$
g(C) = 1 - \chi(\cO_C). \label{pa}
$

\item\label{genus Hilbert} If $C \subset \PP^r = \PP(V)$ is a smooth curve of degree $d$  with homogeneous coordinate ring
$S_C$, then for sufficiently large $d$ the function $d \mapsto p_C(m) := \dim_\CC (S_C)_d$ is equal to a linear function
$$
p_C(m) =  dm - g + 1,
$$
so $g(C) = dm+1-p_C(m)$. 

\item\label{genus 1forms} $g(C)$ is the dimension of the vector space of regular 1-forms (that is, global sections of the
cotangent sheaf) on $C$.
\end{enumerate}


Applied to a possibly singular or even non-reduced 1-dimensional projective scheme, the formula~\ref{pa} and the equivalent~\ref{genus Hilbert}, define
what is called the \emph{arithmetic genus} $p_a(C)$. By contrast, if $C$ is reduced, so that the normalization
$\widetilde C \to C$ makes sense, then the genus of $\widetilde C$ is called the \emph{geometric genus} of $C$.

\subsection{The Riemann-Roch Theorem}

To prove that these formulas for the genus are correct, we use \trr and Serre duality (sometimes called Kodaira-Serre duality, since Kodaira was responsible for the analytic version.)

\begin{theorem}[Riemann-Roch Theorem]\label{RR}
 If $C$ is a smooth, connected projective curve of genus $g$, and $D$ a divisor of degree $d$ on $C$ then
$$
h^0(D) = d - g + 1 + h^0(K_C - D).
$$
\end{theorem}

For example, if we take $D=0$, this tells us that $h^0(K) = g$, proving the characterization~(\ref{genus 1forms}) above. Using this and \trr for $D=K$
we get $\deg K = 2g-2$. Also, since $h^0(D) = 0$ for any divisor $D$ of negative degree, the formula gives the dimension of $h^{0}(D)$ when $\deg D$ is large:

\begin{corollary}\label{nonspecial RR}
For any divisor $D$ we have
 
$
h^0(D) \geq d - g + 1.
$
with equality if $d > 2g-2$.
\end{corollary}
This was the theorem originally proven by Riemann; his student Roch supplied the correction term $h^0(K_C - D)$ for divisors of lower degree.
the dimension $h^0(K_C-D) = h^1(D)$ was called the \emph{superabundance} of $D$.

Corollary~\ref{nonspecial RR} and Proposition~\ref{very ample} together show that all high degree divisors come from hyperplane sections in 
suitable embeddings:

\begin{corollary}\label{degree 2g+1 embedding}
Let $D$ be a divisor of degree $d$ on a smooth, connected projective curve of genus $g$. If $d \geq 2g$, the complete linear series $|D|$ is base point free; and if $d \geq 2g+1$ the associated morphism $\phi_D : C \to \PP^{d-g}$ is an embedding, so that
$D$ is the preimage of the intersection of $C$ with a hyperplane in $ \PP^{d-g}$.
\end{corollary}

Since the complement of a hyperplane in projective space is an affine space, we get an affine embedding result too:

\begin{corollary}
 If $C$ is any smooth, connected projective curve and $\emptyset \neq \Gamma \subset C$ a finite subset then $C \setminus \Gamma$ is affine.
\end{corollary}
\begin{proof}
Let $D$ be the divisor defined by $\Gamma$ By Corollary~\ref{degree 2g+1 embedding} a high multiple of $D$ is very ample,
and gives an embedding $\phi: C\to \PP^n$ such that the preimage of the intersection of $C$ with some hyperplane $H$
is a multiple of $D$. It follows that $C\setminus \Gamma$ is embedded in $\AA^n = \PP^n\setminus H$.
\end{proof}
 
We can  use Corollary~\ref{nonspecial RR} to determine the Hilbert polynomial of a projective curve. To do this, let $C \subset \PP^r$ be a smooth curve of degree $d$ and genus $g$, and consider the exact sequence of sheaves
$$
0 \rTo \cI_{C/\PP^r}(m) \rTo \cO_{\PP^r}(m) \rTo \cO_C(m) \rTo 0
$$
and the corresponding exact sequence
$$
 H^0(\cO_{\PP^r}(m)) \rTo^{\rho_m} H^0(\cO_C(m)) \rTo H^1(\cI_{C/\PP^r}(m)) \rTo 0.
$$

The \emph{Hilbert function} $h_C$ of $C$  is defined by
$$
h_C(m) = \dim_{\CC} (S_{C})_{m} = \rank(\rho_m).
$$
By Theorem~\ref{Serre vanishing} we have $H^1(\cI_{C/\PP^r}(m)) = 0$ for large $m$, so $h_{C}(m) = h^0(\cO_C(m))$, for large $m$, which, by \trr, equals $md-g+1$, again for large $m$. Thus, the Hilbert polynomial of $C \subset \PP^r$ is $p_C(m) = dm-g+1$, establishing the characterization~(\ref{genus Hilbert}).
 
The Riemann-Roch formula does \emph{not} give us a formula for the dimension $h^0(D)$ when $h^0(K_C - D)>0$; such divisors $D$ are called \emph{special divisors}, or \emph{special divisor classes}. The existence or non-existence of divisors $D$ with given $h^{0}(D)$ and $h^{1}(D)$ often serves to distinguish one curve from another, and will be an important part of our study.

\begin{fact}
If $k$ is a field that is not algebraically closed there may be smooth projective genus 0 curves over $k$ that are not isomorphic to $\PP^1$. However, they must be``forms'' of $\PP^1$ in the sense that they become isomorphic to $\PP^1$ after extension of scalars to 
the algebraic closure $\overline k$ of $k$. The unique example with $k = \RR$ is the conic $x^2+y^2+z^2 = 0$. Indeed, any form of $\PP^1$ over any field $k$ can all be embedded in $\PP_{k}^2$  using the anti-canonical linear system.

The curve $\PP_k^1$ itself may be described as the scheme of left ideals of $k$-vector-space dimension 1 in the ring of
$2\times 2$ matrices over $k$ (such an ideal can be embedded in the matrix ring as a linear combination of the 2 columns in an appropriate sense). More generally, any scheme that is a form of $\PP^1$ over $k$
may be described as the scheme of 1-dimensional left ideals in a 4-dimensional central simple ($=$ Azumaya) algebra over $k$. For example, the
conic $x^2+y^2+z^2 = 0$ with no points over $\RR$ is the scheme of left ideals in the algebra of quaternions.\end{fact}

\subsection{A partial proof}

If, following~\ref[Chapter IV]{H} we take the definition of the genus of  a smooth connected curve $C$ to be $h^1(\sO_C)$ (so that $g = h^0(K_C)$ becomes a Corollary of Theorem~\ref{sd}), then it is easy to the following form of the Riemann-Roch Theorem:

\begin{corollary}
 If $C$ is a smooth, connected projective curve and $D$ is a divisor on $C$ then
$$
\chi(\sO_C(C)) := h^0(D) - h^1(D) = d-g+1
$$
or in other words, for any invertible sheaf $\cL$ of degree $d$ on $C$,
$$
\chi(\cL) = d-g+1
$$
\end{corollary}
\begin{proof}
 
\end{proof}
Taking $D=0$ the statement becomes $h^0(\sO) - h^1(\sO) = 1-g$, which is our definition of $g$. On the other hand,
For any divisor $D$ on $C$ and any point $p \in C$ we have an exact sequence of sheaves
$$
0 \to \cO_C(D-p) \to \cO_C(D) \to \frac{\cO_C(D)}{\gm_{C,p}\cO_C(D)} \to 0
$$
Since $\cO_C(D)$ is locally isomorphic to $\sO_C$we see that $cO_C(D)/\gm_{C,p}\cO_C(D)\cong \kappa(p)$ is a sky-scraper sheaf of dimension 1, concentrated at $p$,
and thus has Euler characteristic 1. 

Thus the Riemann-Roch Theorem for $\cO_C(D)$ is equivalent to the Riemann-Roch Theorem for $\cO_C(D-p)$. Since any divisor can be obtained from 0 by adding and subtracting points, the Riemann-Roch formula for an arbitrary $\cL$ follows from the special case $\cL = \cO_C$.

\subsection{Clifford's theorem}

While the Riemann-Roch Theorem gives a lower bound for the dimension of a linear system, $r(\sL) := h^0(\sL)-1 \geq \deg \sL -g$, Clifford's Theorem
gives an upper bound. If $\deg \sL>2g-2$, then the Riemann-Roch inequality becomes an equality, so it is enough to treat the case $\deg \sL \leq 2g-2$.
To state the sharpest form, we define a curve $C$ of genus $\geq 2$ to be \emph{hyperelliptic} if there exists a $g^1_2$ on $C$; that is an
invertible sheaf $\cL$ of degree $2$ with 2 independent global sections. We will see in Chapter~\ref{genus 2 and 3 chapter} that such a sheaf would be unique, and $h^0(\cL) = 2$.


\begin{theorem}\label{Clifford}
Let $C$ be a curve of genus $g$ and $\cL$ a line bundle of degree $d \leq 2g-2$. Then
$$
r(\cL) \leq \frac{d}{2}.
$$
Moreover, if  equality holds then we must have either
\begin{enumerate}
\item $d=0$ and $\cL = \cO_C$;
\item $d = 2g-2$ and $\cL = K_C$; or
\item $C$ is hyperelliptic, and $|\cL|$ is a multiple of the $g^1_2$ on $C$.
\end{enumerate}
\end{theorem}

The proof of the inequality will follow easily from a basic result about the addition of linear series, defined as follows:
$\cD = (\cL,V)$ and $\cE = (\cM, W)$ be two linear series on a curve $C$. By the \emph{sum} $\cD + \cE$ of $\cD$ and $\cE$, we will mean the pair 
$$
\cD + \cE = (\cL \otimes \cM, U) 
$$
where $U \subset H^0(\cL \otimes \cM)$ is the subspace generated by the image of $V \otimes W$, under the multiplication/cup product map $H^0(\cL) \otimes H^0(\cM) \to H^0(\cL \otimes \cM)$---in other words, it's the subspace of the complete linear series $|\cL\otimes \cM|$ spanned by divisors of the form $D+E$, with $D \in \cD$ and $E \in \cE$.
 
\begin{proof}
If $\cD$ and $\cE$ are two nonempty linear series on a curve $C$, then
$$
\dim(\cD + \cE) \geq \dim \cD + \dim \cE.
$$
To see this, we observe that to say $\dim \cD \geq m$ means exactly that we can find a divisor $D \in \cD$ containing any given $m$ points of $C$; since $\cD + \cE$ contains all pairwise sums $D + E$ with $D \in \cD$ and $E \in \cE$, we can certainly find a divisor $F \in cD + \cE$ containing any given $\dim \cD + \dim \cE$ points of $C$.

The proof of the inequality in Clifford's Theorem follows  by applying this observation to the pair $|\cL|$ and $|K_C\otimes \cL^{-1}|$: by 
the Riemann-Roch Theorem, we have
$$
r(K_C\otimes \cL^{-1}) = r(\cL) +g - d - 1
$$
and so we deduce that
$$
g = r(K_C) + 1 \geq r(\cL) + r(K_C\otimes \cL^{-1}) + 1 \geq 2r(\cL) +g - d;
$$
hence $r(\cL) \leq d/2$.

For the proof of the second half of Clifford we will use a basic fact about the geometry of hyperplane sections of a curve in projective space (Proposition~\ref{monodromy of hyperplane section}); we defer it to there.
\end{proof}



\section{The canonical morphism}

Given the central role played by the canonical divisor class, it is natural to look at the geometry of the morphism $\phi_K : C \to \PP^{g-1}$ associated to the complete canonical series $|K|$.  

\begin{definition}
A curve $C$ of genus $g \geq 2$ is said to be \emph{hyperelliptic} if there exists a morphism $f: C \to \PP^1$ of degree 2. \end{definition}

\begin{theorem}\label{canonical system is very ample}
If $C$ is a smooth curve of genus $\geq 2$ then the canonical morphism $\phi_K : C \to \PP^{g-1}$ is an embedding if and only if $C$ is not hyperelliptic.
\end{theorem}

\begin{proof}
By Corollary~\ref{degree 2g+1 embedding} we have to show that for any pair of points $p, q \in C$ we have
$$
h^0(K_C(-p-q)) = h^0(K_C)-2 = g-2.
$$
Applying \trr we see this fails if and only if $h^0(\cO_C(p+q)) \geq 2$ for some $p,q \in C$. By Lemma~\ref{deg 2 morphism}, this implies that $C$ is hyperelliptic.
\end{proof}

\begin{lemma}\label{deg 2 morphism}
Let $C$ be a smooth, projective curve of genus $g\geq 2$. If $C$ has an invertible sheaf $\cL$ of degree 2 with two independent sections, then
$|\cL|$ defines a morphism of degree 2 to $\PP^{1}$, and $C$ is hyperelliptic. In particular, if $g(C) = 2$ then the canonical series $|K_{C}|$ defines a 2 to 1 morphism to $\PP^{1}$, so $C$ is hyperelliptic.
\end{lemma}

\begin{proof}
If $\cO_C(p+q)$ has two independent sections and has $d$ basepoints, then it defines a morphism of degree $2-d$ to $\PP^1$. Since $C$ is not rational,
we must have $d=0$, proving the first statement. 
\end{proof}

%Furthermore, $C$ is hyperelliptic. To finish the proof, by Corollary~\ref{degree 2g+1 embedding} it suffices to show that
% an invertible sheaf $\sL$ of degree 1 on $C$ must have $h^{0}(\sL)\leq 1$.
% 
%Suppose that $\sigma_{0}, \sigma_{1}$ were two linearly independent sections of $\sL$. Each $\sigma_{i}$ vanishes at a unique point $p_{i}$. If $p_{0}= p_{1}$ then a linear combination of $\sigma_{0}, \sigma_{1}$ would be a section vanishing to order $\geq 2$, which is impossible, so $\sL$ is basepoint free, and defines a degree 1 morphism $C\to \PP^{1}$. Such a morphism must be an isomorphism (because $\PP^{1}$ is normal), contradicting $g(C) \geq 2$.
%
%
%\fix{the following argument is only set-theoretic. Admit this or make it precise}
%Note that if $C$ is hyperelliptic, the morphism $\phi_K$ factors through the degree 2 morphism $\pi : C \to \PP^1$: if $\{p,q\} \subset C$ is a fiber of this morphism, we have $h^0(\cO_C(p+q)) = 2$ and hence $\phi_K(p) = \phi_K(q)$. The image of the morphism $\phi_K$ is a nondegenerate curve of degree $g-1$ in $\PP^{g-1}$, which we will see is a \emph{rational normal curve}. This observation implies in particular that if $C$ is hyperelliptic of genus $g \geq 2$, then the invertible sheaf $\cL$ of degree 2 with $h^0(\cL) = 2$ is in fact unique.
%
%\fix{note that the next para needs a parameterization of the curves of genus }
%
%Among curves with $g \geq 3$ the hyperelliptic curves are very special: in the family of all curves, as we'll see, they comprise a closed subvariety. Also, the behavior of linear series and morphisms on a hyperelliptic curve is very different from that of series on a general curve; when we discuss the geometry of curves of low genus in the Chapter~\ref{}, we will exclude  the hyperelliptic case, and deal with this case in a separate chapter.
%
%For non-hyperelliptic curves, however, the geometry of the canonical morphism, and its image, the canonical curve, are the keys to understanding the curve. We'll see this in detail in many cases in the following chapter; for now, we mention one highly useful result along these lines.
%
%\fix{add here: canonical series on plane curves cut by $|\cO_{\PP^2}(d-3)|$; consequence that no smooth plane curve can be hyperelliptic}
%
%\fix{maybe move initial discussion of hyperelliptic curves from Ch. 6 to a section here}
%
%\fix{maybe add to this chapter: differentials on plane curves $C$, possibly with nodes or more general singularities; adjoint conditions; algorithm for determining the complete linear system associated to a divisor $D$ on $C$}
%
\subsection{The geometric Riemann-Roch theorem}

If $C$ is a nonhyperelliptic curve, embedded in $\PP^{g-1}$ by its canonical series and  $D = p_1+\dots + p_d$ is a divisor consisting of $d$ distinct points; let $\overline D$ be the span of the points $p_i \in C \subset \PP^{g-1}$. Since the hyperplanes in $\PP^{g-1}$ containing $\{p_1,\dots,p_d\}$ correspond (up to scalars) to sections of $K_C$ vanishing at all the points $p_i$, we see that
$$
h^0(K_C-D) = g - 1 - \dim \overline D.
$$
Plugging this into the Riemann-Roch formula, we arrive at the statement
$$
r(D) = d - 1 - \dim \overline D;
$$
or in other words, the dimension of the linear series $|D|$ in which the divisor $D$ moves is equal to the number of linear relations on the points $p_i$ on the canonical curve. Thus, for example, if $D = p_1+p_2+p_3$, we see that $D$ moves in a pencil if and only if the points $p_i$ are collinear.

We can extend this statement to the case of arbitrary effective divisors $D$ on any smooth curve if we define our terms correctly. To do this, suppose $f : C \to \PP^d$ is any morphism, and $D \subset C$ any divisor. We define the \emph{span} of  $f(D)$ to be the intersection
$$
\overline{f(D)} = \bigcap_{H \mid f^{-1}(H)\supset D} H 
$$
of all hyperplanes in $\PP^d$ whose preimage in $C$ contains $D$. The argument above applies to prove:

\begin{theorem}[Geometric Riemann-Roch Theorem]\label{geometric RR}
If $C$ is any curve of genus $g \geq 2$,  $\phi : C \to \PP^{g-1}$ its canonical morphism and $D \subset C$ any effective divisor of degree $d$, then
$$
r(D) = d - 1 - \dim \overline{\phi(D)}.
$$
\end{theorem}
 
 \section{Canonical Curves}\label{Noether theorem section}

When we analyze the embedding of a curve $C\subset \PP^n$ we generally ask first whether $C$ is contained in any low-degree
hypersurfaces, and if so how many, in the sense of the dimension $(I_C)_d)$ the degree $d$ part of the homogeneous ideal of $C$.
From the long exact sequence in cohomology
$$
0\to H^0(\sI_{C/\PP^n}(d)) \to H^0(\sO_{\PP^n}(d) \to H^0(\sO_C(d) \to H^1(\sI_{C/\PP^n}(d)) \to H^1(\sO_{\PP^n}(d) \to\cdots
$$
together with the facts that  $H^0(\sO_{\PP^n}(d)$ is the vector space of degree $d$ forms on $\PP^n$ and 
 $H^1(\sO_{\PP^n}(d) = 0$ for $n>1$ and $d\geq 0$, we see that
 $$
 \dim (I_C)_d) = h^0(\sI_{C/\PP^n}(d)) = h^0(\sO_{\PP^n}(d) - h^0(\sO_C(d) + h^1(\sI_{C/\PP^n}(d)).
$$
If $d$ times the degree of $C$ is $>2g-2$, the the Riemann-Roch Theorem tell us the value of
 $h^0(\sO_C(d)$, and thus a lower bound for $ \dim (I_C)_d)$, but if, in addition, the map
 $H^0(\sO_{\PP^n}(d) \to H^0(\sO_C(d)$ is surjective (or equivalently $h^1(\sI_{C/\PP^n}(d))=0$),
 then we get $ \dim (I_C)_d)$ exactly.
 
 
\begin{definition} An embedding of a smooth curve
$C\subset \PP^n$ is said to be \emph{projectively normal} if all the maps $H^0(\sO_{\PP^n}(d) \to H^0(\sO_C(d)$ are surjective,
or equivalently $h^1(\sI_{C/\PP^n}(d))=0$ for all $d\geq 0$.
\end{definition}
 
By Serre's Criterion of Normality  a 1-dimensional scheme  $C\subset \PP^n$
is a smooth, projectively normal curve iff the homogeneous coordinate ring of $C$ is a normal ring, whence the terminology in
this definition. The criterion has two parts, which are interesting separately: A local or graded ring is normal if it is nonsingular in
codimension 1; and
locally of depth $\geq 2$ in codimension $\geq 2$. In the case of the homogeneous coordinate ring of a curve, 
the first condition means that the curve is nonsingular and the second means that $h^1(\sI_{C/\PP^n}(d))=0$ for all $d$.
In particular, for any very ample invertible sheaf $\sL$ on a smooth curve $C$, the ring 
$$
R(C, \sL) := \oplus_{m=0}^\infty H^0(\sL^m)
$$
is normal. See for example \cite[Theorem ***]{Eisenbud1995}.

\begin{example}
For $d\geq 1$ the rational normal curve $C\subset \PP^d$ of degree $d$ is projectively normal. 
We have $H^0(\sO_{\PP^1}(md))  = \CC[s,t]_{md}$. Th natural natural map
$$
\CC[x_0,\dots,x_d] \to\bigoplus_{m = 0}^\infty H^0(\sO_{\PP^1}(md)));\quad x_0,\dots, x_n \mapsto s^n,s^n-1t,\dots, t^n
$$ 
is surjective since every monomial of degree $md$ can be written as a monomial in  
the elements $s^n,s^n-1t,\dots, t^n$, and the ring is normal (Exercise~\ref{normality of RNC}).
\end{example}

\begin{exercise}\label{normality of RNC}
 Show that $\CC[s^n,s^n-1t,\dots, t^n]$ is normal (ie, integrally closed) by noting that its integral closure must be
 contained in $\CC[s,t]$ and then showing that if $f$ is any polynomial
 in the integral closure then the homogeneous components of $f$ are also in the integral closure.
\end{exercise}


We will now show that the canonical image of a smooth curve is always projectively normal. When the curve is hyperelliptic,
the canonical image is the rational normal curve, treated above, so we may assume that the curve $C$ is embedded
by $|K_C|$ as a smooth curve in $\PP^{g-1}$.


To do this we will make use of an auxiliary construction, a \emph{simple $(g-4)$-secant $(g-3)$-plane.}

\begin{lemma}
 If $C\subset \PP^n$ is a reduced, irreducible, nondegenerate curve, and $m\leq n-2$, then the linear span $L := \overline{p_1,\dots, p_m}$
 of $m$ general points of $C$ is a simple $m$-secant; that is, a plane of dimension $m-1$ such that
 $C\cap L = \{p_1,\dots,p_m\}$ scheme-theoretically.
 \end{lemma}
 
 
\begin{proof}
The plane $L$ is contained in a hyperplane $H$, and since the points are general, we may take this to be a general hyperplane. By Bertini's Theorem, $C\cap H$ is reduced, so $C\cap L$ is also reduced.
 If $C\cap L$ had length $>m$, then by Theorem~\ref{uniform position??}\fix{in ch 8-BrillNoether} every set of $m+1$ points of $C\cap H$ would be dependent,
 and the span of $C\cap H$ would thus have dimension $\leq m-1<n-1$, and we could choose a hyperplane section $C\cap H'$ with more points than $C\cap H$, which is absurd.
\end{proof}

The following was proven by Max Noether, Emmy's father:

\begin{theorem}[Max Noether]\label{canonical curves are ACM}
A canonical curve in $\PP^{g-1}$ has degree $2g-2$ and arithmetic genus $g$. If the curve has a simple
$g-2$ secant, then it is arithmetically Cohen-Macaulay; that is,
$\HH^{1}(\sI_{C/\PP^{g-1}}(m)) = 0$ for all $m\in \ZZ$.
\end{theorem}
 
For a canonically embedded irreducible curve the simple $g-3$-dimensional $g-2$ secant planes $\Lambda$  correspond to base-point-free pencils of degree $g = 2g-2 -(g-2)$: Given $\Lambda$, the linear series of hyperplanes containing $\Lambda$ intersects $C$ in $\Lambda$ plus the fibers of this pencil.Conversely, given such a pencil, the plane is the span of the complement of a general  member $P$ of the pencil in  $C\cap \overline P$, where $\overline P$ is the hyperplane that is the linear span of $P$.
 \fix{I worry about the converse; why shouldn't the base locus of $K-g^{1}_{g}$ have a multiple point, or even contain a singular point?}  
  
\begin{proof} The Hilbert polynomial $\chi_{C}(t) = h^{0}\sO(t)-\h^{1}\sO(t)$ of $C$ has degree equal to
$\dim C = 1$, so it is determined by two values.

We begin by showing that $\sO(-m)$ has no global sections for $m>0$.
If $D$ is a divisor equivalent to $m$ times the hyperplane section, we have an exact sequence
$$
0\to \HH^{0}(\sO_{C}(-m)) \to \HH^{0}(\sO_{C}) \to \HH^{0}(\sO_{D}) \to \cdots.
$$
By hypothesis, the vector space $\HH^{0}\sO_{C}$ is spanned by the constant functions, and these
restrict non-trivially to $\sO_{D}$, and $\HH^{0}(\sO_{C}(-m)) = 0$ as claimed.

Using the Riemann-Roch Theorem we can now compute the Hilbert function $\chi_{C}(m)$:
We have 
\begin{align*}
 \chi_{C}(0) &= h^{0}(\sO_{C}) - h^{1}(\sO_{C}) = h^{0}(\sO_{C}) - h^{0}(\omega_{C}) = 1-g.\\
\chi_{C}(1) &= h^{0}(\sO_{C}(1)) - h^{1}(\sO_{C}(1)^{*}\otimes \omega_{C}) = h^{0}(\omega_{C}) - h^{0}(\sO_{C}) = g-1.
\end{align*}
and we deduce
$\chi_{C}(m) = (2g-2)m -g+1$, whence we see that the degree of $C$ is $2g-2$ and $\p(C) = g$ as claimed.

To show that
$C$ is arithmetically Cohen-Macaulay we use the sequence
$$
\cdots \to \HH^{0}(\sO_{\PP^{n}}(m)) \to \HH^{0}(\sO_{C}(m))
\to \HH^{1}(\sI_{C}(m))\to \HH^{1}(\sO_{\PP^{n}}(m)) \to\cdots .
$$
Since $\HH^{0}(\sO_{\PP^{n}}(m)) = 0$, it
is enough to show that the natural map 
$$
\HH^{0}(\sO_{\PP^{n}}(m)) \to \HH^{0}(\sO_{C}(m))
$$
 is surjective for all $m\in \ZZ$. For $m=0,1$ this is immediate from the hypothesis.

For $m <0$ we must show $\HH^{0}(\sO_{C}(m))=0.$ 
If $D$ is a divisor equivalent to $-m$ times the hyperplane section, we have an exact sequence
$$
0\to \HH^{0}(\sO_{C}(m)) 
\to \HH^{0}(\sO_{C}) 
\to \HH^{0}(\sO_{D}) \to \cdots.
$$
By hypothesis, the vector space $\HH^{0}\sO_{C}$ is spanned by the constant functions, and these
restrict non-trivially to $\sO_{D}$, so the kernel, $\HH^{0}(\sO_{C}(m))$, is 0 as claimed. 

To prove surjectivity for $m\geq 2$ we use the remaining hypothesis, the existence of
a simple $g-3$-dimensional $g-2$ secant plane $\Lambda$  and an idea sometimes called the \emph{base-point-free pencil trick}. Let $p_{0},\dots p_{g-3}$ be the points in which $\Lambda$ meets $C$.  Since the
$p_{i}$ are linearly independent by hypothesis, we may choose homogeneous coordinates $x_{i} \in \HH^{0}(\sO_{C}(1))$ so that
$x_{i}(p_{j}) \neq 0$ if and only if $i = j$. It follows that the sections
$x_{i}^{m}$ of $\sO_{C}(m)$ span $\HH^{0}(\sO_{C}(m)|_{\{p_{0}, \dots, p_{g-3}\}}$. Let 
$V\subset \HH^{0}(\sO_{C}(1))$ be the two-dimensional subspace of linear forms vanishing on
$\Lambda$, and thus on the $p_{i}$. 

For $m\geq 2$ there are maps of vector spaces
$$
\wedge^{2} V\otimes \HH^{0}(\sO_{C}(m-2)) \to V\otimes \HH^{0}(\sO_{C}(m-1)) 
\to \HH^{0}(\sO_{C}(m))
$$
where the right hand map is multiplication and the left hand map sends
$s_{1}\wedge s_{2}\otimes \sigma$ to $s_{1}\sigma-s_{2}\sigma$ for any local section $\sigma$.
The sequence is exact because the sections $s_{1},s_{2}$ that span $V$ never vanish simultaneously except on the $p_{i}$, and has image  consisting of sections that vanish on the points $p_{i}$

\end{proof}

\begin{corollary}\label{canonical hilbert function}
If $C\subset \PP^{g-1}$ is a canonical curve with a simple $g-3$-secant, then the Hilbert function of the homogeneous coordinate ring $S_{C}$ of  $C$ depends only on $g$, and is given by:
$$
\dim({S_{C}})_{d} = h^{0}(\cO_{C}(d)) = 
\begin{cases}
 0 &\mbox {if } d<0\\
 1 & \mbox {if }  d=0\\
 g & \mbox {if }  d=1\\
 (2n-1)g+1 & \mbox {if }  d>1\\
\end{cases}
$$
\end{corollary}
\begin{proof}
By Theorem~\ref{canonical curves are ACM} implies, in particular, that the homogeneous coordinate ring of $C$ can be identified with 
$\oplus_{n\in \ZZ}\HH^0\sO_C(n)$.  
\end{proof}

 \section{A bit about surfaces}
 We will often analyze curves  on a smooth surface; here are a few results that will be useful. We refer to~\cite[Chapter V]{Hartshorne1977}
 and~\cite[Chapter I]{B} for proofs.
 
 We suppose for this section that $X$ is a smooth projective surface.
 We define $\Pic(X)$ to be the group whose elements are invertible sheaves on $X$.
When two divisors $D,E$ on $X$ meet transeversally we define $D\cdot E$ to be the number of points in which they meet. If they have not common
components, we can still define a non-negative intersection multiplicity $m_X(D,E,p)$ at each point $p$, and then set
$$
D\cdot E = \sum_p m_X(D,E,p).
$$
Over the complex numbers each codimension 1 subvariety $D$ of $X$ has a fundamental class
$[D]\in H^2(X, \ZZ)$ and the product is the cup product with values in $H^4(X,\ZZ)$, which is $\ZZ$ since $X$ is compact and orientable. Thus
there is an extension of the product to the cases where $D,E$ have common components. From a geometric point of view, we may choose an
appropriate $C^\infty$
normal vector field along $E$ and define $E'$ by ``pushing'' $E$ out slightly along the direction of this field, until $D$ and $E'$ are transverse,
and the intersections can be computed in the usual way. It should be noted that such intersection numbers can be either positive or negative,
since they depend on the relative orientations of $D$ and $E'$ at the intersection points; this is in contrast to the case when $D$ and $E$
themselves meet transversely; in this case the fact that the complex numbers are canonically oriented makes the intersection number non-negative.

The intersection product can also be defined algebraically, over any field: Setting $\sL := \sO_X(C)$ and
$\sM := \sO_X(D)$ to simplify the notation, we set 
$$
D\cdot E = \chi(\sO_X)-\chi(\sL^{-1}) -\chi(\sM^{-1}) -\chi(\sL^{-1}\sM^{-1}) 
$$
\begin{theorem} The pairing $(D,E) \mapsto (D.E)$ is the unique bilinear map
$\Pic(X) \times \Pic(X) \to \ZZ$ extending the case of intersections of two transverse curves on $X$. 
\end{theorem}

A frequent use of the the intersection pairing is to compute the (arithmetic) genus of a curve on a surface,
a result called the adjunction formula.

\begin{theorem}[Adjunction Formula]\label{adjuction formula}
If $C$ is a curve lying on a smooth surface $X$ then 
$$
\omega_C = \omega_X \otimes \sO_X(C) \otimes \sO_C.
$$
In particular 
$$
2p_a(C) -2 = (K_X+C).C.
 $$
\end{theorem}

It is useful to know what happens under mappings of surfaces, particularly the case of the mapping
corresponding to blowing up a point.

\begin{theorem}
If $\pi: X \to Y$ is a birational map of smooth surfaces, then the pullback map on divisors
preserves the intersection pairing. If $X$ is the blowup of $Y$ at a point $p$, with exceptional
divisor $E = \pi^{-1}(p)$ then:

\begin{enumerate}
 \item $\Pic X =\pi^*(\Pic Y) \oplus \ZZ E$.
\item The canonical class on $X$ is given by $K_X = \pi^*(K_Y)+E$.
 \item The intersection pairing on $\Pic X$ is given by
 
\begin{itemize}
\item $\pi^*(D).\pi^*(D') = D.D'$ for all $D,D'\in \Pic Y$.
\item $\pi^*(D).E = 0$ for all $D\in \Pic Y$.
 \item $E.E = -1$.
 \item $K_X.E = -1$.
 \item If $C$ is a curve that has an $m$-fold point at $p$ then $\pi^{-1}(C)$ contains $E$ with multiplicity $m$.
 \end{itemize}
\end{enumerate}
\end{theorem}

Blowups occur frequently in the theory of surfaces, and are easy to characterize:
\begin{theorem}
If $E\subset X$ is a curve on a smooth projective surface $X$ and
 that $E^2 = E.K_X = -1$ then $E$ can be ``blown down'' in the sense that
 $X$ is the blowup of a smooth surface $Y$ at a point $p\in Y$, and $E$ is the exceptional divisor.
\end{theorem}

\input footer.tex