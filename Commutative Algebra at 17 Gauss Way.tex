\documentclass[11pt, oneside]{article}   	% use "amsart" instead of "article" for AMSLaTeX format
\usepackage{geometry}                		% See geometry.pdf to learn the layout options. There are lots.
\geometry{letterpaper}  
\usepackage{amsmath}
                 		% ... or a4paper or a5paper or ... 
%\geometry{landscape}                		% Activate for rotated page geometry
%\usepackage[parfill]{parskip}    		% Activate to begin paragraphs with an empty line rather than an indent
\usepackage{graphicx}				% Use pdf, png, jpg, or eps§ with pdflatex; use eps in DVI mode
								% TeX will automatically convert eps --> pdf in pdflatex		
\usepackage{amssymb}

%SetFonts

%SetFonts


\title{Commutative Algebra at 17 Gauss Way}
\author{David Eisenbud}
%\date{}							% Activate to display a given date or no date
%goal: 1500 words, two pictures
\begin{document}
\maketitle
\section{MSRI/SLMath and Commutative Algebra}

The current program on Commutative Algebra at SLMath follows three others; a three-week microprogram in 1987 and year-long programs in 2002-03 and 2012-13. Many of the senior participants in the current program were graduate students or postdocs in the earlier ones, and the cumulative effect on the field has been very great.

My first experience at MSRI was during my sabbatical in 1986-87---a wonderful year that led to my application for the job of Director, 20 years later. An unforgettable highlight for me was the microprogram at the end of the year, led by Mel Hochster, Craig Huneke (who is a current member) and Judith Sally (whose death last month at 86 saddened us all.)

There are perhaps a dozen intertwined themes being pursued in the the current semester, from singularities and representation theory in positive characteristic to free resolutions, finite and infinite. I could make a very dull article by writing a paragraph on each. 

Instead I'll focus on one topic with a long history: residual intersections: roughly: what's left over when you subtract one algebraic variety from another. The question arose in at least three quite independent areas over the course of the late 19th and early 20th century, and I'll first sketch these origins, sticking for simplicity with projective varieties over the complex numbers.

\section{Linkage of curves in 3-space} The easiest of the three source problems to understand is historically the latest: How can one classify curves in 3-space? A smooth curve in our sense is the same thing as a Riemann surface, and there are just two topological invariants that characterize such a curve in projective space: its degree and genus. So the question becomes: what are the possible degrees and genera
of a smooth curve in (complex projective) 3-space?

A  smooth curve in 2-space is defined by a single equation and if the degree of the equation is $d$ then the curve is a Riemann surface of genus $d-1\choose 2$.
The analogue of degree for a curve $C$ in 3-space is the number of points in which it meets a general plane, but already for curves of degree 4, there are two possible genera (0 and 1), so there is not such a 
sharp connection between degree and genus 

Suppose first that the ideal of homogeneous polynomials vanishing on $C$ is generated by just 2---the smallest possible number, and that their degrees are $d$ and $e$. In this case the degree of the curve
is $de$ and the genus of the curve is $1+\frac{1}{2}de(d+e-4)$. But there are also a curves of any degree having genus 0, such as those parameterized by $t \mapsto (t, t^{d-1}, t^{d})$. A natural guess (after a few more examples) might be that a curve could have degree and genus $(d,g)$ for any
$0\leq g \leq 1+\frac{1}{2}de(d+e-4)$---but this was known to be false!

The landscape of curves in 3-space was explored by 
Max Noether (Emmy's father, whom van de Waerden called the ``father of algebraic geometry'') and Georges-Henri Halphen, who jointly received in the Steiner prize of the Prussian Academy of Sciences in 1880 for their work. A primary technique is what is now called \emph{linkage} of curves (also known by its French term liaison):  Given a curve $C$, find two equations from its defining ideal, and look at the ``curve'' $D$ that is defined by these two; in general it will turn out that $D = C\cup C'$, where $C'$ is another smooth curve, said to be directly linked to $C$.

The simplest nontrivial case is illustrated by the picture in
Figure~\ref{cubicAndLine}. In general, if the equations of the two surfaces in the linkage are $s$ and $t$ then it turns out that
the degrees and genera of $C$ and $C'$ are related by the simple formulas
\begin{align*}
&\deg C+\deg C' = st\\
&g(C) - g({C'}) = \frac{s+t-4}{2}(\deg C-\deg {C'}).
\end{align*}

The theory of linkage has a long tail: curves linked (in many steps) to one another are said to be in the same ``linkage equivalence class''. Curves in 3-space in the linkage class of a line were classified by in the 1940s, and a remarkably simple complete algebraic invariant for linkage of curves in 3-space was discovered by Robin Hartshorne and Prabhakar Rao in ****. 

One of the emphasis areas of the current SLMath program is centered around a dramatic new understanding of linkage of curves in 4-space and beyond by Jerzy Weyman and his collaborators
Lorenzo Guerrieri and Xianglong Ni, (the latter is a Berkeley graduate student participating in this semester's program).




\begin{figure}\label{cubicAndLine}
\centerline {\includegraphics[height=3in]{"main/Fig15-1-TwistAndShout"}}
 \caption{A quadratic cone (red) intersecting a smooth quadratic surves (yellow) in the union of a vertical line and a twisted cubic (credit: Herwig Hauser)}
\end{figure}

%\begin{figure}\label{8 conics}
%\centerline {\includegraphics[height=3in]{""}}
% \caption{****}
%\end{figure}




\section{History I: The Steiner problem (1848 and Chasles' 1864 solution. Different approaches...}
\section{History II: The Riemann-Roch Theorem: Brill, Noether, Macaulay}
\section{History III: The French Academy Prize: Noether and Halphen}
\section{The modern period}
\subsection{Fulton-MacPherson 1978}
\subsection{Artin and Nagata}
\subsection{Peskine and Szpiro}
\subsection{Huneke and Ulrich}




\end{document}  