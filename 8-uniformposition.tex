
%header and footer for separate chapter files

\ifx\whole\undefined
\documentclass[12pt, leqno]{book}
\usepackage{graphicx}
\input style-for-curves.sty
\usepackage{hyperref}
\usepackage{showkeys} %This shows the labels.
%\usepackage{SLAG,msribib,local}
%\usepackage{amsmath,amscd,amsthm,amssymb,amsxtra,latexsym,epsfig,epic,graphics}
%\usepackage[matrix,arrow,curve]{xy}
%\usepackage{graphicx}
%\usepackage{diagrams}
%
%%\usepackage{amsrefs}
%%%%%%%%%%%%%%%%%%%%%%%%%%%%%%%%%%%%%%%%%%
%%\textwidth16cm
%%\textheight20cm
%%\topmargin-2cm
%\oddsidemargin.8cm
%\evensidemargin1cm
%
%%%%%%Definitions
%\input preamble.tex
%\input style-for-curves.sty
%\def\TU{{\bf U}}
%\def\AA{{\mathbb A}}
%\def\BB{{\mathbb B}}
%\def\CC{{\mathbb C}}
%\def\QQ{{\mathbb Q}}
%\def\RR{{\mathbb R}}
%\def\facet{{\bf facet}}
%\def\image{{\rm image}}
%\def\cE{{\cal E}}
%\def\cF{{\cal F}}
%\def\cG{{\cal G}}
%\def\cH{{\cal H}}
%\def\cHom{{{\cal H}om}}
%\def\h{{\rm h}}
% \def\bs{{Boij-S\"oderberg{} }}
%
%\makeatletter
%\def\Ddots{\mathinner{\mkern1mu\raise\p@
%\vbox{\kern7\p@\hbox{.}}\mkern2mu
%\raise4\p@\hbox{.}\mkern2mu\raise7\p@\hbox{.}\mkern1mu}}
%\makeatother

%%
%\pagestyle{myheadings}

%\input style-for-curves.tex
%\documentclass{cambridge7A}
%\usepackage{hatcher_revised} 
%\usepackage{3264}
   
\errorcontextlines=1000
%\usepackage{makeidx}
\let\see\relax
\usepackage{makeidx}
\makeindex
% \index{word} in the doc; \index{variety!algebraic} gives variety, algebraic
% PUT a % after each \index{***}

\overfullrule=5pt
\catcode`\@\active
\def@{\mskip1.5mu} %produce a small space in math with an @

\title{Personalities of Curves}
\author{\copyright David Eisenbud and Joe Harris}
%%\includeonly{%
%0-intro,01-ChowRingDogma,02-FirstExamples,03-Grassmannians,04-GeneralGrassmannians
%,05-VectorBundlesAndChernClasses,06-LinesOnHypersurfaces,07-SingularElementsOfLinearSeries,
%08-ParameterSpaces,
%bib
%}

\date{\today}
%%\date{}
%\title{Curves}
%%{\normalsize ***Preliminary Version***}} 
%\author{David Eisenbud and Joe Harris }
%
%\begin{document}

\begin{document}
\maketitle

\pagenumbering{roman}
\setcounter{page}{5}
%\begin{5}
%\end{5}
\pagenumbering{arabic}
\tableofcontents
\fi



\chapter{Hyperplane sections of a curve}\label{uniform position}

\section{Linearly general position}
In this section we will allow our algebraically closed ground field to have arbitrary characteristic. 

One way to study a curve in $\PP^n$ is to use properties of its hyperplane sections. In this section we show that any $n$ points of a general hyperplane section of a reduced irreducible curve in characteristic 0, or a nonsingular curve in any characteristic, are linearly independent (but see Exercise~\ref{strange curves} for
interesting singular examples in positive characteristic). 

In the next sections we give a series of applications of this result. Then, in Section~\ref{uniform position section} we return to the characteristic 0 case to prove the much stronger \emph{uniform position lemma}. Finally we give some applications
of this stronger result.

Recall that an irreducible reduced curve
$C\subset \PP^n$, with $n\geq 2$ over an algebraically closed field $k$ is called \emph{strange} if
all the tangent lines at smooth points of $C$ meet in a point $p\in \PP^n$. In this case the projection of $C$ from $p$ is inseparable, and thus there are no strange curves in characteristic 0. Pierre Samuel showed that the only smooth curve that is strange (in any characteristic) is the conic in characteristic 2; a proof is given in \cite[Theorem IV.3.9]{Hartshorne1977} 

If $C\subset \PP^n$ is a reduced curve then, by Bertini's theorem, a general hyperplane
section of $C$ consists of smooth points. 

\begin{proposition} \label{basic linear independence}(\cite[Lemma 1.1]{Rathmann})
If $C\subset \PP^n$ is a nondegenerate, irreducible, reduced curve that is not strange, and
and $H$ is a general hyperplane, then any $n$ points of $H\cap C$ are
linearly independent.
\end{proposition}

The following result is a step in the proof that is useful in other contexts too. 
A \emph{multiple secant} or \emph{multisecant} line to a curve $C\subset \PP^n$ is a line whose
scheme-theoretic intersection with $C$ has length $\geq 3$.  A \emph{stationary
secant} is a line joining two smooth points $r,s\in C$ such that the tangent lines
to $C$ at $r$ and $s$ meet. For example, if $C$ is planar then every line in the plane
spanned by $C$ is a stationary secant; if $\deg C\geq 3$ then every
line in the plane is a multisecant as well. 

\begin{lemma}\label{incident tangents}
If $C\subset \PP^n$ is a non-strange, non-planar reduced irreducible curve, then
a general secant line is neither multiple nor stationary.
\end{lemma}

\begin{proof}
 By projection we may reduce to the case $n=3$. Consider a projection $\pi_p$
from a general point $p\in C\subset \PP^3$, with image $C'\subset \PP^2$. If the projection were inseparable, then
every tangent line to $C$ at a smooth point would pass through $p$, so $C$ would be strange.
Thus if $q\in C'$ is a general point, then $\pi_p$ is unramified at $q$,
and the secant line corresponding to $q$ meets $C$ in distinct smooth points $r,s$
in addition to $p$.

Arguing by contradiction, suppose first that every secant is a multi-secant.
Since the tangent lines at $r,s$ both map to the tangent line at $q$,
they span a plane, and thus intersect
in a point. It follows that for an open set of pairs of points
$(r,s)\in C\times C$, the tangent lines at $r$ and $s$ intersect. 

Any three lines in $\PP^3$ that meet each other pairwise must all pass through
a single point or all lie in a plane. Since each pair of tangent lines at general points meet, they must all pass through
a single point, the intersection of any two, or lie in a single plane, the span of any two. Thus if all pairs of tangent lines to $C$ at smooth points intersect, then all either
pass through a common point or all lie in a common plane. In the first case the curve would be strange,
while in the second case the curve would be planar, contradicting our hypotheses. \end{proof}

The family of secant lines is naturally an algebraic variety, and the stationary and multiple secant lines are subvarieties: let $C\subset \PP^n$ be a reduced irreducible curve. Since $C$ has only finitely many singular points, we may ignore them, and let $C_0$ be its smooth locus. Consider first the ``universal secant''
$$
I := \{((p,q), r)\in C_0^{(2)}\times \PP^n \mid r\in \overline{p,q}\}
$$
where $C_0^{(2)}$ denotes the symmetric square of $C_0$, and $\overline{p,q}\cong \PP^1$ denotes the linear span of $p,q$.
The fibers of the projection $I \to C_0^{(2)}$ are all isomorphic to $\PP^1$,
so $I$ is smooth and irreducible of dimension 3. Stationary secants
make sense only for secants at pairs of distinct points, so we 
also consider $C_0^{(2)}\setminus \Delta$, where $\Delta$ denotes the diagonal.

\begin{proposition}
With notation above, subset of $C_0^{(2)}$ corresponding to 
multiple secants  is closed, and the subset of 
$C_0^{(2)}\setminus \Delta$ corresponding to stationary secants is also closed. 
\end{proposition}

\begin{proof}
To deal with multisecants, we consider the projection 
$$
\rho: I\cap C_0^{(2)}\times C \to C_0^{(2)}.
$$
The map is projective, and the preimage of a point $(p,q)\in C_0^{(2)}$ is the scheme-theoretic intersection of $\overline{p,q}$ with
$C$, which is finite, so the map is finite. The degree of the fibers is therefore semicontinuous (it is the local number of generators of the pushforward of
the structure sheaf of the intersection), and thus the set of multisecants, which is the set of points of $C_0^{(2)}$ where the length of the 
fiber is $>2$, which is closed.

To treat the case of stationary tangents, let  
$$
I' := \{((p,q), r)\in (C_0^{(2)}\setminus \Delta)\times \PP^n \mid r \in T_p(C) \cap T_q(C)\}.
$$
which is clearly a closed subset of $((p,q), r)\in (C_0^{(2)}\setminus \Delta)\times \PP^n.$
Since the projection to $(C_0^{(2)}\setminus \Delta)$ is a projective map,
the image, which corresponds to the set of stationary tangents, is closed
in $(C_0^{(2)}\setminus \Delta)$
\end{proof}

\begin{proof}[Proof of Proposition~\ref{basic linear independence}]
The set of hyperplanes in $\PP^n$ whose intersection with $C$ is in linearly general position is open in $\PP^{n*}$ so it suffices to show that this set is nonempty.

Choose general points $q,r\in C$, and let $V\subset \PP^n$ be the projective space spanned by the tangent lines at $q,r$. Since $C$ is not strange, $V$ is 3 dimensional.
Let $p\in C$ be a general point; since $C$ is nondegenerate and $n\geq 4$, the point $p$ is not in $V$. 

Since not every secant is a multisecant and $p\in C$ is general,
projection $\pi_p$ from $p$ is birational onto its
 image $C'\subset \PP^{n-1}$. Because $p\notin V$, the restriction of $\pi_p$ to $V$
 is an isomorphism, so the image of the tangent lines at $q$ and $r$, which are
 the tangent lines to $C'$ at $\pi_p(q)$ and $\pi_p(r)$, do not meet. Thus
 $C'$ is not strange. 

Let $d= \deg C$. By induction there is a hyperplane $H'\subset \PP^{n-1}$ whose
intersection with $C'$ consists of $d-1$ points in linearly general position. In this case the span of $H' \cup \{p\}$ is a hyperplane in
$\PP^n$ whose points project to the points of $H'\cap C'$, together with $p$ and, because $p$ is not in the span of $C'$, this is a set in linearly general position.
\end{proof}


\section{Existence of good projections}\label{projection section}\label{good projections}

We can use Proposition~\ref{basic linear independence} to show that every smooth curve $C$ is birational to a nodal plane curve $C_0 \subset \PP^2$, in many ways.

\begin{proposition}\label{nodal projection}
If $C \subset \PP^n$ is a smooth nondegenerate curve in projective space, let $\Lambda \cong \PP^{k} \subset \PP^n$ be a general $k$-plane, and let
$\pi_\Lambda$ be the projection from $\Lambda$, restricted to $C$. If $n\geq 4$ and $k=n-4$,then
$\pi_\Lambda: C \to \PP^3$ defines an isomorphism of $C$ onto its image, while if $n\geq 3$ then $\pi_\Lambda$ is birational onto its image, which is a curve with only ordinary nodes.
\end{proposition}

\begin{proof} Recall that the secant variety of $C$ consists of the union of the lines $\overline{q,r}$ joining pairs of distinct points $q,r \in C$, plus the tangent lines ${\mathbb T}_q(C)$; altogether, these lines form family, parametrized by the symmetric square $C^{(2)}$of $C$. More precisely every subscheme $\lambda$ of
length 2 in $\PP^n$ spans a line $\overline \lambda$. The incidence variety
$$
I:=\{(\lambda, p)\mid \lambda\in C^{2)} \hbox{ is a divisor of degree 2 on }C,\ p\in \overline \lambda\subset\PP^n\}
$$
projects to $C^{(2)}$ with 1-dimensional fibers isomorphic to $\PP^1$, and thus
is irreducible of dimension 3. Its image in $\PP^n$ under the second projection
is the secant variety of $C$, which is thus irreducible of dimension $\leq 3$.
It follows that a general
$n-4$-plane does not meet the secant variety, and the first statement of the Proposition follows.

If $n>3$ then by first projecting from a general $n-4$-plane inside $\Lambda$ we may reduce to the case $n=3$, and assume that $\Lambda$ is a general point of $\PP^3$. By a variant of the argument above, the union 
of the tangent lines to $C$ is a surface, and thus does not contain $\Lambda$.
It follows that $\pi_\Lambda$ is locally an analytic isomorphism.

To show that the fibers of $\pi_\Lambda$ are subschemes of length at most 2,
we need to show that $\Lambda$ does not lie on any multisecant line. 

By Proposition~\ref{basic linear independence} the family of multisecant lines to $C$ is a proper subscheme of the irreducible two-dimensional family of secant lines, so the union of the trisecant lines is at most 2 dimensional, and we see that the fibers of $\pi_\Lambda$ all have degree $\leq 2$. Furthermore, the general fiber of the projection
from the incidence correspondence $I$ to $\PP^3$ is empty or finite, so only a finite number of secant lines will contain $\Lambda$, and we see that $\pi_\Lambda$ is birational. 

We have shown that the map $\pi_\Lambda$ is an immersion, and at most two-to-one everywhere; thus the image curve $C_0 \subset \PP^2$ will have at most double points, and an analytic neighborhood of each double point will consist of two smooth branches. To complete the proof of Proposition~\ref{nodal projection} we have to show that those two branches have distinct tangent lines; that is, that
if $q, r \in C$ are any two points collinear with $\Lambda$, then the images of the tangent lines ${\mathbb T}_q(C)$ and ${\mathbb T}_r(C)$ in $\PP^2$ are distinct. But if  $\pi_p({\mathbb T}_q(C)) = \pi_p({\mathbb T}_r(C))$ then  ${\mathbb T}_q(C)$ and ${\mathbb T}_r(C)$ lie in a plane, and thus intersect.

Since the family of all secant lines is irreducible of dimension 2,  it will suffice to show that not every secant line to $C$ is a stationary secant or, equivalently, that not every pair of tangent lines to $C$ meet. We saw in Lemma~\ref{incident tangents} that in the contrary case the curve $C$ would be either strange or planar, a contradiction.
\end{proof}


\section{The case of equality in Martens' theorem}

Proposition~\ref{basic linear independence}  allows us to analyze the case of equality in Martens' theorem bounding the dimension of the variety $W^r_d(C)$ parametrizing divisor classes of degree $d$ on a curve $C$ with $r(D) \geq r$
(Theorem~\ref{Martens' inequality}.
To start, recall the statement:
å
\begin{theorem}[Martens' theorem]\label{full Martens}
If $C$ is any smooth projective curve of genus $g$, then for any $r>d-g$ we have
$$
\dim W^r_d(C) \leq d-2r.
$$
Equality holds iff either $C$ is hyperelliptic
or $d=r=0$ or $d=2g-2, r=g-1$; in either of the last two cases $W^r_d$ is
a single point.\end{theorem}

Note that the inequality $\dim W^r_d(C) \leq d-2r$ is equivalent to the inequality $\dim C^r_d \; \leq \; d-r$, which is what we actually showed in Chapter~\ref{new Jacobians chapter}. This in turn followed by combining the geometric form of Riemann-Roch with an elementary bound on the dimension of the variety of secant planes to a curve in projective space:

\begin{lemma}[elementary secant plane lemma]
Let $C \subset \PP^n$ be a smooth, irreducible and nondegenerate curve. If we denote by $\Sigma \subset C_d$ the locus of effective divisors $D$ of degree $d$ on $C$ with $\dim \overline D \leq d-r-1$, then for any $d \leq n$ and $r > 0$,
$$
\dim \Sigma \leq d-r.
$$
\end{lemma}

Proposition~\ref{basic linear independence} allows us to do exactly one better: 

\begin{lemma}[strong secant plane lemma]\label{Strong secant plane lemma}
Let $C \subset \PP^n$ be a smooth, irreducible and nondegenerate curve. If we denote by $\Sigma^r_d \subset C_d$ the locus of effective divisors $D$ of degree $d$ on $C$ with $\dim \overline D \leq d-r-1$, then for any $d \leq n$ and $r > 0$,
$$
\dim \Sigma^r_d \leq d-r-1.
$$
\end{lemma}

\begin{proof}
Consider the incidence correspondence: 
$$
\Gamma := \left\{ (D, H) \in \Sigma^r_d\times {\PP^n}^* \mid \overline D \subset H \right\}.
$$
The curve being nondegenerate, the projection map $\Gamma \to  {\PP^n}^*$ is finite. But the fibers of $\Gamma$ over $\Sigma^r_d$ have dimension at least $n-d+r$; if we had $\dim \Sigma^r_d \geq d-r$, it would follow that $\dim \Gamma \geq n$, and hence that the projection map $\Gamma \to  {\PP^n}^*$ is dominant---contradicting Proposition~\ref{basic linear independence}.
\end{proof}

\begin{proof}[Proof of the case of equality in Martens' theorem]
 Now, if a curve $C$ is non-hyperelliptic, we can apply the strong secant plane lemma to the canonical curve. Except for the trivial cases $d=r=0$ and $d=2g-2, r=g-1$,
 we can apply Lemma~\ref{Strong secan plane lemma} to conclude that $\dim W^r_d(C) \leq d-2r-1$; it follows that if we have $\dim W^r_d(C) = d-2r$ the curve $C$ in question must be hyperelliptic.
\end{proof}

Using the case of equality in Martens' Theorem, we can analyze equality in 
Clifford's Theorem:

\begin{corollary}\label{equality in Clifford from Martens}
If $C$ is a smooth curve of genus $g$ and $D$ a divisor on $C$ of degree $\leq 2g-2$,
such that $\deg D = 2r(D)$, then either $D ~0$ or $D~K_C$ or $C$ is hyperelliptic.
\end{corollary}

\begin{proof}
If $C$ has a divisor $D$ with $\deg D =2 r(D)$ then by the Riemann-Roch theorem,  $\deg D  = 2(\deg D-g+h^1(D))$, 
so $\deg D = 2g-h^1(D)$. If also $\deg D\leq 2g-2$, then $h^1(D) \geq 1$
and thus $r(D) >\deg D-g$. Since $W^{r(D)}_{\deg D}$ contains $D$ it's dimension
is $\geq 0$, and we have a case of equality in Martens' theorem.
\end{proof}

\section{The $g+2$ theorem}\label{g+2 section}

Now that we have the strong form of Martens' theorem, we can prove the analogue of Theorem~\ref{g+3 theorem} for general linear series of degree $g+2$. This was stated in Section~\ref{g+3 section}, but we'll reproduce the statement here.

\begin{theorem}
Let $C$ be any smooth projective curve of genus $g$, and let $D$ be a general divisor of degree $g+2$ on $C$. 
\begin{enumerate}
\item If $C$ is non-hyperelliptic, the map $\phi_D : C \to \PP^2$ is birational onto its image $C_0$, and $C_0$ is a plane curve of degree $g+2$ with exactly $\binom{g}{2}$ nodes and no other singularities; and
\item If $C$ is hyperelliptic, the map $\phi_D : C \to \PP^2$ is birational onto its image $C_0$, and $C_0$ is a plane curve of degree $g+2$ with one ordinary $g$-fold point and no other singularities.
\end{enumerate}
\end{theorem}

\begin{proof}
Let's start by proving that $\phi_D$ is birational onto its image, which is true whether or not $C$ is hyperelliptic. We do this much as in the proof of Theorem~\ref{g+3 theorem}. Since $D$ is general, we know that $h^0(D) = 3$. To say that $\phi_D(p) = \phi_D(q)$ for some pair of points $p, q \in C$, accordingly, means exactly that $h^0(D-p-q) = 2$. This in turn means that $D-p-q = K-E$ for some effective divisor $E$ of degree $g-2$; in other words, to say that $\phi_D(p) = \phi_D(q)$ for some pair of points $p, q \in C$ means that $\mu(D)$ is in the image of the map
$$
\nu : C_2 \times C_{g-2} \to \Pic_{g+2}(C)
$$
sending $(p+q, E)$ to $K_C - E + p + q$. Now, $\nu$ is a map between projective varieties of the same dimension, so we might expect that it is surjective, and indeed the genus formula tells us that it must be: $\phi_D$ cannot be an embedding. But by the same token, the locus in $\Pic_{g+2}(C)$ over which the fibers of $\nu$ are positive-dimensional must be a proper subvariety; thus for general $D$ we conclude that \emph{there are only finitely many pairs $p, q \in C$ such that $\phi_D(p) = \phi_D(q)$}; in other words, $\phi_D$ is birational onto its image.

Let's now suppose that $C$ is non-hyperelliptic and $D$ is a general divisor of degree $g+2$ on $C$. To prove the theorem in this case, we have to show  three things: that the image $C_0 = \phi_D(C)$ does not have cusps; that it does not have triple points, and that it does not have tacnodes. (The fact that $|D|$ is base point free  will follow from the first of these arguments, as noted below.) We'll take these assertions in turn:

\begin{enumerate}


\item Cusps: to say that a point $p \in C$ maps to a cusp of $C_0$ (that is, the differential $d\phi_D$ is zero at $p$) amounts to saying that $h^0(D-2p) \geq 2$; that is, $D-2p$ is a $g^1_g$. But by Riemann-Roch, $W^1_g = K_C - W_{g-2}$; so to say $\phi_D$ has a cusp means that
$$
\mu(D) \in 2W_1 + K_C - W_{g-2},
$$
and since the locus on the right has dimension at most $g-1$, a general point of $J(C)$ will not lie in it. Note that this subsumes the fact that $|D|$ has no base points.

\item Triple points: to say that $C_0$ has a triple point means that for some divisor $E = p+q+r$ of degree 3, $h^0(D-E) \geq 1$; thus we must have 
$$
\mu(D) \in W_3 + W^1_{g-1}
$$
Now, to argue that this is not the case, we need to know that $\dim W^1_{g-1} \leq g-4$. Here we have to invoke Marten's theorem, which says that $\dim W^1_{g-1} \leq g-4$ if $C$ is non-hyperelliptic; given this, we conclude that $C_0$ has no triple points.

\item Tacnodes:  To say that a pair of points $p, q \in C$ map to a tacnode of $C_0$ means two things: that $h^0(D-p-q) \geq 2$; and that $h^0(D-2p-2q) \geq 1$. If this is the case, set $E = D - 2p - 2q$;  the condition $h^0(D-2p-2q) \geq 1$ is simply that $E$ is an effective divisor, and by the geometric Riemann-Roch the condition $h^0(D-p-q) \geq 2$ says that in terms of the canonical embedding of $C$, the secant line $\overline{p,q}$ meets the $g-3$-plane $\overline E$ spanned by $E$. Now, not every secant line to $C$ can meet a linear subspace $\Lambda \cong \PP^{g-3}$ of dimension $g-3$---the projection $\pi_\Lambda : C \to \PP^1$ would be constant---so we see that $\mu(D)$ would have to lie on the image of a proper subvariety of $C_{g-2} \times C_2$ under the map 
$$
(E, p+q) \mapsto E+2p+2q.
$$
By a dimension count it follows that $D$ cannot be a general divisor of degree $g+2$.

\end{enumerate}

Thus, in the non-hyperelliptic case, the image curve $C_0 = \phi_D(C)$ has only nodes as singularities; the fact that there are exactly $\binom{g}{2}$ of them follows from the genus formula.


This concludes the proof in the non-hyperelliptic case. Now suppose that $C$ is hyperelliptic, and let $|E|$ be the (unique) $g^1_2$ on $C$. If  $D$ is any divisor of degree $g+2$, the divisor $D - E$ will have degree $g$, and so be effective; thus we can write
$$
D ~ E + p_1 + \dots + p_g
$$
for some $g$-tuple of points $p_i$. If the divisor $D$ is general, the points $p_i$ will be general as well, and in particular distinct.

Now, the fact that
$$
h^0(D - p_1 - \dots - p_g) = h^0(E) = 2 = h^0(D) - 1
$$
says that $\phi_D$ maps all the points $p_i$ to the same point! The image curve $C_0$ thus has a point of multiplicity at least $g$, with at least $g$ branches; by the genus formula, we see that this is an ordinary $g$-fold point of $C_0$ and that $C_0$ can have no other singularities.

\end{proof}





\section{Castelnuovo's theorem}

Clifford's theorem gives a complete and sharp answer to the question, ``what linear series can exist on a curve of genus $g$?
But maybe that wasn't the question we meant to ask! After all, we're interested in describing curves in projective space as images of abstract curves $C$ under maps given by linear systems on $C$. Observing that the linear series that achieve equality in Clifford's theorem give maps to $\PP^r$ that are 2 to 1 onto a rational curve, we might hope that we would have a different---and more meaningful---answer if we  restrict our attention to linear series $\cD = (\cL,V)$ for which the associated map $\phi_\cD$ is at least  birational. 

A classical result of Castelnuovo gives a sharp bound, which is in some sense explained by Castelnuovo's classification of the curves that achieve it. For positive integers $d$ and $r$, we write
$$
 d = M(r-1) + 1 + \epsilon \quad \text{ with } \quad 0 \leq \epsilon \leq r-2
$$
and set
$$
\pi(d,r) = \frac{M(M-1)}{2}(r-1) + M\epsilon.
$$

\begin{theorem}[Castelnuovo's bound]\label{Castelnuovo's bound}
If $C \subset \PP^r$ is a reduced, irreducible, nondegenerate curve of degree $d$ and arithmetic genus $p_a$, then
$$
p_a \leq \pi(d,r).
$$
\end{theorem}

In our proof we will use characteristic 0, but in fact the result holds in all characteristics.

We will say that a curve achieving the bound is a \emph{Castelnuovo curve}. We will see in Chapter~\ref{} that the bound is sharp: for every $r$ and $d \geq r$, there exist curves with genus exactly $\pi(d,r)$. 

\begin{example}
For plane curves, $r=2, \ \epsilon = 0, \ M = d-1$ so we recover the necessary $p = \pi(d,2) =  {d-1\choose 2}$. Thus every reduced irreducible
plane curve is a Castelnuovo curve.
\end{example}

\begin{example}For a more typical situation, consider the case $r=3$. We have
$$
p_a \leq \pi(d,3) = \lfloor d^2/4 \rfloor-d+1\, .
$$
It is easy to check that curves on a smooth quadric surface of classes
$(d/2, d/2)$ or $((d-1)/2, (d+1)/2)$
achieve this bound, and a variant of this works for singular quadrics as well. The classification  explained in Chapter~\ref{ScrollsChapter}
shows that these are the only Castelnuovo curves in $\PP^3$.
\end{example}

\begin{fact}
A famous theorem of Gruson and Peskine~\ref{Gruson-Peskine} (see \ref{Hartshorne-report} for an exposition and also the cases of characteristic $>0$ and small fields) completes the picture of the possibilities for the degree $d$ and  genus $g$  of a smooth curve in $\PP^3$. First, if the curve does not lie on a plane or a quadric, then
$$
g\leq \pi_1(d,3) := \frac{d^2-3d}{6} +1
$$
so there are  sometimes gaps in the possible genera for a given degree: for example  $\pi_1(9,3) = 10<\pi(9,3) =12$ but there is
no curve of degree 9 and genus 11: an irreducible curve of class $(a,b)$ on a smooth quadric has arithmetic genus $11 = (a-1)(b-1)$
if and only if $\{a,b\} = \{2,12\}$, so such a curve must have degree 14, and a similar computation holds for a curve on a quadratic cone.

On the other hand, there are smooth curves in $\PP^3$ of every genus and degree realizing $g\leq \pi_1(d)$; and these can be realized as curves
on cubic or quartic surfaces.

The case of $\PP^n$ for $n\geq 3$ remains open. 
\end{fact}
To prove Theorem~\ref{Castelnuovo's bound}, we will give lower bounds for the dimensions of the linear series  cut on a curve $C$ by hypersurfaces of degree $m$. For large values of $m$ the line bundle $\cO_C(m)$ is non-special and $H^1(\sI_{C/\PP^r}(m)) = 0$, so a lower bound on the dimension of its space of sections translates, via the Riemann-Roch theorem, into an upper bound on the genus $g$ and this is the Castelnuovo bound.

For the proof, the following definition will be convenient:

\begin{definition}
If $\sV = (V,cL)$ is a linear system on a variety and $\Gamma$ is a subscheme then the number of conditions
imposed by $\Gamma$ on $\sV$ is the dimension of the image of $V$ in $H^0(\sL\mid_\Gamma) = H^0(\sL \otimes \sO_\Gamma)$; or, numerically,
$$
\dim(V) - \dim \left(V \cap H^0(\cL\otimes \cI_{\Gamma/X}) \right).
$$\end{definition}

Thus, for example, if $\Gamma \subset \PP^r$, then the number of conditions imposed by $\Gamma$ on $H^0(\cO_{\PP^r}(m))$ is the value $h_\Gamma(m)$ of the Hilbert function of $\Gamma$ at $m$.
Note that the number of conditions imposed by $\Gamma$ on a linear system $V$ is necessarily less than or equal to the degree $d$ of $\Gamma$; if it is equal we say that $\Gamma$ \emph{imposes independent conditions on $V$}.

\begin{proof}[Proof of Theorem~\ref{Castelnuovo's bound}]
Suppose that $C \subset \PP^r$ is an irreducible, nondegenerate curve, and let $\Gamma = C \cap H$ be a general hyperplane section of $C$. Let $V_m \subset H^0(\cO_C(m))$ be the linear series cut on $C$ by hypersurfaces of degree $m$ in $\PP^r$, that is, the image of the restriction map
$$
H^0(\cO_{\PP^r}(m)) \to H^0(\cO_C(m)).
$$
Since the ideal sheaf of the hyperplane section is $\cI_{\Gamma/C} = \sO_C(-1)$ we have:
\begin{align*}
h^0(\cO_C(m)) - h^0(\cO_C(m-1)) & \geq \text{\# of conditions imposed by $\Gamma$ on $H^0(\cO_C(m))$} \\
&\geq \text{\# of conditions imposed by $\Gamma$ on $V_m$} \\
&\geq \text{\# of conditions imposed by $\Gamma$ on $H^0(\cO_{\PP^r}(m))$} \\
&\geq h_\Gamma(m).
\end{align*}
Summing these relations, we see that the dimension $h^0(\cO_C(m))$ is bounded below by
$$
h^0(\cO_C(m)) \geq \sum_{k=0}^m h_\Gamma(k).
$$

Thus, in order to bound the genus $C$ from above, we have to bound the Hilbert function of its hyperplane section $\Gamma$  from below; and for this, we need to know something about the geometry of $\Gamma$. In fact, all we need to know is Lemma~\ref{general position lemma}, which says that the points of $\Gamma$ are in linear general position! The key fact is the

\begin{proposition}\label{min hilb}
If $\Gamma \subset \PP^n$ is a collection of $d$ points in linearly general position that span $\PP^n$, then 
$$
h_\Gamma(m) \geq \min\{d, mn+1\}
$$
\end{proposition}

\begin{proof}
Suppose first that $d \geq mn+1$, and let $p_1,\dots,p_{mn+1} \in \Gamma$ be any subset of $mn+1$ points. It suffices to show that $\Gamma' = \{p_1,\dots,p_{mn+1}\}$ imposes independent conditions of $H^0(\cO_{\PP^n}(m))$, that is, for any $p_i \in \Gamma'$ there is a hypersurface $X \subset \PP^n$ of degree $m$ containing all the points $p_1,\dots, \hat{p_i},\dots,p_{mn+1}$ but not containing $p_i$.

To construct such an $X$, group the $mn$ points of $\Gamma' \setminus \{p_i\}$ into $m$ subsets $\Gamma_k$ of cardinality $n$; each set $\Gamma_k$ will span a hyperplane $H_k \subset \PP^n$, and we can take $X = H_1 \cup \dots \cup H_m$. 

In the case where $d<mn+1$, we add $mn+1-d$ general points; each one imposes exactly one
additional condition on hypersurfaces of degree $m$.
\end{proof}

This may seem like a crude argument, but the bound derived is sharp, as shown by the example in Exercise~\ref{linear bound is sharp}.

To complete the proof of Theorem~\ref{Castelnuovo's bound} all that remains is to add up the lower bounds in the proposition. To this end, let $C \subset \PP^r$ be as above an irreducible, nondegenerate curve of degree $d$, and set $M = \lfloor{\frac{d-1}{r-1}}\rfloor$, so that we can write
$$
d = M(r-1) + 1 + \epsilon \quad \text{ with } \quad 0 \leq \epsilon \leq r-2.
$$
We have 
\begin{align*}
h^0(\cO_C(M)) &\geq \sum_{k=0}^M h^0(\cO_C(k)) - h^0(\cO_C(k-1)) \\
&\geq  \sum_{k=0}^M k(r-1)+1 \\
&= \frac{M(M+1)}{2}(r-1) + M + 1
\end{align*}
and similarly
$$
h^0(\cO_C(M+m)) \geq \frac{M(M+1)}{2}(r-1) + M + 1 + md.
$$
For sufficiently large $m$, the line bundle $\cO_C(M+m)$ will be nonspecial, so by the Riemann-Roch Theorem,
\begin{align*}
g &= (M+m)d - h^0(\cO_C(M+m)) + 1 \\
&\leq (M+m)d - \bigl(  \frac{M(M+1)}{2}(r-1) + M + 1 + md \bigr) \\
& = M\bigl( M(r-1) + 1 + \epsilon \bigr) - \bigl(  \frac{M(M+1)}{2}(r-1) + M + 1 \bigr) \\
&= \frac{M(M-1)}{2}(r-1) + M\epsilon.
\end{align*}
 \end{proof}

From the proof we see that if  $C\subset \PP^r$ who degree and genus achieve equality, then all the inequalities in the proof
must be equalities; in particular, the map $H^0(\sO_{\PP^r}(d)) \to H^0(\sO_C(d))$ is surjective for every $d$ so the curve is
projectively normal.


Castelnuovo in fact proved a converse to this observation:

\begin{fact}
If $d \geq 2n+3$, then any collection of $d$ points in $\PP^n$ in linearly general position with Hilbert function $h_\Gamma(2) = \min\{d, 2n+1\}$ must lie on a rational normal curve (and then $h_\Gamma(m) = \min\{d, mn+1\}$ for all $m$.) See~\cite[Lemma 3.9]{MR685427} for a proof.
\end{fact}
 
This was a key ingredient in Castelnuovo's theorem characterizing curves of maximal genus as
those lying in certain divisor classes on rational normal scrolls.
\fix{We'll see how it's applied in Chapter~\ref{ScrollsChapter} if we include it}

\section{Uniform position} \label{uniformSection}
We now return to the situation where the ground field $k$ is algebraically closed of characteristic 0.

The Uniform Position lemma deals with the \emph{monodromy group} of the points of a general hyperplane section of a curve $C \subset \PP^r$. To prove it we will use the classical topology (an equivalent definition is described in Cheerful Fact~\ref{Galois equals monodromy} below). 

We may describe the monodromy group informally as follows: Suppose that $C \subset \PP^r$ is an irreducible curve, and $H_0 \subset \PP^r$ a hyperplane transverse to $C$; say the intersection $C \cap H_0 = \{p_1,\dots,p_d\}$. As we vary $H_0$ continuously along a real arc $\{H_t\}$, staying within the open subset $U \subset {\PP^r}^*$ of hyperplanes transverse to $C$, we can ``follow" each of the points $p_i(t)$ of intersection of $C$ with the hyperplane $H_t$.

Now imagine that the hyperplanes $H_t$ come back to the original $H_0$ at some time; that is, we have a continuous family $\{H_t\}_{0 \leq t \leq 1}$ with $H_1 = H_0$. Each of the points $p_i$ then traces out a continuous real arc 
$\{p_i(t) \in C \cap H_t\}_{0 \leq t \leq 1}$. Since $H_1 = H_0$, the end point $p_i(1)$ is one of the original points $p_j \in C \cap H_0$. In this way, we get a permutation of the set $C \cap H_0$; the group of all permutations arrived at in this way will be called the \emph{monodromy group} of the points $C \cap H_0$. 

We will now give a precise definition of the mondromy group in a more general setting, and prove that points of a general hyperplane section of an irreducible curve
the monodromy group is the full symmetric group; this is the uniform position lemma. The rest of the chapter will be a series of applications.

\subsection{The monodromy group of a generically finite morphism}

Let $f : Y \to X$ be a dominant map between varieties of the same dimension over $\CC$, and suppose that $X$ is irreducible. There is then an open subset $U \subset X$ such that $U$ and 
its preimage $V = f^{-1}(U)$ are smooth, and the restriction of $f$ to $V$ is a covering space in the classical topology. Let $d$ be the number of sheets. This is the degree of the extension $K(Y)/K(X)$.

Homotopy theory  associates a \emph{monodromy group} to any finite topological covering map $f : V \to U$, defined as follows: Choose a base point $p_0 \in U$, and suppose $\Gamma := f^{-1}(p_0)  = \{q_1,\dots,q_d\}$. If $\gamma$ is any loop in $U$ with base point $p_0$, for any $i = 1, \dots, d$ there is a unique lifting of $\gamma$ to an arc $\tilde \gamma_i$ in $V$ with initial point $\tilde \gamma_i(0) = q_i$ and end point $\tilde \gamma_i(1) = q_j$ for some $j \in \{1,2,\dots,d\}$. Since we could traverse the loop in the opposite direction, the index $j$ determines $i$, and the map $i\mapsto j$ is a permutation of $\{1,2,\dots,d\}$. 
Since the set $\Gamma$ is discreet, the permutation depends only on the class of $\gamma$ in $\pi_1(U,p_0)$ so we have defined a homomorphism to the symmetric group:
$$
\pi_1(U,p_0)  \to {\rm Perm}(\Gamma) \cong S_d.
$$
The image $M$ of this map is called the \emph{monodromy group} of the map $f$. It depends on the labeling of the points of $\Gamma$, but a change in labeling
only changes the group by conjugation with the corresponding permutation. In our setting $\pi_1(U,p_0)$, and thus the monodromy group, is independent of the choice of open set $U$: if $U' \subset U$ is a Zariski open subset, the complement of $U\setminus U'$ has
real codimension 2 so the map $\pi_1(U', p_0) \to \pi_1(U,p_0)$ is surjective. Thus the image of $\pi_1(U', p_0)$ in $S_d$ is the same. \fix{do we need the boundary to
be a neighborhood retract or some such?? How about a reference?}

\begin{fact}\label{Galois equals monodromy}
The theory of finite coverings of algebraic varieties is not only analogous to Galois theory, it \emph{is} Galois theory: In the situation described above, if we assume that $Y$ is irreducible, then the pullback map $f^*$ expresses the function field $K(Y)$ as a finite algebraic extension of $K(X)$. The degree of this extension is the degree of the map $f$. The monodromy group of $f$  is the Galois group of the Galois normalization of $K(Y)$ over $K(X)$ (see \cite{Harris1979}). Indeed, in early treatments of Galois theory, such as Jordan's book \emph{Trait\'e des Substitutions} (\cite{}), function fields played as large a role as number fields.
\end{fact}

Since we assumed that $X$ is irreducible, the space $U$ is (path) connected, and it follows that the monodromy group is transitive if and only if the space $V$ is (path) connected. But $V$ is a smooth
variety, so this is the case if and only if $V$ is irreducible.

We will next compute the monodromy group of the  universal hyperplane section of a curve, constructed as follows:
Let $C \subset \PP^r$ be an irreducible, nondegenerate curve of degree $d$, and let $X = {\PP^r}^*$ be the space of hyperplanes in $\PP^r$. We define the \emph{universal hyperplane section of $C$} to be the projection  $f: Y\to {\PP^r}^*$ of the incidence variety
$$
Y = \{ (H, p) \in {\PP^r}^* \times C \mid p \in H \}.
$$
The fibers of $f$ are the hyperplane
sections of $C$, so $f$ is a dominant finite map. If we let $U\subset {\PP^r}^*$ be the open subset of hyperplanes
meeting $C$ transversely, then the restriction of $f$ to the preimage $V$ of $U$ is a covering space
whose fibers each consist of $d$ distinct points. The preimage in $Y$ of a point $p\in C$ is the set of hyperplanes containing
$p$, a copy of $\PP^{r-1}$, and thus $Y$ is irreducible. Thus the monodromy group of $f$ is transitive. But much more is true:

\begin{theorem}[Uniform Position Theorem]\label{uniform position lemma}
The monodromy group of the universal hyperplane section of an irreducible curve $C \subset \PP^r$ is the full symmetric group $S_d$.
\end{theorem}

Theorem~\ref{uniform position lemma} fails over fields of finite characteristic, though there is no known counterexample for smooth curves; see Exercise~\ref{strange curves} for singular examples, and \cite{Rathmann} and \cite{Kadets} for what is known. 

Theorem~\ref{uniform position lemma} implies that two subsets of the same cardinality in the general hyperplane section of $C$
are indistinguishable from the point of view of any discrete invariant that is semicontinuous in the Zariski topology. To make this precise,we introduce a definition:

\begin{definition}
Let $\phi : Y \to X$ be a finite morphism. By the \emph{restricted fiber power} $\tilde Y^n/X$ we will mean the complement of all diagonals in the ordinary fiber power; that is,
$$
\tilde Y^n/X := \{ (x, y_1,\dots, y_n) \in X \times Y^n \mid \phi(y_i) = x \text{ and } y_i \neq y_j \; \forall i \neq j \}
$$
\end{definition}

In down-to-earth terms, a point of $\tilde Y^n/X$ is a set of $n$ distinct points in a fiber of $\phi$ together with
the choice of a total order on these points. 

\begin{lemma}\label{transitivity lemma}
Let $f : Y \to X$ be a generically finite cover of degree $d$, with  monodromy group $M \subset S_d$.
$M$ is $n$ times transitive if and only if the restricted fiber power $\tilde Y^n/X$ is irreducible.
\end{lemma}

\begin{proof}
If we restrict ourselves to open subsets $U \subset X$ and $V = f^{-1}(U) \subset Y$ such that $U$ and $V$ are smooth and the restriction $f|_V : V \to U$ is a covering space map in the classical topology, then the restricted fiber powers are unions of connected component of the usual fiber power $Y^n/X$. The condition that the monodromy if $n$-times transitive is equivalent to the condition that the restricted fiber power $\tilde Y^n/X$ is connected; since the fiber powers are all smooth, this is equivalent to $\tilde Y^n/X$ being irreducible.
\end{proof}


\subsection{Proof of the uniform position theorem}

Let $C \subset \PP^r$ be an irreducible, nondegenerate curve of degree $d$ and $f : Y \subset {\PP^r}^* \times C \to  X = {\PP^r}^*$ its universal hyperplane section; let $U \subset {\PP^r}^*$ be the open subset of hyperplanes transverse to $C$ and $V = f^{-1}(U)$; let $M \subset S_d$ be the monodromy group of $V$ over $U$.
To show that  $M$ is the full symmetric group, it suffices to show that $M$ contains all transpositions, for this it is enought to show that $M$ is doubly transitive and contains one transposition.

To prove that $M$ is doubly transitive, we can give a concrete description of the restricted fiber power $\tilde V^2/U$: let
$$
\Sigma := \left\{ (H, p, q) \in {\PP^r}^* \times C \times C \mid p, q \in H \text{ and } p \neq q \right\}.
$$
Projection on the second and third factor expresses $\Sigma$ as a $\PP^{r-2}$-bundle over the complement $C \times C \setminus \Delta$ of the diagonal in $C \times C$. Thus $\Sigma$ is irreducible, and it follows that the restricted fiber square $\tilde V^2/U$, which is a Zariski open subset of $\Sigma$, is as well. Note that this part of the argument does not rely on any assumption about the characteristic.

Next we give a criterion for a monodromy group to contain a transposition:

\begin{lemma}\label{transposition lemma}
Let $f : Y \to X$ be a generically finite cover of degree $d$ over an irreducible variety $X$, with  monodromy group $M \subset S_d$.  
If,  for some smooth point $p \in X$ the fiber $f^{-1}(p)\subset V$ consists of $d-2$ reduced points $p_1,\dots, p_{d-2}$ and one point $q$ of multiplicity 2, where $q$ is also a smooth point of $Y$, then $M$ contains a transposition.
\end{lemma}

\begin{proof} Note that the hypothesis implies that $Y$ is smooth
locally near the fiber of $p$. Let $U \subset X$ be a Zariski open subset of the smooth locus in $X$, as in the definition of the monodromy group, so that  $V := f^{-1}(U)$ is also smooth and the restriction $f|_V : V \to U$ expresses $V$ as a finite $d$-sheeted covering space of $U$, with $U,V$ smooth and $p\in X$.

Let $B_i\subset Y$ be disjoint small connected closed neighborhoods of the points
in $f^{-1}(p)$. Because $f|_V$  is finite, the image $A := \cap f(B_i)$ has is a locally
closed subset of the same dimension as $X$ and thus $A$
 contains a neighborhood
of $p$ in the classical topology.

Let $p' \in A \cap U$. Two of the $d$ points of $f^{-1}(p')$ will lie in the component  of $B$ containing $q$; call these $q'$ and $q''$. Since $B \cap V$ is the complement of a proper subvariety in $B$ it is connected, and we can draw a real arc $\gamma : [0,1] \to B \cap V$ joining $q'$ to $q''$; by construction, the permutation of $f^{-1}(p')$ associated to the loop $f \circ \gamma$ will exchange $q'$ and $q''$ and fix each of the remaining $d-2$ points of $f^{-1}(p')$.
\end{proof}

\begin{proof}[Completion of the proof of the Uniform position lemma]
 It remains to show that a reduced irreducible curve $C$ of degree $d$ (in characteristic 0)
 has a hyperplane section consisting of $d-2$ smooth points and one double point; that is, a hyperplane simply tangent to the curve at one point and transverse everywhere else. By the results of 
 Section~\ref{isolated flexes and bitangents} there is a tangent line $L$ to $C$ at a smooth point that is not flex tangent, and is not tangent to $C$ at any other point. A general hyperplane $H$ containing $L$ meets $C$ doubly in the point of
 tangency. At each of the other points of $L\cap C$ the intersection $C\cap H$ is transverse because there are only finitely many lines to avoid; and at points not on $L$ $H\cap C$ is transverse by Bertini's theorem. This completes the proof.
\end{proof}

\begin{remark}
 The fact that not every tangent line is bitangent is true
even for holomorphic arcs. Using the theory of correspondences it is possible to show that
a general tangent line to $C$ does not meet $C$ at any further points; but
this is not true of analytic arcs
\end{remark}

 \section{Applications of uniform position}

Applying this to the universal hyperplane section of a curve $C \subset \PP^r$, a consequence of the Uniform Position Theorem is an irreducibility result:

\begin{corollary}\label{hyperplane section monodromy} If $C\subset \PP^r$ is a smooth curve of degree $d$, then 
for $k\leq \deg C$, the restricted fibered powers $\tilde Y^n/X$  of the universal hyperplane section 
of $C$ are irreducible.
\end{corollary}


\begin{corollary}[numerical uniform position lemma]\label{numerical uniform position lemma}
Let $C\subset \PP^r$ be an irreducible curve, and let $\Gamma = H\cap C$ be a general hyperplane section. Any two subsets of $\Gamma$ with the same cardinality impose the same number of conditions on forms of any degree; that is, any two subsets of the same cardinality have the same Hilbert function.
\end{corollary}

\begin{proof} Let $U = {\PP^r}^* \setminus C^*$ be the open subset of hyperplanes transverse to $C$, and let $Y\to U$ be the universal hyperplane section.
Corollary~\ref{hyperplane section monodromy} says that the restricted fiber powers $\tilde V^n/U$ are irreducible.

Now, for each $m$ the number of conditions that $\Gamma$ imposes on forms of degree $m$ is lower semicontinuous, so it achieves its maximum on a Zariski open subset of $\tilde V^n/U$. Since $\tilde V^n/U$ is irreducible, the complement $Z$ of this open set has dimension strictly less than $\dim \tilde V^n/U = \dim U$. Thus a general hyperplane $H \in {\PP^r}^*$ will lie outside the image of $Z$, meaning that the number of conditions imposed by all the $k$-element subsets $\Gamma \subset C \cap H$ will have this maximal value.
\end{proof}

This result may be seen as an important strengthening of Proposition~\ref{basic linear independence}, since if $C$ is a reduced, irreducible nondegenerate curve in $\PP^n$ then a general subset of $n$ points of $C$ is linearly independent and spans a hyperplane; Corollary~\ref{numerical uniform position lemma} says that his must
be true for every subset of $n$ points of every general hyperplane section, 
which reproves Proposition~\ref{basic linear independence}, though only in characteristic 0. 

\section{Sums of linear series}

Another consequence of the uniform position lemma is a result about sums of linear series.
Recall that if $D$ is a divisor on a curve $C$ we write $r(D) = \dim |D| = h^0(\cO_C(D))-1$.

\begin{corollary}\label{Clifford equality plus}
If $D,E$ are effective divisors on a curve $C$ then
$$
r(D+E) \geq r(D)+r(E).
$$
If the genus of $C$ is $>0$ and $D+E$ is birationally very ample, then the inequality is strict.
\end{corollary}.

Note that on $\PP^1$ any effective divisor $D$ has $r(D) = \deg D$, so the inequality above is
always an equality for $C = \PP^1$.

\begin{proof}
 The inequality follows in general because the sums of divisors in $|D|$ and divisors in $|E|$ already move in 
 a family of dimension $r(D)+r(E)$; the key point is the strict inequality in case $D+E$ is birationally very ample.
 
If $D+E$ is birationally very ample then, restricting to an open set,
we may identify $C$ with its image under the complete linear series $|D+E|$, and we see that a general hyperplane section $H\cap C$ contains a divisor equivalent to $D$.

Let $Y$ be the $\deg D +\deg E$ restricted fiber power of the universal hyperplane.
A point $y\in Y$ is a hyperplane section plus an ordering of its points.  Let $\phi: Y \to \Pic_d(C)$ be the Abel-Jacobi map taking $y$
 to the class of the divisor that is the sum of first $d$ points in this order. The preimage  $Y'$ of the point of $\Pic_d(C)$ corresponding to the class of $D$ is a closed subset, and
since every divisor in the class of a hyperplane section contains a divisor
linearly equivalent to  $D$, the subvariety $Y'$ dominates $\PP^{n*}$, and thus
has the same dimension as $Y$. Consequently $Y'=Y$, and the sum of the first $d$ points
in any ordering of the general hyperplane section---that is, the sum of any $d$
of the points---is equivalent to $D$.

Thus if $p\in D$ and $q\notin D$, then $D-p+q \equiv D$, whence $q\equiv p$. Thus
$r(p)\geq 1$, so $C\cong \PP^1.$
\end{proof}

A special case of Corollary~\ref{Clifford equality plus} completes the proof of the ``equality'' case of Clifford's Theorem~\ref{Clifford equality}:

\begin{corollary}\label{Clifford equality proven}
If $D$ is an effective divisor  of degree $\leq 2g-2$ on a curve of genus $g\geq 2$ and $r(D) = d/2$, then either $D= 0$ or 
$D=K$ or $C$ is hyperelliptic and $D$ is a multiple of the $g^1_2$.
\end{corollary}

\begin{proof}
  By the argument given in Theorem~\ref{Clifford}, if $r(D) = d/2$, but $D\neq 0,K$, then
$$
r(D) + r(K-D) = r(K) = g-1;
$$
which implies that every canonical divisor is expressible as a sum of a divisor in $|D|$ and a divisor in $|K-D|$.
By Corollary~\ref{Clifford equality plus}, $K$ is not very ample; so $C$ is hyperelliptic. Moreover the
divisors equivalent to $D$ are sums of the fibers of the map given by $|K|$.
\end{proof}



%For every $k\leq \deg C$ we consider the fiber product over $U$ minus the main diagonal $\Delta$
%\begin{align*}
% Y^{k*} :=&Y\times_U Y\times_U \cdots Y\times_U \setminus \Delta  =\\
%                                                 & \{ (H, p_1,\dots, p_k) \in {\PP^r}^* \times C^k \mid p_1,\dots, p_k \in H, p_i \neq p_j \hbox{ for all } i,j \}.
%\end{align*}

%\begin{proof}[Proof of the Uniform Position Theorem]
%We show first that the monodromy group $M$ is twice transitive, and then that it contains a transposition. It follows that $M$ contains all the transpositions so that  $M = S_d$.
%Restricting to an open subset of $C$, we may assume that $C$ is smooth.
%
%For the double transitivity, we introduce a related cover: set
%$$
%\Phi = \{ (H, p, q) \in {\PP^r}^* \times C \times C \mid p + q \subset H \}
%$$
%where $p+q$ is the divisor $p+q$ on $C$, viewed as a subscheme of $C$. The projection $\pi_{2,3} : \Phi \to C \times C$ is a $\PP^{r-2}$-bundle, and so irreducible; applying Lemma~\ref{transitivity lemma}, we deduce that $M$ is twice transitive.
%
%To prove that $M$ contains a transposition, we will use the characteristic 0 hypothesis. It follows from Bertini's theorem and Lemma~\ref{tangent not bitangent} that a general hyperplane $H$ containing
%the tangent line at a general point $p\in C$ meets $C$ with multiplicity exactly 2 at $p$ and meets $C$ transversely elsewhere. 
%Thus the fiber of $\Phi$ over the point $H \in {\PP^r}^*$ consists of the point $p$ with multiplicity 2, and $d-2$ reduced points; applying Lemma~\ref{transposition lemma}, we deduce that $M$ contains a transposition.
%\end{proof}


\section{Exercises}

\begin{exercise}
Let $C\subset \PP^n$ be a smooth curve. If we re-embed $C$ by a Veronese map of sufficiently high degree---that is, we let $\nu_m : \PP^n \to \PP^N$ be the $m$th Veronese map, and let $\widetilde C = \nu_m(C)$ be image of $C$. Prove:

\begin{proposition}[Proposition~\ref{nodal projection} in characteristic $p$]\label{positive characteristic nodes}
The projection of $\widetilde C$ from a general $\PP^{N-3}$ is a nodal plane curve.
\end{proposition}

Hint: show that the items of~\ref{needed for nodes}  are true in this situation.
\end{exercise}

\begin{exercise}
Let $C_0$ be a plane quartic curve with two nodes $q_1, q_2$; let $\nu : C \to C_0$ be its normalization, and let $o \in C$ be any point not lying over a node of $C_0$.
By the genus formula, $C$ has genus 1. Using the construction above, describe the group law on $C$ with $o$ as origin.
\end{exercise}

%\begin{exercise}\label{multisecants to strange} 
%Let $C\subset \PP^n$ be a reduced, irreducible nondegenerate curve, with $n\geq 3$.
%\begin{enumerate}
%\item If the general secant line to $C$ meets $C$ intersects $C$ in a scheme of length $\geq 3$, then the tangent lines to $C$ at general points meet each other. 
%
%Hint: First reduce, by projecting to the case of a curve in $\PP^3$. Then consider the
%projection from a general point
%on the curve. Consider what it would mean for the projection to be inseparable. Supposing
%on the contrary that it is separable, let $p$ be a general point in the image and
%consider tangent lines at the two preimages of $p$.
%
%\item If the tangent lines to $C$ at general points meet each other, then all the tangent lines to $C$ pass through a common point of $\PP^n$, and thus $C$ is strange.
%
%Hint: use the fact that $C$ does not lie in a plane.
%
%\item If $C$ is not strange, then projection from a general point of $C$ is birational.
%
%Hint: See \cite[Proposition IV.3.8]{Hartshorne1977} for the first two parts. The last part follows from the first two.
%\end{enumerate}
%\end{exercise}
%
\begin{exercise}\label{strange curves} \cite{Rathmann}. Let $k$ be an algebraically closed field of characteristic $p>0$, and let $q=p^e$ for some $e\geq 1$. Let $C\subset \PP^n$
be the closure of the image $C_0$ of the morphism
$$
\AA^1 \ni t \mapsto (t, t^q, t^{q^2}, \dots , t^{q^{n}}) \in \AA^n
$$
where $\AA^n\subset \PP^n$ is the open set $x_0=1$. 
\begin{enumerate}
\item Show that $C$ is a complete intersection, defined by the equations
$$
x_0^{q-1}x_2 - x_1^q, x_0^{q-1}x_3 - x_2^q,\dots, 
x_0^{q-1}x_n - x_{n-1}^q.
$$
\item Show that $C$ is singular unless $q = n = 2$.
\item Show every secant line to $C_0$ contains $q$ points of $C_0$; more generally, if
$a_1, \dots, a_r$ are linearly independent points of $C_0$, show that the linear span of
$a_1, \dots, a_r$ contains $q^{(r-1)}$ points of $C_0$.  Compare this with the configuration of
points in affine $n$-space over a field of $q$ elements.
\end{enumerate}
\end{exercise}

Here is a generalization of the uniform position theorem for varieties of higher dimension:

\begin{exercise}
Let $X \subset \PP^r$ be an irreducible variety of dimension $k$, and let $\Phi \subset \GG(r-k, r) \times X$ the universal $(r-k)$-plane section; that is,
$$
\Phi := \{ (\Lambda, p) \in \GG(r-k,r) \times X \mid p \in \Lambda \}.
$$
Show that the monodromy of the projection $\Phi \to \GG(r-k,r)$ is the full symmetric group.
\end{exercise}

As a consequence of this, we can deduce the Bertini irreducibility theorem:

\begin{exercise}
Let $X \subset \PP^r$ be an irreducible variety of dimension $k \geq 2$. Show that a general hyperplane section of $X$ is irreducible.
\end{exercise}

\begin{exercise}
Let $C \subset \PP^r$ be a union of irreducible curves $C_i$ of degrees $d_i$. Prove that the monodromy group of the points of a general hyperplane section of $C$ is the product $\prod S_{d_i}$.
\end{exercise}

Hint: We need to know that the dual hypersurfaces $C_i^* \subset {\PP^r}^*$ are all distinct; given this, we can exhibit loops that induce a given permutation of the points of $H \cap C_i$ while fixing the points of $H \cap C_j$ for $j \neq i$.


\begin{exercise}
Here is another approach to the $g+2$ theorem in the hyperelliptic case: 
Let $C$ be a hyperelliptic curve of genus $g$ and $D$ a general divisor of degree $g+1$ on $C$; let $|E|$ be the $g^1_2$ on $C$.
Consider the map $\phi : C \to \PP^1 \times \PP^1$ given as the product of the maps $\phi_D : C \to \PP^1$ and $\phi_E : C \to \PP^1$ given by the pencils $|D|$ and $|E|$.
\begin{enumerate}
\item Show that $\phi$ embeds the curve $C$ as a curve of bidegree $(g+1,2)$ on $\PP^1 \times \PP^1$.
\item Now embed $\PP^1 \times \PP^1$ into $\PP^3$ as a quadric surface $Q$; pick a general point $p \in C \subset Q$ and project $C$ from the point $p$. Show that the image curve $C_0$ is a plane curve of degree $g+2$ with one ordinary $g$-fold point.
\end{enumerate}
\end{exercise}

\begin{exercise}
Show that with $C$ and $M$ as above, the line bundle $\cO_C(M)$ is nonspecial. (We will see in Section~\ref{} that this is sharp; that is, there exist curves $C \subset \PP^r$ with $\cO_C(M-1)$ special).
\end{exercise}

\begin{exercise}\label{extremal m-ics}
Establish the analog of Proposition~\ref{rnc on most quadrics} for hypersurfaces of any degree $m$, that is to say no irreducible, nondegenerate curve in $\PP^r$ lies on more hypersurfaces of degree $m$ than the rational normal curve.
To do this, let $C\subset \PP^d$ be any irreducible nondegenerate curve. Let $\Gamma$ be a general hyperplane section
of $C$, and use the exact sequences
$$
0 \to \cI_{C/\PP^d}(l-1) \to \cI_{C/\PP^d}(l) \to \cI_{\Gamma/\PP^{d-1}}(l) \to 0.
$$ 
with $2 \leq l \leq m$ to show that
$$
h^0(\cI_{C/\PP^d}(m)) \leq  \binom{d+m}{m} - (md+1)
$$
with equality only if $C$ is a rational normal curve.
\end{exercise}

\begin{exercise}\label{linear bound is sharp}
Let $D \subset \PP^n$ be a rational normal curve. If $\Gamma \subset D$ is any collection of $d$ points on $D$ (or for that matter any subscheme of $D$ of degree $d$) then the Hilbert function of $\Gamma$ is
$$
h_\Gamma(m) = \min\{d, mn+1\}
$$
\end{exercise} 

\begin{exercise}
Let $C \subset \PP^r$ be an irreducible, nondegenerate curve of degree $d$, and set $M = \lfloor{\frac{d-1}{r-1}}\rfloor$ as in the proof of Castelnuovo's theorem.
Show that the line bundle $\cO_C(M)$ is nonspecial. (We will see in Section~\ref{} that this is sharp; that is, there exist curves $C \subset \PP^r$ with $\cO_C(M-1)$ special).
\end{exercise}

\input footer.tex