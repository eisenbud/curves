%header and footer for separate chapter files

\ifx\whole\undefined
\documentclass[12pt, leqno]{book}
\usepackage{graphicx}
\input style-for-curves.sty
\usepackage{hyperref}
\usepackage{showkeys} %This shows the labels.
%\usepackage{SLAG,msribib,local}
%\usepackage{amsmath,amscd,amsthm,amssymb,amsxtra,latexsym,epsfig,epic,graphics}
%\usepackage[matrix,arrow,curve]{xy}
%\usepackage{graphicx}
%\usepackage{diagrams}
%
%%\usepackage{amsrefs}
%%%%%%%%%%%%%%%%%%%%%%%%%%%%%%%%%%%%%%%%%%
%%\textwidth16cm
%%\textheight20cm
%%\topmargin-2cm
%\oddsidemargin.8cm
%\evensidemargin1cm
%
%%%%%%Definitions
%\input preamble.tex
%\input style-for-curves.sty
%\def\TU{{\bf U}}
%\def\AA{{\mathbb A}}
%\def\BB{{\mathbb B}}
%\def\CC{{\mathbb C}}
%\def\QQ{{\mathbb Q}}
%\def\RR{{\mathbb R}}
%\def\facet{{\bf facet}}
%\def\image{{\rm image}}
%\def\cE{{\cal E}}
%\def\cF{{\cal F}}
%\def\cG{{\cal G}}
%\def\cH{{\cal H}}
%\def\cHom{{{\cal H}om}}
%\def\h{{\rm h}}
% \def\bs{{Boij-S\"oderberg{} }}
%
%\makeatletter
%\def\Ddots{\mathinner{\mkern1mu\raise\p@
%\vbox{\kern7\p@\hbox{.}}\mkern2mu
%\raise4\p@\hbox{.}\mkern2mu\raise7\p@\hbox{.}\mkern1mu}}
%\makeatother

%%
%\pagestyle{myheadings}

%\input style-for-curves.tex
%\documentclass{cambridge7A}
%\usepackage{hatcher_revised} 
%\usepackage{3264}
   
\errorcontextlines=1000
%\usepackage{makeidx}
\let\see\relax
\usepackage{makeidx}
\makeindex
% \index{word} in the doc; \index{variety!algebraic} gives variety, algebraic
% PUT a % after each \index{***}

\overfullrule=5pt
\catcode`\@\active
\def@{\mskip1.5mu} %produce a small space in math with an @

\title{Personalities of Curves}
\author{\copyright David Eisenbud and Joe Harris}
%%\includeonly{%
%0-intro,01-ChowRingDogma,02-FirstExamples,03-Grassmannians,04-GeneralGrassmannians
%,05-VectorBundlesAndChernClasses,06-LinesOnHypersurfaces,07-SingularElementsOfLinearSeries,
%08-ParameterSpaces,
%bib
%}

\date{\today}
%%\date{}
%\title{Curves}
%%{\normalsize ***Preliminary Version***}} 
%\author{David Eisenbud and Joe Harris }
%
%\begin{document}

\begin{document}
\maketitle

\pagenumbering{roman}
\setcounter{page}{5}
%\begin{5}
%\end{5}
\pagenumbering{arabic}
\tableofcontents
\fi


\chapter{Scrolls and the curves on them}
\label{ScrollsChapter}

Some of the simplest subvarieties in projective space are the \emph{rational normal scrolls}. One way to define a rational normal scroll is to say that it is the variety cut out by the $2\times 2$ minors of a $2\times m$ matrix of linear forms satisfying the following non-degeneracy condition: 
\begin{definition}
 A matrix of linear forms is said to be \emph{1-generic} if, even after arbitrary row and column transformations, it's entries are all non-zero. Note that if a $2\times n$ matrix
 $M$ of linear forms has rank 2 and is not 1-generic, then the ideal $I_2(M)$ generated by the $2\times 2$ minors of $M$ contains the product of 2 linear forms, and in fact the converse is true as well (see ***). 
\end{definition}
\begin{definition}
A \emph{rational normal scroll} (over an algebraically closed field) is a variety defined by the ideal of $2\times 2$ minors
of a 1-generic matrix. 
\end{definition}
For example, the matrix 
$$
M = \begin{pmatrix}
 x &y\\
 z&x
\end{pmatrix}
$$
over $\CC[x,y,z]$ is  1-generic, since
$\det M = x^2-yz$ is irreducible. The corresponding variety, the conic in $\PP^2$,
is a rational normal curve.

On the other hand, the matrix
$$
M' = \begin{pmatrix}
 x &y\\
 -y&x
\end{pmatrix}
$$
over $\CC[x,y]$ is not, since
$$
\begin{pmatrix}
1&0\\
-i&1 
\end{pmatrix}
M'
\begin{pmatrix}
 1&0\\
 i&1
\end{pmatrix}
= 
\begin{pmatrix}
 x+iy&0\\
 0&x-iy
\end{pmatrix}
$$
(but note that it would be 1-generic if we restricted scalars to $\RR$---thus the definition depends on the field).

In practice, 1-generic matrices arise in the following way. Let $X$ be
an irreducible, reduced variety, and suppose that $(\cL, V)$ is a linear series. Suppose that we there are two linear series $(\cL_1,V_1),  (\cL_2,V_2)$ on $X$
such that $\cL = \cL_1\otimes\cL_1$ and $V_1\otimes V_2 \subset V$. If we
choose
bases $\{u_i\}$ of $V_1$ and $\{v_j\}$ of $V_2$ then we may regard $\ell_{i,j}:=u_iv_j\in V$
as a linear form on $\PP(V)$.

\begin{proposition}
 With notation above, the $\dim V_1 \times \dim V_2$ matrix 
$M :=  (\ell_{i,j})$ is 1-generic. Moreover, its $2\times 2$ minors are contained in 
the image of $X$ under the linear series $(cL, V)$.
\end{proposition}

\begin{proof}
The entries of $M$, after any row and column operations, have the form $uv$, where
$u$ and $v$ are nonzero sections of $\cL_1$ and $\cL_2$, respectively. Since $X$ is irreducible and reduced, $u$ and $v$ can vanish only on nowhere dense subsets of $X$, so $uv$ is a nozero.

We may regard all the sections of the line bundles as elements of the 
field of rational functions on $X$, so a $2\times 2$ minor 
$$
\det
\begin{pmatrix}
 \ell_{i,j}&\ell_{i,j'} \\
 \ell_{i',j}&\ell_{i',j'}
\end{pmatrix}
= (u_iv_j)(u_{i'}v_{j'} - u_{i}v_{j'} u_{i'}v_{j}
$$
vanishes on $X$ by the associativity of multiplication.
\end{proof}

Consider, for example, a rational normal curve $C$ of degree $a$ in $\PP^a$, with $a\geq 2$, 
the image of the complete linear series $|\cL|$ with $\cL = \cO_{\PP^1}(a).$ 
We can  write $\cL = \cL_1\otimes \cL_2$ in many ways, for example with
$\cL_1 = \cO_{\PP^1}(1), \cL_2 = \cO_{\PP^1}(a-1)$, and taking
$V_i = \HH^0(\cL_i)$ we have $V_1V_2 = \HH^0(\cL)$. Choosing bases
\begin{align*}
\{s,t\} &\subset   \HH^0(\cL_1)\\
\{s^{a-1}, s^{a-2}t, \dots, t^{a-1}\} &\subset \HH^0(\cL_i)\\
\end{align*}
writing $z_i = s^{a-i}t^i$, we see that the corresponding 1-generic matrix is
$$
M_a:= \begin{pmatrix}
 z_0&z_1&\dots&z_{a-1}\\
 z_1&z_2&\dots&z_{a}\\
\end{pmatrix}.
$$
\begin{proposition}
 The $2\times 2$ minors of $M_a$ generate the ideal of the rational normal curve of degree $a$
\end{proposition}
\begin{proof}
 
\end{proof}
The first  example to keep in mind is the rational normal curve of degree $a$---the image of $\PP^1$ under complete linear series $|\cO_{\PP^1}(a)|$:
$$
\phi_a:\PP^1 \to \PP^a:\quad (s,t) \mapsto (s^a, s^{a-1}t,\dots, t^a).
$$
In general, if $a_1,\dots, a_m$ is a sequence of non-negative integers, we define the  \emph{rational normal scroll $S(a_1,\dots, a_m)$} 
to be the subvariety of dimension $m$ in $\PP^{\sum_i a_i+m-1}$ constructed as follows:

\begin{enumerate}
 \item Write $\CC^{\sum_i a_i+m} = \oplus_{i=1}^m\CC^{a_i+1}$ 
$$
\PP^n \supset \coprod_{i= 1}^m V_i; \quad \hbox{where} \quad V = \oplus_{i= 1}^m V_i.
$$
\item For each $i = 1...m$, choose an isomorphism 
$$
\phi_{a_i}:\PP^1\rOnto C_{a_i}\subset \PP^{a_i}\subset \PP^m
$$ 
onto a rational normal curve of degree $a_i$.
\item Let 
$$
S(a_1,\dots,a_m) := \bigcup_{p\in \PP^1} L_p
$$
 be the union of the linear spaces 
$$
L_p := \overline{\phi_{a_1}(p), \dots, \phi_{a_m}(p)}; \quad p\in \PP^1
$$
spanned by sets of corresponding point on the curves $C_{a_i}$.
\end{enumerate}

any variet varieties that occur as divisors on these play an important role in our story.
\section{ Geometric Description of surface scrolls}; lines joining 2 rational normal curves, Hirzebruch Surfaces. Mention higher-dim scrolls, but do surface scrolls in more detail: 


\begin{fact} \fix{where should this go?}
 Classification of minimal degree varieties
\end{fact}
 
  
\section{ general emb of degree $g+3$, in $\PP^3$ as divisor on a quadric of type $(2,g+1)$}

\section{ determinantal varieties}
  Map to $\PP^1$ is the line bundle defined by the cokernel. Get the VB as the pushforward of $\cO(1)$.

\section{ Linear series on hyperell curve}
 1) special linear series are mult $g^1_2$+basepoints. 2) Given an embedding, there's a union of lines. If the embedding is complete, we get a matrix...that defines the union of lines. Scrolls in all dimensions as unions of spans of divisors.
 

\section{ dim and genus of curves on scrolls.} Condition for the existence of integral curves in each class iff the int number with the directrix is >0 and the curve is not a multiple of the fiber. 

 
 
\begin{fact}
 Minimal degree varieties; as the varieties of given degree lying on the maximal number of quadrics.
\end{fact}

\section{ canonical image of a trigonal curve}
It lies on a 2-dim scroll (non -subcanonical embedding only on 3-dim scrolls).  embedding of a trigonal curve lies on the same scroll.Stratification of trigonal curves by Maroni invariants. Dimensions via automorphism groups of scrolls.
 
\section{ Statement of Castelnuovo Theorem}




%footer for separate chapter files

\ifx\whole\undefined
%\makeatletter\def\@biblabel#1{#1]}\makeatother
\makeatletter \def\@biblabel#1{\ignorespaces} \makeatother
\bibliographystyle{msribib}
\bibliography{slag}

%%%% EXPLANATIONS:

% f and n
% some authors have all works collected at the end

\begingroup
%\catcode`\^\active
%if ^ is followed by 
% 1:  print f, gobble the following ^ and the next character
% 0:  print n, gobble the following ^
% any other letter: normal subscript
%\makeatletter
%\def^#1{\ifx1#1f\expandafter\@gobbletwo\else
%        \ifx0#1n\expandafter\expandafter\expandafter\@gobble
%        \else\sp{#1}\fi\fi}
%\makeatother
\let\moreadhoc\relax
\def\indexintro{%An author's cited works appear at the end of the
%author's entry; for conventions
%see the List of Citations on page~\pageref{loc}.  
%\smallbreak\noindent
%The letter `f' after a page number indicates a figure, `n' a footnote.
}
\printindex[gen]
\endgroup % end of \catcode
%requires makeindex
\end{document}
\else
\fi
