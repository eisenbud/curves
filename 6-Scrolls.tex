%header and footer for separate chapter files

\ifx\whole\undefined
\documentclass[12pt, leqno]{book}
\usepackage{graphicx}
\input style-for-curves.sty
\usepackage{hyperref}
\usepackage{showkeys} %This shows the labels.
%\usepackage{SLAG,msribib,local}
%\usepackage{amsmath,amscd,amsthm,amssymb,amsxtra,latexsym,epsfig,epic,graphics}
%\usepackage[matrix,arrow,curve]{xy}
%\usepackage{graphicx}
%\usepackage{diagrams}
%
%%\usepackage{amsrefs}
%%%%%%%%%%%%%%%%%%%%%%%%%%%%%%%%%%%%%%%%%%
%%\textwidth16cm
%%\textheight20cm
%%\topmargin-2cm
%\oddsidemargin.8cm
%\evensidemargin1cm
%
%%%%%%Definitions
%\input preamble.tex
%\input style-for-curves.sty
%\def\TU{{\bf U}}
%\def\AA{{\mathbb A}}
%\def\BB{{\mathbb B}}
%\def\CC{{\mathbb C}}
%\def\QQ{{\mathbb Q}}
%\def\RR{{\mathbb R}}
%\def\facet{{\bf facet}}
%\def\image{{\rm image}}
%\def\cE{{\cal E}}
%\def\cF{{\cal F}}
%\def\cG{{\cal G}}
%\def\cH{{\cal H}}
%\def\cHom{{{\cal H}om}}
%\def\h{{\rm h}}
% \def\bs{{Boij-S\"oderberg{} }}
%
%\makeatletter
%\def\Ddots{\mathinner{\mkern1mu\raise\p@
%\vbox{\kern7\p@\hbox{.}}\mkern2mu
%\raise4\p@\hbox{.}\mkern2mu\raise7\p@\hbox{.}\mkern1mu}}
%\makeatother

%%
%\pagestyle{myheadings}

%\input style-for-curves.tex
%\documentclass{cambridge7A}
%\usepackage{hatcher_revised} 
%\usepackage{3264}
   
\errorcontextlines=1000
%\usepackage{makeidx}
\let\see\relax
\usepackage{makeidx}
\makeindex
% \index{word} in the doc; \index{variety!algebraic} gives variety, algebraic
% PUT a % after each \index{***}

\overfullrule=5pt
\catcode`\@\active
\def@{\mskip1.5mu} %produce a small space in math with an @

\title{Personalities of Curves}
\author{\copyright David Eisenbud and Joe Harris}
%%\includeonly{%
%0-intro,01-ChowRingDogma,02-FirstExamples,03-Grassmannians,04-GeneralGrassmannians
%,05-VectorBundlesAndChernClasses,06-LinesOnHypersurfaces,07-SingularElementsOfLinearSeries,
%08-ParameterSpaces,
%bib
%}

\date{\today}
%%\date{}
%\title{Curves}
%%{\normalsize ***Preliminary Version***}} 
%\author{David Eisenbud and Joe Harris }
%
%\begin{document}

\begin{document}
\maketitle

\pagenumbering{roman}
\setcounter{page}{5}
%\begin{5}
%\end{5}
\pagenumbering{arabic}
\tableofcontents
\fi


\chapter{Scrolls and the curves on them}
\label{ScrollsChapter}

\fix{put the following here or in the $g=3, d=3$
Hilbert scheme of curves in $\PP^{3}$.}

\begin{verbatim}
 The naming of cats is a difficult matter,
 It isn't just one of your everyday games.
 You may think that I am as mad as a hatter,
 When I tell you each cat must have three different 	names.
 --T.S.Eliot, Practical Cats
\end{verbatim}
\section{1-generic matrices and scrolls}
Some of the simplest subvarieties in projective space are the \emph{rational normal scrolls}. One way to define a rational normal scroll is to say that it is the variety cut out by the $2\times 2$ minors of a $2\times m$ matrix of linear forms satisfying the following non-degeneracy condition: 
\begin{definition}
 A matrix of linear forms is said to be \emph{1-generic} if, even after arbitrary row and column transformations, it's entries are all non-zero. Note that if a $2\times n$ matrix
 $M$ of linear forms has rank 2 and is not 1-generic, then the ideal $I_2(M)$ generated by the $2\times 2$ minors of $M$ contains the product of 2 linear forms, and thus is not prime.
\end{definition}

\begin{definition}
A \emph{rational normal scroll} (over an algebraically closed field) is a variety defined by the ideal of $2\times 2$ minors
of a 1-generic matrix. 
\end{definition}
For example, the matrix 
$$
M = \begin{pmatrix}
 x &y\\
 z&x
\end{pmatrix}
$$
over $\CC[x,y,z]$ is  1-generic, since
$\det M = x^2-yz$ is irreducible. The corresponding variety, the conic in $\PP^2$,
is a rational normal curve.

On the other hand, the matrix
$$
M' = \begin{pmatrix}
 x &y\\
 -y&x
\end{pmatrix}
$$
over $\CC[x,y]$ is not, since
$$
\begin{pmatrix}
1&0\\
-i&1 
\end{pmatrix}
M'
\begin{pmatrix}
 1&0\\
 i&1
\end{pmatrix}
= 
\begin{pmatrix}
 x+iy&0\\
 0&x-iy
\end{pmatrix}
$$
(but note that it would be 1-generic if we restricted scalars to $\RR$---thus the definition depends on the field).

In practice, 1-generic matrices arise in the following way. Let $X$ be
an irreducible, reduced variety, and suppose that $(\cL, V)$ is a linear series. Suppose that we there are two linear series $(\cL_1,V_1),  (\cL_2,V_2)$ on $X$
such that $\cL = \cL_1\otimes\cL_1$ and $V_1\otimes V_2 \subset V$. If we
choose
bases $\{u_i\}$ of $V_1$ and $\{v_j\}$ of $V_2$ then we may regard $\ell_{i,j}:=u_iv_j\in V$
as a linear form on $\PP(V)$.

\begin{proposition}\label{some generators}
 With notation above, the $\dim V_1 \times \dim V_2$ matrix 
$M :=  (\ell_{i,j})$ is 1-generic. Moreover, its $2\times 2$ minors are contained in 
the image of $X$ under the linear series $(cL, V)$.
\end{proposition}

\begin{proof}
The entries of $M$, after any row and column operations, have the form $uv$, where
$u$ and $v$ are nonzero sections of $\cL_1$ and $\cL_2$, respectively. Since $X$ is irreducible and reduced, $u$ and $v$ can vanish only on nowhere dense subsets of $X$, so $uv$ is a nozero.

We may regard all the sections of the line bundles as elements of the 
field of rational functions on $X$, so a $2\times 2$ minor 
$$
\det
\begin{pmatrix}
 \ell_{i,j}&\ell_{i,j'} \\
 \ell_{i',j}&\ell_{i',j'}
\end{pmatrix}
= (u_iv_j)(u_{i'}v_{j'}) - (u_{i}v_{j'}) (u_{i'}v_{j})
$$
vanishes on $X$ by the associativity of multiplication.
\end{proof}

Consider, for example, a rational normal curve $C$ of degree $a$ in $\PP^a$, with $a\geq 2$, 
the image of the complete linear series $|\cL|$ with $\cL = \cO_{\PP^1}(a).$ 
We can  write $\cL = \cL_1\otimes \cL_2$ in many ways, for example with
$\cL_1 = \cO_{\PP^1}(1), \cL_2 = \cO_{\PP^1}(a-1)$, and taking
$V_i = \HH^0(\cL_i)$ we have $V_1V_2 = \HH^0(\cL)$. Choosing bases
\begin{align*}
\{s,t\} &\subset   \HH^0(\cL_1)\\
\{s^{a-1}, s^{a-2}t, \dots, t^{a-1}\} &\subset \HH^0(\cL_i)\\
\end{align*}
writing $z_i = s^{a-i}t^i$, we see that the corresponding 1-generic matrix is
$$
M_a:= \begin{pmatrix}
 z_0&z_1&\dots&z_{a-1}\\
 z_1&z_2&\dots&z_{a}\\
\end{pmatrix}.
$$


\begin{theorem}Let $M$ be a 1-generic $2\times a$ matrix of linear forms on $\PP^n$, and
let $I = I_2(M)$  be the ideal generated by the $2\times 2$ minors of $M$. 
\begin{enumerate}
\item If $n=a$, and 
$$
M = M_a:= \begin{pmatrix}
 z_0&z_1&\dots&z_{a-1}\\
 z_1&z_2&\dots&z_{a}\\
\end{pmatrix},
$$
then
$I$ is the homogeneous ideal of the rational
normal curve.

\item If $n = a$ then the 1-generic matrix
$M$ is equivalent up to row and column transformations to the matrix $M_{a}$;
in particular, $I$ is prime and $V(I)$ is a rational normal
curve of degree $a$.

\item Independently of $n$, the ideal $I$ is
prime, and the variety $V = V(I) \subset \PP^n$ has degree $a$ and codimension $a-1$.
\end{enumerate}

\end{theorem}

\begin{proof} (1) The homogeneous ideal $J$ of $C$ is the kernel of the ring homomorphism
$$
\psi: S:= \CC[z_0,\dots,z_a] \to \CC[s,t]; \quad z_i \mapsto s^{a-i}t^i.
$$
 and
the coordinate ring $S_C = S/J$ is isomorphic to the ring spanned by
monomials in $s,t$ that have degree a multiple of $a$. Such
a monomial may be written uniquely as a power of
$s^a$
times a power of $t^a$ times a monomial $s^{a-i}t^i$ with $1\leq i\leq a-1$.

Let $I$ be the ideal $I_2(M_a)$ of $2\times 2$ minors of $M_a$. By Proposition~\ref{some generators} we have $I\subset J$. Thus to prove $I=J$ it suffices to show that
every monomial m in the $z_i$ can be written, modulo $I$, in the form
$z_0^pz_iz_a^q$ for some $p,q$ and $1\leq i\leq a-1$; that is, we never need to
use more than the first power of a single variable in the set $\{z_1,\dots, z_{a-1}\}$.

Suppose that $m$ is a monomial that cannot be so expressed, so that $m$contains $z_iz_j$ as a factor, with $1\leq i\leq j\leq a-1$. We do induction on $i$. Since
$M_a$ contains the submatrix
$$
\begin{pmatrix}
 z_{i-1} & z_{j}\\
 z_i & z_{j+1}
\end{pmatrix}
$$
we see that $z_iz_j \equiv z_{i-1}z_{j+1} \mod I$, completing the argument.

(2)use Lemma~\ref{size of 1-generic} prove that the (s,t)-generalized row
defines a point on a nondegenerate curve of degree $a$, thus a RNC; wlog \emph{the} RNC. First row: independent
forms van at (1,0), so $s^{a}\dots,st^{a-1}$. Now second row proportional to $s^{a-1}\dots,t^{a}$.

3) induct on number of vars. Mod a general lin form get a prime of correct. Show this is enough.

\end{proof}




\begin{lemma} \label{size of 1-generic}
There exist 1-generic $p\times q$ matrices of linear forms in $n+1$ variables over $\CC$ (or any algebraically closed field) if and only if $n\geq p+q$;
In particular, the dimension of the space of linear forms spanned by the $1\times 1$ minors of a  1-generic matrix $M$ of size $p\times q$ is at least $p+q-1$. Moreover, if this space of linear forms has dimension $>p+q-1$, then the restriction of $M$ to a general hyperplane is still 1-generic.
\end{lemma}

\begin{proof}
If we think of a polynomial ring $\CC[z_0,\dots,z_n]$ as the symmetric algebra
of a vector space $V$ of rank $n+1$, then we may regard a $p\times q$ matrix of
linear forms $M$ as coming from a map $m: \CC^{p}\otimes \CC^{q}\to V$. The matrix is 1-generic
if and only if no ``pure'' tensor $r\otimes s$ goes to zero, that is, iff the kernel $K$ of $m$ intersects the cone of
pure tensors only in 0. The cone of pure tensors is the cone over the Segre embedding of $\PP^{p-1}\times \PP^{q-1}$, 
an thus has dimension $(p-1)+(q-1)+1$. Thus a general subspace $K$ of codimension $\geq p+q-1$ will intersect the cone
only in 0, but any larger subspace $K$ will intersect the cone non-trivially,  and the first two statements follow.

Moreover, if $K$ is any space of codimension $>p+q-1$ that intersects the cone only in 0, then the general subspace $K'\supset K$
of dimension one larger still intersects the cone only in 0, proving the last statement.
\end{proof}

Note that the argument given is the the same as the proof of the bound in Clifford's Theorem: if $\cL$ is a special
line bundle on a curve $C$ then the map 
$$
\HH^0(\cL) \otimes \HH^0(\cL^{-1}\otimes \omega_C) \to \HH^0 (\omega_C)
$$
is 1-generic by Proposition~\ref{some generators}, and thus
$$
h^0(\cL)+h^1(\cL)-1 \leq g.
$$
By Riemann-Roch,  $h^1(\cL) = h^0(\cL) -d+g-1$, so this last relation becomes
$h^0(\cL)+(h^0(\cL) -d+g-1) -1 \leq g$, or $2(h^0(\cL)-1) \leq d$.


\section{Other descriptions and classification}

The $s$-dim scroll is 1 param union of 
$s-1$-dim spaces -- in fact spanned by
corresponding points on $s$ rational normal curves.
\begin{fact}
 With a little more effort one can show that any
 1-generic matrix is equivalent to a matrix of the 
 form
 $$
 M_{a_{1}, \dots, a_{s}} := 
\begin{pmatrix}
 z_{1,0}& \dots z_{1,a_{1}-1}&&
 z_{2,0}& \dots z_{2,a_{2}-1}&&\dots&&
 z_{s,0}& \dots z_{s,a_{s}-1}\\
 %
  z_{1,1}& \dots z_{1,a_{1}}&&
 z_{2,1}& \dots z_{2,a_{2}}&&\dots&&
 z_{s,1}& \dots z_{s,a_{s}}\\
\end{pmatrix}\,.
$$
\end{fact}

In general, if $a_1,\dots, a_m$ is a sequence of non-negative integers, we define the  \emph{rational normal scroll $S(a_1,\dots, a_m)$} 
to be the subvariety of dimension $m$ in $\PP^{\sum_i a_i+m-1}$ constructed as follows:

\begin{enumerate}
 \item Write $\CC^{\sum_i a_i+m} = \oplus_{i=1}^m\CC^{a_i+1}$ 
$$
\PP^n \supset \coprod_{i= 1}^m V_i; \quad \hbox{where} \quad V = \oplus_{i= 1}^m V_i.
$$
\item For each $i = 1...m$, choose an isomorphism 
$$
\phi_{a_i}:\PP^1\rOnto C_{a_i}\subset \PP^{a_i}\subset \PP^m
$$ 
onto a rational normal curve of degree $a_i$.
\item Let 
$$
S(a_1,\dots,a_m) := \bigcup_{p\in \PP^1} L_p
$$
 be the union of the linear spaces 
$$
L_p := \overline{\phi_{a_1}(p), \dots, \phi_{a_m}(p)}; \quad p\in \PP^1
$$
spanned by sets of corresponding point on the curves $C_{a_i}$.
\end{enumerate}


\begin{fact} 
 Classification of minimal degree varieties.
 
 Minimal degree varieties; as the varieties of given degree lying on the maximal number of quadrics.

\end{fact}
 
description as projectivized VB emb by $\cO(1)$.

  Map to $\PP^1$ is the line bundle defined by the cokernel. Get the VB as the pushforward of $\cO(1)$.
  
  
\section{Degrees and genera of curves on scrolls.} Condition for the existence of integral curves in each class iff the int number with the directrix is >0 and the curve is not a multiple of the fiber. 


\section{Hyperelliptic curves}

In our early encounters with curves, we frequently assumed that the curve we were consider was non-hyperelliptic, since the behavior of hyperelliptic curves is so atypical. In this section, we'll describe the geometry of hyperelliptic curves.

\subsection{Basic models of hyperelliptic curves}

We start by establishing some basic facts about hyperelliptic curves. Many of these follow from general theorems like Riemann-Roch; but since they can be established by direct examination we will carry that out here.

Suppose $C$ is a smooth, projective hyperelliptic curve of genus $g \geq 2$. By definition, $C$ admits a degree 2 map $\pi : C \to \PP^1$; and as we've observed (\ref{**}) this map is unique.

By Riemann-Hurwitz, the map $\pi : C \to \PP^1$ will have $2g+2$ distinct simple branch points, say $\lambda_1,\dots,\lambda_{2g-2} \in \PP^1$. $C$ can then be realized as the smooth projective completion of the affine curve given as
$$
C^\circ = \big\{ (x,y) \in \AA^2 \; \mid \; y^2 = \prod_{i=1}^{2g-2} (x - \lambda_i) \big\}.
$$ 
Note that if we simply take the closure of this locus in $\PP^2$, the resulting curve will be highly singular at the point $[1,0,0]$, as can be seen either  directly by making an appropriate change of variables, or by invoking the genus formula for plane curves: if the closure were smooth, it would have genus $\binom{2g+1}{2}$. We can, however, complete the curve simply in $\PP^1 \times \PP^1$, for example by setting
$$
y' = \frac{y}{\prod_{i=1}^{g-1} (x - \lambda_i)};
$$
we can then write the equation of an open subset of $C^\circ$ as
$$
{y'}^2 \cdot \prod_{i=1}^{g+1} (x - \lambda_i) \; = \; \prod_{i=g+2}^{2g+2} (x - \lambda_i).
$$
If we now take the closure of this locus in $\PP^1 \times \PP^1$, we get a curve of type $(2,g+1)$ on $\PP^1 \times \PP^1$; this curve is smooth, as can be seen again either directly in coordinates or by invoking the genus formula for curves on $\PP^1 \times \PP^1$. In other words,
$$
C \; = \; V\Big(Y_0^2\cdot \prod_{i=1}^{g+1} (X_1 - \lambda_iX_0) - Y_1^2 \cdot \prod_{i=g+2}^{2g+2} (X_1 - \lambda_iX_0) \Big)
$$

Next, let's describe the space of regular differentials on $C$. For this, it's convenient to work with the affine model $C^\circ = V(f) \subset \AA^2$, where
$$
f(x,y) = y^2 - \prod_{i=1}^{2g-2} (x - \lambda_i).
$$

We'll denote the two points at infinity---that is, the two points of $C \setminus C^\circ$---as $p$ and $q$.

To start, consider the simple differential $dx$. This is clearly regular on $C^\circ$, with zeros at the ramification points $r_i = (\lambda_i, 0)$. But it does not extend to a regular differential on all of $C$: it will have double poles at $p$ and $q$, as can be seen either directly or by degree considerations: as we said, $dx$ has $2g+2$ zeros, while the degree of $K_C$ is $2g-2$, meaning that there must be poles at the points $p$ and $q$.

To kill these poles, we can of course divide by $x^2$ (or any quadratic polynomial in $x$). But that just introduces new poles in the finite part $C^\circ$ of $C$. Instead, we want to multiply $dx$ by a rational function with zeros at $p$ and $q$, but \emph{whose poles occur only at the points where $dx$ has zeroes}---that is, the points $r_i$.  A natural choice is simply the reciprocal of the partial derivative $f_y = \partial f/ \partial y$, which vanishes exactly at the points $r_i$, and has correspondingly a pole of order $g+1$ at each of the points $p$ and $q$. In other words, the differential
$$
\omega = \frac{dx}{f_y}
$$
is regular, with divisor
$$
(\omega) = (g-1)p + (g-1)q.
$$
The remaining regular differentials on $C$ are now easy to find: we can simply multiply $\omega$ by any power $x^k$ with $k = 0, 1, \dots, g-1$; these then form a basis for $H^0(K_C)$.


 1) special linear series are mult $g^1_2$+basepoints. 2) Given an embedding, there's a union of lines. If the embedding is complete, we get a matrix...that defines the union of lines. Scrolls in all dimensions as unions of spans of divisors.
 
 
\subsection{General embeddings of degree
genus$+3$} 

It's a divisor on a quadric in $\PP^{3}$ of type $(2,g+1)$


\section{Trigonal curves}

\subsection{Special linear series on trigonal curves}

In analyzing special linear series on a hyperelliptic curve, we made crucial use of the facts that the canonical image of a hyperelliptic curve is a rational normal curve, and that any collection of points on a rational normal curve $C \subset \PP^n$ either are linearly independent or span $\PP^n$. In a similar (though necessarily less complete) way, we can use the fact that the canonical image of a trigonal curve lies on a rational normal surface scroll to describe special linear series on it.

 

\begin{lemma}
Let $S = S_{a,b} \subset \PP^n$ be a rational normal surface scroll. Any hyperplane section $H \cap S$ consists of the union of a rational normal curve $E$, which is a section of the scroll, and a union of lines of the ruling of the scroll.
\end{lemma}

Note that the curve $E$ must be a reduced component of $S \cap H$, but the lines $L_i$ may coincide, i.e., may be non-reduced components of the intersection. In the following proof, we'll assume for clarity that the lines $L_i$ are distinct (that is, $S \cap H$ is reduced); we leave it as an exercise to rewrite the proof to accommodate the remaining cases.

\begin{proof}
Let $F \in \Pic(S)$ be the class of a line of the ruling. Since $F^2 = 0$ and $H\cdot F = 1$, exactly one of the components of $S \cap H$ must have intersection number 1 with $F$; all other components must have intersection number 0 with $F$ and so must be lines of the ruling.

It remains to show that the unique component $E$ of $H \cap S$ having intersection number 1 with $F$ is a rational normal curve. This can be seen directly, but there's a shortcut. Suppose that we have
$$
S \cap H = E \cup L_1 + \dots + L_k,
$$
so that in particular $\deg(E) = n-1-k$. Since each of the lines $L_i$ of the ruling must meet $C$, we have that
\begin{align*}
n-1 &= \dim(\overline{S \cap H}) \\
&\leq \dim(\overline E) + k\\
&\leq (n-1-k) + k \\
&= n-1.
\end{align*}
We conclude that $\dim(\overline E) = n-k-1$, and hence that $E$ is a rational normal curve.
\end{proof}

Note that if $S = S_{a,b}$ with $a \leq b$, we must have either $0 \leq k \leq a$ or $k = b$: as soon as $k > a$, the span of the lines $L_i$ will contain the directrix of the scroll, and so must consist of the union of the directrix with $n-1-a = b$ lines.

Now let $C$ be a trigonal curve of genus $g \geq 5$, embedded in $\\P^{g-1}$ as a canonical curve, and let $S$ be the scroll containing $C$. We want to describe special linear series $\cD = |D|$. If our linear series has base points, we can delete them; so we'll assume that $|D|$ and $|K-D|$ are base point free. Note that this implies that   both  $r(D) \geq 1$ and $r(K-D) \geq 1$. In addition, it follows by Bertini that a general divisor $D \in \cD$ is reduced, that is, consists of distinct points $p_1,\dots,p_d$.

Now, the first hypothesis, that $r(D) \geq 1$, says that the points $p_1,\dots,p_d$ are linearly dependent. The second hypothesis, that $r(K-D) \geq 1$, says that the points $p_i$ span a subspace of codimension at least 2 in $\PP^{g-1}$ They therefore lie on at least a pencil of hyperplanes; let $H$ be a general hyperplane containing $D$.




. is effective, says that the divisor $D$ lies in a hyperplane section $C \cap H$; let $H$ be a general such hyperplane. At the same time 

canonical image lies on a 2-dim scroll (non -subcanonical embedding only on 3-dim scrolls).  embedding of a trigonal curve lies on the same scroll.Stratification of trigonal curves by Maroni invariants. Dimensions via automorphism groups of scrolls.

\section{Castelnuovo's Theorem}
(Statement only)

\section{Appendix: Centennial Account}
Our paper as appendix.
%footer for separate chapter files

\ifx\whole\undefined
%\makeatletter\def\@biblabel#1{#1]}\makeatother
\makeatletter \def\@biblabel#1{\ignorespaces} \makeatother
\bibliographystyle{msribib}
\bibliography{slag}

%%%% EXPLANATIONS:

% f and n
% some authors have all works collected at the end

\begingroup
%\catcode`\^\active
%if ^ is followed by 
% 1:  print f, gobble the following ^ and the next character
% 0:  print n, gobble the following ^
% any other letter: normal subscript
%\makeatletter
%\def^#1{\ifx1#1f\expandafter\@gobbletwo\else
%        \ifx0#1n\expandafter\expandafter\expandafter\@gobble
%        \else\sp{#1}\fi\fi}
%\makeatother
\let\moreadhoc\relax
\def\indexintro{%An author's cited works appear at the end of the
%author's entry; for conventions
%see the List of Citations on page~\pageref{loc}.  
%\smallbreak\noindent
%The letter `f' after a page number indicates a figure, `n' a footnote.
}
\printindex[gen]
\endgroup % end of \catcode
%requires makeindex
\end{document}
\else
\fi
