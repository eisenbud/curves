%header and footer for separate chapter files

\ifx\whole\undefined
\documentclass[12pt, leqno]{book}
\usepackage{graphicx}
\input style-for-curves.sty
\usepackage{hyperref}
\usepackage{showkeys} %This shows the labels.
%\usepackage{SLAG,msribib,local}
%\usepackage{amsmath,amscd,amsthm,amssymb,amsxtra,latexsym,epsfig,epic,graphics}
%\usepackage[matrix,arrow,curve]{xy}
%\usepackage{graphicx}
%\usepackage{diagrams}
%
%%\usepackage{amsrefs}
%%%%%%%%%%%%%%%%%%%%%%%%%%%%%%%%%%%%%%%%%%
%%\textwidth16cm
%%\textheight20cm
%%\topmargin-2cm
%\oddsidemargin.8cm
%\evensidemargin1cm
%
%%%%%%Definitions
%\input preamble.tex
%\input style-for-curves.sty
%\def\TU{{\bf U}}
%\def\AA{{\mathbb A}}
%\def\BB{{\mathbb B}}
%\def\CC{{\mathbb C}}
%\def\QQ{{\mathbb Q}}
%\def\RR{{\mathbb R}}
%\def\facet{{\bf facet}}
%\def\image{{\rm image}}
%\def\cE{{\cal E}}
%\def\cF{{\cal F}}
%\def\cG{{\cal G}}
%\def\cH{{\cal H}}
%\def\cHom{{{\cal H}om}}
%\def\h{{\rm h}}
% \def\bs{{Boij-S\"oderberg{} }}
%
%\makeatletter
%\def\Ddots{\mathinner{\mkern1mu\raise\p@
%\vbox{\kern7\p@\hbox{.}}\mkern2mu
%\raise4\p@\hbox{.}\mkern2mu\raise7\p@\hbox{.}\mkern1mu}}
%\makeatother

%%
%\pagestyle{myheadings}

%\input style-for-curves.tex
%\documentclass{cambridge7A}
%\usepackage{hatcher_revised} 
%\usepackage{3264}
   
\errorcontextlines=1000
%\usepackage{makeidx}
\let\see\relax
\usepackage{makeidx}
\makeindex
% \index{word} in the doc; \index{variety!algebraic} gives variety, algebraic
% PUT a % after each \index{***}

\overfullrule=5pt
\catcode`\@\active
\def@{\mskip1.5mu} %produce a small space in math with an @

\title{Personalities of Curves}
\author{\copyright David Eisenbud and Joe Harris}
%%\includeonly{%
%0-intro,01-ChowRingDogma,02-FirstExamples,03-Grassmannians,04-GeneralGrassmannians
%,05-VectorBundlesAndChernClasses,06-LinesOnHypersurfaces,07-SingularElementsOfLinearSeries,
%08-ParameterSpaces,
%bib
%}

\date{\today}
%%\date{}
%\title{Curves}
%%{\normalsize ***Preliminary Version***}} 
%\author{David Eisenbud and Joe Harris }
%
%\begin{document}

\begin{document}
\maketitle

\pagenumbering{roman}
\setcounter{page}{5}
%\begin{5}
%\end{5}
\pagenumbering{arabic}
\tableofcontents
\fi


\chapter{Scrolls and the curves on them}
\label{ScrollsChapter}

\begin{verbatim}
 The naming of cats is a difficult matter,
 It isn't just one of your everyday games.
 You may think that I am as mad as a hatter,
 When I tell you each cat must have three different 	names.
 The first is the name that the family use daily ...
 But I tell you, a cat needs a name that's particular ...
 But above and beyond there's still one name left over,  ...
 [his] deep and inscrutable, singular name.

 
 --T.S.Eliot, Practical Cats
\end{verbatim}

Some of the simplest subvarieties in projective space are the \emph{rational normal scrolls}. They appear in many contexts in algebraic geometry, and are useful, in particular, for describing the embeddings of curves of low degree and genus. 

We begin this chapter by giving three different characterizations of these varieties, each useful in a different context: First a classical geometric construction that gives a good picture, then an algebraic description that allows one to ``find" the scrolls containing a given variety, and then a more modern geometric definition that makes it easy to understand the divisors on a scroll. Finally, we turn to some of the applications to the embeddings of curves.

In each section we will focus on the 2-dimensional case, both because this is the case that occurs in our applications, and to simplify the discussion. We will also indicate the surprisingly simple extensions to higher dimensions.

In this chapter we will refer to rational normal scrolls simply as scrolls. The third characterization we will give lends itself to a natural generalization to families of irrational ruled varieties, which we'll briefly mention, and the reader should be aware that in the literature the word ``scroll'' is used for this wider class.

\section{Some classical geometry (the name the family use daily)}

Recall that the image of the map 
$$
\PP^1\to\PP^a: (s,t) \mapsto (s^a, s^{a-1}t, \dots, t^a),
$$
corresponding to the complete linear series
$|\cO_{\PP^1}(a)|$, is called a \emph{rational normal curve}; it is, up to linear transformations, the unique nondegenerate curve of degree $a$ in $\PP^a$.

Scrolls are one answer to the question: ``What is the higher-dimensional analogue of rational normal curves?'' To construct a scroll of dimension 2, we choose integers $0<a_1, a_2$ and consider  a projective space 
$$
\PP^{a_1+a_2+1} = \PP(\CC^{a_1+1}\oplus \CC^{a_2+1})
$$
with its two subspaces $\PP^{a_i} = \PP(\CC^{a_i+1})$. We choose a rational normal curve $C_i\subset \PP^{a_i}$, and we choose an isomorphism $\phi: C_1\to C_2$. We define the scroll $S(a_1, a_2)$ to be the union of the lines
$$
S(a_1,a_2) := \bigcup_{p\in C_1} \overline{p, \phi(p)}.
$$
It is not hard to prove directly that this is an algebraic variety, but as we shall soon write down its defining equations we will not bother to do so. 

From this description we can immediately deduce the dimension and degree of the scroll:
\begin{proposition}
\begin{enumerate}
\item $S(a_1,a_2)$ is a nondegenerate surface.
 \item $S(a_1,a_2)$ has degree $a_1+a_2$, and codimension $a_1+a_2-1.$
 \item Let 
\end{enumerate}

\end{proposition}\label{deg and codim}
\begin{proof}
 The rational normal curves separately span the spaces $\PP^{a_i}$, so a hyperplane containing both of them would contain $\overline{\PP^{a_1}, \PP^{a_{2}+1}} = \PP$, proving nondegeneracy. 
 
 It is clear from our description that $S$ is 2-dimensional, and thus of
codimension $a_{1}+a_{2}+1 -2 = a_{1}+a_{2}-1$. 

To compute the degree, we choose a general hyperplane $H$ containing $\PP^{a_{1}}$. The intersection $H\cap C_{2}$ consists of $a_{2}$ distinct points. Thus the intersection $H\cap S$ consists of $C_{1}$ and the $a_{2}$ lines connecting 
the points of $H\cap C_{2}$ with their corresponding points on $C_{1}$; this union obviously has degree $a_{1}+a_{2}$.
\end{proof}

A completely parallel construction creates rational normal scrolls of dimension $r$. Set $N = \sum_{i=1}^{r}(a_{i}+1)$, where each $a_{i}>0$ and
decompose $\CC^{N}$ as
$$
\CC^{\sum_{i=1}^{r}(a_{i}+1)} = \oplus_{i=1}^{r}\CC^{a_{i}+1}.
$$
Let $\PP^{a_{i}}\subset \PP^{N}$ be the subspaces corresponding to the summands,  choose
rational normal curves $C_{i}\subset \PP^{a_{i}}$ and choose isomorphisms $C_{1}\to C_{i}$. 
Set
$$
S:=S(a_{1}, \dots, a_{r}) = \bigcup_{p\in C_{1}}\overline{p, \phi_{2}(p), \dots, \phi_{r}(p)}.
$$
The variety $S$ is nondegenerate of codimension $N-r$ and degree $\sum_{i}a_{i} = N-r+1$. The proof follows exactly the same line as the one we gave for $r=2$, but with more notation.
Note the case $r=1$, in which $S(a)$ is simply the rational normal curve of degree $a$. 

To put this result in context, we recall an elementary fact of projective geometry:
 
\begin{proposition}\label{minimal degree}
 Any nondegenerate variety of codimension $c$ in $\PP^{N}$ has degree $\geq c +1$.
\end{proposition}

\begin{proof} We do induction on $c$, the case $c=0$ being trivial.
 Otherwise, assume that $c\geq1$. A general plane $L\subset \PP^{N}$ meets $X$ in $\deg X$
 distinct general points, which must be nonsingular points of $X$.
 
Let $p\in L\cap X$ be a point. If every secant to $X$ through $p$ lies entirely in $X$, then $X$ is a cone over $p$; but since $p$ was a general point, this would imply that $X$ is a plane, contradicting non-degeneracy. 

It follows that $\pi_{p}$ is a generically finite (rational) map from $X$ to $X' := \pi_{p}(X)$. The plane 
$\pi_{p}(L)$ meets $X'$ in the images of the points of $L\cap X$ other than $p$, so
$\deg X>\deg X'$. But 
$\dim X' = \dim X$, so by induction, $\deg X' \geq \codim X'+1 = \codim X$. 
\end{proof}

Thus we see that scrolls are \emph{varieties of minimal degree}. The reader already knows that the rational normal curve of degree $a$ in $\PP^{a}$ is the only curves of degree $a$ and codimension $a-1$. A celebrated Theorem of Del Pezzo (for surfaces) and Bertini (in general) generalizes this statement:

\begin{theorem}\label{classification of scrolls}
 Any nondegenerate variety of minmal degree in $\PP_{k}^{N}$, for any algebraically closed field $k$, is either a scroll or the Veronese surface in $\PP^{5}$ or a cone over one of these.
\end{theorem}

A proof will be given in an appendix to this chapter.

One interesting way to view the construction above is that we chose subvarieties $C_{i}\subset \PP^{a_{i}}$ and a one-to-one correspondence between them, that is, a subscheme
$\Gamma\subset \prod_{i}C_{i}$ that projects isomorphically onto each $C_{i}$; the scroll is then the
union of the planes spanned by sets of points $p_{i}\in C_{i}$ that are ``in correspondence''. There are other interesting varieties constructed starting with other choices of subvarieties $C_{i}$ and other subschemes---not necessarily reduced---of $\prod_{i}C_{i}$. See \cite{Eisenbud-Sammartano} for an exploration of this idea.

We tend to speak of ``the'' rational normal scroll rather than ``a'' rational normal scroll, despite the choices made in the definition, for the following reason:

\begin{proposition}\label{uniqueness of scrolls}
The scroll $S(a_1,a_2)$ is nondegenerate, and independent of the choices made in its
 definition up to a linear automorphism of $\PP^{a_1+a_2+1}$. 
\end{proposition}
\begin{proof} 
To simplify the notation, set $S := S(a_{1}, a_{2})$ and $\PP := \PP^{a_1+a_2+1}$.
To construct $S$ we chose 
\begin{enumerate}
 \item disjoint subspaces $\PP^{a_i}\subset \PP$;
 \item a rational normal curve in each subspace; and
 \item an isomorphism between these curves.
\end{enumerate}
Elementary linear algebra shows that there are automorphisms of $\PP$ carrying any choice of disjoint subspaces to any other choice. Further since the rational normal curve of degree $a$ is unique up to an automorphism of $\PP^{a}$ the choice in (2) can be undone by a linear automorphism. Finally, any automorphism of $C_{a_{2}}$ extends to an automorphism of $\PP^{a_{2}}$, and this extends to an automorphism of $\PP$ fixing $\PP^{a_{1}}$ pointwise,
showing that $S(a_{1}, a_{2})$ is independent, up to an automorphism of the ambient space, of the choice in (3)  as well.
\end{proof}


\section{1-generic matrices and the equations of scrolls (the name that's particular)}
From the definition above it is easy to find the equations of a rational normal scroll. To warm up, we look at the case of the rational normal curve, $S(a)\subset \PP^{a}$. For $a=1$ there is of course no problem, so we assume $a>1$. The idea, to be generalized later in this section, is to use the decomposition
$$
 \cO_{\PP^{1}}(1)\otimes_{\PP^{1}} \cO_{\PP^{1}}(a-1) = \cO_{\PP^{1}}(a) 
$$
which leads to the map of vector spaces
$$
 \mu: H^{0} \cO_{\PP^{1}}(1)\otimes_{\CC} H^{0}\cO_{\PP^{1}}(a-1) \to H^{0}\cO_{\PP^{1}}(a),
$$
which, in coordinates $s,t$ on $\PP^{1}$, is just the multiplication map
$$
<s,t> \otimes_{\CC}<s^{a-1}, s^{a-2}t,\dots,t^{a-1}.
$$
We may represent this map as a multiplication table, with matrix
$$
\bordermatrix{
& s^{a-1}&s^{a-2}t&\dots&t^{a-1}\cr
s&  s^{a}& s^{a-1}t&\dots&st^{a-1}\cr
t&  s^{a-1}t& s^{a-2}t^{2}&\dots&t^{a}\cr
}$$
More abstractly, what we have done is to use the equivalence between maps of vector spaces $U\otimes_{\CC}V\to W$ and maps
$W^{*}\to Hom_{\CC}(V, U^{*})$.
Because this is a multiplication table, the $2\times 2$ minors of any of the $2\times 2$ submatrices
$$
\begin{pmatrix}
s^{a-i}t^{i}& s^{a-j}t^{j}\\
s^{a-i-1}t^{i+1}& s^{a-j-1}t^{j+1}
\end{pmatrix}
$$
vanish on $\PP^{1}$. Thus, if we give the parametrization $\PP^{1}\to \PP^{a}$ of the rational normal curve  by
$$
x_{i} = s^{a-i}t^{i}
$$
we see that $2\times 2$ minors of the 
matrix
$$\leqno{(*)}\qquad
M_{a} = 
\begin{pmatrix}
 x_{0}&x_{1}&\dots&x_{a-1}\\
  x_{1}&x_{2}&\dots&x_{a}\\
\end{pmatrix}
$$
vanish on  the  curve. In fact:

\begin{proposition}\label{RNC generators} The ideal of forms vanishing on the rational normal curve in $\PP^{a}$ given parametrically by
 $x_{i} = s^{a-i}t^{i}$ is generated by the
 $2\times 2$ minors of $M_{a}$.
 \end{proposition}
 
\begin{proof}
The ideal of forms vanishing on the rational normal curve is the kernel of the map of graded rings
$$
\CC[x_{0},\dots, x_{a}] \twoheadrightarrow \CC[s^{a}, s^{a-1}t, \dots,t^{a}],
$$
and the map is homogeneous of degree 0 if we take the monomials $s^{a-i}t^{i}$ in the target to have 
degree 1. Write $I_{2}(M_{a})$ for the ideal generated by the $2\times 2$ minors of $M_{a}$. By what we have seen above, this map factors through 
$\CC[x_{0},\dots, x_{a}]/I_{2}(M_{a})$. But,
modulo the $2\times 2$ minors, any monomial in the $x_{i}$ of degree $d$ can be reduced to the form 
$$
x_{0}^{m} x_{i}^{\epsilon} x_{a}^{m-d-\epsilon}
$$
for some $i$, with $\epsilon\in \{0,1\}$ and $0\leq m\leq d+\epsilon$. To see this,
Suppose that $m$ is a monomial that cannot be so expressed, so that $m$contains $z_iz_j$ as a factor, with $1\leq i\leq j\leq a-1$. We do induction on $i$. Since
$M_a$ contains the submatrix
$$
\begin{pmatrix}
 z_{i-1} & z_{j}\\
 z_i & z_{j+1}
\end{pmatrix}
$$
we see that $z_iz_j \equiv z_{i-1}z_{j+1} \mod I$, completing the argument.

Thus the dimension of the $d$-th graded
component of $\CC[x_{0},\dots, x_{a}]/I_{2}(M_{a})$ is at most $ad+1$. The
$d$-graded component of $\CC[s^{a}, s^{a-1}t, \dots,t^{a}]$ is the $ad$-th graded component of
$\CC[s,t]$, which has dimension $ad+1$. Thus the  map 
$$
\CC[x_{0},\dots, x_{a}]/I_{2}(M_{a})\cong \CC[s^{a}, s^{a-1}t, \dots,t^{a}]
$$ 
is an isomorphism, as required.
\end{proof}

By a \emph{generalized row} of $M_{a}$, we mean a $\CC$-linear combination of the given rows of $M_{a}$. Note that the points at which the $2\times 2$ minors of $M_{a}$ vanish are the points at which the evaluations of the two rows are linearly dependent; that is, the points at which some
generalized row of $M_{a}$ vanishes identically. Given Proposition~\ref{RNC generators}, we see that the points of the rational normal curve are exactly the points where all the linear forms in some generalized row
of $M_{a}$ vanish.

The matrix $M_{a}$ has a special property: it is 1-generic in the sense below:

\begin{definition}
 A matrix of linear forms $M$ is said to be \emph{1-generic} if every generalized row of $M$
 consists of $\CC$-linearly independent forms.. 
 \end{definition}

\begin{exercise}
Show that a matrix $M$ of linear form is 1-generic iff, even after arbitrary row and column transformations, it's entries are all non-zero.
\end{exercise}
 For example, the matrix 
$$
M = \begin{pmatrix}
 x &y\\
 z&x
\end{pmatrix}
$$
over $\CC[x,y,z]$ is  1-generic, since if a row and column transformation produced a 0 the determinant would be a product of linear forms, whereas
$\det M = x^2-yz$ is irreducible. 

On the other hand, the matrix
$$
M' = \begin{pmatrix}
 x &y\\
 -y&x
\end{pmatrix}
$$
over $\CC[x,y]$ is not 1-generic, since
$$
\begin{pmatrix}
1&0\\
-i&1 
\end{pmatrix}
M'
\begin{pmatrix}
 1&0\\
 i&1
\end{pmatrix}
= 
\begin{pmatrix}
 x+iy&0\\
 0&x-iy
\end{pmatrix}
$$
(but note that it would be 1-generic if we restricted scalars to $\RR$---thus the definition depends on the field).

The fact that $M_{a}$ is 1-generic is most conveniently seen as a special case of the next proposition:

Let $X$ be
an irreducible, reduced variety, and suppose that $(\cL, V)$ is a linear series. Suppose that there are two linear series $(\cL_1,V_1),  (\cL_2,V_2)$ on $X$
such that $\cL = \cL_1\otimes\cL_2$ and $V_1\otimes V_2 \subset V$. If we
choose
bases $\{u_i\}$ of $V_1$ and $\{v_j\}$ of $V_2$ then we may regard $\ell_{i,j}:=u_iv_j\in V$
as a linear form on $\PP(V)$.

\begin{proposition}\label{some generators}
 With notation above, the $\dim V_1 \times \dim V_2$ matrix 
$M :=  (\ell_{i,j})$ is 1-generic. Moreover, its $2\times 2$ minors are contained in 
the image of $X$ under the linear series $(cL, V)$.
\end{proposition}

\begin{proof}
The entries of $M$, after any row and column operations, have the form $uv$, where
$u$ and $v$ are nonzero sections of $\cL_1$ and $\cL_2$, respectively. Since $X$ is irreducible and reduced, $u$ and $v$ can vanish only on nowhere dense subsets of $X$, so $uv$ is nonzero. This proves the first statement.

To see that the $2\times 2$ minors of $M$ vanish on $X$ we interpret all the sections of the line bundles $\cL_{1}, \cL_{2}$ and $\cL$ as elements of the 
field of rational functions on $X$, so a $2\times 2$ minor 
$$
\det
\begin{pmatrix}
 \ell_{i,j}&\ell_{i,j'} \\
 \ell_{i',j}&\ell_{i',j'}
\end{pmatrix}
= (u_iv_j)(u_{i'}v_{j'}) - (u_{i}v_{j'}) (u_{i'}v_{j})
$$
vanishes on $X$ by the associativity of multiplication.
\end{proof}

The fact that the matrix $M'$ above is \emph{not} 1-generic can be seen from a more general point of view as well:

\begin{lemma} \label{size of 1-generic}
There exist 1-generic $p\times q$ matrices of linear forms in $n+1$ variables over $\CC$ (or any algebraically closed field) if and only if $n\geq p+q$;
In particular, the dimension of the space of linear forms spanned by the $1\times 1$ minors of a  1-generic matrix $M$ of size $p\times q$ is at least $p+q-1$. Moreover, if this space of linear forms has dimension $>p+q-1$, then the restriction of $M$ to a general hyperplane is still 1-generic.
\end{lemma}

\begin{proof}
If we think of a polynomial ring $\CC[z_0,\dots,z_n]$ as the symmetric algebra
of a vector space $V$ of rank $n+1$, then we may regard a $p\times q$ matrix of
linear forms $M$ as coming from a map $m: \CC^{p}\otimes \CC^{q}\to V$. The matrix is 1-generic
if and only if no ``pure'' tensor $r\otimes s$ goes to zero, that is, iff the kernel $K$ of $m$ intersects the cone of
pure tensors only in 0. The cone of pure tensors is the cone over the Segre embedding of $\PP^{p-1}\times \PP^{q-1}$, 
an thus has dimension $(p-1)+(q-1)+1$. Thus a general subspace $K$ of codimension $\geq p+q-1$ will intersect the cone
only in 0, but any larger subspace $K$ will intersect the cone non-trivially,  and the first two statements follow.

Moreover, if $K$ is any space of codimension $>p+q-1$ that intersects the cone only in 0, then the general subspace $K'\supset K$
of dimension one larger still intersects the cone only in 0, proving the last statement.
\end{proof}

Note that the idea behind this argument is present in the usual proof of the bound in Clifford's Theorem: if $\cL$ is a special
line bundle on a curve $C$ then the map 
$$
\HH^0(\cL) \otimes \HH^0(\cL^{-1}\otimes \omega_C) \to \HH^0 (\omega_C)
$$
is 1-generic by Proposition~\ref{some generators}, and thus
$$
h^0(\cL)+h^1(\cL)-1 \leq g.
$$
By Riemann-Roch,  $h^1(\cL) = h^0(\cL) -d+g-1$, so this last relation becomes
$h^0(\cL)+(h^0(\cL) -d+g-1) -1 \leq g$, or $2(h^0(\cL)-1) \leq d$.



%Consider, for example, a rational normal curve $C$ of degree $a$ in $\PP^a$, with $a\geq 2$, 
%the image of the complete linear series $|\cL|$ with $\cL = \cO_{\PP^1}(a).$ 
%We can  write $\cL = \cL_1\otimes \cL_2$ in many ways, for example with
%$\cL_1 = \cO_{\PP^1}(1), \cL_2 = \cO_{\PP^1}(a-1)$, and taking
%$V_i = \HH^0(\cL_i)$ we have $V_1V_2 = \HH^0(\cL)$. Choosing bases
%\begin{align*}
%\{s,t\} &\subset   \HH^0(\cL_1)\\
%\{s^{a-1}, s^{a-2}t, \dots, t^{a-1}\} &\subset \HH^0(\cL_i)\\
%\end{align*}
%writing $z_i = s^{a-i}t^i$, we see that the corresponding 1-generic matrix is
%$$
%M_a:= \begin{pmatrix}
% z_0&z_1&\dots&z_{a-1}\\
% z_1&z_2&\dots&z_{a}\\
%\end{pmatrix}.
%$$

We have already seen that the ideal of minors of the 1-generic $2\times a$ matrix $M_{a}$ associated to the rational normal curve is a prime ideal of codimension $a-1$ and degree $a$. This too is part of a more general pattern:

\begin{theorem}\label{1-generic basics} Let $M$ be any 1-generic $2\times a$ matrix of linear forms on $\PP^n$, and
let $I = I_2(M)$  be the ideal generated by the $2\times 2$ minors of $M$. By Lemma~\ref{size of 1-generic} we have $a\geq n$.
\begin{enumerate}
%\item If $n=a$, and 
%$$
%M = M_a:= \begin{pmatrix}
% z_0&z_1&\dots&z_{a-1}\\
% z_1&z_2&\dots&z_{a}\\
%\end{pmatrix},
%$$
%then
%$I$ is the homogeneous ideal of a rational
%normal curve of degree $a$.

\item If $n = a$ then the 1-generic matrix
$M$ is equivalent up to row and column transformations to the matrix $M_{a}$;
in particular, $I$ is prime and $V(I)$ is a rational normal
curve of degree $a$.

\item Independently of $n$, the ideal $I$ is
prime, and the variety $V = V(I) \subset \PP^n$ has degree $a$ and codimension $a-1$.
\end{enumerate}

\end{theorem}

\begin{proof} 
We will write $S = \CC[x_{0},\dots,x_{n}]$ for the homogeneous coordinate ring of
$\PP^{n}$.

(1) The points of  $X = V(I_{2}(M))$ are the points where some generalized
row of $M$ vanish. Since the family of generalized rows is $\P^{1}$, this variety is
at most one-dimensional. If it were 0-dimensional then, since $\PP^{1}$ is irreducible
it would have to be a single point---that is, all the generalized rows would be contained
in a single vector space of linear forms of dimension $n$, contradicting Lemma~\ref{size of 1-generic}.
Thus $X$ is a curve in $\PP^{n}$, and $I_{2}(M)$ has codimension $n-1 = a-1$.

We now use an important general result from commutative algebra, Theorem~\ref{Macaulay's Theorem}, from which we see that $I_{2}(M)$ is unmixed, that, even after factoring out two ($=$ dimension $I_{2}(M)$ general linear forms $\ell_{1}, \ell_{2}$, the $2\times 2$ minors of $M$ remain linearly indedpendent.   Since the dimension of the space of quadratic forms in 
$T := S/(\ell_{1}, \ell_{2})$ in $n-1$ variables is ${n\choose 2}$,  the same as the number of minors,we see that the vector space dimension of $\overline S/(I_{2}(M)+(\ell_{1}, \ell_{2})$
is $1+(n-1) = n$; thus the curve defined (in the scheme-theoretic sense) by $I_{2}(M)$ 
has degree $n$. As it is nondegenerate in $\PP^{n}$, $X_{\rm red}$ must be the rational normal curve, and since $I_{2}(M)$ is unmixed it follows that $I_{2}(M)$ is the whole homogeneous ideal
defining the rational normal curve.

Let $\phi: \PP^{1}\to \PP^{n}$ be the parametrization of this rational normal curve, and let
$$
\begin{pmatrix}
 \ell_{0,0},\dots, \ell_{0,a-1}\\
  \ell_{1,0},\dots, \ell_{1,a-1}\\
\end{pmatrix}
$$
 Write $\overline{\ell_{i,j}}$ for the restriction of $\ell_{i,j}$ to $X \cong \PP^{1}$; we thus consider
  $\overline{\ell_{i,j}}$ as a form of degree $a$ in 2 variables. 
 Rechoosing coordinates on $\PP^{1}$, we may assume that the first row vanishes at the point $\phi(0,1)$, and the second row vanishes at $\phi(1,0)$ so that each $\overline{\ell_{0,j}}$ is divisible by $s$ and each $\overline{\ell_{1,j}}$ is divisible by $t$. Since the vector space of forms of degree $a$ divisible by $s$ has dimension $a$, we may, rechoosing coordinates on $PP^{n}$, assume that $\overline{\ell_{0,i}} = s^{a-i}t^{i}$. It follows that
each $\overline{\ell_{1,i}}$ is divisible by $t$, and that the restriction to $\PP^{1}$ of the second
row is proportional to $(t/s)(s^{a},\dots,st^{a-1}$; thus, after multiplying by a scalar, it will become
equal to $s^{a-1}t,\dots,t^{a}$; that is, the matrix $M$ is equivalent under row and column operations to the matrix $M_{a}$ defined above.

2) Now suppose that $a<n$, and let $\ell$ be a general linear form. By induction on $n$, the image of $I_{2}(M)$ in $S/\ell \cong \CC[x_{0},\dots, x_{n-1}$ is prime of codimension $a-1$, and degree $a$ so
$V(I_{2}(M))$ is irreducible of degree $a$ and codimension $a-1$. It remains to show
that $I_{2}(M)$ is prime.

It follows from the induction that the image $\overline \ell$ of $\ell$ in
$R :=  S/I_{2}(M)$  generates a prime ideal containing the
unique minimal prime $P$ of $R$. 
Every element $f\in P$ is divisible by $\ell$. If $f\in P$ is a minimal generator, then $f = \ell f'$, and
since $\ell\notin P$ we must have $f'\in P$, a contradiction. Thus $P = 0$; that is, $I_{2}(M)$ is
prime, as required.
\end{proof}

\begin{corollary}\label{equations of scrolls} Let $a_{1}, \dots a_{r}$ be positive integers, and let $N = r-1+sum_{i=1}^{r} a_{i}$.
The ideal of $S(a_{1},\dots,a_{r})\subset \PP^{N}$ is generated by the $2\times 2$ minors of the matrix
$$
\setcounter{MaxMatrixCols}{20}
M = \begin{pmatrix}
x_{1,0}&x_{1,1}&\dots&x_{1, a_{1}-1}&|&x_{2,0}&\dots&x_{2, a_{1}-1}&|&x_{r,0}&\dots&x_{r, a_{1}-1}\\
x_{1,1}&x_{1,2}&\dots&x_{1, a_{1}}.  &|&x_{2,1}&\dots&x_{2, a_{1}}&|&x_{r,1}&\dots&x_{r, a_{1}}
\end{pmatrix}
$$
\end{corollary}

\begin{proof} We may think of the matrix $M$ as consisting of $r$ blocks, $M_{a_{i}}$ of the form (*). These blocks are 1-generic by Proposition~\ref{some generators}. Since they involve distinct variables, it follows that $M$ is 1-generic. Thus by
Theorem~\ref{}, the ideal $I_{2}(M)$ is prime and of codimension $\sum a_{i}-1$, as is the ideal of the scroll. Thus it suffices to show that the minors of $M$ vanish on the scroll.

Let $C_{i}$ be the rational normal curve in the subspace $\PP^{a_{i}}\subset\PP^{N}$.
As always, the set $V(M)$ is the union of the linear spaces on which generalized rows of $M$ vanish; and each such space is the space spanned by the points in the curves $C_{a_{i}}$ corresponding to the part of that row in the block $M_{a_{i}}$---that is, $V(I_{2}(M))$ is the union of the spans of sets of corresponding points on the $C_{a_{i}}$, as required.
\end{proof}

More is true: 
\begin{fact}
 Every
 1-generic matrix of linear forms is equivalent to one of the type shown in
Corollary~\ref{equations of scrolls}, and thus the minors of any 1-generic matrix defines a scroll or the cone over a scroll.
\end{fact}

\begin{proof}[References]
A $2\times a$ matrix of linear forms in $N+1$ variables may be though of as a tensor
in $\CC^{2}\otimes \CC^{a}\otimes \CC^{N+1}$, or, equivalently, as an $a\times N+1$ matrix of linear forms in 2 variables. This, in turn is equivalent to a \emph{pencil} (that is, a projective line) in the vector space of scalar $a\times N+1$ matrices. These were first classified by Kronecker; see 
\cite[Theorems *** and ***]{Gantmacher} for a modern exposition. 
\end{proof}

In fact it will be convenient to widen the definition of scrolls to allow cones over scrolls as well: indeed, if we think of a \emph{rational normal curve of degree 0}, then the cone over $S(a_{1}, \dots, a_{r})$with $b$-dimensional  vertex is $S(0,\dots,0, a_{1}, \dots a_{r})$, where there are 
$b$ zeros in the sequence. With this in mind, we can say that giving a scroll of codimension $a-1$ is ``the same'' as giving a 1-generic $2\times a$ matrix of linear forms.

There are not many types of varieties of minimal degree:

\begin{fact}
Suppose that $X\subset \PP^{N}$ is a nondegenerate variety of minimal degree; that is, 
$\deg X - 1 = \codim X$. Then $X$ is one of the following:
\begin{itemize}
\item a quadric hypersurface;
\item a (cone over the) Veronese surface $\PP^{2}\subset \PP^{5}$; or
\item a (cone over a) rational normal scroll.
\end{itemize}
\end{fact}


We have used an important result from commutative algebra.
We say that a matrix of forms is \emph{homogeneous} if the entries $f_{i,j}$ satisfy
$$
\deg f_{i,j} + \deg f_{k,l} = \deg f_{i,l}+\deg f_{k,j} \hbox{ for all } i,j,k,l \hbox{ where this makes sense;}
$$
that is, if the determinant of each $2\times 2$ submatrix is ``naturally'' homogeneous.

\begin{fact}\label{Macaulay's Theorem} Every minimal prime over the ideal $I$ of $p\times p$ minors of a homogeneous $p\times q$ matrix forms has codimension $\leq q-p+1$. If $I$ has codimension $q-p+1$, then it is unmixed---that is, there are no embedded primes---and the $p\times p$ minors are linearly independent over the ground field.
\end{fact}

\begin{proof}[References]
\cite[ Theorem *** ]{Ei}
\end{proof}


\section{Scrolls as Images of Projective Bundles (the deep and inscrutable name)}

\fix{Need: the scroll as image of a PE bundle; cor: Pic of PE , which divisor classes move, etc.
 Talk about the singular case too.}
 
A nonsingular rational normal scroll $X := S(a_{1}, \dots, a_{r})\in \PP^{N}$ is fibered over
$\PP^{1}$ by linear subspaces $L_{\lambda} \cong \PP^{r-1}$, where $\lambda\in \PP^{1}$, which may be thought of either as the spans of sets of corresponding points on the rational normal curves
$C_{a_{i}}$, as in Section~\ref{}.

More precisely, describing the scroll as $X = V(I_{2}(M)$ for a 1-generic matrix $M$, then
\fix{to here!}

 Section~\ref{}. We may package this fact as a map $\pi:X\to \PP^{1}$, whose fiber over $\lambda$ is $L_{\lambda}$.

 If we describe the scroll by the zero Because the $L_{\lambda}$ are embedded linearly, the restriction of $\sO_{\PP^{N}}(1)$ to
$L_{\lambda} \cong \PP^{r-1}$ is $\sO_{\PP^{r-1}}$. This implies:

\begin{theorem}
With notation as above, 
$
X = \PP(\sE),
$
where $\sE =\pi_{*}\sO_{X}(1)$, embedded in $\PP^{N}$ by the complete linear series 
of the tautological line bundle $\sO_{\PP(\sE)}(1)$ 
\end{theorem}

Map to $\PP^1$ is the line bundle defined by the cokernel. Get the VB as the pushforward of $\cO(1)$.


\begin{fact} 
 Classification of minimal degree varieties.
 
 Minimal degree varieties; as the varieties of given degree lying on the maximal number of quadrics.

\end{fact}
 
  
  
\section{Degrees and genera of curves on scrolls.} 
\fix{this should probably start a new Chapter.}

Condition for the existence of integral curves in each class iff the int number with the directrix is >0 and the curve is not a multiple of the fiber. 


\section{Hyperelliptic curves}

In our early encounters with curves, we frequently assumed that the curve we were consider was non-hyperelliptic, since the behavior of hyperelliptic curves is so atypical. In this section, we'll describe the geometry of hyperelliptic curves.

\subsection{Basic models of hyperelliptic curves}\fix{move this section to ch 2; add discussion of adjoints--- perhaps as exercises?}

We start by establishing some basic facts about hyperelliptic curves. Many of these follow from general theorems like Riemann-Roch; but since they can be established by direct examination we will carry that out here.

Suppose $C$ is a smooth, projective hyperelliptic curve of genus $g \geq 2$. By definition, $C$ admits a degree 2 map $\pi : C \to \PP^1$; and as we've observed (\ref{**}) this map is unique.

By Riemann-Hurwitz, \fix{attibution?} the map $\pi : C \to \PP^1$ will have $2g+2$ distinct simple branch points, say $\lambda_1,\dots,\lambda_{2g-2} \in \PP^1$. An open subsect $C^\circ$ of $C$ can then be realized as the smooth projective completion of the affine curve given as
$$
C^\circ = \big\{ (x,y) \in \AA^2 \; \mid \; y^2 = \prod_{i=1}^{2g+2} (x - \lambda_i) \big\}.
$$ 
\fix{if two of the $\lambda_i$ coinncide, then the curve develops a singular point. Much of what we will do carries over to the singular case.} \fix{say the smooth model has 2 points at $\infty$.} Note that if we simply take the closure of this locus in $\PP^2$, the resulting curve will be highly singular at the point $[1,0,0]$, as can be seen either  directly by making an appropriate change of variables, or by invoking the genus formula for plane curves: if the closure were smooth, it would have genus $\binom{2g+1}{2}$. We can, however, complete the curve simply in $\PP^1 \times \PP^1$, for example by setting \fix{this is a rabbit from a hat. Consider either saying that by the previous section, if there's an emb in P3 then its on P1 x P1 as a divisor of type
2,g+1; and then "finding" this embedding as below; or moving this page to the early place where hyperelliptic curves are first mentioned.}
$$
y' = \frac{y}{\prod_{i=1}^{g+1} (x - \lambda_i)};
$$
we can then write the equation of a still smaller open subset of $C$ as
$$
{y'}^2 \cdot \prod_{i=1}^{g+1} (x - \lambda_i) \; = \; \prod_{i=g+2}^{2g+2} (x - \lambda_i).
$$
If we now take the closure of this locus in $\PP^1 \times \PP^1$, we get a curve of type $(2,g+1)$ on $\PP^1 \times \PP^1$; this curve is smooth, as can be seen again either directly in coordinates or by invoking the genus formula for curves on $\PP^1 \times \PP^1$. In other words,
$$
C \; = \; V\Big(Y_0^2\cdot \prod_{i=1}^{g+1} (X_1 - \lambda_iX_0) - Y_1^2 \cdot \prod_{i=g+2}^{2g+2} (X_1 - \lambda_iX_0) \Big)
$$

Next, let's describe the space of regular differentials on $C$. For this, it's convenient to work with the affine model $C^\circ = V(f) \subset \AA^2$, where
$$
f(x,y) = y^2 - \prod_{i=1}^{2g-2} (x - \lambda_i).
$$

We'll denote the two points at infinity---that is, the two points of $C \setminus C^\circ$---as $p$ and $q$.

To start, consider the simple differential $dx\in \Omega_{C^\circ/k}$. This is clearly regular on $C^\circ$, with zeros at the ramification points $r_i = (\lambda_i, 0)$. But it does not extend to a regular differential on all of $C$: it will have double poles at $p$ and $q$, as can be seen either directly or by degree considerations: as we said, $dx$ has $2g+2$ zeros, while the degree of $K_C$ is $2g-2$, meaning that there must be poles at the points $p$ and $q$.

To kill these poles, we can of course divide by $x^2$ (or any quadratic polynomial in $x$). But that just introduces new poles in the finite part $C^\circ$ of $C$. Instead, we want to multiply $dx$ by a rational function with zeros at $p$ and $q$, but \emph{whose poles occur only at the points where $dx$ has zeroes}---that is, the points $r_i$.  A natural choice is simply the reciprocal of the partial derivative $f_y = \partial f/ \partial y = 2y$, which vanishes exactly at the points $r_i$, and has correspondingly a pole of order $g+1$ at each of the points $p$ and $q$ (reason: the involution $y\to -y$ fixes $C^\circ$ and $x$), and exchanges the points $p,q$. In other words, the differential
$$
\omega = \frac{dx}{f_y}
$$
is regular, with divisor
$$
(\omega) = (g-1)p + (g-1)q.
$$
The remaining regular differentials on $C$ are now easy to find: Since $x$ has only a simple pole
at the two points at infinity \fix{say why}. we can  multiply $\omega$ by any $x^k$ with $k = 0, 1, \dots, g-1$. Since this gives us $g$ independent differentials, these  form a basis for $H^0(K_C)$.


 1) special linear series are mult $g^1_2$+basepoints. 2) Given an embedding, there's a union of lines. If the embedding is complete, we get a matrix...that defines the union of lines. Scrolls in all dimensions as unions of spans of divisors.
 
 
\subsection{General embeddings of degree
genus$+3$} 

It's a divisor on a quadric in $\PP^{3}$ of type $(2,g+1)$


\section{Trigonal curves}

\subsection{Special linear series on trigonal curves}

In analyzing special linear series on a hyperelliptic curve, we made crucial use of the facts that the canonical image of a hyperelliptic curve is a rational normal curve, and that any collection of points on a rational normal curve $C \subset \PP^n$ either are linearly independent or span $\PP^n$. In a similar (though necessarily less complete) way, we can use the fact that the canonical image of a trigonal curve lies on a rational normal surface scroll to describe special linear series on it.

 

\begin{lemma}
Let $S = S_{a,b} \subset \PP^n$ be a rational normal surface scroll. Any hyperplane section $H \cap S$ consists of the union of a rational normal curve $E$, which is a section of the scroll, and a union of lines of the ruling of the scroll.
\end{lemma}

Note that the curve $E$ must be a reduced component of $S \cap H$, but the lines $L_i$ may coincide, i.e., may be non-reduced components of the intersection. In the following proof, we'll assume for clarity that the lines $L_i$ are distinct (that is, $S \cap H$ is reduced); we leave it as an exercise to rewrite the proof to accommodate the remaining cases.

\begin{proof}
Let $F \in \Pic(S)$ be the class of a line of the ruling. Since $F^2 = 0$ and $H\cdot F = 1$, exactly one of the components of $S \cap H$ must have intersection number 1 with $F$; all other components must have intersection number 0 with $F$ and so must be lines of the ruling.

It remains to show that the unique component $E$ of $H \cap S$ having intersection number 1 with $F$ is a rational normal curve. This can be seen directly, but there's a shortcut. Suppose that we have
$$
S \cap H = E \cup L_1 + \dots + L_k,
$$
so that in particular $\deg(E) = n-1-k$. Since each of the lines $L_i$ of the ruling must meet $C$, we have that
\begin{align*}
n-1 &= \dim(\overline{S \cap H}) \\
&\leq \dim(\overline {E}) + k\\
&\leq (n-1-k) + k \\
&= n-1.
\end{align*}
We conclude that $\dim(\overline E) = n-k-1$, and hence that $E$ is a rational normal curve.
\end{proof}

Note that if $S = S_{a,b}$ with $a \leq b$, we must have either $0 \leq k \leq a$ or $k = b$: as soon as $k > a$, the span of the lines $L_i$ will contain the directrix of the scroll, and so must consist of the union of the directrix with $n-1-a = b$ lines.

Now let $C$ be a trigonal curve of genus $g \geq 5$, embedded in $\\P^{g-1}$ as a canonical curve, and let $S$ be the scroll containing $C$. We want to describe special linear series $\cD = |D|$. If our linear series has base points, we can delete them; so we'll assume that $|D|$ and $|K-D|$ are base point free. Note that this implies that   both  $r(D) \geq 1$ and $r(K-D) \geq 1$. In addition, it follows by Bertini that a general divisor $D \in \cD$ is reduced, that is, consists of distinct points $p_1,\dots,p_d$.

Now, the first hypothesis, that $r(D) \geq 1$, says that the points $p_1,\dots,p_d$ are linearly dependent. The second hypothesis, that $r(K-D) \geq 1$, says that the points $p_i$ span a subspace of codimension at least 2 in $\PP^{g-1}$ They therefore lie on at least a pencil of hyperplanes; let $H$ be a general hyperplane containing $D$.




. is effective, says that the divisor $D$ lies in a hyperplane section $C \cap H$; let $H$ be a general such hyperplane. At the same time 

canonical image lies on a 2-dim scroll (non -subcanonical embedding only on 3-dim scrolls).  embedding of a trigonal curve lies on the same scroll.Stratification of trigonal curves by Maroni invariants. Dimensions via automorphism groups of scrolls.

\section{Castelnuovo's Theorem}
(Statement only) \fix{we'll need the existence of smooth curves in given classes --- base point freeness of certain divisor classes on the scroll. Theorem: bpf iff they meet both a,b rational normal curves positively. reference to Montreal? better to make a tex file of the essential bit and put it in, as appendix. or ACGH?}

\section{Appendix: Centennial Account}
Our paper as appendix.
%footer for separate chapter files

\ifx\whole\undefined
%\makeatletter\def\@biblabel#1{#1]}\makeatother
\makeatletter \def\@biblabel#1{\ignorespaces} \makeatother
\bibliographystyle{msribib}
\bibliography{slag}

%%%% EXPLANATIONS:

% f and n
% some authors have all works collected at the end

\begingroup
%\catcode`\^\active
%if ^ is followed by 
% 1:  print f, gobble the following ^ and the next character
% 0:  print n, gobble the following ^
% any other letter: normal subscript
%\makeatletter
%\def^#1{\ifx1#1f\expandafter\@gobbletwo\else
%        \ifx0#1n\expandafter\expandafter\expandafter\@gobble
%        \else\sp{#1}\fi\fi}
%\makeatother
\let\moreadhoc\relax
\def\indexintro{%An author's cited works appear at the end of the
%author's entry; for conventions
%see the List of Citations on page~\pageref{loc}.  
%\smallbreak\noindent
%The letter `f' after a page number indicates a figure, `n' a footnote.
}
\printindex[gen]
\endgroup % end of \catcode
%requires makeindex
\end{document}
\else
\fi
