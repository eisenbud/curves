%header and footer for separate chapter files

\ifx\whole\undefined
\documentclass[12pt, leqno]{book}
\usepackage{graphicx}
\input style-for-curves.sty
\usepackage{hyperref}
\usepackage{showkeys} %This shows the labels.
%\usepackage{SLAG,msribib,local}
%\usepackage{amsmath,amscd,amsthm,amssymb,amsxtra,latexsym,epsfig,epic,graphics}
%\usepackage[matrix,arrow,curve]{xy}
%\usepackage{graphicx}
%\usepackage{diagrams}
%
%%\usepackage{amsrefs}
%%%%%%%%%%%%%%%%%%%%%%%%%%%%%%%%%%%%%%%%%%
%%\textwidth16cm
%%\textheight20cm
%%\topmargin-2cm
%\oddsidemargin.8cm
%\evensidemargin1cm
%
%%%%%%Definitions
%\input preamble.tex
%\input style-for-curves.sty
%\def\TU{{\bf U}}
%\def\AA{{\mathbb A}}
%\def\BB{{\mathbb B}}
%\def\CC{{\mathbb C}}
%\def\QQ{{\mathbb Q}}
%\def\RR{{\mathbb R}}
%\def\facet{{\bf facet}}
%\def\image{{\rm image}}
%\def\cE{{\cal E}}
%\def\cF{{\cal F}}
%\def\cG{{\cal G}}
%\def\cH{{\cal H}}
%\def\cHom{{{\cal H}om}}
%\def\h{{\rm h}}
% \def\bs{{Boij-S\"oderberg{} }}
%
%\makeatletter
%\def\Ddots{\mathinner{\mkern1mu\raise\p@
%\vbox{\kern7\p@\hbox{.}}\mkern2mu
%\raise4\p@\hbox{.}\mkern2mu\raise7\p@\hbox{.}\mkern1mu}}
%\makeatother

%%
%\pagestyle{myheadings}

%\input style-for-curves.tex
%\documentclass{cambridge7A}
%\usepackage{hatcher_revised} 
%\usepackage{3264}
   
\errorcontextlines=1000
%\usepackage{makeidx}
\let\see\relax
\usepackage{makeidx}
\makeindex
% \index{word} in the doc; \index{variety!algebraic} gives variety, algebraic
% PUT a % after each \index{***}

\overfullrule=5pt
\catcode`\@\active
\def@{\mskip1.5mu} %produce a small space in math with an @

\title{Personalities of Curves}
\author{\copyright David Eisenbud and Joe Harris}
%%\includeonly{%
%0-intro,01-ChowRingDogma,02-FirstExamples,03-Grassmannians,04-GeneralGrassmannians
%,05-VectorBundlesAndChernClasses,06-LinesOnHypersurfaces,07-SingularElementsOfLinearSeries,
%08-ParameterSpaces,
%bib
%}

\date{\today}
%%\date{}
%\title{Curves}
%%{\normalsize ***Preliminary Version***}} 
%\author{David Eisenbud and Joe Harris }
%
%\begin{document}

\begin{document}
\maketitle

\pagenumbering{roman}
\setcounter{page}{5}
%\begin{5}
%\end{5}
\pagenumbering{arabic}
\tableofcontents
\fi


\chapter{Hilbert Schemes}
\label{HilbertSchemesChapter}

In Chapter~\ref{}, we looked at curves of low genus and described the linear systems on them; that is, their maps to (and in particular their embeddings in) projective space. We'd now like to revisit this topic, but now with a different question: can we describe the family of all such curves in projective space?

To set this up, let's start with some notation and conventions. To start with, we'll limit ourselves to looking at curves in $\PP^3$. We'll denote by $\cH = \cH_{dm-g+1}(\PP^3)$ the Hilbert scheme parametrizing subschemes of $\PP^3$ with Hilbert polynomial $p(m) = dm-g+1$ (which includes
curves of degree $d$ and genus $g$ in $\PP^3$), and by $\cH^\circ \subset \cH$ the open subset parametrizing smooth, irreducible, nondegenerate curves $C \subset \PP^3$. 

We're going to ask two basic questions about the schemes $\cH^\circ$:

\begin{enumerate}
\item[$\bullet$] Is $\cH^\circ$ irreducible? and
\item[$\bullet$]  What is its dimension or dimensions?
\end{enumerate}

Of course, there are many more questions we can ask about the geometry of $\cH^\circ$: for example, can we say where it is smooth or singular? Can we characterize the closure $\overline{\cH^\circ} \subset \cH$ in the whole Hilbert scheme? (In other words, can we say when a subscheme $X \subset \PP^3$ with Hilbert polynomial $dm-g+1$ is \emph{smoothable}, that is, the flat limit of a family of smooth curves?) And can we determine the Picard groups of $\cH^\circ$ and its closure? We will for the most part not address these, though we will indicate the answers in special cases.

We will proceed systematically, starting with curves of the lowest possible degree and going on to successively higher degrees.

\section{Degree 3}

The smallest possible degree of an irreducible, nondegenerate curve $C \subset \PP^3$ is 3, so we'll start there. We also know that any irreducible, nondegenerate curve $C \subset \PP^3$ is a twisted cubic, so that in this case $\cH^\circ$ is simply the parameter space for twisted cubics.

\begin{proposition}\label{hilb of twisted cubics}
The open subset $\cH^\circ$ of the Hilbert scheme $\cH_{3m+1}$ parametrizing twisted cubics is irreducible of dimension 12.
\end{proposition}

\begin{proof}  There are in fact several ways of establishing this statement. To start with the simplest, let $C_0 \subset \PP^3$ be any given twisted cubic, and consider the family of translates of $C_0$ by automorphisms $A \in \PGL_4$ of $\PP^3$: that is, the family
$$
\cC = \{ (A, p) \in \PGL_4 \times \PP^3 \; \mid \; p \in A(C_0) \}.
$$
Via the projection $\pi : \cC \to \PGL_4$, this is a family of twisted cubics, and so it induces a map
$$
\phi : \PGL_4 \to \cH^\circ.
$$
Since every twisted cubic is a translate of $C_0$, this is surjective, with fibers isomorphic to the stabilizer of $C_0$, that is, the subgroup of $\PGL_4$ of automorphisms of $\PP^3$ carrying $C_0$ to itself. Since this group is isomorphic to $\PGL_2$ and has dimension 3, and since $\PGL_4$ is irreducible of dimension 15, we conclude that \emph{$\cH^\circ$ is irreducible of dimension 12}.
\end{proof}


This argument is based on a rather special fact, that all irreducible nondegenerate cubic curves $C \subset \PP^3$ are translates of one another. There is another, less ad-hoc way of arriving at the conclusion above which we'll now describe. While it seem much more involved, as we'll see in the next couple of sections it's a broadly applicable technique in general.

\begin{proof}[Second Proof] The idea behind this approach is the fact the intersection of any two distinct quadrics $Q, Q' \supset C$ containing a twisted cubic curve $C$ is the union of $C$ and a line $L \subset \PP^3$, which follows from B\' ezout's theorem. 
Conversely, suppose that $L \subset \PP^3$ is any line and  $Q, Q'$ two general quadrics containing $L$; write the intersection $Q \cap Q'$ as a union $L \cup C$. By Bertini, $Q$ is smooth, and the quadric $Q'$ will intersect it in a curve of type $(2,2)$, and so the curve $C$ will have class $(2,1)$ or $(1,2)$. Since the quadrics $Q'$ containing $L$ cut out on $Q$ the complete linear system of curves of type $(2,1)$, which has no base locus, Bertini's theorem tells us that $C$ will be smooth, so that the intersection $Q \cap Q' = L \cup C$ will be the union of $L$ and a twisted cubic. This suggests that we set up an incidence correspondence: let $\PP^9$ denote the projective space of quadrics in $\PP^3$, and consider
$$
\Phi = \{ (C, L, Q, Q') \in \cH^\circ \times \GG(1,3) \times \PP^9 \times \PP^9 \; \mid \; Q \cap Q' = C \cup L \}.
$$

We'll analyze $\Phi$ by considering the projection maps to $\cH^\circ$ and $\GG(1,3)$; that is, by looking at the diagram

\begin{diagram}
& &  \Phi & & \\
& \ldTo^{\pi_1} & & \rdTo^{\pi_2} & \\
\cH^\circ & & & & \GG(1,3)
\end{diagram}

Consider first the projection map $\pi_2 : \Phi \to \GG(1,3)$ on the second factor. By what we just said, the fiber over any point $L \in \GG(1,3)$ is an open subset of $\PP^6 \times \PP^6$, where $\PP^6$ is the space of quadrics containing $L$; it follows that $\Phi$ is irreducible of dimension $4 + 2\times 6 = 16$. Going down the other side, we see that the map $\pi_1 : \Phi \to \cH^\circ$ is surjective, with fibers open subsets of $\PP^2 \times \PP^2$; we conclude again that \emph{$\cH^\circ$ is irreducible of dimension 12}.
\end{proof}

Yet another proof of Proposition~\ref{hilb of twisted cubics} is based on a remarkable fact about twisted cubics, described in the next exercise; the application to $\cH^\circ$ is carried out in the following one.

\begin{exercise}
Show that if $p_1,\dots, p_6 \in \PP^3$ are any six points in $\PP^3$ in \emph{linear general position}, that is, with no four lying in a plane, then there exists a unique twisted cubic curve $C \subset \PP^3$ containing them.
\end{exercise}

\begin{exercise}
Consider the incidence correspondence
$$
\Phi = \{ (p_1,\dots,p_6, C) \in (\PP^3)^6 \times \cH^\circ \; \mid p_1,\dots,p_6 \in C  \}.
$$
Use the result of the preceding problem to show that $\cH^\circ$ is irreducible of dimension 12.
\end{exercise}


Note that $\cH^\circ$ is open in the Hilbert scheme $\cH = \cH_{3m+1}(\PP^3)$, but its closure is not all of $\cH$! There is a second irreducible component of $\cH$, of dimension 15. To see this, observe that any plane cubic $C \subset \PP^2 \subset \PP^3$ has Hilbert polynomial $p(m) = 3m$. If $p \in \PP^3 \setminus C$ is any point not on $C$, then, the union $C \cup \{p\}$ is a subscheme of $\PP^3$ with Hilbert polynomial $3m+1$, and so corresponds to a point of $\cH$. But the family of such subschemes has dimension 15: we have to specify a plane in $\PP^3$ (3 parameters), a cubic curve $C$ in that plane (9 parameters) and a point $p \in \PP^4$ (3 parameters). In fact, these schemes are dense in a second irreducible component $\cH'$ of $\cH$.

In general, the Hilbert scheme $\cH_{dm-g+1}(\PP^3)$ will have many components (we don't know in general how many, or what their dimensions are), few of which actually parametrize reduced, irreducible and nondegenerate curves in $\PP^3$. This is why, for the most part, we'll be restricting our attention to the closure of $\cH^\circ$. 

\begin{exercise}
Describe the intersection of the closure  of $\cH^\circ$ in $\cH_{3m+1}$ with the second component described above. In particular, show that the locus $\Sigma$ of schemes $X$ consisting of a nodal plane cubic curve $C$ with a spatial embedded point of multiplicity 1 at the node is dense in the intersection $\overline{\cH^\circ} \cap \cH'$.
\end{exercise}

\section{Linkage} \label{SLinkage}

As the second proof of Proposition~\ref{hilb of twisted cubics} suggests, when the union of two curves $C$ and $D$ forms a complete intersection we can use this fact to relate the geometry of their respective Hilbert schemes. This is a technique we'll use repeatedly in this chapter. One thing we need in order to apply it in general is a formula relating the genera of the curves $C$ and $D$, which we'll derive in this section. This is one aspect of the general theory of \emph{liaison}, or \emph{linkage}, of curves in $\PP^3$; we'll discuss this theory in detail in Chapter~\ref{SyzygiesChapter}.

In our present situation, we assume $C$ and $D \subset \PP^3$ are curves of degrees $d$ and $e$ and genera $g$ and $h$ respectively, with no common components. We assume also that the union $C \cup D$ is a complete intersection $S \cap T$ of surfaces of degrees $s$ and $t$, with $S$ smooth. In this situation, B\'ezout tells us that $d+e = st$; we want a formula similarly relating the genera $g$ and $h$ of $C$ and $D$.

To do this, we work in the Chow ring of $S$. (This is why we assume $S$ is smooth, although the treatment in Section~\ref{} will show that this hypothesis is unnecessary.) We know that the canonical bundle $K_S = \cO_S(s-4)$, so that by adjunction 
$$
2g-2 = (C\cdot C) + (K_S\cdot C) = C\cdot C + (s-4)d, 
$$
or in other words,
$$
(C \cdot C) = 2g-2 - (s-4)d.
$$
Next, since $C \cup D$ is a complete intersection of $S$ with a surface of degree $t$, we have $C + D\sim tH$, where $H$ denotes the hyperplane class on $S$; thus we have
$$
(C \cdot D) = (C \cdot (tH - C)) = td - (C \cdot C) = td - 2g + 2 + (s-4)d
$$
and similarly
$$
(D \cdot D) = (D \cdot (tH - C)) = te - td + 2g - 2 - (s-4)d.
$$
Finally, we can apply the adjunction formula to $D$ to arrive at
$$
2h - 2 = (D \cdot D) + (K_S \cdot D) = (s-4)e  + te - td + 2g - 2 - (s-4)d.
$$
Collecting terms, we can write this in the convenient form
\begin{equation}\label{linked genus formula}
h - g = (e-d)\frac{s+t-4}{2};
\end{equation}
in words, \emph{the difference between the genera of $C$ and $D$ is proportional to the difference in their degrees, with constant of proportionality $(s+t-4)/2$}. 

We will see this formula used repeatedly in this chapter, and as we indicated it will be discussed as part of the larger theory of liaison for space curves in Chapter~\ref{}. For now, you should just take a moment and reassure yourself that the right hand side of~(\ref{linked genus formula}) is indeed an integer!

%We will generalize the technique used in second proof of Proposition~\ref{hilb of twisted cubics}. Will do first for $C, D$ smooth; at the end, have discussion of full definition and theorem.
%
%To be done: First, derive formula for genus of linked curves via intersection theory of a (presumably smooth) surface containing the curves. Similarly introduce the Rao module and show that Rao modules of linked curves are dual (with a twist) via cohomology groups of line bundles on this surface and Serre duality. Unirationality of one Hilbert scheme equiv. to unirationality of the other; discussion of historical significance and ref. to result that general curves are not linked to anything simpler.
%
%References to Chapter 10, where the 
%
%Question: Is the Hilbert scheme of twisted cubics rational?
%
%\subsection{formulas for degree and genus of linked curves}
%
%\subsection{Cheerful facts: Hartshorne-Rao; smoothness of residual curve when $I_C$ generated in degree $d$ (Prop. 5.6 in 3264)}
%
%\subsection{Unirationality of Hilbert schemes}

\section{Degree 4}

Let's move on to curves $C \subset \PP^3$ of degree 4 (always assumed smooth, irreducible and nondegenerate). The first thing to observe here is that by Clifford such a curve must have genus 0 or 1; we consider these cases in turn.

\subsection{Genus 0}\label{degree 4 genus 0}

We can deal with rational quartics by a slight variant of the first method we used to deal with twisted cubics; in fact, this method will answer our question for rational curves of any degree. Specifically, a rational curve of degree 4 is the image of a map $\phi_F : \PP^1 \to \PP^3$ given by a four-tuple $F = (F_0,F_1,F_2,F_3)$ with $F_i \in H^0(\cO_{\PP^1}(4))$. The space of all such four-tuples up to scalars is a projective space of dimension $4 \times 5 - 1 = 19$; let $U \subset \PP^{19}$ be the open subset of four-tuples such that the map $\phi$ is a nondegenerate embedding. We then have a surjective map $\pi : U \to \cH^\circ$, whose fiber over a point $C$ is the space of maps with image $C$. Since any two such maps differ by an automorphism of $\PP^1$---that is, an element of $\PGL_2$---the fibers of $\pi$ are three-dimensional; we conclude that \emph{$\cH^\circ_{4m+1}$ is irreducible of dimension 16}.

As we said, the same analysis can be used on rational curves of any degree $d$: the space $U$ of nondegenerate embeddings $\PP^1 \to \PP^3$ of degree $d$ is an open subset of the projective space $\PP^{4(d+1)-1}$ of four-tuples of homogeneous polynomials of degree $d$ on $\PP^1$ modulo scalars; and the fibers of the corresponding map $U \to \cH^\circ_{dm+1}$ are copies of $\PGL_2$. This yields the

\begin{proposition}\label{dimension of rational curves}
The open set $\cH^0 \subset \cH_{dm+1}$ parametrizing smooth, irreducible nondegenerate rational curves $C \subset \PP^3$ is irreducible of dimension $4d$.
\end{proposition}

\begin{exercise}
Just to get some practice with the method of linkage, give an argument for Proposition~\ref{dimension of rational curves} in case $d=4$ along the lines of the second argument for twisted cubics.
\end{exercise}

\subsection{Genus 1}

What about genus 1? In fact, this is relatively simple: as we saw in Section~\ref{}, a quartic curve $C \subset \PP^3$ of genus 1 is the intersection of two quadric surfaces, and by the Noether-Lasker theorem every quadric containing $C$ is a linear combination of those two. Conversely, the intersection of two general quadrics in $\PP^3$ is a quartic curve of genus 1. The space of such curves is thus an open subset of the Grassmannian $G(2,10) = \GG(1,9)$, and we conclude that \emph{$\cH^\circ_{4m}$ is irreducible of dimension 16}.

\section{Degree 5}

Let $C \subset \PP^3$ be a smooth, irreducible, nondegenerate quintic curve of genus $g$. To start with, we can use Clifford plus Riemann-Roch to bound the genus of $C$: by Clifford, the bundle $\cO_C(1)$ must be nonspecial, and then by Riemann-Roch we must have $g \leq 2$.

Now, we have already seen that the space $\cH^\circ_{5m+1}$ of rational quintic curves is irreducible of dimension 20. We'll consider, accordingly, the two remaining cases, $g=1$ and $g=2$.

\subsection{Genus 2}

We start with genus 2, since this is a case we've considered already in Section~\ref{}.  To recap the analysis, let $C \subset \PP^3$ be a smooth, irreducible, nondegenerate curve of degree 5 and genus 2. Since $h^0(\cO_C(2)) = 10-2+1 = 9$ by Riemann-Roch, the restriction map
$$
H^0(\cO_{\PP^3}(2)) \to H^0(\cO_C(2))
$$
must have a kernel; by B\'ezout, this kernel must be one-dimensional, that is, $C$ lies on a unique quadric surface $Q$. Similarly, the restriction map
$$
H^0(\cO_{\PP^3}(3)) \to H^0(\cO_C(3))
$$
must have at least a 6-dimensional kernel; since at most 4 of these are of the form $LQ$ for $L$ a linear form, we see that $C$ lies on a cubic surface not containing $Q$. Thus we can say that $C$ is residual to a line in the complete intersection of a quadric and a cubic surface, or, if $Q$ is smooth, in terms of the isomorphism $Q \cong \PP^1 \times \PP^1$ we can say $C$ is a curve of type $(2,3)$ on the quadric $Q$. Note that conversely if $L \subset \PP^3$ is a line and $Q$ and $S \subset \PP^3$ are general quadric and cubic surfaces containing $L$, and if we write
$$
Q \cap S = L \cup C
$$ 
then the curve $C$ is a curve of type $(2,3)$ on the quadric $Q$ and hence a quintic of genus 2.

This suggests two ways of describing the family $\cH^\circ$ of all such curves. First, we can use the fact that $C$ is linked to a line in much the same way we did in the case of twisted cubics: we set up an incidence correspondence
$$
\Psi = \{ (C, L, Q, S) \in \cH^\circ \times \GG(1,3) \times \PP^9 \times \PP^{19} \; \mid \; Q \cap S = C \cup L \},
$$
where the $\PP^9$ (respectively, $\PP^{19}$) is the space of quadric (respectively, cubic) surfaces in $\PP^3$. Given a line $L \in \GG(1,3)$, the space of quadrics containing $L$ is a $\PP^6$, and the space of cubics containing $L$ is a $\PP^{15}$; thus the fiber of the projection $\pi_2 : \Psi \to \GG(1,3)$ over $L$ is an open subset of $\PP^6 \times \PP^{15}$, and we see that \emph{$\Psi$ is irreducible of dimension $4 + 6 + 15 = 25$}.

Going back down the other way, the fiber of $\Psi$ over a point $C \in \cH^\circ$ is an open subset of the $\PP^5$ of cubics containing $C$; and we conclude that \emph{$\cH^\circ$ is irreducible of dimension $20$}.

Another, in some ways more direct, approach would be to use the fact that the quadric surface $Q$ containing a quintic curve $C \subset \PP^3$ of genus 2 is unique. We thus have a map
$$
\cH^\circ \to \PP^9,
$$
whose fiber over a point $Q \in \PP^9$ is the space of quintic curves of genus 2 on $Q$. 

The problem is, the space of quintic curves of genus 2 on a given quadric $Q$ is not in general irreducible: if $Q$ is smooth, it consists of the disjoint union of two $\PP^{11}$s (or rather open subsets of these $\PP^{11}$s), corresponding to the families of curves of type $(2,3)$ and $(3,2)$. Thus we can conclude immediately that $\cH^\circ$ is of pure dimension 20; but to conclude that it's irreducible we need to verify that, in the family of all smooth quadric surfaces, the monodromy exchanges the two rulings. This is not hard: it amounts to the assertion that the family
$$
\Gamma = \{ (Q,L) \in \PP^9 \times \GG(1,3) \; \mid \; L \subset Q \}
$$
is irreducible, which can be seen readily via projection on the second factor.

There is another approach to the problem of describing $\cH^\circ$, which is to describe such curves parametrically rather than via the equations defining them as subsets of $\PP^3$, which is a direct generalization of the approach we took to the proof of Proposition~\ref{dimension of rational curves} above. We'll describe this in general in Section~\ref{estimating dim hilb}. It covers the case of quintics of genus 1, so we won't deal with that case separately, except in the form of an exercise:

\begin{exercise}
Show that a smooth, irreducible, nondegenerate curve $C \subset \PP^3$ of degree 5 and genus 1 is residual to a rational quartic in the complete intersection of two cubics, and use the result of subsection~\ref{degree 4 genus 0} to deduce that the space of genus 1 quintics is irreducible of dimension 20.
\end{exercise}

\section{Degree 6}

Moving on to curves of degree 6, the first question to ask would be: what are the possible genera of smooth, irreducible, nondegenerate curves $C \subset \PP^3$ of degree 6? Once more (and for the last time, we're afraid) Clifford and Riemann-Roch suffice to answer this: if the line bundle $\cO_C(1)$ is nonspecial, then by Riemann-Roch we have $g \leq 3$; and if $\cO_C(1)$ is special, $d \leq 2g-2$ and hence $g \leq 4$.

The cases of general 0, 1 and 2 are covered under Proposition~\ref{nonspecial Hilbert}, leaving us the cases $g = 3$ and 4. Both are well-handled by the Cartesian approach of describing their ideals; we'll start with the relatively simple case of $g=4$.

\subsection{Genus 4}

This is easy: by Riemann-Roch, a curve of degree 6 and genus 4 in $\PP^3$ is necessarily a canonical curve, and as we've seen a canonical curve of genus 4 is the complete intersection of a (unique) quadric $Q$ and a cubic surface $S$. We thus have a map
$$
\alpha : \cH^\circ \rTo \PP^9
$$
sending a curve $C$ to the quadric $Q$ containing it. Moreover, the fibers of this map are open subsets of the projective space $\PP V$, where $V$ is the quotient
$$
V = \frac{H^0(\cO_{\PP^3}(3))}{H^0(\cI_{Q/\PP^3}(3))}
$$
of the space of all cubic polynomials modulo cubics vanishing on $Q$. Since this vector space has dimension 16, the fibers of $\alpha$ are irreducible of dimension 15, and we deduce that \emph{the space $\cH^\circ_{6m-3}$ is irreducible of dimension 24}.

\subsection{Genus 3}

\begin{exercise}
Let $C$ be a curve of degree 6 and genus 3, and assume that $C$ does not lie on any quadric surface. Show that $C$ is residual to a twisted cubic in the complete intersection of two cubic surfaces, and use this to deduce that the space of such curves is irreducible of dimension 24.
\end{exercise}


\begin{exercise}
Now let $C$ again be a curve of degree 6 and genus 3, but now assume that $C$ \emph{does} lie on a quadric surface $Q$. Show that such a curve is a specialization/flat limit of curves of the type described in the last exercise, and conclude that $\cH^\circ_{6m-2}(\PP^3)$ is irreducible of dimension 24.
\end{exercise}


\begin{exercise}
For another approach, try tweaking the argument for Proposition~\ref{nonspecial Hilbert} to cover the case $d=2g$, and use this to deduce again that $\cH^\circ_{6m-2}(\PP^3)$ is irreducible of dimension 24.
\end{exercise}

\section{Why  $4d$?}\label{estimating dim hilb}

The sharp-eyed reader will have noticed that, in every case analyzed so far, the dimension of the Hilbert scheme parametrizing smooth curves of degree $d$ and genus $g$ in $\PP^3$ is $4d$. While this is not the case in general (we will see shortly an example where it fails), the persistence of $4d$ suggests that there may be a way of estimating the dimension of the Hilbert scheme that yields, in the case of curves in $\PP^3$, the ``expected" dimension $4d$.

In fact, there are two, and in the following subsections we'll describe both approaches. For the remainder of this section, we will step outside $\PP^3$ and consider the restricted Hilbert scheme $\cH^\circ$ of smooth, irreducible, nondegenerate curves in $\PP^r$.

\subsection{Estimating $\dim \cH^\circ$ by Brill-Noether}

Our first approach to the problem of estimating the dimension of $\cH^\circ$ is a direct generalization of the approach we took to the proof of Proposition~\ref{dimension of rational curves} above, with two additional wrinkles, which involve invoking the existence of  spaces $M_g$ parametrizing abstract curves of genus $g$ and $\Pic_d(C)$ parametrizing line bundles of degree $d$ on a given curve $C$.

To set this up, again let $\cH^\circ$ be the space of smooth, irreducible, nondegenerate curves $C \subset \PP^3$ of degree 5 and genus 2. By forgetting the information of the particular embedding of $C$ in $\PP^3$ and remembering only the isomorphism class of $C$, we have a map
$$
\mu : \cH^\circ \rTo M_2.
$$
What does the fiber $\Sigma_C =\mu^{-1}(C)$ of the map $\mu$ over a point $C \in M_2$ look like? To start, we have a map
$$
\nu : \Sigma_C \rTo \Pic_5(C),
$$
obtained by sending a point in $\Sigma_C$ to the line bundle $\cO_C(1)$. Moreover, by Proposition~\ref{**}, which says that any line bundle of degree 5 on a curve of genus 2 is very ample, this map is surjective. Finally, the fiber of $\nu$ over a point $\cL \in \Pic_5(C)$ is simple to describe: once we've specified the abstract curve $C$, and the line bundle $\cL \in \Pic_5(C)$ giving the embedding, since $h^0(\cL) = 4$ all we have to do is to choose a basis for $H^0(\cL)$ up to scalars. In other words, the fiber of $\nu$ are isomorphic to $\PGL_4$. We can now work our way up from $M_2$:

\begin{enumerate}

\item[$\bullet$] We know (or at least have asserted) that $M_2$ is irreducible of dimension 3.

\item[$\bullet$] It follows that the space of pairs $(C,\cL)$ with $C \in M_2$ a smooth curve of genus 2 and $\cL \in \Pic_5(C)$ is irreducible of dimension 3 + 2 = 5; and finally

\item[$\bullet$] It follows that $\cH^\circ$ is irreducible of dimension $5 + 15 = 20$.

\end{enumerate}

In fact, this approach applies to a much wider range of examples: whenever $d \geq 2g+1$ and $r \leq d-g$, we can look at the tower of spaces

\begin{diagram}
\cH^\circ = \cH^\circ_{dm-g+1}(\PP^r) \\
\dTo \\
\cP_{d,g} = \{(C,\cL) \mid \cL \in \Pic_d(C) \} \\
\dTo \\
M_g.
\end{diagram}

Exactly as in the special case $(d,g,r) = (5,2,3)$ above, we can work our way up the tower:


\begin{enumerate}

\item[$\bullet$]  $M_g$ is irreducible of dimension $3g-3$;

\item[$\bullet$] it follows that $\cP_{d,g}$ is irreducible of dimension $3g-3+g = 4g-3$; and finally

\item[$\bullet$] since the fibers of $\cH^\circ \to \cP_{d,g}$ consist of $(r+1)$-tuples of linearly independent sections of $\cL$ (mod scalars), it follows that $\cH^\circ$ is irreducible of dimension $4g-3 + (r+1)(d-g+1) - 1$.

\end{enumerate}

In sum, we have the

\begin{proposition}\label{nonspecial Hilbert}
Whenever $d \geq 2g+1$, the space $\cH^\circ$ of smooth, irreducible, nondegenerate curves $C \subset \PP^r$ is either empty (if $d-g < r$) or irreducible of dimension $4g-3 + (r+1)(d-g+1) - 1$.
\end{proposition}

Note that in case $r=3$, this reduces to $4d$. Note also that we can modify this analysis to extend this beyond the range $d \geq 2g+1$: as long as the Brill-Noether number $\rho(d,g,r) \geq 0$, the Brill-Noether theorem tells us that for a general curve $C$, the variety $W^r_d(C)$ has dimension $\rho$, and (assuming $r \geq 3$) the general point of $W^r_d(C)$ corresponds to a very ample line bundle wirth exactly $r+1$ sections. In this situation, there is a unique component of $\cH_0 \subset \cH^\circ$ dominating $M_g$, and the map $\cH^\circ \to \cP_{d,g}$ carries this component to a subvariety $\cW^r_d \subset \cP_{d,g}$ of dimension $3g-3 + \rho$; we have then
$$
\dim \cH_0 = 3g-3+\rho + (r+1)^2 - 1 = 4g-3 + (r+1)(d-g+1) - 1,
$$
extending the calculation of Proposition~\ref{nonspecial Hilbert}. The component $\cH_0$ is called the \emph{principal component} of the Hilbert scheme; there may be others as well, of possibly different dimension, and we do not know precisely for which $d,g$ and $r$ these occur.

\subsection{Estimating $\dim \cH^\circ$ by Euler characteristic of the normal bundle}

It's interesting to compare the estimate of  $\dim \cH^\circ$ above with what we get via a completely different approach. To set this up, let $\cH$ be a component of the scheme $\cH^\circ$, with $C \subset \PP^r$ a curve corresponding to a general point $[C]$ of $\cH$.

We start with the idea that the dimension of the scheme $\cH$ is approximated by the dimension of its Zariski tangent space $T_{[C]}\cH$ at a general point $[C]$. But we've seen that the tangent space to $\cH$ at $[C]$ is the space $H^0(\cN_{C/\PP^r})$ of global sections of the normal bundle $\cN = \cN_{C/\PP^r}$. And we can think of the dimension $h^0(\cN)$ as approximated by the Euler characteristic $\chi(\cN)$, with ``error term" $h^1(\cN)$ coming from its first cohomology group.

Given these two approximations, we arrive at a number we can actually compute! From the exact sequence
$$
0 \to T_C \to T_{\PP^r}|_C \to \cN \to 0
$$
we deduce that
\begin{align*}
c_1(\cN) &= c_1(T_{\PP^r}|_C) - c_1(T_C) \\
&= (r+1)d - (2-2g).
\end{align*}

Now we can apply the Riemann-Roch formula for vector bundles on curves (\cite{3264}) to conclude that
\begin{align*}
\chi(\cN) &= c_1(\cN) - \rank(\cN)(g-1) \\
&= (r+1)d - (r-3)(g-1).
\end{align*}

Note that our two ``estimates" are actually inequalities. But, unfortunately, they go in opposite directions: we have
$$
\dim \cH \leq \dim T_{[C]}\cH,
$$
but 
$$
\dim T_{[C]}\cH \geq \chi(\cN).
$$


\subsection{They're the same!} Proposition~\ref{nonspecial Hilbert} suggests that the ``expected dimension" of the restricted Hilbert scheme $\cH^\circ$ of curves of degree $d$ and genus $g$ in $\PP^r$ should be 
$$
h(g,r,d) = 4g-3 + (r+1)(d-g+1) - 1.
$$
But the calculation immediately above suggests it should be $(r+1)d - (r-3)(g-1)$. Which is it? The answer is both: they're the same number!

\section{Degree 8}

At this point, the reader may be growing impatient: in every case dealt with so far, the space $\cH^\circ_{dm-g+1}$ is irreducible of dimension $4d$. Is that the case in general?

The answer is very much no: in general, neither the number of irreducible components of $\cH^\circ_{dm-g+1}$, nor their dimensions, are known. In this section and the next we'll give the first two examples where $\cH^\circ_{dm-g+1}$ is either reducible, or of dimension $>4d$.

The first example occurs in degree 8: specifically, we consider the space $\cH^\circ = \cH^\circ_{8m-8}(\PP^3)$ of smooth, irreducible, nondegenerate curves of degree 8 and genus 9. To describe such curves $C \subset \PP^3$, we look first at the restriction map
$$
\rho_2 : H^0(\cO_{\PP^3}(2)) \rTo H^0(\cO_C(2)).
$$
The space on the left has dimension 10, as always. As for the space on the right, Riemann-Roch is ambivalent: it says that
\begin{align*}
h^0(\cO_C(2)) =
\begin{cases}
9, \quad &\text{if } \cO_C(2) \cong K_C; \\
8,  \quad &\text{if } \cO_C(2) \not\cong K_C.
\end{cases}
\end{align*}

In the latter case, $C$ would have to lie on two distinct quadrics, which would violate B\'ezout; we deduce that we must have $\cO_C(2) \cong K_C$, and hence $C$ lies on a unique quadric surface $Q$.

What about cubics and higher degree surfaces? For cubics, there is no question: since $C$ lies on a (necessarily irreducible) quadric $Q$, by B\'ezout it cannot lie on any cubic not containing $Q$. Moving on to quartics, we look again at the restriction map
$$
\rho_4 : H^0(\cO_{\PP^3}(4)) \rTo H^0(\cO_C(4)).
$$
The dimensions here are, respectively, 35 and $4\cdot 8 - 9 + 1 = 24$; and we deduce that $C$ lies on at least an 11-dimensional vector space of quartic surfaces. On the other hand, only a 10-dimensional vector subspace of these consist of quartics vanishing on Q; and so we conclude that \emph{$C$ lies on a quartic surface not containing $Q$}. Next, B\'ezout tells us that we must have $C = Q \cap S$, and the Noether-Lasker theorem tells us that the equations of $Q$ and $S$ generate the homogeneous ideal of $C$; that is, $\ker(\rho_4)$ has dimension exactly 11, and so $S$ is unique modulo quartics vanishing on $Q$.

What can we say about the space $\cH^\circ$ parametrizing such curves? Well, we have a natural map $\cH^\circ \to \PP^9$ with dense image; and the fibers, by what we've just said, is an open subset of the projective space $\PP V$, where $V$ is the 25-dimensional vector space
$$
V = \frac{H^0(\cO_{\PP^3}(4))}{H^0(\cI_{Q/\PP^3}(4))}.
$$
It follows that \emph{the space $\cH^\circ_{8m-8}(\PP^3)$ is irreducible of dimension 33}---one larger than the expected $4d$.


\section{Degree 9}

For the next example, consider the space $\cH^\circ = \cH^\circ_{9m-9}(\PP^3)$ of curves of degree 9 and genus 10. Once more, to describe such curves we look to the restriction maps $\rho_m$; and again, we have an ambiguity coming from Riemann-Roch, which tells us that
\begin{align*}
h^0(\cO_C(2)) =
\begin{cases}
10, \quad &\text{if } \cO_C(2) \cong K_C \; \text{(``the first case,") and } \\
9,  \quad &\text{if } \cO_C(2) \not\cong K_C  \; \text{(``the second case.")}
\end{cases}
\end{align*}
Unlike the last case, however, both are possible; we'll analyze each of these cases in turn.

1. Suppose first that $C$ does not lie on any quadric surface (so that we are necessarily in the first case above). We consider next the map $\rho_3 : H^0(\cO_{\PP^3}(3)) \to H^0(\cO_C(3))$. By Riemann-Roch, the dimension of the target is $3\cdot 9 - 10 + 1 = 18$, from which we conclude that $C$ lies on at least a pencil of cubic surfaces. Since $C$ lies on no quadrics, all of these cubic surfaces must be irreducible, and it follows by B\'ezout that the intersection of two such surfaces is exactly $C$. At this point, Noether-Lasker assures us that $C$ lies on exactly two cubics.

The space of curves of this type is thus an open subset of the Grassmannian $G(2,20)$ of pencils of cubic surfaces, which is irreducible of dimension 36.

2. Next, suppose that $C$ does lie on a quadric surface. In this case, we claim that $C$ must be a curve of type $(3,6)$, or equivalently $C$ is residual to three skew lines in the complete intersection of a quadric $Q$ and a sextic surface $S$.

In sum, there are two types of smooth, irreducible, nondegenerate curves $C \subset \PP^3$ of degree 9 and genus 10: type 1, which are complete intersections of two cubics; and type 2, which are curves of type $(3,6)$ on a quadric surface. Moreover, the family of curves of each type is irreducible of dimension 36; and we conclude that \emph{the space $\cH^\circ_{9m-9}(\PP^3)$ is reducible, with two components of dimension 36}.


\begin{exercise}
In the preceding argument, we used a dimension count to conclude that a general curve of type 1 could not be a specialization of a curve of type 2, and vice versa. Prove these assertions directly: specifically, argue that
\begin{enumerate}
\item by upper-semicontinuity of $h^0(\cI_{C/\PP^3}(2))$, argue that a curve $C$ not lying on a quadric cannot be the specialization of curves $C_t$ lying on quadrics; and
\item show that for a general curve of type $(3,6)$ on a quadric, $K_C \not\cong \cO_C(2)$, and deduce that a general curve of type 2 is not a specialization of curves of type 1.
\end{enumerate}
\end{exercise}

\begin{exercise}
Let $\Sigma_1$ and $\Sigma_2 \subset \cH^\circ_{9m-9}(\PP^3)$ be the loci of curves of types 1 and 2 respectively. 
\begin{enumerate}
\item What is the intersection of the closures of $\Sigma_1$ and $\Sigma_2$ in $\cH_{9m-9}(\PP^3)$?
\item What is the intersection of the closures of $\Sigma_1$ and $\Sigma_2$ in the whole Hilbert scheme $\cH_{9m-9}(\PP^3)$?
\end{enumerate}
\end{exercise}

\section{Degree 14: the training wheels come off}

We have seen, in the last two sections, examples of Hilbert schemes of smooth curves that are reducible or of greater than the expected dimension. In this final section, we'll see an example of yet another phenomenon: a component of the Hilbert scheme that is everywhere nonreduced. 

We should say at the outset that this analysis, while not requiring ideas or techniques beyond those introduced above, does involve a substantial amount of verification and case-checking. In the interests of maintaining a reasonable flow, we have relegated many of these verifications to the exercises.

To fix the numerical invariants: we are going to be analyzing here the open subscheme $\cH^\circ$ of the Hilbert scheme parametrizing smooth, irreducible curves of degree 14 and genus 24 in $\PP^3$. As we'll see, there are three irreducible components of this locus. 

We start our analysis in the by now familiar way: we ask, given a smooth, irreducible curve $C \subset \PP^3$ of degree 14 and genus 24, what sort of surfaces may contain it? By applying the genus formula for plane curves and curves on quadrics we see right off the bat that $C$ cannot lie on a quadric, so we start our usual chart in degree 3:

\begin{center}
\begin{tabular}{ c | c | c }\label{postulation table}
 $m$ & $h^0(\cO_C(m))$ & $h^0(\cO_{\PP^3}(m))$ \\
 \hline
 3 & 19, 20 or 21 & 20 \\
 4 & 33 & 35 \\
 5 & 47 & 56 \\
 6 & 61 & 84
\end{tabular}
\end{center}

Here we see from degree considerations that $\cO_C(m)$ is nonspecial for $m \geq 4$, so Riemann-Roch gives an exact value of $h^0(\cO_C(m))$ in those cases. For $m=3$, on the other hand, the degree of $K_C(-3)$ is 4, and so 
$$
h^1(\cO_C(3)) = h^0(K_C(-3)) = 0, 1 \text{ or } 2.
$$
(The curve $C$ cannot be hyperelliptic, since it's embedded in $\PP^3$ by a special linear series, and as we saw in Chapter~\ref{} a special linear series on a hyperelliptic curve can never be very ample; thus by Clifford $h^0(K_C(-3))$ cannot be 3 or more.) Riemann-Roch thus gives three a priori possible values for $h^0(\cO_C(3))$. In particular, as far as the table is concerned $C$ may or may not lie on a cubic surface; we'll consider these cases in turn.

\subsection{Case 1: $C$ does not lie on a cubic surface}

In this case, the table says that $C$ lies on at least two linearly independent quartic surfaces $S$ and $S'$; and since $C$ does not lie on any surface of smaller degree, neither can be reducible. It follows that the intersection $S \cap S'$ must consist of the union of the curve $C$ and a curve $D$ of degree 2; and the linkage formula~(\ref{linked genus formula}) says that
$$
g(C) - g(D) = (14 - 2)\frac{4+4-4}{2} = 24,
$$
so $D$ has arithmetic genus 0. It follows that $D$ must be a plane conic curve (****); so we see that in this case $C$ is residual to a plane conic curve $D$ in the complete intersection of two quartics. Conversely, if $C$ is any curve residual to a conic curve $D$ in the complete intersection of two quartics, it must have degree 14 and genus 24, and by B\'ezout it cannot lie on a cubic surface, so it must be of this type. We can thus describe the family of all smooth curves of degree 14 and genus 24 not lying on a cubic surface via the incidence correspondence
$$
\Phi = \{ (C, D, S, S') \in \cH^\circ \times \cH_D \times \PP^{34} \times \PP^{34} \mid S \cap S' = C \cup D\}.
$$
where $\cH_D$ denotes the Hilbert scheme of plane conics. The Hilbert scheme $\cH_D$ is irreducible of dimension 8; and for any conic $D$ the space of quartic surfaces containing it is a linear subspace of $\PP^{34}$ of codimension 9. The fibers of $\Phi$ over $\cH_D$ are thus open subsets of $\PP^{25} \times \PP^{25}$, and we deduce that \emph{$\Phi$ is irreducible of dimension 58}. 

Finally, if $C$ is a curve of degree 14 and genus 24 residual to a conic $D$ in the intersection of two quartic surfaces, we see from the derivation of the linkage formula that $(C\cdot D) = 10$. It follows that any quartic surface containing $C$ must contain $D$ as well, and so by Lasker-Noether must be a linear combination of $S$ and $S'$. The fibers of $\Phi$ over its image in $\cH_C$ are thus open subsets of $\PP^1 \times \PP^1$, and we have established the

\begin{proposition}
The locus in $\cH^\circ$ of curves not lying on a cubic surface is irreducible of dimension 56.
\end{proposition} 

Since the condition of not lying on a cubic surface is open, this locus forms an irreducible component of $\cH^\circ$, which we'll call $\cH_1$.

Before we go on to Case 2, let's establish one more fact about the geometry of $\cH_1$: that it is generically smooth. To do this, we have to find the dimension of its Zariski tangent space at a general point $[C]$; that is, the dimension $h^0(\cN_{C/\PP^3})$ of the space of global sections of the normal bundle of $C$ in $\PP^3$.

To do this, we choose a quartic surface $S$ containing $C$, and consider the standard exact sequence for the normal bundle $\cN_{C/\PP^3}$:
$$
0 \to \cN_{C/S} \to \cN_{C/\PP^3} \to \cN_{S/\PP^3}|_C \to 0
$$
The bundle $\cN_{S/\PP^3}|_C \cong \cO_C(4)$, which as we've seen is nonspecial; we have $h^0(\cO_C(4)) = 33$ and $h^1(\cO_C(4)) = 0$. As for the bundle $\cN_{C/S}$,  since $S$ has trivial canonical bundle adjunction tells us that $\cN_{C/S} \cong K_C$, so we have $h^0(\cN_{C/S}) = 24$ and $h^1(\cN_{C/S}) = 1$.

This leaves two possibilities for the dimension $h^0(\cN_{C/\PP^3})$: 56 and 57. To say which occurs, we have to observe a key fact: \emph{a general quartic surface does not contain any conic curves}. In fact, there is an infinitesimal version of this: if $S$ is a smooth quartic surface containing a conic curve $C$, there exist first-order deformations of $S$ containing no first-order deformations of $C$. What this says is that the map
$H^0(\cN_{C/\PP^3}) \to H^0(\cN_{S/\PP^3}|_C)$ cannot be surjective; it follows that the coboundary map $H^0(\cN_{S/\PP^3}|_C) \to H^1(\cN_{C/S})$ must be surjective and we have
$$
h^1(\cN_{C/\PP^3}) = 0 \quad \text{and} \quad h^0(\cN_{C/\PP^3}) = 56,
$$
showing that $\cH_1$ is smooth at $[C]$.

\begin{exercise}
Let $\PP^{34}$ be the space of quartic surfaces in $\PP^3$ and $\cH = \cH^\circ_{2m+1}$ the space of conic plane curves in $\PP^3$, and consider the standard incidence correspondence
$$
\Phi = \{(S,D) \in \PP^{34} \times \cH \mid D \subset S \}.
$$
\begin{enumerate}
\item Show that $\dim \Phi = 33$, so that the map $\Phi \to \PP^{34}$ cannot be dominant.
\item Use Lemma 6.23 of \cite{3264} to show that if $S$ is any smooth quartic surface, a general first-order deformation of $S$ contains no conics.
\end{enumerate}
\end{exercise}


\subsection{Case 2: $C$ lies on a cubic surface $S$}

Having shown that the open subset of $\cH^\circ$ corresponding to curves that do not lie on any cubic surface is dense in one irreducible component  $\cH_1 \subset \cH^\circ$, we turn now to the case of curves $C$ that do, that is, to the complement of $\cH_1$ in $\cH^\circ$. In this case, B\'ezout tells us that the cubic surface $S$ containing $C$ is unique, and we will restrict ourselves to the open subset $\cH_2 \subset \cH^\circ \setminus \cH_1$ where the surface $S$ is smooth. (The question of whether there are irreducible components of $\cH^\circ$ whose general members lie on singular cubic surfaces is a side issue that requires an insane amount of case-checking to resolve, and we choose to ignore it.) \fix{find the correct statement}

As always, the first question to ask about a smooth curve $C \subset \PP^3$ of degree $14$ and genus $24$ lying on a smooth cubic surface $S$ is, ``what other surfaces contain $C$?" B\'ezout immediately tells us that $C$ cannot lie on a quartic surface not containing $S$, and with a little more work we can see that it cannot lie on a quintic surface not containing $S$, either: if $C$ were residual to a line in a complete intersection of a cubic and a quintic, the liaison formula~\ref{} would tell us that 
$$
g(C) = (14-1)\frac{3+5-4}{2} = 26.
$$
On the other hand, Table~\ref{postulation table} tells us that there is at least a $84-61 = 23$-dimensional vector space of sextic polynomials vanishing on  $C$, only a 20-dimensional subspace of which can vanish on $S$. Thus there is a sextic surface $T$ containing $C$ but not containing $S$, and we can write
$$
S \cap T = C \cup D
$$
with $D$ a curve of degree 4. We can again apply the liaison formula, which says that
$$
g(C) - g(D) = (14 - 4)\frac{3+6-4}{2} = 25,
$$
so the arithmetic genus of $D$ is $-1$. Note also that $D$ moves in at least a 2-dimensional linear series on $S$: as we observed, the space of sextics vanishing on $S$ has codimension at least 3 in the space of sextics vanishing on $C$, so that $h^0(\cO_S(D)) \geq 3$. We will henceforth take $T$ to be general among sextics containing $C$, so that $D$ will be a general member of the (at least) 2-dimensional linear system cut by sextics containing $C$.

With this said, we have the

\begin{proposition}
$D$ must either be (a) the disjoint union of a line and a twisted cubic on $S$; or (b) a union of two disjoint conics on $S$.
\end{proposition}

\begin{exercise}
(Guided exercise to prove this proposition: first, $D$ cannot have multiple components; then, must be disconnected.)
\end{exercise}

Since neither of these cases can be a specialization of the other, we conclude that the locus $\cH_2$ is the union of two loci $\cH_{2a}$ and $\cH_{2b}$ corresponding to these two cases. We consider these in turn.


\begin{exercise}
(Guided exercise to prove this that $\cH_{2a}$ and $\cH_{2b}$ are irreducible, either by the incidence correspondences or by monodromy.)
\end{exercise}


\subsubsection{Case 2a: $D$ is the disjoint union of a twisted cubic and a line}

To describe the locus in $\cH_2$ corresponding to curves $C$ of this type, let $\cH$ be the locus in the Hilbert scheme $\cH_{4m+2}$ corresponding to disjoint unions of twisted cubics and lines, and set up the usual liaison correspondence
$$
\Phi = \{(C,D,S,T) \in \cH_{2a} \times \cH \times \PP^{19} \times \PP^{83} \mid S \cap T = C \cup D \}.
$$
We have $\dim \cH = 16$, and the fiber of $\Phi$ over a point $[D] \in \cH$ is an open subset of the product $\PP^5 \times \PP^{37}$; so we see that $\Phi$ is irreducible of dimension 58. As for the fibers of $\Phi$ over $ \cH_{2a}$, these are 2-dimensional, and we may ultimately conclude that $\cH_{2a}$ is irreducible of dimension 56.

Finally, we calculate the dimension of the Zariski tangent space $H^0(\cN_{C/\PP^3})$ to $\cH_{2a}$ at a general point $[C]$. We do this, as before, by considering the exact sequence associated to the inclusion of $C$ in $S$:
$$
0 \to \cN_{C/S} \to \cN_{C/\PP^3} \to \cN_{S/\PP^3}|_C \to 0
$$ 
Here there is no ambiguity about the first term: by adjunction, the degree of the normal bundle of $C$ in $S$---that is, the self-intersection of $C$ in $S$---is 60, which is greater than $2g(C) - 2 = 46$; so $h^1(\cN_{C/S}) = 0$ and $h^0(\cN_{C/S}) = 37$.

The issue here is the cohomology of the third term, $\cN_{S/\PP^3}|_C \cong \cO_C(3)$; as we observed in Table~\ref{postulation table}, this can a priori have either 19, 20 or 21 global sections. But given the explicit description of $C$ in this case, we can determine which it is. To start with, some notation: we let $L$ and $T$ denote the line component and the twisted cubic component of $D$ respectively; and let $H$ denote the hyperplane class on $S$. By adjunction, the self-intersection of $L$ and $T$ on $S$ are given by $(L \cdot L) = -1$ and $(T \cdot T) = 1$. Since $C \sim 6H - D$ on $S$, we have
$$
(C\cdot L) = (6H - L - T \cdot L) = 7; \quad \text{and} \quad (C\cdot T) = (6H - L - T \cdot L) = 17
$$
In other words, the curves $L$ and $T$ intersect $C$ in divisors $E_L$ and $E_T$ of degrees $7$ and $17$ respectively. To determine $h^1(\cO_C(3))$, we can write
$$
h^1(\cO_C(3)) = h^0(K_C(-3)) 
$$
and by adjunction,
$$
K_C(-3) = K_S(C)(-3)|_C = \cO_S(-H + 6H - D - 3H)|_C = \cO_C(2)(-E_L-E_T).
$$
Now, the quadrics in $\PP^3$ cut out on $C$ the complete linear series $|\cO_C(2)|$ (****), so we may finally conclude that \emph{$h^1(\cO_C(3))$ is the dimension of the space of quadratic polynomials vanishing on $E_L$ and $E_T$}. But $E_L$ consists of seven points on the line $L$, so any quadric containing $E_L$ contains $L$; and likewise since $E_T$ has degree $17 > 2\cdot 3$, any quadric containing $E_T$ contains $T$. Finally, we simply observe that \emph{no quadric contains the disjoint union of a line and a twisted cubic}; we conclude that $h^1(\cO_C(3))=0$ and $h^0(\cO_C(3)) = 19$.

Putting this all together, we conclude that $h^0(\cN_{C/\PP^3}) = 56$; so the component $\cH_{2a}$ of the Hilbert scheme $\cH^\circ$ is generically smooth of dimension 56.

\subsubsection{Case 2b: $D$ is the disjoint union of two conics}

The analysis of the remaining case---the locus in $\cH^\circ$ of curves $C$ residual to a union of two disjoint conics in the intersection of a smooth cubic and a sextic---follows precisely the same path as the preceding, right up until the very last step. To start with, we let $\cH$ now be the locus in the Hilbert scheme $\cH_{4m+2}$ corresponding to disjoint unions of two conics, and consider the correspondence
$$
\Phi = \{(C,D,S,T) \in \cH_{2a} \times \cH \times \PP^{19} \times \PP^{83} \mid S \cap T = C \cup D \}.
$$
Once more we have $\dim \cH = 16$, and the fiber of $\Phi$ over a point $[D] \in \cH$ is again an open subset of the product $\PP^5 \times \PP^{37}$ (unions of two disjoint conics imposes the same number of conditions on cubics and sextics as the disjoint union of a line and a twisted cubic); so we see that $\Phi$ is as before irreducible of dimension 58. The fibers of $\Phi$ over $ \cH_{2a}$ are likewise 2-dimensional, and we may ultimately conclude that $\cH_{2b}$ is irreducible of dimension 56.

Similarly, the calculation of the dimension of the Zariski tangent space $H^0(\cN_{C/\PP^3})$ to $\cH_{2a}$ at a general point $[C]$ proceeds exactly as in the last case: we start with the exact sequence
$$
0 \to \cN_{C/S} \to \cN_{C/\PP^3} \to \cN_{S/\PP^3}|_C \to 0.
$$ 
Just as before, there is no ambiguity about the first term: the line bundle $\cN_{C/S}$ has degree 60 and so is nonspecial; thus $h^1(\cN_{C/S}) = 0$ and $h^0(\cN_{C/S}) = 37$.

It's only when we come to the last part of the calculation---the determination of the cohomology of the third term, $\cN_{S/\PP^3}|_C \cong \cO_C(3)$---that things are different. We set up in the same way: we let $Q$ and $Q'$ be the  two conics comprising the residual curve $D$; and let $H$ denote the hyperplane class on $S$. (Note that $Q$ and $Q'$ are linearly equivalent: each is residual to a line in the intersection of $S$ with a plane, and if $Q$ and $Q'$ are disjoint these  must be the same line $L$. Thus we can write the class of $C$ on $S$ as either $6H-2Q$ or, equivalently, $4H+2L$.)
By adjunction, we have $Q \cdot Q = 0$; and since $C \sim 6H - 2Q$ on $S$, we have
$$
(C\cdot Q) = (6H - 2Q \cdot Q) = 12.
$$
In other words, the curves $Q$ and $Q'$ intersect $C$ in divisors $E_Q$ and $E_{Q'}$ of degree $12$. As before, we can write
$$
h^1(\cO_C(3)) = h^0(K_C(-3)) = h^0(\cO_C(2)(-E_Q-E_{Q'})
$$
so we may finally conclude that \emph{$h^1(\cO_C(3))$ is the dimension of the space of quadratic polynomials vanishing on $E_Q$ and $E_{Q'}$}; again, since $12 > 2\cdot 2$, this is the same as the space of quadrics containing the two curves $Q$ and $Q'$. And here, finally, is where the stories diverge: whereas there is no quadric containing the disjoint union of a line and a twisted cubic, there is indeed a quadric containing the union of two given disjoint conics, namely, the sum of the planes of the conics. Conversely, any quadric containing $Q$ and $Q'$ must intersect the plane $\overline Q$ in $Q$ plus the two points of $Q' \cap \overline Q$ and so must contain $\overline Q$; thus $\overline Q \cup \overline {Q'}$ is the unique quadric containing $Q \cup Q'$.  
 We may conclude in this case that $h^1(\cO_C(3))=1$,  $h^0(\cO_C(3)) = 20$ and correspondingly $h^0(\cN_{C/\PP^3}) = 57$.
 
 In sum, we see that the irreducible component $\cH_{2b} \subset \cH^\circ$ has dimension 56, but its tangent space at a general point has dimension 57; in other words, \emph{the closure of $\cH_{2b}$ is an everywhere nonreduced component of the Hilbert scheme}.

\subsubsection{What's going on here?}

What accounts for the different behaviors of curves in cases 2a and 2b? Here is one explanation:

To start, let $C$ be a curve corresponding to a general point of $\cH_{2a}$. As we've seen, we have
$$
h^1(\cO_C(3)) = 0 \quad \text{and} \quad h^0(\cO_C(3)) = 19,
$$
and so $C$ is \emph{forced} to lie on a cubic surface. Moreover, by upper-semicontinuity, the same is true of any deformation of $C$, and so in an \'etale neighborhood of $[C]$ the Hilbert scheme looks like a projective bundle over the space of cubic surfaces.

By contrast, if $C$ is the curve corresponding to a general point of $\cH_{2b}$, we have
$$
h^1(\cO_C(3)) = 1 \quad \text{and} \quad h^0(\cO_C(3)) = 20.
$$
In other words, $C$ is not forced to lie on a cubic surface, it just chooses to! And, correspondingly, it turns out that \emph{there are finite-order deformations of $C$ that are not contained in any deformation of $S$}. If we could extend these deformations to arbitrary order, we would arrive at a family of curves whose general member lay in the first component $\cH_1$; but we know that a general point of $\cH_{2b}$ is not in the closure of $\cH_1$, and so \emph{these deformations of $C$ must be obstructed}.

One final note: it may seem that the phenomenon described in this last example---a component of the Hilbert scheme that is everywhere nonreduced, even though the objects parametrized are perfectly nice smooth, irreducible curves in $\PP^3$---represents a pathology. (Indeed, it was first described by David Mumford, in a paper entitled ``Pathologies"!) But, as Ravi Vakil has shown, it is to be expected: Vakil shows that, in effect, \emph{every singularity occurs as a singularity of a Hilbert scheme of smooth curves}. (reference to Vakil's paper, and more precise statement of Ravi's theorem)
 
%footer for separate chapter files

\ifx\whole\undefined
%\makeatletter\def\@biblabel#1{#1]}\makeatother
\makeatletter \def\@biblabel#1{\ignorespaces} \makeatother
\bibliographystyle{msribib}
\bibliography{slag}

%%%% EXPLANATIONS:

% f and n
% some authors have all works collected at the end

\begingroup
%\catcode`\^\active
%if ^ is followed by 
% 1:  print f, gobble the following ^ and the next character
% 0:  print n, gobble the following ^
% any other letter: normal subscript
%\makeatletter
%\def^#1{\ifx1#1f\expandafter\@gobbletwo\else
%        \ifx0#1n\expandafter\expandafter\expandafter\@gobble
%        \else\sp{#1}\fi\fi}
%\makeatother
\let\moreadhoc\relax
\def\indexintro{%An author's cited works appear at the end of the
%author's entry; for conventions
%see the List of Citations on page~\pageref{loc}.  
%\smallbreak\noindent
%The letter `f' after a page number indicates a figure, `n' a footnote.
}
\printindex[gen]
\endgroup % end of \catcode
%requires makeindex
\end{document}
\else
\fi
