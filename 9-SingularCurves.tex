%header and footer for separate chapter files

\ifx\whole\undefined
\documentclass[12pt, leqno]{book}
\usepackage{graphicx}
\usepackage{eps-to-pdf}
\input style-for-curves.sty
%\input sl-macros.sty
\usepackage{hyperref}
\usepackage{showkeys} %This shows the labels.
\usepackage{msribib}
\usepackage{pdfpages}
\usepackage{draftwatermark}
\SetWatermarkText{DRAFT:\ \today}
\SetWatermarkScale{2}
\SetWatermarkColor[gray]{0.9}

%\usepackage{SLAG,msribib,local}
%\usepackage{amsmath,amscd,amsthm,amssymb,amsxtra,latexsym,epsfig,epic,graphics}
%\usepackage[matrix,arrow,curve]{xy}
%\usepackage{graphicx}
%\usepackage{diagrams}
%
%%\usepackage{amsrefs}
%%%%%%%%%%%%%%%%%%%%%%%%%%%%%%%%%%%%%%%%%%
%%\textwidth16cm
%%\textheight20cm
%%\topmargin-2cm
%\oddsidemargin.8cm
%\evensidemargin1cm
%
%%%%%%Definitions
%\input preamble.tex
%\input style-for-curves.sty
%\def\TU{{\bf U}}
%\def\AA{{\mathbb A}}
%\def\BB{{\mathbb B}}
%\def\CC{{\mathbb C}}
%\def\QQ{{\mathbb Q}}
%\def\RR{{\mathbb R}}
%\def\facet{{\bf facet}}
%\def\image{{\rm image}}
%\def\cE{{\cal E}}
%\def\cF{{\cal F}}
%\def\cG{{\cal G}}
%\def\cH{{\cal H}}
%\def\cHom{{{\cal H}om}}
%\def\h{{\rm h}}
% \def\bs{{Boij-S\"oderberg{} }}
%
%\makeatletter
%\def\Ddots{\mathinner{\mkern1mu\raise\p@
%\vbox{\kern7\p@\hbox{.}}\mkern2mu
%\raise4\p@\hbox{.}\mkern2mu\raise7\p@\hbox{.}\mkern1mu}}
%\makeatother

%%
%\pagestyle{myheadings}

%\input style-for-curves.tex
%\documentclass{cambridge7A}
%\usepackage{hatcher_revised} 
%\usepackage{3264}
   
\errorcontextlines=1000
%\usepackage{makeidx}
\let\see\relax
\usepackage{makeidx}
\makeindex
% \index{word} in the doc; \index{variety!algebraic} gives variety, algebraic
% PUT a % after each \index{***}

\overfullrule=5pt
\catcode`\@\active
\def@{\mskip1.5mu} %produce a small space in math with an @

\title{A Chapter from ``The Practice of Algebraic Curves"}
\author{\copyright David Eisenbud and Joe Harris}
%%\includeonly{%
%0-intro,01-ChowRingDogma,02-FirstExamples,03-Grassmannians,04-GeneralGrassmannians
%,05-VectorBundlesAndChernClasses,06-LinesOnHypersurfaces,07-SingularElementsOfLinearSeries,
%08-ParameterSpaces,
%bib
%}

\date{\today}
%%\date{}
%\title{Curves}
%%{\normalsize ***Preliminary Version***}} 
%\author{David Eisenbud and Joe Harris }
%
%\begin{document}

\begin{document}
\maketitle

\pagenumbering{roman}
\setcounter{page}{5}
%\begin{5}
%\end{5}
\pagenumbering{arabic}
\tableofcontents
\fi


\chapter{Singular Curves}
\label{SingularCurvesChapter}

Throughout this book thus far, we have developed techniques for dealing with smooth, projective curves; to the extent that we have considered singular curves we have studied them by applying the ideas and constructions we've developed to their normalizations. But singular curves have a fascinating geometry in their own right, not only for the singularities themselves but the effect singularities have on associated objects such as the Jacobian. In this chapter, we will undertake a brief survey of the geometry of singular curves and what we can say about linear series on them.

To say what classes of curves we'll be dealing with here: for the most part, we will confine ourselves to working with reduced, projective curves. Many of the results we will derive will in fact will be applicable to a larger class of curves, namely those that are \emph{Cohen-Macaulay}; these may well be non-reduced, but cannot have embedded points. In Section~\ref{} below, we'll mention some examples of nonreduced curves to which we can apply our ideas (such as, for example, \emph{ribbons}), and we'll develop these ideas further in the following chapter, where we introduce more algebraic techniques. But for the rest of this chapter, we will take the objects of our study to be reduced, projective curves over $\CC$.


\section{The arithmetic genus and singularities}

To start at the beginning, when we first defined the notion of the \emph{genus} of a smooth projective curve, we gave several different but equivalent characterizations of the genus. As you might expect, these diverge in the presence of singularities, so we adopt the following universal definition.

\begin{definition}
Let $C$ be an arbitrary one-dimensional projective scheme over a field $\CC$. By the \emph{arithmetic genus} $p_a(C)$ of $C$ we mean 1 minus the Euler characteristic of the structure sheaf of $C$:
$$
p_a(C) \; = \; 1 - \chi(\cO_{C}).
$$
In contrast, if $C$ is reduced, we define the \emph{geometric genus} to be the genus of the normalization $C^\nu$ of $C$.
\end{definition}

Note that the arithmetic genus satisfies many of the formulas derived above in the smooth case: for example, if $C \subset S$ is a divisor on a smooth surface, the adjunction formula holds:
$$
p_a(C) \; = \; \frac{C\cdot C + K_S\cdot C}{2} + 1.
$$
Thus, for example, a double conic curve $C = V((XY-Z^2)^2) \subset \PP^2$, like every other plane quartic curve, has arithmetic genus 3. (If you're curious, we'll see what the Jacobian of $C$ looks like in Section\ref{} below.)

\subsection{Relation between the arithmetic and geometric genus}

Our first order of business is to understand the relationship between the arithmetic and geometric genera of a reduced projective curve $C$. To this end, let $\nu : C^\nu \to C$ be the normalization of $C$, and consider the exact sequence of sheaves on $C$:
$$
0 \to \cO_C \to \nu_*\cO_{C^\nu} \to \cF \to 0
$$
where $\cF$ is simply defined to be the quotient $\nu_*\cO_{C^\nu}/\cO_C$; note that $\cF$ is supported exactly at the singular points of $C$.

\section{canonical sheaf and Gorenstein condition}
Adjoint ideal and conductor ideal

\section{RR for singular curves}

\section{Picard group and Jacobian}
\begin{fact}
 something about the compactification
\end{fact}

%footer for separate chapter files

\ifx\whole\undefined
\makeatletter\def\@biblabel#1{#1]}\makeatother
\gdef\urlhook{\url}
\bibliography{slag}
\bibliographystyle{msribib}


%%%% EXPLANATIONS:

% f and n
% some authors have all works collected at the end

\catcode`\^\active
%if ^ is followed by 
% 1:  print f, gobble the following ^ and the next character
% 0:  print n, gobble the following ^
% any other letter: print letter
\makeatletter
\def^#1{\ifx1#1f\expandafter\@gobbletwo\else
        \ifx0#1n\expandafter\expandafter\expandafter\@gobble\else#1\fi\fi}
\makeatother
\let\moreadhoc\relax
\def\indexintro{%An author's cited works appear at the end of the
%author's entry; for conventions
%see the List of Citations on page~\pageref{loc}.  
%\smallbreak\noindent
The letter `f' after a page number indicates a figure, `n' a footnote.}
\printindex[gen]
%requires makeindex
\end{document}
\else
\fi
