%header and footer for separate chapter files

\ifx\whole\undefined
\documentclass[12pt, leqno]{book}
\usepackage{graphicx}
\usepackage{eps-to-pdf}
\input style-for-curves.sty
%\input sl-macros.sty
\usepackage{hyperref}
\usepackage{showkeys} %This shows the labels.
\usepackage{msribib}
\usepackage{pdfpages}
\usepackage{draftwatermark}
\SetWatermarkText{DRAFT:\ \today}
\SetWatermarkScale{2}
\SetWatermarkColor[gray]{0.9}

%\usepackage{SLAG,msribib,local}
%\usepackage{amsmath,amscd,amsthm,amssymb,amsxtra,latexsym,epsfig,epic,graphics}
%\usepackage[matrix,arrow,curve]{xy}
%\usepackage{graphicx}
%\usepackage{diagrams}
%
%%\usepackage{amsrefs}
%%%%%%%%%%%%%%%%%%%%%%%%%%%%%%%%%%%%%%%%%%
%%\textwidth16cm
%%\textheight20cm
%%\topmargin-2cm
%\oddsidemargin.8cm
%\evensidemargin1cm
%
%%%%%%Definitions
%\input preamble.tex
%\input style-for-curves.sty
%\def\TU{{\bf U}}
%\def\AA{{\mathbb A}}
%\def\BB{{\mathbb B}}
%\def\CC{{\mathbb C}}
%\def\QQ{{\mathbb Q}}
%\def\RR{{\mathbb R}}
%\def\facet{{\bf facet}}
%\def\image{{\rm image}}
%\def\cE{{\cal E}}
%\def\cF{{\cal F}}
%\def\cG{{\cal G}}
%\def\cH{{\cal H}}
%\def\cHom{{{\cal H}om}}
%\def\h{{\rm h}}
% \def\bs{{Boij-S\"oderberg{} }}
%
%\makeatletter
%\def\Ddots{\mathinner{\mkern1mu\raise\p@
%\vbox{\kern7\p@\hbox{.}}\mkern2mu
%\raise4\p@\hbox{.}\mkern2mu\raise7\p@\hbox{.}\mkern1mu}}
%\makeatother

%%
%\pagestyle{myheadings}

%\input style-for-curves.tex
%\documentclass{cambridge7A}
%\usepackage{hatcher_revised} 
%\usepackage{3264}
   
\errorcontextlines=1000
%\usepackage{makeidx}
\let\see\relax
\usepackage{makeidx}
\makeindex
% \index{word} in the doc; \index{variety!algebraic} gives variety, algebraic
% PUT a % after each \index{***}

\overfullrule=5pt
\catcode`\@\active
\def@{\mskip1.5mu} %produce a small space in math with an @

\title{A Chapter from ``The Practice of Algebraic Curves"}
\author{\copyright David Eisenbud and Joe Harris}
%%\includeonly{%
%0-intro,01-ChowRingDogma,02-FirstExamples,03-Grassmannians,04-GeneralGrassmannians
%,05-VectorBundlesAndChernClasses,06-LinesOnHypersurfaces,07-SingularElementsOfLinearSeries,
%08-ParameterSpaces,
%bib
%}

\date{\today}
%%\date{}
%\title{Curves}
%%{\normalsize ***Preliminary Version***}} 
%\author{David Eisenbud and Joe Harris }
%
%\begin{document}

\begin{document}
\maketitle

\pagenumbering{roman}
\setcounter{page}{5}
%\begin{5}
%\end{5}
\pagenumbering{arabic}
\tableofcontents
\fi


\chapter{Singular Curves}
\label{SingularCurvesChapter}

Throughout this book thus far, we have developed techniques for dealing with smooth, projective curves; to the extent that we have considered singular curves we have studied them by applying the ideas and constructions we've developed to their normalizations. But singular curves have a fascinating geometry in their own right, not only for the singularities themselves but the effect singularities have on associated objects such as the Jacobian. In this chapter, we will undertake a brief survey of the geometry of singular curves and what we can say about linear series on them.

To say what classes of curves we'll be dealing with here: for the most part, we will confine ourselves to working with reduced, projective curves. Many of the results we will derive will in fact will be applicable to a larger class of curves, namely those that are \emph{Cohen-Macaulay}; these may well be non-reduced, but cannot have embedded points. In Section~\ref{} below, we'll mention some examples of nonreduced curves to which we can apply our ideas (such as, for example, \emph{ribbons}), and we'll develop these ideas further in the following chapter, where we introduce more algebraic techniques. But for the rest of this chapter, we will take the objects of our study to be reduced, projective curves over $\CC$.


\section{The arithmetic genus and singularities}

To start at the beginning, when we first defined the notion of the \emph{genus} of a smooth projective curve, we gave several different but equivalent characterizations of the genus. As you might expect, these diverge in the presence of singularities, so we adopt the following universal definition.

\begin{definition}
Let $C$ be an arbitrary one-dimensional projective scheme over a field $\CC$. By the \emph{arithmetic genus} $p_a(C)$ of $C$ we mean 1 minus the Euler characteristic of the structure sheaf of $C$:
$$
p_a(C) \; = \; 1 - \chi(\cO_{C}).
$$
In contrast, if $C$ is reduced, we define the \emph{geometric genus} to be the genus of the normalization $C^\nu$ of $C$.
\end{definition}

Note that the arithmetic genus satisfies many of the formulas derived above in the smooth case: for example, if $C \subset S$ is a divisor on a smooth surface, the adjunction formula holds:
$$
p_a(C) \; = \; \frac{C\cdot C + K_S\cdot C}{2} + 1.
$$
Thus, for example, a double conic curve $C = V((XY-Z^2)^2) \subset \PP^2$, like every other plane quartic curve, has arithmetic genus 3. (If you're curious, we'll see what the Jacobian of $C$ looks like in Section\ref{} below.)

\subsection{Relation between the arithmetic and geometric genus}

Our first order of business is to understand the relationship between the arithmetic and geometric genera of a reduced projective curve $C$. To this end, let $\nu : C^\nu \to C$ be the normalization of $C$, and consider the exact sequence of sheaves on $C$:
$$
0 \to \cO_C \to \nu_*\cO_{C^\nu} \to \cF \to 0
$$
where $\cF$ is simply defined to be the quotient $\nu_*\cO_{C^\nu}/\cO_C$; note that $\cF$ is supported exactly at the singular points of $C$.

The point here is that, because the map $\nu$ is finite, there are no higher direct images of $\nu_*\cO_{C^\nu}$; so the Leray-Serre spectral sequence tells us that
$$
\chi(\nu_*\cO_{C^\nu}) \; = \chi(\cO_C).
$$
Thus, the difference between the arithmetic and geometric genera of $C$ is
$$
g(C) \; = \; p_a(C) - h^0(\cF).
$$
We can refine this a little: for each singular point $p \in C$, denote by $\delta_p$ the dimension (as vector space over $\CC$) of the stalk $\cF_p$ of $\cF$ at $p$. This is called the \emph{delta-invariant} of $p$, and is the most fundamental numerical invariant of a curve singularity. In these terms, we can write
$$
g(C) \; = \; p_a(C) - \sum_{p \in C} \delta_p.
$$

\subsection{The delta invariant}

To see how this works in practice, let's calculate the delta invariant of some relatively simple singularities.

\subsubsection{the $\delta$-invariant of a node}

To start with the simplest singularity, suppose that $p$ is a node of $C$: that is, a neighborhood of $p \in C$\footnote{we can take this to be either a complex analytic neighborhood or an \'etale neighborhood} is the union of two smooth curves intersecting transversely at $p$. In this case, there will be two points $q, r \in C^\nu$ in the normalization lying over $p$, and we see that if $U$ is a suitably small neighborhood of $p \in C$ then a function $\tilde f \in \cO_{C^\nu}(\nu^{-1}(U)$ on the preimage of $U$ in $C^\nu$  is the pullback of a function $f \in \cO_C(U)$ if and only if the values $\tilde f(q) = \tilde f(r)$ agree. It follows that the stalk of the sheaf $\cF$ at $p$ is one-dimensional, so $\delta_p = 1$.

\subsubsection{the $\delta$-invariant of a cusp}

By definition, a \emph{cusp} of a curve $C$ is a point $p \in C$ such that the normalization map is given in terms of suitable local coordinates in a neighborhood $U$ of $p$ as
$$
\nu : t \mapsto (t^2, t^3).
$$
We see from this that a function  is the pullback of a function $f \in \cO_C(U)$ if and only if the derivative $f'(0) = 0$; this being again one linear condition, we see that $\delta_p = 1$.

\subsubsection{the $\delta$-invariant of a tacnode}
$\tilde f \in \cO_{C^\nu}(\nu^{-1}(U))$
As with a node, a suitably small neighborhood $U$ of a tacnode $p \in C$ is a union of two smooth curves; but this time the two branches are simply tangent to one another rather than transverse. It is no longer the case, accordingly, that a function $\tilde f \in \cO_{C^\nu}(\nu^{-1}(U))$ is a pullback if and only if the values of $\tilde f$ at the points  $q, r \in C^\nu$ lying over $p$ agree: for example, a function that has a simple zero at $q$ but vanishes to order 2 or more at $r$ cannot be a pullback. The correct statement is that $\tilde f$ will descend to $C$ if and only if $f(q) = f(r)$ and the derivatives $f'(q)$ and $f'(r)$ (with respect to suitably chosen local coordinates) agree. Thus we see that $\delta_p = 2$.

\subsubsection{the $\delta$-invariant of an ordinary triple point}

An ordinary triple point is again a \emph{planar} singularity, meaning its Zariski tangent space is 2-dimensional, or, equivalently, it is embeddable in a smooth surface. A small neighborhood of such a point $p \in C$ consists of a union of three smooth branches, intersecting pairwise transversely at $p$. In order for a function on the preimage of $U$ to be a pullback, naturally, its values at the three points $q,r,s \in \nu^{-1}(p)$ lying over $p$ have to agree, so that $\delta_p \geq 2$. But this is not a sufficient condition: a function  $\tilde f$ with a simple zero at $q$ but vanishing to order at least 2 at $r$ and $s$ cannot be a pullback. Rather, the fact that the tangent lines to the three branches are linearly dependent means that in order for $\tilde f$ to descend its derivatives at $q, r$ and $s$ must satisfy a linear relation as well, and so we have $\delta_p = 3$. 

\subsubsection{the $\delta$-invariant of a spatial triple point}

By way of contrast with the last example, suppose now that $p$ is a \emph{spatial triple point}: in other words, a neighborhood of $p \in C$ is a union of three smooth branches with linearly independent tangent lines (so that $\dim T_p(C) = 3$). In this case, it is a necessary and sufficient condition for a function $\tilde f$ to descend is simply that its values at the points lying over $p$ agree, from which we see that $\delta_p = 2$.

\begin{exercise}
Consider the following curves $C \subset \PP^3$ of degree 3 in $\PP^3$. In each case, determine the arithmetic genus; and, in those cases where $p_a(C) = 0$ show that the curve $C$ is indeed the flat limit of a family of twisted cubics.
\begin{enumerate}
\item the union of a line and a conic curve meeting transversely at one point;
\item the union of a conic and a tangent line;
\item the union of three concurrent, coplanar lines;
\item the union of three concurrent but not coplanar lines.
\end{enumerate}
\end{exercise}

\section{The dualizing sheaf and Riemann-Roch for singular curves}

Without question, the most fundamental object we deal with in analyzing the geometry of a smooth curve $C$ is its canonical bundle/divisor class $K_C$, defined simply to be the cotangent bundle of $C$ or equivalently the sheaf of regular 1-forms. Is there an analog of this in the case of singular curves?

The answer is an emphatic ``yes:" if $C$ is Cohen-Macaulay (and in particular, if $C$ is reduced), we can introduce the \emph{dualizing sheaf} $\omega_C$, which is a more than adequate understudy for the role of canonical bundle. (Indeed, in a large range of cases, those of \emph{Gorenstein} curves, it is locally free, as we'll see in the subsection below.) We will give a relatively concrete description of the dualizing sheaf here, and a more abstract, algebraic definition in the next chapter.

To understand the definition of the dualizing sheaf, it is useful (and amusing) to recall a bogus proof of the Riemann-Roch formula, which highlights the key property of the canonical bundle. To set this up, suppose now that $C$ is a smooth projective curve of genus $g$, and imagine we have a divisor $D = p_1 + \dots + p_d$


\subsection{the Gorenstein condition}

\section{Picard group and Jacobian}
\begin{fact}
 something about the compactification
\end{fact}

%footer for separate chapter files

\ifx\whole\undefined
\makeatletter\def\@biblabel#1{#1]}\makeatother
\gdef\urlhook{\url}
\bibliography{slag}
\bibliographystyle{msribib}


%%%% EXPLANATIONS:

% f and n
% some authors have all works collected at the end

\catcode`\^\active
%if ^ is followed by 
% 1:  print f, gobble the following ^ and the next character
% 0:  print n, gobble the following ^
% any other letter: print letter
\makeatletter
\def^#1{\ifx1#1f\expandafter\@gobbletwo\else
        \ifx0#1n\expandafter\expandafter\expandafter\@gobble\else#1\fi\fi}
\makeatother
\let\moreadhoc\relax
\def\indexintro{%An author's cited works appear at the end of the
%author's entry; for conventions
%see the List of Citations on page~\pageref{loc}.  
%\smallbreak\noindent
The letter `f' after a page number indicates a figure, `n' a footnote.}
\printindex[gen]
%requires makeindex
\end{document}
\else
\fi
