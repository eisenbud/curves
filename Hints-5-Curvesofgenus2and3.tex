%header and footer for separate chapter files

\ifx\whole\undefined
\documentclass[12pt, leqno]{book}
\usepackage{graphicx}
\input style-for-curves.sty
\usepackage{hyperref}
\usepackage{showkeys} %This shows the labels.
%\usepackage{SLAG,msribib,local}
%\usepackage{amsmath,amscd,amsthm,amssymb,amsxtra,latexsym,epsfig,epic,graphics}
%\usepackage[matrix,arrow,curve]{xy}
%\usepackage{graphicx}
%\usepackage{diagrams}
%
%%\usepackage{amsrefs}
%%%%%%%%%%%%%%%%%%%%%%%%%%%%%%%%%%%%%%%%%%
%%\textwidth16cm
%%\textheight20cm
%%\topmargin-2cm
%\oddsidemargin.8cm
%\evensidemargin1cm
%
%%%%%%Definitions
%\input preamble.tex
%\input style-for-curves.sty
%\def\TU{{\bf U}}
%\def\AA{{\mathbb A}}
%\def\BB{{\mathbb B}}
%\def\CC{{\mathbb C}}
%\def\QQ{{\mathbb Q}}
%\def\RR{{\mathbb R}}
%\def\facet{{\bf facet}}
%\def\image{{\rm image}}
%\def\cE{{\cal E}}
%\def\cF{{\cal F}}
%\def\cG{{\cal G}}
%\def\cH{{\cal H}}
%\def\cHom{{{\cal H}om}}
%\def\h{{\rm h}}
% \def\bs{{Boij-S\"oderberg{} }}
%
%\makeatletter
%\def\Ddots{\mathinner{\mkern1mu\raise\p@
%\vbox{\kern7\p@\hbox{.}}\mkern2mu
%\raise4\p@\hbox{.}\mkern2mu\raise7\p@\hbox{.}\mkern1mu}}
%\makeatother

%%
%\pagestyle{myheadings}

%\input style-for-curves.tex
%\documentclass{cambridge7A}
%\usepackage{hatcher_revised} 
%\usepackage{3264}
   
\errorcontextlines=1000
%\usepackage{makeidx}
\let\see\relax
\usepackage{makeidx}
\makeindex
% \index{word} in the doc; \index{variety!algebraic} gives variety, algebraic
% PUT a % after each \index{***}

\overfullrule=5pt
\catcode`\@\active
\def@{\mskip1.5mu} %produce a small space in math with an @

\title{Personalities of Curves}
\author{\copyright David Eisenbud and Joe Harris}
%%\includeonly{%
%0-intro,01-ChowRingDogma,02-FirstExamples,03-Grassmannians,04-GeneralGrassmannians
%,05-VectorBundlesAndChernClasses,06-LinesOnHypersurfaces,07-SingularElementsOfLinearSeries,
%08-ParameterSpaces,
%bib
%}

\date{\today}
%%\date{}
%\title{Curves}
%%{\normalsize ***Preliminary Version***}} 
%\author{David Eisenbud and Joe Harris }
%
%\begin{document}

\begin{document}
\maketitle

\pagenumbering{roman}
\setcounter{page}{5}
%\begin{5}
%\end{5}
\pagenumbering{arabic}
\tableofcontents
\fi



\chapter{Curvesofgenus2and3 hints}\label{Curvesofgenus2and3 hints}

\begin{exercise}
  We have seen that a curve $C$ of genus $g=1$ is expressible as a 2-sheeted cover of $\PP^1$ branched over four points; that is, as the smooth projective curve associated to the affine curve $C^\circ$ given by $y^2 - \prod_{i=1}^4 (x-\lambda_i)$. Show that the closure $\overline{C^\circ}$ of $C^\circ \subset \AA^2$ in either $\PP^2$ or $\PP^1 \times \PP^1$ consists of the union of $C^\circ$ with one additional point, with that point a tacnode of $\overline{C^\circ}$ in either case.
  
  Hint: Observe that in either case the complement $\overline{C^\circ} \setminus C^\circ$ consists of a single point, with two points of $C$ mapping to it; now use the genus formula in either $\PP^2$ or $\PP^1 \times \PP^1$. 
  \end{exercise}

\begin{exercise}
Find the number of 3-sheeted covers $C \to \PP^1$ of genus $g$ with simple branching except for one point of total ramification (that is, one point with just a single preimage point.)

Hint: such a cover is specified by giving $2g+2$ transpositions, not all equal, whose product is a nontrivial 3-cycle, modulo simultaneous conjugation. We have already worked out the number of such tuples whose product is the identity; just subtract.
\end{exercise}


\begin{exercise}
Let $B$ be a curve of genus $h$. How many unramified double covers of $B$ are there?

Hint: Topologically, such covers are in 1-1 correspondence with subgroups of index 2 in $\pi_1(C)$; and such a subgroup is necessarily the preimage of a subgroup of index 2 in the abelianization $H_1(C, \ZZ) \cong \ZZ^{2g}$.
\end{exercise}

\begin{exercise}
Show that unramified double covers of a smooth curve $C$ are in one-to-one correspondence
with invertible sheaves $\sL$ on $C$ such that $\sL^2 \cong \sO_C$, that is with the 2-torsion points
of $\Jac(C)$.

Hint: If $f : X \to C$ is an unramified double cover, consider the direct image $f_*(\cO_X)$. This is a locally free sheaf of rank 2 on $C$, on which the group $\ZZ/2$ acts; the $+1$-eigenspace is the structure sheaf $\cO_C$, and the $-1$-eigenspace is an invertible sheaf $\sL$ on $C$ such that $\sL^2 \cong \sO_C$.
\end{exercise}


\begin{exercise} Let $E$ be a curve of genus 1, and $q_1,\dots,q_b \in E$. How many double covers $C \to E$ are there branched over the $q_i$?

Hint: By our analysis, to specify such a cover, we have to specify the monodromy around representative loops generating $H_1(E, \ZZ) \cong \ZZ^2$; thus there are four possibilities.
\end{exercise}

%\fix{Do we do anywhere the correspondence between double covers and square roots of the branch divisor?}

%\begin{exercise} Let $E$ be a curve of genus 1, and $q, q' \in E$. How many triple covers $C \to E$ are there simply branched over $q$ and $q'$?
%\end{exercise}

\begin{exercise}\label{ideal of genus 2 degree 5} 
Show that for any pair of lines $L, L'$ of the appropriate ruling of $Q$, the three polynomials $Q$, $S_L$ and $S_{L'}$ generate the homogeneous ideal $I(C)$. Find relations among them. Write out the minimal resolution of $I(C)$.
\end{exercise}


\begin{exercise}\label{theta char on genus 2} % this is an easy case the reader can do before reading the general treatment. 
 Let $C$ be a curve of genus 2, expressed as a 2-sheeted cover of $\PP^1$ with ramification points $p_1,\dots,p_6$. In this exercise we will
 count the number of
 even and odd theta characteristics.
The text contains the count for a hyperelliptic curve of any genus; we offer 
the case of genus 2 as a warmup.
 \begin{enumerate}
 \item Show that the theta-characteristics on $C$ are either of the form $\cL = \cO_C(p_i)$ or of the form $\cL = \cO_C(p_i + p_j - p_k)$ with $i, j, k$ distinct. 
 \item Show that in the first case we have $h^0(\cL) = 1$, and in the second case we have $h^0(\cL) = 0$. 
 \item Finally, show that there are six of the former kind, and 10 of the latter, making $2^4 = 16$ in all.
 \end{enumerate} 
 \end{exercise}
 
 
\begin{exercise}\label{nodal quartic}
Let $C$ be a  curve of genus 2 and let $\sL\in \Pic^4(C)$ be an invertible sheaf of the form $L = K_C(p+q)$ with $p \neq q$ and $p+q \not\sim K_C$ as in~\ref{p+q not g12}. Show that
\begin{enumerate}
\item $h^0(L(-2p)) = h^0(L(-2q)) = 1$, and
\item $h^0(L(-2p-2q)) = 0$.
\end{enumerate}
Deduce from this that the map $\phi_L$ is an immersion, and that the tangent lines to the two branches of $\phi_L(C)$ at the point $\phi_L(p) = \phi_L(q)$ are distinct, meaning the point $\phi_L(p) = \phi_L(q)$ is a node of $\phi_L(C)$.

Hint: the first statement implies that the map $\phi_L$ is an immersion at $p$ and $q$, while the second says that the images of the differential $d\phi_L$ at $p$ and $q$ are distinct. 
\end{exercise}


 
 
\begin{exercise}\label{G13}
We can represent any line in $\PP^3$ by 2 points on it, and using their coordinates as the two rows of a 
$2\times 4$ matrix. The \emph{Pl\"ucker coordinates} of the line are the six $2\times 2$ minors
$$
\{p_{i,j}\}_{0\leq i<j\leq 3}
$$
of this matrix. They are independent, up to a common scalar multiple, of the two points chosen, and define the \emph{Pl\"ucker embedding} of the Grassmannian $\GG(1,3)$ in $\PP^5$.

The minors $p_{i,j}$  satisfy a nonsingular quadratic equation: if we stack two copies of the $2\times 2$
matrix to produce a $4\times 4$ matrix, its determinant is zero, and the Laplace expansion of this determinant
is the \emph{Pl\"ucker equation}
$$
p_{0,1}p_{2,3}-p_{0,2}p_{1,3}+p_{0,3}p_{1,2} = 0.
$$

\begin{enumerate}
\item Show that the quadratic form
$
Q = p_{0,1}p_{2,3}-p_{0,2}p_{1,3}+p_{0,3}p_{1,2}
$
is nonsingular, and deduce that it generates the ideal of $\GG(1,3)$ in $\PP^5$.
\item
Write the bilinear form corresponding to $Q$ as the determinant of a matrix, and deduce that 
two points in $\GG(1,3)$ correspond to vectors that pair to 0 iff and only if they correspond to lines that intersect.
\item Deduce that a maximal isotropic plane for $Q$ corresponds either to the set of lines containing a given point or the set of lines contained in a given plane; and that two such sets of lines meet in a single point or coincide.
\end{enumerate}
\end{exercise}



%footer for separate chapter files

\ifx\whole\undefined
%\makeatletter\def\@biblabel#1{#1]}\makeatother
\makeatletter \def\@biblabel#1{\ignorespaces} \makeatother
\bibliographystyle{msribib}
\bibliography{slag}

%%%% EXPLANATIONS:

% f and n
% some authors have all works collected at the end

\begingroup
%\catcode`\^\active
%if ^ is followed by 
% 1:  print f, gobble the following ^ and the next character
% 0:  print n, gobble the following ^
% any other letter: normal subscript
%\makeatletter
%\def^#1{\ifx1#1f\expandafter\@gobbletwo\else
%        \ifx0#1n\expandafter\expandafter\expandafter\@gobble
%        \else\sp{#1}\fi\fi}
%\makeatother
\let\moreadhoc\relax
\def\indexintro{%An author's cited works appear at the end of the
%author's entry; for conventions
%see the List of Citations on page~\pageref{loc}.  
%\smallbreak\noindent
%The letter `f' after a page number indicates a figure, `n' a footnote.
}
\printindex[gen]
\endgroup % end of \catcode
%requires makeindex
\end{document}
\else
\fi
