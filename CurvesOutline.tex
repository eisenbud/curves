\documentclass[12pt, leqno]{article}
\usepackage{amsmath,amscd,amsthm,amssymb,amsxtra,latexsym,epsfig,epic,graphics}
\usepackage[matrix,arrow,curve]{xy}
\usepackage{graphicx}
\usepackage{diagrams}
%\usepackage{amsrefs}
%%%%%%%%%%%%%%%%%%%%%%%%%%%%%%%%%%%%%%%%%
%\textwidth16cm
%\textheight20cm
%\topmargin-2cm
\oddsidemargin.8cm
\evensidemargin1cm

%%%%%Definitions
\input preamble.tex
\def\TU{{\bf U}}
\def\AA{{\mathbb A}}
\def\BB{{\mathbb B}}
\def\CC{{\mathbb C}}
\def\QQ{{\mathbb Q}}
\def\RR{{\mathbb R}}
\def\facet{{\bf facet}}
\def\image{{\rm image}}
\def\cE{{\cal E}}
\def\cF{{\cal F}}
\def\cG{{\cal G}}
\def\cH{{\cal H}}
\def\cHom{{{\cal H}om}}
\def\h{{\rm h}}
 \def\bs{{Boij-S\"oderberg{} }}

\makeatletter
\def\Ddots{\mathinner{\mkern1mu\raise\p@
\vbox{\kern7\p@\hbox{.}}\mkern2mu
\raise4\p@\hbox{.}\mkern2mu\raise7\p@\hbox{.}\mkern1mu}}
\makeatother

%%
%\pagestyle{myheadings}
\date{March 5, 2009}
%\date{}
\title{Twenty-seven Lines: An Undergraduate Semester on Algebraic Geometry and Commutative Algebra}
%{\normalsize ***Preliminary Version***}} 
\author{David Eisenbud and Joe Harris }

\begin{document}

\maketitle


\section*{Principles}
\begin{enumerate}

\item Model book: Beauville Complex Surfaces

\item 10-12 Chapters, each of 10-15 pages, each of which can be covered in 1 week (3 hours lecture)

\item Work over $\CC$, singular curves are images of smooth ones (no multiple components or embedded points.)

\item Background: able to write $H^0, H^1$, and use the terms scheme and Hilbert Function  without blushing. Assume Ch 2 of Geometry of Schemes.

\end{enumerate}



\section*{Outline}
 \begin{enumerate}

\item Background (corresponds to material in Hartshorne Ch 4, sects 1,2,4(?)---all except elliptic curves and "classification")
\begin{enumerate}
\item B\'ezout and the  weak B\'ezout (ex 8.4.6 in Fulton).
\item Divisors and maps; canonical divisor; canonical map as example
\item Projections from on and off a curve; desingularization and removeable singularities of a map to $\PP^n$.
\item Riemann-Roch---canonical series is bpf, $2g+1$ is very ample. Canonical Curves and geometric RR
\item Adjunction formula for curves in a surface.
\item Clifford, including strong form, canonical series is va except in hyperelliptic case.
\item Riemann-Hurwitz
\item Scrolls: determinantal representation, equivalent characterizations: as Hirzebruch surfaces $\PP(\E)$; as Minimal degree varieties; as the varieties of given degree lying on the maximal number of quadrics. Divisors on surface scrolls. Scrolls in all dimensions as unions of spans of divisors. 

\end{enumerate}

\item Short chapter with Key Questions and references to text
\begin{enumerate}
 \item Lowest degree of a covering of $\PP^1$?
 \item Lowest degree of a plane model?
 \item Lowest degree of an embedding?
 \item Number of forms of of degree d vanishing on the curve---ideals of the embeddings, Hilbert Functions.
\end{enumerate}

\item Personalities
\begin{enumerate}
\item Geometric RR
\item $g=0$: rational curves as projections of rational normal curves. Rational quartic in $\PP^3$. Mention SCI problem.
\item $g=1$. Wonderful subject; refer to somewhere else. Plane cubic, quartic in $\PP^3$, state that an elliptic quintic is Pfaffian (give proof??).
\item $g=2$. Canonical map to $\PP^1$. Embedding in $\PP^3$ as $(2,3)$ on a quadric, ideal is 1 quadric, 2 cubics.
\item $g=3$. Plane quartics. Sextic curves in $\PP^3$: general $g^3_6$ is an embedding, det variety.  $g\times g+1$ matrix of linear forms in $\PP^3$ gives curve of genus $g$, degree ${g\choose 2}$
\item $g=4$. Canonical model is ci. Therefore $C$ is a 3-sheeted cover of $\PP^1$, in 1 or 2 ways, depending on the singularity of the unique quadric in the ideal.
\item $g=5$. Canonical model lies on 3 quadrics. Either a CI, or trigonal. In latter case, Pfaffians. Introduce Green's conjecture?
\item $g=6$. Canonical model lies on at least 6 quadrics. We will see exactly 6, and what they are...(projective normality of canonical embedding; and Brill-Noether for the plane model with 4 nodes.)
\end{enumerate}

\item Hyperelliptic curves (maybe this is just a section of another ch)
\begin{enumerate}
\item no special embeddings
\item a hyperelliptic curve lies on surface swept out by lines -- see section on scrolls.
\end{enumerate}

\item Castelnuovo Theory
\begin{enumerate}
\item Ineq on Hilbert functions, leading to bound on $g,d$. 
\item Scrolls and their divisors
\begin{enumerate}
 \item Hirzebruch surfaces embedded by complete series
 \item lines joining 2 rational normal curves
 \item determinantal varieties
\end{enumerate}
\item Give the scroll examples
\item Castelnuovo's lemma (2n+3 points in $\PP^n$ by determinantal proof) 
\item Characterization of Castelnuovo curves
\item Projective normality of the canonical curve.
\end{enumerate}

\item Abel map and Jacobian
\begin{enumerate}
\item $J = H^0(\omega)^\vee/\hbox{lattice of periods}$. $\dim J = g$ general $g^3_{g+3}$ is very ample. Do this as an example of Hodge Theory.

\item Symmetric powers, Abel-Jacobi map, Abel's Theorem (statement), differential of the Abel map.
\item connection with addition formulas for integrals
\end{enumerate}

\item Projective Normality
\begin{enumerate}
\item Castelnuovo's theorem: $2g+1$ is projectively normal.
\item Liaison of curves in $\PP^3$.
 \end{enumerate}

\item Parameter space (open problems to be discussed in BN chapter or following.
\begin{enumerate}
\item Survey: Hilbert Scheme, Hurwitz Variety, Moduli
\item Irreducibility (or not) and dimension. 
\end{enumerate}

\item Inflectionary behavior and the Pl\"ucker formulas
\begin{enumerate}
 \item Pl\"ucker formulas
 \item Weierstrass Points
 \item finiteness of the automorphism group
\end{enumerate}

\item Brill Noether
\begin{enumerate}

\item Proof of 1/2 BN by cuspidal curves
\item Reference to the appendix of 3264 for the existence half
\item generic $g^2_*$ is birational; $g^3_*$ is very ample (on a general curve)
\end{enumerate}




%\item
%\begin{enumerate}
%\item Broader View: 
%\end{enumerate}



\end{enumerate}



\begin{thebibliography}{ABC99}

%\bibitem[1965]{BR} D. Buchsbaum and D.S.Rim.
%A generalized Koszul complex. III. A remark on generic acyclicity.
%Proc. Amer. Math. Soc. 16 (1965) 555--558. 

\bibitem[Walker]{Walker} Walker.
\bibitem[Hartshorne]{Hartshorne} Hartshorne Ch 4
\bibitem[Fulton]{Fulton} Fulton Alg curves
\bibitem[Schemes]{Eisenbud-Harris} Schemes
\bibitem[3264]{Eisenbud-Harris} 3264
\bibitem[Griffiths-Harris]{Griffiths-Harris} (for the Abel-Jacobi stuff)
\bibitem[Griffiths]{Griffiths-Chinese}(for the Abel-Jacobi stuff)
\bibitem[Mumford]{Mumford}-Curves and their Jacobians
\bibitem[Voisin]{Voisin} Hodge Theory
%\bibitem[]{Smith} Smith, K.

\end{thebibliography}
\bigskip

\vbox{\noindent Author Addresses:\par
\smallskip
\noindent{David Eisenbud}\par
\noindent{Department of Mathematics, University of California, Berkeley,
Berkeley CA 94720}\par
\noindent{eisenbud@math.berkeley.edu}\par
}

\end{document}


