%header and footer for separate chapter files

\ifx\whole\undefined
\documentclass[12pt, leqno]{book}
\usepackage{graphicx}
\usepackage{eps-to-pdf}
\input style-for-curves.sty
%\input sl-macros.sty
\usepackage{hyperref}
\usepackage{showkeys} %This shows the labels.
\usepackage{msribib}
\usepackage{pdfpages}
\usepackage{draftwatermark}
\SetWatermarkText{DRAFT:\ \today}
\SetWatermarkScale{2}
\SetWatermarkColor[gray]{0.9}

%\usepackage{SLAG,msribib,local}
%\usepackage{amsmath,amscd,amsthm,amssymb,amsxtra,latexsym,epsfig,epic,graphics}
%\usepackage[matrix,arrow,curve]{xy}
%\usepackage{graphicx}
%\usepackage{diagrams}
%
%%\usepackage{amsrefs}
%%%%%%%%%%%%%%%%%%%%%%%%%%%%%%%%%%%%%%%%%%
%%\textwidth16cm
%%\textheight20cm
%%\topmargin-2cm
%\oddsidemargin.8cm
%\evensidemargin1cm
%
%%%%%%Definitions
%\input preamble.tex
%\input style-for-curves.sty
%\def\TU{{\bf U}}
%\def\AA{{\mathbb A}}
%\def\BB{{\mathbb B}}
%\def\CC{{\mathbb C}}
%\def\QQ{{\mathbb Q}}
%\def\RR{{\mathbb R}}
%\def\facet{{\bf facet}}
%\def\image{{\rm image}}
%\def\cE{{\cal E}}
%\def\cF{{\cal F}}
%\def\cG{{\cal G}}
%\def\cH{{\cal H}}
%\def\cHom{{{\cal H}om}}
%\def\h{{\rm h}}
% \def\bs{{Boij-S\"oderberg{} }}
%
%\makeatletter
%\def\Ddots{\mathinner{\mkern1mu\raise\p@
%\vbox{\kern7\p@\hbox{.}}\mkern2mu
%\raise4\p@\hbox{.}\mkern2mu\raise7\p@\hbox{.}\mkern1mu}}
%\makeatother

%%
%\pagestyle{myheadings}

%\input style-for-curves.tex
%\documentclass{cambridge7A}
%\usepackage{hatcher_revised} 
%\usepackage{3264}
   
\errorcontextlines=1000
%\usepackage{makeidx}
\let\see\relax
\usepackage{makeidx}
\makeindex
% \index{word} in the doc; \index{variety!algebraic} gives variety, algebraic
% PUT a % after each \index{***}

\overfullrule=5pt
\catcode`\@\active
\def@{\mskip1.5mu} %produce a small space in math with an @

\title{A Chapter from ``The Practice of Algebraic Curves"}
\author{\copyright David Eisenbud and Joe Harris}
%%\includeonly{%
%0-intro,01-ChowRingDogma,02-FirstExamples,03-Grassmannians,04-GeneralGrassmannians
%,05-VectorBundlesAndChernClasses,06-LinesOnHypersurfaces,07-SingularElementsOfLinearSeries,
%08-ParameterSpaces,
%bib
%}

\date{\today}
%%\date{}
%\title{Curves}
%%{\normalsize ***Preliminary Version***}} 
%\author{David Eisenbud and Joe Harris }
%
%\begin{document}

\begin{document}
\maketitle

\pagenumbering{roman}
\setcounter{page}{5}
%\begin{5}
%\end{5}
\pagenumbering{arabic}
\tableofcontents
\fi


\chapter{Interlude on Moduli}
\label{Moduli chapter}

\section{Moduli problems}

It is a fundamental aspect of algebraic geometry that the objects we deal with often vary in families, and can often be parametrized by a ``universal" such family. This notion of objects varying with parameters underlies many of the constructions and theorems we'll be discussing in this book, and so it seems like a good idea to establish the basic facts about moduli spaces in general, and the particulars of the ones we'll be dealing with here.

\subsection{What is a moduli problem?}

Briefly, a \emph{moduli problem} consists of two things: a class of objects, or isomorphism classes of objects; and a notion of what it means to have a \emph{family} of these objects parametrized by a given scheme $B$. To make this relatively explicit, the four main examples of moduli problems we'll be discussing here are:

\begin{enumerate}
\item objects are isomorphism classes  of smooth, projective curves $C$ of a given genus $g$; by a family we'll mean a flat, smooth, projective morphism $\cX \to B$.

\item objects are smooth, projective curves $C \subset \PP^r$ of degree $d$ and genus $g$; a family over $B$ will be a subscheme $\cX \subset B \times \PP^r$, smooth, projective and flat over $B$, whose fibers are curves of degree $d$ and genus $g$

\item objects are effective divisors of a given degree $d$ on a given smooth, projective curve $C$; families over $B$ will be subschemes $\cD \subset B \times C$ flat over $B$ with fibers of degree $d$

\item line bundles of a given degree $d$ on a given smooth, projective curve $C$; by a family of such line bundles over a scheme $B$ we'll mean a line bundle on the product $B \times C$ whose restriction to each fiber over $B$ has degree $d$, modulo tensor product with line bundles pulled back from $B$.
\end{enumerate}

Given a moduli problem, our goal will be to describe a corresponding \emph{moduli space}. By this we mean a scheme $M$ whose points are in natural one-to-one correspondence with the objects in our moduli problem; or, to put it differently, we want to realize the set of objects in our moduli problem as the underlying set of a scheme $M$.

The issue here, as it so often is,  is the word ``natural." If we're working over $\CC$, for example, all positive dimensional varieties have the same cardinality ($\aleph_1$), so saying that we have a bijection between the points of a variety $M$ and the set of isomorphism classes of curves doesn't characterize $M$. Rather, we need some basic condition on the bijection, for which the word ``natural" is a stand-in. 

In the pre-Grothendieckian world of varieties, it was easy to express this condition, though the result (as we'll see) was not always satisfactory. Given a family of the objects in our moduli problem over a variety $B$, we get a map from the underlying set of $B$ to the underlying set of $M$; and the requirement was simply that this map of sets defined a regular morphism of varieties.

In the post-Grothendieckian world of schemes, however, this doesn't work: a morphism of schemes is not determined by the associated map of sets. The solution to this 

The definition is straightforward: by a family of smooth, projective curves $C$ of a given genus $g$ over a given base variety $B$, we mean a flat morphism
$\cC \to B$ whose fibers are smooth, projective curves $C$ of genus $g$

\

In Chapter~\ref{new Jacobians chapter}, we saw how to parametrize the set of line bundles of degree $d$ on a given curve $C$ by the points of an algebraic variety, the \emph{Picard variety} $\Pic^d(C)$. The theory of  Jacobians of curves is an enormously rich and deep one, but leaving all that aside, just the existence of a parameter space makes a big difference: it allows us, for example, to use dimension estimates to prove theorems like Theorem~\ref{g+3 theorem}, and will allow us to give a much stronger version of the Brill-Noether theorem in Section~\ref{BNomnibus} below.

In fact, parameter space are ubiquitous in algebraic geometry. The fact that under reasonable circumstances the set of objects of a given type---schemes, sheaves on a scheme, morphisms between schemes---may be put naturally in one-to-one correspondence with the points of a variety is one of the most essential characteristics of the subject.

In this chapter, we'll introduce some of the basic parameter spaces relevant to algebraic curves. We won't actually construct any of them---God forbid!---but we'll indicate basic facts about each.

\section{What is a parameter space?}

\section{Hilbert schemes, Severi varieties and Hurwitz spaces}

\section{Moduli spaces of curves}

\section{The Brill-Noether theorem: omnibus version}\label{BNomnibus}

%footer for separate chapter files

\ifx\whole\undefined
\makeatletter\def\@biblabel#1{#1]}\makeatother
\gdef\urlhook{\url}
\bibliography{slag}
\bibliographystyle{msribib}


%%%% EXPLANATIONS:

% f and n
% some authors have all works collected at the end

\catcode`\^\active
%if ^ is followed by 
% 1:  print f, gobble the following ^ and the next character
% 0:  print n, gobble the following ^
% any other letter: print letter
\makeatletter
\def^#1{\ifx1#1f\expandafter\@gobbletwo\else
        \ifx0#1n\expandafter\expandafter\expandafter\@gobble\else#1\fi\fi}
\makeatother
\let\moreadhoc\relax
\def\indexintro{%An author's cited works appear at the end of the
%author's entry; for conventions
%see the List of Citations on page~\pageref{loc}.  
%\smallbreak\noindent
The letter `f' after a page number indicates a figure, `n' a footnote.}
\printindex[gen]
%requires makeindex
\end{document}
\else
\fi
