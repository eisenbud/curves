%header and footer for separate chapter files

\ifx\whole\undefined
\documentclass[12pt, leqno]{book}
\usepackage{graphicx}
\usepackage{eps-to-pdf}
\input style-for-curves.sty
%\input sl-macros.sty
\usepackage{hyperref}
\usepackage{showkeys} %This shows the labels.
\usepackage{msribib}
\usepackage{pdfpages}
\usepackage{draftwatermark}
\SetWatermarkText{DRAFT:\ \today}
\SetWatermarkScale{2}
\SetWatermarkColor[gray]{0.9}

%\usepackage{SLAG,msribib,local}
%\usepackage{amsmath,amscd,amsthm,amssymb,amsxtra,latexsym,epsfig,epic,graphics}
%\usepackage[matrix,arrow,curve]{xy}
%\usepackage{graphicx}
%\usepackage{diagrams}
%
%%\usepackage{amsrefs}
%%%%%%%%%%%%%%%%%%%%%%%%%%%%%%%%%%%%%%%%%%
%%\textwidth16cm
%%\textheight20cm
%%\topmargin-2cm
%\oddsidemargin.8cm
%\evensidemargin1cm
%
%%%%%%Definitions
%\input preamble.tex
%\input style-for-curves.sty
%\def\TU{{\bf U}}
%\def\AA{{\mathbb A}}
%\def\BB{{\mathbb B}}
%\def\CC{{\mathbb C}}
%\def\QQ{{\mathbb Q}}
%\def\RR{{\mathbb R}}
%\def\facet{{\bf facet}}
%\def\image{{\rm image}}
%\def\cE{{\cal E}}
%\def\cF{{\cal F}}
%\def\cG{{\cal G}}
%\def\cH{{\cal H}}
%\def\cHom{{{\cal H}om}}
%\def\h{{\rm h}}
% \def\bs{{Boij-S\"oderberg{} }}
%
%\makeatletter
%\def\Ddots{\mathinner{\mkern1mu\raise\p@
%\vbox{\kern7\p@\hbox{.}}\mkern2mu
%\raise4\p@\hbox{.}\mkern2mu\raise7\p@\hbox{.}\mkern1mu}}
%\makeatother

%%
%\pagestyle{myheadings}

%\input style-for-curves.tex
%\documentclass{cambridge7A}
%\usepackage{hatcher_revised} 
%\usepackage{3264}
   
\errorcontextlines=1000
%\usepackage{makeidx}
\let\see\relax
\usepackage{makeidx}
\makeindex
% \index{word} in the doc; \index{variety!algebraic} gives variety, algebraic
% PUT a % after each \index{***}

\overfullrule=5pt
\catcode`\@\active
\def@{\mskip1.5mu} %produce a small space in math with an @

\title{A Chapter from ``The Practice of Algebraic Curves"}
\author{\copyright David Eisenbud and Joe Harris}
%%\includeonly{%
%0-intro,01-ChowRingDogma,02-FirstExamples,03-Grassmannians,04-GeneralGrassmannians
%,05-VectorBundlesAndChernClasses,06-LinesOnHypersurfaces,07-SingularElementsOfLinearSeries,
%08-ParameterSpaces,
%bib
%}

\date{\today}
%%\date{}
%\title{Curves}
%%{\normalsize ***Preliminary Version***}} 
%\author{David Eisenbud and Joe Harris }
%
%\begin{document}

\begin{document}
\maketitle

\pagenumbering{roman}
\setcounter{page}{5}
%\begin{5}
%\end{5}
\pagenumbering{arabic}
\tableofcontents
\fi


\chapter{Moduli of Curves}
\label{Moduli chapter}

Philosophy of this chapter: the book \emph{Moduli of Curves} has already been written, and we don't want to write it again. But a summary of the basic information would be useful

\section{$M_g$}

Assert existence as a coarse, rather than a fine moduli space (say what this means and give ref to Geometry of Schemes?)

\section{Compactifying $M_g$}

Describe the Deligne-Mumford compactification; mention alternatives?

\section{Auxilliary constructions}

It is hard to specify an abstract curve. It is much easier if the curve $C$ comes to us with some additional structure, such as a map to projective space; if the map is a birational embedding, we can specify the curve just by specifying a set of polynomial equations cutting it out. 

\subsection{Hurwitz spaces}

Fix integers $d \geq 2$ and $g \geq 0$. By the \emph{small Hurwitz space} $\cH^\circ_{d,g}$ we will mean a space parametrizing simply branched covers $f : C \to \PP^1$ of degree $d$, with $C$ a smooth projective curve of genus $g$. Here ``simply branched" means that every fiber  either is reduced---that is, consists of $d$ reduced points---or consists of one double point and $d-2$ reduced points. If a map $f : C \to \PP^1$ is simply branched, in particular, the branch divisor $B \subset \PP^1$ of the map will consist of $b = 2d+2g-2$ distinct points in $\PP^1$

What does such a space look like? The answer is easiest to see if we work over $\CC$ and use the classical or \'etale topology; in this setting, we consider the incidence correspondence
\begin{diagram}
& & \cH_{d,g} & & \\
& \ldTo^\alpha & & \rdTo^\beta & \\
U \subset \PP^b & & & & M_g
\end{diagram}
Here $\alpha$ is the map associating to a branched cover $f : C \to \PP^1$ its branch divisor, and $\beta$ the map sending $f : C \to \PP^1$ to the point $[C] \in M_g$; the open set $U \subset \PP^b$ is the open set of $b$-tuples of distinct points in the space $\PP^b$ of all effective divisors of degree $b$ on $\PP^1$.

Over $\CC$, we can describe a branched cover concretely: if we make a collection of cuts in $\PP^1$ joining a base point $p$ to each of the branch points $p_1, p_2, \dots, p_b$ of the map, the preimage in $C$ of the complement of the cuts will consist of $d$ disjoint copies of the complement of the cuts in $\PP^1$ (the ``sheets" of the cover), which we can label with the integers $1, 2, \dots, d$. In these terms, we can associate to each branch point $p_i \in \PP^1$ the transposition $\tau_i \in S_d$ exchanging the two sheets that come together over $p_i$. We arrive at a sequence of transpositions $\tau_1, \tau_2, \dots, \tau_b \in S_d$, with two conditions:

\begin{enumerate}
\item the product $\tau_1\cdot \tau_2 \cdots \tau_b$ is the identity; and
\item the $\tau_i$ together generate a transitive subgroup of $S_d$.
\end{enumerate}

Note that the sequence $\tau_1, \tau_2, \dots, \tau_b \in S_d$ is determined by the cover $f : C \to \PP^1$ up to simultaneous conjugation in $S_d$: we can revise our labelling of the sheets, which has the effect of conjugating all the $\tau_i$ by the relabelling permutation.

The conclusion is simply that the map $\alpha : \cH_{d,g} \to U$ is a covering space map, which gives us our first picture of the geometry of $\cH_{d,g}$. For example, this was the basis of the first proof that $M_g$ is irreducible: Clebsch, Hurwitz and others (**??**) analyzed the monodromy of the cover $\alpha : \cH_{d,g} \to U$, and showed that it was indeed transitive; they concluded that \emph{$\cH_{d,g}$ is irreducible for all $d$ and $g$} and hence, since $\cH_{d,g}$ dominates $M_g$ for $d \gg g$, that $M_g$ is irreducible for all $g$.

The Hurwitz spaces also give us a way to estimate the dimension of the moduli space $M_g$. The point is, while it may not be immediately obvious what the dimension of $M_g$ is, the dimension of $\cH_{d,g}$ is clear: it's a finite-sheeted cover of an open subset $U \subset \PP^b$, and so has dimension  $b = 2d+2g-2$. To find the dimension of $M_g$, accordingly, we simply have to choose $d \gg g$ (so that $\cH_{d,g}$ dominates $M_g$), and estimate the dimension of the fibers of $\cH_{d,g}$ over $M_g$.

This is straightforward, based on our previous constructions. Given a curve $C$, to specify a map $f : C \to \PP^1$ we have to specify first a line bundle $L$ of degree $d$ on $C$ ($g$ parameters, as described in Chapter~\ref{new Jacobians chapter}. We then have to specify a pair of sections of $L$ (up to multiplying the pair by a scalar). By Riemann-Roch, we will have $h^0(L) = d-g+1$, so to specify a pair of sections (mod scalars) is $2(d-g+1)-1$ parameters. Altogether, we have
$$
2d+2g-2 = \dim \cH_{d,g} = \dim M_g + g + 2(d-g+1)-1;
$$
and solving, we arrive at
$$
\dim M_g \; = \; 3g-3.
$$




\subsection{Severi varieties}

Deduce: $M_g$ is irreducible of dimension $3g-3$

\section{Unirationality}

Can you write down a general curve of genus $g$?

%footer for separate chapter files

\ifx\whole\undefined
\makeatletter\def\@biblabel#1{#1]}\makeatother
\gdef\urlhook{\url}
\bibliography{slag}
\bibliographystyle{msribib}


%%%% EXPLANATIONS:

% f and n
% some authors have all works collected at the end

\catcode`\^\active
%if ^ is followed by 
% 1:  print f, gobble the following ^ and the next character
% 0:  print n, gobble the following ^
% any other letter: print letter
\makeatletter
\def^#1{\ifx1#1f\expandafter\@gobbletwo\else
        \ifx0#1n\expandafter\expandafter\expandafter\@gobble\else#1\fi\fi}
\makeatother
\let\moreadhoc\relax
\def\indexintro{%An author's cited works appear at the end of the
%author's entry; for conventions
%see the List of Citations on page~\pageref{loc}.  
%\smallbreak\noindent
The letter `f' after a page number indicates a figure, `n' a footnote.}
\printindex[gen]
%requires makeindex
\end{document}
\else
\fi
