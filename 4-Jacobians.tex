\input header.tex

\chapter{Jacobians}\label{new Jacobians chapter}


An essential construction in studying a curve $C$ is the association to a divisor  of an invertible sheaf---in other words, the map
$$
\big\{ \text{effective divisors of degree $d$ on C}\big\} \rTo^\mu \big\{ \text{invertible sheaves of degree $d$ on C} \big\}.
$$
sending $D$ to $\cO_C(D)$.

A priori, this is a map of sets. But it is a fundamental fact that each set may  be given the structure of an algebraic variety in a natural way, so that the map between them is regular. The geometry of this map governs the geometry of the curve in many ways.
In this chapter we will describe the source and target of $\mu$, and give references to proofs of their properties. 

We start with the effective divisors. Since $C$ is smooth, an effective divisor of degree $d$ on $C$ is the same thing as a subscheme $D \subset C$ of dimension 0 and degree $d$, and thus
the family $C_d$ of effective divisors of degree $d$ on $C$ is a Hilbert scheme; see~\ref{hilbert scheme section}. This Hilbert scheme may be identified with
the $d$-th \emph{symmetric power} $C^(d)$  of $C$, described in Section~\ref{symmetric section}. 

The parametrization of the set of invertible sheaves on $C$ of a given degree $d$ by the variety $\Pic_d(C)$ requires different techniques. We will define it by a universal property in the category of schemes, and exhibit its construction as an analytic variety, actually a complex torus $\Jac(C)$, whose group structure reflects the tensor product of
sheaves in $\Pic_0(C)$.
Historically, the algebraic construction was a major milestone first reached in the work of Andre Weil in the middle of
the 20th century, and was the reshaped by Grothendieck and his school. The interested reader will find a beautiful, detailed account both of the history and the 
modern theory of the scheme of divisors and the Picard scheme in the exposition~\cite{Kleiman-PicardScheme}. We will give
more specific references to this text as we go, and the reader will find extensive references to the original literature there.

As an application of the mere existence of the spaces $C_d$ and $\Pic_d$, we show in Theorem~\ref{g+3 theorem} that a general divisor of degree $g+3$ on any curve of genus $g$ gives rise to an embedding in $\PP^3$ as a curve of degree $g+3$. Similar techniques, in Section~\ref{g+2 section} show that a general divisor of degree $g+2$ on any curve gives a birational morphism to a plane curve of degree $g+2$. Moreover, if the curve is not hyperelliptic then its image in $\PP^2$ has  only nodes
as singularities, while in the hyperelliptic case its only singularity is one ordinary $g$-fold point. 

\section{Symmetric products and the universal divisor}\label{symmetric section}

Let $C$ be a smooth curve. The space of effective divisors on $C$ can be characterized by a universal property. 

\begin{definition}
Let $B$ be any scheme. A \emph{family} of effective divisors of degree $d$ on $C$, parametrized by the scheme $B$, is a Cartier divisor $X\subset B\times C$ whose intersection with fibers $\{b\} \times C \cong C$ over points of $B$ are divisors of degree $d$ on $C$.
\end{definition}

Given this, we have a contravariant functor 
$$
F : (schemes) \to (sets),
$$
defined by taking a scheme $B$ to the set of families of divisors of degree $d$ over $B$; if $\pi : B' \to B$ is any morphism, the induced map $F(B) \to F(B')$ is defined by taking a family $\cD \subset B \times C$ to the preimage of $\cD$ under the map $\pi \times Id : B' \times C \to B \times C$. We say that a scheme $C_d$ is a fine moduli space for divisors of degree $d$ on $C$ if we have an isomorphism of functors
$$
F \cong \Hom_{\rm{Schemes}}( -, C_d).
$$
By Yoneda's Lemma, this is equivalent to the existence of a \emph{universal family} $\cD \subset C_d \times C$, with the property that for any family $X \subset B \times C$ of divisors on $C$ over any scheme $B$, there is a unique map $\phi : B \to C_d$ such that $X = (\phi \times id_{C})^{-1}(\cD)$.

From the universal property it is clear that a fine moduli space for divisors of degree $d$ on $C$ is unique if it exists. Indeed, it does exist, and we'll sketch the construction, using symmetric products. This construction relies on the existence of quotients of schemes by finite groups, and we'll pause here to discuss such quotients.

\subsection{Finite group quotients}

If $G$ is a finite group acting by automorphisms on an affine scheme $X:=\Spec A$ then $X/G$ is by definition $\Spec(A^G)$, the spectrum of the ring $A^G$ of invariant elements of $A$. The following result shows that this definition reflects the desired
geometry: 

\begin{theorem}\label{finite invariant theory}
 The map $\pi: X\to X/G$ induced by the inclusion of rings is finite. If $X$ is a normal variety, then the fibers of $\pi$  are the orbits of $G$.
\end{theorem}
For a proof, see for example \cite[Proposition 13.10]{Eisenbud1995}).  

Since the map $X\to X/G$ is finite, we have $\dim X/G = \dim X$. 

The construction commutes with the passage to $G$-invariant open affine sets, and thus passes to more general schemes---and in particular to projective schemes (Exercise~\ref{quotient of projective})---as well.

When the group $G$ is infinite, the situation becomes much more complex---see Chapter~\ref{ModuliChapter} for some  idea of what can and cannot be done.

For example, if $X$ is any scheme or any quasi-projective variety $X$ we define the \emph{$d$-th symmetric power $X^{(d)}$ of $X$} to be the quotient of the Cartesian product $X^d$ of $d$ copies of $X$ by the action of the group of permutations of the factors. 

%Since an effective divisor of degree $d$ on a curve $C$ is an unordered $d$-tuple of points on $C$, with repetitions allowed, it corresponds to a point in the \emph{$d$th symmetric power} $C^{(d)}$. For this reason we will write the points of $\Cd$ as $d$-tuples.

If $X=\AA^{1}$ then $X^{d} = \AA^{d} = \Spec \CC[x_{1}, \dots, x_{d}]$. The ring of invariants of the symmetric group acting on
$\CC[x_{1}, \dots, x_{d}]$ by permuting the variables is generated by the $d$ elementary symmetric functions, which generate a polynomial subring, and $X^{(d)}$ is isomorphic to $\AA^{d}$. Since the symmetric functions of the roots of a polynomial in one variable are the coefficients of
that polynomial, we may identify $X^{(d)}$ with the affine space of monic polynomials of degree $d$ in $\CC[z]$. (\cite[Exercises 1.6, 13.2-13.4]{Eisenbud1995})

Next consider $X = \PP^1$ and the product $(\PP^1)^d$. Taking the homogeneous coordinates of the
$i$-th copy of $\PP^1$ to be $(s_i,t_i)$, the $d+1 $multilinear symmetric functions of degree $d$,
$$
t_1t_2\cdots t_d,\ s_1t_2\cdots t_d,\ \dots,\ s_1\cdots s_d
$$
are the coefficients of the polynomial
$$
(s_1\lambda + t_1\mu)(s_2\lambda + t_2\mu)\cdots(s_d\lambda + t_d\mu)
$$
defining a general subcheme of $\PP^1$ of degree $d$ and  define
an isomorphism $(\PP^1)^{(d)} \to \PP^d$.
Again, we may think of this map as taking a $d$-tuple of points to the unique-up-to-scalars
homogeneous form of degree $d$ vanishing on it.

We shall see that when $C$ is a smooth curve of higher genus, the global geometry of $C^{(d)}$ is quite nontrivial, but at least
the local geometry is simple:

\begin{proposition}
If $C$ is a smooth curve then each symmetric power $C^{(d)}$ is smooth.
\end{proposition}

\begin{proof}
 The general case follows from the case of $\AA^{1}$ because locally analytically the action of the symmetric group on $C^d$ is the same as for $\AA^1$: If  $\overline p \in X^{(d)}$, then it suffices to
 show that the quotient of an invariant formal neighborhood of the preimage $p_1,\dots, p_s$ of
 $\overline p$ is smooth. After completing the local rings, we get an action of the symmetric group
 $G$ on the product of the completions of $X$ at the $p_i$, and this depends only on the orbit
 structure of $G$ acting on $\{p_1,\dots, p_s\}$. Thus it would be the same for some orbit of
 points on $\AA^1$.
 \end{proof}

By contrast, if $\dim X \geq 2$ then the symmetric powers $X^{(d)}$ are singular for all $d \geq 2$.
See Exercise~\ref{sym2A2} for the case of $(\AA^2)^{(2)}$ and Exercise~\ref{free actions} for a well-behaved case.

It is clear from Theorem~\ref{finite invariant theory} that the points of $C^{(d)}$ are in one-to-one correspondence with the effective divisors of
degree $d$ on $C$, but much more is true:

\begin{theorem}
 If $C$ is a smooth projective curve, then the $d$th symmetric power $C^{(d)}$ of $C$ is the fine moduli space $C_d$ for effective divisors of degree $d$ on $C$.
\end{theorem}
For a proof in the analytic category, see \cite[]{ACGH}; for the algebraic fact, see \cite[Remark 9.3.9]{Kleiman-PicardScheme}.
Henceforward we will write $C_d$ in place of $C^(d)$ for this space.


%Finally we come to the definition of the universal family:
%
%\begin{fact}
% The universal family of degree $d$ divisors on $C$ is the incidence correspondence $\sD\rTo^\alpha \Cd$ where
% $\sD \subset C \times \Cd$ is the incidence correspondence 
%$$
%\sD := \{(x,(x_1,\dots x_d)) \mid x = x_i\hbox{ for some }i\}.
%$$
%and thus the Hilbert scheme of degree $d$ subschemes of $C$ is $\Cd$.
%
%If $X \to C\times B$ is a family of divisors of degree $d$ on $C$ then we may define a set-theoretic map $\phi: B\to \Cd$ by sending $b\in B$ to the
%unordered $d$-tuple of points of the divisor that is the fiber over $b$. This together with the composition $X \to C \rTo^1 C$
%gives us the diagram in the Theorem, and it can be shown that the map $\phi$ is a map of schemes.
%\end{fact}

%\fix{seems a pity not to prove this -- we prove so little in this Chapter Kleiman's reference are to Deligne and the book by Bosch et al on Neron-Severi.}

\section{The Picard varieties}

As with $C_d$, we may define $\Pic_d(C)$ by a universal property. We start by saying what we mean by a family of invertible sheaves on a smooth curve $C$:

\begin{definition}
 For any scheme $B$, a \emph{family of invertible sheaves on $C$ over $B$} is an equivalence class of invertible sheaves $\sL$ on $B\times C$, where two such
 families $\sL$ and $\sL'$are equivalent if they differ by an invertible sheaf pulled back from $B$, that is, if
 $$
 \sL' = \sL \otimes \pi_1^*\cF
 $$
for some invertible sheaf $\cF$ on $B$.
 \end{definition}

If $p \in C$ and $\cF$ is a sheaf on $B$ then $\pi_1^*(\cF)\mid_{B\times p} = \cF$, so we could have eliminated the
equivalence relation at the expense of choosing a point by insisting that the restriction of $\sL$ to $B \times \{p\}$ be trivial.
 
% Thus, given a point $p\in C$, and family of invertible sheaves
% is equivalent to a unique one whose restriction to $\{p\}\times B$ is trivial. 
 

 
We say that $\sL$  is a family of sheaves of degree $d$ if the restriction of $\sL$
 to $C\times \{b\}$ is $d$ for each point $b\in B$. 
 
We define the functor
 $$
 Pic_d : (schemes) \to (sets)
 $$
 by associating to any scheme $B$ the set of invertible sheaves of degree $d$ on $B \times C$, modulo tensoring with pullbacks of invertible sheaves on $B$. Since the tensor product and inverse of an invertible sheaf of degree 0 again has degree 0, 
 $Pic_0$ factors through the category of abelian groups. These functors are representable by schemes:
  
 \begin{fact}\cite[Theorem 9.4.8]{Kleiman-PicardScheme}
 There exists a fine moduli space $\Pic_d(C)$ for invertible sheaves of degree $d$ on $C$; that is, a scheme $\Pic_d(C)$ such that for any scheme $B$ we have a natural bijection between families of invertible sheaves of degree $d$ over $B$, modulo invertible sheaves on $B$, and morphisms $B \to \Pic_d(C)$. The tensor product of invertible sheaves makes $\Pic_0(C)$ an algebraic group, which acts on each $\Pic_d(C)$.
 \end{fact}
 
If $\sL$ is any invertible sheaf of degree $e$ on $C$, we can define a bijection between families of invertible sheaves of degree $d$ over $B$ and families of invertible sheaves of degree $d+e$ over $B$, uniformly for all $B$, by tensoring with the pullback $\pi_2^*\sL$. Thus $\Pic_d(C) \cong \Pic_{d+e}(C)$ (but not canonically, since the isomorphism depends on the choice of $\sL$).
 
 Note also that the variety $\Pic_0(C)$ is a group, with group law given by tensor product of invertible sheaves of degree 0; and for each $d$, $\Pic_d(C)$ is a principal homogeneous space for $\Pic_0(C)$. 
 
The description of the functors represented by the symmetric product $C_d$ and the Picard scheme $\Pic_d$ implies that the 
set-theoretic map
$$
C_d \to \Pic_d \quad D \mapsto \sO_c(D)
$$
is the underlying set map of a morphism of schemes: on the level of functors the map takes a family of divisors $\cD\subset B\times C$
to the 

The characterization of $\Pic_d(C)$ above does shed much light on the geometry of $\Pic_d(C)$: whether it's irreducible, for example, or what its dimension is. But we can get a  good picture from the classical (19th century) construction of the Jacobian, $\Jac(C)$.
 
\section{Jacobians}

The history leading to the analytic construction of the Jacobian starts from a calculus problem. A goal of the 19th century mathematicians was  to make sense of integrals of algebraic functions. In the early development of calculus, mathematicians figured out how to evaluate explicitly integrals such as
$$
\int_{t_0}^t \frac{dx}{\sqrt{x^2+1}}.
$$
Such integrals can be thought of as path integrals of meromorphic differentials on the Riemann surface associated to the equation $y^2 = x^2+1$. This surface is isomorphic to $\PP^1$, meaning that $x$ and $y$ can be expressed as rational functions of a single variable $z$; making the corresponding change of variables transformed the integral into one of the form
$$
\int_{s_0}^s R(z)dz,
$$
with $R$ a rational function, and such integrals can be evaluated by the technique of partial fractions.

When they tried to extend this to similar-looking integrals, such as
$$
\int_{t_0}^t \frac{dx}{\sqrt{x^3+1}},
$$
which arises when one studies the length of an arc of an ellipse, they were stymied. The reason gradually emerged: the problem is that the Riemann surface associated to the equation $y^2 = x^3+1$ is not $\PP^1$, but rather a curve of genus 1, and so has nontrivial homology group $H_1(C, \ZZ) \cong \ZZ^2$. In particular, if one expresses this ``function'' of $t$  as a path integral, then the value depends on a choice of path; it is defined only modulo a lattice $\ZZ^2 \subset \CC$. This implies that the inverse function is a doubly periodic meromorphic function on $\CC$, and not an elementary function. Many new special functions, such as the Weierstrass $\sP$-function were studied as a result. The name ``elliptic curve'' arose from these considerations.

Once this case was understood, the next step was to extend the theory to path integrals of holomorphic differentials on curves of arbitrary genus. One problem is that the dependence of the integral on the choice of path is much worse; the set of homology classes of paths between two points $p_0, p \in C$ is identified with $H_1(C,\ZZ) \cong \ZZ^{2g}$ rather than $\ZZ^2$, rendering the expression $\int_p^q \omega$ for a given 1-form $\omega$ virtually meaningless.

The solution is to  consider the integrals of \emph{all} holomorphic differentials on $C$ simultaneously---in other words, to consider the expression $\int_p^q$ as a linear function on the space $H^0(K_C)$ of all holomorphic differentials on $C$.

To express the resulting construction in relatively modern terms, let $C$ be a smooth projective curve of genus $g$ over $\CC$, and let $\omega_{C}$ be the sheaf of differential forms on $C$. We will consider $C$ as a complex manifold. Every meromorphic differential form is in fact algebraic
\cite{SerreGAGA}, and we consider $\omega_{C}$ as a sheaf in the analytic topology.

We consider the space $V = H^0(\omega_C)^*$ of linear functions on the space of differentials $H^0(\omega_C)$.  Integration over a closed loop in $C$ defines a linear function on 1-forms, so that we have a map
$$
\iota: \ZZ^{2g} = H_1(C,\ZZ) \; \to \;  H^0(\omega_C)^* \cong H^1(\sO_C) = \CC^{g}.
$$
By Hodge theory, 
$$
H^1(C, \CC) \cong H^1(C, \cO_C) \oplus \overline{H^1(C, \cO_C)}
$$
where the bar denotes complex conjugation $H^1(C, \CC)$, and the map $\iota$ is the composition of 
 the natural inclusion with the projection to the first summand.
 Now
$H_1(C,\CC) = \CC\otimes_\ZZ H_1(C,\ZZ)$, so any basis of $H_1(C,\ZZ)$ maps to a basis of 
 $H^1(C, \CC)$ invariant under conjugation in $H^1(C, \CC)$---See~\cite{Voisin} or~\cite[p. 116]{Griffiths-Harris1978}. 

One can show that the image of $\iota$ is a lattice in $H^0(\omega_C)^*$, and thus the quotient
is a torus of real dimension $2g$. Moreover, the
complex structure on $H^0(\omega_C)^*$ yields a complex analytic structure on the quotient $\CC^{g}/\iota(\ZZ^{2g})$, which is thus a complex torus of  dimension $g$.  

\begin{definition}
 The complex torus $\CC^{g}/\iota(\ZZ^{2g})$ is called the \emph{Jacobian} of $C$.
\end{definition}

The point of this construction is that for any pair of points $p, q \in C$, the expression $\int_q^p$ describes a linear functional on $H^0(\omega_C)$, defined up to functionals obtained by integration over closed loops, and thus a point of $J(C)$. We can think of $p-q$ as a divisor of
degree 0, and the map $p-q \mapsto \int_q^p$ extends to a well defined map $\mu_0$ from divisors of degree 0 to $J(C)$ because
$$
\int_q^p +\int_{q'}^{p'} - (\int_q^{p'} +\int_{q'}^p) 
$$
is the integral around a closed path $q\to p\to q'\to p' \to q$.

Further, if we choose a ``base point''  $p\in C$, we can define the holomorphic map
$$
\mu \; : \; C \; \to \; J(C); \quad q\mapsto \int_{p}^{q}
$$
and more generally maps
$$
\mu_d \; : \; C_d \; \to \; J(C): \quad  \quad (q_1,\dots, q_d) \mapsto \sum_i \int_{p}^{q_i}.
$$

These maps are called the \emph{Abel-Jacobi} maps. When there is no ambiguity about $d$, we will denote all these maps  by $\mu$,  and  
we define $\mu(-D)$ to be $-\mu(D)$. 
The map $\mu$ is a group homomorphism in the sense that if $D, E$ are divisors, then
$\mu (D+E) = \mu(D) + \mu(E)$; this is immediate when the divisors are effective, and 
follows in general because the group of divisors is a free group.

\section{Abel's theorem}
 The connection between the discussion above and the geometry of linear series is made by one of the landmark theorems of the 19th century :

\begin{theorem}[Abel's Theorem]\label{abel}
Two divisors $D, D'$ on $C$ are linearly equivalent if and only if their images under the Abel-Jacobi map are equal, $\mu(D) = \mu(D')$; in other words, the fibers of $\mu_d$ are the complete linear systems of degree $d$ on $C$. Thus $\mu_0$ induces a canonical isomorphism
$\Pic_0(C) \to J(C)$ and after choosing a base point $p$ an isomorphism $\Pic_d(C) \to Jac(C)$, factoring through the isomorphism
$$
\Pic_d(C) \rTo^{-\otimes \sO_C(-dp)} \Pic_0(C) \to J(C).
$$
\end{theorem}


See \cite[Section 2.2]{Griffiths-Harris1978}  for a complete proof of Theorem~\ref{abel}; we will just prove the ``only if" part. This was in fact the only part proved by Abel; the converse, which is substantially more subtle, was proved by Clebsch.

\begin{proof}[Proof of ``only if'']
Suppose that $D$ and $D'$ are linearly equivalent; that is, $\cO_C(D) \cong \cO_C(D')$. Call this invertible sheaf $\cL$, and suppose that $D$ and $D'$ are the zero divisors of sections $\sigma, \sigma' \in H^0(\cL)$.
Taking linear combinations of $\sigma$ and $\sigma'$, we get a pencil $\{D_\lambda\}_{\lambda \in \PP^1}$ of divisors on $C$, with
$$
D_\lambda \; = \; V(\lambda_0\sigma + \lambda_1\sigma'),
$$
and by the universal property of the symmetric product, this corresponds to a regular map $\alpha : \PP^1 \to C^{(d)}$. 

Consider the composition
$$
\phi = \mu \circ \alpha \; : \; \PP^1 \; \to \; J(C).
$$
 If $z$ is any linear functional on $V = H^0(\omega_C)^*$, then, the differential $dz$  on $V$ descends to a global holomorphic 1-form on
 $J(C)$, which is the quotient of $V$ by a discrete lattice. Thus the regular one-forms on $J(C)$ generate the cotangent space to $J(C)$ at every point. But for any 1-form $\omega$ on $J(C)$, the pullback $\phi^*\omega$ is a global holomorphic 1-form on $\PP^1$, and hence identically zero. It follows that the differential $d\phi$ vanishes identically, and hence that $\phi$ is constant, proving that $\mu(D) = \mu(D')$.
\end{proof}

Andre Weil applied Abel's theorem to describe the structure of the Jacobian algebraically:

\begin{corollary}
If $C$ is a smooth curve of genus $g$ then the Abel-Jacobi map $\mu_{g}: C^{(g)} \to J(C)$ is a surjective birational map.
More generally, $\mu_{d}$ is surjective for $d\geq g$ and generically injective for $d\leq g$.
\end{corollary}

\begin{proof}
For $d\leq g = \dim H^{0}(\omega_{C})$,  a divisor $D$ that is the sum of $d$ general points $p_{1}, \dots,  p_{d} \in C$ will impose independent vanishing conditions on the sections of $\omega_{C}$, and thus
$$
h^0(\omega_C(-D)) = g-d,
$$
 Using this, the Riemann-Roch formula gives $h^{0}\cO_{C}(D) = 1$, so the fiber of 
$\mu_{d}$ consists of a single point, proving generic injectivity. In particular when $d= g$, the image of $\mu_{d}$ has
dimension $g$, and since $C^{(g)}$ is compact, the image is closed, so it must be equal to $J(C)$.

Similarly, if $d \geq g$ then $h^0(\omega_C(-D)) = 0$ and hence $r(D) = d-g= \dim C^{(d)} - \dim J(C)$, and it follows that
$\mu_{d}$ is surjective.
\end{proof}

The image of $C_d$ in $J(C)\cong \Pic_d(C)$ can be identified with the set of invertible sheaves 
A priori, these constructions take place in the analytic category; but in fact they are all algebraic:

\begin{fact}
$J(C)$ has the structure of an algebraic group of finite type over $\CC$ (and can be defined over any field where $C$ is defined) and the Abel-Jacobi maps are
maps of algebraic varieties~\cite[]{Kleiman-PicardScheme}.
\end{fact}

\subsubsection{The differential of the Abel-Jacobi map}

Consider once more the  Abel-Jacobi map
$$
\begin{aligned}
\mu_1 : \; &C \to J(C) := H^0(K_C)^*/H_1(C, \ZZ) \\
&q \mapsto \int_p^q .
\end{aligned}
$$
To compute the differential of $\mu_1$ at a point $p\in C$ we choos a local parameter $z$ at $p$.
We can then write any differential, locally at $p$, in the form $f(z)dz$ where $f$ is a function analytic on
a neighborhood of $p$. The choice of $z$ also serves to identify the tangent and cotangent space of $C$ with $\CC$.

 For any $q\in C$ close to $p$ we have
$$
\frac{\partial}{\partial q} \int_q^p f(z)dz = f(q)
$$

The tangent space to $J(C) =  H^0(K_C)^*/H_1(C, \ZZ)$ at any point is $H^0(K_C)^*$, so the cotangent space is $H^0(K_C)$. The co-differential 
$T^*(\mu_1)O(p)$ of $\mu_1$, evaluated at $p$, thus sends the element $\omega\in H^0(K)$ to the functional on $f(p)\in \CC = T^*_{C,p}$. 
The kernel of $T^*(\mu_1)O(p)$ is thus the set of global differential forms that vanish at $p$.
Since $(\omega_C, H^0(\omega_C))$ is base-point free, there is a global differential $\omega$ 
with local expression $\omega = f(z)dz$ such that $f(p) \neq 0$; this shows that the codifferential is a surjection, and dually the
differential $T(\mu_1)(p)$ carries the tangent space to $C$ at $p$ to the line in $H^0(K_C)^*$ corresponding to the hyperplane $H^0(K_C(-q)) \subset H^0(K_C)$.

If $g(C)>0$ then $p$ cannot be linearly equivalent to $q$ for distinct points $p, q \in C$ so the Abel-Jacobi map $\mu_1$ is one-to-one, and since its differential is nonzero, $\mu_1 : C \to J(C)$ is an embedding. Indeed, since the tangent spaces to $J(C)$ are all identified with $H^0(K_C)^*$, if $Z \subset J(C)$ is a smooth subvariety of dimension $k$ we can define a \emph{Gauss map} $\cG : Z \to G(k, H^0(K_C)^*)$ by sending each point $z \in Z$ to the tangent space $T_z Z\subset H^0(K_C)^*$.
In particular, the Gauss map applied to the curve $(\mu_1)(C)$ is the canonical map.


\section{The $g+3$ theorem}\label{g+3 section}

Even after the description of the Picard varieties $\Pic_d(C) \cong J(C)$ in the last section, the Picard varieties may seem like mysterious objects. They are! But even the bare-bones facts---that there exists a  moduli space for invertible sheaves of degree $d$ on $C$, and that this space is irreducible of dimension $g$---suffice to prove a non-trivial theorem about curves: 

\begin{theorem}\label{g+3 theorem}
Let $C$ be a smooth projective curve of genus $g$. If $D \in C_{g+3}$ is a general divisor of degree $g+3$ on $C$, then 
$D$ is very ample. In particular, every curve of genus $g$ may be embedded in $\PP^{3}$ as a curve of degree $g+3$.
\end{theorem}

A parallel result, which we will prove in Chapter~\ref{uniformpositionchapter} shows that a general divisor of degree $g+2$ maps the curve birationally to $\PP^2$, onto a curve that either has $\binom{g}{2}$ ordinary nodes and no other singularities unless the curve
is hyperelliptic, in which case it has an ordinary $g$-fold point and no other singularities.

We proved in Theorem~\ref{very ample} that every divisor of degree $\geq 2g+1$ is very ample, and this is sharp: the canonical divisor plus 2 points is not very ample. The difference here is that we are taking a general divisor. Theorem~\ref{g+3 theorem} is also sharp: hyperelliptic curves cannot be embedded in any projective space as curves of any degree less than $g+3$, as we'll see in Chapter~\ref{ScrollsChapter}. 

If we consider only general divisors on \emph{general} curves, we can do still better: the omnibus Brill-Noether theorem (\ref{BN omnibus}) says that``most" curves of genus $g$ can in fact be embedded in $\PP^{3}$ as curves of degree $d = \lceil 3g/4 \rceil + 3$.

\begin{proof}
If $D$ is general of degree $g+3$ then $D$ is nonspecial, so $h^0(\cO_C(D)) = 4$. By Proposition~\ref{very ample} we must show that
for any pair of points $p, q \in C$, we have $h^0(\cO_C(D-p-q)) = 2$.

If, on the contrary, $h^0(\cO_C(D-p-q)) \geq 3$ then by the Riemann-Roch theorem $h^0(\omega_C(-D + p + q)) \geq 1$; fixing a divisor 
$K_{C}\in |\omega_{C}|$, this is the condition that there exists  
an effective divisor $E$ of degree $g-3$ linearly equivalent to a divisor in $|K_C - D + p + q|$. 

Consider the map 
$$
\nu : C^{(g-3)} \times C^{(2)} \; \to \; J(C)
$$
given by
$$
\nu : (E,F) \; \mapsto \; \mu_{2g-2}(K_C) - \mu_{g-3}(E) + \mu_{2}(F), 
$$
where the $+$ and $-$ on the right refer to the group law on $J(C)$. 

By what we have just said and Abel's theorem, a divisor $D$ fails to be very ample only if
$\mu(D) \in {\rm Im}(\nu)$. But the source $C^{(g-3)} \times C^{(2)}$ of $\nu$ has dimension $g-3+2 = g-1$, and so its image in $J(C)$ must be a proper subvariety; since $\mu_{g+3}$ is dominant, the image of a general divisor $D \in C^{(g-3)}$ is a general point of $J(C)$ and thus will not lie in ${\rm Im}(\nu)$. 
\end{proof}

Thus Abel's theorem, which was born out of an effort to evaluate calculus integrals, yields a basic fact in about algebraic curves!

\subsection{Variants}

We mention two variants of the $g+3$ theorem, describing the maps $\phi_D : C \to \PP^1$ and $\phi_D : C \to \PP^2$ associated to a general divisor of degrees $g+1$ and $g+2$. The first is completely elementary:

\begin{theorem}\label{g+1 theorem}
Let $C$ be a smooth projective curve of genus $g$. If $D$ is a general divisor class of degree $g+1$, then $h^0(D) = 2$ and the map $\phi_D : C \to \PP^1$ given by $|D|$ expresses $C$ as a \emph{simply branched} cover of $\PP^1$; that is, all ramification is simple and no two ramification points have the same image.
\end{theorem}

\begin{proof}
To say that the map $\phi_D : C \to \PP^1$ has a ramification point with ramification index $\geq 2$ means that $D$ is linearly equivalent to a divisor $E + 3p$, with $p \in C$ any point and $E$ an effective divisor of degree $g-2$. But the locus of divisor classes of this form has dimension $g-1$, and so a general divisor $D$ of degree $g+1$ is not linearly equivalent to a divisor of this form. Similarly, to say that a fiber of $\phi_D : C \to \PP^1$ contains two ramification points means that $D$ is linearly equivalent to a divisor $E + 2p + 2q$, with $p, q \in C$ any point and $E$ an effective divisor of degree $g-3$; again, the locus of such divisors has dimension $g-1$.
\end{proof}

\fix{instead of giving the proof here, this might make a good exercise}

There is likewise an analog of Theorems~\ref{g+3 theorem} and Theorem~\ref{g+1 theorem} for a general divisor of degree $g+2$, though it is the trickiest of the three. We'll state it here, but the proof will have to be deferred until we have established the strong form of Martens' theorem in Chapter~\ref{uniform position}.

\begin{theorem}
Let $C$ be any smooth projective curve of genus $g$, and let $D$ be a general divisor of degree $g+2$ on $C$. 
\begin{enumerate}
\item If $C$ is non-hyperelliptic, the map $\phi_D : C \to \PP^2$ is birational onto its image $C_0$, and $C_0$ is a plane curve of degree $g+2$ with exactly $\binom{g}{2}$ nodes and no other singularities; and
\item If $C$ is hyperelliptic, the map $\phi_D : C \to \PP^2$ is birational onto its image $C_0$, and $C_0$ is a plane curve of degree $g+2$ with one ordinary $g$-fold point and no other singularities.
\end{enumerate}
\end{theorem}



\section{The schemes $W^r_d(C)$}

One of our principal questions, in dealing with a curve $C$, has been to describe the linear series on $C$---the invertible sheaves $\cL$ of a given degree $d$ with an $(r+1)$-dimensional  vector space $V \subset H^0(\cL)$ of sections. We have primarily asked the simple-minded, ``yes-or-no" question, do there exist such linear series or not? But now that we have a parameter space $\Pic_d(C)$ for invertible sheaves of degree $d$, we can substantially refine the question, and ask about the geometry of the locus of such linear series; that is the geometry of
$$
W^r_d(C) := \{ \cL \in \Pic_d(C) \mid h^0(\cL) \geq r+1 \}.
$$
Thus, for example, $W^0_d(C)$ is simply the locus of effective divisor classes, which is to say the image of the natural map $\mu : C_d \to \Pic_d(C)$. (We often omit the ``0" in this case and write this simply as $W_d(C)$.)

Our first step is the observation that \emph{$W^r_d(C)$ is a closed subset of $\Pic_d(C)$}. This follows from upper-semicontinuity of fiber dimension, since we can write
$$
W^r_d(C) = \left\{ \cL \in \Pic_d(C) \mid \dim(\mu^{-1}(\cL)) \geq r \right\}.
$$
Thus $W^r_d(C)$ has the structure of an algebraic variety, and we can talk about its dimension, irreducibility, smoothness or singularity and so on.

We can also introduce the corresponding subvarieties of the symmetric products $C_d$: the preimage $\mu^{-1}(W^r_d) \subset C_d$ parametrizes effective divisors $D$ with $r(D) \geq r$, and is denoted $C^r_d$.

In fact, $W^r_d(C)$ can be given the structure of a scheme representing the functor of families of invertible sheaves with $r+1$ or more sections (that is, the functor that associates to a scheme $B$ the set of invertible sheaves $\cL$ on $B \times C$ such that the pushforward $(\pi_2)_*\cL$ has a locally free subsheaf of rank $r+1$ over every curvilinear subscheme of $B$, as always modulo tensoring with pullbacks of invertible sheaves on $B$). We will later see examples of curves $C$ genus 4 that suggest that the scheme $W^1_3(C)$ is non-reduced, and  curves $C$ of genus 6 suggesting that the scheme $W^1_4(C)$ is non-reduced (in fact isomorphic to $\Spec \CC[\epsilon]/(\epsilon^5)$). For a definition of the scheme structure, see~\cite[Section IV.3]{ACGH}. 

\section{Examples in low genus}

\subsection{Genus 1} 

If $C$ is a curve of genus 1, then since the Abel-Jacobi map $\mu : C \to \Pic_1(C)$ is always an embedding, it is an isomorphism, and also $\Pic_d(C) \cong C$ for all $d$. If we fix any point $q \in C$, we get a map $C \to \Pic_0(C)$ sending $p \in C$ to the invertible sheaf $\cO_C(p-q)$, which is again an isomorphism.

The isomorphism $C \cong \Pic_0(C)$  depends on the presence of a point; over non-algebraically closed fields, the Jacobian of a curve $C$ of genus 1 will not in general be isomorphic to $C$. The isomorphisms $C \cong \Pic_d(C)$
for $d \neq 1$ depend on the existence of a divisor of degree $d-1$ on $C$.

\subsection{Genus 2}

The map $\mu_2 : C^{(2)} \to J(C)$ is an isomorphism except along the locus $\Gamma \subset  C^{(2)} $ of divisors of the unique $g^1_2$ on $C$. In fact, the symmetric square $ C^{(2)} $ of $C$ is the blow-up of $J(C)$ at the point corresponding to the invertible sheaf corresponding to the
$g^1_2$; see Exercise~\ref{blow-up of $J(C)$ at a point}.


\subsection{Genus 3}
\begin{fact}
 Let $C$ be a curve of genus 3. The geometry of the map $\mu_2$ depends on whether or not $C$ is hyperelliptic. If it is, $\mu_2$ will collapse the locus in $C_2$ of divisors of the hyperelliptic $g^1_2$ to a point, but is otherwise one-to-one, and $W_2(C)$ is singular  If $C$ is not hyperelliptic, by contrast, $\mu_2$ will be an embedding, and $W_2(C)$ smooth. 
 %\fix{Mumford does all this set-theoretically, and just asserts the singularity---with a nice picture.}

In genus 3, the birational surjection $\mu_3 = \mu_g$ is the blow-up of $J(C) \cong \Pic_3(C)$ along the locus $W^1_3(C)$. At the same time, we know that
$$
W^1_3(C) = K - W_1(C),
$$
in the sense that any invertible sheaf $\sL$ defining a $g^1_3$ has, by the Riemann-Roch theorem, the form $\omega_C\otimes \sL'$,
where $\sL'$ has degree 1 and $h^0(\sL') = 1$,
so $W^1_3$ is isomorphic to $C$. Thus $C^{(3)}$ is the blow-up of $J(C)$ along the curve $C$.
(see~\cite[pp. 53--4]{MumfordCJ} for a discussion).
\end{fact}

\section{Martens' theorem}

The general theorems we have described so far dealing with linear series on a curve $C$, like the Riemann-Roch and Clifford theorems, have to do with the existence or non-existence of linear series on $C$. Now that we've seen how to parametrize the set of linear series on $C$ by the varieties $W^r_d(C)$, we can ask ``how many": what can the dimension of $W^r_d(C)$ be? The following result is a generalization and strengthening
of Clifford's Theorem:

\begin{theorem}[Martens' theorem]\cite{Martens}
If $C$ is a smooth projective curve of genus $g$, then for any $d\leq 2g-2$  we have
$$
\dim(W^r_d(C)) \leq d-2r;
$$
moreover, equality holds for some $0<r\leq d/2$ if and only if $C$ is hyperelliptic, and then it holds for all such $r$.
\end{theorem}

We will prove the inequality here, postponing the proof that strict inequality holds in the non-hyperelliptic case to Theorem~\ref{full Martens}.

\begin{proof}[Proof of Martens' inequality]
Since
$$
W^r_d(C) = K_C - W^{g-1-d+r}_{2g-2-d}
$$
and
$$
2g-2-d - 2(g-1-d+r) = d-2r,
$$
it will suffice to prove the theorem in case $d \leq g-1$.

Since the fibers of $C^r_d$ over $W^r_d(C)$ have dimension at least $r$, Marten's inequality will follow from the corresponding inequality
$$
\dim C^r_d \; \leq \; d-r.
$$
In this form, it will follow immediately from the geometric  Riemann-Roch theorem by applyig Lemma~\ref{elementary secant plane lemma}, below, to the canonical image of $C$.
\end{proof}

\begin{lemma}[elementary secant plane lemma]\label{elementary secant plane lemma}
Let $C \subset \PP^n$ be a smooth, irreducible and nondegenerate curve. If we denote by $\Sigma \subset C_d$ the locus of effective divisors $D$ of degree $d$ on $C$ with $\dim \overline D \leq d-r-1$, then for any $d \leq n$ and $r > 0$,
$$
\dim \Sigma \leq d-r.
$$
\end{lemma}

\begin{proof}
Let $C^d$ be the Cartesian product and $\alpha : C^d \to C_d$ the quotient map; let $\Phi = \alpha^{-1}(\Sigma)$. Since the fibers of $\pi$ are finite, the lemma is equivalent to the assertion that $\dim \Phi \leq d-r$. Let $\pi : C^d \to C^{d-r}$ on the first $d-r$ factors.

We will proceed by induction on $d$. A general $(d-r)$-tuple of points on the curve are linearly independent, so the general fiber of the restriction $\pi_\Phi : \Phi \to C^{d-r}$ is finite. By  induction, we may assume that the locus of $(d-r)$-tuples of points on $C$ that span only a $(d-r-1-s)$-plane has codimension at least $s$ in $C^{d-r}$. Since the fibers over this locus have dimension $s$, we conclude that $\dim \Phi \leq d-r$.
\end{proof}

If $C$ is hyperelliptic with $g^1_2 = |D|$ then $\mu(rD) \in W^r_{2r}$ and
$$
W^r_d(C) \supset W_{d-2r}(C) + \mu(rD).
$$
(and in fact this is an equality by~\ref{RR}). Since $W_{d-2r}(C)$ has dimension $d-2r$, we see that Martens' theorem is sharp. 
\fix{put in real ref when there is one}

There are extensions of Martens' theorem to the cases $\dim(W^r_d(C)) < d-2r$ in \cite{Mumford-Prym1} \cite{Keem},
and \cite{Coppens}.

\section{Exercises}

\begin{exercise}
 Let $G$ be a finite group acting on a quasi-projective scheme $X$. Show that there is a finite covering of $X$ by invariant open affine sets. (Hint: consider the sum of the $G$-translates of a very ample divisor.)
\end{exercise}


\begin{exercise}\label{free actions}
We say that a group $G$ acts freely on $X$ if $gx = gy$ only when $g =1$ or $x=y$. Show that
 if $G$ is a finite group acting freely on a smooth affine variety $X$ then the quotient $X/G$ is smooth.
\end{exercise}


\begin{exercise}
 \label{sym2A2} 
 \begin{enumerate}
 \item Let $X = (\AA^{2})^{2}$ and let $G := \ZZ/2$ act on $X$ by permuting the two copies of  $\AA^{2}$; algebraically,
$(\AA^{2})^{2} = \Spec S$, with $S = k[x_{1},x_{2}, y_{1}, y_{2}]$ and the nontrivial element $\sigma\in G$ acts by
$\sigma(x_{i}) = y_{i}$. 
\item Show that $G$ acts freely on the complement of the diagonal, but fixes the diagonal pointwise.
\item Show that the algebra $S^{G}$ has dimension 4 and is generated by the 5 elements
$$ 
f_{1} = x_{1}+y_{1}, f_{2} = x_{2}+y_{2}, g_{1} = x_{1}y_{1}, g_{2} = x_{2}y_{2}, h = x_{1}y_{2}+x_{2}y_{1},
$$
perhaps by appropriately modifying the steps given in \cite[Exercise 1.6]{Eisenbud1995}. 
\item Show that $h^2$ lies in the subring generated by $f_1,\dots, f_4$, and thus $S^{(2)}$ is a hypersurface, singular
along the  codimension 2 subset $f_{1} = f_{2} = 0$, which is the image of the diagonal subset of the 
cartesian product $(\AA^{2})^{2}$.
\end{enumerate}
\end{exercise}

\begin{exercise}
Show that if $r \geq d-g$, then $W^r_d(C) \setminus W^{r+1}_d(C)$ is dense in $W^r_d(C)$ (that is, $W^{r+1}_d(C)$ does not contain any irreducible component of $W^r_d(C)$).

\end{exercise}

\begin{exercise}
Let $C$ be a curve of genus 2, and let $C \subset J(C)$ be the image of the Abel-Jacobi map $\mu_1$. Show that the self-intersection of the curve $C$ is 2,
\begin{enumerate}
\item by applying the adjunction formula to $C \subset J(C)$; and
\item by calculating the self-intersection of its preimage $C + p \subset C_2$ and using the geometry of the map $\mu_2$.
\end{enumerate}
\end{exercise}

\begin{exercise}
Again, let $C$ be a curve of genus 2, and consider the map $\nu : C \times C \to \Pic_0(C)$ defined by sending $(p, q)\in C \times C$ to the invertible sheaf $\cO_C(p-q)$. What is the degree of this map?
\end{exercise}

\begin{exercise} \label{comparison with geometric RR}
Show that the image of the differential of the Abel-Jacobi map $C^{(d)} \to J(C)$ at a point of $C^{(d)} \setminus C^1_d$  corresponding to a reduced divisor is  the plane in $\PP^{g-1}$ spanned by the divisor $D$ on the canonical curve.
\end{exercise}
%\fix{can we do this for non-reduced divisors?}

\begin{exercise}\label{blow-up of $J(C)$ at a point}
Let $C$ be a curve of genus 2. The canonical map $\phi_K : C \to \PP^1$ expresses $C$ as a 2-sheeted cover of $\PP^1$, and we have correspondingly an involution $\tau : C \to C$ exchanging points in the fibers of $\phi_K$ (equivalently, for any $p \in C$, we have $h^0(K_C(-p)) = 1$; $\tau$ will send $p$ to the unique zero of the unique section $\sigma \in H^0(K_C(-p))$). Let $\Gamma \subset C \times C$ be the graph of $\tau$.
\begin{enumerate}
\item Using the fact that a birational morphism of smooth surfaces must be the inverse of a sequence of blow-ups of reduced points (\cite[V.??]{H}) show the self-intersection of the image $C^1_2$ of $\Gamma$ in $C_2$ is $-1$.
\item Find the self-intersection of $\Gamma$ in $C \times C$.
\end{enumerate}
\end{exercise}

\input footer.tex

