%\documentclass[12pt, leqno]{article}
%\usepackage{amsmath,amscd,amsthm,amssymb,amsxtra,latexsym,epsfig,epic,graphics}
%\usepackage[matrix,arrow,curve]{xy}
%\usepackage{graphicx}
%\usepackage{diagrams}
%%\usepackage{amsrefs}
%%%%%%%%%%%%%%%%%%%%%%%%%%%%%%%%%%%%%%%%%%
%%\textwidth16cm
%%\textheight20cm
%%\topmargin-2cm
%\oddsidemargin.8cm
%\evensidemargin1cm


\ifx\whole\undefined
\documentclass[12pt, leqno]{book}
\input style-for-curves.sty

%%%%%Definitions
%\input preamble.tex
%\input style-for-curves.sty
%\def\TU{{\bf U}}
%\def\AA{{\mathbb A}}
%\def\BB{{\mathbb B}}
%\def\CC{{\mathbb C}}
%\def\QQ{{\mathbb Q}}
%\def\RR{{\mathbb R}}
%\def\facet{{\bf facet}}
%\def\image{{\rm image}}
%\def\cE{{\cal E}}
%\def\cF{{\cal F}}
%\def\cG{{\cal G}}
%\def\cH{{\cal H}}
%\def\cHom{{{\cal H}om}}
%\def\h{{\rm h}}
% \def\bs{{Boij-S\"oderberg{} }}

\makeatletter
\def\Ddots{\mathinner{\mkern1mu\raise\p@
\vbox{\kern7\p@\hbox{.}}\mkern2mu
\raise4\p@\hbox{.}\mkern2mu\raise7\p@\hbox{.}\mkern1mu}}
\makeatother

%%
%\pagestyle{myheadings}
\date{August 4, 2017}
%\date{}
\title{To Fix}
%{\normalsize ***Preliminary Version***}} 
\author{David Eisenbud and Joe Harris }

\begin{document}

\maketitle

\setlength{\parskip}{5pt}

 
 \section{Items to discuss with Silvio}
 
 \begin{enumerate}
 
 \item The exercises, with whatever hints there are, should be copied to a separate document, and the format of hints regularized: 
 exercise/blank line or hrule/hing. The hints should then be deleted from the original document. The numbering in the new document should of course correspond to that in the book. Watch out for references to the text from the hints.
 
 This will be a document on the book's website at the AMS, 
 mentioned in but not distributed with the book.
 
 We don't yet know the ams website address: it needs to be filled in a line in the intro when it is knownl
   
\item theta characteristic not theta-characteristic 

 \item Index: Silvio will create this, with a model similar to what Devlin did in 3264. Joe and I will mark a printed copy
 
 \item Could we use a different font for Cheerful Facts? Italic is kind of wearying if it goes on for more than a paragraph.
 
 \item Some symbols (wedge product, direct sums) are awkwardly sized. Could we see about intermediate sizes?
 
 \item Capitalization: named chapters, sections etc get a capital; not generic references. First word of a title only gets a cap.
 
 \item definitions get an \emph{}
 
 \item statements sometime have a superfluous future tense (``the point will be smooth'',``this will be proved..." and the like.) Make present tense.
 
 \item Hurwitz' theorem not Hurwitz's 
 
 \item even named theorems not referred to by number should be lower case: eg the Riemann-Roch theorem. 
 
 \item  ``parameterize" should be  ``parametrize"
 
\item ``homogenous" should b ``homogeneous " 

\item in running head such as the one n section 5.5 the name of the theorem is the g+3 theorem, but it come out the G+3 theorem.
Is there a way to avoid the capitalization of g?

\item in the syzygies chapter (ch 18 in the book) there is a mix of $\wedge$ and $\bigwedge$ applied to modules, mostly F and G.
the $\wedge$ looks puny and the $\bigwedge$ is too big and sticks out. It would be nice to uniformly use a symbol in between the too.
(I think you found a symbol that we liked for 3264.)
 
 \end{enumerate}
\end{document}



