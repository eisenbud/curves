
%header and footer for separate chapter files

\ifx\whole\undefined
\documentclass[12pt, leqno]{book}
\usepackage{graphicx}
\input style-for-curves.sty
\usepackage{hyperref}
\usepackage{showkeys} %This shows the labels.
%\usepackage{SLAG,msribib,local}
%\usepackage{amsmath,amscd,amsthm,amssymb,amsxtra,latexsym,epsfig,epic,graphics}
%\usepackage[matrix,arrow,curve]{xy}
%\usepackage{graphicx}
%\usepackage{diagrams}
%
%%\usepackage{amsrefs}
%%%%%%%%%%%%%%%%%%%%%%%%%%%%%%%%%%%%%%%%%%
%%\textwidth16cm
%%\textheight20cm
%%\topmargin-2cm
%\oddsidemargin.8cm
%\evensidemargin1cm
%
%%%%%%Definitions
%\input preamble.tex
%\input style-for-curves.sty
%\def\TU{{\bf U}}
%\def\AA{{\mathbb A}}
%\def\BB{{\mathbb B}}
%\def\CC{{\mathbb C}}
%\def\QQ{{\mathbb Q}}
%\def\RR{{\mathbb R}}
%\def\facet{{\bf facet}}
%\def\image{{\rm image}}
%\def\cE{{\cal E}}
%\def\cF{{\cal F}}
%\def\cG{{\cal G}}
%\def\cH{{\cal H}}
%\def\cHom{{{\cal H}om}}
%\def\h{{\rm h}}
% \def\bs{{Boij-S\"oderberg{} }}
%
%\makeatletter
%\def\Ddots{\mathinner{\mkern1mu\raise\p@
%\vbox{\kern7\p@\hbox{.}}\mkern2mu
%\raise4\p@\hbox{.}\mkern2mu\raise7\p@\hbox{.}\mkern1mu}}
%\makeatother

%%
%\pagestyle{myheadings}

%\input style-for-curves.tex
%\documentclass{cambridge7A}
%\usepackage{hatcher_revised} 
%\usepackage{3264}
   
\errorcontextlines=1000
%\usepackage{makeidx}
\let\see\relax
\usepackage{makeidx}
\makeindex
% \index{word} in the doc; \index{variety!algebraic} gives variety, algebraic
% PUT a % after each \index{***}

\overfullrule=5pt
\catcode`\@\active
\def@{\mskip1.5mu} %produce a small space in math with an @

\title{Personalities of Curves}
\author{\copyright David Eisenbud and Joe Harris}
%%\includeonly{%
%0-intro,01-ChowRingDogma,02-FirstExamples,03-Grassmannians,04-GeneralGrassmannians
%,05-VectorBundlesAndChernClasses,06-LinesOnHypersurfaces,07-SingularElementsOfLinearSeries,
%08-ParameterSpaces,
%bib
%}

\date{\today}
%%\date{}
%\title{Curves}
%%{\normalsize ***Preliminary Version***}} 
%\author{David Eisenbud and Joe Harris }
%
%\begin{document}

\begin{document}
\maketitle

\pagenumbering{roman}
\setcounter{page}{5}
%\begin{5}
%\end{5}
\pagenumbering{arabic}
\tableofcontents
\fi


\chapter{Moduli of curves} 
\label{CurvesModuli chapter}\label{CurvesModuliChapter}


In the preceding chapter, we described the \emph{Hilbert scheme}, a fine moduli space for curves in projective space. In
this chapter we will discuss the second moduli space central to the theory of algebraic curves: $M_g$, which parametrizes isomorphism classes of smooth projective curves of genus $g$. As we'll see, $M_g$ is not a fine moduli space, but it comes close.

To describe the situation,  we will start with the case of curves of genus 1, where everything can be made explicit.


\section{Curves of genus 1}\label{Curves of genus 1}

Let $C$ be a smooth curve of genus 1. Any invertible sheaf of degree 2 on $C$ can be written as
$\sO_C(2p)$, and defines
a morphism to $\PP^1$ with 4 distinct branch points. Since the automorphism group of $C$ is transitive,
these 4 points in $\PP^1$ are independent of the choice of $p$, and are well-defined
up to an automorphism of $\PP^1$.    As explained in Section~\ref{branched covers},  this means that every such curve $C$ can be realized as the completion of an affine curve
$$
y^2 = f(x)
$$
where $f$ is a quartic polynomial with distinct roots:
$$
f(x) = \prod_{i=1}^4 (x - \lambda_i).
$$
Thus we would like to define $M_1$ to be the set of 4-tuples of distinct points $\{\lambda_{1}, \dots, \lambda_{4}\}$ of $\PP^1$ modulo the action of $\Aut(\PP^1) = PGL_2$.

As we will explain in the next sections, quotients by infinite groups can behave badly,
but in this case we can compute the quotient in a much simpler way:
There is a unique automorphism of $\PP^1$ carrying the three points $\lambda_1, \lambda_2,\lambda_3$ to the points $0, 1$ and $\infty \in \PP^1$ respectively, so that we can write $C$ as the zero locus of
$$
y^2 = x(x-1)(x-\lambda)
$$
for some complex number $\lambda  \in \PP^1 \setminus \{0,1,\infty\}$; we'll call this curve $C_\lambda$. 
This expression is not unique, since if we reordered the original  four points $\lambda_i$, we might arrive at a different value of $\lambda$; for example, if we exchanged 0 and $\infty$ and fixed 1, $\lambda$ would be replaced by $1/\lambda$. Thus the symmetric group $S_4$ acts on the set $\AA^1 \setminus \{0,1,\infty\}$
and one can show that the orbit of $\lambda$ under this action is
$$
 \left\{ \lambda, 1-\lambda, \frac{1}{\lambda}, \frac{1}{1-\lambda}, \frac{\lambda-1}{\lambda}, \frac{\lambda}{\lambda - 1} \right\}.
$$
There are 6 points in the orbit rather than 24 because the Klein 4-group
$K = \ZZ/2\times \ZZ/2 \subset S_4$ of fixed-point free involutions acts trivially, so what we really have is an action of $S_4/K \cong S_3$.

Since $S_3$ is finite and $\PP^1 \setminus \{0,1,\infty\}$ is a normal affine curve, the quotient space by the action is again a normal affine curve whose points are in one-to-one
correspondence with the orbits, and thus with the set of curves of genus 1. 
 Moreover, by L\"uroth's theorem, Theorem~\ref{Lueroth}, the quotient is rational, meaning that the field of rational functions on the quotient---that is, the subfield of $\CC(\lambda)$ invariant under the action of $S_3$---is of the form $\CC(j)$ for some rational function $j(\lambda)$ of degree 6. Of course, there are many possible generators of the field of rational functions on the quotient; one that works is
\begin{equation}\label{formula for j}
j(\lambda) := 256\frac{(\lambda^2-\lambda + 1)^3}{\lambda^2(\lambda-1)^2},
\end{equation}
known as the \emph{$j$-function}. As $\lambda$ varies in $\PP^1 \setminus \{0,1,\infty\}$, $j(\lambda)$ can assume any value in $\AA^1$.
 
 
%One can check that the quotient map $\PP^1 \setminus \{0,1,\infty\} \to \PP^1\setminus \infty = \AA^1$
%is given by the rational function
%$$
%j(\lambda) := 256\frac{(\lambda^2-\lambda + 1)^3}{\lambda^2(\lambda-1)^2},
%$$
%known as the $j$-function. A

Summarizing, we have proven:

\begin{theorem}
The set of isomorphism classes of smooth projective curves of genus 1 is in bijection with the points of the affine line $M_1 \cong \AA^1$. The bijection maps the curve defined by $y^2 = x(x-1)(x-\lambda)$
to  $j(\lambda)\in \AA^1$.
\end{theorem}

\subsection{$M_1$ is a coarse moduli space}

 As we will see in Exercises *** and ***,
$M_1$ is not a fine moduli space, but it comes close in two senses. 

\begin{proposition}\label{M1 is coarse}
To any family $\pi : \cC \to B$  of smooth projective curves of genus 1 over a reduced base $B$ we can associate a natural morphism of schemes $\phi : B \to M_1$ whose value at any point $b \in B$ is the $j$-invariant of the corresponding fiber $C_b$.
\end{proposition} 

This is slightly weaker than saying that $M_1$ is a coarse moduli space, because we are restricting ourselves to families with reduced bases; in fact, the statement is true in general, but requires more machinery to prove.



\begin{proof}
To start, we will work locally in $B$: for a given $b_0 \in B$, we will choose a suitably small neighborhood $U$ of $b_0 \in B$ and restrict ourselves to the preimage $\cC_U = \pi^{-1}(U)$. The first thing to do is to express the curves $C_b$ in our family as 2-sheeted covers of $\PP^1$, which is to say we want to choose an invertible sheaf on $\cC_U$ having degree 2 on each fiber $C_b$. Since we're working locally in $B$,
we can find a section $\rho : U \to \cC_U$ of $\pi : \cC \to B$. If we let $D = \rho(U) \subset \cC_U$ be the image, then we can take our invertible sheaf to be $\cL := \cO_{\cC_U}(2D)$.

Next, we use the following result, which is a special case of the theorem on ``cohomology and base change''
(see for example \cite[Appendix, Theorems B.5 and B.9]{3264} or, for a treatment that some of our students find harder to understand, \cite[Theorem 12.11]{Hartshorne1977}.)
\begin{theorem}(Cohomology and base change) If $f: X\to Y$ is a morphism and $\sF$ is
a coherent sheaf on $X$ such that $H^{1}(\sF|_{f^{-1}(y)})= 0$ for all $y\in Y$, then 
$h^{0}(\sF|_{f^{-1}(y)})$ is a constant function of $y$, and 
$f_{*}(\sF)$ is a vector bundle of this rank.\qed
\end{theorem}

This result implies that the direct image $\cE := \pi_*(\cO_{\cC_U}(2D))$ is locally free of rank 2, and we get a morphism $\cC_U \to \PP(\cE)$ expressing each curve $C_b$ as a two-sheeted cover of the corresponding fiber $\PP(\cE_b)$. Again, since we are working locally in $B$, we can trivialize the bundle $\cE$, so that we get a diagram
$$
\begin{diagram}
\cC_U & & \rTo & & U \times \PP^1 \\
& \rdTo & & \ldTo & \\
& & U & &
\end{diagram} 
$$
Once more restricting to a smaller neighborhood $U$ if necessary, we can write the family $\cC_U \to U$ as the locus
$$
y^2 = \prod_1^4 (x - \lambda_i),
$$
where the $\lambda_i$ are regular functions on $U$. The $j$-function of the $\lambda_i$  yields a map $U \to M_1$; since the value of the $j$-function at a point is determined by the isomorphism type of the fiber over this point, these maps agree on overlaps to give  the desired morphism $B \to M_1$.
\end{proof}



\subsection{The good news}

However, $M_1$ comes close to being a fine moduli space in the following sense:

\begin{proposition}\label{families on pullbacks} Let $B$ be a reduced scheme.
\begin{enumerate}
\item If $j : B \to \AA^1$ is any regular function on $B$, then there exists a finite cover $\alpha : B' \to B$ such that $j \circ \alpha$ is the $j$-function of a family of curves of genus 1 on $B'$; and
\item If $\pi : \cC \to B$ and $\eta : \cD \to B$ are two families of curves of genus 1 with the same associated $j$-function, then there exists a finite cover $\alpha : B' \to B$ and an isomorphism $\cC \times_B B' \cong \cD \times_B B'$ such that the diagram
$$
\begin{diagram}
\cC \times_B B' & & \rTo & & \cD \times_B B' \\
& \rdTo & & \ldTo & \\
& & B' & &
\end{diagram} 
$$
commutes.
\end{enumerate}
\end{proposition}

\begin{proof} For the first of these assertions, let
$$
B' := \{(b, \lambda) \in B \times (\AA^1 \setminus \{0,1\}) \mid j(b) = j(\lambda)\}
$$
where $j(\lambda)$ is as given in formula~(\ref{formula for j}). We have already described a family of curves of genus 1 over the ``$\lambda$-line" $\AA^1 \setminus \{0,1\}$; the pullback to $B'$ is the desired family.

For the second half, we want to do something similar. Specifically, we want to choose sections $\sigma : B \to \cC$ and $\tau: B \to \cD$ and take
$$
B' := \{ (b, \phi) \mid b \in B, \; \phi : C_b \rTo^{\cong} D_b  \text{ and } \phi(\sigma(b)) = \tau(b) \};
$$
by Corollary~\ref{finite automorphism genus 1}, $B'$ is a finite cover of $B$ and when we pull back the two families to $B'$ we have a tautological isomorphism between them. The only issue is how to give $B'$ an appropriate scheme structure, and for this we can use the Isom scheme described at the end of Section~\ref{maps between curves}.
\end{proof}

Thus, $M_1$ is not a fine moduli space for smooth curves of genus 1, but it is the next best thing: we don't get a bijection between families of curves of genus 1 over a given base $B$ and maps $j : B \to M_1$; but we do get a map from the former to the latter with ``finite kernel and cokernel".

\subsection{Compactifying $M_1$}

A natural question to ask is, if every value of $j \in \AA^1$ corresponds to an isomorphism class of curves $C_j$ of genus 1, what happens to the curves $C_j$ as $j$ goes to $\infty$? Or, equivalently, what happens to the curve $C_\lambda$ given as the double cover
$$
y^2 = x(x-1)(x - \lambda)
$$
as $\lambda$ approaches 0, 1 or $\infty$---the other branch points of the double cover? The answer is clear from the equation: when two branch points of a double cover of smooth curves come together, the limiting curve has a node. In fact, there is a unique isomorphism class of irreducible curves of arithmetic genus 1 having a node; it's represented by the curve with equation $y^2=x^2(x-1)$.

\begin{figure}\label{Fig7.0}
\begin{center}
\centerline {\includegraphics[height=2in]{"Fig7.0.pdf"}}
\caption{A curve of genus 1 degenerating to a rational curve with a node
in the family $y^2 = x(x-1)(x - \lambda)$}
\label{default}
\end{center}
\end{figure}

The upshot of this is that if we enlarge the original class of curves parametrized by $M_1$---``smooth projective curves of genus 1"---to the slightly larger class, ``irreducible nodal projective curves of arithmetic genus 1," we still have a coarse moduli space $\overline M_1$ for this slightly larger class of objects. This enlarged moduli space is obtained by adding one point ``at $\infty$" to the existing space $M_1 \cong \AA^1$ to form $\overline M_1 \cong \PP^1$.

This is an example of what is called a \emph{modular compactification}. There is no precise definition, but if we have a class of objects parametrized by a (non-compact) moduli space $M$ we may be able enlarge the class of objects to be parametrized, with the result that the moduli space $\overline M$ of the larger class is compact. 

Modular compactifications of a given moduli problem may or may not exist. It's sometimes a tricky problem to find a suitable class of objects to parametrize: if we don't add enough additional isomorphism classes, not every 1-parameter family of objects in our original class will have a limit in the larger class, meaning the enlarged moduli space will still not be compact; if we add too many,  1-parameter families may have more than one possible limit, meaning the enlarged space won't be separated. For example in the family
 of curves $C_t$ given as
$$
C_t = V(y^2 -x^3 - t^2x - t^3)
$$
the $j$ function is constant when $t\neq 0$, but  the limiting curve $C_0$ has a cusp. This shows that
we could not have added cuspidal curves to $M_1$.

\begin{figure}\label {Fig7.A}
\begin{center}
\centerline {\includegraphics[height=2.5in]{"Fig7.A.pdf"}}
\caption{A curve of genus 1 degenerating to a cuspidal curve in the family $
C_t = V(y^2 -x^3 - t^2x - t^3)$.}
\label{default}
\end{center}
\end{figure}


 When modular compactifications do exist, they are extremely valuable for the study of both the space $M$ and of the objects parametrized by $M$: compactness allows us to apply the techniques of modern algebraic geometry to the space $\overline M$, while the fact that it is still a moduli space gives us a handle on its geometry. In the following section, we will describe a modular compactification of $M_g$. The objects parametrized are called ``stable curves."

Getting back to the moduli space $\overline M_1$, if we have a family where
$j(\lambda)$ has a pole, we would like to say that the limit of the curves in the family is an irreducible nodal curve,
but this is not necessarily true! For example, the limit of the curves
$$
y^2 = x(x-t)(tx-1)
$$
as $t \to 0$ is reducible, with two components meeting in two points, 0 and $\infty$.
What is true is that a process called \emph{semistable reduction} shows that after a base change and a birational
modification of the family around the pole we can replace the family with one where the singular fiber
is indeed an irreducible nodal curve. (Figure 7.3) See~\cite{MR1631825} for a description of this process in general, and several explicit examples.

\fix{Silvio: The upper picture is supposed to show the degeneration $y^2 = x(x-t)(tx-1)
$ as $t\to 0$.}
\begin{figure}
\inprogress
\centerline{\includegraphics[width=2in]{"main/Fig07-1"}\llap{\hskip-0.5in
\small\openup-2pt
\raise140pt\vbox{\hbox{blow-down on}\hbox{component}\hbox{of central fiber}}}}
\caption{In this case a birational modification of the total space of the family changes the unstable reducible curve to a stable curve.}
\label{default}
\end{figure}

\section{Higher genus}

The  idea  is  analogous to the one used  for genus 1 curves: to construct a moduli space, first parametrize curves with a choice of some additional structure, such as a map to projective space, and then mod out by the choices made. For any smooth projective curve $C$ of genus $g\geq 2$, the tricanonical linear series $|3K_C|$ is very ample; it embeds $C$ as a curve of degree $6g-6$ in $\PP^{5g-6}$. Thus we have a way of realizing a given abstract curve $C$ as a curve in projective space, unique up to automorphisms of $\PP^{5g-6}$.

We claim next that the set of smooth, tricanonically embedded curves is a locally closed subset $X$ of the Hilbert scheme $Hilb_{(6g-6)m+1-g}(\PP^{5g-6})$ parametrizing curves of genus $g$ and degree $6g-6$ in $\PP^{5g-6}$. By Lemma~\ref{smooth is open}, the set of points in the base over which the curves are smooth is open.  Let 
$$
Hilb^\circ = Hilb^\circ_{(6g-6)m+1-g}(\PP^{5g-6})\subset Hilb_{(6g-6)m+1-g}(\PP^{5g-6})
$$
be this open set.

Next, on the universal family $\cC \subset Hilb^\circ \times \PP^{5g-6}$, we have two families of invertible sheaves: we have the pullback of $\cO_{\PP^{5g-6}}(1)$; and we have the cube $K^3$ of the dualizing sheaf. Each gives rise to a section of the relative Picard variety over $Hilb^\circ$, and the locus where they agree is thus a closed subset $X \subset Hilb^\circ$.

The group $PGL_{5g-5}$ of automorphisms of $\PP^{5g-6}$ acts on the variety $X$ and its orbits
are the isomorphism classes of smooth curves of genus $g$; thus, we might hope to realize the moduli space $M_g$ as the quotient of $X$ by $PGL_{5g-5}$. But here things go awry in a hurry: unlike the case of an action of a finite group on a variety,
the orbit spaces of infinite groups are often not algebraic varieties. (Think of the action of $\CC^*$ on $\CC$ by multiplication.) What is needed is a tool to extract the ``best possible approximation" to a quotient. Happily, David Mumford created a tool that does exactly this:  \emph{geometric invariant theory} (GIT).  To see how GIT can be used in this  setting to produce the space $M_g$, see the wonderful introduction in~\cite{Mumford-Suominen} or the more technical version in~\cite{MR0450272}.

%Nearly all of the properties of the moduli spaces $M_1$ and $\overline M_1$ of curves of genus 1 have analogues for curves of higher genus $g$,  though the techniques needed to verify the statements are vastly more sophisticated. In this section, we'll state the corresponding results about the moduli spaces $M_g$, and then describe one way in which they may be constructed.
%
%\fix{explain that we're going to take the tricanonical embedding -- this comes from nowhere no
%The singular locus of the fibers in any family of curves is closed, so the set of points in the base over which the curves are smooth is open.  Let 
%$$
%\cH^\circ_{(6g-6)m+1-g}(\PP^{5g-6}) \subset \cH_{(6g-6)m+1-g}(\PP^{5g-6})
%$$
%be this open set, and let $H$ be the pullback of $\sO_{\PP^5g-6}(1)$ to the universal curve over it.
%The canonical bundle of each fiber is the restriction of the relative cotangent bundle $\Omega$ of the family,
%so a fiber $C_\lambda$ is tricanonically embedded if $(\Omega_\lambda)^{\otimes 3}(-H)$ has a nonzero section. By the semicontinuity of cohomology, this is a closed condition on $\lambda.$  Thus
%the set of tricanonically embedded smooth curves is a closed subset of the open set of smooth
%curves---a locally closed subset $X$ of the Hilbert scheme.

%The group $PGL_{5g-5}$ of automorphisms of $\PP^{5g-6}$ acts on variety $X$ and its orbits
%are the isomorphism classes of curves. Unlike the case of an action of a finite group on a variety,
%the orbit spaces of infinite groups are often not algebraic varieties, but this case is well-behaved. This is the content of the main (and last) theorem of \cite{GIT}:

\begin{theorem}(Mumford)
The space of orbits of $PGL_{5g-5}$ acting on the subset of the Hilbert scheme representing
tricanonical curves has the structure of an algebraic variety $M_g$ which is a \emph{coarse moduli
space} in the sense that
\begin{enumerate}
 \item Given any flat family $Y\to B$ of smooth curves of genus $g$ there is a morphism of schemes
 $B\to M_g$ sending each closed point $p\in B$ to the point of $M_g$ representing the fiber $Y_b$;
 \item These maps form a natural transformation from the functor $G(-)$ of families of smooth curves to the functor 
 $\Mor_{\rm schemes}(-, M_g)$ through which any natural transformation $G \to \Mor_{\rm schemes}(-, M')$
 factors;
%Given any variety $M$ with this  property\fix{clarify what "this" property is}, there is a unique morphism $M\to M_g$ that
 %sending each closed point $p\in M$ to the point of $M_g$ representing the same curve as $p$.
 

\end{enumerate}
\end{theorem}

The power of the theory of the moduli space of curves was greatly increased when compactifications of the space (there are many interesting ones) were understood. One of these, the compactification
of $M_1 = \AA^1$ to $\overline M_1 = \PP^1$ by adding a nodal curve, has already been mentioned. This has the desirable properties that the subset added to $M_1$ is a divisor; and the compactification is \emph{modular} in the sense
that the point added corresponds to a curve almost of the same type as the curves in $M_1$.

There are two reasons why a compactification is  important:

First, the great majority of the techniques that algebraic geometers have developed for dealing with varieties apply directly only to projective varieties. For example, the Satake compactification is a projective variety containing $M_g$ in such a way that the complement---usually referred to as the ``boundary''---has codimension 2. Taking successive hyperplane sections that pass through a given point but don't meet the boundary, we see that for $g\geq 2$ there is a complete one-dimensional family of \emph{smooth} curves containing any smooth curve of genus $\geq 2$. 

Often, though, we can learn the most from a compactification where the ``boundary''---the part that is added---is a divisor, and this is the case for the Deligne-Mumford compactification
$\overline M_g$ introduced by Deligne and Mumford in their groundbreaking 1969 paper~\cite{Deligne-Mumford}, described below. A central example of how this is used is given in Section~\ref{mgunirational}, where we take up the question, ``can we write down a general curve of genus $g$?" 

To describe this compactification, we first explain some of the language and results of geometric
invariant theory.

%The variety $\overline M_g$ has an important extra property: it is a \emph{modular}  compactification in the sense that the points of the boundary correspond to slightly more general
%objects of the same type as the points of $M_g$. 
%Briefly, a projective curve $C$ of arithmetic genus $g$ is said to be \emph{stable} if its singularities, if any, are all nodes, and its automorphism group is finite. These are precisely the stable points in the Hilbert scheme of tri-canonical embeddings in the sense of geometric invariant theory. \fix{is this true? Check Harris-Morrison}

%But while orbit spaces of finite groups always exist  and behave well, quotients by positive-dimensional algebraic groups are a very different story; to deal with these, we have to introduce the methods of \emph{geometric invariant theory}, which we'll do now.


%The morphism $\sC \subset U \times \PP^{5g-6} \to U$ from the universal family to the Hilbert scheme is a smooth morphism over the open set of smooth curves, and therefore the relative cotangent sheaf is a bundle whose top exterior power is the family of canonical bundles. 
%\fix{I think the morphism has relative dim 1, so the bundle IS the canonical bundle.}
%The third tensor power  of this bundle times the inverse of the restriction of $\sO_\PP{5g-6}(1)$ has a nonzero global section at a point if and only if
%that point represents a tricanonical curve, and thus by semicontinuity of cohomology, the locus of smooth tricanonical curves is closed
%in the open set of smooth curves. We would now like to take the quotient by the group of automorphisms
%of $\PP^{5g-6}.$ This requires geometric invariant theory.


\subsection{Stable, semistable, unstable}

Given a quasi projective variety $X \subset \PP^N$ and a group $G \subset PGL_{N+1}$ that carries $X$ into itself, we wish to construct as good a map as possible from the set of orbits
to a projective space. Whatever map we take, the closure of the
image will correspond to a graded ring. We want to preserve as much of the structure of the orbit space as possible, and on an open affine cover
this means finding as many functions as possible that are invariant on the orbits. Thus it is natural to take the ring of invariants
of the homogeneous coordinate ring $A$ of the closure of $X$ as the homogeneous coordinate ring of the closure
of the image of $X$. 

The first difficulty is that the elements of $A$ are not functions on $X$, so $G$ may not even act on $A$. However, 
it is possible to lift the action of $G$ to an action on $A$ of the slightly larger group, $SL_{N+1}$, a process called \emph{linearization}. The kernel of the map $SL_{N+1} \to PGL_{N+1}$ consists of diagonal matrices of finite order dividing $N+1$, and the choice of
a linearization amounts to a choice of a character of this abelian group. However, the choice doesn't matter: since the kernel acts trivially on forms of degree a multiple
of $N+1$, and thus the action of $PGL_{N+1}$ itself  extends to an action on the homogeneous coordinate ring of the $(N+1)$-st Veronese embedding. 

The second difficulty in this program is that the ring of invariants of an infinite group may not even be finitely generated,
so it may not correspond to a projective variety. Hilbert showed that if $G= SL_{N+1}$, then the ring of invariants
is finitely generated. Since Hilbert's time this result has been extended to the class of 
\emph{linearly reductive} groups---see~\cite{MR0382294}.
Thus the subring $A^G \subset A$ of invariant elements is finitely generated over the ground field.
%\def\quot{{/\kern-3pt/}}

The third difficulty is that the points of $\Proj(A^G)$, usually denoted $X\quot G$, are generally not in one-to-one correspondence with the orbits of $G$ on $X$! 
is $\Proj(A^G)$, usually denoted $X\quot G$. 

Geometric invariant theory explains the relationship of $X\quot G$ to the set of orbits. To do this, it performs a sort of triage on the points of $X$ (or their orbits), dividing them into three classes: stable, semi-stable and unstable. The theory also provides tools for determining this stratification. 
\begin{enumerate}


\item  \emph{Stable} points. These are the points whose orbits are closed. They comprise an open subset $X^{\rm stable} \subset X$, and the points of an open subset of $X\quot G$ correspond one-to-one to the stable orbits, that is, an open subset that is set-theoretically $X^{\rm stable}/G$. In general, this set may be empty, but in the case of the action of $PGL_3$ on the $\PP^9$ of plane cubics, the stable points are the smooth plane cubics, and the quotient is the affine $j$-line.

\item \emph{Strictly semistable} points. These are the points $p$ such that there exists an invariant form not vanishing at $p$.  Together with the stable points, comprise a larger open subset $X^{\rm{semistable}} \subset X$, called the \emph{semistable} locus. Two  semistable points $p,q$ map to the same point in $X\quot G$ if and only if $\overline{Gp}\cap \overline{Gq} \neq \emptyset$. In the example of the action of $PGL_3$ on  $\PP^9$, the semistable  locus contains  the orbits of smooth and nodal plane cubics; that is, smooth cubics together with the three orbits consisting of irreducible cubics with a node, unions of lines and conics meeting transversely, and triangles. In the quotient, these last three orbits correspond to just one additional point, and this quotient is the compactification of the affine line to the projective line obtained by adding one point.

\item  \emph{Unstable} orbits. These are the points $p$ on which all invariant polynomials vanish, so that the induced map
$\Proj A \to \Proj (A^G)$ is not even defined at $p$. Thus unstable points do not correspond to any points of $X\quot G$; in fact, they cannot be included in any topologically separated quotient of an open subset of $X$ defined in this way (but there may well be other compactifications, coming from
other representations of $M_g$ as $X'\quot  G'$; see \cite{MR3044128}).
\end{enumerate}

%\subsection{Construction and characterization of the moduli space of curves}
%
%Even so, these curves do \emph{not} form a family over $M_g$; that is, there is no universal family, for reasons discussed in Section~\ref{almost fine}. But it is a ``coarse moduli space", the best possible approximation to a fine moduli space in the category of varieties. Even though the
%functor $\Mor_{\rm schemes}(-, M_g)$ is not isomorphic to the functor $F(-)$ of families of curves of genus $g$, there is
%a natural transformation
%$$
%\Psi: F\to \Mor_{\rm schemes}(-, M_g)
%$$
%such that 
%for every scheme $M'$ and natural transformation $\Psi': F \to \Mor(-, M')$
%there is a unique morphism $\eta: M\to M'$ so that $\Psi' = \Psi\circ \Mor(-, \eta)$.
%
%Since $M_g$ does not support a universal family of curves, we cannot expect an arbitrary family $\pi: X \to B$ of smooth curves of genus $g$ to arise as the pull-back along a morphism $B\to M_g$. However, the property of functors above
%says that the set-theoretic
%map $\rho_\pi: B \to M_g$ sending each closed point  $b\in B$ to the point of $M_g$ corresponding to the class of the fiber
%over $b$ is actually a morphism of schemes. \fix{ I think we're dropping the idea expressed in the next sentence} Even more is true, as explained below in Section~\ref{almost fine}. 
%
%%%%Edited to here Jul 14
%
%To summarize:
%\begin{enumerate}
%  \item The points of $M_g$ correspond one-to-one to isomorphism classes of smooth curves.
% \item For every family $\cC \to B$ of smooth curves there is a map $B\to M_g$ carrying
% each closed point  $b \in B$ to the point representing the isomorphism class of the fiber of $\cC$ over $b$,
% \end{enumerate}
%in a ``maximal'' way.
%

\section{Stable curves}

The compactification $\overline M_g$ is also a \emph{modular}  compactification in the sense that the points of the boundary correspond to slightly more general
objects of the same type as the points of $M_g$. 
\begin{definition}
A reduced irreducible connected curve is \emph{stable} if it has at most nodes as singularities,  and if every smooth rational component meets the other components at least three times.

The last phrase of the definition could be replaced by the equivalent condition that the automorphism group
of $C$ is finite. 
\end{definition}

These are stable points in the Hilbert scheme of tri-canonical embeddings in the sense of geometric invariant theory, and the result is that $M_g$ has a modular compactification that is a projective variety:


\begin{theorem}(Deligne, Mumford, Knudsen) \cite{Deligne-Mumford}, \cite{MR702954}\label{DM is coarse}
$\overline M_g$ is a projective variety such that
\begin{enumerate}
 \item The points of $\overline M_g$ correspond one-to-one to isomorphism classes of smooth curves.
 \item For every family $\cC \to B$ of stable curves there is a morphism of schemes $B\to M_g$ carrying
 each closed point  $b \in B$ to the point representing the isomorphism class of the fiber of $\cC$ over $b$. 
 These maps form a natural transformation from the functor $G(-)$ of families of stable curves to the functor 
 $\Mor_{\rm schemes}(-, \overline M_g)$ through which any natural transformation $G \to \Mor_{\rm schemes}(-, M')$
 factors.
\end{enumerate}
\end{theorem}

\begin{figure}\label{Fig7.2}
\begin{center}
\centerline {\includegraphics[height=1.2in]{"main/Fig07-2"}}
\caption{A stable curve.}
\label{default}
\end{center}
\end{figure}

%There is also a modular compactification of $M_g$. 
%
%\begin{theorem}(Deligne, Mumford, Knudsen) \cite{Deligne-Mumford}, \cite{MR702954}
%\label{DM is coarse}
%\label{stable moduli}
%There exists a coarse moduli space $\overline M_g$, whose points correspond bijectively to isomorphism classes of stable curves of genus $g$ and such that for any family $\cC \to B$ of stable curves we have an induced map $\phi : B \to \overline M_g$. Moreover, $\overline M_g$ is a projective variety.
%\end{theorem}
%
The deepest theorems about $M_{g}$ have been proven using the divisor class group of $\overline M_{g}$,
and many of the divisors that play a role are actually supported on the complement $\overline M_{g} \setminus M_{g}$, often called the \emph{boundary}.

\begin{fact}
\begin{proposition}
For $g\geq 1$ the boundary $\overline M_{g}\setminus M_{g}$ is the union of $1+\lfloor{g-1/2}\rfloor$ divisors whose generic points are of the form:
\begin{enumerate}
 \item irreducible nodal curves of geometric genus $g-1$
 \item for $i = 1, \dots \lfloor{(g-1)/2}\rfloor$ the union of two smooth curves $C_{i}\cup C_{g-i-1}$ of genera
 $i$ and $g-1-i$ meeting in a point.
\end{enumerate}
\end{proposition}
\end{fact}

We will not prove either of Theorems~\ref{moduli} and~\ref{stable moduli}. For an introduction to the actual proofs, with references, see~\cite{MR1631825}.

\subsection{How we deal with the fact that $\overline M_g$ is not fine}

The fact that $\overline M_g$ is not a fine moduli space---and that correspondingly there does not exist a universal family of curves over it---is unquestionably a nuisance. Nonetheless, there are ways of dealing with the situation. The first step is to identify the cause of the problem, which is that some curves have nontrivial automorphisms. There are three ways to proceed:

\begin{enumerate}
\item \emph{Kill the automorphisms}. The idea here is to add additional structure to the objects parametrized, so as to eliminate automorphisms. Here is an example of such a construction. We saw in Chapter~\ref{JacobianChapter} that on a smooth projective curve $C$ of genus $g$, the collection of invertible sheaves $\cL$ with $\cL^m \cong \cO_C$ forms a group isomorphic to $(\ZZ/m)^{2g}$. We define a \emph{curve with level $m$ structure} to be such a curve, together with a choice of $2g$ generators $\cL_1,\dots,\cL_{2g}$ for this group. On every curve $C$ of genus $\geq 2$ an automorphism fixing all line bundles of order $m$ is trivial,  and there does 
exist a fine moduli space $\overline M_g[m]$ for curves with level $m$ structure; this space is a finite cover of $M_g$. Thus, while a universal family does not exist over $M_g$, one does exist over a finite cover of $M_g$, and this is sufficient for many purposes.

\item \emph {Ignore the automorphisms}.
Here we use a basic fact, which we'll establish in Section~\ref{curves with automorphisms}: in $M_g$, the locus $A \subset M_g$ of curves that do have automorphisms other than the identity has codimension $g-2$. If we restrict to the complement $M_g^\circ = M_g \setminus A$, accordingly, there does exist a universal family, and again this is sufficient for many purposes; for example, if $g \geq 4$ then a divisor class on $M_g$ is determined by its restriction to $M_g^\circ$, so we can just work over that open set.

\item \emph{Embrace the automorphisms}. We mentioned above that there does not exist a fine moduli space for curves of genus $g$ in the category of schemes. But there is a larger category, called \emph{stacks}, in which a fine moduli space does exist. This solution to the problem, pioneered by Deligne and Mumford, has
many advantages but involves a substantial investment in mastering the technical issues; readers who wish to pursue this avenue may consult~\cite{Deligne-Mumford} or~\cite{Olsson}.
\end{enumerate} 


\section{Can one write down a general curve of genus $g$?}\label{mgunirational}

We have made a fuss over the value of compactifying $M_g$ to a projective variety. To see an example of the usefulness of $\overline M_g$, we'll take up a question we've raised before: Can one write down a general curve of genus $g$?
More precisely,  does there exist a family of curves depending freely on parameters that includes all the curves in an open subset of $M_{g}$---for example, as the equation $y^{2} = x(x-1)(x-\lambda)$ represents general curves of genus 1? Still more precisely,
we say that a variety is \emph{unirational} if it admits a dominant morphism from an open subset of $\AA^{n}$, and our question is: ``Is $M_g$ unirational?". 

We have produced  families with free parameters in genera 2 and 3. Essentially
the same approach works in genera $4$ and $5$; in each case a general canonical curve is a complete intersection, so that if we take the coefficients of its defining polynomials to be general scalars we have a general curve.

This method breaks down when we get to genus 6, where a canonical curve is not a complete intersection. But it's close enough: as discussed in Chapter~\ref{Brill-Noether}, a general canonical curve of genus 6 is the intersection of a smooth del Pezzo surface $S \subset \PP^5$ with a quadric hypersurface $Q$; since all smooth del Pezzo surfaces in $\PP^5$ are isomorphic, we can just fix one such surface $S$ and let $Q$ be a general quadric.

It gets harder as the genus increases. Already genus 7 calls for a different approach. Here we want to argue that, by Brill-Noether theory, a general curve of genus $7$ can be realized as (the normalization of) a plane septic curve with 8 nodes $p_1,\dots,p_8 \in \PP^2$. Conversely, if $p_1,\dots,p_8 \in \PP^2$ are general points then having nodes at the points $p_i$ imposes $ 3\times 8 = 24$ independent conditions on the $\PP^{35}$ of curves of degree 7, so that we would expect that the septic curves double at the $p_i$ form a linear series, parametrized by a projective space $\PP^{11}$.

This suggests that we consider the space
$$
\Sigma := \{ (p_1,\dots,p_8,C) \in (\PP^2)^8 \times \PP^{35} \mid C \text{ is double at } p_1,\dots,p_8 \}
$$
With a little work, we can see that there is a unique component $\Sigma^\circ$ of $\Sigma$ dominating $(\PP^2)^8$, which is a $\PP^{11}$-bundle over an open subset of $(\PP^2)^8$ and hence rational; this component dominates $M_7$, showing that $M_7$ is unirational.

A similar approach works through genus 10, and Severi conjectured that it would be possible to do something similar for all genera. The approach through plane curves, however, fails in genus 11: by the Brill-Noether theorem, the smallest degree of a planar embedding of a general curve of genus 11 is 10; by our $g+2$ theorem, such a curve has ${9\choose 2}-11 = 25$ nodes. But $3 \times 25 > 65$, the dimension of the space of plane curves of degree 10. Thus, if we introduce the analog of the incidence correspondence we used in the case of genus 7---that is,
$$
\Sigma := \{ (p_1,\dots,p_{25},C) \in (\PP^2)^{25} \times \PP^{65} \mid C \text{ is double at } p_1,\dots,p_{25} \}
$$
then the projection $\Sigma \to (\PP^2)^{25}$ is not dominant, and we have no idea if $\Sigma$ is rational.
 Ad hoc (and much more difficult) arguments have been given in genera 11, 12 13 and 14, but so far no-one can go further in producing general curves; in genus 15 it is only known that
 any two general curves can be connected by a chain of rational curves that passes through
 the locus of irreducible nodal curves in $\overline M_{g}$ (\cite{MR2202246}). In genera 15 and 16 Chang and Ran showed that $\overline M_{g}$ has no pluricanonical divisors a (possibly) weaker statement. 
 

 
However the issue is resolved for all genera $\geq 22$. Surprisingly, this depends (in the current state of our knowledge) on an understanding of the complement
$\overline M_{g}\setminus M_{g}$ and its image in the divisor class group of $\overline M_{g}$. The starting point is the fact that a smooth $n$-dimensional projective variety $X$ with an effective pluricanonical canonical divisor---that is, a nonzero section of the sheaf $\omega_{X}^{\otimes p}$ for some $p>0$---cannot be unirational: if there were a dominant rational map $\PP^n \to X$, we could pull this section back to get an effective pluricanonical divisor on $\PP^n$, which doesn't exist because
the canonical divisor on $\PP^{n}$ has negative degree. At the same time, we can analyze the divisor class theory of the space $\overline M_g$ and for large $g$ exhibit an effective pluricanonical divisor on $M_g$ by using components of  $\overline M_{g}\setminus M_{g}$. 
The upshot is 
\begin{theorem}(\cite{Harris-MumfordModuli}, \cite{HarrisModuli}, \cite{Eisenbud-HarrisModuli}, \cite{M22-23})
For all $g \geq 22$, $M_g$ is not unirational.
\end{theorem}
In each case, what is actually proven is the stronger but more technical statement that $\overline M_g$ has \emph{general type}.This line of argument requires a great deal of work---the interested reader can find more details, plus a guide to the literature, in~\cite{MR1631825}. 

%But the sequence cannot go on much longer! To say that there exists a family $\cC \to B$ over an open subset $B \subset \AA^n$ such that the induced map $\phi_\cC : B \to M_g$ is dominant is to say that $M_g$ is \emph{unirational}. Using $\overline M_g$ one can show that for higher $g$ this is not the case, and in fact:
%
%\begin{theorem}
%If $g\geq 23$ then there is no rational curve through a general point of $M_g$; that is, $M_g$ is not uniruled.
%\end{theorem}
%\begin{proof}[Proof sketch]
%The set of curves of genus $g$ posessing a divisor $D$ with $\rho(D) = g - (r(D)+1)(\deg(D) -g + r(D)) = -1$ is an effective divisor
%in the moduli space, and this leads to the construction of an effective pluricanonical divisor on the desingularization of $\overline M_g$. The proof of this
%was carried out in
%\cite{Harris-Mumford-Moduli}, \cite{HarrisModuli}, and \cite{Eisenbud-HarrisModuli}
% for all genera $g \geq 23$.
%If $M_g$ were uniruled, then there would be non-trivial deformations of a general rational curve in $M_g$, 
%and thus the normal bundle of the general rational curve $\phi: \PP^1 \to M_g$, pulled back to $\PP^1$, would have positive degree. 
%Furthermore, an effective pluricanonical form is by definition represented by a nontrivial global section of a tensor power
%of $\wedge^{\dim M_g}\Omega_{M_g}$ and the pull-back of this section section 
%along a nontrivial map from $\PP^1$ to $\overline M_g$, and thus to a desingularization, that met the pluricanonical divisor properly, would lead to an effective canonical divisor on $\PP^1$, a contradiction.
%\fix{say something about restricting a plurican div. Change this vague statement to the one in Mumford.} 
%\end{proof}
% 
% A consequence is that the sort of descriptions of embeddings with which much of this book is concerned, where we produce a surface on which a general curve of a certain sort lies, cannot be continued to high genus:
%
%\begin{fact}
% A general curve $C$ of  genus $\geq 22$ does not lie in a nontrivial linear series on any surface
% except those birational to $C\times \PP^1$. 
% \fix{citation for the theorem "a surface with an isotrivial linear series of divisors is implies birat to product".}
%\end{fact}
%
%
\section{Hurwitz spaces}\label{Hurwitz spaces}

Hurwitz spaces are spaces parametrizing branched covers. They are fascinating objects; we know quite a bit about their geometry but there is much that is unknown as well. In this discussion, we'll stick to the simplest case, that of the \emph{small Hurwitz spaces}, parametrizing simply branched covers of $\PP^1$.

To start with the definition: the small Hurwitz space $Hur^\circ_{g,d}$ parametrizes pairs $(C, f)$ where $C$ is a smooth curve of genus $g$ and $f : C \to \PP^1$ a map of degree $d$ with simple branching; that is,
$$
Hur^\circ_{g,d} = \{ (C, f) \mid C \in M_g  \text{ and } f:C \to \PP^1 \text{ simply branched of degree } d\}.
$$

There are two natural maps from the Hurwitz space to other spaces. First, we can ``project on the first factor;" that is, simply forget the map $f$ to arrive at a map $\pi : Hur^\circ_{g,d} \to M_g$. Secondly, we can associate to a point $(C,f) \in Hur^\circ_{g,d}$ the branch divisor $B \subset \PP^1$, which is an unordered $b$-tuple of distinct points in $\PP^1$, which we can think of as a point in the $b$th symmetric product $(\PP^1)_b  \cong \PP^b$. We thus have a diagram
$$
\begin{diagram}
& & Hur^\circ_{g,d} & & \\
& \ldTo^\pi & & \rdTo^\beta & \\
M_g & & & & U \subset \PP^b
\end{diagram}
$$
where $U \subset \PP^b$ is the complement of the hypersurface in $\PP^{b}$ where at least 2 of the
$b$ points are equal, called the discriminant hypersurface. Thus the Hurwitz space is positioned between an object $U$ we understand relatively well, and an object $M_g$ about which we would like to know more; this accounts for the historical importance of Hurwitz spaces. We'll now illustrate how this can be exploited.

To begin with, by the analysis in Section~\ref{branched covers}, we see that \emph{the map $\beta$ is a covering space}: for any reduced divisor $B \subset \PP^1$ there are a finite number of simply branched covers of $\PP^1$ with branch divisor $B$; and as we vary the points of $B$ locally we can deform the cover along with them. This allows us to give the Hurwitz space $Hur^\circ_{g,d}$ the structure of a smooth variety, and also tells us that
$$
\dim(Hur^\circ_{g,d}) = b = 2d+2g-2
$$

\subsection{The dimension of $M_g$} 

Next, we look at the projection $\pi : Hur^\circ_{g,d} \to M_g$. To start, let's assume $d$ is large relative to $g$; $d \geq g+1$ suffices, but you can take $d$ as large as you like; taking $d > 2g$ may make the argument simpler. We have then the

\begin{proposition}
If $d \geq g+1$, the map $\pi : Hur^\circ_{g,d} \to M_g$ is surjective, with fibers of dimension $2d-g+1$.
\end{proposition}

\begin{proof}
The question is, given a curve $C$, how many simply branched maps $f : C \to \PP^1$ of degree $d$ are there? To begin with, the $g+1$ theorem (\ref{g+1 theorem}) tells us that there is one, whence we see that $\pi$ is surjective. 

We can compute the dimension of the fibers, too: To specify a map $f : C \to \PP^1$, we can start by choosing a divisor $D \in C_d$, which will be the divisor $f^{-1}(\infty)$; this can be a general divisor of degree $d$ on $C$. Second, we choose a divisor $E$ which will be $f^{-1}(0)$; this can be a general member of the linear system $|D|$, which has dimension $d-g$. Finally, specifying $f^{-1}(\infty)$ and $f^{-1}(0)$ determines the map $f$ up to scalar multiplication on $\PP^1$; adding up the degrees of freedom, we see that the fibers of $\pi$ have dimension
$$
d + (d-g) + 1 = 2d-g+1.
$$ 
\end{proof}

Finally, we conclude that
$$
\dim(M_g) = (2d+2g-2) - (2d - g + 1) = 3g-3.
$$

We can use this in turn to analyze the cases of smaller $d$. As a basic application, note that the group $PGL_2$ of automorphisms of $\PP^1$ acts on the Hurwitz space: given $\varphi \in PGL_2$, we can send $(C,f)$ to $(C, \varphi \circ f)$. Moreover, the orbits of this action lie in fibers of the projection $\pi : Hur^\circ_{g,d} \to M_g$, meaning that the fibers of $\pi$ have dimension at least 3. 

\begin{corollary}\label{branched cover BN}
If $d < \lceil \frac{g}{2} \rceil + 1$, then a general curve $C$ of genus $g$ does not admit a map of degree $d$ to $\PP^1$.
\end{corollary}

This is one-half of the case $r=1$ of the Brill-Noether theorem, about which we will say much more later.

\subsection{Irreducibility of $M_g$}

Another important application is the original proof of the irreducibility of $M_g$: in~\cite{Hurwitz}, Hurwitz analyzed the monodromy of the map $\beta: Hur^\circ_{g,d} \to U \subset \PP^b$---what happens, in other words, when you let the branch points of a cover wander around in $U$ before coming back to their original locations. He proved that the monodromy is transitive, and hence that the Hurwitz space $Hur^\circ_{g,d}$ is irreducible; since $Hur^\circ_{g,d}$ dominates $M_g$ for $d$ large, he deduced that $M_g$ must be irreducible as well.

Hurwitz' argument illustrates a fundamental point: in practice, moduli spaces of curves ``with extra structure," such as a map to projective space, are often easier to work with, and provide a useful tool for understanding the geometry of abstract moduli spaces. For example, if we're given an abstract curve $C$ of genus $g$, it's hard---without developing a fair amount of deformation theory---to show that $C$ varies in a nontrivial family. But if $C$ is expressed as a branched cover, we can find such families just by varying the branch points.

There are many open problems connected with the Hurwitz scheme; here are a few:
\begin{enumerate}
 \item A compactification of the Hurwitz scheme by \emph{admissible covers} (allowing both source and target
of the covering to be reducible in a controlled way) is known~\cite{MR1631825}, but the boundary is very complicated, and it would be interesting to find a simpler one.

\item It is conjectured that  Picard group of the Hurwitz scheme is torsion; see~\cite{MR3320849}, who prove the conjecture for $g\leq 5$, for more information.

\item There is active work and many open problems around computing the \emph{Hurwitz numbers}, that is,
the number of curves having maps to $\PP^{1}$ with specified degree and branching; see for example~\cite{Hurwitz2} and \cite{ELSV}.
\end{enumerate}

\section{The Severi variety}\label{severi variety}

Despite having been studied for so long, many questions about plane curves remain open---for example: which ones degenerate into which others, and in what way. All plane curves of degree $d$ have the same Hilbert function, and thus the same arithmetic genus
$\binom{d-1}{2}$, but since curves of degree $d$ can have different sorts and numbers of singularities, they can have geometric genera from 0 to $\binom{d-1}{2}$. In this section we will explore the subset of (reduced, irreducible) curves of degree $d$ with a fixed geometric genus. We will focus on the open set consisting of curves with only nodes as singularities, which we call \emph{nodal curves}, and compute its dimension. 

\def\Vdg{{V_{g,d}}}
%\def\Vdgtilde{{\widetilde{V}_{g,d}}} 
\def\Vdgbar{{\overline{V}_{g,d}}} 

Let $\PP^N := \PP^{{d_1+2\choose 2} - 1}$ be the projective space parametrizing plane curves of degree $d$.
Within $\PP^N$ the set of reduced irreducible curves is open---it is the complement of the union of the images of the maps 
$$
\PP^{{d_1+2\choose 2} - 1}\times\PP^{{d_2+2\choose 2}-1} \to \PP^N
$$ 
with $d_1+d_2 = d$ given by multiplication of forms. 

\begin{propdef}
The \emph{Severi variety} $V_{g,d} \subset \Vdgbar$ is the locus of irreducible plane curves of degree $d$ with $\delta = \binom{d-1}{2} - g$ nodes and no other singularities . This is a locally closed subset of $\PP^N$.
(Reason: having only nodes as singularities is an open condition; having at least a certain number of them
is a closed condition.)
It is sometimes
called the \emph{small Severi variety}, since we are excluding curves with more complicated singularities.
\end{propdef}

%\begin{proof}
%\fix{ How about proving that $V_{g,d}$ is locally closed, and at least mentioning that $V_{g,d}$ is open in $\overline V_{g,d}$}
%\end{proof}

We will see that in a neighborhood of  $ {V}_{g,d}$,  the closure $\overline V_{g,d}$  is well behaved; but away from this,
even the singularities of $\overline V_{g,d}$  are not well understood. It is is an interesting open problem to find a simpler partial compactification of $ V_{g,d}$. 


\begin{fact}\label{severi irreducible}
As we shall see, the variety $V_{g,d}$ is smooth. In 1921 F. Severi gave an incorrect proof that $\Vdg$ was connected, and thus irreducible. This was finally proven in~\cite{MR837522}.
\end{fact}


\subsection{Local geometry of the Severi variety}\label{local severi geometry}

We first consider the \emph{universal singular point}
$$
\Phi := \left\{ (C, p) \in \PP^N \times \PP^2 \mid p \in C_{sing} \right\}
$$
and its image $\Delta\subset \PP^N$, the \emph{discriminant} variety. 

\begin{proposition}
 $\Phi$ is smooth and irreducible of dimension $N-1$, and the discriminant $\Delta$ is a hypersurface in $\PP^N$.
\end{proposition}
\begin{proof}
Projection on the second factor expresses $\Phi$ as a $\PP^{N-3}$-bundle over $\PP^2$. Explicitly, if $[X,Y,Z]$ are homogeneous coordinates on $\PP^2$, and $\{a_{i,j,k} \mid i+j+k = d \}$ are homogeneous coordinates on $\PP^N$, then the universal curve 
$$
\CC := \left\{ (C, p) \in \PP^N \times \PP^2 \mid p \in C \right\}
$$
is given as the zero locus of the single bihomogenous polynomial of bidegree $(1, d)$
$$
F([a_{i,j,k}], [X,Y,Z] ) = \sum a_{i,j,k} X^iY^jZ^k;
$$
and the universal singular point is the common zero locus of the three partial derivatives $\partial F/\partial X$, $\partial F/\partial Y$ and  $\partial F/\partial Z$. 

The set of forms $F$ that define curves singular at a given point is defined by 3 independent linear conditions, and since the set of 
points is 2-dimensional, the set $\Delta$ of singular forms has dimension $N-1$.
\end{proof}
 
We next compute the differential of the map $\pi : \Phi \to \PP^N$:

\begin{lemma}\label{tangent space to discriminant}
Suppose that $(C,p)\in \Phi$, with $p$ a node of $C$.  The differential 
$$
d\pi : T_{(C,p)}\Phi \to T_C \PP^N
$$
is injective, with image the hyperplane $H_p \subset \PP^N$ of plane curves containing the point $p$.
\end{lemma}

Thus, if $p$ is a node of $C$ and the only singularity of $C$, then $\Delta$ is smooth at $C$; and more generally the image of a small analytic neighborhood of $(C,p) \in \Phi$ is smooth, and we can identify its tangent space at $p$ with the hyperplane $H_p$. 

\begin{proof}
We will prove this using affine coordinates on $\PP^2$ and $\PP^N$. Changing coordinates if necessary, we may assume that the point $[1,0,0] \notin C$, and that the point $p$ is $[0,0,1]$. let $x = X/Z$ and $y = Y/Z$ be coordinates on the affine plane $Z \neq 0$ and write the polynomial $F(x,y,1)$ above as
$$
f(x,y) = \sum_{i+j \leq d} a_{i,j} x^iy^j 
$$
with $a_{d,0}$ normalized to 1. 

Let $g,h$ be the two partial derivatives of $f$, that is:
\begin{align*}
 g(x,y) := \frac{\partial f}{\partial x} &= \sum_{i+j \leq d} i a_{i,j} x^{i-1}y^j\\
h(x,y) := \frac{\partial f}{\partial y} &= \sum_{i+j \leq d} j a_{i,j} ix^{i}y^{j-1}.
\end{align*}
The functions $f, g$ and $h$ are local defining equations for $\Phi$; we consider their partial derivatives with respect to $x, y$ and $a_{0,0}$, evaluated at the point $(C,p)$, as in in Table~\ref{tang to Delta}.

\begin{table}[h!]\label{tang to Delta}
  \begin{center}
     \begin{tabular}{c|c|c|c} % <-- Alignments: 1st column left, 2nd middle and 3rd right, with vertical lines in between
            & $f$ & $g$ & $h$ \\
      \hline
$\frac{\partial}{\partial x}$ & 0 & $a_{2,0}$ & $a_{1,1}$ \\
$\frac{\partial}{\partial y}$ & 0 & $a_{1,1}$ & $a_{0,2}$ \\
$\frac{\partial}{\partial a_{0,0}}$ & 1 & 0 & 0 
    \end{tabular}
  \end{center}
\end{table}

The fact that $p$ is a node of $C$ (and not a more complicated singularity) implies that the upper right $2 \times 2$ submatrix is nonsingular, which shows that the differential $d\pi$ is injective, and its image is the hyperplane $a_{0,0} = 0$ in $\PP^N$, which is exactly the hyperplane of curves containing $p$.
\end{proof}

\begin{lemma}\label{adjoint independent}
The nodes $q_i$ of an irreducible nodal plane curve $C$ of degree $d$ impose independent conditions on curves of degree $d-3$, and hence on curves of any degree $m \geq d-3$.
\end{lemma}

\begin{proof}
We will prove in Chapter~\ref{PlaneCurvesChapter} that the $g$ sections of the canonical sheaf on the normalization $\widetilde C$ of
$C$ are the preimages of the sections of $\sO_C(d-3)$ that vanish at the nodes. On the other hand, 
$h^0(\sO_C(d-3)) = \binom{d-1}{2}$, and the difference is exactly the number of nodes.
\end{proof}

\begin{corollary}\label{local geometry of Severi}
If $C$ is a nodal curve of degree $d$ with geometric genus $g = \binom{d-1}{2}-\delta$, then in a neighborhood of $C\in \PP^N$
the discriminant hypersurface of all singular curves consists of $\delta$ smooth sheets, meeting transversely, and hence
$V_{g,d}$ is smooth. 

\begin{figure}
\centerline {\includegraphics[height=1.6in]{"main/Fig07-3"}}
 \caption{In a neighborhood of the point corresponding to a plane curve with 2 nodes, $V_{\binom{d-1}{2}-2, d}$ is the intersection of two smooth hypersurfaces intersecting transversely.}
\end{figure}

Moreover, in a neighborhood  $C \in \PP^N$ 
the variety $\overline V_{g,d'}$ with $g' =  \binom{d-1}{2}-\delta' > g$ is the union of $\binom{\delta}{\delta'}$ smooth branches, each of dimension $N - \delta'$, corresponding bijectively with subsets of $\{p_1,\dots,p_{\delta'}\}$ of cardinality $\delta'$.
\end{corollary}
\begin{proof}
Lemma~\ref{tangent space to discriminant} shows that in an analytic neighborhood of $C\in \PP^N$ the discriminant hypersurface $\Delta$  consists of $\delta$ smooth sheets, each corresponding to one node, and Lemma~\ref{adjoint independent} implies that the tangent spaces to these sheets are linearly independent. 
\end{proof}


\begin{corollary}\label{dim Severi}
The  Severi variety $V_{g,d}$ has pure dimension $N - \delta$, where $\delta = \binom{d-1}{2} - g$.
\end{corollary}

In Section~\ref{estimating dim hilb}, we give a heuristic calculation of the ``expected dimension'' $h(d,g,r)$ of the variety parametrizing curves of degree $d$ and genus $g$ in $\PP^r$:
$$
h(g,r,d) := 4g-3 + (r+1)(d-g+1) - 1.
$$
The actual dimension of the restricted Hilbert scheme may be quite different. But  Corollary~\ref{dim Severi} shows that in case $r=2$ (as in the case of $r=1$), the actual dimension is always the expected.



\section{Exercises}

\begin{exercise}
Consider the action of the multiplicative group $G_m$ on the affine line $\AA^1$. Does the action extend
to the projective line? What are the invariant functions? Find the sets $(\AA^1)^{\rm stable}, (\AA^1)^{\rm semistable}, (\AA^1)^{\rm unstable}.$
\end{exercise}

\begin{exercise}
Consider the action of $G_m$ on $\PP^3$ given in coordinates by 
$$t: (x_0,x_1,x_2,x_3) \mapsto (tx_0,\ tx_1,\ t^{-1}x_2,\ t^{-1}x_3)
$$
for $t\in G_m = \CC^*$.
\begin{enumerate}
 \item Show that the ring  of forms in $\CC[x_0, \dots, x_3]$ that are invariant is generated by
$$
x_0x_3, \ x_0x_2,\ x_1x_3,\ x_1x_2
$$
and thus $\PP^3\quot G_m\cong \PP^1\times \PP^1$.
\item Show that the unstable locus for this action is the union of the two lines $x_0=x_1=0$ and
$x_2=x_3=0$.
\item Show that the orbits of $G_m$ are the points on the unstable lines and, for each
point $p$ not on an unstable line, a copy of
$\PP^1\setminus \{0,\infty\}\cong G_m$ whose closure is the unique line containing $p$ and
meeting both unstable lines.
\end{enumerate}
\end{exercise}

\begin{exercise}
Consider the action of $G_m$ on $\PP^3$ given in coordinates by 
$$t: (x_0,x_1,x_2,x_3) \mapsto (tx_0,\ tx_1,\ tx_2,\ t^{-1}x_3)
$$
for $t\in G_m = \CC^*$.
\begin{enumerate}
 \item Show that the ring  of forms in $\CC[x_0, \dots, x_3]$ that are invariant is generated by
 forms
$$
F(x_0,  x_1, x_2)x_3
$$
where $F$ is a cubic form on $\PP^2$, and thus 
$\PP^3\quot G_m\cong \PP^2$, with the embedding given by the third Veronese map.
\item Show that the unstable locus for this action is the union of the point  $x_0=x_1=x_2 = 0$ and
the plane $x_3=0$.
\item Show that the orbits of $G_m$ are the points on the components of the unstable locus and, for each
point $p$ that is not unstable, a copy of
$\PP^1\setminus \{0,\infty\}\cong G_m$ whose closure is the unique line containing $p$ and the unstable
point. Thus the quotient map is the composition of the linear projection from the unstable point with the 3-uple
embeddi
\end{enumerate}
\end{exercise}

\begin{exercise}\label{not fine 1}
Show from the explicit formula~\ref{formula for j} for the $j$-function that if $j : B \to M_1 = \AA^1$ is a map  associated to a family $\cC \to B$ of curves of genus 1, then every zero of the $j$-function has multiplicity divisible by 3, and conclude that some maps $B\to M_1$ do not correspond to families of curves; in particular there is no universal family over $M_1$, and thus $M_1$ is not a fine moduli space for curves
of genus 1. There is a similar problem at $j(\lambda)=1728$. 
\end{exercise}

\begin{exercise}\label{not fine 2}
In Exercise~\ref{not fine 1} we saw a local obstruction to the existence of a universal family over $M_1$. There is also a global obstruction, coming from the fact that some genus 1 curves have extra automorphisms. Show that there is a ``tautological'' family over the punctured $j$-line $L := \AA^1\setminus \{0,1728\}$---that is, a family
$\sX \to L$ whose fiber over $t$ has $j$-invariant $t$; but show that this family is not universal as folows:
Let $B$ be any variety with a fixed-point free involution $\tau$, and let $E$ be a fixed elliptic curve (that is, curve of genus 1 with a chosen point, so that we may identify the points of $E$ with an abelian group.)
Let $\cX\to L$ be the family $E\times B$ modulo the equivalence relation $(e,b) \sim (-e, \tau(b))$.
The projection to $B/\tau$ has all fibers isomorphic to $E/(\pm) \cong E$. But the family is not
isomorphic to the trivial family $E\times B/\tau \to B/\tau$ \fix{Why??}
\end{exercise}

%footer for separate chapter files

\ifx\whole\undefined
%\makeatletter\def\@biblabel#1{#1]}\makeatother
\makeatletter \def\@biblabel#1{\ignorespaces} \makeatother
\bibliographystyle{msribib}
\bibliography{slag}

%%%% EXPLANATIONS:

% f and n
% some authors have all works collected at the end

\begingroup
%\catcode`\^\active
%if ^ is followed by 
% 1:  print f, gobble the following ^ and the next character
% 0:  print n, gobble the following ^
% any other letter: normal subscript
%\makeatletter
%\def^#1{\ifx1#1f\expandafter\@gobbletwo\else
%        \ifx0#1n\expandafter\expandafter\expandafter\@gobble
%        \else\sp{#1}\fi\fi}
%\makeatother
\let\moreadhoc\relax
\def\indexintro{%An author's cited works appear at the end of the
%author's entry; for conventions
%see the List of Citations on page~\pageref{loc}.  
%\smallbreak\noindent
%The letter `f' after a page number indicates a figure, `n' a footnote.
}
\printindex[gen]
\endgroup % end of \catcode
%requires makeindex
\end{document}
\else
\fi
