%header and footer for separate chapter files

\ifx\whole\undefined
\documentclass[12pt, leqno]{book}
\usepackage{graphicx}
\input style-for-curves.sty
\usepackage{hyperref}
\usepackage{showkeys} %This shows the labels.
%\usepackage{SLAG,msribib,local}
%\usepackage{amsmath,amscd,amsthm,amssymb,amsxtra,latexsym,epsfig,epic,graphics}
%\usepackage[matrix,arrow,curve]{xy}
%\usepackage{graphicx}
%\usepackage{diagrams}
%
%%\usepackage{amsrefs}
%%%%%%%%%%%%%%%%%%%%%%%%%%%%%%%%%%%%%%%%%%
%%\textwidth16cm
%%\textheight20cm
%%\topmargin-2cm
%\oddsidemargin.8cm
%\evensidemargin1cm
%
%%%%%%Definitions
%\input preamble.tex
%\input style-for-curves.sty
%\def\TU{{\bf U}}
%\def\AA{{\mathbb A}}
%\def\BB{{\mathbb B}}
%\def\CC{{\mathbb C}}
%\def\QQ{{\mathbb Q}}
%\def\RR{{\mathbb R}}
%\def\facet{{\bf facet}}
%\def\image{{\rm image}}
%\def\cE{{\cal E}}
%\def\cF{{\cal F}}
%\def\cG{{\cal G}}
%\def\cH{{\cal H}}
%\def\cHom{{{\cal H}om}}
%\def\h{{\rm h}}
% \def\bs{{Boij-S\"oderberg{} }}
%
%\makeatletter
%\def\Ddots{\mathinner{\mkern1mu\raise\p@
%\vbox{\kern7\p@\hbox{.}}\mkern2mu
%\raise4\p@\hbox{.}\mkern2mu\raise7\p@\hbox{.}\mkern1mu}}
%\makeatother

%%
%\pagestyle{myheadings}

%\input style-for-curves.tex
%\documentclass{cambridge7A}
%\usepackage{hatcher_revised} 
%\usepackage{3264}
   
\errorcontextlines=1000
%\usepackage{makeidx}
\let\see\relax
\usepackage{makeidx}
\makeindex
% \index{word} in the doc; \index{variety!algebraic} gives variety, algebraic
% PUT a % after each \index{***}

\overfullrule=5pt
\catcode`\@\active
\def@{\mskip1.5mu} %produce a small space in math with an @

\title{Personalities of Curves}
\author{\copyright David Eisenbud and Joe Harris}
%%\includeonly{%
%0-intro,01-ChowRingDogma,02-FirstExamples,03-Grassmannians,04-GeneralGrassmannians
%,05-VectorBundlesAndChernClasses,06-LinesOnHypersurfaces,07-SingularElementsOfLinearSeries,
%08-ParameterSpaces,
%bib
%}

\date{\today}
%%\date{}
%\title{Curves}
%%{\normalsize ***Preliminary Version***}} 
%\author{David Eisenbud and Joe Harris }
%
%\begin{document}

\begin{document}
\maketitle

\pagenumbering{roman}
\setcounter{page}{5}
%\begin{5}
%\end{5}
\pagenumbering{arabic}
\tableofcontents
\fi


\chapter{Free resolutions and canonical curves}
\label{SyzygiesChapter}

\def\length{\mathrm{ length}}

In Chapter~\ref{LinkageChapter} we related the free resolution of the homogeneous
coordinate ring of a curve $C$
in $\PP^3$ to the Hartshorne--Rao module $H^{1}_{*}(I_{C})$ of the curve. These 
are extrinsic invariants \emdash they depend on the particular embedding of the curve.
 In this chapter we will develop the
language and machinery to understand Green's conjecture, which posits a
subtle relationship between the intrinsic geometry of a curve and the free resolution
of the canonical image of that curve.

The first 2 sections of this
chapter constitute a quick review of some necessary parts of
 homological commutative algebra; see~\cite[Part III]{Eisenbud1995}
 for a more complete exposition. In particular, we
  explain how the condition that a curve
in $\PP^r$ is arithmetically Cohen--Macaulay manifests itself in the
free resolution of the homogeneous ideal of the curve, and we show
how the uniqueness of minimal free resolutions leads to the
classification of the matrices whose minors define 
scrolls. In section 3 we
introduce the Eagon--Northcott complexes and give a full proof of their
properties. We also explain their relation to
the free resolutions of rational normal scrolls. We close the chapter in section 4 with an
explanation of Green's conjecture.

\section{Free resolutions}

The basic results described here depend on 
\blue{Nakayama's lemma} and hold 
\index{Nakayama's lemma}%
 in two parallel contexts:
\blue{modules over a local ring,}
\index{module!over a local ring}%
\index{local ring!module over}%
and
\blue{graded modules over a polynomial ring}
\index{graded module}%
\index{polynomial ring}%
whose variables have positive degree. 
We will work both with local rings and with
polynomial rings whose variables have degree 1,
as appropriate, and leave the translations to the reader.
%%These are similar
%The similarity is essentially because
%%because
%\blue{Nakayama's lemma}
%\index{Nakayama's lemma}%
%works in both cases. We will work in either context, as convenient,
%%with local rings or with polynomial rings over a field with variables in
%%degree 1 as convenient, 



Let $M$ be a finitely generated graded module over $S \colonequals
\CC[x_0, \dots x_r]$. A \emph{free resolution} of $M$ is an
\index{exact complex}%
exact complex
\index{free resolution|defi}%
of graded free modules, with maps of degree~0:
$$
\let\quad\enspace
(\FF, \phi):\quad 0\leftarrow M \luto{\ \epsilon}
\tsty\bigoplus_jS(-j)^{@\beta_{0,j}}
\leftarrow\cdots
\luto{\ \phi_t}\bigoplus_jS(-j)^{@\beta_{t,j}}\leftarrow 0.
$$
Here $S(-j)$ denotes the graded free module of rank 1 with generator in
degree~$j$.
The map to $M$ is not considered part of the free resolution. The
resolution is called \emph{minimal} if a minimal set of generators of $F_i
\index{minimal free resolution}%
\index{free resolution!minimal}%
\colonequals  \bigoplus_jS(-j)^{\beta_{t,j}}$ maps to a minimal set of
generators of the kernel of the following map,
or equivalently (by Nakayama's lemma) the maps in $\FF\otimes_S \CC$
\index{Nakayama's lemma}%
are all 0, so $\beta_{i,j} = \dim_{\CC}\Tor_{i}^{S}(M, \CC).$ 
\index{Tor}%
The numbers $\beta_{i,j}$ are called the \blue{\emph{Betti numbers}}
\index{Betti numbers}%
of $M$. 

The Betti numbers of a graded module over the polynomial ring determine
the Hilbert function of the module as an alternating sum of the Hilbert functions of
the free modules in the resolutions \emdash this was Hilbert's original motivation
for proving the  syzygy  theorem. As we shall see in the last section of this chapter,
the Betti numbers are a much finer invariant.

Koszul complexes were
\index{Koszul complex}%
\index{historic context}%
first defined, despite the name, in \cite{Cayley}, and found
\index{Cayley, David}%
\index{Hilbert, David}%
as examples \cite{Hilbert1890}).
They can be used to resolve $S/I$ when
$I = (f_1,\dots, f_t)$
is a complete intersection, that is, $f_1,\dots, f_t$ is a regular
sequence.  For $t = 2,3$, if $\deg f_i = d_i$, these have the form
$$
\xymatrix@C=30pt{
S
&S(-d_1)\oplus S(-d_2) \ar[l]_{\null\hskip-10pt(f_1\ f_2)\hskip10pt\null}
&S(-d_1-d_2)\ar[l]_{\hskip5pt\tbinom{-f_2}{f_1}}
&0\ar[l]
}
$$
and
$$
\xymatrix@C=40pt{
S
&F_1 \ar[l]_{\hskip-3pt(f_1\ f_2\ f_3)}
&F_2 \ar[l]_{\Biggl(
\begin{smallmatrix}
0&f_3&\!-f_2\!\\
\!-f_3&0&f_1\!\\
f_2&\!-f_1&0\!
\end{smallmatrix}\Biggr)}
&F_3 \ar[l]_{\Biggl(
\begin{smallmatrix}
f_1\\f_2\\f_3
\end{smallmatrix}\Biggr)}
&0\ar[l]
}
$$
where
$$
\tsty
F_1 = \bigoplus\limits_{j=1}^3S(-d_j),\quad
F_2=
\bigoplus\limits_{1\leq i <j\leq 3} S(-d_i-d_j), \quad
F_3 =
S(-d_1-d_2-d_3).
$$
Minimal free resolutions of a given module $M$ are all isomorphic, and thus provide
interesting invariants of $M$.

To construct the minimal free resolution of a finitely generated graded module $M$, suppose that a minimal homogeneous set of generators
of $M$
contains $\beta_{0,j}$ generators of degree $j$ for each $j$; the choice of generators
defines a degree 0 map $\epsilon$
from
$
\oplus_jS(-j)^{\beta_{0,j}}
$
onto $M$. We proceed to do the same with the kernel of $\epsilon$ to construct $\phi_{1}$, and continue similarly to construct $\phi_{2}\dots$.
Hilbert's basis theorem
\index{Hilbert's basis theorem}%
and
syzygy theorem
\index{Hilbert's syzygy theorem}%
together imply that 
that for some $t\leq r+1$ we will find that $\phi_t$ has
kernel equal to 0; that is,
every finitely
generated $S$-module has a finite free resolution
of length $t\leq r+1$
\cite[Corollary 19.7]{Eisenbud1995}.
The minimal such $t$ is called the \emph{projective dimension} of $M$,
written $\pd\ M$.
\index{projective dimension}%
\index{pd@$\pd$}%
Computing $\Ext(M,-)$ from such a resolution, Nakayama's lemma implies that,
$$
t = \max\{s \mid \Ext_S^s(M,k) \neq 0 \}.
$$
It follows from the Auslander--Buchsbaum
\index{Auslander--Buchsbaum theorem}%
theorem \cite[Theorem 19.9]{Eisenbud1995} that $t \geq \codim \ann_S(M)$,
the codimension of the support of $M$.

\begin{fact}
 Fundamental
results of Auslander, Buchsbaum and Serre say that a local ring $R$
\index{Serre, Jean-Pierre}%
\index{Buchsbaum, David A.}%
\index{Auslander, Maurice}%
is \emph{regular}\emdash that is, the
\index{regular!ring|defi}%
\blue{Krull dimension}
\index{Krull dimension}%
of $R$ is equal to the
minimal number
of generators of is maximal ideal \emdash if and only~if the minimal free resolution
of the residue field is finite, in which case
the minimal free resolution of every module is finite.
\end{fact}
 



\subsection*{The classification of 1-generic \texorpdfstring{$2\times f$}{2 x f}
matrices}\label{Kronecker}

We can use the uniqueness of minimal free resolutions to give a simple
proof of Kronecker's classification of 1-generic $2\times f$
\index{Kronecker, Leopold}%
matrices, which was announced in Chapter~\ref{ScrollsChapter}.

\begin{theorem}\label{matrix pencils}
Every
$1$-generic $2 \times f$ matrix of linear forms can be transformed by
row and column operations and a linear change
of variables to one of the type shown in
Corollary~\ref{equations of scrolls}, and thus the minors of any 1-generic
matrix define a scroll.
\unif
\end{theorem}

To prove this result we first reinterpret the 1-generic condition:
We have observed that an
$a\times b$ matrix of linear forms in $c$ variables is the same as a
$\CC$-linear map of vector spaces
$A \otimes B \to C$, where $A, B$ and $C$ have dimensions $a,b$ and $c$
respectively. Such a map
can be viewed in several ways, for example as a map $C^{*} \otimes B\to
A^{*}$\emdash in other words, a $c\times b$ matrix in $a$ variables\emdash
or equivalently an $a$-dimensional family of $c\times b$ matrices (and
similarly for other permutations of $A,B,C$).
This bit of trivial formalism pays off in the following observation:

\begin{proposition}\label{reinterpretation of 1-generic}
An  $a\times b$ matrix of linear forms in $c$ variables $M$ corresponding
to $A\otimes B \to C$ is 1-generic if and only~if the $c \times b$ matrix
$N$ of linear forms in $a$ variables corresponding to $C^{*}\otimes B
\to A$ has constant rank $b$; that is,
for any point $x$ in $\PP^{a-1}$, the rank of $N$ evaluated at $x$ is $b$.
\unif
\end{proposition}

\begin{proof}
The $b$ columns of $N$ correspond to the $b$ columns of $M$, while the
rows of $N$ are indexed
by the $c$ variables in $M$ and the variables in $N$ are indexed by the
$a$ rows of $M$. Thus the
evaluation of $N$ at a point $x$ corresponds to a generalized row of $M$;
and the image of $N(x)$
is the span of the variables in that generalized row. The matrix $M$
is 1-generic if the dimension
of that span is $b$ for every generalized row.
\end{proof}

Thus the classification of 1-generic $2\times f$ matrices of linear
forms in $r+1$ variables of Theorem~\ref{matrix pencils} is equivalent
to the classification
of \emph{matrix pencils} of constant maximal rank\emdash that is $f \times
\index{matrix pencil}%
(r+1)$ matrices of linear forms over $\PP^{1}$ with constant rank $f$.
Such ``matrix pencils'' were first classified by Kronecker; see
\index{Kronecker, Leopold}%
\cite[Chapter 12]{Gantmacher} for an exposition, and
\cite{Eisenbud-Harris-Centennial} for a geometric approach.

\begin{proof}[Proof of Theorem~\ref{matrix pencils}]
Let $M$ be a 1-generic $2\times b$ matrix in $r+1$ variables. We may
assume that the span of the entries
is equal to the vector space of linear forms in $\PP^{r}$. The associated
$(r+1)\times b$ matrix
$$
N: \sO_{\PP^{1}}(-1)^{b} \to \sO_{\PP^{1}}^{r+1}
$$
of linear forms over $\PP^{1}$ has constant rank $b$.
For every point $x\in \PP^{1}$ the scalar matrix $N(x)$ is a split
inclusion, and thus
$\coker N$ is a vector bundle on $\PP^{1}$ that is generated by its global sections, and thus, necessarily of the form $\sum_{i
=1}^{d}\sO_{\PP^{1}}(a_{i})$ for some
integers $a_{i}\geq 0$.

Regarding $N$ as a matrix over the homogeneous coordinate ring $S\colonequals \CC[s,t]$ of $\PP^{1}$, and taking global sections of all
non-negative twists, we get
$$
\tsty
0\to S(-1)^{b} \ruto {N}
S^{r+1}  \to \bigoplus\limits_{i = 1}^{d}\,\bigl(
\bigoplus\limits_{j=0}^{\infty}H^{0}(\sO_{\PP^{1}}(j+a_{i}))\bigr)\to 0,
$$
which is a minimal free resolution because $H^{1}(\sO_{\PP^{1}}(j)) = 0$ for all $j\geq -1$.
It follows that the map $N$ is the direct
sum of the minimal $S$-free resolutions
of the modules
$$
\tsty
\bigoplus\limits_{j=0}^{\infty}H^{0}(\sO_{\PP^{1}}(j+a_{i})) = (s,t)^{a_{i}},
$$
where we take the generators to be in degree 0.
As we will explain in Example~\ref{res of max ideal power}, the minimal
presentation of $(s,t)^{a_{i}}$ is the $(a_{i}+1)\times a_{i}$ matrix
$$
\phi_a=
\left(\,
\begin{matrix}
s&0&0&\dots&0\\
\!-t&s&0&\dots&0\\
0&\!-t&s&\dots&0\\
0&0&\!-t&\dots&0\\
\vdots&\vdots&\rlap{\,$\ddots$}&&\vdots\\
0&0&0&\ddots&s\\
0&0&0&\cdots&\!-t\\
\end{matrix}
\right)
.
\label{phia} % for page number
$$
 Thus, by a change of bases and variables, the matrix $N$ can be
transformed into the direct sum of such matrices.
The direct sum decomposition of $N$ corresponds to a block decomposition
of $M$. Translating this
back to a $2\times (r+d-1)$ matrix $M$, we have
transformed $M$ into a matrix of
the desired form.
\end{proof}

\subsection*{How to look at a resolution}
As is apparent even in the example $t=3$ above, free resolutions can be
bulky to describe; the
\emph{Betti table} is a compact representation of the numerical
\index{Betti table}%
information in the resolution.
Suppose that
$F$ is a minimal free resolution of a graded module $M$ as illustrated in the
beginning of the previous section. Since we choose a minimal set of
generators at each stage, the matrices of the $\phi_i$ have entries in
the maximal
ideal $(x_0,\dots,x_r)$, and thus each $\beta_{i+1, j}$ must be strictly
greater than some $\beta_{i,j}$. For this reason it
is convenient to tabulate the Betti numbers so that $\beta_{i,j}$ is in the
$i$-th column and $(j-i)$-th row:
$$
\small
\begin{tabular}{r|@{\hskip17pt}ccccc}
\downstrut
$j$&\llap{$i={}$}0&1&$\cdots$&$n$\\
\hline\upstrut
$\vdots$&$\vdots$&$\vdots $&$     $&$\vdots     $\cr
0&$\beta_{0,0}$&$\beta_{1,1}$&$\cdots$&$\beta_{n,n}$\cr
1&$\beta_{0,1}$&$\beta_{1,2}$&$\cdots$&$\beta_{n,n+1}$\cr
$\vdots$&$\vdots$&$\vdots$ &    &$\vdots$     \cr
\end{tabular}
$$

\begin{example}
The Koszul complex that resolves the homogeneous coordinate ring $S/(Q,
\index{Koszul complex}%
F_1, F_2)$ of the complete intersection of  2 quadrics and a cubic in
$\PP^3$ has the form
$$
\let\lTo\leftarrow
S \lTo S(-2)\oplus S(-3)^2 \lTo S(-5)^2\oplus S(-6) \lTo S(-8) \lTo 0,
$$
which has Betti table
\index{Betti table}%
$$
\small
\begin{tabular}{r|@{\hskip20pt}ccccc}
\downstrut
$j$&\llap{$i={}$}0&1&2&3\\
\hline\upstrut
0&1&--&--&--\\
1&--&1&--&--\\
2&--&2&--&--\\
3&--&--&$2$&--\\
4&--&--&$1$&--\\
5&--&--&--&$1$\\
\end{tabular}
$$
where 
a dash
represents 0.  
\end{example}

To simplify our language, we will speak of the \emph{Betti table of a
scheme $X$} rather than the ``Betti table of the minimal free resolution
of the ideal of $X$''.

\subsection*{When is a finite free complex a resolution?}
How does a free resolution over $S \colonequals  \CC[x_0, \dots x_r]$
``know'' to end no later than the $(r+1)$-st step?
The main theorem of \cite{WMACE}, describes a sense in which the maps in the resolution
change as the resolution continues.See \cite[Theorem 20.9]{Eisenbud1995}
for further exposition.

A central role in the theorem is played by the ideals of minors of the
differentials in the complex: If $\phi: F\to G$ is a map of finitely
generated free $R$-modules with cokernel $M$, then
\blue{$I_t(\phi)$} 
\index{I@$I_t$}
denotes the ideal generated by all the $t\times t$ minors
(subdeterminants) of a matrix representing $\phi$; this is independent
of the choice of bases used to represent $\phi$ as a matrix.
The ideal $I_{\rank G -j}(\phi)$ depends only on $M$; it called the
$j$-th \emph{Fitting ideal} of $\coker \phi$  and
\index{ideal!Fitting}%
\index{Fitting ideal}%
\index{Fitting, Hans, 1906--1938}
usually written $\Fitt_{j} M$. Some basic properties of these ideals are
given in Exercise~\ref{Fitt}, where the reader may show that the
annihilator of $M$ has the
same radical as $\Fitt_{0} M$. Actually more is true:

\begin{fact}
If $F\to G \to M \to 0$ is an exact sequence of finitely generated
$R$-modules, then
$$
(\ann_{R}(\coker \phi))^{\rank G}\subset \Fitt_{0}\phi \subset
\ann_{R}(\coker \phi).
$$
For this and other such inequalities, see \cite{MR476736}.
\end{fact}

The \emph{rank} of $\phi: F\to G$ is by definition the largest size of a nonvanishing
\index{rank!of map of free modules|defi}%
minor of a matrix representing $\phi$,
or equivalently the largest $k$ such that the
\blue{exterior power}
\index{exterior power}%
$$\mwedge^{k}\phi : \mwedge^{k}@F \to \mwedge^{k}@G$$
is nonzero. 

We define
$I(\phi): = I_{\rank \phi}(\phi)$; if the source or target of $\phi$ is 0 then, since an empty
product is equal to 1,
we set $I(\phi) = R$.  The ideal $I(\phi)$ plays a special role: the cokernel
of $\phi$
is  
\blue{locally free}
index{free!locally}
 if and only~if $I(\phi) = R$. When
$I(\phi)$ contains a 
\blue{nonzerodivisor,}
\index{nonzerodivisor}%
 the construction localizes. We consider the local case:

\begin{lemma}\label{free coker}
If $R$ is a local ring and $\phi: F\to G$ is a map of free $R$ modules
with $G$ finitely generated,
then $\coker \phi$ is free if and only if $I(\phi) = R$, the unit ideal.
\end{lemma}

\begin{proof}
 If $H \colonequals \coker \phi$ is free, then we may write
 $G = G'\oplus H$ and $F = F'\oplus G'$, where $\phi$ is the projection to $G'$.
 Thus $\phi$ is the direct sum of a unit matrix and zero, whence $I(\phi) = R$.
 
 For the converse, set $r\colonequals \rank \phi$. If $I(\phi) = R$ then, because $R$ is local, some $r \times r$
 submatrix $\phi_{1}$ of $\phi$ has unit determinant, and after row and column operations
 we can write $\phi \cong \phi_{1}\oplus \phi_{2}$. If $\phi_{2} \neq 0$ the rank of $\phi$
 would be greater than $r$, so $\phi_{2} = 0$, and $\coker \phi$ is a free module
 isomorphic to the target of $\phi_{2}$.
\end{proof}

The \emph{grade} of an ideal $I$ is defined to be the length of a maximal regular
\index{grade!of ideal}%
sequence
contained $I$, or $\infty$ if $I=R$. If $R$ is a
\blue{Cohen--Macaulay ring}
\index{Cohen--Macaulay ring}%
such as
%say
$\CC[x_{0},\dots, x_{r}]$ then the grade
of any proper ideal is equal to its codimension, so grade becomes a
geometric notion.
Readers less familiar with commutative algebra will lose little if they
stick with the case when $R$ is
regular, or even the case when $R$ is $\CC[x_{0},\dots, x_{r}]$. This
suffices, for example, for the applications
\index{Eagon--Northcott complex}%
of the Eagon--Northcott complex described below.

\begin{theorem}\label{WMACE}  (\cite{WMACE}) 

Let $R$ be a Noetherian ring and let
$$
\FF:\quad
F_0\luto{\phi_1}F_1 \leftarrow \cdots \leftarrow F_{n-1}\luto{\phi_n}
F_n\leftarrow 0
$$
be a finite complex of free $S$-modules. Set $r_i \colonequals  \rank
\phi_i$.
The complex $@\FF$ is
\blue{\emph{acyclic}}
\index{acyclic|defi}%
(that is, $H_i(\FF) = 0$ for all $i>0$) if and only~if,
for all $i$,
$$
\rank F_i = r_i+r_{i+1}\quad\hbox{ and }\quad
\grade @I_{r_{i}}(\phi_i) \geq i.
\eqno\qed
$$
\end{theorem}


A familiar case occurs when  $r=1$ and $R$ is a domain. In this case the
theorem says that a map $F_1\to F_0$ is a monomorphism if and only~if
it becomes a monomorphism after tensoring with the field of rational
functions $K$, which follows from the flatness of
localization and the fact that $F_1$ is torsion-free, so that
$F_1 \subset F_1 \otimes K$. More generally, 
if $R$ is Noetherian, then a map of finitely generated
free modules
$\phi:F\to G$
is injective
if and only~if $I_{\rank F}(\phi)$ contains a nonzerodivisor.


To see the relevance of the grade hypothesis to the conclusion, suppose
for a moment that $R$ is
a regular local ring of dimension $r$, and suppose that the complex
$\FF$ is acyclic. The hypothesis $\grade I(\phi_{d+1}) \geq d+1$ can
only be satisfied if $I(\phi_{d+1}) = R$ (so that its grade is $\infty$
by convention). This  is equivalent to the cokernel of $\phi_{d+1}$
being free. Thus the theorem ``explains'' why a minimal free resolution
has length $\leq r+1$.

\section{Depth and the Cohen--Macaulay property}

If $M$ is a graded  $\CC[x_0, \dots x_r]$-module then an \emph{$M$-regular
sequence} is a sequence of homogeneous polynomials
$f_1,\dots,f_m \in (x_0,\dots, x_r)$ such that $f_1$ is a nonzerodivisor
on $M$, $f_2$ is a nonzerodivisor on $M/(f_1M)$, and so on.
The maximal length of such a sequence is called the \emph{depth} of $M$,
\index{depth!of a module|defi}%
or more properly the depth of $(x_0,\dots, x_r)$ on $M$.
The lengths of
all maximal $M$-regular sequences
\index{maximal $M$-regular sequence}%
are
the same:

\begin{theorem}[Auslander--Buchsbaum]\label{Auslander--Buchsbaum}
If $M$ is a finitely generated graded module
over $S \colonequals  \CC[x_0, \dots x_r]$,
then the length of every $M$-regular sequence is
$$m = r+1 - \pd\  M,$$
and $m$ is the smallest integer $m$ such that
$\Ext_S^m(S/(x_0, \dots x_r), M) \neq 0$.
\index{Ext@$Ext_S^m$}%
\qed
\end{theorem}

The depth of a module $M$ is bounded above by $\dim M$, the Krull
dimension. The reason is that if the dimension of $M$
is $d$, and $f_1 \in (x_0, \dots x_r) $ is a nonzerodivisor on $M$,
then $\dim M/(f_1)M= \dim M-1$. Thus by induction, if
$f_1,\dots, f_d$ is $M$-regular then $M/(f_1, \dots, f_d)M$ has
dimension 0, which is equivalent to its being Artinian. Thus any
$ f_{d+1} \in(x_0, \dots x_r) $ acts as a nilpotent endomorphism of
$M/(f_1, \dots, f_d)M$.

It follows from these facts that the depth of an $S$-module $M$ is equal
to the dimension of $M$ if and only~if the projective dimension
of $M$ is equal to the codimension of $M$; in this case we say that $M$
is a
\index{Cohen--Macaulay module}%
\emph{Cohen--Macaulay module}.

As we showed in Chapter~\ref{linkageChapter}, a curve $C\subset \PP^3$
is linked to a complete intersection
if and only~if
$H^1_*(\sI_C) \colonequals  \bigoplus_{m\in \ZZ} H^1(\sI_C(m)) = 0$,
in which case we said that $C$ was
\blue{arithmetically Cohen--Macaulay.}
\index{ACM}%
This is equivalent to the condition that the homogeneous coordinate
ring $R_{C}$ of $C$ is Cohen-Macaulay.
In general, we say that a projective scheme $X\subset \PP^{r}$
is arithmetically Cohen-Macaulay if its homogeneous coordinate ring $R_{X}$
is Cohen-Macaulay, and this is true
if and only~if $\pd R_X = \codim X$. 

\subsection*{The Gorenstein property}
An important homological condition on a scheme $X$ is the condition that
$\omega_X$ is an invertible sheaf; when this holds, we say that $X$
is \emph{quasi-Gorenstein}. When, in addition, $X$ is Cohen--Macaulay
\index{quasi-Gorenstein}%
\index{Gorenstein!module|defi}%
we say that $X$ is \emph{Gorenstein}. As with the Cohen--Macaulay property,
the Gorenstein property is interpreted locally
on a scheme. 
Any scheme that is locally a
complete intersection, such as any smooth scheme, is Gorenstein. Since
the restriction
of $\sO_{\PP^r}(1)$ to a subvariety is always invertible, saying that
a scheme $X$ is canonically embedded implies that
$X$ is at least quasi-Gorenstein. We say that a projective scheme $X$ is
\emph{arithmetically Gorenstein}
\index{arithmetically Gorenstein}%
if its homogeneous coordinate ring is Gorenstein, and it follows that
$\omega_{S/I} \cong S/I(a)$ for some integer $a = a(X)$.

In Chapter~\ref{LinkageChapter},  we expressed $\omega_X$
for a subscheme $X\subset Y$ of a scheme $Y$ as
$\Ext^{\codim X}_{\sO_Y}(\sO_X, \omega_Y)$. Slightly extending this idea,
if $C\subset \PP^r$ is a curve
with homogeneous coordinate ring $R_{C}$,
we define $\omega_{R_C}$ to be $\Ext^{r-1}_S(R_C, S(-r-1))$, where $S$
is the homogeneous coordinate ring of $\PP^r` `$.
Since $\omega_{\PP^r} = \sO_{\PP^r}(-r-1)$, the sheafification of this
module is  $\omega_C$.

If $C$ is arithmetically Cohen--Macaulay, so that
$\pd (C) = \codim C = r-1$,  then by
computing $\Ext^{r-1}_S(R_C, S(-r-1))$ from the 
\blue{minimal free resolution}
\index{minimal free resolution}%
$$
(\FF, \phi):\quad 
0\leftarrow R_C \leftarrow S\luto{\phi_1} F_1 \leftarrow \cdots \leftarrow 
F_{r-2} \luto{\!\phi_{r-1}}\hskip-15pt-\ F_r\leftarrow 0
$$
we see that $\omega_{R_C} = \coker \phi_{r-1}^*$. 
\index{omega@$\omega_{R_C}$}%
Theorem~\ref{WMACE}
implies that the complex $(\FF^*, \phi^*)$ which is the dual
of the  resolution $(\FF, \phi)$ is again acyclic, so it is the minimal
free resolution of $\omega_{R_C}$. Thus
$\omega_{R_C}$ is a Cohen--Macaulay module. Just as the Cohen--Macaulay
property of
$R_C$ implies that $R_C = H^0_*(\sO_C)$, it follows that $\omega_{R_C}
= H^0_*(\omega_C)$.

If, in addition, $C$ is a canonical curve, so that $\omega_C = \sO_C(1)$,
then we derive:
$$
\omega_{R_C} = \coker(\phi_{r-2}^*)(-r-1) = R_C(1)
$$
so $\coker \phi_{r-2}^* = R_C(r)$. Thus $(\FF^*(-r), \phi^*)$ is a
minimal free resolution of $R_C$, and is therefore isomorphic to
$(\FF^*, \phi^*)$; that is, $(\FF, \phi)$ is self-dual.
We have seen an example already
in the Koszul complex (a complete intersection is arithmetically Gorenstein).

Taking into account that in a resolution $(\FF, \phi)$ each summand
$S(-j)$ of $F_{i+1}$ can only
map to summands $S(-l)$ of $F_i$ with $\ell < j$, and similarly for the
dual, we see that the
Betti table of the minimal free resolution of a canonical curve of genus
\index{Betti table}%
$g\geq 4$ must have the form
$$
\small
\begin{tabular}{r|@{\hskip20pt}cccccc}
\downstrut
$j$&\llap{$i={}$}0&1&2&$\cdots$&$g{-}3$&$g{-}2$\\
\hline\upstrut
0&1&--&--&$\cdots$&--&--\\
1&--&$b_1$&$b_2$&$\cdots$&$b_{g-3}$&--\\
2&--&$b_{g-3}$&$b_{g-4}$&$\cdots$&$b_1$&--\\
3&--&--&--&$\cdots$&--&$1$\\
\end{tabular}
$$
It turns out that the $b_i$ depend
on the particular canonical curve,
but since the Hilbert function of the curve is the alternating sum of
\index{Hilbert function}%
the Hilbert functions in the resolution,
the differences $b_i- b_{g-i-1}$ are independent of the curve.
This together with the self-duality implies that the whole Betti table
of a canonical curve
is determined by the numbers $b_{1}, \dots, b_{g-2}$.

The geometry of linear series on $C$ influences these Betti numbers.
If $|\sL|$ is a $g^{r}_{d}$ on $C$ then the multiplication map
$H^0(\sL) \otimes H^0(\sL^{-1}\otimes \omega_C) \to H^0(\omega_C) =
H^0(\sO_C(1))$
corresponds to a $2\times h^{1}(\sL)$ matrix of linear forms on $\PP^{g-1}$
as in Chapter~\ref{ScrollsChapter}.
and
the ideal $I$ generated by the minors of this matrix is contained in the homogeneous
ideal of $C$. It turns out that the whole resolution of $I$ is a subcomplex of
the resolution of $R_{C}$. Resolutions of such determinantal ideals can be
described explicitly, and we turn now to this description.

\section{The Eagon--Northcott complex}\label{EN section}

The Eagon--Northcott complex $EN(\phi)$ \cite{MR0142592} associated with
\index{Eagon--Northcott complex}%
\index{EN@$EN(\phi)$}%
a matrix, or a map of free modules $\phi: F\to G$,
is a generalization of the Koszul complex, which is the case $\rank G =
\index{Koszul complex}%
1$. Like the Koszul complex,
it is tautological: its existence depends only on the properties
of commutative rings; and like the Koszul complex it is exact or
not depending on a property of the matrix $\phi$ related to regular
sequences. It is part of a family of complexes described in
\cite[Appendix A2]{Eisenbud1995}, and, from a more conceptual and general
point of view, in \cite{Weyman-book}.

We
are interested in $EN(\phi)$ because its shape
influences the shape of the free resolutions of canonical curves in an
interesting way,
leading to Green's conjecture. This conjecture, one of the central open
problems in the theory of algebraic curves, is described in the last
section of this chapter. We will also use the the Eagon--Northcott
complex, in a special case, to give a proof of the classification of
matrix pencils and an analysis of the ideals of ACM curves in $\PP^{3}$.

The special cases of $EN(\phi)$ where $\rank G = 1$, and where
$\rank G$ is arbitrary but $\rank F = \rank G + 1$, are of interest in their
own right and we will describe these first. The general case
is notationally more complicated, but the  ideas necessary for
describing it and proving its properties
 are all
present in these two special cases.

\smallbreak
\noindent
\underline{The case $\rank G = 1$}
\smallbreak

In this case $EN(\phi) = K(\phi)$ is just a
 Koszul complex. Let $\phi:F = R^{f}\to R$ be a homomorphism from
a free module to a ring $R$. We may write $EN(\phi)$ in the form
$$
S \luto{\,\delta_{1}} \!F \luto{\,\delta_{2}}\!\! \mwedge^{2}F
\luto{\,\delta_{3}} \!\cdots \luto{\,\delta_{f}}\!\!\mwedge^{f}F\leftarrow 0, 
$$
where $\delta_{1} = \phi$.

To define the complex, we must construct the differentials $\delta_{i}$
and prove that
$\delta_{i}\delta_{i+1} = 0$. Since the modules are free, it suffices
to do this for the
dual maps
$$
\partial_{i}: \mwedge^{i}F^{*} \to \mwedge^{i+1}F^{*},
$$
and it turns out that this is in a sense even more natural.

It is convenient to think of $R$ as an $S \colonequals  \ZZ[x_{1},\dots,
x_{f}]$-algebra by the map sending
$x_{i}$ to $\phi_{i}$; we  define the Koszul complex of $\phi$ over $R$
\index{Koszul complex}%
by tensoring
the Koszul complex of $(x_{1}, \dots, x_{f})$ with $R$.

Thus for the definition we take the map $\phi$ to be
$$
\phi: S^{f}\ruuuto {\ \
(x_{1} \,\dots \, x_{f})
} \hskip0.7em S.
$$

First of all, the map $\partial_{i}$ (like the map $\delta_{i}$) is
\emph{linear}: the image of a basis vector of $\mwedge^{i}@F^{*} $ is a
sum of variables times basis vectors
of $\mwedge^{i+1}@F^{*}$. We may write $S$ as $\Sym(V)$, where $V$ is the
free $\ZZ$-module generated by $x_{1}, \dots, x_{f}$, and we may think
of $F$ as the module $V\otimes S$ with the map
$\phi$ sending $V\otimes 1\subset F$ by the identity to $V = S_{1}\subset S$
\emdash the 
\blue{\emph{tautological map.}}
\index{tautological map}%
Let $t\in V\otimes V^{*}\subset S\otimes \mwedge \,V^{*}$ be the \blue{\emph{trace element}}
\index{trace}%
represented in terms of any basis $\{x_{i}\}$ of $V$
and dual basis $\{\hat e_{i}\}$ of $V^{*}$ as $t = \sum x_{i}\otimes
\hat e_{i}$. Because $\mwedge \,V^{*}$ is
an anti-commutative algebra, we have $t^{2} = 0$.

We define the map
$$
\partial_{i}: S\otimes_{\CC} \mwedge^{i}@V^{*} = \mwedge^{i}@F^{*}  \to
\mwedge^{i+1}@F^{*} = S\otimes_{\CC} \mwedge^{i+1}@V^{*}
$$
to be multiplication by $t$, and thus $\partial_{i+1}\partial_{i}$ is
multiplication by $t^{2} = 0$.

Having defined the complex $K(\phi) = EN(\phi)$ in the case $\rank G =
1$, we next ask what conditions on $\phi$
make it acyclic (that is, a free resolution of $\coker \delta_{1}$).

\begin{theorem}\label{rankG1}
Suppose that $R$ is a ring and $\phi: F\to R$ is a map from a free
$R$-module of rank $f$.
The complex $K(\phi)$ is acyclic if and only~if the ideal $I \colonequals
I_{1}(\phi)$ has grade $\geq f$.
\index{grade}%
\end{theorem}

\begin{proof}
Theorem~\ref{WMACE} 
directly 
implies that if $K(\phi)$ is acyclic then
$\rank \delta_{f} = 1$ and $\grade I(\delta_{f}) \geq f$. Since
$I(\delta_{f}) = I$, this proves one implication.

Now assume that $\grade I \geq f$. We first prove that
$K(\phi)$ is 
\blue{split exact}
\index{split exact}%
when $I = R$; that is, 
$\mwedge^{i}@F = \ker \delta_{i} \oplus \im \delta_{i}$ 
for every $i$, or equivalently
$\mwedge^{i}@F^{*} = \im \partial_{i}\oplus\coker \partial_{i}$ for
every $i$. The condition $I=R$ implies that $\delta_{1}$ is s split
surjection,
or equivalently that
$\partial_{1}$ is a split injection. In this case we may write $F^{*}
= R\oplus F'^{*}$ in such a way that $\partial_{1}$ is the injection
into the first summand, and we may
choose a basis $\{\hat e_{i}\}$ of $F^{*}$ so that the last $f-1$ basis
elements are a basis for $F'^{*}$.
Specializing the sequence $x_{1},x_{2}, \dots, x_{f}$ to the sequence $1,
0,\dots, 0$, the differential of $K(\phi)^{*}$
becomes the multiplication by  $1\otimes e_{1}$.

The module
$\mwedge^{i}@F^{*}$  decomposes as
$$
\mwedge^{i}@F^{*} = \bigl(Re_{1}\otimes_{R} \mwedge^{i-1}@F^{*} \bigr)
\oplus \mwedge^{i}@F'^{*}.
$$
Because $e_{1}\wedge e_{1} = 0$ the differential $\partial_{i}: \wedge^{i}F^{*} \to \wedge^{i+1}F^{*}$ has the form
$$
\begin{diagram}
&&Re_{1}\otimes \wedge^{i-1}F'^{*} &\rTo^{0}&  Re_{1}\otimes \wedge^{i}F'^{*}\\
\wedge^{i}F^{*}&=& \bigoplus&\ruTo^{\cong}&\bigoplus&=&\wedge^{i+1}F^{*}\\
 &&\wedge^{i}F'^{*}&\rTo^{0}& \wedge^{i+1}F'^{*}
\end{diagram}.
$$
Thus we see that $K(\phi)$ is split exact when $\phi$ is a split
surjection.

We now assume only that $\grade I\geq f$. From what we just proved we
see that if we localize
$R$ by inverting any element of $I$ the complex $K(\phi)$ becomes split
exact. Since $\grade f\geq 1$,
we can find such  a nonzerodivisor in $I$, and inverting it does not
change the ranks of the
maps $\phi_{i}$. Because ranks of free modules are additive in direct
sums, it is obvious that
in the split exact case the condition on the ranks of the $\phi_{i}$
is satisfied; more precisely,
$\rank(\delta_{i}) = \tbinom{i-1}{f-1}$. We also see that after inverting
a nonzerodivisor in $I(\delta_{1})$ we have $I(\delta_{i}) = R$;
equivalently,
$$
\tsty % tame the square root
I  \subset \sqrt {I(\delta_{i})}.
$$
(In fact $I(\delta_{i}) = I^{\sbinom{i-1}{f-1}}$, though this requires a
separate argument.) Thus if $\grade I = f$ then  $\grade I(\delta_{i})
\geq i$ for all $i$, so $K(\phi)$ is acyclic.
\end{proof}

\smallbreak
\noindent
\underline{The case $\rank F=\rank G + 1$}
\smallbreak

We set $g=\rank G$ and $f = \rank F = g+1$. In this case the
Eagon--Northcott complex has the form
$$
EN(\phi):\quad 0\to G^{*}\otimes \mwedge^{f}F \ruto {\delta_{2}}
\mwedge^{f-1}F \ruto {\delta_{1}} \mwedge^g@G.
$$
Here $\delta_{1} = \mwedge^{g}\phi$, so that the entries of a matrix for
$\delta_{1}$ are the $g\times g$ minors of
$\phi$.

We choose an identification $\mwedge^{f}F = S$, called an
\index{orientation}%
\blue{\emph{orientation}}
of $F$, and get a perfect pairing
$$
\mwedge^{g}F \times F \to \mwedge^{f}F = S
$$
so that we may identify
$\mwedge^{g}@F$ with $F^{*}$. With this identification, we define
$\delta_{2}$ as
$$
\delta_{2}:  G^{*}\ruto {\phi^{*}}  F^{*} = \mwedge^{g}@F.
$$
We also choose an orientation $\mwedge^{g}@G = S$, in terms of which the
image of $\delta_{1}$ is
the ideal generated by the $(f{-}1)\times (f{-}1)$  minors of $\phi$.

We first claim that $EN(\phi)$ is a complex; that is,
$\delta_{1}\delta_{2} = 0$.  As with the Koszul complex,
\index{Koszul complex}%
it is convenient to dualize and consider the maps
$$
EN(\phi)^{*}:\quad 0\to \mwedge^{g} @
G \to \mwedge^{g}@F^{*} = F\ruto{\,\phi} G
.
$$
The fact that this composition is 0 is often taught as 
\blue{Cramer's rule}
\index{Cramer's rule}%
for solving a system of
homogeneous equations represented by a $g\times (g+1)$ matrix of rank $g$.
The solutions\emdash that is, the elements of $\ker \phi$\emdash are
multiples of the column
$\Delta_{1}, \dots, \Delta_{g}$ where the $\Delta_{j}$ is 
$(-1)^{j}$
times the determinant
of the matrix obtained from $\phi$ by leaving out the $j$-th column. This
works
because the composition of the two maps is a column matrix whose $i$-th
entry is the
expansion of the $(g+1)\times (g+1)$ determinant of the matrix obtained
from $\phi$ by
repeating the $i$-th row.

\begin{theorem}\label{EN grade 2}
Suppose that $R$ is a ring and $\phi: F\to G$ is a map of  free
$R$-modules,
where $G$ has rank $g$ and $F$ has rank $f = g+1$.
The complex $EN(\phi)$ is acyclic if and only~if the ideal $I
\colonequals  I_{g}(\phi)$ has grade $\geq 2$.
\end{theorem}

\begin{example}
We have seen that the ideal of the twisted cubic is generated by the
$2\times 2$ minors of the matrix
$$
\phi \colonequals
\begin{pmatrix}
x_{0}&x_{1}&x_{2}\\
x_{1}&x_{2}&x_{3}
\end{pmatrix}
$$
and it follows that the free resolution of its homogeneous coordinate
ring is the Eagon--Northcott complex
$$
0\to S^{2}(-3)
\hskip1pt
-\hskip-4pt-\hskip-4pt\ruuto {\hskip-13pt\vbox{\hbox{$\biggl(\!\begin{smallmatrix}
x_{0}&x_{1}\\
x_{1}&x_{2}\\
x_{2}&x_{3}
\end{smallmatrix}\!\biggr)$}\vskip0pt}}
S^{3}(-2)
\ruuto {\phi\wedge\phi}
S
\marginparhere{used $\phi\wedge\phi$ instead of big wedge, ok?}
$$
\end{example}

\begin{example}\label{res of max ideal power}
In Section~\ref{Kronecker} we asserted that  the $(a+1)\times a$  matrix
$\phi_{a}$ given on page~\pageref{phia}
is the minimal presentation of the ideal $(s^{a},s^{a-1}t, \dots, t^{a})
\subset R\colonequals \CC[s,t]$. It is not hard to check this
directly, but in any case it's easy to see that its $a\times a$ minors of
$\phi_{a}$ generate this ideal, which has grade 2, so the Eagon--Northcott
resolution $EN(\phi)$
has the form
$$
R\luto{\ \ssty\bigwedge^a\!\phi_a}\hskip-23pt-\hskip-4pt-
\hskip2pt R^{a+1} 
\luto{@\phi_{a}} R^{a} \leftarrow 0
,
$$
verifying the assertion.
\end{example}

\begin{proof}
Once having shown that $EN(\phi)$ is a complex, as we did above,  the
proof of the equivalence in the theorem follows the same pattern as the
proof given above for the Koszul complex.
\index{Koszul complex}%

If $EN(\phi)$ is acyclic, then by Theorem~\ref{WMACE} the $g\times g$
minors of $\phi = \delta_{2}$ must
have grade $\geq 2$.
For the converse, suppose first that
$I_{g}(\phi)$ is the
unit ideal. We may split  $F$ as  $S\oplus G$ with $\Delta_{1} = 1$
and $\Delta_{j} = 0$
for $j>1$, and then $EN(\phi)^{*}$ has the form
$0\to G^{*} \to F^{*} \to S$ with maps as given below:
$$
\begin{diagram}
&&0&\rTo&  G^{*}&\rTo^{0}&S\\
0&\rTo& \bigoplus&\ruTo^{\cong}&\bigoplus&\ruTo^{\cong}&\bigoplus\\
 &&G^{*}&\rTo^{0}&S&\rTo&0
\end{diagram}.
$$
%
%\marginpar{\vskip10pt transposition ok?}
%\vspace*{-5pt}
%$$
%\def\messy{\vbox{\hbox{$\ssty\smash{\cong}$}\vskip-3pt}}
%\xymatrix@C=3pt@R=13pt{&0\ar[d]\\
%G\ar[d]_0\ar[drr]^{\messy}  & \oplus&0\ar[d]^0\\
%S\ar[d]\ar[drr]^{\messy} &\oplus&G\ar[d]^0\\
%0&\oplus& S}
%$$
Thus 
$EN(\phi)$ is split exact in this case.

We now apply Theorem~\ref{WMACE}: From what we just proved we see that
if we localize
$S$ by inverting any element of $I$ then the complex $EN(\phi)$ becomes
split exact,
and therefore, before localizing,
$\rank(\delta_{1}) = 1$ and $\rank \delta_{2} = g$ . In this
case it follows from the definition that $I_{1}(\delta_{1}) = I_{g}(\phi)
= I_{g}(\delta_{2})$
so if $I_{g}(\phi)$ has grade 2 then both conditions of
Theorem~\ref{WMACE} are
satisfied.
\end{proof}

\subsection*{The Hilbert--\kern-0.5pt Burch theorem}

\hskip-3pt
In a regular local ring any ideal of whose primary components all have 
\null codimension\kern1.5pt 1 is 
principal (divisors are all Cartier). What about
ideals of codimension 2? The answer is the content of the
\emph{Hilbert--Burch} theorem, proven
in 1890 by David Hilbert in the case of homogeneous ideals in
\index{Hilbert, David}%
\index{Burch Lindsay}%
$\CC[x_{0},x_{1}]$ and in general by
Lindsay Burch \citeyear{MR212008}. We can deduce it as an application of
the Eagon--Northcott complex
in the case $f =g+1$:

\begin{corollary}[Hilbert--Burch theorem]\label{Hilbert--Burch}
Suppose that $R$ is a local ring. Any ideal $I\subset R$ of projective
\index{Hilbert--Burch theorem}%
dimension \1 has the form
$aI'$ where $I'$ is an ideal of grade 2 generated by the $g\times g$
minors
of a $g \times (g+1)$ matrix and $a$ is a nonzerodivisor of $R$; and
conversely any ideal of this form
has projective dimension \1.

In particular, if $C\subset \PP^{3}$ is an 
\blue{ACM curve}
\index{ACM!curve}%
whose homogeneous
ideal $I$ is generated by
$f$ elements, then $I$ is minimally generated by the $(f-1)\times (f-1)$
minors of the 
\blue{syzygy matrix}
\index{syzygy matrix}%
 of $I$.
\unif
\end{corollary}

\begin{proof}
If $C\subset \PP^{3}$ is ACM, then the projective dimension of the
homogeneous coordinate ring $R_{C}$
is 2  by the 
\blue{Auslander--Buchsbaum theorem,}
\index{Auslander--Buchsbaum theorem}%
and thus the ideal of $C$
has projective dimension 1.

Now suppose that $I\subset R$ is an ideal of projective dimension 1 in
any local ring, and suppose
that $I$ is generated by $f$ elements, so that we have a surjection
$F\colonequals  R^{f} \to I$.  The module $R/I$
has free resolution of the form
$$
\FF:\quad R\luto{\ \alpha} R^{f}\luto{\ \phi} G\leftarrow 0,
$$
where $I = I_{1}(\alpha)$, so by Theorem~\ref{WMACE} the free module $G$
has rank $g = f-1$, the $g\times g$
minors of $\phi$ generate an ideal $I'$ of grade $\geq 2$, and the ideal
$I$ has grade 
at least
$1$. Theorem~\ref{WMACE} implies
that both the Eagon--Northcott complex $EN(\phi)$
and its dual are acyclic.

The dual of the complex $\FF$ will not be acyclic unless 
$\grade I = 2$, but there is at least a comparison map
\vspace*{-5pt}
$$
\xymatrix@C=30pt{
\llap{$\FF^{*}:\quad$}
G^{*}\ar[d]^=&F^*\ar[l]_{\phi^*}\ar[d]^=&R\ar[l]_{\alpha^*}\ar[d]_a &0\ar[l]\\
\llap{$EN(\phi)^{*}:\quad$}
G^{*}&F^*\ar[l]_{\phi^*}&R\ar[l]_{\ssty\bigwedge^g``\phi^{*}}&0\ar[l] \\
}
$$
It follows that $I = aI'$, and since $I$ has grade at least 1, $a$ must be a
nonzerodivisor.

Conversely, if $\phi: R^{f}\to R^{g}$ is a map with $f = g+1$ and $\grade
I_{g}(\phi)\geq 2$,
then the acyclicity of $EN(\phi)$ shows that $I_{g}(\phi)$ has projective
dimension 1; and if
$a$ is a 
\blue{nonzerodivisor,}
\index{nonzerodivisor}%
then $I \colonequals  aI_{g}(\phi) \cong
I_{g}(\phi)$ as $R$-modules, so
$I$ has projective dimension 1 as well.
\unif
\end{proof}

This argument applies, for example,
to the case of a 
\blue{nonhyperelliptic curve} 
\index{nonhyperelliptic curve!of genus 3 and degree 6}%
of genus 3 and
degree 6 in $\PP^{3}$, discussed on page~\pageref{other genus 3}.

\subsection*{The general case of the Eagon--Northcott complex}

With these two special cases in hand, we are ready for the general
case. 

\begin{definition}
If $\phi: F\to G$ is a map of free $S$-modules with 
$f\colonequals \rank F\geq  g\colonequals  \rank G$, then the
\emph{Eagon--Northcott complex of $\phi$}
\index{Eagon--Northcott complex}%
is a complex of free $S$-modules
\begin{multline*}
EN(\phi): \quad
S \luto{\delta_1}
%\hskip-30pt-\hskip-4pt-\hskip-4pt-
%\hskip3pt
\mwedge^g @F
\luto{@\delta_{2}}
G^*\otimes \mwedge^{g+1}@ F  \luto{@\delta_{3}}
(\Sym^2G)^*\otimes\mwedge^{g+2}@F  \\
\luto{\delta_{4}}\cdots\luto{\delta_{f-g+1}}\hskip-23pt-\hskip-4pt-
(\Sym^{f-g}G)^*\otimes\mwedge^fF
\leftarrow 0
\end{multline*}
with the following properties:
\begin{enumerate}

\item After identifying $\mwedge^{g}G$ with $S$, the map $\delta_{1}$
is identified with $\mwedge^{g}\phi$.

\item For $i>1$ the dual of the differential $\delta_{i}$,
$$
\partial_{i} \colonequals \delta_{i}^{*}: \Sym^{i-2}G \otimes \mwedge^{g+i-2}@F^{*} \to
\Sym^{i-1}G \otimes \mwedge^{g+i-1}@F^{*}
$$ 
is multiplication by the trace element element
\index{trace}%
$\sum_{i = 1}^{f} \phi(e_{i}) \otimes \hat e_{i}$
where
$\{e_{i}\}$ and $\{\hat e_{i}\}$ are dual bases for $F$ and $F^{*}$, respectively.

%\item It is convenient to give a formula for $\partial_{i} =
%\delta_{i}^{*}$ by taking advantage of the
%algebra structures of $\Sym(G)$ and $\mwedge F^{*}$. To do this, choose
%dual bases $\{e_{i}\}$ and $\{\hat e_{i}\}$ for $F$ and $F^{*}$. In
%these terms
%$$
%\delta_{i}^{*} = \partial_{i}:
%\Sym^{i-2}G \otimes \mwedge^{g+i-2}@F^{*} \to
%\Sym^{i-1}G \otimes \mwedge^{g+i-1}@F^{*}
%$$
%is multiplication by the element
%$\sum_{i = 1}^{f} \phi(e_{i}) \otimes \hat e_{i}$.
\end{enumerate}
\end{definition}

\begin{proof}[Proof that $EN(\phi)$ is a complex]
 The proof that $\delta_{1}\delta_{2} = 0$ is almost the same as in the
case $f = g+1$ because
a basis element
$$
b: = e_{i_{1}}\wedge \cdots \wedge e_{i_{g+1}} \in \mwedge^{g+1}@F
$$
can be thought of as coming from a rank $g+1$ summand of $F$, and the
value of $\delta_{2}\delta_{1}b$
is the same as it would be if $F$ were replaced by this summand. Thus
from the case $f=g+1$ we see
that $\delta_{1}\delta_{2}b = 0$, and thus $\delta_{2}\delta_{1} = 0$.

On the other hand if $i\geq 1$ then, just as in the case $f=1$,
 the map $\partial_{i+1}\partial_{i}$
is multiplication by
$$
\biggl(@\sum_{i = 1}^{f} \phi(e_{i}) \otimes \hat e_{i}\biggr)^{\!2},
$$
which we may think of as the square of an element of degree 1 in the
exterior algebra
of the free $\Sym(G)$-module $\mwedge@(\Sym(G)\otimes F^{*}) = \Sym(G)
\otimes \mwedge \,F^{*}$, and hence this square is 0.
\end{proof}

\begin{example}
If $\phi$ is a matrix of linear forms, then the first map of $EN(\phi)$
is represented by the
row of $g\times g$ minors of $\phi$, which are forms of degree $g$,
but all the rest of the maps
are represented by matrices of linear forms. Thus, for example, the
Betti table of the Eagon--Northcott complex of
\index{Betti table}%
a $2\times f$ matrix of linear forms is
$$
\small
\begin{tabular}{r|@{\hskip20pt}ccccc}
\downstrut
$j$&\llap{$i={}$}0&1&2&$\cdots$&$f{-}1$\\
\hline
\upstrut
0&1&--&--&$\cdots$&--\\
1&--&$\tbinom{f}{2}$&$2\tbinom{f}{3}$&$\cdots$&$(f-1)\tbinom{f}{f}$\\
\end{tabular}
$$
\end{example}



\begin{theorem}\label{ENgeneral}
Suppose that $R$ is a ring and $\phi: F\to G$ is a map of  free
$R$-modules,
where $G$ has rank $g$ and $F$ has rank $f \geq g$.
The complex $EN(\phi)$ is acyclic if and only~if the ideal 
\index{acyclic!EN complex}%
$I \colonequals  I_{g}(\phi)$ has 
\index{grade}%
grade $\geq f-g+1$.
\end{theorem}

\begin{proof}
The dual of the last differential of $EN(\phi)$ is
$$
\partial_{f-g+1}: \Sym^{f-g-1}G \otimes \mwedge^{f-1}@F^{*} \to
\Sym^{f-g}(G) \otimes \mwedge^{f}F^{*}.
$$
After choosing a generator of $\mwedge^{f}F^{*}$ we
may identify $\mwedge^{f}F^{*}$ with $S$ and $\mwedge^{f-1}F^{*}$ with $F$,
and the map $\partial_{f-g+1}$
is then identified with the multiplication map
$$
\Sym^{f-g-1}G \otimes F \ruto {\!1\cdot \phi} \Sym^{f-g}G
,
$$
whose cokernel is $\Sym^{f-g}(\coker \phi)$ by the right exactness of
the symmetric algebra functor \cite[Proposition A2.2]{Eisenbud1995}. 

Because $\Sym$ is a multilinear functor,
the
support of $\Sym^{f-g}(\coker \phi)$ is contained in the support
of $\coker \phi$, but in fact they are equal,
because
if a localization of
$\coker \phi$, over a local ring $S_{P}$,
is nonzero, then by
\blue{Nakayama's lemma}
it surjects onto $S_{P}/P_{P} = \kappa(P)$, and again
\index{Nakayama's lemma}%
by the right exactness
of the symmetric algebra functor $\Sym^{f-g}(\coker \phi)_{P}$ surjects
onto
$\Sym^{f-g}(\kappa(P)) = \kappa(P)$.

By Theorem~\ref{WMACE} we see from this that if $EN(\phi)$ is acyclic,
then the support of $\coker \phi$
has grade $\geq f-g+1$. This support is defined by the radical of
$I_{g}(\phi)$, so $\grade I_{g}(\phi)\geq f-g+1$ as required.

Conversely, to show that $EN(\phi)$ is acyclic under the given hypothesis
we first treat the case where $R$ is local and 
$I_{g}(\phi) = S$, and prove in this case that $EN(\phi)^{*}$ is split exact. This
is the most complicated part of the proof,
but, given our understanding of the Koszul complex, it is purely formal.

In this case the cokernel of $\phi$ is 0, so we may split $F$ and
assume that $F = G\oplus F'$, the map $\phi$ being the projection onto
the first summand. Using the splitting of $F$ we may write
split the exterior powers of $F$ as  
$$
\mwedge^{i}F =\tsty\bigoplus\limits_{j=0}^{i}\mwedge^{j}G\otimes \mwedge^{i-j}F'.
$$

In terms of the splitting of $\mwedge^{g}F$, the map $\delta_{1}$ 
%$$
%\mwedge^{g}G\lTo^{\delta_{1}}
%\tsty\bigoplus\limits_{j=0}^{g}\mwedge^{j}G\otimes \mwedge^{g-j}F'
%$$
is the projection onto the  summand with $j=g$, so we must show that
that in this case
 $EN(\phi)$ is a split exact resolution of $\mwedge^{g}G$ together with
 its augmentation map, a projection onto $\mwedge^{g}G$.
 
It is simplest to prove the corresponding statement for the dual complex. 
Assuming that the summand $G\subset F$ has basis $e_{1},
\dots, e_{g}$,
the dual differential $\partial_{i}$ takes the form $\sum_{i=1}^{g}
e_{i}\otimes \hat e_{i}$.
For $i\geq 1$ the terms of the $EN(\phi)^{*}$ decompose as
$$
EN_{i}(\phi)^{*} = \Sym^{i-1}G \otimes  \mwedge^{g+i-1}F^{*}  =
\tsty\bigoplus\limits_{j=0}^{g} \Sym^{i-1}G \otimes  \mwedge^{j}G^{*} \otimes
\mwedge^{g+i-1-j}F'^{*}.
$$
The map $\partial_{i}= \delta_{i}^{*}$ is a direct sum
of the maps
$$
\Sym^{i-1}G \otimes  \mwedge^{j}G^{*} \otimes \mwedge^{g+i-1-j}F'^{*}
\to
\Sym^{i}G \otimes  \mwedge^{j+1}G^{*} \otimes \mwedge^{g+i-1-j}F'^{*}
$$
given by the tensor product of multiplication by $\sum_{k=1}^{g} e_{k}\otimes
\hat e_{k}$
on the first two tensor factors and the identity on the last tensor factor,
$\mwedge^{g+i-1-j}F'^{*}$.

Thus, setting $\ell = g+i-1-j$, we see that   $EN(\phi)^{*}$ is the direct sum for
$0\leq \ell \leq f-g$ of the free module 
$\mwedge^{\ell}F'^{*}$ tensored with the complex
\begin{equation}
%\Sym^{0}G\otimes\mwedge^{j}G^{*}\to\cdots \to 
\cdots\to \Sym^{\ell + j -g}G \otimes  \mwedge^{j}G^{*}  \to
 \Sym^{\ell + j -g+1}G \otimes  \mwedge^{j+1}G^{*}\cdots\to .
\tag{$*_\ell$}
\label{starj}
\end{equation}


Let $R \colonequals \Sym G = S[e_{1}, \dots, e_{g}]$,  write 
$\phi': R^{g} \to R$ for the map sending the $i$-th basis
element of $R^{g}$ to the element $e_{i}\in R$, and let $K(\phi') = EN(\phi')$
be the Koszul complex. The $R$-dual $\Hom(K(\phi'), R)$ 
\index{Koszul complex}%
decomposes as
$$
R\otimes_{S}\mwedge G^* =   \Sym(G)\otimes_{S} \mwedge G^* = 
\tsty\bigoplus\limits_{p,q}\Sym^{p}G\otimes\mwedge^{q}@G^*
$$
and has differential equal to multiplication by the trace element $\sum_{k=1}^{g} e_{k}\otimes
\hat e_{k}$. Each complex $(*_{\ell})$ is a direct summand of this complex.

% 1-term complex 
%\begin{equation}
%\Sym^{0}G\otimes\mwedge^{j}G^{*}\to\cdots \to \Sym^{i}G
%\otimes  \mwedge^{i+j}G^{*}  \to \cdots
%\tag{$*_g$}
%\label{starj}
%\end{equation}

We 
proved
in Theorem~\ref{rankG1} that $K(R)$ is a free resolution
of $R/(e_1, \dots, e_g)=S$ via the isomorphism
 $\Sym^{0}G\otimes \mwedge^{0}@G^*\to S` `$. Thus,
 as a complex of $S$-modules, $K(R)$
 is exact with homology $S$ in degree 0, and since it is free as an $S$-module,
 it becomes a split exact complex if we add this isomorphism.
 
It follows
that each of the complexes $(*_{\ell})$ is split exact except for the
one-term complex $(*_{0})$, which is
$$
0\to \Sym^{0}G\otimes\mwedge^{g}G^{*}\to 0,
$$
and we see finally that $EN(\phi)^{*}$ is a direct sum of split exact complexes
plus the one-term complex
$$
0\to \Sym^{0}G\otimes\mwedge^{g}G^{*}\otimes \mwedge^{0}F'^{*} \to 0.
$$
Adjoining the isomorphism
$$
\partial_{1}: \wedge^{g}G^{*} \to
\Sym^{0}G\otimes\mwedge^{g}G^{*}\otimes \mwedge^{0}F'^{*}
$$
  we see at last that $EN(\phi)$ is split exact in this case, as required.

The argument above shows that the
complex $EN(\phi)$ becomes split exact after localizing at any prime $P$ not containing
the ideal $I$ of $g\times g$ minors of $\phi$. 

To establish the rank condition in Theorem~\ref{WMACE}, we first consider the
generic case when $I_{g}(\phi)$ contains a nonzerodivisor $x$. Inverting $x$
does not change the rank of any map, and after inverting $x$ the
complex $EN(\phi)$ becomes
split exact, so we see that 
$$
\rank \delta_{i}+\rank \delta_{i+1} \leq \rank (EN(\phi)_{i}).
$$

Now suppose that $I_{g}(\phi)$ has grade $\geq f-g+1$. Since this number is
at least 1, it follows from the preceding argument that the rank condition
of Theorem~\ref{WMACE} is satisfied. Moreover after localizing at any prime
not containing $I_{g}(\phi)$ the ideals $I(\delta_{i})$ become equal to the unit ideal.
By Lemma~\ref{free coker},
 $$
\tsty % tame the square root
I_{g}(\phi) \subset \sqrt{I(\delta_i)}.
$$
so that all the ideals have grade at least $\geq f-g+1$, verifying the grade condition
of Theorem~\ref{WMACE} and completing the proof of Theorem~\ref{ENgeneral}.
\end{proof}

\begin{corollary}\label{resolution of a scroll}
If $X\subset \PP^{r}$ is the rational normal scroll defined by $I_{2}(\phi)$, then the resolution of the homogeneous
coordinate ring of $X$ is $EN(\phi)$.
\end{corollary}

\begin{proof}
The map $\phi$ is represented by a 1-generic matrix of size $2\times (1+\codim X)$, and thus $I_{2}(\phi)$ has grade equal to $\codim X = (1+\codim X) - 2 +1$.
\end{proof}

\begin{corollary}\label{E-N cor}
With notation as in Theorem~\ref{ENgeneral} suppose that $R$ is a local ring
with residue field $k$ and that
and $k\otimes_{R}\phi = 0$.
If $I_{g}(\phi)$
has codimension $\geq f-g+1$ then it has
codimension exactly $f-g+1$, the ring $S/I_g(M)$ is Cohen--Macaulay,
and the $\tbinom{f}{g}$ forms
that are the $g\times g$ minors of a matrix for $\phi$ are linearly
independent over $\CC$.
\end{corollary}

\begin{proof}
From the resolution $EN(M)$ we see that the projective dimension of
$S/I_g(M)$ is at most $f-g+1$. Since the projective dimension of a module
is at least the codimension of its annihilator, the equality follows,
and the Auslander--Buchsbaum formula implies that $S/I_g(M)$ is
\index{Auslander--Buchsbaum formula}%
\index{Cohen--Macaulay}%
Cohen--Macaulay. The linear independence of the minors of $M$ follows
because $EN(M)$ is a resolution and the differential $\delta_{2}$, which 
exhibits the relations on the $g\times g$ minors, has image in the
maximal ideal time $EN(\phi)_{1}$.
\end{proof}

In general when $X\subset Y\subset \PP^r` `$, so that $I_X \supset I_Y$,
it may be hard to see which syzygies of $X$ come
from syzygies of $Y$. But when the degrees of the syzygies of $Y$ are
smaller than those from $X$, the situation is simpler.
Here is the special case we will use:

\begin{proposition}\label{containment of resolutions}
Let $C\subset \PP^r$ be
a nondegenerate curve. If $C\subset X
\subset \PP^r` `$, where $X$ is a rational
\index{rational normal scroll}%
normal scroll, then the Eagon--Northcott complex that is  the minimal
free resolution of $I_X$ is termwise a direct summand
of the minimal free resolution of $I_C$. Thus each number in the Betti table of the
\index{Betti table}%
resolution of $I_C$ is  
no less than
the corresponding number in the Betti table of the resolution of $I_X$.
\unif
\end{proposition}

\begin{proof}
Let $EN$ be the minimal resolution of $I_X$, and let $\FF$ be the minimal
resolution of $I_C$.
The inclusion $I_X \subset I_C$ induces a map $\psi: EN\to \FF$, unique
up to homotopy. Since the minimal generators of $I_X$ are quadratic,
and $I_C$ contains no linear forms, $\phi_0: EN_0\to \FF_0$ is a split
monomorphism.

By Corollary~\ref{resolution of a scroll} the minimal free resolution of the homogeneous
coordinate ring of $X$ is $EN(\phi)$ for a suitable $2\times (1+\codim X)$ matrix $\phi$.
By induction, we may assume that $\psi_{i-1}$ is a split monomorphism. The
free module $EN(\phi)_{i}$ is generated in
degree $i+1$, while the free module $\FF_i$ is generated in degrees
$\geq i+1$. It follows that the relations
represented by $EN_i$, extended by 0, are among the minimal generators
of the relations represented by $\FF_i$,
completing the proof.
\end{proof}

\section{Green's Conjecture}

To begin to describe Green's conjectured relationship of the geometry of
a curve and the Betti table of its homogeneous coordinate ring
in its canonical embedding, we begin at the beginning of the resolution, with 
the number of quadratic and cubic generators of the homogeneous ideal.

Corollary~\ref{canonical hilbert function} implies that the dimension
\index{Green's Conjecture}%
of the vector space of forms of degree $d$
vanishing on a canonical curve is independent of the curve; for example,
if $C$ is a curve of genus $g\geq 3$ that is not hyperelliptic,
$
\dim ({I_{C}})_{2} = \tbinom{g-2}{2}
$
and since there are no linear forms in the ideal this implies that in the 
Betti table of the homogeneous coordinate ring of every canonically embedded
curve of genus $g$ we have 
$\beta_{1,2} =  \tbinom{g-2}{2}$.

However, $\beta_{2,2}$, the number of cubic generators required, may vary, and reflects the geometry of the curve.  We have seen in Chapter~\ref{genus 4, 5 Chapter}
that if $C$ is a canonically embedded curve of genus 5 in $\PP^{4}$ that is not trigonal,
then $C$ is the complete intersection of three quadrics, and thus the 
Betti table of the homogeneous coordinate ring $R_{C}$ is
$$
\small
\begin{tabular}{r|@{\hskip20pt}ccccc}
\downstrut
$j$&\llap{$i={}$}0&1&2&3\\
\hline\upstrut
0&1&--&--&--\\
1&--&3&--&--\\
2&--&--&3&--\\
3&--&--&--&1\\
\end{tabular}
$$
On the other hand the ideal of a trigonal curve $C'$ of genus 5
requires two cubic generators in addition to three quadrics. 
 Since the Hilbert functions are the same in the two cases, and there
 were no linear relations on the quadrics in the nontrigonal case, there must be
2 linear relations on the quadrics in the trigonal case. Since the homogeneous
coordinate ring of a canonical curve is Gorenstein, the free resolution is symmetric,
and is therefore the Betti table for $C'$ is
$$
\small
\begin{tabular}{r|@{\hskip20pt}ccccc}
\downstrut
$j$&\llap{$i={}$}0&1&2&3\\
\hline\upstrut
0&1&--&--&--\\
1&--&3&2&--\\
2&--&2&3&--\\
3&--&--&--&1\\
\end{tabular}
$$

Using the knowledge of scrolls from Chapter~\ref{ScrollsChapter} we can refine this
observation. By the
\index{Riemann--Roch theorem!geometric}%
\blue{geometric Riemann--Roch theorem,} 
$C$ has
\index{trigonal}%
a 1-dimensional family of 
\blue{trisecant lines,}
\index{trisecant}%
and any quadric containing $C$
must contain all these. As we have seen in Chapter~\ref{ScrollsChapter},
these lines sweep
out the 2-dimensional rational normal scroll defined by the 1-generic
$2\times (g-2)$ matrix $M$ corresponding to the decomposition of
$\sO_C(1)$
into a tensor product of the  line bundle $\sL$ associated to the
$g^{1}_{3}$ and the residual line bundle $\omega_{C}\otimes \sL^{-1}$. The
latter has $g-2$ sections, and we see from section~\ref{particular name}
that the scroll itself lies on the $\tbinom{g-2}{2}$ quadrics defined
by the minors of $M$. The exactness of the Eagon--Northcott complex
associated to this matrix shows that there are no relations of degree
0 on these minors\emdash that is, they are linearly independent over
the ground field. It follows that they generate the vector space of all
quadrics containing $C$. We have seen that the minimal resolution
of the ideal of the scroll is, term by term, a summand of the minimal
resolution of $I_{C}$, and indeed we see the Betti table 
$$
\small
\begin{tabular}{r|@{\hskip20pt}ccccc}
\downstrut
$j$&\llap{$i={}$}0&1&2&3\\
\hline\upstrut
0&1&--&--&--\\
1&--&3&2&--\\
\end{tabular}
$$
of the scroll in the first two rows of the Betti table for $C$ above.

Another example of a canonical curve whose ideal requires cubic generators occurs in genus 6:
If $C$ is isomorphic to a 
\blue{plane quintic} 
\index{quintic curve!plane}%
curve,
then the canonical series of the plane quintic is $5-3 = 2$ times the
hyperplane series, and it follows that the canonical image of $C$ lies on
the Veronese surface in $\PP^{5}$. As explained in the general construction
at the beginning of section~\ref{particular name},  the Veronese surface is contained
in (in fact, equal to) the intersection of the quadrics defined by the
$2\times 2$ minors of a generic symmetric matrix, coming from the
multiplication map
$$
H^{0}(\sO_{\PP^{2}}(1))\otimes H^{0}(\sO_{\PP^{2}}(1)) \to
H^{0}(\sO_{\PP^{2}}(2)) = H^{0}(\sO_{\PP^{5}}(1))
$$
and there are $6 = \tbinom{g-2}{ 2}$ independent quadrics in this
ideal, so these are all the quadrics in $I_{C}$. Thus $I_{C}$ requires cubic generators.
One can show that if $C$ is not trigonal and not isomorphic to a plane quintic,
then the resolution of $R_{C}$ has Betti table
$$
\small
\begin{tabular}{r|@{\hskip20pt}ccccc}
\downstrut
$j$&\llap{$i={}$}0&1&2&3&4\\
\hline\upstrut
0&1&--&--&--&--\\
1&--&6&5&--&--\\
2&--&--&5&6&--\\
3&--&--&--&--&1\\
\end{tabular}
$$
whereas if $C$ is either trigonal or isomorphic to a plane quintic, the Betti table of
$R_{C}$ is
$$
\small
\begin{tabular}{r|@{\hskip20pt}ccccc}
\downstrut
$j$&\llap{$i={}$}0&1&2&3&4\\
\hline\upstrut
0&1&--&--&--&--\\
1&--&6&8&3&--\\
2&--&3&8&6&--\\
3&--&--&--&--&1\\
\end{tabular}
$$
and the first two rows are the Betti tables of either the rational normal scroll
swept out by the trisecant lines (in the trigonal case) or the Veronese surface
(in the plane quintic case). 

One might fear that these examples were the beginning of a long series of
types of curves whose canonical image is not cut out by quadrics, but
this is not the case:

\begin{theorem} [Petri]\label{Petri}
The ideal of a canonical curve of genus $\geq 5$ is generated by the
\index{Petri's theorem}%
$\tbinom{g-2}{ 2}$-dimensional space of quadrics it contains unless the curve
is either trigonal or isomorphic to a plane quintic; in the latter cases,
the ideal of the curve is generated by quadrics and cubics.
\unif
\end{theorem}
For a modern treatment of Petri's theorem in this level of generality see
\cite{Schreyer}; for a different treatment see \cite{Arbarello-Sernesi}.

The Betti tables for curves of genus 6
do not distinguish between the two exceptional types of 
curves of genus 6, and this is a clue to the general case.
The two exceptions in Petri's theorem are unified  by  the 
notion of the
\index{Clifford index:}%
Clifford index:

\begin{definition}
The \emph{Clifford index} Cliff $\sL$ of a line bundle $\sL$ on a curve
\index{Cliff@Cliff $\sL$}%
$C$ is $d-2r$, where $d \colonequals  \deg \sL$ and $r \colonequals
h^0(\sL)-1$. The Clifford index Cliff $C$ of
a curve $C$ of genus $\geq 2$ is the minimum of the Clifford indices
of special line bundles with at least 2 sections.
\unif
\end{definition}

Clifford's theorem (Corollaries \ref{Clifford bound} and ~\ref{equality
in Clifford from Martens}) shows that, for any curve $C$,
 Cliff $C \geq 0$, and Cliff $C =
\index{Clifford's theorem}%
0$ if and only~if $C$ is hyperelliptic. If $C$ is not hyperelliptic, then
 Cliff $C=1$ if and only~if $C$ is either trigonal or
isomorphic to a plane quintic, the two exceptional cases in Theorem~\ref{Petri}.
 The Clifford index of any smooth curve of
genus $g\geq 2$ is $\leq \lceil g/2\rceil+1$, with equality for a general
curve, as one sees from the Brill--Noether Theorem~\ref{basic BN}. The line bundle $\sL$ of maximal Clifford index often has only
2 sections, though there is an infinite sequence of examples where this
``Clifford dimension'' is greater beginning with plane quintics
and complete intersections of two cubics in $\PP^{3}$~\cite{MR1030141}.

%
%Moving to cubic forms, we see that $\dim ({I_C})_3 = 
%\tbinom{g+2}{ 3}-(5g-5)$. Comparing this number with the number of (possibly linearly
%dependent)
%cubics obtained by multiplying $g$ linear forms and $\tbinom{g-2}{ 2}$
%quadrics, we see that the ideal of the curve has at least
%$
%\tbinom{g-2}{ 2} - \tbinom{g+2}{ 3}-(5g-5)
%$
%independent syzygies of total degree 3 (that is, linear syzygies on the
%quadrics). For example when $g=4$ so that $C\subset \PP^3$ there is one
%quadric and 5 independent
%cubics, at most 4 of which are multiples of the quadric. Since the curve
%has degree $6 = 2\times 3$, the ideal of the curve must be generated by
%the quadric and one cubic. When $g=5$ there are genuinely two
%possibilities: the three quadrics in the ideal might be a complete
%intersection
%(then they generate the ideal), so the Betti table would be
%\index{Betti table}%
%$$
%\small
%\begin{tabular}{r|@{\hskip20pt}ccc}
%\downstrut
%$j$&\llap{$i={}$}0&1&2\\
%\hline
%\upstrut
%0&1&--&--\\
%1&--&2&--\\
%2&--&--&$1$\\
%\end{tabular}
%$$
%\emdash or the curve could be trigonal, in which case the 3 quadrics
%generate the ideal of a surface scroll $F$. In the latter
%case, the Eagon--Northcott complex resolves the 
%\index{homogeneous coordinate ring}%
%\blue{homogeneous coordinate ring}
%$S_F$ of the scroll,
%$$
%0\to S^2(-3) \to S^3(-2) \to S \to S_F \to 0
%,
%$$
%which has Betti table
%$$
%\small
%\begin{tabular}{r|@{\hskip20pt}ccc}
%\downstrut
%$j$&\llap{$i={}$}0&1&2\\
%\hline
%\upstrut
%0&1&--&--\\
%1&--&3&$2$\\
%\end{tabular}
%$$
%and we see that there are 2 linear relations among the
%quadrics. Thus the minimal generators of $I_C$ must include exactly
%2 cubics as well as the 3 quadrics. Since the homogeneous ring of a
%canonical curve is Gorenstein, its minimal free resolution is symmetric,
%and this is enough for us to fill in its Betti table:
%$$
%\small
%\begin{tabular}{r|@{\hskip20pt}cccc}
%\downstrut
%$j$&\llap{$i={}$}0&1&2&3\\
%\hline
%\upstrut
%0&1&--&--&--\\
%1&--&3&$2$&--\\
%2&--&$2$&$3$&--\\
%3&--&--&--&$1$\\
%\end{tabular}
%$$
%Note that we can see the scroll reflected in the top two
%lines of the table.
%
%From the analogue of the 
%\blue{Hilbert--Burch theorem}
%\index{Hilbert--Burch theorem}%
% for 
%\blue{Gorenstein rings} 
%\index{Gorenstein rings}%
%of
%codimension 3 one can show that the 5 generators can be written as the
%\blue{Pfaffians}
%\index{Pfaffian}%
%of a skew symmetric $5\times 5$ matrix whose entries are of
%degrees 1 and 2, in the following pattern (we give just the degrees,
%and 
%use -- for 0):
%$$
%\catcode`\-=\active \def-{\hbox{\char`\-\char`\-}}
%\left(@
%\begin{matrix}
%-&-&1&1&1\\
%-&-&1&1&1\\
%1&1&-&2&2\\
%1&1&2&-&2\\
%1&1&2&2&-
%\end{matrix}
%@\right)
%.
%$$
%Here the $2\times 2$ minors of the upper $2\times 3$ block of linear
%forms generate the ideal of the scroll.
%
%Applying this logic more generally we get the following result about
%the canonical embedding of curves with low degree maps to $\PP^1$:
%
Putting together the construction of section~\ref{particular name} with
Proposition~\ref{containment of resolutions} allows us to say
something about the meaning of the Betti table of a canonical curve.

\begin{theorem}
Let $C\subset \PP^{g-1}$ be a reduced, irreducible canonical curve. If
$C$ has a line bundle $\sL$ of degree $d \leq g-1$ with $h^0(\sL) =
2$  then
there is a 1-generic  $2\times (g+1-d)$ matrix of linear forms whose
minors define a scroll of codimension $g-d$ containing $C$; and thus an
Eagon--Northcott complex of length $g-d$ is a subcomplex of the minimal
free resolution of $R_C$. In particular, the Betti table of $R_C$ is
\index{Betti table}%
termwise $\geq$ that of the homogeneous coordinate ring of the scroll.
\unif
\end{theorem}

%We have stated this theorem for canonical curves, but in fact the
%construction applies much more generally to a linearly normal variety
%$X \subset \PP^n$ of any dimension: if $X$ has a divisor $D$ that moves
%in a pencil and is contained in a subspace $\PP^k$ with $k \leq n-2$,
%the planes spanned by the divisors of the pencil $|D|$ sweep out a
%rational normal scroll.

Thus the existence of the $g^1_d$ on $C$, together with the symmetry
of the resolution of the Gorenstein ring $R_C$,
implies that the Betti table of $R_C$ has the form%
\index{Betti table}%
$$
\small
\medmuskip1mu
\begin{tabular}{r|@{\hskip20pt}cccccccccccc}
\downstrut
$j$&\llap{$i={}$}0&1&2&\dots&$d-3$&$d-2$&\dots&$g-d-1$&$g-d$&\dots&$g-3$&$g-2$\\
\hline
\upstrut
0&1&--&--&$\cdots$&--&--&--&--&--&--&--&--\\
1&--&*&*&$\cdots$&*&*&$\cdots$&*&$?$&$\cdots$&?&?\\
2&--&?&?&$\cdots$&?&*&$\cdots$&*&*&$\cdots$&*&*\\
3&--&--&--&$\cdots$&--&--&--&--&--&--&--&1
\end{tabular}
$$
where we have assumed for illustration that $d-2<g-d-1$. 
As before, a dash indicates a place that is
definitely 0; 
asterisks indicate some that are
definitely
nonzero. The entries of the rows marked 0 and 1 are 
greater than or equal to
the
corresponding entries of the Betti table of the scroll.

We can summarize this by saying
that if the curve $C$ has a line bundle $\sL$ of degree $d$ with exactly
2 sections (which is thus of Clifford index $c = d-2$) the row labeled  2
in the Betti diagram definitely has $\beta_{c, c+2} \neq 0$. As with
the case of the plane quintics above, one can
make a similar argument for \emph{any} line bundle of Clifford index
$c$. Thus:

\begin{corollary}
If\, $\mathrm{Cliff}\, C \leq c$ then $\beta_{c,c+2}(S/I_C)) \neq 0$.
\unif
\end{corollary}

Starting from examples such as the case of genus 6, Mark Green made a
\index{Green, Mark}%
bold conjecture that is still open as of this writing:

\begin{conjecture}[Green's conjecture]
If $C$ is a smooth canonical curve of genus $g$ and $R_{C}$ is the
\index{Green's conjecture}%
homogeneous coordinate ring of $C$ in its canonical embedding,
then the Clifford index of $C$ is $\leq c$ if and only~if
$\beta_{c,c+2}(R_{C}) \neq 0$.
\unif
\end{conjecture}

The conjecture was made for curves over a field of characteristic 0, and
is known in many cases, though it is also known to fail in small finite
characteristics (see \cite{Bopp-Schreyer} for an amended conjecture that
may hold in all characteristics.)
All the possibilities of Betti tables for
canonical curves of genus up to 8 (in characteristic 0) were
analyzed
in \cite{Schreyer-canonical}),
and the project was carried further to
genus 9 in \cite{Sagraloff}; in particular,
these computations
establish the conjecture for these values of $g$.
Claire Voisin
\citeyear{MR1941089,MR2157134} 
\index{Voisin, Claire}%
proved
the conjecture 
for an open set of curves of each Clifford index; 
simpler proofs then appeared in
\cite{MR4022070,MR4213770,arXiv:2205.00266}.

See the excellent survey~\cite{Farkas-progress-on-syzygies}
for more information.


\section{Exercises}

\begin{exercise}\label{WMACE corollary}
Let $S = k[x_0,\dots, x_r]$, and let
$$
\FF:\quad  
F_0\luto{\phi_1}F_1 \leftarrow \cdots \leftarrow F_{n-1}\luto{\phi_n} F_n\leftarrow 0
$$
be a finite complex of free $S$-modules. Set
$$
X_i =\big\{p\in \AA^{n+1} \mid  H_i(\FF \otimes \kappa(p)) \neq 0 \big\}.
$$
Use Theorem~\ref{WMACE} to prove that $\FF$ is 
\blue{acyclic}
\index{acyclic}%
if and only~if
$
\codim X_i \geq i
$
for all $i>0$. Moreover, $X_{0}\supseteq X_{1}\supseteq \cdots \supseteq
X_{n}$.

Hint: Elementary linear algebra shows that, if $k$ is a field, then a
complex $$k^p \lTo^{\phi} k^q \lTo^\psi k^r$$ is exact at $k^q$ if and
only~if $\rank \phi +\rank \psi = q$.
\end{exercise}

\begin{exercise}
Prove that if $X\subset \PP^r$ is 
\blue{arithmetically Cohen--Macaulay}
\index{ACM}%
then
the dual of the minimal free resolution of $S/I_X$
is the minimal free resolution of $\omega_X$.

Hint: Use Theorem~\ref{WMACE}
and the characterization of $\omega_{S/I_X}$
as an Ext module.
\end{exercise}

\begin{exercise}
The \emph{depth lemma} states that if
\index{depth lemma}%
$$
0\to A\to B\to C \to 0
$$
is an exact sequence of nonzero finitely generated modules over a local
ring $R$ then
$$
\begin{aligned}
\depth C &\geq \min\{\depth B, \depth A-1\},\\
\depth A & \geq \min\{\depth B, \depth C +1\}.
\end{aligned}
$$
Prove this in the special case when $R$ is regular using the
characterization of depth
via projective dimension.
\end{exercise}

\begin{exercise}
\begin{enumerate}
\item Prove that if $\phi: F\to G$ is a free presentation of a finitely
generated module $M$
then
$$
\ann_{R}(\coker \phi)^{\rank G}\subset \Fitt_{0}\phi \subset
\ann_{R}(\coker \phi).
$$
\item Prove that over a local ring every projective module is free;
and show that
$I(\phi)$ defines the nonfree locus of $\coker \phi$.
\end{enumerate}
\end{exercise}

\begin{exercise}
In dealing with arithmetically Gorenstein schemes, we used the fact that
\blue{arithmetically Gorenstein}
\index{arithmetically Gorenstein}%
if $\omega_{S/I}$ is an invertible
sheaf (over $\Spec(S/I)$) then it is isomorphic to $S/I(a)$ for some
$a$. Why is this true?

Hint: Nakayama's lemma can be used to prove that projectives are free
\index{Nakayama's lemma}%
in some cases.
\end{exercise}

\begin{exercise}
Find a degree 6 embedding of a curve of genus 3 that is not arithmetically
Cohen--Macaulay, and another that is.

Hint: Show that the $3\times 3$ minors of a general $4\times 3$ matrix
of linear forms defines a Cohen--Macaulay curve
of genus 3. Show that a curve of type $(2,4)$ on a smooth quadric in
$\PP^3$ is not arithmetically Cohen--Macaulay.
\end{exercise}

\begin{exercise}
Referring to the Betti table of a canonical curve just before
\index{Betti table}%
Section~\ref{EN section}, give a formula
for the differences $b_i- b_{g-i-1}$ that depends only on $i$ and $g$.
\end{exercise}

\begin{exercise}
Show that if $I \subset S \colonequals  \CC[x_0,\dots, x_r]$ is a
codimension 2 ideal, then $S/I$ is Cohen--Macaulay if and only
if the minimal $S$-free resolution of $S/I$ has the form
$$
0\to S^{n-1} \to S^n \to S
$$
for some $n$. Show that $S/I$ is Gorenstein if and only~if $I$ is a
complete intersection.

Hint: for the first part, tensor with
the field of rational functions.
\end{exercise}

\begin{exercise}
Which sets of 4 distinct points in $\PP^2$ are arithmetically
Gorenstein? Which sets of 5 points? Which sets of 6 points?
\end{exercise}

\begin{exercise}
Give an example of a set of points in $\PP^3$ that is arithmetically
Gorenstein but not a complete intersection.

Hint: take the
hyperplane section of a trigonal canonical curve of genus 5.
\end{exercise}

\begin{exercise}
Let $q:F\to G$, with $F = S^2(-1)$ and $G = S^2$,
be described by the matrix $\bigl(` `\begin{smallmatrix} x_0&x_1\\ x_2&x_3 \end{smallmatrix}` `\bigr)@$, and let $Q\subset \PP^3$ be the quadric defined by the determinant of $q$.
\begin{enumerate}

\item Show that the sheafification of the graded module $M \colonequals
\coker q$ is $\sO_Q(1,0)$ and the sheafification
of $\coker q^*$ is $\sO_Q(0,1)$ by computing the vanishing locus
of the two sections corresponding to the generators of the module.

\item Show that the sheafification of $\Sym^a(M)$ is
$\sO_Q(a,0)$. Conclude that
if $a\leq \nobreak b$ then the relative ideal sheaf $\sI_{C/Q} = \sO(-a,-b)$
of a curve $C$ of type $(a,b)$
is the sheafification of the module $\Sym^{b-a}(M)(-b)$.

\item Let $S = \CC[x_0,\dots, x_3]$. Show that the minimal $S$-free
resolution of $\Sym^a(M)$
has the form
$$
0 \to \mwedge^2 F \otimes \Sym^{a-2} G(-2)\to F\otimes \Sym^{a-1}
G(-1)\to \Sym^a G.
$$

Hint: Use multilinear algebra (as in \cite{Eisenbud1995}) to define the
maps, and use
Theorem~\ref{WMACE} to prove that this is a resolution.

\item Show that if $C\subset \PP^3$ is a curve of type $(a, b)$ with
$a\leq b$ then
there is a free resolution of a module
whose sheafification is $\sI_{C/\PP^3}$ that is a mapping cone of the
map of complexes
\vspace*{-10pt}
$$
\small
\quad
\medmuskip0mu
\xymatrix@C=15pt{
0\ar[r]  &0\ar[r]& \mwedge^2 F \ar[r]^ {\ssty\bigwedge^{`2}``q}&\mwedge^2 G\\
0\ar[r]\ar[u]  & \mwedge^2 F @\otimes@ \Sym^{b-a-2} G(-b-2) \ar[r]\ar[u]  
&F @\otimes@\Sym^{b-a-1} G(-b-1) \ar[r]\ar[u]& \Sym^{b-a} G(-b)\ar[u]
}
\vspace*{5pt}
$$

\item Conclude that the deficiency module of a curve of type $(a, b)$
with $a\leq b$ is the cokernel of a map
$$
\mwedge^2 F^* \otimes \Sym^{b-a-2} (G^*)(b+2) \luto{\,F^*}\otimes \Sym^{b-a-1}
G^*(b+1)
$$
and thus, after choosing a basis of $F = S^2` `$, may be identified as
the sheafification of the cokernel of
$$
\Sym^{b-a-2} (G^*)(b+2) \luto {\ F}\otimes \Sym^{b-a-1} G^*(b+1)
$$
where the map is the action of $F$ on $\mwedge \,G^*$ via the map $q:F\to G$.
\end{enumerate}

Hint:  Imitate the proof of Theorem~\ref{ENgeneral} to prove
the exactness of the given resolution of $\Sym^a(M)$. See also
\cite[Appendix A2.6]{Eisenbud1995}.
\end{exercise}

\begin{exercise}
\label{Fitt}
Let $\phi: F\to G \to M\to 0$ be an exact sequence of finitely generated
$R$-modules,
with $F$ and $G$ free.

\begin{enumerate}
\item Show that the annihilator of  $\coker \phi$ has the
same radical as the ideal 
of minors
$I_{\rank G}(\phi)$.
\item 
Show that the cokernel of $\phi$
is locally free if and only~if $I_{\rank \phi}(\phi)= R$.
\end{enumerate}
\end{exercise}

%footer for separate chapter files

\ifx\whole\undefined
%\makeatletter\def\@biblabel#1{#1]}\makeatother
\makeatletter \def\@biblabel#1{\ignorespaces} \makeatother
\bibliographystyle{msribib}
\bibliography{slag}

%%%% EXPLANATIONS:

% f and n
% some authors have all works collected at the end

\begingroup
%\catcode`\^\active
%if ^ is followed by 
% 1:  print f, gobble the following ^ and the next character
% 0:  print n, gobble the following ^
% any other letter: normal subscript
%\makeatletter
%\def^#1{\ifx1#1f\expandafter\@gobbletwo\else
%        \ifx0#1n\expandafter\expandafter\expandafter\@gobble
%        \else\sp{#1}\fi\fi}
%\makeatother
\let\moreadhoc\relax
\def\indexintro{%An author's cited works appear at the end of the
%author's entry; for conventions
%see the List of Citations on page~\pageref{loc}.  
%\smallbreak\noindent
%The letter `f' after a page number indicates a figure, `n' a footnote.
}
\printindex[gen]
\endgroup % end of \catcode
%requires makeindex
\end{document}
\else
\fi

