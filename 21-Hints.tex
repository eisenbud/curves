
\chapter*{Hints to marked exercises}
\label{hints}

\begingroup
\def\addcontentsline#1#2#3{\relax} % we don't "Exercises..." in the .toc

\subsection*{Exercises in Chapter 1\nopunct}

\hinthead{1.1}
Use B\'ezout's theorem.

\hinthead{1.4}
For the first part, use B\'ezout's theorem.

\hinthead{1.7}
For parts (3) and (4), observe that if $\beta \neq \pm \alpha$ then
$\Gamma_{\alpha, \beta}$  is locally the intersection of $C$ with the
line $L \subset \AA^2$ given by $\alpha x + \beta y = 0$, but when
$\beta = \pm \alpha$
the intersection $L \cap C$ has multiplicity 3 at $p$,
and is not equal to $\Gamma_{\alpha, \beta}$.

\subsection*{Exercises in Chapter 2\nopunct}

\hinthead{2.1}
For (1) and (3), use the
\index{Serre--Grothendieck vanishing theorem}%
\index{vanishing!theorem (Serre--Grothendieck)}%
\index{Riemann--Roch formula}%
Serre--Grothendieck vanishing theorem, plus the
Riemann--Roch formula in the case of (3).
For (2), compare the Hilbert polynomial of $X$ with that of a hyperplane section of $X$.

\hinthead{2.2}
For
the degree formula, note that $\smash{\sum_{p\in C}}
\mult_p(C,H) p$ is the divisor associated to the morphism
$\widetilde C \to \PP^r$ with image $C$.

\hinthead{2.3}
For (3), let $r$ be the third point of intersection of the
  line $\overkern20{pq}$ with the closure of $C^\circ$, and consider the
  pencil of lines through $r$.)

\hinthead{2.6}
For (1), find rational functions in $x$ and $y$ whose pullback to $C$
has poles along $D = p + q + r_0 + r_3$ but nowhere else. For the last
part, observe that the equation of the image corresponds to the kernel
of the map $\Sym^4 H^0(\cO_C(D)) \to H^0(\cO_C(4D))$.
\index{Sym@$\Sym$, $\Sym^i$}%

\hinthead{2.11}
First show that there are at most two points in the normalization of
$C$ lying over $p$.

\hinthead{2.13}
The linear forms in $\PP^{\sbinom{m+2}{2} - 1}$ vanishing on the image
of $C_0$ correspond to elements of the kernel of the map
$H^0(\cO_{\PP^2}(m)) \to H^0(\cO_{C}(m))$.

\subsection*{Exercises in Chapter 3\nopunct}

\hinthead{3.1}
Use Proposition~\ref{very ample}.

\hinthead{3.4}
Find the rank of the restriction map
$H^0(\cO_{\PP^N}(1)) \to H^0(\cO_{\nu_d(C)}(1))$.

\hinthead{3.8}
Use B\'ezout's theorem.
\index{Bezout@B\'ezout's theorem}%

\hinthead{3.9}
Look at the linear series cut on $C$ by the two rulings of the quadric.

\hinthead{3.10}
To any such embedding we can associate the $g^1_3$ given by one of the
rulings of the quadric surface containing the image curve. Modulo
$\PGL_2$, there is a one-parameter family of $g^1_3$s on $\PP^1` `$.

\hinthead{3.11} Use induction on $n$, starting with the case $n=2$.

\hinthead{3.13} For any point $p \in C$, let $L_p \subset \cN_{C/\PP^3}$ be the
line bundle
of $\cN_{C/\PP^3}$ whose fiber over any point $q \neq p \in C$ is the
one-dimensional subspace of $(\cN_{C/\PP^3})_q$ spanned by the line
$\overkern20{pq}$. (This of course only defines a line subbundle of
$\cN_{C/\PP^3}$ over $C \setminus \{p\}$, but there is a unique
extension to a line subbundle of $\cN_{C/\PP^3}$ over all of $C$.)
Show that for $p \neq p'$ we have
$$
\cN_{C/\PP^3} = L_p \oplus L_{p'}.
$$

\hinthead{3.14}
For $p \in C$, define a line subbundle $L_p \subset \cN_{C/\PP^d}$ as
in the preceding problem, and show that for $p_1,\dots,p_{d-1}$
distinct points the bundles $L_{p_i}$ are independent.

\hinthead{3.15}
It is the projectivization of the action of $\SL(V)$ on $\Sym^{d-2}V` `$.
\index{Sym@$\Sym$, $\Sym^i$}%

\hinthead{3.16}
Since the $r-1$ surfaces intersect only in codimension $r-1$, they form
a regular sequence; and the length of any maximal regular sequence in
$\CC[x_0,\dots, x_r]$ is $r+1$.

\hinthead{3.17} Let $p\in L\cap X$ be a point. If every secant to $X$ through
$p$ lies entirely in $X$, then $X$ is a cone over $p$; but since $p$
was a general point, this would imply that $X$ is a linear space,
contradicting nondegeneracy.
Hence
the projection $\pi_{p}:X \to \PP^{N-1}$ is a generically
finite (rational) map from $X$ to $X' \colonequals  \pi_{p}(X)$,
and thus $\dim X' = \dim X$ and $\codim X' = \codim X-1$. The plane
$\pi_{p}(L)$ meets $X'$ in the images of the points of $L\cap X$ other
than $p$, so
$\deg X\geq \deg X'+1$. By induction, $\deg X' \geq \codim X'+1 = \codim
X$, completing the argument.

\subsection*{Exercises in Chapter 4\nopunct}

\hinthead{4.1}
Show that any $d-2$ points impose independent conditions on
forms of degree $d-3$, and use
Bertini's theorem
\index{Bertini's theorem}%
to reduce to this case.

\hinthead{4.2}
First, the space of sextics double at three noncollinear points
visibly has dimension 19 (take the points to be the coordinate points
and count monomials). Then, since this includes the triangle with
vertices at the three points plus arbitrary cubics, the subspace of
those double at the fourth point will have codimension 3.

\subsection*{Exercises in Chapter 5\nopunct}

\hinthead{5.1} Use the fact that the intersection of two affine open
subsets of a quasiprojective scheme is again affine.

\hinthead{5.2}
For (2), change coordinates to
$z_i \colonequals x_i+y_i$, $@w_i \colonequals x_i - y_i$.

\hinthead{5.5}
If $\cL$ is an invertible sheaf of degree $d$ with $h^0(\cL) > d-g+1$,
show that for general $p, q \in C$ we have $h^0(\cL(p-q))  = h^0(\cL) - 1$.

\hinthead{5.6}
For (2), use the fact that $\mu_2$ is the blowup
\index{blowup}%
of $J(C)$ at a point.

\hinthead{5.7}
Show that for general points $p, q \in C$, there exists a unique pair
$(p',q') \neq (p,q)$ with $p - q \sim p'-q'$.

\hinthead{5.9}
A general divisor of degree $\geq g+1$ is nonspecial, so
$$h^0(\sO_C(D)) = (g+1)-g+1 = 2.$$
To say that the linear series $|D|$ has no basepoints means that $D$ is not equivalent to a divisor of the form
$D'+p$, where $h^0(D')$ has degree $g$ and two independent sections.
By the
Riemann--Roch theorem,
\index{Riemann--Roch theorem}%
$h^0(D') = 1+h^0(K-D')$,
so the condition on $D'$ means that $K-D'$ is effective.
But $K-D'$ is a general divisor of degree $g-2$, and
the set of classes of effective divisors, the image of $C_{g-2}$, has dimension only $g-2$.
Thus the set of divisors of the form $D'+p$ with $h^0(D') \geq 2$ has dimension only $g -2 +1$
so a general divisor $D$ of degree $g+1$ is basepoint free and defines a map $C\to \PP^1` `$.

By
Hurwitz's theorem
\index{Hurwitz's theorem}%
the total ramification of such a map is $2g+2$. The divisors with points of ramification
index 2 or more are linearly equivalent to divisors of the form $D'+3p$, and the divisors that have two ramification
points mapping to the same point are equivalent to divisors of the form $D''+2p+2q$. Each of these sets
of divisors fills only a
$(g{-}1)$-dimensional family of equivalence classes, the images of
$C_{g-2}\times C$ or $C_{g-3}\times C_2$ respectively; so not every divisor $D$ is of this form.

\subsection*{Exercises in Chapter 6\nopunct}

\hinthead{6.1}
 In either case the complement $\overkern22{C^\circ}
  \setminus C^\circ$ consists of a single point, with two points of $C$
  mapping to it; now use the genus formula in either $\PP^2$ or $\PP^1
  \times \PP^1` `$.

\hinthead{6.2}
Such a cover is specified by giving $2g+2$ transpositions, not
all equal, whose product is a nontrivial 3-cycle, modulo simultaneous
conjugation. We have already worked out the number of such tuples whose
product is the identity; just subtract.

\hinthead{6.3} Topologically, such covers are in 1-1 correspondence with
subgroups of index 2 in $\pi_1(C)$; and such a subgroup is necessarily
the preimage of a subgroup of index 2 in the abelianization $H_1(C, \ZZ)
\cong \ZZ^{2g}$.

\hinthead{6.4}
If $f : X \to C$ is an unramified double cover, consider the direct
image $f_*(\cO_X)$. This is a locally free sheaf of rank 2 on $C$,
on which the group $\ZZ/2$ acts; the $+1$-eigenspace is the structure
sheaf $\cO_C$, and the $-1$-eigenspace is an invertible sheaf $\sL$
on $C$ such that $\sL^2 \cong \sO_C$.

\hinthead{6.5}
By our analysis, to specify such a cover, we have to specify the
monodromy around representative loops generating $H_1(E, \ZZ) \cong\ZZ^2$;
thus there are four possibilities.

\hinthead{6.6}
Choose any line $M \subset Q$ of the opposite ruling, and look at the
linear forms $H, H'$ on $\PP^3$ vanishing on $L \cup M$ and $L' \cup M$.

\hinthead{6.7} If $h^0(\cL) = 0$,
we have $h^0(\cL(p_k)) = 1$ for any ramification point $p_k$;
show that  the unique effective divisor in
 $|\cL(p_k)|$ must be the sum of two ramification points.

\hinthead{6.8}
For (1) \emdash which implies that the map $\phi_\sL$ is an
immersion\emdash observe that $h^0(\sL\otimes K_C^{-1}) = 1$, meaning
$p$ and $q$ are unique. Part (2) says that the images of the
differential $d\phi_\sL$ at $p$ and $q$ are distinct.

\subsection*{Exercises in Chapter 7\nopunct}

\hinthead{7.1}
The Hilbert polynomials satisfy $p_X = p_Y + p_Z$, which follows
from the vanishing of $h^1(\cI_{Y\cup Z} (m))$ for large $m$; the
Hilbert functions satisfy $h_X \leq h_Y + h_Z$. (When $h_X(m) = h_Y(m)
+ h_Z(m)$, we say that $Y$ and $Z$
\emph{impose independent conditions}
\index{independent conditions}%
on $|\cO_{\PP^n}(m)|$.)

\hinthead{7.2}
Use the exact sequence
$
0 \to \cI_{Y\cup Z} \to \cI_{Y} \oplus \cI_{Z} \to \cI_{Y \cap Z} \to 0
$.

\hinthead{7.3}
Any cubic vanishing on $X$ vanishes identically on $H$.

\hinthead{7.4}
As in the twisted cubic case, the group $\PGL_{r+1}$
\index{PGL@$\PGL_{r+1}$}%
acts transitively on $\cH^\circ$ with stabilizer $\PGL_2$.

\hinthead{7.5}
The normal bundle is $\sN = \sO_C(d)+\sO_C(e)$. To prove smoothness, use
Exercise~\ref{ci is acm} to compute $H^0(\sN)$.

\subsection*{Exercises in Chapter 8\nopunct}

\hinthead{8.4} Show that the canonical sheaf $K_\cX$ is nontrivial.

\subsection*{Exercises in Chapter 9\nopunct}

\hinthead{9.1}
Recall that the $g^1_3$s on $C$ are cut by the rulings of the quadric $Q$.

\hinthead{9.2}
Try blowing up
\index{blowup}%
the plane at the nodes. Look at
Section~\ref{canonical series on nodal plane curves} if you get stuck.

\hinthead{9.3}
For the first part: if the curve $C \subset \PP^3$ lay on a quadric,
what would be its class? For the second, if
$S \cap T = C \cup D$, calculate the degree and genus of $D$.
See Chapter~\ref{LinkageChapter} if you get stuck.

\hinthead{9.4} 
Show that each of the divisors $E$ of the $g^1_2$ span a line in $\PP^3` `$.

\hinthead{9.5} 
Consider the incidence correspondence
$$
\Gamma \colonequals  \{ (Q, L) \in \PP^9 \times \GG(1,3) \mid L \subset Q \}
$$
where $\PP^9$ is the space of quadrics $Q\subset \PP^3` `$.

\hinthead{9.6} 
By projection, show that $C$ would have to be double at the vertex of the cone.

\hinthead{9.7} 
Show that the linear series $|K_C - p|$ is very ample if and only~if
$C$ is not trigonal.

\hinthead{9.8} 
Consider separately the cases where $\Gamma$ contains a fat point
\index{fat point|defi}%
(that is, the scheme defined by the square of the maximal ideal at a
point) or is
curvilinear
\index{curvilinear|defi}%
(that is, has Zariski tangent space of
dimension at most 1 everywhere).

\subsection*{Exercises in Chapter 10\nopunct}

\hinthead{10.1} 
Show that the items of Theorem\ref{needed for nodes}  are true in this
situation.

\hinthead{10.2} 
For (1), show that $h^0(D - K_C) = 0$, and conclude
that the map $\phi_D \times \phi_E$ is birational onto its image;
using
the genus formula, conclude the image curve is smooth. For (2),
observe that if $L$ is a line through $p$ meeting $C$ more than
once elsewhere, the line $L$ must be contained in $Q$.

\hinthead{10.5} 
If $d > mr$, then the hypersurfaces of degree $m$ containing
$\Gamma$ are exactly the hypersurfaces of degree $m$ containing $D$,
and we know how many there are. If $d \leq mr$, then the hypersurfaces
of degree $m$ containing $\Gamma$ cut out on $D$ the linear series
$|\cO_D(m)(-\Gamma))| = |\cO_{\PP^1}(mr-d)|$ and again we know the
dimension.

\hinthead{10.6} 
Show that if $h^{0}(\sO_{C}(m+1))-h^{0}(\sO_{C}(m)) \leq \deg C$,
and that if equality holds for
all $m\geq m_{0}$, then $\sO_{C}(m) $ is nonspecial for all $m\geq m_{0}$.

\hinthead{10.7} 
If $|D'|$ were another $g^r_d$ on $C$, show that $h^0(mD + D') > h^0((m+1)D)$.

\hinthead{10.8} 
For (1), we can calculate the dimension of the Hilbert scheme $\cH^\circ$ of
Castelnuovo curves
\index{Castelnuovo curve}%
by calculating the dimension
of the incidence correspondence of pairs $(Q, C)$ with $Q$ a quadric
surface in $\PP^3$ and $C \subset Q$ a curve of type $(k,k)$; then invoke
Exercise~\ref{castelnuovo unique} to say the fibers of $\cH^\circ \to
M_g$ are isomorphic to
\index{PGL@$\PGL_4$}%
\index{PGL@$\PGL_2$}%
$\PGL_4$.
For (2), we already know the
dimension of the locus of hyperelliptic curves is $2g-1$ ($2g+2$-tuples
of points in $\PP^1$ mod $\PGL_2$); the point is just that Castelnuovo
curves are rarer than hyperelliptic ones.

\subsection*{Exercises in Chapter 11\nopunct}

\hinthead{11.1} 
Say $k=2$. If the general hyperplane section of $X$ were reducible,
there would be distinguished subsets of the intersection of $X$ with
a general $(r-2)$-plane $\Lambda$ (the intersections of $\Lambda$ with
the components of $H \cap X$, for $H$ a general hyperplane containing
$\Lambda$); but we know the monodromy on $X \cap \Lambda$ is the full
symmetric group.

\hinthead{11.2} 
Choose a basepoint of the pencil, say $p = [1, i, 0, 0]$. The
lines of the two rulings of $Q_t$ passing through $p$ are
$Y-iX = Z-\pm i@\sqrt{t}@W = 0$, which are exchanged under the monodromy
as $t$ goes around 0.

\hinthead{11.4} 
We need to know that the dual hypersurfaces 
$C_i^* \subset (\PP^r)^*$ 
are all distinct; given this, we can exhibit loops that induce
a given permutation of the points of $H \cap C_i$ while fixing the points
of $H \cap C_j$ for $j \neq i$. To see that the dual hypersurfaces $C_i^*
\subset (\PP^r)^*$ are all distinct, invoke the duality theorem $(C_i^*)^*
= C_i$ (see for example \cite{3264}).

\hinthead{11.5} 
For (1), we can fix the curve $E$ and prove the
a priori stronger statement that the monodromy on the points of
$D \cap E$ as $D$ varies is the symmetric group; this follows from
Theorem~\ref{uniform position lemma} applied to the Veronese embedding
\index{Veronese!map}%
$\nu_d(E)$. Alternatively, we can exhibit a transposition by finding a
pair $(D,E)$ such that $D \cap E$ consists of $de-2$ simple points and
one double point, and prove double transitivity with the usual incidence
correspondence argument.

\hinthead{11.6} 
The line joining any two flexes of a plane cubic $E$ contains a third.

\hinthead{11.7} 
Show that a transitive subgroup of $S_n$ generated by transpositions
\index{transposition}%
is all of $S_n$.

\subsection*{Exercises in Chapter 12\nopunct}

\hinthead{12.2} 
For (2), we know that $\phi_F(E)$ lies on at least 5
quadrics by the usual restriction sequence; if it lay on 6 or more it
would be a rational normal curve. 
(Alternatively, see Section~\ref{g=1 in P4}.)

\hinthead{12.3} 
Use the description of the canonical model of $C$ 
together with the geometric
Riemann--Roch theorem.
\index{Riemann--Roch theorem!geometric}%
\index{canonical model}%

\hinthead{12.5} 
Take the image of $C$ under the map associated to the $g^2_6$
cut out by conics through two of the three collinear nodes and the one
remaining node.

\hinthead{12.7} 
The curve in question is the normalization of a plane sextic curve
with one double point, consisting of two smooth branches with contact of
order 4 with each other and contact of order 3 with their common tangent
line. The exercise asks you to both prove that such a curve exists,
and that the $g^1_4$ cut out by lines through the double point is the
unique $g^1_4$ on $C$.

\subsection*{Exercises in Chapter 13\nopunct}

\hinthead{13.2} 
Since we know that $f$ has finite order, we can take the quotient
$B = C/\langle f \rangle$ of $C$ by the cyclic group $\langle f \rangle$;
apply
the Riemann--Hurwitz formula
\index{Riemann--Hurwitz formula}%
to the quotient map $C \to B$.

\hinthead{13.3}  
Use the Hodge index theorem~\ref{hodge index} and the adjunction
\index{Hodge index theorem}%
\index{adjunction formula}%
formula to show that the maximum possible genus of such
a curve $C$ is obtained when $C$ is linearly equivalent to a sum of
fibers of the two projections, in which case the inequality
becomes an equality.

\hinthead{13.4} Since we know that $\Aut C$ has finite order, we can take
the quotient $B = C/\Aut C$ of $C$; again, apply
the Riemann--Hurwitz formula
\index{Riemann--Hurwitz formula}%
\index{automorphism!group, size of}%
to the quotient map $C \to B$. (Warning: the idea is the same as in
Exercise~\ref{2g+2fixedpoints}, but the execution is substantially
more complicated.)

\hinthead{13.5} 
For $\Lambda \in \Sigma_{\bbeta}({\cal V})$, consider bases of
$\Lambda$ such that $\Lambda \cap V_i$ is the span of basis vectors.

\hinthead{13.8} 
Let each pair of points come together, and use the result
for the tangent lines, a special case of Theorem~\ref{osculating
intersection}.

\hinthead{13.9}
It's enough to look at the case $d=2$ and $r=1$.

\hinthead{13.11} 
The inflection points are the points $p \in E$ such
that $\cO_E(np) \cong \cO_E(1)$.

\subsection*{Exercises in Chapter 14\nopunct}

\hinthead{14.1} Using the canonical embedding to realize $C$ as a
plane quartic
\index{canonical embedding}%
and project from the point of intersection of the tangent lines at $p$ and $q$.

\hinthead{14.2} Show, using dimension counts:
\begin{enumerate}
\item $\Lambda$ does not contain any tangent line to $C$.
\item $\Lambda$ does not meet any secant line to $C$ other than the
lines  $\overline{p_{i} q_{i}}$.
\item $\Lambda$ does not meet the 3-plane $\overline{\TT_{p_i}C,
\TT_{q_i}C}$ in a line.
\end{enumerate}

\hinthead{14.3} 
Let the $p_{i}, q_{i}$ approach each other, reducing to the case
of tangent lines. Then use
Theorem~\ref{osculating intersection}. 
For a direct proof see \cite[Lemma, p.~259]{Griffiths-Harris-BN}.

\hinthead{14.4}
Let $C_{0}$ be a curve with a node $p$, and $C \ruto {\,\nu} C_{0}$
\index{partial normalization}%
its
partial normalization
at $p$. Denote by $q,r \in \widetilde C$
the points lying over $p$. If $\cL$ is an invertible sheaf on $C$, and
$\cM \colonequals  \nu^*(\cL)$ the pullback of $\cL$ to $\widetilde C$,
then $\cM$ is an invertible sheaf on $\widetilde C$. Its fibers over $q$
and $r$ are both identified with the fiber $\cL_p$ of $\cL$ at $p$, and
hence with each other. Conversely, given an invertible sheaf $\cM$ on
$\widetilde C$ and an identification of the fibers $\cM_q$ and $\cM_r$,
we can form an invertible sheaf $\cL$ on $C$ by taking the subsheaf
of $\nu_*\cM$ whose sections agree at $q$ and $r$, in terms of the
identification.

\subsection*{Exercises in Chapter 15\nopunct}

\hinthead{15.1} 
If $D = q_1 + \dots + q_{d-2}$ had $r(D) \geq 1$, the points 
$q_1 + \dots + q_{d-2}$ and $p$ would fail to impose
independent conditions on
\index{independent conditions}%
plane curves of degree $d-3$ and hence lie on a line.

\hinthead{15.2} 
If $D = q_1 + \dots + q_{d-2}$ had $r(D) \geq 1$, the points 
$q_1 +\dots + q_{d-2}$ and $p, p'$ would fail to impose independent conditions
on plane curves of degree $d-3$ and hence by Proposition~\ref{independent
conditions} below $d-1$ of them would lie on a line.

\hinthead{15.3} 
A $g^1_{d-2}$ is a set of points that, together with the nodes,
impose dependent conditions on forms of degree $d-3$.

\hinthead{15.4} 
Be careful to subtract the right multiples of the points that are
preimages of the singular points.

\hinthead{15.5} To add two points $s$ and $t \in C$, choose a conic curve $D$
passing though $s, t, q_1$ and $q_2$, and let $u$ and $v$ be the remaining
points of $C_0 \cap D$; then take the conic
$D'$ passing though $u, v,
q_1, q_2$ and $o$. The sum $s+t$ will then be the remaining point of $D'
\cap C_0$.

\hinthead{15.6} 
Show that in addition to the triple point at $p$, the curve $C_{0}$
has one infinitely near point of multiplicity 3.

\hinthead{15.7} 
For (1), the adjoint ideal is the ideal of functions vanishing
to order 4 on each branch (so that the general member of the ideal with
have zero locus consisting of two smooth branches simply tangent to the
branches of the triple point). For (3), the adjoint is simply the
square of the maximal ideal.

\hinthead{15.8} The computation can be done locally analytically. Let 
$R = \widehat{\sO_{C_0, p}}$ be the completion of the local ring
of $C_0$ at $r_i$. The integral closure is then the product of rings
$R_i = \widehat{\sO_{B_i}} \cong k\[t_i\]$,
with $R_i = R/P_i$ as $P_i$ runs over the minimal primes of $R$. The
multiplicity
$m_i$ is the colength of the ideal $\sum_{j\neq i}P_j \subset R$.

\subsection*{Exercises in Chapter 16\nopunct}

\hinthead{16.3} First show that any 
locally free sheaf
\index{locally free sheaf}%
on $C$ is an iterated
extension of invertible sheaves.

\hinthead{16.5} 
The ideal sheaf of the curve on the quadric $Q$ is an extension
\index{ideal!sheaf}%
of the ideal sheaf of the quadric in $\PP^3$
with the ideal sheaf of the curve on the quadric, which~is
$$
\sO_Q(-a,-b) = \pi_1^*(\sO_{\PP^1}(-a)) \otimes \pi_2^*(\sO_{\PP^1}(-b)),
$$
where $\pi_1, \pi_2$ are the projections to $\PP^1` `$. Use the
\index{Kunneth@K\"unneth formula}%
K\"unneth formula
$$
H^1(\sO_Q(p,q)) = H^1(\sO_{\PP^1}(p)) \otimes H^0(\sO_{\PP^1}(q)) \oplus
H^0(\sO_{\PP^1}(p)) \otimes H^1(\sO_{\PP^1}(q))
$$
to compute the necessary cohomology.

\hinthead{16.6} 
Use the exact sequence
$$
0\to (fg) \to gI \oplus fJ \to gI+fJ \to 0
$$
and the corresponding exact sequence of quotients by these ideals.

\hinthead{16.9} 
Count the monomials of each degree in square of the ideal of a line.

\hinthead{16.10} 
Use the description of $I(X)$ as the ideal of $2\times 2$ minors of
$$
\begin{pmatrix}
x_0 &x_1&x_2\\
x_1& x_2& x_3
\end{pmatrix}
.
$$

\hinthead{16.12} 
Look at hyperplane sections of $C$.

\subsection*{Exercises in Chapter 17\nopunct}

\hinthead{17.3} 
Imitate the proof of Lemma~\ref{codim of 2,n 1-generic}.

\hinthead{17.5} 
Let $Y = X \cap H$ be a general hyperplane section of $X$
and consider the restriction map $H^0(\cI_{X/\PP^r}(2)) \to
H^0(\cI_{Y/\PP^{n-1}}(2))$; repeat $n-c$ times.

\hinthead{17.7} 
See \cite[Section V.1]{Hartshorne1977}.
Part (2) also follows from
the length of the 
\index{Eagon--Northcott complex}%
Eagon--Northcott complex
described in Section~\ref{EN section}.

\hinthead{17.9} 
See \cite[Section 3c]{Montreal}.

\hinthead{17.10} 
$a_1$ is again the maximal number of rulings. What's the cleanest
statement after that? Does one have
to project the scroll to $S(a_2, \dots)$ to see the rest?

\hinthead{17.14} 
$S(0,3)$ does not occur, because by Theorem~\ref{curves on a
singular scroll} the smooth curves on $S(0,3)$ are either hypersurface
sections, and thus of degree $3$ for
some integer $a$, or in the rational equivalence class of a hypersurface
section plus one line,
of degree $3a+1$, and thus not of degree 8. One can see this directly too:
if the scroll were the cone over
a twisted cubic, then the lines on the cone would cut out the $g^1_3$
on $C$. Since a pair of lines on the cone span
only a 2-plane, the sum $D$ of the two divisors would be a divisor of
degree 6  and thus $K-D$ would be a $g^1_2$,
so the curve would be hyperelliptic.

In the case when $C$ lies on $S \colonequals  S(1,2)$, we may write
the class of $C$ in terms of the hyperplane class $H$ and the class $F$
of a ruling, $C\sim pH+qF$, and we see from the
projection of $S$ to $\PP^1$ that $C$ admits
a degree $p$ covering of $\PP^1` `$. Since we have assumed that $C$
is not hyperelliptic,
we must have $p\geq 3$. By Theorem~\ref{where are the curves?} we
must have
$q\geq -p$. Since $\deg C = C\cdot H = 3p+q = 8$, we must have either
$C\sim 3H-F$ or $C\sim 4H-4F$. By the adjunction formula, in the second
case,
$$
2g(C)-2 = 8 = (C+K_S)\cdot C = (4H-4F)+(-2H+F))\cdot(4H-4F) =4
$$
whereas a similar computation in the first case yields 8; thus $C\sim
3H-F$.

The key thing to note here is that the curve $C$ has intersection number
3 with the lines of the ruling of $S$, meaning that $C$ is a trigonal
curve. (We also see that the $g^1_3$ on $C$ is unique: if $D = p +
q + r$ is a divisor moving in a pencil, the points $p, q$ and $r$ must
lie on a line; since the surface $S$ is the intersection of quadrics,
those lines must lie on $S$ and so must be  lines  of the ruling.) We
see also that the locus $W^1_4(C)$ has two components: there are pencils
with a basepoint, that is, consisting of the $g^1_3$ plus a basepoint;
and there are the residual series $K_C - g^1_3 - p$. Each of these
components of $W^1_4(C)$ is a copy of the curve $C$ itself, and they
meet in two points, corresponding to the points of intersection of $C$
with the directrix of the scroll $S$.

\hinthead{17.15} 
Use intersection theory on the scroll.
\index{intersection theory}%

\subsection*{Exercises in Chapter 18\nopunct}

\hinthead{18.1}
Use the Riemann--Roch theorem.

\hinthead{18.2}
For (3), use multilinear algebra (as in \cite{Eisenbud1995}) to define the
maps, and imitate the proof of Theorem~\ref{ENgeneral} to prove
the exactness of the given resolution of $\Sym^a(M)$. See also
\index{Sym@$\Sym$, $\Sym^i$}%
\cite[Appendix A2.6]{Eisenbud1995}.

\hinthead{18.3}
Elementary linear algebra shows that, if $k$ is a field, then a
complex 
$$k^p \luto{\ @\phi} k^q \luto{\ \psi} k^r$$ 
is exact at $k^q$ if and
only~if $\rank \phi +\rank \psi = q$.

\hinthead{18.6}
Nakayama's lemma can be used to prove that projectives are free
\index{Nakayama's lemma}%
in some cases.

\hinthead{18.7}
Show that the $3\times 3$ minors of a general $4\times 3$ matrix
of linear forms defines a Cohen--Macaulay curve
of genus 3. Show that a curve of type $(2,4)$ on a smooth quadric in
$\PP^3$ is not arithmetically Cohen--Macaulay.

\hinthead{18.8}
Use the formula for the Hilbert function of the canonical curve.

\hinthead{18.9}
To show that the rank of the syzygy matrix is $n-1$, tensor with
the field of rational functions.

\hinthead{18.11}
Consider the
hyperplane section of a trigonal canonical curve of genus~5.

\goodbreak

\subsection*{Exercises in Chapter 19\nopunct}

\hinthead{19.1}
By
restricting the presentation matrix of $I_C$ to $C$\emdash that is,
sub\-sti\-tuting the forms of degree 3 in 2 variables for the variables in
the presentation \null matrix\emdash we get a presentation
$$
R(-9)^2 \ruto {\,A} R(-6)^3 \to I_C/I_C^2 \to 0
.
$$
The kernel of the dual of $A$ is the normal bundle; 
as a module over $\CC[s,t]$ it has 2 linear
generators, the columns of the matrix
$$
\biggl(
\begin{smallmatrix}
t&0\\[2pt]
\!-s&t\\[2pt]
0&\!-s
\end{smallmatrix}
\biggr).
$$

\hinthead{19.3}
The dimensions are 11, 10 and 11.

\hinthead{19.4}
A rational quartic curve is residual to the union of two skew
lines in the complete intersection of a quadric and a cubic.

\hinthead{19.5}
On a smooth rational curve any locally free sheaf that is generated
by global sections is nonspecial.

\hinthead{19.6}
$\cN_{C/\PP^3} \cong \cO_C(2)^{\oplus 2}$.

\hinthead{19.7}
If $C = S \cap T$, show that every deformation of $C$ lies on a
deformation of $S$ and a deformation of $T$.

\hinthead{19.14}
Let $L$, $Q$ and $F$ denote a general linear form,
a general quadratic form and a general cubic form, and consider the
pencil of surfaces $S_t = V(tF + LQ) \subset \PP^3$ specializing from
the cubic surface $V(F)$ the to reducible cubic $V(LQ)$.

\hinthead{19.17}
The ideals in question are sandwiched between two successive powers
of the maximal ideal of a point in $\PP^3$.

\hinthead{19.21}
For (1), see Corollary~\ref{CBM cor 2}.
\endgroup

