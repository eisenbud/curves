%header and footer for separate chapter files

\ifx\whole\undefined
\documentclass[12pt, leqno]{book}
\usepackage{graphicx}
\usepackage{eps-to-pdf}
\input style-for-curves.sty
%\input sl-macros.sty
\usepackage{hyperref}
\usepackage{showkeys} %This shows the labels.
\usepackage{msribib}
\usepackage{pdfpages}
\usepackage{draftwatermark}
\SetWatermarkText{DRAFT:\ \today}
\SetWatermarkScale{2}
\SetWatermarkColor[gray]{0.9}

%\usepackage{SLAG,msribib,local}
%\usepackage{amsmath,amscd,amsthm,amssymb,amsxtra,latexsym,epsfig,epic,graphics}
%\usepackage[matrix,arrow,curve]{xy}
%\usepackage{graphicx}
%\usepackage{diagrams}
%
%%\usepackage{amsrefs}
%%%%%%%%%%%%%%%%%%%%%%%%%%%%%%%%%%%%%%%%%%
%%\textwidth16cm
%%\textheight20cm
%%\topmargin-2cm
%\oddsidemargin.8cm
%\evensidemargin1cm
%
%%%%%%Definitions
%\input preamble.tex
%\input style-for-curves.sty
%\def\TU{{\bf U}}
%\def\AA{{\mathbb A}}
%\def\BB{{\mathbb B}}
%\def\CC{{\mathbb C}}
%\def\QQ{{\mathbb Q}}
%\def\RR{{\mathbb R}}
%\def\facet{{\bf facet}}
%\def\image{{\rm image}}
%\def\cE{{\cal E}}
%\def\cF{{\cal F}}
%\def\cG{{\cal G}}
%\def\cH{{\cal H}}
%\def\cHom{{{\cal H}om}}
%\def\h{{\rm h}}
% \def\bs{{Boij-S\"oderberg{} }}
%
%\makeatletter
%\def\Ddots{\mathinner{\mkern1mu\raise\p@
%\vbox{\kern7\p@\hbox{.}}\mkern2mu
%\raise4\p@\hbox{.}\mkern2mu\raise7\p@\hbox{.}\mkern1mu}}
%\makeatother

%%
%\pagestyle{myheadings}

%\input style-for-curves.tex
%\documentclass{cambridge7A}
%\usepackage{hatcher_revised} 
%\usepackage{3264}
   
\errorcontextlines=1000
%\usepackage{makeidx}
\let\see\relax
\usepackage{makeidx}
\makeindex
% \index{word} in the doc; \index{variety!algebraic} gives variety, algebraic
% PUT a % after each \index{***}

\overfullrule=5pt
\catcode`\@\active
\def@{\mskip1.5mu} %produce a small space in math with an @

\title{A Chapter from ``The Practice of Algebraic Curves"}
\author{\copyright David Eisenbud and Joe Harris}
%%\includeonly{%
%0-intro,01-ChowRingDogma,02-FirstExamples,03-Grassmannians,04-GeneralGrassmannians
%,05-VectorBundlesAndChernClasses,06-LinesOnHypersurfaces,07-SingularElementsOfLinearSeries,
%08-ParameterSpaces,
%bib
%}

\date{\today}
%%\date{}
%\title{Curves}
%%{\normalsize ***Preliminary Version***}} 
%\author{David Eisenbud and Joe Harris }
%
%\begin{document}

\begin{document}
\maketitle

\pagenumbering{roman}
\setcounter{page}{5}
%\begin{5}
%\end{5}
\pagenumbering{arabic}
\tableofcontents
\fi


\chapter{Hilbert Schemes}
\label{HilbertSchemesChapter}

In Chapter~\ref{}, we looked at curves of low genus and described the linear systems on them; that is, their maps to (and in particular their embeddings in) projective space. We'd now like to revisit this topic, but now with a different question: can we describe the family of all such curves in projective space?

To set this up, let's start with some notation and conventions. To start with, we'll limit ourselves to looking at curves in $\PP^3$. We'll denote by $\cH = \cH_{dm-g+1}(\PP^3)$ the Hilbert scheme parametrizing subschemes of $\PP^3$ with Hilbert polynomial $p(m) = dm-g+1$ (which includes
curves of degree $d$ and genus $g$ in $\PP^3$), and by $\cH^\circ \subset \cH$ the open subset parametrizing smooth, irreducible, nondegenerate curves $C \subset \PP^3$. 

We're going to ask two basic questions about the schemes $\cH^\circ$:

\begin{enumerate}
\item[$\bullet$] Is $\cH^\circ$ irreducible? and
\item[$\bullet$]  What is its dimension or dimensions?
\end{enumerate}

Of course, there are many more questions we can ask about the geometry of $\cH^\circ$: for example, is it smooth or singular? Can we characterize the closure $\overline{\cH^\circ} \subset \cH$ in the whole Hilbert scheme? (In other words, can we say when a subscheme $X \subset \PP^3$ with Hilbert polynomial $dm-g+1$ is the flat limit of a family of smooth, irreducible, nondegenerate curves?) And can we determine the Picard groups of $\cH^\circ$ and its closure? We will for the most part not address these, though we will indicate the answers in special cases.

We will proceed systematically, starting with curves of the lowest possible degree and going on to successively higher degrees.

\section{Degree 3}

The smallest possible degree of an irreducible, nondegenerate curve $C \subset \PP^3$ is 3, so we'll start there. We also know that any irreducible, nondegenerate curve $C \subset \PP^3$ is a twisted cubic, so that in this case $\cH^\circ$ is simply the parameter space for twisted cubics.

As such, we have several ways of answering our two basic questions in this case. To start with the simplest, let $C_0 \subset \PP^3$ be any given twisted cubic, and consider the family of translates of $C_0$ by automorphisms $A \in \PGL_4$ of $\PP^3$: that is, the family
$$
\cC = \{ (A, p) \in \PGL_4 \times \PP^3 \; \mid \; p \in A(C_0) \}.
$$
Via the projection $\pi : \cC \to \PGL_4$, this is a family of twisted cubics, and so it induces a map
$$
\phi : \PGL_4 \to \cH^\circ.
$$
Since every twisted cubic is a translate of $C_0$, this is surjective, with fibers isomorphic to the stabilizer of $C_0$, that is, the subgroup of $\PGL_4$ of automorphisms of $\PP^3$ carrying $C_0$ to itself. Since this group is isomorphism to $\PGL_2$ and has dimension 3, and since $\PGL_4$ is irreducible of dimension 15, we conclude that \emph{$\cH^\circ$ is irreducible of dimension 12}.

This argument is based on a rather special fact, that all irreducible nondegenerate cubic curves $C \subset \PP^3$ are translates of one another. There is another, less ad-hoc way of arriving at the conclusion above which we'll now describe. 

The idea behind this approach is the fact the intersection of any two distinct quadrics $Q, Q' \supset C$ containing a twisted cubic curve $C$ is the union of $C$ and a line $L \subset \PP^3$, and the converse fact that if $L \subset \PP^3$ is any line and $Q, Q'$ two general quadrics containing $L$, then the intersection $Q \cap Q' = L \cup C$ will be the union of $L$ and a twisted cubic. This suggests that we set up an incidence correspondence: let $\PP^9$ denote the projective space of quadrics in $\PP^3$, and consider
$$
\Phi = \{ (C, L, Q, Q') \in \cH^\circ \times \GG(1,3) \times \PP^9 \times \PP^9 \; \mid \; Q \cap Q' = C \cup L \}.
$$
Now consider the projection map $\pi_2 : \Phi \to \GG(1,3)$ on the second factor. By what we just said, the fiber over any point $L \in \GG(1,3)$ is an open subset of $\PP^6 \times \PP^6$, where $\PP^6$ is the space of quadrics containing $L$; it follows that $\Phi$ is irreducible of dimension $4 + 2\times 6 = 16$. Going down the other side, we see that the map $\pi_1 : \Phi \to \cH^\circ$ is surjective, with fibers open subsets of $\PP^2 \times \PP^2$; we conclude again that \emph{$\cH^\circ$ is irreducible of dimension 12}.

Note that $\cH^\circ$ is open in the Hilbert scheme $\cH = \cH_{3m+1}(\PP^3)$, but its closure is not all of $\cH$! There is a second irreducible component of $\cH$, of dimension 15. To see this, observe that any plane cubic $C \subset \PP^2 \subset \PP^3$ has Hilbert polynomial $p(m) = 3m$. If $p \in \PP^3 \setminus C$ is any point not on $C$, then, the union $C \cup \{p\}$ is a subscheme of $\PP^3$ with Hilbert polynomial $3m+1$, and so corresponds to a point of $\cH$. But the family of such subschemes has dimension 15: we have to specify a plane in $\PP^3$ (3 parameters), a cubic curve $C$ in that plane (9 parameters) and a point $p \in \PP^4$ (3 parameters). In fact, these schemes are dense in a second irreducible component $\cH'$ of $\cH$.

In general, the Hilbert scheme $\cH_{dm-g+1}(\PP^3)$ will have many components (we don't know in general how many, or what their dimensions are), few of which actually parametrize reduced, irreducible and nondegenerate curves in $\PP^3$. This is why, for the most part, we'll be restricting our attention to the closure of $\cH^\circ$. 

\begin{exercise}
Describe the intersection of the closure  of $\cH^\circ$ in $\cH_{3m+1}$ with the second component described above. In particular, show that the locus $\Sigma$ of schemes $X$ consisting of a nodal plane cubic curve $C$ with a spatial embedded point of multiplicity 1 at the node is dense in the intersection $\overline{\cH^\circ} \cap \cH'$.
\end{exercise}

\section{Degree 4}

Let's move on to curves $C \subset \PP^3$ of degree 4 (always assumed smooth, irreducible and nondegenerate). The first thing to observe here is that by Clifford such a curve must have genus 0 or 1; we consider these cases in turn.

\subsection{Genus 0}

We can deal with rational quartics by a slight variant of the first method we used to deal with twisted cubics; in fact, this method will answer our question for rational curves of any degree. Specifically, a rational curve of degree 4 is the image of a map $\phi_F : \PP^1 \to \PP^3$ given by a four-tuple $F = (F_0,F_1,F_2,F_3)$ with $F_i \in H^0(\cO_{\PP^1}(4))$. The space of all such four-tuples up to scalars is a projective space of dimension $4 \times 5 - 1 = 19$; let $U \subset \PP^{19}$ be the open subset of four-tuples such that the map $\phi$ is a nondegenerate embedding. We then have a surjective map $\pi : U \to \cH^\circ$, whose fiber over a point $C$ is the space of maps with image $C$. Since any two such maps differ by an automorphism of $\PP^1$---that is, and element of $\PGL_2$---the fibers of $\pi$ are three-dimensional; we conclude that \emph{$\cH^\circ_{4m+1}$ is irreducible of dimension 16}.

As we said, the same analysis can be used on rational curves of any degree $d$: the space $U$ of nondegenerate embeddings $\PP^1 \to \PP^3$ of degree $d$ is an open subset of the projective space $\PP^{4(d+1)-1}$ of four-tuples of homogeneous polynomials of degree $d$ on $\PP^1$ modulo scalars; and the fibers of the corresponding map $U \to \cH^\circ_{dm+1}$ are copies of $\PGL_2$. This yields the

\begin{proposition}\label{dimension of rational curves}
The open set $\cH^0 \subset \cH_{dm+1}$ parametrizing smooth, irreducible nondegenerate rational curves $C \subset \PP^3$ is irreducible of dimension $4d$.
\end{proposition}

\begin{exercise}
Just to get some practice with the method of linkage, give an argument for Proposition~\ref{dimension of rational curves} in case $d=4$ along the lines of the second argument for twisted cubics.
\end{exercise}

\subsection{Genus 1}

What about genus 1? In fact, this is relatively simple: as we saw in Section~\ref{}, a quartic curve $C \subset \PP^3$ of genus 1 is the zero locus of a pencil $\{Q_\lambda\}_{\lambda \in \PP^1}$ of quadric surfaces, and conversely the base locus of a general such pencil if a quartic curve of genus 1. The space of such curves is thus an open subset of the Grassmannian $G(2,10) = \GG(1,9)$, and we conclude that \emph{$\cH^\circ_{4m}$ is irreducible of dimension 16}.

\section{Degree 5}

Let $C \subset \PP^3$ be a smooth, irreducible, nondegenerate quintic curve of genus $g$. To start with, we can use Clifford plus Riemann-Roch to bound the genus of $C$: by Clifford, the bundle $\cO_C(1)$ must be nonspecial, and then by Riemann-Roch we must have $g \leq 2$.

Now, we have already seen that the space $\cH^\circ_{5m+1}$ of rational quintic curves is irreducible of dimension 20. We'll consider, accordingly, the two remaining cases, $g=1$ and $g=2$.

\subsection{Genus 2}

We start with genus 2, since this is a case we've considered already in Section~\ref{}.  To recap the analysis, let $C \subset \PP^3$ be a smooth, irreducible, nondegenerate curve of degree 5 and genus 2. Since $h^0(\cO_C(2)) = 10-2+1 = 9$ by Riemann-Roch, the restriction map
$$
H^0(\cO_{\PP^3}(2)) \to H^0(\cO_C(2))
$$
must have a kernel; by B\'ezout, this kernel must be one-dimensional, that is, $C$ lies on a unique quadric surface $Q$. Similarly, the restriction map
$$
H^0(\cO_{\PP^3}(3)) \to H^0(\cO_C(3))
$$
must have at least a 6-dimensional kernel; since at most 4 of these are of the form $LQ$ for $L$ a linear form, we see that $C$ lies on a cubic surface not containing $Q$. Thus we can say that $C$ is residual to a line in the complete intersection of a quadric and a cubic surface, or, if $Q$ is smooth, in terms of the isomorphism $Q \cong \PP^1 \times \PP^1$ we can say $C$ is a curve of type $(2,3)$ on the quadric $Q$. Note that conversely if $L \subset \PP^3$ is a line and $Q$ and $S \subset \PP^3$ are general quadric and cubic surfaces containing $L$, and if we write
$$
Q \cap S = L \cup C
$$ 
then the curve $C$ is a curve of type $(2,3)$ on the quadric $Q$ and hence a quintic of genus 2.

This suggests two ways of describing the family $\cH^\circ$ of all such curves. First, we can use the fact that $C$ is linked to a line in much the same way we did in the case of twisted cubics: we set up an incidence correspondence
$$
\Psi = \{ (C, L, Q, S) \in \cH^\circ \times \GG(1,3) \times \PP^9 \times \PP^{19} \; \mid \; Q \cap S = C \cup L \},
$$
where the $\PP^9$ (respectively, $\PP^{19}$) is the space of quadric (respectively, cubic) surfaces in $\PP^3$. Given a line $L \in \GG(1,3)$, the space of quadrics containing $L$ is a $\PP^6$, and the space of cubics containing $L$ is a $\PP^{15}$; thus the fiber of the projection $\pi_2 : \Psi \to \GG(1,3)$ over $L$ is an open subset of $\PP^6 \times \PP^{15}$, and we see that \emph{$\Psi$ is irreducible of dimension $4 + 6 + 15 = 25$}.

Going back down the other way, the fiber of $\Psi$ over a point $C \in \cH^\circ$ is an open subset of the $\PP^5$ of cubics containing $C$; and we conclude that \emph{$\cH^\circ$ is irreducible of dimension $20$}.

Another, in some ways more direct, approach would be to use the fact that the quadric surface $Q$ containing a quintic curve $C \subset \PP^3$ of genus 2 is unique. We thus have a map
$$
\cH^\circ \to \PP^9,
$$
whose fiber over a point $Q \in \PP^9$ is the space of quintic curves of genus 2 on $Q$. 

The problem is, the space of quintic curves of genus 2 on a given quadric $Q$ is not in general irreducible: if $Q$ is smooth, it consists of the disjoint union of two $\PP^{11}$s (or rather open subsets of these $\PP^{11}$s), corresponding to the family of curves of type $(2,3)$ and $(3,2)$. Thus we can conclude immediately that $\cH^\circ$ is of pure dimension 20; but to conclude that its irreducible we need to verify that, in the family of all smooth quadric surfaces, the monodromy exchanges the two rulings. This is not hard: it amounts to the assertion that the family
$$
\Gamma = \{ (Q,L) \in \PP^9 \times \GG(1,3) \; \mid \; L \subset Q \}
$$
is irreducible, which can be seen readily via projection on the second factor.

\subsection{Genus 1}

\section{degree 6}
\section{degree 8}
\section{degree 9}


%footer for separate chapter files

\ifx\whole\undefined
\makeatletter\def\@biblabel#1{#1]}\makeatother
\gdef\urlhook{\url}
\bibliography{slag}
\bibliographystyle{msribib}


%%%% EXPLANATIONS:

% f and n
% some authors have all works collected at the end

\catcode`\^\active
%if ^ is followed by 
% 1:  print f, gobble the following ^ and the next character
% 0:  print n, gobble the following ^
% any other letter: print letter
\makeatletter
\def^#1{\ifx1#1f\expandafter\@gobbletwo\else
        \ifx0#1n\expandafter\expandafter\expandafter\@gobble\else#1\fi\fi}
\makeatother
\let\moreadhoc\relax
\def\indexintro{%An author's cited works appear at the end of the
%author's entry; for conventions
%see the List of Citations on page~\pageref{loc}.  
%\smallbreak\noindent
The letter `f' after a page number indicates a figure, `n' a footnote.}
\printindex[gen]
%requires makeindex
\end{document}
\else
\fi
