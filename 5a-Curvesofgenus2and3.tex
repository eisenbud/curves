%header and footer for separate chapter files

\ifx\whole\undefined
\documentclass[12pt, leqno]{book}
\usepackage{graphicx}
\usepackage{eps-to-pdf}
\input style-for-curves.sty
%\input sl-macros.sty
\usepackage{hyperref}
\usepackage{showkeys} %This shows the labels.
\usepackage{msribib}
\usepackage{pdfpages}
\usepackage{draftwatermark}
\SetWatermarkText{DRAFT:\ \today}
\SetWatermarkScale{2}
\SetWatermarkColor[gray]{0.9}

%\usepackage{SLAG,msribib,local}
%\usepackage{amsmath,amscd,amsthm,amssymb,amsxtra,latexsym,epsfig,epic,graphics}
%\usepackage[matrix,arrow,curve]{xy}
%\usepackage{graphicx}
%\usepackage{diagrams}
%
%%\usepackage{amsrefs}
%%%%%%%%%%%%%%%%%%%%%%%%%%%%%%%%%%%%%%%%%%
%%\textwidth16cm
%%\textheight20cm
%%\topmargin-2cm
%\oddsidemargin.8cm
%\evensidemargin1cm
%
%%%%%%Definitions
%\input preamble.tex
%\input style-for-curves.sty
%\def\TU{{\bf U}}
%\def\AA{{\mathbb A}}
%\def\BB{{\mathbb B}}
%\def\CC{{\mathbb C}}
%\def\QQ{{\mathbb Q}}
%\def\RR{{\mathbb R}}
%\def\facet{{\bf facet}}
%\def\image{{\rm image}}
%\def\cE{{\cal E}}
%\def\cF{{\cal F}}
%\def\cG{{\cal G}}
%\def\cH{{\cal H}}
%\def\cHom{{{\cal H}om}}
%\def\h{{\rm h}}
% \def\bs{{Boij-S\"oderberg{} }}
%
%\makeatletter
%\def\Ddots{\mathinner{\mkern1mu\raise\p@
%\vbox{\kern7\p@\hbox{.}}\mkern2mu
%\raise4\p@\hbox{.}\mkern2mu\raise7\p@\hbox{.}\mkern1mu}}
%\makeatother

%%
%\pagestyle{myheadings}

%\input style-for-curves.tex
%\documentclass{cambridge7A}
%\usepackage{hatcher_revised} 
%\usepackage{3264}
   
\errorcontextlines=1000
%\usepackage{makeidx}
\let\see\relax
\usepackage{makeidx}
\makeindex
% \index{word} in the doc; \index{variety!algebraic} gives variety, algebraic
% PUT a % after each \index{***}

\overfullrule=5pt
\catcode`\@\active
\def@{\mskip1.5mu} %produce a small space in math with an @

\title{A Chapter from ``The Practice of Algebraic Curves"}
\author{\copyright David Eisenbud and Joe Harris}
%%\includeonly{%
%0-intro,01-ChowRingDogma,02-FirstExamples,03-Grassmannians,04-GeneralGrassmannians
%,05-VectorBundlesAndChernClasses,06-LinesOnHypersurfaces,07-SingularElementsOfLinearSeries,
%08-ParameterSpaces,
%bib
%}

\date{\today}
%%\date{}
%\title{Curves}
%%{\normalsize ***Preliminary Version***}} 
%\author{David Eisenbud and Joe Harris }
%
%\begin{document}

\begin{document}
\maketitle

\pagenumbering{roman}
\setcounter{page}{5}
%\begin{5}
%\end{5}
\pagenumbering{arabic}
\tableofcontents
\fi


\chapter{Curves of genus 2 and 3}\label{genus 2 and 3 chapter}


\section{Curves of genus 2}

Canonical map to $\PP^1$. Embedding in $\PP^3$ as $(2,3)$ on a quadric, via any degree 5 line bundle. Ideal is 1 quadric, 2 cubics.
Plane model of degree 4 with node or cusp.

\subsubsection{Representations as double covers of $\PP^1$}

As with curves of genus 1, there are no nontrivial linear series of degree 0 or 1 on a curve of genus 2; the first positive-dimensional linear series occurs in degree 2. Unlike the case of genus 1, however, this series is unique: by Riemann-Roch, if $D$ is any divisor of degree 2 on a curve $C$ of genus 2, we have
$$
h^0(D) = 1 + h^1(D) = 1 + h^0(K-D);
$$
since $K-D$ has degree 0, this says that $h^0(D) > 1$ if and only if $D=K$, in which case $|D| = |K|$ is the canonical $g^1_2$ on $C$.

The canonical series gives a map $\phi_K : C \to \PP^1$ expressing $C$ as a double cover of $\PP^1$; as in the case of genus 1, this means we can realize $C$ as the smooth projective compactification of the affine curve given by
$$
y^2 = x(x-1)(x - \alpha)(x - \beta)(x - \gamma)
$$
for some triple $\alpha,\beta,\gamma \in \CC$ distinct from each other and from 0 and 1. This representation shows us that the moduli space $M_2$ is the space of 6-tuples of distinct points in $\PP^1$ modulo the action of $PGL_2$. This tells us immediately that $M_2$ is irreducible of dimension 3; with a fair amount of additional work, we can also use this to describe the coordinate ring of $M_2$ (\ref{**}).

\subsubsection{Embeddings in $\PP^3$}

For line bundles $L$ of degree $d \geq 3$ on $C$, Riemann-Roch tells us simply that $h^0(D) = d - 1$; if we want to embed our curve $C$ in projective space, accordingly, we had better take $d \geq 5$. Conversely, Corollary~(\ref{degree 2g+1 embedding}) tells us that any line bundle of degree 5 on $C$ is very ample, so we'll consider first the embeddings of $C$ given by those.

So: for the following, let $L$ be any line bundle of degree 5 on our curve $C$, and $\phi_L : C \to \PP^3$ the embedding given by the complete linear system $|L|$. By a mild abuse of language, we'll also denote the image $\phi_L(C) \subset \PP^3$ by $C$.

The first question to ask is once more, what degree surfaces in $\PP^3$ contain the curve $C$? We start with degree 2, where we consider the restriction map
$$
H^0(\cO_{\PP^3}(2)) \to H^0(\cO_C(2)) = H^0(L^2).
$$
The space on the left has dimension 10 as always; on the right, Riemann-Roch tells us that $h^0(L^2) = 2\cdot5 - 2 + 1 = 9$. It follows that $C$ must lie on a quadric surface $Q$; and by Bezout that $Q$ is unique (since $C$ can't lie on a union of planes, any quadric containing $C$ must be irreducible; if there were more than one such, Bezout would imply that $\deg(C) \leq 4$).

We might ask at this point: is $Q$ smooth or a quadric cone? The answer depends on the choice of line bundle $L$:

\begin{proposition}
Let $C \subset \PP^3$ be a smooth curve of degree 5 and genus 2 and $Q \subset \PP^3$ the unique quadric containing $C$. If $L = \cO_C(1) \in \pic^5(C)$, then $Q$ is singular if and only if we have
$$
L \cong K^2(p)
$$
for some point $p \in C$.
\end{proposition}

(Note that there is a 2-parameter family of line bundles of degree 5 on $C$ **we don't know this yet, unless we want to state it somewhere**, of which a one-dimensional subfamily are of the form $K^2(p)$, conforming to our naive expectation that ``in general" $Q$ should be smooth, and that it should become singular in codimension 1.)

\begin{proof}
First, suppose that the line bundle $L \cong K^2(p)$ for some $p \in C$. Then $L(-p) \cong K^2$, meaning that the map $\pi : C \to \PP^2$ given by projection from $p$ is the map $\phi_{K^2} : C \to \PP^2$ given by the square of the canonical bundle.

What does this map look like?
\end{proof}

Whether the quadric $Q$ is smooth or not, we can describe a minimal set of generators of the homogeneous ideal $I(C) \subset \CC[x_0, x_1, x_2, x_3]$ similarly. First, we look at the restriction map
$$
H^0(\cO_{\PP^3}(3)) \to H^0(\cO_C(3));
$$
since the dimensions of these spaces are 20 and $15-2+1 = 14$ respectively, we see that  vector space of cubics vanishing on $C$ has dimension at least 6. Four of these are already accounted for: we can take the defining equation of $Q$ and multiply it by any of the linear forms on $\PP^3$; we conclude, accordingly, that \emph{there are at least two cubics vanishing on $C$ linearly independent modulo those vanishing on $Q$}.

In fact, we can prove the existence of these cubics geometrically, and show that there are no more than 2 linearly independent modulo the ideal of $Q$. Suppose first that $Q$ is smooth, so that $C$ is a curve of type $(2,3)$ on $Q$. In that case, if $L \subset Q$ is any line of the first ruling, the sum $C+L$ is the complete intersection of $Q$ with a cubic $S_L$, unique modulo the ideal of $Q$; conversely, if $S$ is any cubic containing $C$ but not containing $S$, the intersection $S \cap Q$ will be the union of $C$ and a line $L$ of the first ruling; thus, mod $I(Q)$, $S = S_L$. A similar argument applies in case $Q$ is a cone, and $L$ is any line of the (unique) ruling of $Q$.

\begin{exercise}
Show that for any pair of lines $L, L'$ of the appropriate ruling of $Q$, the three polynomials $Q$, $S_L$ and $S_{L'}$ generate the homogeneous ideal $I(C)$. Find relations among them. Write out the minimal resolution of $I(C)$.
\end{exercise}

\subsubsection{Projective normality III}

\begin{theorem}
 Let $C$ be a smooth (is reduced, irreducible enough?) curve of arithmetic genus $g$, and let $\cL$ be a line bundle on $C$ of degree $\geq 2g+1$. The image of 
 $C$ under the complete linear series $|\cL|$ is projectively normal ( when $C$ is singular, aritmetically Cohen-Macaulay).
\end{theorem}

\begin{proof}
 The line bundle $\cL$ is very ample by \ref{?}. Thus it suffices We must show that the multiplication map $H^0(\sL)\otimes H^0(\cL^{m}) \to H^0(\cL^{m+1})$ is surjective for all $m\geq 1$.
 For $m=1$ do it by number of quadrics, uniform position. For m>1 the bpf pencil trick.
\end{proof}

\section{Curves of genus 3}

This will be, perhaps somewhat counter-intuitively, the shortest of the sections in this chapter. The reason is simple: for a non-hyperelliptic curve of genus 3, the canonical model is virtually the only one we will deal with. For curves $C$ of other genera, different representations of $C$---as a branched cover of $\PP^1$, as the normalization of a plane curve $C_0 \subset \PP^2$, as embedded in $\PP^3$ and higher-dimensional projective spaces---display different aspects of the geometry of the curve; and it's correspondingly valuable to understand all these different models of $C$ and their relation to one another. For a non-hyperelliptic curve of genus 3, by contrast, the canonical embedding is  the only one we deal with; virtually all the aspects of the geometry of $C$ are best seen in this model.

So: let $C$ be a smooth projective curve of genus 3. The is an immediate bifurcation into two cases, hyperelliptic and non-hyperelliptic curves; we will discuss hyperelliptic curves of any genus in Section~\ref{**}, and so for the following we'll assume $C$ is nonhyperellitic. By our general theorem~\ref{**}, this means that the canonical map $\phi_K : C \to \PP^2$ embeds $C$ as a smooth plane quartic curve; and conversely, by adjunction any smooth plane of degree 4 has genus 3 and is canonical (that is, $\cO_C(1) \cong K_C$).

Note that this gives us a way to determine the dimension of the moduli space $M_3$ of smooth curves of genus $3$: if $\PP^{14}$ is the space of all plane quartic curves, and $U \subset \PP^{14}$ the open subset corresponding to smooth curves, we have a dominant map $U \to M_3$ whose fibers are isomorphic to the 8-dimensional affine group $PGL_3$. (Actually, the fiber over a point $[C] \in M_3$ is isomorphic to the quotient of $PGL_3$ by the automorphism group of $C$; but since $Aut(C)$ is finite this is still 8-dimensional.) We conclude, therefore, that
$$
\dim M_3 = 14 - 8 = 6.
$$

What about other linear series on $C$, and the corresponding models of $C$? To start with, by hypothesis $C$ has no $g^1_2$s; that is, it is not expressible as a 2-sheeted cover of $\PP^1$. On the other hand, it is expressible as a 3-sheeted cover: if $L \in \pic^3(C)$ is a line bundle of degree 3, by Riemann-Roch we have
$$
h^0(L) = 
\begin{cases}
2, &\text{if $L \cong K-p$ for some point $p \in C$; and} \\
1 &\text{otherwise.}
\end{cases}
$$
There are thus a 1-dimensional family of representations of $C$ as a 3-sheeted cover of $\PP^1$. In fact, these are plainly visible from the canonical model: the degree 3 map $\phi_{K-p} : C \to \PP^1$ is just the composition of the canonical embedding $\phi_K : C \to \PP^2$ with the projection from the point $p$.

There are of course other representations of $C$ as the normalization of a plane curve. By Riemann-Roch, $C$ will have no $g^2_3$s and the canonical series is the only $g^2_4$, but there are plenty of models as plane quintic curves: by Proposition~\ref{**}, if $L$ is any line bundle of degree 5, the linear series $|L|$ will be a base-point-free $g^2_5$ as long as $L$ is not of the form $K+p$, so that $\phi_L$ maps $C$ birationally onto a plane quintic curve $C_0 \subset \PP^2$. But these can also be described geometrically in terms of the canonical model: any such line bundle $L$ is of the form $2K-p-q-r$ for some trio of  points $p, q, r \in C$ that are not colinear in the canonical model, and we see correspondingly that $C_0$ is obtained from the canonical model of $C$ by applying a Cremona transform with respect to the points $p, q$ and $r$. 

We can also embed $C$ in $\PP^3$ as a smooth sextic curve by Proposition~\ref{**}; in fact, a line bundle $L \in \pic^6(C)$ of degree 6 will be very ample if and only if it is not of the form $K+p+q$ for any $p, q \in C$. One cheerful fact in this connection is that these curves are determinantal:

\begin{exercise}
Let $C \subset \PP^3$ be a smooth non-hyperelliptic curve of degree 3 and genus 6. Show that there exists a $3 \times 4$ matrix $M$ of linear forms on $\PP^3$ such that 
$$
C = \{ p \in \PP^3 \mid \rank(M(p)) \leq 2 \}.
$$
\end{exercise}

\input footer.tex


