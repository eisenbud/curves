%header and footer for separate chapter files

\ifx\whole\undefined
\documentclass[12pt, leqno]{book}
\usepackage{graphicx}
\input style-for-curves.sty
\usepackage{hyperref}
\usepackage{showkeys} %This shows the labels.
%\usepackage{SLAG,msribib,local}
%\usepackage{amsmath,amscd,amsthm,amssymb,amsxtra,latexsym,epsfig,epic,graphics}
%\usepackage[matrix,arrow,curve]{xy}
%\usepackage{graphicx}
%\usepackage{diagrams}
%
%%\usepackage{amsrefs}
%%%%%%%%%%%%%%%%%%%%%%%%%%%%%%%%%%%%%%%%%%
%%\textwidth16cm
%%\textheight20cm
%%\topmargin-2cm
%\oddsidemargin.8cm
%\evensidemargin1cm
%
%%%%%%Definitions
%\input preamble.tex
%\input style-for-curves.sty
%\def\TU{{\bf U}}
%\def\AA{{\mathbb A}}
%\def\BB{{\mathbb B}}
%\def\CC{{\mathbb C}}
%\def\QQ{{\mathbb Q}}
%\def\RR{{\mathbb R}}
%\def\facet{{\bf facet}}
%\def\image{{\rm image}}
%\def\cE{{\cal E}}
%\def\cF{{\cal F}}
%\def\cG{{\cal G}}
%\def\cH{{\cal H}}
%\def\cHom{{{\cal H}om}}
%\def\h{{\rm h}}
% \def\bs{{Boij-S\"oderberg{} }}
%
%\makeatletter
%\def\Ddots{\mathinner{\mkern1mu\raise\p@
%\vbox{\kern7\p@\hbox{.}}\mkern2mu
%\raise4\p@\hbox{.}\mkern2mu\raise7\p@\hbox{.}\mkern1mu}}
%\makeatother

%%
%\pagestyle{myheadings}

%\input style-for-curves.tex
%\documentclass{cambridge7A}
%\usepackage{hatcher_revised} 
%\usepackage{3264}
   
\errorcontextlines=1000
%\usepackage{makeidx}
\let\see\relax
\usepackage{makeidx}
\makeindex
% \index{word} in the doc; \index{variety!algebraic} gives variety, algebraic
% PUT a % after each \index{***}

\overfullrule=5pt
\catcode`\@\active
\def@{\mskip1.5mu} %produce a small space in math with an @

\title{Personalities of Curves}
\author{\copyright David Eisenbud and Joe Harris}
%%\includeonly{%
%0-intro,01-ChowRingDogma,02-FirstExamples,03-Grassmannians,04-GeneralGrassmannians
%,05-VectorBundlesAndChernClasses,06-LinesOnHypersurfaces,07-SingularElementsOfLinearSeries,
%08-ParameterSpaces,
%bib
%}

\date{\today}
%%\date{}
%\title{Curves}
%%{\normalsize ***Preliminary Version***}} 
%\author{David Eisenbud and Joe Harris }
%
%\begin{document}

\begin{document}
\maketitle

\pagenumbering{roman}
\setcounter{page}{5}
%\begin{5}
%\end{5}
\pagenumbering{arabic}
\tableofcontents
\fi


\chapter{Hilbert Schemes}
\label{HilbertSchemesChapter}

In earlier chapters, we described some  degree $d$ embeddings of curves of  genus $g$ in projective spaces $\PP^r$ for small $g,r,d$. In this chapter, we will try to describe the \emph{restricted Hilbert scheme} $\cH^\circ_{g,3,d}$, defined to be the open subscheme of the Hilbert scheme $\cH_{g,3,d} := Hilb_{dm-g+1}(\PP^3)$ parametrizing smooth, irreducible and nondegenerate curves of degree $d$ and genus $g$ in $\PP^3$.

Three basic questions about the schemes $\cH^\circ_{g,r,d}$ are:

\begin{enumerate}
\item[$\bullet$] Is $\cH^\circ_{g,r,d}$ irreducible? 
\item[$\bullet$]  What is its dimension (or the dimensions of its components)?
\item[$\bullet$] Where is it smooth, and where is it singular?
\end{enumerate}

Of course, there are many more questions one could ask about the geometry of $\cH^\circ_{g,r,d}$, many of which are open. For example,  what is the closure $\overline{\cH^\circ_{g,r,d}} \subset \cH_{g,r,d}$ in the whole Hilbert scheme? (In other words, when is a subscheme $X \subset \PP^r$ with Hilbert polynomial $dm-g+1$ \emph{smoothable}, in the sense that it is the flat limit of a family of smooth curves?) In general, no one knows!


\section{Degree 3}

The first case to consider is that of the Hilbert scheme  $\cH_{0,3,3}$. The corresponding restricted Hilbert scheme $\cH^\circ_{0,3,3}$, parameterizing twisted cubics, is one we have encountered and described already: in Proposition~\ref{hilb of twisted cubics} we showed that $\cH^\circ_{0,3,3}$ is irreducible of dimension 12,
and we gave another proof, based on linkage, in Chapter~\ref{LinkageChapter}. 
By Exercise~\ref{twisted cubic normal bundle}, the normal bundle of a twisted cubic $C$ is $\sN_{C/\PP^3}=\sO_{\PP^1}(5)\oplus \sO_{\PP^1}(5)$
so by Theorem~\ref{tangent space of Hilb} the tangent space to the Hilbert scheme at $C$ is
$H^0(\sN_{C/\PP^3}) = \CC^{12}$. Thus $\cH_{0,3,3}$ is smooth at this point.

\subsection{The other component of $\cH_{0,3,3}$}

We might naively expect that the closure $\overline{\cH^\circ_{0,3,3}}$ would be all of $\cH_{0,3,3}$, but this is not the case:  $\cH_{0,3,3}$ has two irreducible components. 

One component is the closure of $\cH^\circ_{0,3,3}$. To describe the second (``extraneous'') component, suppose we start with a plane cubic curve $C_0 \subset \PP^2 \subset \PP^3$. This is of course a curve of degree 3, but its Hilbert polynomial is $p(m) = 3m$ rather than $3m+1$, reflecting the fact that the genus of $C_0$ is 1, not 0.

But that can be fixed: adding a point $p \in \PP^3 \setminus C_0$ to $C_0$ has the effect of increasing the Hilbert polynomial by 1, so that the Hilbert polynomial of $C := C_0\cup \{p\}$ is $3m+1$. Thus $C$  corresponds to a point of $\cH_{0,3,3}$. In fact, the locus in $\cH_{0,3,3}$ of curves $C$ of this form is open, and its closure, which includes plane cubics with an embedded point, is a second irreducible component of $\cH_{0,3,3}$. 


\begin{theorem}
$\cH_{0,3,3}$ has two irreducible components. One has generic point corresponding to  a twisted cubic,
and the other has generic point corresponding to the union of a smooth plane cubic and a point outside the plane.
They have dimension 12 and 15 respectively.
\end{theorem}

\begin{proof}
Let $C'$ be the purely 1-dimensional scheme defined by the intersection of the 1-dimensional primary ideals in the decomposition of $I_C$. If the Hilbert polynomials of $C$ and $C'$ are $p(m)$ and  $p'(m)$ then
$p(m) \geq p'(m)$ for all large $m$; equality for large $m$ would imply that $C'=C$.

Whatever 0-dimensional components $C'$ may have do not contribute to the degree (= leading coefficient of the Hilbert polynomial) so $\deg C' = \deg C = 3$. Thus the curve $C'$ is either irreducible or the union of two or three irreducible components. In the first case $C'$ is either nondegenerate, in which case it is a twisted cubic by Theorem~\ref{characterization of P1} and $C' = C$; or a plane cubic. If $C'$ is a plane cubic, then it has Hilbert function $3m$, so $\sI_{C'}/\sI_C$
corresponds to a point in $\PP^3$, either embedded in $C'$ or not. Such a union is specified by the choice of the
plane, the cubic in it, and the point,\footnote{if the point is an embedded point, specifying $C'$ and $p$ does not determine $C$---if $p$ is a smooth point of $C'$, for example, there is a one-parameter family of curves $C$ with support $C'$ and an embedded point at $p$; the tangent space to $C$ at $p$ can be any plane containing the tangent line to $C'$ at $p$---but these are all flat limits of disjoint unions $C' \sqcup \{p\}$, so they don't contribute a separate component of $\cH_{0,3,3}$.} and thus has dimension $3+9+3 = 15.$

On the other hand, if $C$ is not planar and not irreducible, then $C'$ consists of 3 lines or the union of a (planar) conic
and a line not in the plane. In this case each connected component has arithmetic genus 0 and thus Hilbert polynomial
with constant term 1; so the curve must be connected. All such curves can be realized as divisors of type $(1,2)$
on a quadric, and thus have Hilbert function equal to that of $C$, whence again $C' = C$.
\end{proof}


\begin{figure}
\centerline {%
\includegraphics[width=1.7in]{"main/Fig18-1A"}\quad
\includegraphics[width=1.7in]{"main/Fig18-1B"}%
\includegraphics[width=1.7in]{"main/Fig18-1C"}%
}
\caption{$\sH_{0,3,3}$ has two components, whose generic members are shown on the left and right,
with the generic member of the intersection shown in the middle. On the left is the ``principal component'' whose generic point is a smooth twisted cubic. This degenerates to a singular plane
cubic with an embedded point at the node. In the other component, the singular plane cubic
becomes smooth, and the extra point is free to move anywhere in space.
{Silvio: The left-hand picture is (mathematically) a twisted cubic; the middle curve is a plane cubic with a node
and the vertical arrow represents an infinitesimal (non-planar) thickening; the right picture is supposed to be a smooth
plane elliptic curve, and a point, floating in 3-space away from the plane.
I suggest droppint the $\cup$ and moving the floating point above the plane, to indicate that all three pictures are in the same
3-space. The three curves should all be red. The elliptic curve can be the usual beautiful one (two ovals, one extending
to the edge of the plane. 
The middle picture actually represents a degeneration of each of the ones on the left and right. Maybe place it a little lower, and put in arrows??
An additional possibility to consider: add a "plane" that would be seen as "below" the curve on the left. }}
\label{Fig18.1}
\end{figure}


\begin{fact}
In \cite{Piene-Schlessinger} it is also shown that the two components of $\cH_{0,3,3}$ are smooth and rational and that
their  intersection is  smooth and rational of dimension 11.
\end{fact}

The presence of components of the Hilbert scheme whose general member is not smooth, irreducible and nondegenerate is not exceptional, and in the following section we'll describe some of the ``extraneous components" more generally. But one aspect of the geometry of $\cH_{0,3,3}$ is special: the action of the group $PGL_4$ on $\cH_{0,3,3}$ has only finitely many orbits (if you want to see the orbits in the principle component $\overline{\cH}^\circ_{0,3,3}$, see~\cite{Montreal}). It's not known if there are other examples of this, beyond Grassmannians, Hilbert schemes of quadric hypersurfaces and Hilbert schemes of triples of points.  In Section~\ref{rigid?} we will discuss the opposite situation: the possibility of ``rigid curves,"  corresponding to points in the Hilbert scheme $\cH^{\circ}_{dm-g+1}$ with $g>0$ whose $PGL_{r+1}$ orbit is open.

\section{Extraneous components}

 Let $\cH_d(\PP^r)$ be the Hilbert scheme of subschemes of $\PP^r$ with Hilbert polynomial of degree 0, say equal to the constant $d$. There is an open subset $\cH^\circ_d(\PP^r)$ of dimension $dr$ whose points correspond to reduced $d$-tuples of points in $\PP^r$. This open subset is isomorphic to the complement of the diagonals in the $d$th symmetric power of $\PP^r$. We call the closure of this open set the \emph{principal component} of $\cH_d(\PP^r)$. 
 
The Hilbert scheme  $\cH_d(\PP^2)$ is smooth and irreducible of dimension $2d$, and a similar result holds for 
0-dimenensional schemes on any smooth surface, but Iarrobino showed that, for any $r \geq 3$ and any sufficiently large $d$, there are components of $\cH_d(\PP^r)$ having dimension strictly larger than $dr$ \cite{Iarrobino1985}; see Exercise~\ref{bigger component}. There are also examples of extraneous components
of dimension $<dr$; see~\cite{MR2579394}. No one knows how many irreducible components the Hilbert scheme $\cH = \cH_d(\PP^r)$ has, or what their dimensions might be.

This in turn infects the Hilbert schemes of curves. For example, the Hilbert scheme $\cH_{g,r,d}$ has a component whose general point corresponds to a union of a plane curve of degree $d$ and $\binom{d-1}{2} - g$ points; moreover, if $\Gamma$ is any irreducible component of the Hilbert scheme of zero-dimensional subschemes of degree $\binom{d-1}{2} - g$ in $\PP^3$, there is a component of $\cH_{g,3,d}$ whose  general point corresponds to a union of a plane curve of degree $d$ and the subscheme corresponding to a general point of $\Gamma$. Further, we can replace the plane curves in this construction with any component of the Hilbert scheme of curves of degree $d$ and genus $g' > g$. There can also be components of $\cH_{d,r,3}$ whose general point corresponds to a subscheme of $\PP^3$ with a spatial embedded point---see~\cite{Chen-Nollet}.

Bottom line: it's a mess. For many $g,d$ the Hilbert scheme $\cH_{g,3,d}$ has many components. In most cases no one knows how many, or what their dimensions are, which is why we most often focus on the restricted Hilbert scheme. 

\section{Degree 4}

By Clifford's theorem, an irreducible nondegenerate curve of degree 4 in $\PP^{3}$ must have genus 0 or 1; we consider these cases in turn.

\subsection{Genus 0}\label{degree 4 genus 0}

We can deal with rational quartics by a slight variant of the first method we used to deal with twisted cubics. A rational curve of degree 4 is the image of a map $\phi_F : \PP^1 \to \PP^3$ given by a four-tuple $F = (F_0,F_1,F_2,F_3)$ with $F_i \in H^0(\cO_{\PP^1}(4))$. The space of all such four-tuples up to scalars is a projective space of dimension $4 \times 5 - 1 = 19$; let $U \subset \PP^{19}$ be the open subset of four-tuples such that the map $\phi$ is a nondegenerate embedding. By the universal property of the Hilbert scheme there is a surjective map $\pi : U \to \cH^\circ$, whose fiber over a point $C$ is the space of maps with image $C$. Since any two such maps differ by an automorphism of $\PP^1$---that is, an element of $PGL_2$---the fibers of $\pi$ are three-dimensional, proving that $\cH^\circ_{0,3,4}$ is irreducible of dimension 16. 

The same analysis can be used on rational curves of any degree $d$ in any projective space $\PP^r$:

\begin{proposition}\label{dimension of rational curves}
The open set $\cH^\circ_{0,r,d}$ parametrizing smooth, irreducible nondegenerate rational curves $C \subset \PP^3$ is irreducible of dimension $(r+1)(d+1)-4$; in case $r=3$ in particular it has dimension $4d$.
\end{proposition}

\begin{proof}
The space $U$ of nondegenerate embeddings $\PP^1 \to \PP^r$ of degree $d$ is an open subset of the projective space $\PP^{(r+1)(d+1)-1}$ of $(r+1)$-tuples of homogeneous polynomials of degree $d$ on $\PP^1$ modulo scalars; and the fibers of the corresponding map $U \to \cH^\circ_{dm+1}$ are copies of $PGL_2$. 
\end{proof}

In contrast to the case of twisted cubics, smooth rational curves in $\PP^r$ of the same degree may have different normal bundles. This gives an interesting stratification of the restricted Hilbert scheme of rational curves; see \cite{MR3778979} for a discussion.

\subsection{Genus 1}
 As we saw in Section~\ref{g=1 in P3,P4}, a quartic curve $C \subset \PP^3$ of genus 1 is the intersection of two quadric surfaces, and by Lasker's theorem, every quadric containing $C$ is a linear combination of those two. Conversely, the intersection of two general quadrics in $\PP^3$ is a quartic curve of genus 1, as follows from B\'ezout's theorem and the adjunction formula. We can thus construct a family of quartics of genus 1: let $V = H^0(\cO_{\PP^3}(2))$ be the 10-dimensional vector space of homogeneous quadric polynomials in $\PP^3$ and $G(2,V)$ the Grassmannian of 2-planes in $V$, and consider the incidence correspondence
$$
\Gamma = \{ (\Lambda, p) \in G(2,V) \times \PP^3 \mid F(p)=0 \; \forall \; F \in \Lambda \}.
$$
The fiber of $\Gamma$ over a point $\Lambda \in G(2,V)$ is thus the base locus of the pencil of quadrics represented by $\Lambda$; let $B \subset G(2,V)$ be the Zariski open subset over which the fiber is smooth, irreducible and nondegenerate of dimension 1. By the universal property of Hilbert schemes, the family $\pi_1 : \Gamma_B \to U$ induces a map $\phi : B \to \cH^\circ_{1,3,4}$ that is one-to-one on points; it follows that the reduced subscheme of $\cH^\circ_{1,3,4}$ is birational to an open subset of the Grassmannian $G(2,10)$. We conclude that $\cH^\circ_{1,3,4}$ is irreducible of dimension 16. Exercise~\ref{hilb 1,3,4} shows that this map is actually an isomorphism.

The  argument  here---where we constructed a family $\cC \to B$ of curves of given type, and then invoked the universal property of the Hilbert scheme to get a map $B \to \cH$---is typical in analyses of Hilbert schemes. 

\section{Degree 5}

Let $C \subset \PP^3$ be a smooth, irreducible, nondegenerate quintic curve of genus $g$. By Clifford's theorem the bundle $\cO_C(1)$ must be nonspecial, so  by the Riemann-Roch theorem we must have $0\leq g \leq 2$. We have already seen that the space $\cH^\circ_{0,3,5}$ of rational quintic curves is irreducible of dimension 20. We will treat the case $g=2$ in detail, and leave the case $g=1$ as Exercise~\ref{quintics genus 1}. (Degree 5 will be covered in a different way in Section~\ref{estimating dim hilb}.)

\subsection{Genus 2}

We have considered curves of genus 2 in Section~\ref{genus 2 section}.  Recall that a curve of genus 2 and degree 5 in 
$\PP^3$ is contained in the intersection of a unique quadric surface $Q$ and a cubic surface $S$ not containing $Q$.
The intersection $Q\cap S$
has degree 6, and is thus the union of $C$ and a line. If $Q$ is smooth then, in terms of the isomorphism $Q \cong \PP^1 \times \PP^1$, the curve $C$ is a divisor of type $(2,3)$ on the quadric $Q$. Note that conversely if $L \subset \PP^3$ is a line and $Q$ and $S \subset \PP^3$ are general quadric and cubic surfaces containing $L$, and if we write
$$
Q \cap S = L \cup C
$$ 
then the curve $C$ is a curve of type $(2,3)$ on the quadric $Q$ and hence, by the adjunction formula,
 a quintic of genus 2.

This suggests two ways of describing the family $\cH^\circ_{2,3,5}$ of such curves. First, we can use the fact that $C$ is linked to a line to make an incidence correspondence
$$
\Psi = \{ (C, L, Q, S) \in \cH^\circ \times \GG(1,3) \times \PP^9 \times \PP^{19} \; \mid \; Q \cap S = C \cup L \},
$$
where the $\PP^9$ (respectively, $\PP^{19}$) is the space of quadric (respectively, cubic) surfaces in $\PP^3$. Given a line $L \in \GG(1,3)$, the space of quadrics containing $L$ is a $\PP^6$, and the space of cubics containing $L$ is a $\PP^{15}$; thus the fiber of the projection $\pi_2 : \Psi \to \GG(1,3)$ over $L$ is an open subset of $\PP^6 \times \PP^{15}$, and we see that $\Psi$ is irreducible of dimension $4 + 6 + 15 = 25$.

On the other hand, the fiber of $\Psi$ over a point $C \in \cH^\circ_{2,3,5}$ is an open subset of the $\PP^5$ of cubics containing $C$; and we conclude that $\cH^\circ_{2,3,5}$ is irreducible of dimension $20$.

Another approach to describing the restricted Hilbert scheme $\cH^\circ_{2,3,5}$ would be to use the fact that the quadric surface $Q$ containing a quintic curve $C \subset \PP^3$ of genus 2 is unique. We thus have a map
$$
\cH^\circ \to \PP^9,
$$
whose fiber over a point $Q \in \PP^9$ is the space of quintic curves of genus 2 on $Q$. 

The general fiber of this map, the space of quintic curves of genus 2 on a smooth quadric $Q$ is  reducible: it consists of the disjoint union of the open subsets of smooth elements in the two linear series of curves of type $(2,3)$ and $(3,2)$ on $Q$, each of which is a $\PP^{11}$, while over a 
quadric cone the fiber is irreducible, since the cone has a unique family of lines.  We can conclude immediately that $\cH^\circ_{2,3,5}$ is of pure dimension 20; to conclude that it is irreducible we use the fact that in the family of all smooth quadric surfaces, the monodromy exchanges the two rulings (Example~\ref{monodromy of rulings}).

\section{Degree 6}

Again the Clifford and Riemann-Roch theorems suffice to compute the possible genera of a curve of degree 6. To start with,  if the line bundle $\cO_C(1)$ is nonspecial, then by the Riemann-Roch theorem we have $g \leq 3$. Suppose on the other hand that $\cO_C(1)$ is special. Since   $h^{0}(\cO_C(1)) \geq 4$, we have equality in Clifford's theorem, and either $C$ is hyperelliptic and $\cO_C(1)$ is a multiple of the $g^{1}_{2}$ or  $C$ is  a canonically embedded curve of genus 4. The first case cannot occur, since no special series on a hyperelliptic curve is very ample; thus $C$ must be a canonical curve of genus 4. Thus a smooth irreducible, nondegenerate curve of degree 6 in $\PP^3$ has genus at most 4.


The cases of genera 0, 1 and 2 are covered under Proposition~\ref{nonspecial Hilbert} below, leaving us the cases $g = 3$ and 4. In both these cases we can describe the ideal of the curve.

\subsection{Genus 4}

As we've seen in Section\ref{canonical genus 4} a canonical curve of genus 4 is the complete intersection of a (unique) quadric $Q$ and a cubic surface $S$. We thus have a map
$$
\alpha : \cH^\circ_{4,3,6} \rTo \PP^9
$$
sending a curve $C$ to the quadric $Q$ containing it. Moreover, the fibers of this map are open subsets of the projective space $\PP V$, where $V$ is the quotient of the space of all cubic polynomials modulo cubics containing $Q$, 
$$
V = \frac{H^0(\cO_{\PP^3}(3))}{H^0(\cI_{Q/\PP^3}(3))}.
$$
Since this vector space has dimension 16, the fibers of $\alpha$ are irreducible of dimension 15, and we deduce that the space $\cH^\circ_{4,3,6}$ is irreducible of dimension 24.

See Exercise~\ref{second complete intersection exercise} for the generalization to arbitrary smooth complete intersections in $\PP^3$.

\subsection{Genus 3}
We leave this to the reader in Exercises~\ref{6,3:1}, \ref{6,3:2}


\section{Degree 7}

Using the tools above, we invite the reader to show that each component of the restricted Hilbert schemes of curves of degree 7
in $\PP^3$ has expected dimension 28. See Exercise~\ref{degree 7 analysis} for an outline.

 
\section{The expected dimension of $\cH^\circ_{g,r,d}$}\label{chi N}


The sharp-eyed reader will have noticed that in, every case analyzed so far,  the Hilbert scheme
$\cH^\circ_{g,3,d}$ parametrizing smooth curves of degree $d$ and genus $g$ in $\PP^3$ has dimension $4d$. 
 This is the ``expected dimension,"  in a sense we will now make precise:


Let $C\subset \PP^r$ be a smooth curve of genus $g$ and degree $d$. In Section~\ref{tan hilbert section}
we computed the tangent space to $\cH^\circ_{g,r,d}$ at the point $[C]$ as $H^0(\sN_{C/\PP^r})$, so 
the dimension $h^0( \sN_{C/\PP^r})$ is an upper bound for $\dim \cH^\circ_{g,r,d}$. We can compute a lower bound as well:


\begin{fact}\label{deformation bound}
The completion of the local ring 
$$
(R,\gm) = \CC[[ x_1,\dots, x_t]]/J
$$ 
of $\cH^\circ_{g,r,d}$ at the point $[C]$ representing a smooth curve can, in principle, be computed by deformation theory.
Though this can actually be carried out in small cases, it is hard to get much qualitative information from the process
except for two numbers: the dimension of the Zariski tangent space $(\gm/\gm^{2})^{*}$, which we have computed as 
$t:= h^0(\sN_{C/\PP^r})$; 
and an upper bound for the number of generators of the ideal $J$, which is 
$h^1(\sN_{C/\PP^r})$. See
\cite[Corollaries 6.2.5, 6.4.11 and Proposition 6.5.2]{MR2223408}, where a similar result is given for any
locally complete intersection subscheme of $C\subset \PP^r$.

Thus from the principal ideal theorem~\cite[Theorem 10.2]{Eisenbud1995} it follows that if $C$ is a smooth curve, then 
$$
\chi(\sN_{C/\PP^r}) \leq \dim (\cH^\circ_{g,r,d})\leq h^0(\sN_{C/\PP^r}) \hbox{ locally at } [C].
$$
If the upper bound is achieved, then $[C]$ is a smooth point of $\cH^\circ_{g,r,d}$, and if the lower bound is achieved
then the local ring of $\cH^\circ_{g,r,d}$ at $[C]$ is a complete intersection.
\end{fact}

Using the deformation bound~\ref{deformation bound} we can at least control the Hilbert scheme at
a point representing a nonspecial embedding. 

\begin{theorem}\label{nonspecial Hilbert}
For any smooth curve $C\subset \PP^{r}$ of genus $g$ and degree $d$
$$
\chi(\cN_{C/\PP^{r}}) = (r+1)d - (r-3)(g-1),
$$
which is a lower bound for the dimension of the Hilbert scheme locally at $[C]$. 
If $\sO_{C}(1)$ is nonspecial, 
 then $H^1(\sN_{C/\PP^r})=0$ so
 $\cH^\circ_{g,r,d}$ is smooth and of dimension $ (r+1)d - (r-3)(g-1)$ at $[C]$.
 If $d>2g-2$,
then $\cH^\circ_{g,r,d}$ is irreducible as well.
\end{theorem}

Note that in the case of $\PP^3$ we have $\chi(\sN_{C/\PP^r}) = 4d$, and we have
seen that when $d\leq 7$ this is always equal to the dimension of the restricted Hilbert
scheme, whether or not the embedding is special. In view of this,
we define the \emph{expected dimension} of $\cH^\circ_{g,r,d}$ to be
$$
h(g,r,d) := (r+1)d - (r-3)(g-1).
$$
We will give yet another argument for calling this the expected dimension in Section~\ref{estimating dim hilb}.

\begin{proof} 
From the exact sequence
$$
0 \to \sT_C \to \sT_{\PP^r}|_C \to \sN_{C/\PP^r} \to 0
$$
we see that $\chi(\sN_{C/\PP^r}) = \chi(\sT_{\PP^r}|_C) - \chi(\sT_C)$
and that if $H^1( \sT_{\PP^r}|_C) = 0$ then $\chi(\sN_{C/\PP^r}) = h^0(\sN_{C/\PP^r}).$
Since $\sT_C = \omega_C^{-1}$, the Riemann-Roch theorem gives $\chi(\sT_C) = -3g+3$.

To compute $\chi(\sT_{\PP^r}|_C)$ we restrict the Euler sequence
$$
0\to \sO_{\PP^r} \to \sO_{\PP^r}(1)^{r+1} \to \sT_{\PP^r} \to 0
$$
to $C$.
Using the Riemann-Roch theorem again we deduce that
$$
\chi( \sT_{\PP^r} ) = (r+1)\chi(\sO_C(1)) - \chi(\sO_C) = (r+1)(d-g+1) - (1-g) = (r+1)d + r(1-g)
$$
From the restriction of the sequence above we also see that
if $\sO_C(1)$ is nonspecial then $H^1(\sT_{\PP^r}\mid_C) = 0$.

Putting these values together, we get
$
\chi(\sN_{C/\PP^r}) = (r+1)d - (r-3)(g-1)
$
as required.

Whenever $\dim \cH^\circ_{g,r,d} = h^{0}(\cN_{C/\PP^{r}}) $ the Hilbert scheme is smooth at $C$ by
Theorem~\ref{tangent space of Hilb}. 
From the deformation theory argument
in~\ref{deformation bound} we know that this dimension equality is true for any
nonspecial embedding since in this case 
$H^1(\sN_{C/\PP^r}) = 0$, so the upper and lower estimates for the dimension
coincide. If $d>2g-2$, then every curve in $\cH^\circ_{g,r,d}$ is nonspecial.

We can also prove the dimension statement for nonspecial embeddings invoking only the existence of the relative Picard scheme from
Chapter~\ref{JacobianChapter}: 
If $\sO_C(1)$ is nonspecial, then the
nonspecial invertible sheaves of degree $d$ on curves of genus $g$
 form an open subset of the $(3g-3+ g)$-dimensional relative Picard scheme. The dimension
 of the space of sections is $d-g+1$, so the family of $r$-dimensional linear series associated to
 each invertible sheaf is the dimension of the Grassmannian, $\dim G(r+1, d-g+1) = (r+1)(d-g-r)$,
 and the choice of a basis of the linear series, up to scalars, adds $(r+1)^2-1 = \dim PGL_{r+1}$. 
 Thus the dimension of  $\cH^\circ_{g,r,d}$ near $[C]$ is
$$
3g-3+ g + (r+1)(d-g-r) + (r+1)^2-1 = (r+1)d - (r-3)(g-1)
$$
as required. 

The irreducibility of the Hilbert scheme in the case $d>2g-2$ also follows from this argument 
together with the existence of a connected family containing all curves of genus $g$, for example over the
Hilbert scheme of tricanonical curves.
\end{proof}


\section{Some open problems}\label{open problems}

\subsection{Brill-Noether in low codimension}
 
If $C\subset \PP^r$ is a nonspecial embedding, then the isomorphism classes of the curves represented
in the component of $\cH^\circ_{d,g,r}$ containing $[C]$ is open in $M_g$. We shall see in Section~\ref{estimating dim hilb} that every component of $\cH^\circ_{d,g,r}$  dominating $M_g$ in this sense has dimension $h(g,r,d)$. 
What about ``smaller'' components?
Observations suggest that components of $\cH^\circ$ whose images in $M_g$ have low codimension still have the expected dimension $h(g,r,d)$: among the examples we know of components of the Hilbert scheme whose dimension is strictly greater than the expected $h(g,r,d)$, there are none whose image in $M_g$ has codimension less than $g-4$. 

\begin{conjecture}\label{large rho hilb dimension}
If $\cK \subset \cH^\circ_{d,g,r}$ is any component of a restricted Hilbert scheme, and the image of $\cK$ in $M_g$ has codimension $\leq g-4$, then $\dim \cK = h(g,r,d)$.
\end{conjecture}

A few cases of this conjecture are known; see~\ref{Hilb with rho geq -2}

\subsection{Maximally special  curves} 
How special can a linear series on a special curve be?

To make such a question precise, let $\widetilde M^r_{g,d} \subset M_g$ be the closure of the image of the map $\phi : \cH^\circ_{g,r,d}\to M_g$ sending a curve to its isomorphism class. We ask,
\begin{enumerate}
\item What is the smallest possible dimension of $\cH^\circ_{g,r,d}$? 
\item What is the smallest possible dimension of $\widetilde M^r_{g,d}$? and
\item Modifying the last question slightly, let $M^r_{g,d} \subset M_g$ be the closure of the locus of curves $C$ that possess a $g^r_d$ (in other words, we are dropping the condition that the $g^r_d$ be very ample). We can ask what is the smallest possible dimension of $M^r_{g,d}$?
\end{enumerate}

One might guess that the most special curves, from the point of view of questions 2 and 3, are hyperelliptic curves. A hyperelliptic curve is determined by a set of $2g+2$ points in $\PP^{1}$, modulo the action of $PGL_{2}$, so the locus in $M_g$ of hyperelliptic curves has dimension $2g-1$. Smooth plane curves are a better guess  --  the locus in $M_g$ of smooth plane curves has dimension asymptotic to $g$ (Exercise~\ref{moduli of plane curves}) -- but there are still a lot of them.

Can we do better?  Consider the locus of smooth complete intersections of two surfaces of degree $m$ in $\PP^3$. (Exercise~\ref{balanced CI}  suggests why we are choosing complete intersections of surfaces of the same degree.) As we saw in Exercise~\ref{complete intersection open}, these comprise an open subset $\cH^\circ_{ci}$ of the Hilbert scheme of curves of degree $d = m^2$, and genus $g$ given by the relation
$$
2g-2 = \deg K_C = m^2(2m-4),
$$
or, asymptotically,
$$
g \sim m^3.
$$

Moreover, the dimension of this component of the Hilbert scheme is easy to compute, since as we saw in Exercise~\ref{first complete intersection exercise} that it is isomorphic to an open subset of the Grassmannian $G(2, \binom{m+3}{3})$, and so has dimension
$$
2\left(\binom{m+3}{3} - 2\right) \; \sim \; \frac{m^3}{3}.
$$

To compute the dimension of the fibers of the map from this Hilbert scheme to $M_{g}$ we use 
the facts that if $C \subset \PP^r$ is a complete intersection curve of genus $g >1$ then the canonical bundle $K_C$ is a positive power of $\cO_C(1)$, and by Exercise~\ref{ci is acm}  $C$ is arithmetically Cohen-Macaulay.
For a given abstract curve $C$ there are only finitely many invertible sheaves having the canonical sheaf as a power; and since an arithmetically Cohen-Macaulay curve is necessarily embedded by a complete linear series, there are only finitely many embeddings of a given curve as a complete intersection, up to $PGL_{r+1}$. Thus the fibers of $\cH^\circ_{ci}$ over $M_g$ have dimension $\dim(PGL_{r+1}) = r^2 + 2r$.

This construction gives us a sequence of components of the restricted Hilbert scheme $\cH^\circ_{g,3,d}$ whose images in $M_g$ have dimension tending asymptotically to $g/3$.

More generally, we can consider complete intersections of $r-1$ hypersurfaces of degree $m$ in $\PP^r$. Such curves have
genus $g = m^{r-1}((r-1)m-r-1)/2 +1$ in a similar fashion we can calculate that their images in $M_g$ have dimension asympotically approaching $2g/r!= (m-1)(r-1)/r!$
 as $m \to \infty$, as we ask you to verify in Exercise~\ref{balanced CI in higher codim}.


These components have the smallest images in $M_{g}$ of any we know. To pose a precise question: if we fix $r$, can we find a sequence of components $\cH_n$ of  restricted Hilbert schemes  $\cH^\circ_{g_n,r,d_n}$ of curves in $\PP^r$ such that
$$
\lim \frac{\dim \cH_n}{g_n} \; = \; 0?
$$

\subsection{Rigid curves?}\label{rigid?}

In the last section, we considered components of the restricted Hilbert scheme whose image in $M_g$ was ``as small as possible." Let's go now all the way to the extreme, and ask: is there a component of the restricted Hilbert scheme $\cH^\circ_{g,r,d}$ whose image in $M_g$ is a single point? (Of course $M_0$ itself is a single point, so we exclude genus 0.) We can give three flavors of this question, in order of ascending preposterousness. Let $C \subset \PP^r$ be
a smooth irreducible nondegenerate curve.

\begin{enumerate} 
\item We call $C$ \emph{moduli rigid} if it lies in a component of the restricted Hilbert scheme whose image in $M_g$ is just the point $[C] \in M_g$---in other words, if the linear series $|\cO_C(1)|$ does not deform to any nearby curves.


\item We call $C$ \emph{rigid} if it lies in a component $\cH^\circ_{g,r,d}$ of the restricted Hilbert scheme such that $PGL_{r+1}$ acts transitively on $\cH^\circ_{g,r,d}$. This is saying that $C$ is moduli rigid, plus the line bundle $\cO_C(1)$ does not deform to any other $g^r_d$ on $C$.

\item We call $C$ \emph{deformation rigid} if the curve $C \subset \PP^r$ has no nontrivial infinitesimal deformations other than those induced by $PGL_{r+1}$---in other words, every global section of the normal bundle $\cN_{C/\PP^r}$ is the image of the restriction of a vector field on $\PP^r$.
\end{enumerate}

Do any such curves exist? This is not so much a question  as a howl of frustration. The existence of irrational rigid curves seems outlandish; we don't know anyone who thinks there are such things. But then, why can't we prove that they don't exist?



\section{Degree 8, genus 9}\label{degree 8 section}

So far, with the reader's presumed assistance, we have examined all the $\cH^\circ_{g,3,d}$ with $d\leq 7$
and found only irreducible varieties whose components have the expected dimension $4d = h(g,3,d)$. But it turns out that
this is not typical.

We start with an example of a component of $\cH^\circ_{9,3,8}$ whose dimension is strictly greater than $4d$.  Let $C$ be  a curve in this Hilbert scheme, and consider the restriction map
$$
\rho_2 : H^0(\cO_{\PP^3}(2)) \rTo H^0(\cO_C(2)).
$$
The source of $\rho_2$ has dimension 10. The Riemann-Roch theorem admits two possibilities for the dimension
of the target.
\begin{align*}
h^0(\cO_C(2)) =
\begin{cases}
9, \quad &\text{if } \cO_C(2) \cong K_C; \\
8,  \quad &\text{if } \cO_C(2) \not\cong K_C\,.
\end{cases}
\end{align*}
However, if $h^0(\cO_C(2))$ were 8 then $C$ would  lie on two distinct quadrics $Q$ and $Q'$. Since $C$ is nondegenerate, it cannot lie on a reducible quadric; thus $Q$ and $Q'$ would  be irreducible,  violating B\'ezout's theorem. We deduce that $\cO_C(2) \cong K_C$, and thus that $C$ lies on a unique quadric surface $Q$.

Similarly, since $\deg C > 2\cdot 3$, the curve $C$ cannot lie on any cubic not containing $Q$. Moving on to quartics, we look again at the restriction map
$$
\rho_4 : H^0(\cO_{\PP^3}(4)) \rTo H^0(\cO_C(4)).
$$
The dimensions here are, respectively, 35 and $4\cdot 8 - 9 + 1 = 24$; and we deduce that $C$ lies on at least an 11-dimensional vector space of quartic surfaces. On the other hand, only a 10-dimensional vector subspace of these vanish on Q. Thus $C$ lies on a quartic surface not containing $Q$. It follows from B\'ezout's theorem and 
Lasker's theorem that  $C = Q \cap X$
and moreover, by Lasker's theorem, the homogeneous ideal of $C$ is generated by the forms defining $Q$ and $X$. Thus $\ker(\rho_4)$ has dimension exactly 11, and  $X$ is unique modulo quartics vanishing on $Q$.

From these facts it is easy to compute the dimension of  $\cH^\circ_{9,3,8}$: Associating to $C$ the unique quadric on which it lies gives a map $\cH^\circ_{9,3,8} \to \PP^9$ with dense image, and each fiber is an open subset of the projective space $\PP V$, where $V$ is the 25-dimensional vector space
$$
V = \frac{H^0(\cO_{\PP^3}(4))}{H^0(\cI_{Q/\PP^3}(4))}.
$$
By Exercise~\ref{hilb at a ci}, $\cH^\circ_{9,3,8}$ is generically smooth, as well.

In sum, we have proven:
\begin{proposition}
 The scheme $\cH^\circ_{9,3,8}$ of curves of genus  is generically smooth and irreducible of dimension 33---one larger than the ``expected'' $4d$.
\end{proposition}
This is a special case of Exercise~\ref{second complete intersection exercise}.
See Exercise~\ref{many large components} for a rich set of examples of components whose dimension
is $>h(g,r,d)$.


\section{Degree 9, genus 10}\label{deg9 section}

For an example of a restricted Hilbert scheme that is reducible, consider $\cH^\circ_{9,3,10}$, the
scheme of smooth irreducible curves of degree 9 and genus 10 in $\PP^{3}$.

\begin{proposition}\label{types of 10,3,9}
 The scheme $\cH^\circ_{10,3,9}$ of smooth, irreducible, nondegenerate curves $C \subset \PP^3$ of degree 9 and genus 10 has two irreducible components, each generically smooth of dimension 36, the expected dimension. One consists of the complete intersections of two cubics. The other consists of curves of type $(3,6)$ or $(6,3)$ on a smooth quadric surface. \end{proposition}

\begin{proof}
To describe a smooth irreducible nondegenerate curve $C$ of degree 9 and genus 10 in $\PP^{3}$  we look at the restriction maps $\rho_m: H^0(\cO_{\PP^3}(m)) \rTo H^0(\cO_C(m))$. The Riemann-Roch theorem admits the possibilities:
\begin{align*}
h^0(\cO_C(2)) =
\begin{cases}
10, \quad &\text{if } \cO_C(2) \cong K_C \; \text{(``type 1,") and } \\
9,  \quad &\text{if } \cO_C(2) \not\cong K_C  \; \text{(``type 2.")}
\end{cases}
\end{align*}
Unlike the situation in degree 8, both occur.

1. Suppose first that $C$ does not lie on any quadric surface (so that $C$ is of type 1), and consider the map $\rho_3 : H^0(\cO_{\PP^3}(3)) \to H^0(\cO_C(3))$. By the Riemann-Roch theorem, the dimension of the target is $3\cdot 9 - 10 + 1 = 18$, from which we conclude that $C$ lies on at least a pencil of cubic surfaces. Since $C$ lies on no quadrics, all of these cubic surfaces must be irreducible, and it follows by B\'ezout's theorem that the intersection of two such surfaces is exactly $C$. At this point, Lasker's theorem assures us that $C$ lies on exactly two cubics.

By Exercise~\ref{first complete intersection exercise} the space $\cH^\circ_1$ of curves of this type is an open subset of the Grassmannian $G(2,20)$ of pencils of cubic surfaces, which is irreducible of dimension $36 = 4d$. By Exercise~\ref{hilb at a ci}, $\cH^\circ_1$ is generically smooth.

2. Next, suppose that $C$ has type 2---that is, $C$ does lie on a quadric surface $Q \subset \PP^3$; let $\cH^\circ_2 \subset \cH^\circ_{10,3,9}$ be the locus of such curves. First, suppose that the quadric $Q$ is singular. Since $C$ is irreducible and nondegenerate,
$Q$ must be the cone over a conic, that is, the scroll $S(0,2)$. Since $\deg C = 9$ we must have 
$C\sim 4H+F$, where $H$ is the hyperplane and $F$ the ruling on $Q$. By Theorem~\ref{curves on a singular scroll}
the genus of such a curve is 12, not 10, a contradiction proving that the quadric $Q$ is smooth.

If $C$ has class $(a,b)$ on $Q$ then $a+b= 9, (a-1)(b-1) = 10$ has the solutions $(a,b) = (3,6)$ or $(6,3)$.

We can show that $\cH^\circ_2$ is generically smooth by computing its tangent space
$H^0(\sN_{C/\PP^3})$. We start with the normal bundle sequence
$$
0\to \sN_{Q/\PP^3}\mid_C \to \sN_{C/\PP^3} \to \sN_{C/Q} \to 0
$$
and note that, by the adjunction theorem $\omega_C = \sO_Q(1,4)\mid_C$.
Thus both $\sN_{Q/\PP^3}|_C = \sO_Q(2,2)|_C$ and
$\sN_{C/Q} = \sO_Q(3,6)|_C$ are nonspecial, so 
$$
h^0 (\sN_{C/\PP^3}) = h^0(\sN_{\sO_Q(2,2)}|_C ) + h^0(\sO_Q(3,6)|_C)
= 10-1 + (4\cdot 7-1) = 36
$$

We outline two proofs that 
$\cH^\circ_2$ is irreducible in Exercise~\ref{degree 9 type 2 is irreducible}.
\end{proof}

These examples are far from exhausting the possibilities for the schemes $\cH^{\circ}_{g,3,d}$. For example
the scheme $\cH^{\circ}_{14,3,24}$ has 3 components, one of which is everywhere non-reduced~\cite{Mumford1962} and
\cite{Nasu2008}.


\section{Estimating the dimension of the restricted Hilbert schemes using the Brill-Noether theorem}\label{estimating dim hilb}

The Brill-Noether theorems lead to an understanding of at least one component of $\cH^{\circ}_{g,r,d}$ when
the Brill-Noether number is nonnegative:

\begin{theorem}\label{principal component}
Let $g, d$ and $r$ be any nonnegative integers such that the Brill-Noether number  $\rho(g,r,d) = g - (r+1)(g-d+r) \geq 0$
and $r\geq 3$.  There is a unique component $\cH_0$ of the restricted Hilbert scheme $\cH^\circ_{g,r,d}$ dominating the moduli space $M_g$; and this component has the ``expected dimension''
$$
\dim \cH_0 = h(g,r,d).
$$
\end{theorem}

 The component $\cH_0$ identified in Theorem~\ref{principal component} is called the \emph{principal component} of the Hilbert scheme; there may be others as well, of possibly different dimension, and we do not know precisely for which $d,g$ and $r$ these occur. In case $\rho < 0$, the Brill-Noether theorem tells us that there is no component of $\cH^\circ_{g,r,d}$ dominating $M_g$

%$3g-3+\rho + (r+1)^2 - 1 = 4g-3 + (r+1)(d-g+1) - 1.$

\begin{proof}
Consider the spaces
$$
\cH^\circ_{g,r,d}  
\rTo 
\cP_{d,g} = \{(C,\cL) \mid \cL \in \Pic_d(C) \} 
\rTo 
M_g.
$$
Starting from the right, we compute dimensions:

\begin{enumerate}

\item[$\bullet$]  $M_g$ is irreducible of dimension $3g-3$. Let $C$ be a general curve of genus $g$.

\item[$\bullet$]  
The Brill-Noether theorem tells us that the variety $W^r_d(C)$ has dimension $\rho$, and the general point of $W^r_d(C)$ corresponds to a very ample line bundle with exactly $r+1$ sections. 
If $\rho>0$ then $W^r_d(C)$ is irreducible, while if $\rho = 0$ then the monodromy action on $W^r_d(C)$
is transitive.

\item[$\bullet$] Therefore, over a general point of $W^r_d(C)$ the fiber of $\cH^\circ_{g,r,d} $ is
isomorphic to $PGL_{r+1}$. Thus there is a unique component of $\cH^\circ_{g,r,d}$ dominating
$W^r_d(C)$ and therefore a unique component of $\cH^\circ_{g,r,d}$ dominating $M_g$.

\item[$\bullet$] Adding up the dimensions of base and fiber, we see that the dimension
of this unique component is $(3g-3)+\rho(g,r,d) +((r+1)^2-1)$,
and arithmetic shows that this is $h(g,r,d)$.
\end{enumerate}
\end{proof}

\begin{fact}\label{Hilb with rho geq -2}
 Although when $\rho(g,r,d)<0$ there is no component of $\cH^{\circ}_{g,r,d}$ dominating moduli, one could hope
that when $\rho$ is not very negative one could understand a component dominating a subset of $M_{g}$ of codimension $-\rho$, leading to a proof of Conjecture~\ref{large rho hilb dimension} in such cases.
Indeed, this can be done when $\rho(g,r,d)\geq -2$: In~\cite{BrillNoether-1}, it is shown that if $\Sigma \subset M_g$ is any subvariety of codimension 1, then the curve $C$ corresponding to a general point of $\Sigma$ has no linear series with Brill-Noether number $\rho < -1$; and Edidin in ~\cite{Edidin} proves the analogous (and much harder) result for subvarieties of codimension 2.
\end{fact}


\section{Exercises}
\begin{exercise}\label{twisted cubic normal bundle}
Let $C \cong \PP^1 \subset \PP^3$ be a twisted cubic. Show that the normal bundle $\cN_{C/\PP^3} \cong \cO_{\PP^1}(5)^{\oplus 2}$, the direct sum of two line bundles of degree 5. Use this to prove that the restricted Hilbert scheme $\cH^\circ$ of twisted cubics is everywhere smooth. \end{exercise}

Hint: Restricting the presentation matrix of $I_C$ to $C$---that is, substituting the forms of degree 3 in 2 variables for the variables in the presentation matrix---we get a presentation 
$$
R(-9)^2 \rTo^A R(-6)^3 \to I_C/I_C^2 \to 0 
$$
The kernel of the dual of $A$ is the normal bundle; it has 2 linear generators, 
$$
\begin{pmatrix}
t&0\\
-s&t\\
0&-s
\end{pmatrix}
$$
as a module over $\CC[s,t].$


\begin{exercise}\label{hilb intersection}
Show that the locus $\Sigma$ of schemes $X$ consisting of a nodal plane cubic curve $C$ with a spatial embedded point of multiplicity 1 at the node is dense in the intersection $\overline{\cH^\circ} \cap \overline{\cH'}$.
\end{exercise}

\begin{exercise}
 Compute the dimension of each of the following subsets of $Hilb_{3m+1}$:
 
\begin{enumerate}
 \item unions of a conic and a line meeting it in 1 point.
 \item the connected union of 3 lines not all contained in the same plane
 \item nodal plane cubics together with an embedded point at the node that is not contained in the plane of
 the cubic.
\end{enumerate}
\end{exercise}

\begin{exercise}
Give an argument for Proposition~\ref{dimension of rational curves} in case $d=4$ using linkage. 
\end{exercise}
\begin{exercise}
Let $C \cong \PP^1 \subset \PP^3$ be a smooth rational curve of any degree $d$. 
\begin{enumerate}
\item Show that $h^1(\cN_{C/\PP^3}) = 0$; that is, the normal bundle of $C$ is nonspecial.
\item Using this, the Riemann-Roch formula for vector bundles on a curve and Proposition~\ref{dimension of rational curves}, show that the Hilbert scheme $\cH$ is smooth at the point $[C]$.
\end{enumerate} 
\end{exercise}

\begin{exercise}\label{hilb 1,3,4}
Let $C = Q \cap Q' \subset \PP^3$ be a smooth curve of degree 4 and genus 1. Identify the normal bundle $\cN_{C/\PP^3}$ of $C$, and use this to conclude that $\cH^\circ_{1,3,4}$ is itself reduced, and even smooth, and thus isomorphic to an open subset of the Grassmannian $G(2,10)$.
\end{exercise}

\begin{exercise}\label{complete intersection open}
Let $m \geq n >0$ be two positive integers. Show that the locus $U_{m,n} \subset \cH^\circ$ of curves $C \subset \PP^3$ that are smooth complete intersections of surfaces of degrees $m$ and $n$ is an open subset of the Hilbert scheme.
\end{exercise}

\begin{exercise}\label{first complete intersection exercise}
Consider  the locus $U_{n,n} \subset \cH^\circ$ of curves $C \subset \PP^3$ that are smooth complete intersections of two surfaces of degrees $n$. Show that $U_{n,n}$ 
is isomorphic to an open subset of the Grassmannian $G(2, H^0(\cO_{\PP^3}(n)))$.
\end{exercise}

\begin{exercise}
Let $C \subset \PP^3$ be a smooth curve of degree 5 and genus 2, and assume that the quadric surface $Q$ containing $C$ is smooth. From the exact sequence
$$
0 \to \cN_{C/Q} \to  \cN_{C/\PP^3} \to  \cN_{Q/\PP^3}|_C \to 0,
$$
calculate $h^0(\cN_{C/\PP^3})$ and deduce that \emph{$\cH^\circ_{2,3,5}$ is smooth at the point $[C]$}. Does  this conclusion still hold if $Q$ is singular?
\end{exercise}

\begin{exercise}\label{quintics genus 1}
Show that a smooth, irreducible, nondegenerate curve $C \subset \PP^3$ of degree 5 and genus 1 is residual to a rational quartic in the complete intersection of two cubics, and use the result of subsection~\ref{degree 4 genus 0} to deduce that the space of genus 1 quintics is irreducible of dimension 20.
\end{exercise}

\begin{exercise}
\begin{enumerate}
\item Show that all genera $g \leq 4$  occur among curves of degree 6 in $\PP^3$; that is, there exists a smooth irreducible, nondegenerate curve of degree 6 and genus $g$ in $\PP^3$ for all $g \leq 4$.
\item What is the largest possible genus of a smooth irreducible, nondegenerate curve $C \subset \PP^3$ of degree $d=7$? Can you do this with Clifford and Riemann-Roch, or do you need to invoke Castelnuovo?
\end{enumerate}
\end{exercise}

\begin{exercise}\label{hilb at a ci}
Show that the complete intersection locus in the Hilbert scheme is open and smooth.
If $X\subset \PP^r$ is a codimension $c$ complete intersection of forms of degrees $d_1,\dots d_c$,
Show that the dimension of the Hilbert scheme at $X$ is $\sum_{i = 1}^c h^0(\sO_X(c_i))$.
 
\end{exercise}
\begin{exercise}\label{second complete intersection exercise}
As before, let $U_{m,n} \subset \cH^\circ$ be the locus of curves $C \subset \PP^3$ that are smooth complete intersections of surfaces of degrees $m$ and $n$.
 In case $m > n$, show that $U_{m,n}$ is isomorphic to an open subset of a projective bundle over the projective space $\PP(H^0(\cO_{\PP^3}(n))) \cong \PP^{\binom{n+3}{3}-1}$ of surfaces of degree $n$, with fiber over the point $[S] \in \PP(H^0(\cO_{\PP^3}(n)))$ the projective space $\PP(H^0(\cO_{\PP^3}(m)))/H^0(\cI_{S/\PP^3}(m)) \cong \PP^{\binom{m+3}{3} - \binom{m-n+3}{3} - 1}$.
 
 Note that  by Exercise~\ref{hilb at a ci}  the scheme $\cH^\circ$ is smooth.
\end{exercise}

\begin{exercise}\label{6,3:1}
Let $C$ be a curve of degree 6 and genus 3, and assume that $C$ does not lie on any quadric surface. Show that $C$ is residual to a twisted cubic in the complete intersection of two cubic surfaces, and use this to deduce that the space of such curves is irreducible of dimension 24.
\end{exercise}


\begin{exercise}\label{6,3:2}
Now let $C$ again be a curve of degree 6 and genus 3, but now assume that $C$ \emph{does} lie on a quadric surface $Q$. Show that such a curve is a flat limit of curves of the type described in the last exercise, and conclude that $\cH^\circ_{3,3,6}$ is irreducible of dimension 24. (Hint: Let $L$, $Q$ and $F$ denote a general linear form, a general quadratic form and a general cubic form, and consider the pencil of surfaces $S_t = V(tF + LQ) \subset \PP^3$ specializing from the cubic surface $V(F)$ the to reducible cubic $V(LQ)$.)
\end{exercise}

\begin{exercise}
By analyzing the geometry of linear series of degrees $2g-1$ and $2g$ on a curve of genus $g$, extend Proposition~\ref{nonspecial Hilbert} to the cases $d = 2g-1$ and $2g$. What goes wrong if $d \leq 2g-2$?
\end{exercise}

\begin{exercise}\label{degree 7 analysis}
Show that each component of the restricted Hilbert schemes of curves of degree 7
in $\PP^3$ has expected dimension 28:
\begin{enumerate}
 \item $g\geq 6$
By Castelnuovo's theorem, the largest possible genus of a curve of degree 7 in $\PP^3$ is 6, and 
by Theorem~\ref{Castelnuovo examples} or a comparison of the Hilbert polynomial
with the Hilbert polynomial of $\PP^3$, these lie on a quadric.
 \item $g=5$: Show that there is a unique component, and it dominates the moduli space of curves of genus 5.
 Show that such a  curve is residual to a conic in the complete intersection of two cubics.
 \item $g\leq 4$: Since these curves are nonspecial,  there is a unique irreducible component.
\end{enumerate}
\end{exercise}


\begin{exercise}\label{bigger component}(Iarrobino)
Show that ideals generated by $a$ independent forms of degree $d$ together with all the forms of degree $d+1$ in 3 variables
form a family of 0-dimensional schemes of degree $d:={3+d\choose 3} -a$ which can be interpreted as a subvariety
of the Hilbert scheme $Hilb_d(\PP^3)$. Show that the dimension of this subvariety is $h := a({2+d\choose 2}-a)$. Find values of
$d$ and $a$ such that $h>3d$, and conclude that the Hilbert scheme $Hilb_d(\PP^3)$ has more than one component.
\end{exercise}

%%%%%Exercises originally in Ch 19:
 The next three exercises refer to the discussion of the two types of curves of degree 9 and genus 10 in Section~\ref{deg9 section}:
\begin{exercise}\label{degree 9 type 2 is irreducible}
In Section~\ref{deg9 section} we identified a locus $\cH^\circ_2$ of degree 9, genus 10 curves lying on a quadric.
One can show that $\cH^\circ_2$  is irreducible by a monodromy argument using Example~\ref{monodromy of rulings}, but one can also prove it via a liaison:  a curve in $\cH^\circ_2$ is residual to a union of three skew lines in the intersection of a quadric and a sextic surface. Use this to establish that $\cH^\circ_2$ is irreducible.
\end{exercise}

\begin{exercise}
 In this and the next exercise we refer to the classification of curves of degree 9 and genus 10 in $\PP^3$ given in Proposition~\ref{types of 10,3,9}. We used a dimension count to conclude that a general curve of type 1 could not be a specialization of a curve of type 2, and vice versa. Prove these assertions directly: specifically, argue that
\begin{enumerate}
\item by upper-semicontinuity of $h^0(\cI_{C/\PP^3}(2))$, argue that a curve $C$ not lying on a quadric cannot be the specialization of curves $C_t$ lying on quadrics; and
\item show that for a general curve of type $(3,6)$ on a quadric, $K_C \not\cong \cO_C(2)$, and deduce that a general curve of type 2 is not a specialization of curves of type 1.
\end{enumerate}
\end{exercise}

\begin{exercise}
Let $\Sigma_1$ and $\Sigma_2 \subset \cH^\circ_{10,3,9}$ be the loci of curves of types 1 and 2 respectively. 
\begin{enumerate}
\item What is the intersection of the closures of $\Sigma_1$ and $\Sigma_2$ in $ \cH^\circ_{10,3,9}$?
\item What is the intersection of the closures of $\Sigma_1$ and $\Sigma_2$ in the whole Hilbert scheme $\cH_{9m-9}(\PP^3)$?
\end{enumerate}
\end{exercise}


\begin{exercise}\label{many large components}
Let $\cH^\circ$ be a component of the Hilbert scheme parametrizing curves of degree $d$ and genus $g$ in $\PP^3$ that dominates the moduli space $M_g$. For $s, t \gg d$, let $\cK^\circ$ be the family of smooth curves residual to a curve $C \in  \cH^\circ$ in a complete intersection of surfaces of degrees $s$ and $t$.
\begin{enumerate}
\item Show that $\cK^\circ$ is open and dense in a component of the Hilbert scheme of curves of degree $st-d$ and the appropriate genus.
\item Calculate the dimension of $\cK^\circ$, and in particular show that it is strictly greater than $h(g,r,d)$.
\end{enumerate}
\end{exercise}

\begin{exercise}\label{moduli of plane curves}
\begin{enumerate}
\item Let $C \subset \PP^2$ be a smooth plane curve of degree $d\geq 4$. Show that the $g^2_d$ cut by lines on $C$ is unique; that is, $W^2_d(C)$ consists of one point.
\item Using this, find the dimension of the locus of smooth plane curves in $M_g$.
\end{enumerate}
\end{exercise}

\begin{exercise}\label{balanced ci}
Consider the locus of curves $C \subset \PP^3$ that are complete intersections of a quadric surface and a surface of degree $m$. Show that these comprise components of the restricted Hilbert scheme, and that their images in moduli have dimension asymptotically approaching $g$ as $m \to \infty$.
\end{exercise}

\begin{exercise}\label{balanced CI in higher codim}
Consider the locus $\cH^\circ_{ci}$, in the Hilbert scheme $\cH^\circ$, of smooth, irreducible, nondegenerate curves $C \subset \PP^r$ that are complete intersections of $r-1$ hypersurfaces of degree $m$. 
\begin{enumerate}
\item Show that $\cH^\circ_{ci}$ is open in $\cH^\circ$;
\item Calculate the dimension of $\cH^\circ_{ci}$ (and observe that it is irreducible); and
\item Show that the dimension of the image of $\cH^\circ_{ci}$ in $M_g$ is asymptotically $2g/r!$ as $m \to \infty$
\end{enumerate}
\end{exercise}


%footer for separate chapter files

\ifx\whole\undefined
%\makeatletter\def\@biblabel#1{#1]}\makeatother
\makeatletter \def\@biblabel#1{\ignorespaces} \makeatother
\bibliographystyle{msribib}
\bibliography{slag}

%%%% EXPLANATIONS:

% f and n
% some authors have all works collected at the end

\begingroup
%\catcode`\^\active
%if ^ is followed by 
% 1:  print f, gobble the following ^ and the next character
% 0:  print n, gobble the following ^
% any other letter: normal subscript
%\makeatletter
%\def^#1{\ifx1#1f\expandafter\@gobbletwo\else
%        \ifx0#1n\expandafter\expandafter\expandafter\@gobble
%        \else\sp{#1}\fi\fi}
%\makeatother
\let\moreadhoc\relax
\def\indexintro{%An author's cited works appear at the end of the
%author's entry; for conventions
%see the List of Citations on page~\pageref{loc}.  
%\smallbreak\noindent
%The letter `f' after a page number indicates a figure, `n' a footnote.
}
\printindex[gen]
\endgroup % end of \catcode
%requires makeindex
\end{document}
\else
\fi



