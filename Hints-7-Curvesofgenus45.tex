%header and footer for separate chapter files

\ifx\whole\undefined
\documentclass[12pt, leqno]{book}
\usepackage{graphicx}
\usepackage{eps-to-pdf}
\input style-for-curves.sty
%\input sl-macros.sty
\usepackage{hyperref}
\usepackage{showkeys} %This shows the labels.
\usepackage{msribib}
\usepackage{pdfpages}
\usepackage{draftwatermark}
\SetWatermarkText{DRAFT:\ \today}
\SetWatermarkScale{2}
\SetWatermarkColor[gray]{0.9}

%\usepackage{SLAG,msribib,local}
%\usepackage{amsmath,amscd,amsthm,amssymb,amsxtra,latexsym,epsfig,epic,graphics}
%\usepackage[matrix,arrow,curve]{xy}
%\usepackage{graphicx}
%\usepackage{diagrams}
%
%%\usepackage{amsrefs}
%%%%%%%%%%%%%%%%%%%%%%%%%%%%%%%%%%%%%%%%%%
%%\textwidth16cm
%%\textheight20cm
%%\topmargin-2cm
%\oddsidemargin.8cm
%\evensidemargin1cm
%
%%%%%%Definitions
%\input preamble.tex
%\input style-for-curves.sty
%\def\TU{{\bf U}}
%\def\AA{{\mathbb A}}
%\def\BB{{\mathbb B}}
%\def\CC{{\mathbb C}}
%\def\QQ{{\mathbb Q}}
%\def\RR{{\mathbb R}}
%\def\facet{{\bf facet}}
%\def\image{{\rm image}}
%\def\cE{{\cal E}}
%\def\cF{{\cal F}}
%\def\cG{{\cal G}}
%\def\cH{{\cal H}}
%\def\cHom{{{\cal H}om}}
%\def\h{{\rm h}}
% \def\bs{{Boij-S\"oderberg{} }}
%
%\makeatletter
%\def\Ddots{\mathinner{\mkern1mu\raise\p@
%\vbox{\kern7\p@\hbox{.}}\mkern2mu
%\raise4\p@\hbox{.}\mkern2mu\raise7\p@\hbox{.}\mkern1mu}}
%\makeatother

%%
%\pagestyle{myheadings}

%\input style-for-curves.tex
%\documentclass{cambridge7A}
%\usepackage{hatcher_revised} 
%\usepackage{3264}
   
\errorcontextlines=1000
%\usepackage{makeidx}
\let\see\relax
\usepackage{makeidx}
\makeindex
% \index{word} in the doc; \index{variety!algebraic} gives variety, algebraic
% PUT a % after each \index{***}

\overfullrule=5pt
\catcode`\@\active
\def@{\mskip1.5mu} %produce a small space in math with an @

\title{A Chapter from ``The Practice of Algebraic Curves"}
\author{\copyright David Eisenbud and Joe Harris}
%%\includeonly{%
%0-intro,01-ChowRingDogma,02-FirstExamples,03-Grassmannians,04-GeneralGrassmannians
%,05-VectorBundlesAndChernClasses,06-LinesOnHypersurfaces,07-SingularElementsOfLinearSeries,
%08-ParameterSpaces,
%bib
%}

\date{\today}
%%\date{}
%\title{Curves}
%%{\normalsize ***Preliminary Version***}} 
%\author{David Eisenbud and Joe Harris }
%
%\begin{document}

\begin{document}
\maketitle

\pagenumbering{roman}
\setcounter{page}{5}
%\begin{5}
%\end{5}
\pagenumbering{arabic}
\tableofcontents
\fi



\chapter{Curvesofgenus45 exercises}\label{Curvesofgenus45 exercises}


\begin{exercise}\ref{ex7.1}
Let $C$ be a smooth projective non-hyperelliptic curve of genus 4, and $|D|$ a $g^1_3$ on $C$ (that is, a linear equivalence class $D$ of degree 3, with $r(D) = 1$). Show that the following are equivalent:
\begin{enumerate}
\item $D \sim K-D$
\item the multiplication map $\mu : H^0(D) \otimes H^0(K-D) \to H^0(K)$ fails to be surjective
\item the unique quadric $Q$ containing the canonical curve of $C$ is singular; and
\item $|D|$ is the unique $g^1_3$ on $C$.
\end{enumerate}
\end{exercise}

Hint: We have seen that a $g^1_3$ on $C$ is cut out by the lines of a ruling of the unique quadric containing the canonical model of $C$; this is unique iff the quadric is a cone. Likewise,  $D \sim K-D$ iff there is a hyperplane $H \subset \PP^3$ intersecting $C$ in the divisor $2D$, that is, tangent to $C$ at each of the three points of $D$; by the description of the Gauss map of a quadric, this will be the case iff the quadric is a cone. Finally, if $D \sim K-D$ and we write $H^0(D) = \langle \sigma, \tau \rangle$, then the kernel of $\mu$ contains the element $\sigma \otimes \tau - \tau \otimes \sigma$.

\begin{exercise}\ref{ex7.2}
Let $C$ again be a smooth projective non-hyperelliptic curve of genus 4. We have seen that $C$ is birational to a quintic plane curve $C_0$ with two nodes $p, q \in C_0$. Show that the canonical series of $C$ is cut out by the system of plane conic curves passing through $p$ and $q$ in the sense
that if $D$ is a curve meeting C at each node and transverse to each branch at the nodes, then
the sum of the points of $D\cap C$ \emph{other than the nodes} is a canonical divisor.
\end{exercise}

Hint: Let $S = Bl_{\{p,q\}}\PP^2$ be the blow-up of the plane at the two points $p$ and $q$, with $C$ the proper transform of $C_0$; let $E$ and $F$ be the exceptional divisors, and let $L$ be the class of a line in $\PP^2$. We have
$$
C \sim 5L -2E - 2F \quad \text{and} \quad K_S \sim -3L + E + F
$$
in the Picard group of $S$; applying adjunction, we see that $K_C \sim (2L - E - F)|_C$ and the statement follows.

\begin{exercise}\ref{ex7.3}
Let $C$  be a smooth projective non-hyperelliptic curve of genus 4 and let $D$ be a general divisor of degree 7 on $C$. By the $g+3$ theorem (Theorem~\ref{}), $h^0(D) = 4$ and the map $\phi_D : C \to \PP^3$ is an embedding. Show that the image $C \subset \PP^3$ does not lie on any quadric surfaces, but does lie on two cubic surfaces $S$ and $T$; describe the intersection $S \cap T$.
\end{exercise}

Hint: We can see that $C$ does not lie on any quadric surface $Q$ by the genus formula for curves on $Q$. Given this, the cubics $S$ and $T$ cannot have a component in common, and so must intersect in the union of $C$ and (by Bezout) a curve of degree 2. Using the adjunction formula for curves of $S$, we may conclude that $D$ is the union of two skew lines.

\begin{exercise}\ref{ex7.4}
In the setting of the preceding exercise, suppose now that $C$ \emph{is} hyperelliptic. Show that in this case the image of $C$ under the map $\phi_D : C \to \PP^3$ does lie on a quadric surface $Q$, and in fact is a curve of type $(2,5)$ on $Q$. Show also that if $D$ is of the form $D \sim 2g^1_2 + p + q + r$ then the quadric surface $Q$ is singular, and the image curve $\phi_D(C)$ has a triple point at the vertex of $Q$.
\end{exercise}

Hint: Show that each of the divisors $E$ of the $g^1_2$ span a line in $\PP^3$. In either case, the quadric surface $Q$ will be swept out by the 1-parameter family of lines spanned by the divisors $E$ of the $g^1_2$ on $C$. In the latter case, since $h^0(D-p-q-r) = 3$ the map $\phi_D$ will send the three points $p, q, r$ to the same point. (Note also that a smooth curve of degree 7 on a quadric cone cannot have genus 4.)

\begin{exercise}\ref{ex7.5}
In Chapter~\ref{}, we introduced a moduli space $M^r_{d,g}$ parametrizing $g^r_d$s on curves of genus $g$; that is,
$$
M^r_{d,g} = \{ (C, L) \mid C \text{ a smooth curve of genus } g, L \in \pic^d(C) \text{ and } h^0(L) \geq r+1 \}.
$$
The analysis of this chapter shows that $M^1_{3,4}$ is a two-sheeted cover of $M_4$. Show that it is in fact irreducible.
\end{exercise}

Hint: This will follow from the observation that, in the family of all quadric surfaces $Q \subset \PP^3$, the monodromy acts nontrivially on the two rulings of a smooth quadric $Q$. This will in turn follow from the fact that the incidence correspondence
$$
\Gamma := \{ (Q, L) \in \PP^9 \times \GG(1,3) \mid L \subset Q \}
$$
is irreducible, which we can see via projection on the second factor.

\begin{exercise}\ref{ex7.6}
The arguments in the chapter show that the canonical model of a non-hyperelliptic trigonal curve of genus 5 lies on an irreducible, nondegenerate cubic surface $S \subset \PP^4$. In Chapter~\ref{ScrollsChapter}, we'll see that such a surface is either smooth or a cone over a twisted cubic curve. Show that the latter case cannot occur.
\end{exercise}

Hint: Let $D = p + q + r$ be a divisor of the $g^1_3$ on $C$; that is, the intersection of $C$ with a line of the ruling of $S$. If $S$ were indeed a cone, the tangent plane to $S$ at the points $p, q$ and $r$ would be tangent to $C$ at all three points; that is, we would have $\dim \overline{2D} = 2$. By the geometric Riemann-Roch and Clifford's theorem, this would contradict the hypothesis that $C$ is not hyperelliptic.


\begin{exercise}\ref{ex7.7}
Let $C$ be a smooth projective curve of genus 5. The $g+3$ theorem (Theorem~\ref{}) says that $C$ admits an embedding in $\PP^3$ as a curve of degree 8. Does it admit an embedding of degree 7?
\end{exercise}

Hint: if $|D|$ is a $g^3_7$ on $C$, by Riemann-Roch it must be of the form $D \sim K - p$ for some point $p \in C$; that is, the map $\phi_D$ is the composition of the canonical map with projection from a point $p \in C$. If $C$ is trigonal, there is a unique divisor $D = p + q + r$ of the $g^1_3$ containing $p$, and the map $\phi_{K-p} : C \to \PP^3$ will map $q$ and $r$ to the same point, so it cannot be an embedding. Conversely, if $C$ is not trigonal, the canonical model is a complete intersection of three quadrics, and it follows that projection from any point is an embedding.


\begin{exercise}\label{non-red 4-tuples}\ref{ex7.8}
Complete the proof of Lemma~\ref{4-tuples}
\end{exercise}

\begin{exercise}\ref{ex7.9}
Verify that if $C$ is a nonhyperelliptic curve of genus 5 then the variety $W^1_4(C)$ is a curve of arithmetic genus 11.
\end{exercise}

%footer for separate chapter files

\ifx\whole\undefined
\makeatletter\def\@biblabel#1{#1]}\makeatother
\gdef\urlhook{\url}
\bibliography{slag}
\bibliographystyle{msribib}


%%%% EXPLANATIONS:

% f and n
% some authors have all works collected at the end

\catcode`\^\active
%if ^ is followed by 
% 1:  print f, gobble the following ^ and the next character
% 0:  print n, gobble the following ^
% any other letter: print letter
\makeatletter
\def^#1{\ifx1#1f\expandafter\@gobbletwo\else
        \ifx0#1n\expandafter\expandafter\expandafter\@gobble\else#1\fi\fi}
\makeatother
\let\moreadhoc\relax
\def\indexintro{%An author's cited works appear at the end of the
%author's entry; for conventions
%see the List of Citations on page~\pageref{loc}.  
%\smallbreak\noindent
The letter `f' after a page number indicates a figure, `n' a footnote.}
\printindex[gen]
%requires makeindex
\end{document}
\else
\fi
