%header and footer for separate chapter files

\ifx\whole\undefined
\documentclass[12pt, leqno]{book}
\usepackage{graphicx}
\usepackage{eps-to-pdf}
\input style-for-curves.sty
%\input sl-macros.sty
\usepackage{hyperref}
\usepackage{showkeys} %This shows the labels.
\usepackage{msribib}
\usepackage{pdfpages}
\usepackage{draftwatermark}
\SetWatermarkText{DRAFT:\ \today}
\SetWatermarkScale{2}
\SetWatermarkColor[gray]{0.9}

%\usepackage{SLAG,msribib,local}
%\usepackage{amsmath,amscd,amsthm,amssymb,amsxtra,latexsym,epsfig,epic,graphics}
%\usepackage[matrix,arrow,curve]{xy}
%\usepackage{graphicx}
%\usepackage{diagrams}
%
%%\usepackage{amsrefs}
%%%%%%%%%%%%%%%%%%%%%%%%%%%%%%%%%%%%%%%%%%
%%\textwidth16cm
%%\textheight20cm
%%\topmargin-2cm
%\oddsidemargin.8cm
%\evensidemargin1cm
%
%%%%%%Definitions
%\input preamble.tex
%\input style-for-curves.sty
%\def\TU{{\bf U}}
%\def\AA{{\mathbb A}}
%\def\BB{{\mathbb B}}
%\def\CC{{\mathbb C}}
%\def\QQ{{\mathbb Q}}
%\def\RR{{\mathbb R}}
%\def\facet{{\bf facet}}
%\def\image{{\rm image}}
%\def\cE{{\cal E}}
%\def\cF{{\cal F}}
%\def\cG{{\cal G}}
%\def\cH{{\cal H}}
%\def\cHom{{{\cal H}om}}
%\def\h{{\rm h}}
% \def\bs{{Boij-S\"oderberg{} }}
%
%\makeatletter
%\def\Ddots{\mathinner{\mkern1mu\raise\p@
%\vbox{\kern7\p@\hbox{.}}\mkern2mu
%\raise4\p@\hbox{.}\mkern2mu\raise7\p@\hbox{.}\mkern1mu}}
%\makeatother

%%
%\pagestyle{myheadings}

%\input style-for-curves.tex
%\documentclass{cambridge7A}
%\usepackage{hatcher_revised} 
%\usepackage{3264}
   
\errorcontextlines=1000
%\usepackage{makeidx}
\let\see\relax
\usepackage{makeidx}
\makeindex
% \index{word} in the doc; \index{variety!algebraic} gives variety, algebraic
% PUT a % after each \index{***}

\overfullrule=5pt
\catcode`\@\active
\def@{\mskip1.5mu} %produce a small space in math with an @

\title{A Chapter from ``The Practice of Algebraic Curves"}
\author{\copyright David Eisenbud and Joe Harris}
%%\includeonly{%
%0-intro,01-ChowRingDogma,02-FirstExamples,03-Grassmannians,04-GeneralGrassmannians
%,05-VectorBundlesAndChernClasses,06-LinesOnHypersurfaces,07-SingularElementsOfLinearSeries,
%08-ParameterSpaces,
%bib
%}

\date{\today}
%%\date{}
%\title{Curves}
%%{\normalsize ***Preliminary Version***}} 
%\author{David Eisenbud and Joe Harris }
%
%\begin{document}

\begin{document}
\maketitle

\pagenumbering{roman}
\setcounter{page}{5}
%\begin{5}
%\end{5}
\pagenumbering{arabic}
\tableofcontents
\fi


\chapter{Syzygies}
\label{SyzygiesChapter}

\section{Introduction} 
%Motivation: canonical embedding turns intrinsic invariants into projective invariants. Hilbert Function. Projective Normality; Canonical Module.
%
%What are the projective invariants that correspond to Clifford index? Conjecturally, Green's conjecture. Inequality from Eagon-Northcott.
%
%Hilbert Syzygy theorem, Hilbert function derivation, Unique minimal resolution, Betti table, 
%\fix{ introduce tools as they are used}

It is much easier to talk about the geometry of a curve once it is embedded in projective space, but of course this is really the geometry of a pair, the curve \emph{and} the embedding. However if we take the embedding corresponding to the complete linear series associated to the canonical line bundle (or a fixed multiple of it), then we can be sure we are talking about an intrinsic invariant of the curve itself. In genus 0,1 the canonical bundle is not ample, so this approach doesn't work; but in genus $\geq 2$,as we have seen in Chapter ****, the canonical bundle itself determines an embedding for all but hyperelliptic curves; the square of the canonical bundle gives an embedding for all  curves of genus $>2$; and the cube of the canonical bundle gives an embedding for every curve of genus $\geq 2$. 

Once a curve is embedded in projective space, the first algebraic invariant we have looked at is it's Hilbert function; but this simply recovers the genus and the degree. A much finer invariant is the minimal free resolution of the homogeneous coordinate ring of the curve. For example according to the conjecture of Mark Green (now verified in many special cases), the free resolution determines the ``smallest'' base point free linear series on the curve, in a precise sense that we will describe.

We begin by reviewing some basic facts about free resolutions. They are usually stated for
modules over a local ring, and in that form the results here can be found in \cite[Ch ?]{E}. The proofs are essentially identical in the graded case presented here.

Let 
$R = R_{0} \oplus R_{1}\oplus\cdots$ be a positively graded ring, with $R_{0}$ a field,
 and let $M$ be a finitely generated graded $S$-module. An  \emph{$R$-free resolution} of $M$ is a sequence of free graded $R$ modules $F_{i}$ and homogeneous degree 0 maps $d_{i}$ of the form
$$
\FF: F_{0}:=\oplus_{j}R(-j)^{\beta_{0,j}} \lTo^{d_{1}} F_{1}:=\oplus_{j}R(-j)^{\beta_{0,j}}\lTo^{d_{2}} F_{2}\cdots
$$
and an \emph{augmentation} map $F_{0} \rOnto^{d_{0}} M$ such that the kernel of $d_{i}$ is equal to the image of $d_{i+1}$ for every $i$. 

The resolution $\FF$ is \emph{minimal} if every entry of matrices representing the $d_{i}$ (for $i>0$ is of strictly positive degree. A minimal free resolution of $M$ can be constructed inductively by first choosing a minimal set of homogeneous generators of $M$, determining the map $d_{0}$, then choosing a minimal set of generators of $\ker d_{0}$, determining a map
$F_{1} \rOnto \ker d_{0} \subset F_{0}$, and so on.

The minimal free resolution can be used to derive invariants of $M$
\cite[Theorem ***]{E} by virtue of the following:

\begin{theorem}[Uniqueness]\label{uniqueness} With hypotheses as above, the minimal free resolution of of a finitely generated graded is unique up to (non-unique) isomorphism.
\end{theorem}

From now on in this section, we let $S = \CC[x_{0},\dots, x_{n}]$ be the homogeneous coordinate ring of $\PP^{n}$, and we consider only free resolutions of finitely generated graded $S$-modules, and consider resolutions over this ring. In this case something quite special happens \cite[****]{E}:

\begin{theorem}[Hilbert's Syzygy Theorem]\label{hst}
The minimal $S$-free resolution $\FF$ of any finitely generated graded $S$-module $M$ is finite: in fact, $F_{i}=0$ for all $i>n+1$.
\end{theorem}

\begin{fact}
 There are in fact finite resolutions for all $S$-modules; however there is generally no reasonable notion of a  minimal resolution outside the finitely generated graded or local cases. See \cite{???}.
\end{fact}

The numbers $\beta_{i,j}$ in  the minimal $S$-free resolution of $M$ are called the \emph{graded Betti numbers} of $M$; It follows from this that the graded Betti numbers, and all the invariants of the matrices $d_{i}$, are invariants of $M$. In particular since we know the Hilbert function of $S$, and thus of $S(-j)$,
$$
H_{S(-j)}(m) = 
\begin{cases}
 {n+m-j \choose n} & \hbox{for $t\geq 0$}
  \\ 
 0 & \hbox{for $t<0$}
\end{cases}
$$
we can compute the Hilbert function for a finitely generated graded $S$-module $M$ from its graded Betti numbers 
$$
H_{M}(m) = \sum_{i,j} (-1)^{i}\beta_{i,j}{n+m-j\choose n}.
$$
\begin{example} [The homogeneous coordinate ring of a plane curve] 
For the simplest non-trivial example, if $C$ is a plane curve with equation $F(x_{0},x_{1}x_{2})$ of 
degree $d$, then the minimal free resolution of the homogeneous coordinate ring $S_{C}$ has the form
$$
\FF: S\lTo^{F}S(-d)\lTo 0
$$
Thus the only nonzero graded Betti numbers of $S_{C}$ as an $S$-module are
 $\beta_{0,0} = \beta_{1,d} = 1$, so
$$
 H_{S_{C}}(m) = {2+m\choose 2} - {2-d+m\choose 2}
 $$
 which has the value  $dm + 1+(2-d)(2-d-1)/2 = dtm+ (d-1)(d-2)/2 +1$. Since the degree is the leading coefficient and the arithmetic genus is the constant term minus 1, we get
$\deg C = d, p_{a}(C) = (d-1)(d-2)/2$ as we should, by now, expect. Note that this method of computation is not sensitive at all to whether the curve is smooth and irreducible; it works for any divisor in $\PP^{2}$.
\end{example}

\begin{exercise} In Section **** we computed the genus of a smooth complete intersection curve inductively, by using the adjunction formula for curves on a smooth surface along with Bertini's Theorem.
Show that the same formula gives the arithmetic genus of an arbitrary 1-dimensional complete intersection $C$ hypersurfaces 
$X_{i} := \{F_{i} = 0\}$ of degrees $d_{i}$,
using the fact that the minimal free resolution of the homogeneous coordinate ring 
$S/(F_{1}, \dots, F_{n-1})$ of $C$  is the Koszul complex (see \cite{???}):
$$
 S \lTo \oplus_{1\leq i<n}S(-d_{j}) \lTo  \oplus_{1\leq j<k<n}S(-d_{j}-d_{k}) \cdots S(-\sum_{1\leq j<n}d_{j})\lTo 0.
 $$
\end{exercise}

Computing the Hilbert function was the application for which Hilbert proved the Syzygy Theorem; but if that were the only use of free resolutions, we would not be introducing it here. The first new invariant that we can read from the minimal $S$-free resolution of a module $M$ is its length; that is, the number of nonzero maps, which is finite by the Syzygy Theorem. This is called the \emph{projective dimension} of $M$ as an $S$-module, written $pd_{S}M$. 

\begin{proposition}
If $M$ is a graded $S$-module then $\pd(M)$ is at least the codimension of the support of $M$.
\end{proposition}

In case $\pd(M)$ is equal to the codimension of the support of $M$, we say that $M$ is a
Cohen-Macaulay $S$-module, or equivalently that the sheaf $\widetilde M$ is 
\emph{arithmetically Cohen-Macaulay}. When $M = S_{X}$, the homogeneous coordinate ring of a projective scheme $X$, we say that $X$ is itself. is arithmetically Cohen-Macaulay.

Thus if $C\subset \PP^{n}$ is 1-dimensional the projective dimension of $S_{C}$ is at least $n-1$. But we can be much more precise. Recall that a curve $C\subset \PP^{n}$ is said to be \emph{projectively normal} if the homogeneous coordinate ring of $C$ is integrally closed (which implies, in particular, that $C$ is smooth).

\begin{theorem}
 Let $C\subset \PP^{n}$ be a purely 1-dimensional subscheme. The projective dimension of the homogeneous coordinate ring $S_{C}$ of $C$  is
$$
 pd_{S}S_{C} = 
\begin{cases}
n-1 &\hbox{if $H^{1}(\sI_{C}(t)) = 0$ for all $t\in \ZZ$}\\
n &\hbox{otherwise}.
\end{cases}.
$$
Thus in the first case $S_{C}$ is Cohen-Macaulay. In particular, if $C$ is a smooth curve, the $pd_{S}(S_{C}) = n-1$ if and only if $C$ is projectively normal. \fix{we used Serre's Criterion. Ref?}
\end{theorem}

Here is a version that gives a measure of how far $S_C$ is  from being Cohen-Macaulay:

\begin{theorem}
Let $C\subset \PP^{n}$ be a purely 1-dimensional subscheme, and let  
$$
\FF: F_{0}\lTo^{d_{1}} F_{1}\lTo^{d_{2}} F_{2}\lTo \cdots \lTo F_{n-1}\lTo {d_{n}}F_{n}\lTo F_{n+1} \lTo 0
$$
be the minimal $S$-free resolution of the homogeneous coordinate ring of $C$. We have $F_{n+1}=0$, and 
$$
\oplus_{t\in \ZZ} (H^{1}\sO_{C}(t)) = \Hom_{\CC}(\Ext^{n}(S_{C}, S(-n-1)),\CC)
$$ 
which is sometimes called the \emph{Rao module} of $C$. Thus, up to a shift in grading,
the Rao module of $C$ is the vector space dual of the cokernel of the dual $d_{n}^{*}: F_{n-1}^{*}\to F_{n}$. This is a graded module of finite length.
\end{theorem}


\fix{it's not clear how much of the material here to prove; or, if we do prove it, whether we should introduce the general machinery of depth and admit that local cohomology makes the statements look cleaner. The rest of this section is a possible start on this...}

A famous result of Auslander and Buchsbaum clarifies the meaning of projective dimension. We define the 
\emph{depth} of $M$ to be the maximum length $\ell$ of a \emph{regular sequence on $M$}; that is, a sequence $G_{1},\dots,G_{\ell}$ of homogeneous forms of strictly positive degree such that 
\begin{align*}
G_{1} &\hbox{ is a nonzerodivisor on } M;\\
G_{2} &\hbox{ is a nonzerodivisor on } M/G_{1}M;\\
\vdots&\phantom{\hbox{ is a nonzerodivisor on } }\vdots\\
G_{\ell} &\hbox{ is a nonzerodivisor on } M/(G_{1},\dots,G_{\ell-1})M.
\end{align*}

\begin{theorem}
If $M$ is a finitely generated graded module over the polynomial ring $S := \CC[x_{0},\dots,x_{n}]$, and $M$ has depth $\ell$, then the projective dimension of $M$ is $n+1-\ell$.
\end{theorem}

Though this is not obvious from the definition, all maximal regular sequences on $M$ have the same length, and if the depth of $M$ is $\ell$ then a sequence of general linear forms of length $\ell$ is a regular sequence. This makes the depth easier to compute. Even better, the depth has an interpretation in terms of sheaf cohomology:

Suppose for simplicity that $M = \oplus_{t\in \ZZ}H^{0}\cO_{X}(t)$ for some scheme $X\subset \PP^{n}$ without 0-dimensional (isolated or embedded) components, so that $M$ is a finitely generated $S$-module. 
\begin{theorem}
 With hypotheses as above, $\depth M$ is the largest integer $\ell$ such that $H^{i}(\cO_{X}(t) = 0$ for all $0<i<ell-1$ and all 
\end{theorem}


\section {Dualizing sheaves} 

Other than  the structure sheaf, the most important line bundle on a smooth curve $C$  is the cotangent line bundle, usually called the canonical line bundle or canonical sheaf $\omega_{C}$, and  we have many times used it's key property, that if  $\cF$  is a line bundle on $C$, then $\HH^{0}(\sF^{-1}\otimes \omega_{C})$ is the vector space dual of $\HH^{1}(\cF)$. Because $\sF$ is locally free, we may rewrite the formula in the attractively symmetric form:
$$
\HH^{0}(\sHom_{C}(\sF, \omega_{C})) 
\cong 
\Hom_{k}(\HH^{1}(\cF), k).
$$
Setting $\sF = \omega_{C}$ we let $\eta: \HH^{1}(\omega_{C})\to k$ be the image of 
the identity map $\omega_{C} \to \omega_{C}$, and call $\eta$ the residue map corresponding to the isomorphism; and, inversely, any nonzero functional 
$\eta: \HH^{1}(\omega_{C})\to k$
determines, for every $\sF$ a map
$$
f_{\eta}: \HH^{0}(\sHom_{C}(\sF, \omega_{C}))
\to
\Hom_{k}(\HH^{1}(\cF), k).
$$
sending a homomorphism
$\alpha \in \HH^{0}(\sHom_{C}(\sF, \omega_{C})) = \Hom_{C}(\sF, \omega_{C})$
to $\eta \circ \HH^{1}(\alpha)$ so that, for example, $\eta$ itself corresponds to the 
identity map of $\omega_{C}$  Since $\HH^{1}(\omega_{C})$ is 1-dimensional, such a nonzero functional $\eta$ is unique up to scalars. (When $C$ is smooth over $\CC$, we could take $\eta$ to be the classical residue map of complex analysis.)

It turns out that in this form the result is true for \emph{any} coherent sheaf $\cF$, and moreover can be generalized to any purely 1-dimensional projective scheme (and beyond: see Cheerful Fact\ref{fact****}). By way of terminology, if $X$ is a purely $r$-dimensional projective scheme over the field $k$, we say that a coherent sheaf $\omega$ on $X$, together with a linear
functional $\eta:\HH^{r}(\omega) \to k$ is \emph{dualizing} if the map
$$
f_{\eta}: \HH^{0}(\sHom_{X}(\sF, \omega))
\to
\Hom_{k}(\HH^{r}(\cF), k).
$$
defined as above is an isomorphism for every coherent sheaf $\sF$.

We often abuse the terminology, and say simply that $\omega$ is a dualizing sheaf. If $X$ is reduced and connected, so that $\HH^{0} (\sHom(\sO_{X} , \sO_{X})) = k$, then
$\HH^{r}(\omega) \cong k$, whence $\eta$ is unique up to a scalar. 

\fix{I think the proof given below works for a pure-dimensional scheme $X$ of any dimension, as in the book of Altman-Kleiman. Put this in? The intermediate cohomology groups are dual to each other iff $X$ is Cohen-Macaulay, a cheerful fact.}

\begin{theorem}
Let $X$ be a purely 1-dimensional projective scheme. There is a coherent sheaf
$\omega$ on $X$ and a map $\eta_{X}: \HH^{1}(\omega) \to k$ such that
the corresponding maps
$$
f_{\eta}: \HH^{0}(\sHom_{X}(\sF, \omega))
\to
\Hom_{k}(\HH^{1}(\cF), k): \qquad \alpha \mapsto \eta\circ \HH^{1}(\alpha)
$$
are isomorphisms for any coherent sheaf $\sF$ on $X$. Moreover, this property determines the pair $(\omega_{X}, \eta)$ up to isomorphism in the sense that, given another such pair
$(\omega'_{X}, \eta')$,  there is a unique isomorphism $g: \omega\to \omega'$ making the diagram
$$
\begin{diagram}
 \omega&\rTo^{g}&\omega'\\
 &\rdTo_{\eta}&\dTo_{\eta'}\\
 &&k
\end{diagram}
$$
commute. 
\end{theorem}

Any  sheaf $\omega$ with the property given in the Theorem is called a \emph{dualizing sheaf} for $X$, sometimes denoted $\omega_{X}$.

\begin{proof}
 
We first consider the case $C=\PP^{1}$. If $\cF$ is any coherent sheaf, then considering the torsion subsheaf we see that there is an exact sequence
$$
0\to \cF' \to \cF \to \cF''\to 0
$$ 
where $\cF'$ has finite support and $\cF''$ is torsion free, and thus a sum of line bundles. Since
$\cF'$ is locally free, the sequence is locally split (in fact it is globally split too, but we don't need this.)

Since $\omega_{\PP^{1}} = \cO_{\PP^{1}}(-2)$ is a line bundle, 
$$
\sHom_{\PP^1}(\sF, \omega_{\PP^{1}}) = 
\sHom_{\PP^1}(\sF', \omega_{\PP^{1}}) \oplus \sHom_{\PP^1}(\sF'', \omega_{\PP^{1}})  = 
\sHom_{\PP^1}(\sF', \omega_{\PP^{1}}) 
$$
and 
$\HH^{1}(\cF) = \HH^{1}(\cF'')
$
because $\HH^{1}(\cF') = 0$, so the duality formula for arbitrary coherent sheaves follows from the case of line bundles.

Now let $X$ be a purely 1-dimensional projective scheme. To construct a pair
$(\omega_{X}, \eta)$ with the desired property, we choose a finite mapping 
$\pi: X\to \PP^{1}$ (for example by embedding $X$ in some projective space $\PP^{n}$ and
projecting from an $(n-2)$-plane disjoint from $X$). We define $\omega_{X}$ to be a sheaf
on $X$ satisfying
$$
\pi_{*}(\omega_{X}) = \sHom(\pi_{*}\cO_{X}, \omega_{\PP^{1}}).
$$
To see that there is such a sheaf, take an open affine cover $\{U_{i}\}$ of $\PP^{1}$ and pull it back to an open affine cover $\{V_{i} = \pi^{-1}(U_{i})\}$ of $X$. Because $\pi$ is finite, the restriction of 
$\pi_{*} \cO_{X}$ to $U_{i}$ is naturally isomorphic to $\sO_{V_{i}}$, regarded as an $\sO_{U_{i}}$-module, and thus the restriction of
$\sHom(\pi_{*}\cO_{X}, \omega_{\PP^{1}}$ to $U_{i}$ is 
$Hom_{U_{i}}(\cO_{V_{i}}, \omega_{\PP^{1}}|_{U_{i}})$,
which is naturally a module over $\cO_{V_{i}}$. This gives $\omega_{X}:= \sHom(\pi_{*}\cO_{X}, \omega_{\PP^{1}}$
the structure of a sheaf on $X$, and it is obvious from the construction that this has the desired pushforward. Uniqueness (up to isomorphism) follows easily, but we will soon prove a stronger version.

Because $\pi$ is finite we have
$$
\HH^{1}(\omega) = \HH^{1}(\pi_{*}\omega) 
= \HH^{1}(\pi_{*}\sHom(\cO_{C}, \omega_{\PP^{1}}).
$$
We may use the structure map $\cO_{\PP^{1}} \to \pi_{*}\cO_{C}$ to deduce
a map  
$$
\HH^{1}(\omega) = 
\HH^{1}(\pi_{*}\sHom(\cO_{C}, \omega_{\PP^{1}})
\to
\HH^{1}(\sHom(\cO_{\PP^{1}}, \omega_{\PP^{1}})
= \HH^{1}\omega_{\PP^{1}}
$$
and thus any residue map $\eta_{\PP^{1}}: \HH^{1}(\omega_{\PP^{1}}) \to k$
gives rise, by composition, to a residue map 
$\eta: \HH^{1}(\omega)\to k$.

We next prove that the pair $(\omega, \eta)$ has the dualizing property given in the 
Theorem.  Because $\pi$ is finite, the cohomology of a sheaf on $X$ is the same as the cohomology of its pushforward. In view of the construction of $\omega$, and the fact that $\omega_{\PP^{1}}$ is a dualizing sheaf for $\PP^{1}$, it suffices to show that there is a natural isomorphism
$$
\pi_{*}\sHom_{X}(\sF,\sHom_{\PP^{1}}(\pi_{*\sO_{X}}, \omega_{\PP^{1}})) \to \sHom_{\PP^{1}}(\pi_{*}\sF, \omega_{\PP^{1}}).
$$
Passing to an affine open set $U_{i}\subset \PP^{1}$ and its preimage $V_{i}\subset X$
as in the definition of $\omega$,  the left hand side becomes
$$
\Hom_{V_{i}}(\sF|_{V_{i}}, \Hom_{U_{i}}(\cO_{X}|V_{i}, \omega_{\PP^{1}}|_{U_{i}}))
$$
where all the $V_{i}$-modules are considered $U_{i}$-modules via the structure
map $\cO_{U_{i}} \to \cO_{V_{i}}$. The natural map sending an element $a$ to
the map 
$$
f(a) \in \sHom_{U_{i}}(\sF|_{V_{i}}, \omega_{\PP^{1}}|_{U_{i}})
$$
that takes $t\in \sF|_{V_{i}}$ to $a(t)(1)\in \omega_{\PP^{1}}|_{U_{i}}$ is the natural
isomorphism required.

Finally, to prove the uniqueness statement, suppose that $(\omega',\eta')$ is also a dualizing module for $X$. The duality property of $\omega'$ yields
$$
\Hom_{X}(\omega, \omega') = \HH^{0}(\sHom(\omega, \omega')) = \Hom(\HH^{1}(\omega), k).
$$
Let $g: \omega \to \omega'$ be the map corresponding under this isomorphism to $\eta'$. It follows from the relation of $\eta'$ to the duality isomorphism, that 
$\eta'g = \eta$. Similarly, we get a map $g':\omega'\to \omega$ such that
$\eta g = \eta'$, and it also follows that $\eta gg' = f_{\eta}(gg') =  \eta$, so that
$gg' = 1_{\omega'}$. Similarly, $g'g = 1_{\omega}$, and we are done.
\end{proof}

\begin{fact}[Dualizing sheaves in higher dimension] The proof given for curves above actually works for a purely $r$-dimensional projective scheme $X$ over $k$ if we replace the occurrences of  $\HH^{1}$ by $\HH^{r}$, and shows that every such scheme has a dualizing sheaf $\omega$. However, the isomorphisms
$$
\HH^{i} (\sHom(\sF, \omega)) \not\cong \Hom_{k}(\HH^{r-i}(\sF), k) 
$$
hold for all coherent $\sF$ if and only if $X$ is Cohen-Macaulay. In our situation
this condition means that, if $\pi: X\to \PP^{r}$ is a finite map, then $\pi_{*}\cO_{X}$
is locally free.

A scheme is said to be Gorenstein if it has a dualizing sheaf that is locally free---the canonical bundle. A more general form of Theorem~\ref{canonical is ext} says that if $\pi: X -> Y$ is a finite morphism of pure dimensional Gorenstein schemes, and $Y$ has a canonical bundle $\omega_{Y}$, then the canonical bundle of $X$ may be
represented as 
$$
\omega_{X} = \sExt^{\dim Y-\dim X}_{\sO_{Y}}
(\pi_{*}\sO_{X}, \omega_{Y}),
$$ 
regarded as a sheaf on $\sO_{X}$.

For all this and more, see \cite{AltmanKleiman}.
\end{fact}

 

\begin{exercise}[Adjunction formula] Prove directly that if $\omega_{Y}$ is a dualizing sheaf on a surface $Y$ and $X$ is a divisor on $Y$, then
$\omega := \cO_{X}\otimes_{Y}\omega_{Y}(X))$ is a dualizing sheaf for $X$.  
\end{exercise}

%%The classical adjunction formula gives the canonical line bundle on a curve $C$ on a smooth surface $X$ in terms of the canonical bundle $\omega_{X}$ of $X$ as 
%%$$
%%\omega_{C} = \omega_{X}(C) |_{C},
%%$$
%%and thus, even more familiarly, gives the genus of $C$ as an intersection number,
%%$2\g(C)-2 = (K_{X}+C).C$. 
%%
%%We can reinterpret this formula homologically by noting that there is a short exact sequence
%%$$
%%0\to \sO_{X}(-C) \to \sO_{X}\to \sO_{C}\to 0.
%%$$
%%giving a long exact sequence in $\sExt_{\sO_{X}}(-, \omega_{X})$. Taking into account
%%that $\sExt_{\sO_{X}}(\sO_{X}, \omega_{X}) =0$ because $\sO_{X}$ is locally free, we get the
%%exact sequence
%%$$
%%\to \sHom_{\sO_{X}}(\sO_{X}, \omega_{X}) \to \sHom_{\sO_{X}}(\sO_{X}(-C), \omega_{X})
%%\to \sExt^{1}_{\sO_{X}}(\sO_{C}, \omega_{X}) \to 0,
%%$$ 
%%proving that
%% $\sExt^{1}_{\sO_{X}}(\sO_{C}, \omega_{X}) = \omega_{X}(C)|_{C}$, 
%% and thus showing in this case that 
%% $$
%% \omega_{C} = \sExt^{1}_{\sO_{X}}(\sO_{C}, \omega_{X}).
%% $$
%%
%% It turns out that the ``1'' in the $\sExt^{1}_{\sO_{X}}$ in this formula is just the codimension of $C$ in $X$, and that the result holds in great generality:
% 
%\begin{theorem}\label{canonical is ext}
%Let $C\subset X$ be a purely 1-dimensional subscheme of $\PP^{n}$. The canonical sheaf of $C$ is 
%$$
%\omega_{C} \cong \sExt^{n-1}_{\sO_{\PP^{n}}}(\sO_{C}, \omega_{\PP^{n}}) = 
%\sExt^{n-1}_{\sO_{\PP^{n}}}(\sO_{C}, \sO_{\PP^{n}})(-n-1).
%$$
%which is the sheafification of the graded module $\Ext^{n-1}(S_{C},S)(-n-1)$.
%\end{theorem}
%
%For an abstract definition of the canonical bundle as a ``dualizing sheaf'' see \cite{AltmanKleiman}. For a smooth variety over $\CC$ the definition is satisfied by the the top exterior power of the cotangent bundle, and for $\PP^{n}$ this is $\sO(-n-1)$. 
%\section{Noether's Theorem}\label{sec:canonical curves} 
%
%\fix{we need to do this in a setting where we know the statement of Riemann-Roch. How general is that? Not much harm in retreating to the case of irreducible curves or even smooth ones, and adding a ``fact'' for the general case.}

As we remarked at the beginning of this Chapter, all canonically embedded smooth curves have the same Hilbert functions. We shall now show that they are all arithmetically Cohen-Macaulay.
We follow the treatment in \cite{Schreyer}, and treats a more general situation than that of the 
images of smooth curves under the canonical embeddings.

We define a \emph{canonical curve} in $\PP^{g-1}$ to be a purely one-dimensional, nondegenerate closed subscheme  such that
$$
 h^{0}(\sO_{C}) = 1, h^{0}(\sO_{C}(1) = g, \hbox{ and } \omega_{C} = \sO_{C}(1).
$$

We say that $C$ has a \emph{simple} $g-3$-dimensional $g-2$ secant if $C$ contains $g-2$ reduced points spanning a $(g-3)$ dimensional
plane in $\PP^{g-1}$ that meets $C$ only in the $g-2$s points.

\begin{theorem}[Max Noether]
A canonical curve in $\PP^{g-1}$ has degree $2g-2$ and arithmetic genus $g$. If the curve has a simple
$g-3$-dimensional $g-2$ secant, then it is arithmetically Cohen-Macaulay; that is,
$\h^{1}\sI_{C/\PP^{g-1}}(m) = 0$ for all $m\in \ZZ$.
\end{theorem}
 
Note that for a canonically embedded irreducible curve, the hypothesis of having a simple $g-3$-dimensional $g-2$ secant plane $\Lambda$ is the same as having a base-point free pencil of degree $g-2$; this is the linear series 
cut out by the hyperplanes containing $\Lambda$.

 Note that these hypotheses imply that $C$ is connected and Gorenstein.
 
\begin{proof}
We begin by showing that $\sO(-m)$ has no global sections for $m>0$.
If $D$ is a divisor equivalent to $n$ times the hyperplane section, we have an exact sequence
$$
0\to \HH^{0}(\sO_{C}(-n)) \to \HH^{0}(\sO_{C}) \to \HH^{0}(\sO_{D}) \to \cdots.
$$
By hypothesis, the vector space $\HH^{0}\sO_{C}$ is spanned by the constant functions, and these
restrict non-trivially to $\sO_{D}$, and $\HH^{0}(\sO_{C}(-n)) = 0$ as claimed.

Using the Riemann-Roch Theorem we can now compute the Hilbert function $\chi_{C}(m)$:
We have 
$$
\chi_{C}(0) = h^{0}(\sO_{C}) - h^{1}(\sO_{C}) = h^{0}(\sO_{C}) - h^{0}(\omega_{C}) = 1-g.
$$
On the other hand 
$$
\chi_{C}(0) = h^{0}(\sO_{C}(1)) - h^{1}(\sO_{C}(1)) = h^{0}(\sO_{C}(1_{}) - h^{0}(\sO_{C}) = g-1.
$$
Since $\chi_{C}(m)$ is a polynomial of degree1 in $m$, we deduce
$\chi_{C}(m) = (2g-2)m -g+1$, whence we see that the degree of $C$ is $2g-2$ and $\p(C) = g$ as claimed.

To show that
$C$ is arithmetically Cohen-Macaulay we use the sequence
$$
\cdots \to \HH^{0}(\sO_{\PP^{n}}(m)) \to \HH^{0}(\sO_{C}(m))
\to \HH^{1}(\sI_{C}(m))\to \HH^{1}(\sO_{\PP^{n}}(m)) \to\cdots .
$$
Since $\HH^{0}(\sO_{\PP^{n}}(m)) = 0$, it
is enough to show that the natural map 
$$
\HH^{0}(\sO_{\PP^{n}}(m)) \to \HH^{0}(\sO_{C}(m))
$$
 is surjective for all $m\in \ZZ$. For $m=0,1$ this is immediate from the hypothesis.

For $m <0$ we must show $\HH^{0}(\sO_{\PP^{n}}(m))=0.$ 
If $D$ is a divisor equivalent to $-m$ times the hyperplane section, we have an exact sequence
$$
0\to \HH^{0}(\sO_{C}(m)) 
\to \HH^{0}(\sO_{C}) 
\to \HH^{0}(\sO_{D}) \to \cdots.
$$
By hypothesis, the vector space $\HH^{0}\sO_{C}$ is spanned by the constant functions, and these
restrict non-trivially to $\sO_{D}$, so the kernel, $\HH^{0}(\sO_{C}(m))$, is 0 as claimed. 

To prove surjectivity for $m\geq 2$ we use the remaining hypothesis, the existence of
a simple $g-3$-dimensional $g-2$ secant plane $\Lambda$  and an idea sometimes called the \emph{base-point-free pencil trick}. Let $p_{0},\dots p_{g-3}$ be the points in which $\Lambda$ meets $C$.  Since the
$p_{i}$ are linearly independent by hypothesis, we may choose homogeneous coordinates $x_{i} \in \HH^{0}(\sO_{C}(1))$ so that
$x_{i}(p_{j} \neq 0$ if and only if $i = j$. It follows that the sections
$x_{i}^{m}$ of $\sO_{C}(m)$ span $\HH^{0}(\sO_{C}(m)|_{\{p_{0}, \dots, p_{g-3}\}}$. Let 
$V\subset \HH^{0}(\sO_{C}(1))$ be the two-dimensional subspace of linear forms vanishing on
$\Lambda$, and thus on the $p_{i}$. 

For $m\geq 2$ there are maps of vector spaces
$$
\wedge^{2} V\otimes \HH^{0}(\sO_{C}(m-2)) \to V\otimes \HH^{0}(\sO_{C}(m-1)) 
\to \HH^{0}(\sO_{C}(m))
$$
where the right hand map is multiplication and the left hand map sends
$s_{1}\wedge s_{2}\otimes \sigma$ to $s_{1}\sigma-s_{2}\sigma$ for any local section $\sigma$.
The sequence is exact because the sections $s_{1},s_{2}$ that span $V$ never vanish simultaneously except on the $p_{i}$, and has image  consisting of sections that vanish on the points $p_{i}$

\end{proof}
\section{Castelnuovo's theorem: $2g+1$ is projectively normal.}

 Regularity in terms of resolution and in terms of local cohomology. (has the relation of local to global coho been done already?)

We'll need local cohomology, local duality, 
 WMACE
recall the $2g+1$ theorem from Ch 4. Say that it's ACM+smooth. relate it to regularity as in "geom of syz". Say that the projective
normality of smooth CI's is really ACM for all CI's. Say that ACM is measured by the length of the free resolution.


\section{Liaison of curves in $\PP^3$.}
%Rao module, Hartshorne-Rao theorem. Resolution of the rational quartic.

\begin{fact} We say that curves $C,C'\subset \PP^{3}$ are ``connected by a linkage'' if $C\cup C'$
is a complete intersection.
 The Hartshorne-Rao Theorem \cite{} asserts that 
$C$ and $C'$ can be connected by an even length sequence of linkages if and only if their Rao modules are equal up to a shift; and by an odd length sequence of linkages if and only if their Rao modules differ by a shift and dualization into $\CC$.
\end{fact}


Genus formula by filtration: Suppose that $D$ is a complete intersection of surfaces of degrees $d_1,d_2$, and 
$C\subset D$ is unmixed of dim 1. Define $C'$ by $\cI_{C'} = (\cI_D:\cI_C)$. Prove by unmixedness that 
$\cI_C = (\cI_D:\cI_{C'})$,  and by local duality (?) that 
$$
\omega_{C'} = \cHom_{\cO_D}(\cO_C, \omega_D) = 
\cHom_{\cO_D}(\cO_C, \cO_D(d_1+d_2-4))
=  (\cI_D:\cI_C)(d_1+d_2-4).
$$
Thus there is an exact sequence
$$
0\to \omega_{C'}(-d_1-d_2+4) \to \cO_D \to \cO_C \to 0.
$$
Computing Hilbert polynomials via RR gives the genus formula.

Construction of curves with given Rao module: Bourbaki's theorem.


\subsection{How syzygies can reflect geometry}

One of the main ways in which syzygies can be seen to reflect the geometry of an embedded  reduced, irreducible curve $C\subset \PP^r$
(or other variety) is linked to the possibility of factoring the line bundle $\sO_C(1)$ as the tensor product of two bundles with sections. Suppose for example that $\sO_C(1) = \sL_1\otimes \sL_2$. Choose 2 independent global sections
$\sigma_1, \sigma_2$ of  $H^0(\sL_1)$ and a basis $\tau_1,\dots, \tau_n$ of $H^0(\sL_2)$. Set
$l_{i,j}= \sigma_i\tensor \tau_j$ and consider the matrix 
$$
M = 
\begin{pmatrix}
 l_{1,1}&l_{1,2}&\dots&l_{1,n}\\
  l_{2,1}&l_{2,2}&\dots&l_{2,n}
\end{pmatrix}.
$$
We claim that  the $2\times 2$ minors $l_{1,j} l_{2,j'}-l_{1,j'}l_{2,j}$ are in the homogeneous ideal of $I_C$ of $C$ in $\PP^r$. 

\begin{example}
The most familiar example is that of the twisted cubic. In this case the global sections $x_0\dots x_3$ of $\sO_C(1)$ may be identified with the forms of degree 3 in $k[s,t]$, and if $p\in C \cong \PP^1$ then the factorization $\sO_C(1) = \sO_C(p) \otimes \sO_C(2p)$ leads to the familiar matrix
$$
\begin{pmatrix}
x_0&x_1&x_2\\
x_1&x_2&x_3 
\end{pmatrix}.
$$
In this special case we see that in fact $I_2(M) = I_C$, and the same idea works for the rational normal curve of any degree.

\end{example}

To see that $I_C$ contains the ideal $I_2(M)$ generated by the $2\times 2$ minors of $M$ in general, 
let $K(C)$ be the ring of rational functions on $C$ \fix{have we made this definition somewhere?}. Choosing identifications $\sL_i\otimes K(C) \cong K(C)$ we see that the $\sigma_i$ and the $\tau_j$ commute with each other as elements of $K(C) \otimes_{\sO_X} K(C)$, and thus 
$$
\bigl(l_{1,j} l_{2,j'}-l_{1,j'}l_{2,j}\bigr)|_C = \sigma_1\tau_j\sigma_2\tau_j' - \sigma_1\tau_j'\sigma_2\tau_j =0.
$$

In case $C$ is reduced and irreducible the matrix above has a special property: $K(X)$ is a domain, so no product of a nonzero
section of $\sL_1$ with a nonzero section of $\sL_2$ can be zero. We can state this without any reference to $C$:

\begin{definition}
Let $R$ be a commutative ring. A map $M:R^n\to R^m$ is 1-generic if the kernel of the corresponding
 map $R^{n}\otimes R^{m*} \to R$  contains no pure tensor $a\otimes b$. In more concrete terms, a matrix
$M$ is \emph{1-generic} if there are no invertible matrices $A,B$ such that  $AMB$ has some entry equal to 0.
\end{definition}

We shall see that if $C$ is reduced and irreducible, then  $I_2(M)$ is the homogeneous ideal of a rational normal 
scroll
of codimension $n-1$ and degree $n$, and it has a free resolution of a special form called the 
Eagon-Northcott complex. If, in addition $C$ is non-degenerate, then the minimal free resolution of $I_2(M)$ is
a subcomplex of the minimal free resolution of $I_C$. The presence of such a variety containing $C$ or
a subcomplex of this special form in the minimal free resolution of $C$ is thus necessary for the 
factorization of the line bundle $\sO_C(1)$ as above, and it is sometimes sufficient, as well.

\section{The Eagon-Northcott Complex of a $2\times n$ matrix of linear forms}

The Eagon-Northcott complex is a complex of free modules associated to any matrix over any commutative ring. The most familiar special case is the Koszul complex, which one may think of as the Eagon-Northcott complex of a $1\times n$ matrix, and  even in the general case the Eagon-Northcott complex is in a sense built out of the Koszul complexes. A full treatment of the Eagon-Northcott complex and a whole family of related constructions can be found in 
\cite[Appendix ***]{E}, and, from a more conceptual and general point of view, in \cite{Weyman}. Here we will only
make use of the case of a matrix such as the one above, we will present a simplified account in that case only. Here is the result we need:

\begin{theorem}
 Let $S = k[x_0,\dots, x_r]$ be a polynomial ring,  and let $M: F\to G$ be a homomorphism with
 $F = S^n(-1), G= S^2$. If $M$ is 1-generic, then the ideal
 $I_2(M)$ is the homogeneous ideal of a rational normal scroll of codimension $n-1$, and the minimal free resolution of $S/I_2(M)$ has the form:
\begin{align*}
EN(M) := 
S \lTo{\bigwedge^2 M} 
 \bigwedge^2 F&
 \lTo^{\delta_{2}}
 S^{2*}\otimes \bigwedge^3 F  \lTo^{\delta_{3}}
  (\sym^2S^{2})^*\otimes\bigwedge^4F  \\
 &\lTo^{\delta_{4}}\cdots\lTo^{\delta_{n-1}} 
(\sym^{n-2}S^{2})^*\otimes\bigwedge^nF 
 \lTo 0.
\end{align*}
\end{theorem}

\begin{proof} 
 We begin by defining the maps $\delta_i$ and and proving that the given sequence is indeed a complex---that is, consecutive maps compose to 0. For simplicity of notation, we drop the twists corresponding to the 
 grading of $S$ and choose a generator, and identify $\wedge^2 S^2$ with $S$, which gives a sense to the map labeled $\bigwedge^2M$.
 
  Although it is not hard to do this directly, the dual maps
 $$
 \partial_i: \sym^{i-2} G \otimes \bigwedge^i F^* \rTo \sym^{i-1} G \otimes \bigwedge^{i+1} F^*
 $$
 have a more familar-looking description, so we define these instead. Indeed, the map $M$ corresponds to an
 element $\mu\in G\otimes F^*$. We may think of $ \sym^{i-2} G \otimes \bigwedge^i  F^*$
 as a (bigraded) component of the exterior algebra over $ \sym G$ of 
 $$
  \sym G \otimes \bigwedge_S  F^*= \bigwedge_{ \sym G} (\sym G \otimes  F^*).
 $$
We define $\partial_i$ to be  multiplication by $\mu$ in the sense of this exterior algebra. Since $\mu$ has degree 1
in this sense, its square is 0. 

To show that $(\bigwedge^2 M)\circ \delta_2$ is zero, it is simplest to choose a matrix representing $M$.
Direct computation using only the usual expansion of a determinant
along a row shows that, up to sign,
pure basis vector $e\otimes f_i\wedge f_j\wedge f_k$ of $G^*\otimes \bigwedge^3 F$
maps under the composition $(\bigwedge^2) M\circ \delta_2$ to the determinant
of the $3\times 3$ matrix obtained from $M$ by repeating the row corresponding to $e$ and
the columns $i,j,k$, which is 0 because of the repeated row.

We next prove the split exactness of a complex of the form $EN(M')$ where $M'$ is surjective, so that we
may write $F = G\oplus F'$ and the map $M': G\oplus F' \to G$ as projection on the first factor. Since this
is a "trivial" case, it is perhaps surprising that the proof is a little complicated:

Of course
it suffices to prove the split exactness of the dual sequence. We begin by proving split exactness at the 
positions $\sym^{i} G \otimes \bigwedge^{i+2}  F^*$ where $i\geq 1$.

The module
$ \sym^{i} G \otimes \bigwedge^{i+2}  F^*$
decomposes as
\begin{align*}
&\sym^{i} G \otimes \bigwedge^2 G^* \otimes \bigwedge^{i} F'^*\oplus \\
&\sym^{i} G \otimes  G^*\otimes \bigwedge^{i+1} F'^* \oplus \\
&\sym^{i} G \otimes  \bigwedge^{i+2} F'^* 
\end{align*}
Note that under our hypothesis, the element $\mu' \in G\otimes F^* = G\otimes G^* \oplus G\otimes F'^*$
has the form $(\mu_G, 0)$, where $\mu_G$ represents the identity map $G \to G$. Thus the complex
$EN(M')$ is a direct sum over $i$ of 3-term complexes of the form
$$
\sym^{i+1} G \otimes \bigwedge^2 G^* 
\lTo^{-\wedge \mu'} 
\sym^{i} G \otimes  G^*
\lTo^{-\wedge \mu'} 
\sym^{i-1} G
$$
tensored with various $\bigwedge^j F'^*$, and it suffices to show that the former are split exact when
$i\geq 0$. Now $\sym S^2$ may be identified with $R:= S[x,y]$, where $x,y$ are a basis of $S^2$, and
as such the sequences above may be identified with components of the Koszul complex of $x,y$ over $R$,
$$
R\otimes \wedge^2 G\lTo R\otimes G \lTo R .
$$
The only homology of this sequence is $R/(x,y)R$, at the left, so if we replace  $R$ by the ideal $(x,y)R$, this sequence becomes split exact, and this is the desired result.

We must also prove that $EN(\mu')^*$ is exact at 
$$
\wedge^2 F^* = \bigwedge^2 G^* \oplus (G^*\otimes F'^*) \oplus \bigwedge^2 F'*
$$
Consider the pair of maps
\begin{align*}
S \rTo{\bigwedge^2 M'^*} 
 &\bigwedge^2 G^* \oplus G^*\otimes F' \oplus \bigwedge^2 F'^*
 \rTo^{-\wedge \mu'} \\
 &G\otimes \bigwedge^2 G^*\otimes F'^* \oplus (G\otimes G^*\otimes \bigwedge^2 F'^*) \oplus G\otimes \bigwedge^3 F'^*
\end{align*}
where we have written $\delta_{2, M'}$ in place of $\delta_{2}$ to indicate the dependence on $M'$.
The map $\bigwedge^2 M'$ is the projection to $\bigwedge^2 G$ composed with the chosen isomorphism
$\bigwedge^2 G \cong S$, so we must show that the image of 
$\delta_{2, M'}$ is 
$G\otimes F' \oplus \bigwedge^2 F'$.
The map labeled $-\wedge \mu'$ obviously kills $\wedge^2 G^*$ and is a split monomorphism
on the other two summands. On the other hand, the map $\bigwedge^2M'*$ is an isomorphism onto
$\wedge^2 G^*$, completing the proof of split exactness of $EN(M')$. \end{proof}
%
%The first non-trivial example of a finite free resolution is the Koszul complex on 3 variables, which is the minimal $S = k[x,y,z]$-free resolution of the module $S/(x,y,z)$:
%$$
%0\to S(-3) \rTo^{
%\begin{pmatrix}
%x\\y\\z 
%\end{pmatrix}}
% S^3(-2) \rTo^{\begin{pmatrix}
%0&-z&y\\
%z&0&-x\\
%-y&x &0
%\end{pmatrix}}
%S^3(-1) \rTo^{
%\begin{pmatrix}
%x&y&z
%\end{pmatrix}}
%S
%$$
%In fact this is the first example that Hilbert presented in his famous paper \cite{}. 																											

\section{Syzygies and the Clifford index}
Green's Conjecture, Voisin's Theorem, Ein-Lazarsfeld Theorem --- all cheerful facts. proof that low gonality implies long linear part.

\subsection{Low genus canonical embeddings} Schreyer's table (include $g=9$)?

\section{Cheerful facts about Stillman's problem}
%\section{Low degree}

%footer for separate chapter files

\ifx\whole\undefined
\makeatletter\def\@biblabel#1{#1]}\makeatother
\gdef\urlhook{\url}
\bibliography{slag}
\bibliographystyle{msribib}


%%%% EXPLANATIONS:

% f and n
% some authors have all works collected at the end

\catcode`\^\active
%if ^ is followed by 
% 1:  print f, gobble the following ^ and the next character
% 0:  print n, gobble the following ^
% any other letter: print letter
\makeatletter
\def^#1{\ifx1#1f\expandafter\@gobbletwo\else
        \ifx0#1n\expandafter\expandafter\expandafter\@gobble\else#1\fi\fi}
\makeatother
\let\moreadhoc\relax
\def\indexintro{%An author's cited works appear at the end of the
%author's entry; for conventions
%see the List of Citations on page~\pageref{loc}.  
%\smallbreak\noindent
The letter `f' after a page number indicates a figure, `n' a footnote.}
\printindex[gen]
%requires makeindex
\end{document}
\else
\fi
