%header and footer for separate chapter files

\ifx\whole\undefined
\documentclass[12pt, leqno]{book}
\usepackage{graphicx}
\input style-for-curves.sty
\usepackage{hyperref}
\usepackage{showkeys} %This shows the labels.
%\usepackage{SLAG,msribib,local}
%\usepackage{amsmath,amscd,amsthm,amssymb,amsxtra,latexsym,epsfig,epic,graphics}
%\usepackage[matrix,arrow,curve]{xy}
%\usepackage{graphicx}
%\usepackage{diagrams}
%
%%\usepackage{amsrefs}
%%%%%%%%%%%%%%%%%%%%%%%%%%%%%%%%%%%%%%%%%%
%%\textwidth16cm
%%\textheight20cm
%%\topmargin-2cm
%\oddsidemargin.8cm
%\evensidemargin1cm
%
%%%%%%Definitions
%\input preamble.tex
%\input style-for-curves.sty
%\def\TU{{\bf U}}
%\def\AA{{\mathbb A}}
%\def\BB{{\mathbb B}}
%\def\CC{{\mathbb C}}
%\def\QQ{{\mathbb Q}}
%\def\RR{{\mathbb R}}
%\def\facet{{\bf facet}}
%\def\image{{\rm image}}
%\def\cE{{\cal E}}
%\def\cF{{\cal F}}
%\def\cG{{\cal G}}
%\def\cH{{\cal H}}
%\def\cHom{{{\cal H}om}}
%\def\h{{\rm h}}
% \def\bs{{Boij-S\"oderberg{} }}
%
%\makeatletter
%\def\Ddots{\mathinner{\mkern1mu\raise\p@
%\vbox{\kern7\p@\hbox{.}}\mkern2mu
%\raise4\p@\hbox{.}\mkern2mu\raise7\p@\hbox{.}\mkern1mu}}
%\makeatother

%%
%\pagestyle{myheadings}

%\input style-for-curves.tex
%\documentclass{cambridge7A}
%\usepackage{hatcher_revised} 
%\usepackage{3264}
   
\errorcontextlines=1000
%\usepackage{makeidx}
\let\see\relax
\usepackage{makeidx}
\makeindex
% \index{word} in the doc; \index{variety!algebraic} gives variety, algebraic
% PUT a % after each \index{***}

\overfullrule=5pt
\catcode`\@\active
\def@{\mskip1.5mu} %produce a small space in math with an @

\title{Personalities of Curves}
\author{\copyright David Eisenbud and Joe Harris}
%%\includeonly{%
%0-intro,01-ChowRingDogma,02-FirstExamples,03-Grassmannians,04-GeneralGrassmannians
%,05-VectorBundlesAndChernClasses,06-LinesOnHypersurfaces,07-SingularElementsOfLinearSeries,
%08-ParameterSpaces,
%bib
%}

\date{\today}
%%\date{}
%\title{Curves}
%%{\normalsize ***Preliminary Version***}} 
%\author{David Eisenbud and Joe Harris }
%
%\begin{document}

\begin{document}
\maketitle

\pagenumbering{roman}
\setcounter{page}{5}
%\begin{5}
%\end{5}
\pagenumbering{arabic}
\tableofcontents
\fi


\chapter{Syzygies}
\label{SyzygiesChapter}

\section{Introduction} 
%Motivation: canonical embedding turns intrinsic invariants into projective invariants. Hilbert Function. Projective Normality; Canonical Module.
%
%What are the projective invariants that correspond to Clifford index? Conjecturally, Green's conjecture. Inequality from Eagon-Northcott.
%
%Hilbert Syzygy theorem, Hilbert function derivation, Unique minimal resolution, Betti table, 
%\fix{ introduce tools as they are used}

It is much easier to talk about the geometry of a curve once it is embedded in projective space, but of course this is really the geometry of a pair, the curve \emph{and} the embedding. However if we take the embedding corresponding to the complete linear series associated to the canonical line bundle (or a fixed multiple of it), then we can be sure we are talking about an intrinsic invariant of the curve itself. In genus 0,1 the canonical bundle is not ample, so this approach doesn't work; but in genus $\geq 2$,as we have seen in Chapter ****, the canonical bundle itself determines an embedding for all but hyperelliptic curves; the square of the canonical bundle gives an embedding for all  curves of genus $>2$; and the cube of the canonical bundle gives an embedding for every curve of genus $\geq 2$. 

Once a curve is embedded in projective space, the first algebraic invariant we have looked at is it's Hilbert function; but this simply recovers the genus and the degree. A much finer invariant is the minimal free resolution of the homogeneous coordinate ring of the curve. For example according to the conjecture of Mark Green (now verified in many special cases), the free resolution determines the ``smallest'' base point free linear series on the curve, in a precise sense that we will describe.

We begin by reviewing some basic facts about free resolutions from \cite[Ch ?]{E}. Let 
$R = R_{0} \oplus R_{1}\oplus\cdots$ be a positively graded ring, with $R_{0}$ a field,
 and let $M$ be a finitely generated graded $S$-module. An  \emph{$R$-free resolution} of $M$ is a sequence of free graded $R$ modules $F_{i}$ and homogeneous maps $d_{i}$ of the form
$$
\FF: F_{0}:=\oplus_{j}R(-j)^{\beta_{0,j}} \lTo^{d_{1}} F_{1}:=\oplus_{j}R(-j)^{\beta_{0,j}}\lTo^{d_{2}} F_{2}\cdots
$$
and an \emph{augmentation} map $F_{0} \rOnto^{d_{0}} M$ such that the kernel of $d_{i}$ is equal to the image of $d_{i+1}$ for every $i$. 

The resolution $\FF$ is \emph{minimal} if every entry of matrices representing the $d_{i}$ (for $i>0$ is of strictly positive degree. A minimal free resolution of $M$ can be constructed inductively by first choosing a minimal set of homogeneous generators of $M$, determining the map $d_{0}$, then choosing a minimal set of generators of $\ker d_{0}$, determining a map
$F_{1} \rOnto \ker d_{0} \subset F_{0}$, and so on.

Now let $S = \CC[x_{0},\dots, x_{n}]$ be the homogeneous coordinate ring of $\PP^{n}$. In case $R=S$ something quite special happens \cite[****]{E}:

\begin{theorem}[Hilbert's Syzygy Theorem]\label{hst}
The minimal $S$-free resolution $\FF$ of $M$ is finite: in fact, $F_{i}=0$ for all $i>n+1$.
\end{theorem}

A much easier result shows that the minimal free resolution can be used to derive invariants of $M$
\cite[Theorem ***]{E}:

\begin{theorem}[Uniqueness]\label{uniqueness}
With hypotheses as above, the minimal free resolution of $M$ is unique up to (non-unique) isomorphism.
\end{theorem}

The numbers $\beta_{i,j}$ in  the minimal resolution of $M$ are called the \emph{graded Betti numbers} of $M$; It follows from this that the graded Betti numbers, and all the invariants of the matrices $d_{i}$, are invariants of $M$. In particular since we know the Hilbert function of $S$, and thus of $S(-j)$,
$$
H_{S(-j)}(t) = 
\begin{cases}
 {n+t-j \choose n} & \hbox{for $t\geq 0$}
  \\ 
 0 & \hbox{for $t<0$}
\end{cases}
$$



\section {general adjunction formula} homological definitions of the canonical sheaf. (The canonical module as dual of a local cohomology, 
and local duality proves it's  an Ext (compare with the adjunction f
ormula.))



\section{Castelnuovo's theorem: $2g+1$ is projectively normal.}

 Regularity in terms of resolution and in terms of local cohomology. (has the relation of local to global coho been done already?)

We'll need local cohomology, local duality, 
 Local duality, WMACE
recall the $2g+1$ theorem from Ch 4. Say that it's ACM+smooth. relate it to regularity as in "geom of syz". Say that the projective
normality of smooth CI's is really ACM for all CI's. Say that ACM is measured by the length of the free resolution.

\section{Liaison of curves in $\PP^3$.}
Rao module, Hartshorne-Rao theorem. Resolution of the rational quartic.

Genus formula by filtration: Suppose that $D$ is a complete intersection of surfaces of degrees $d_1,d_2$, and 
$C\subset D$ is unmixed of dim 1. Define $C'$ by $\cI_{C'} = (\cI_D:\cI_C)$. Prove by unmixedness that 
$\cI_C = (\cI_D:\cI_{C'})$,  and by local duality (?) that 
$$
\omega_{C'} = \cHom_{\cO_D}(\cO_C, \omega_D) = 
\cHom_{\cO_D}(\cO_C, \cO_D(d_1+d_2-4))
=  (\cI_D:\cI_C)(d_1+d_2-4).
$$
Thus there is an exact sequence
$$
0\to \omega_{C'}(-d_1-d_2+4) \to \cO_D \to \cO_C \to 0.
$$
Computing Hilbert polynomials via RR gives the genus formula.

Construction of curves with given Rao module: Bourbaki's theorem.


\section{Syzygies and the Clifford index}
Green's Conjecture, Voisin's Theorem, Ein-Lazarsfeld Theorem --- all cheerful facts. proof that low gonality implies long linear part.
\subsection{The Eagon-Northcott complex (lite)}: 
--Koszul complex.
--The inequality in Green's conjecture.
--resolutions of the rational normal curves and scrolls.
--Green's conjecture inequality.

\subsection{Low genus canonical embeddings} Schreyer's table (include $g=9$)?

\section{Cheerful facts about Stillman's problem}
%\section{Low degree}

%footer for separate chapter files

\ifx\whole\undefined
%\makeatletter\def\@biblabel#1{#1]}\makeatother
\makeatletter \def\@biblabel#1{\ignorespaces} \makeatother
\bibliographystyle{msribib}
\bibliography{slag}

%%%% EXPLANATIONS:

% f and n
% some authors have all works collected at the end

\begingroup
%\catcode`\^\active
%if ^ is followed by 
% 1:  print f, gobble the following ^ and the next character
% 0:  print n, gobble the following ^
% any other letter: normal subscript
%\makeatletter
%\def^#1{\ifx1#1f\expandafter\@gobbletwo\else
%        \ifx0#1n\expandafter\expandafter\expandafter\@gobble
%        \else\sp{#1}\fi\fi}
%\makeatother
\let\moreadhoc\relax
\def\indexintro{%An author's cited works appear at the end of the
%author's entry; for conventions
%see the List of Citations on page~\pageref{loc}.  
%\smallbreak\noindent
%The letter `f' after a page number indicates a figure, `n' a footnote.
}
\printindex[gen]
\endgroup % end of \catcode
%requires makeindex
\end{document}
\else
\fi
