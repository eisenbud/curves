%header and footer for separate chapter files

\ifx\whole\undefined
\documentclass[12pt, leqno]{book}
\usepackage{graphicx}
\input style-for-curves.sty
\usepackage{hyperref}
\usepackage{showkeys} %This shows the labels.
%\usepackage{SLAG,msribib,local}
%\usepackage{amsmath,amscd,amsthm,amssymb,amsxtra,latexsym,epsfig,epic,graphics}
%\usepackage[matrix,arrow,curve]{xy}
%\usepackage{graphicx}
%\usepackage{diagrams}
%
%%\usepackage{amsrefs}
%%%%%%%%%%%%%%%%%%%%%%%%%%%%%%%%%%%%%%%%%%
%%\textwidth16cm
%%\textheight20cm
%%\topmargin-2cm
%\oddsidemargin.8cm
%\evensidemargin1cm
%
%%%%%%Definitions
%\input preamble.tex
%\input style-for-curves.sty
%\def\TU{{\bf U}}
%\def\AA{{\mathbb A}}
%\def\BB{{\mathbb B}}
%\def\CC{{\mathbb C}}
%\def\QQ{{\mathbb Q}}
%\def\RR{{\mathbb R}}
%\def\facet{{\bf facet}}
%\def\image{{\rm image}}
%\def\cE{{\cal E}}
%\def\cF{{\cal F}}
%\def\cG{{\cal G}}
%\def\cH{{\cal H}}
%\def\cHom{{{\cal H}om}}
%\def\h{{\rm h}}
% \def\bs{{Boij-S\"oderberg{} }}
%
%\makeatletter
%\def\Ddots{\mathinner{\mkern1mu\raise\p@
%\vbox{\kern7\p@\hbox{.}}\mkern2mu
%\raise4\p@\hbox{.}\mkern2mu\raise7\p@\hbox{.}\mkern1mu}}
%\makeatother

%%
%\pagestyle{myheadings}

%\input style-for-curves.tex
%\documentclass{cambridge7A}
%\usepackage{hatcher_revised} 
%\usepackage{3264}
   
\errorcontextlines=1000
%\usepackage{makeidx}
\let\see\relax
\usepackage{makeidx}
\makeindex
% \index{word} in the doc; \index{variety!algebraic} gives variety, algebraic
% PUT a % after each \index{***}

\overfullrule=5pt
\catcode`\@\active
\def@{\mskip1.5mu} %produce a small space in math with an @

\title{Personalities of Curves}
\author{\copyright David Eisenbud and Joe Harris}
%%\includeonly{%
%0-intro,01-ChowRingDogma,02-FirstExamples,03-Grassmannians,04-GeneralGrassmannians
%,05-VectorBundlesAndChernClasses,06-LinesOnHypersurfaces,07-SingularElementsOfLinearSeries,
%08-ParameterSpaces,
%bib
%}

\date{\today}
%%\date{}
%\title{Curves}
%%{\normalsize ***Preliminary Version***}} 
%\author{David Eisenbud and Joe Harris }
%
%\begin{document}

\begin{document}
\maketitle

\pagenumbering{roman}
\setcounter{page}{5}
%\begin{5}
%\end{5}
\pagenumbering{arabic}
\tableofcontents
\fi



\chapter{Curves of genus 0 and 1 exercises}\label{genus 0 and 1 exercises}

\begin{exercise}\label{veronese inverse}
With notation as in Section~\ref{rational normal curves section}, show that the sheaf associated to the graded module $\coker M$,
that is, the cokernel of the map $\sO_{\PP^d}^d(-1) \to \sO_{\PP^d}^2$ defined by $M$, is the unique invertible sheaf of degree 1
on the rational normal curve $C$, and that thus the associated complete linear series defines the isomorphism $C\to \PP^1$ inverse
to the Veronese map.

Hint: The images of the two generators of $\sO_{\PP^d}^2$ are global sections that generate the
sheaf. The vanishing locus of the first is the point of the curve where the second row is 0.
 \end{exercise}

\begin{exercise}\label{equations of Veroneses}
Considering $\PP^n$ as $\Proj \CC[x_0,\dots,x_n]$, we may index the variables $z_p$ of $\PP^{\binom{n+d}{d}-1}$ by  monomials $p$
of degree $d$ in the $x_i$. Let $M_{n,d}$ be an $(n+1)\times \binom{n+d-1}{n}$ matrix of linear forms
on $\PP^{\binom{n+d,d}-1}$ whose rows are indexed by the variables $x_i$, whose columns are indexed by the monomials $m$ of degree $d-1$ in the $x_i$ and
whose $(i,m)$ entry is $z_{x_im}$. (For example the matrix
$M$ of Section~\ref{rational normal curves section} is the matrix $M_{1,d}$.) Show that the $2\times 2$ minors of $M_{n,d}$ generate the ideal of the image $V_{n,d}$ of the Veronese map 
$\PP^n\to \PP^{\binom{n+d}{d}-1}$, and that the cokernel of $M_{n,d}$ is the unique invertible sheaf of degree 1 supported on $V_{n,d}$.

Hint: Do as in Exercise~\ref{veronese inverse}
\end{exercise}

\begin{exercise}
 Let $\nu_d: \PP^r \to \PP^{\binom{r+d}{r}-1}$ be the $d$-Veronese map, and let $C\subset \PP^r$ be the rational normal curve of degree $r$. Is $\nu_d(C)$ nondegenerate? If not, what is the dimension of its linear span (that is, of the smallest linear
space that contains it)?
\end{exercise}

\begin{exercise}
Show that the twisted cubic is the unique irreducible, nondegenerate space curve lying on three quadrics by considering the possible
intersections of two of the quadrics.

Hint: To be nondegenerate, the curve has to have degree $\geq 3$.
\end{exercise}

\begin{exercise}
As a consequence of our description of rational quartic curves on a smooth quadric in Proposition~\ref{ideal of rational quartic},
show that a general $g^3_4$ on $\PP^1$ is uniquely expressible as a sum of the $g_1^1$ and a $g^1_3$
(in other words, a general 4-dimensional vector space of quartic polynomials on $\PP^1$ is uniquely expressible as the product of a 2-dimensional vector space of cubics and the 2-dimensional space of linear forms.
\end{exercise}

\begin{exercise}\label{distinguishing rational quartics}
Show that, up to projective equivalence, there is a 1-parameter family of embeddings of $\PP^1$ as a 
smooth quartic curve in $\PP^3$ 
by constructing an invariant that distinguishes them. 

Hint: Use Exercise ~\ref{decomposition of a $g^3_4$} and the classification of
$g^1_3$s.
\end{exercise}

\begin{exercise}\label{Castelnuovo uniqueness}
Complete the proof of Proposition~\ref{points on rnc} by showing that if $C, C' \subset \PP^n$ are two rational normal curves meeting in at least $n+3$ distinct points, then $C = C'$.

Hint: Consider the projection from a point of $C\cap C'$, and use induction. 
\end{exercise}

\begin{exercise}\label{rnc and representations}
Let $V = \CC\cdot e_1\oplus \CC\cdot e_2$ be a 2-dimensional vector space. 

The group $SL_2= SL(V)$ acts on the rational normal curve of degree $d$ through automorphisms induced from its action on
 on the ambient space $\PP^d$ of the rational normal curve, which may be identified with $\PP(\Sym^d(V))$.

In~\cite[pp. 146--150]{Fulton-Harris} it is shown that
 every finite dimensional rational 
representation of $V$ is a direct sum of representations of the form $\Sym^e(V)$ for various $e\geq 0$. Moreover, it is often easy to understand
how a given representation decomposes by looking at the action of
$$
\alpha := \begin{pmatrix}
t&0\\
0&t^{-1}
\end{pmatrix}
\in SL(V).
$$
Note that $\Sym^e(V)$ is spanned by ``weight vectors" ($\equiv$ eigenvectors of $\alpha$) $w_s := e_1^{e-s} e_2^{s}$ 
which satisfy $\alpha w_s = t^{e-2s}$ for $s = 0, \dots e$.
To decompose an arbitrary representation $W$, knowing that $W$ is a direct sum of $\Sym^{e_i}V$, it is enough to know the 
eigenvalues for the action of $\alpha$: We begin by finding an element $w\in W$ that
is an eigenvector of $\alpha$ and transforms by $\alpha$ as
as $\alpha w = t^mw$ with the highest possible $m$ (this is called a ``highest weight vector''). This element $w$ must be contained
in a summand $\Sym^m(V)$, and after removing the eigenvalues of the action of $SL_2$ on $\Sym^m(V)$, we continue. 
\begin{enumerate}
 \item Use this method to show that 
\begin{align*}
&\Sym^d(V)\otimes Sym^d(V)= \Sym^{2d}(V) \oplus  \Sym^{2d-2}(V) \oplus \Sym^{2d-4}(V) \cdots\\
 &\Sym^2(\Sym^d(V))= \Sym^{2d}(V) \oplus \Sym^{2d-4}(V)\oplus \Sym^{2d-8}(V) \cdots\\
 &\bigwedge^2(\Sym^d(V))= \Sym^{2d-2}(V) \oplus \Sym^{2d-6}(V)\oplus \Sym^{2d-10}(V) \cdots
\end{align*}
  (where we take $\Sym^{m}(V)=0$ when $m<0$
 \item Show that the space of quadrics containing the rational normal curve is a representation of $SL_2$ of the form
 $$
 \Sym^{2d-4}(V)\oplus \Sym^{2d-8}(V) \cdots
 $$
  \item Show  there is a distinguished nonsingular skew-symmetric form (up to scalars) on the ambient space of the twisted cubic; in particular
  is, given a twisted cubic in $\PP^3$ there is a distinguished plane containing each point of $\PP^3$.
 
 \item Show that if $d$ is divisible by 4 there is a distinguished quadric in the ideal of the rational normal curve.
\end{enumerate}
 Hint: For the last two parts, show that $\Sym^0(V)$ is a summand of the appropriate representation.
\end{exercise}

\begin{exercise}
Let $\PP^1 \hookrightarrow C \subset \PP^3$ be a twisted cubic. Show that the normal bundle $\cN_{C/\PP^3}$ (defined to be the quotient of the restriction $T_{\PP^3}|_C$ to $C$ of the tangent bundle  of $\PP^3$  by the tangent bundle $T_C$) is 
$$
\cN_{C/\PP^3} \cong \cO_{\PP^1}(5) \oplus  \cO_{\PP^1}(5).
$$
Hint: for any point $p \in C$, let $L_p \subset \cN_{C/\PP^3}$ be the sub-line bundle of $\cN_{C/\PP^3}$ whose fiber over any point $q \neq p \in C$ is the one-dimensional subspace of $(\cN_{C/\PP^3})_q$ spanned by the line $\overline{p,q}$. (This of course only defines a sub-line bundle of $\cN_{C/\PP^3}$ over $C \setminus \{p\}$, but there is a unique extension to a sub-line bundle of $\cN_{C/\PP^3}$ over all of $C$.) Show that for $p \neq p'$ we have
$$
\cN_{C/\PP^3} = L_p \oplus L_{p'}.
$$
Hint: for any point $p \in C$, let $L_p \subset \cN_{C/\PP^3}$ be the sub-line bundle of $\cN_{C/\PP^3}$ whose fiber over any point $q \neq p \in C$ is the one-dimensional subspace of $(\cN_{C/\PP^3})_q$ spanned by the line $\overline{p,q}$. (This of course only defines a sub-line bundle of $\cN_{C/\PP^3}$ over $C \setminus \{p\}$, but there is a unique extension to a sub-line bundle of $\cN_{C/\PP^3}$ over all of $C$.) Show that for $p \neq p'$ we have
$$
\cN_{C/\PP^3} = L_p \oplus L_{p'}.
$$

\end{exercise}

\begin{exercise}
Let $\PP^1 \hookrightarrow C \subset \PP^d$ be a rational normal curve. Show that the normal bundle $\cN_{C/\PP^d}$  is 
$$
\cN_{C/\PP^d} \cong \bigoplus_{i=1}^{d-1} \cO_{\PP^1}(d+2).
$$
Hint: choose $d-1$ points on the curve, and construct a sub-bundle
analogous to that in Exercise~\ref{Normal bundle of cubic}.
\end{exercise}

\begin{exercise}
In the situation of the preceding problem, the set  of direct summands of $\cN_{C/\PP^d} $ is a projective space $\PP^{d-2}$. How does the  group of automorphisms of $\PP^d$ carrying $C$ to itself act on this $\PP^{d-2}$?
(For more on normal bundles of rational curves, see for example~\cite{MR3778979}.
\end{exercise}





\input footer.tex