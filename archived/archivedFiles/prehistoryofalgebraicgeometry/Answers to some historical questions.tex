\documentclass[11pt]{book}
\usepackage{graphicx}
\usepackage{amsmath,amssymb, amsthm}
\usepackage{color}
%\usepackage[hidelinks]{hyperref}
\usepackage{hyperref}
 \usepackage{amsfonts}
 \usepackage{makeidx}
 \usepackage{xfrac}
 % \usepackage{ulem}
 \usepackage{tikz}
 \usetikzlibrary{arrows}
 \usepackage{natbib}
 %\bibliographystyle{chicagoa}
 \makeindex
      
\textwidth = 6.5 in \textheight = 8.5 in \oddsidemargin = 0.0 in
\evensidemargin = 0.0 in \topmargin = 0.0 in \headheight = 0.2 in
\headsep = 0.5 in
\parskip = 0.2in
\parindent = 0.2in

\newcommand{\cyp}{\citeyearpar}
\newcommand{\ha}{\frac{1}{2}}
\newcommand{\R}{\mathbb{R}}
\newcommand{\C}{\mathbb{C}}
\newcommand{\CP}{\mathbb{CP}}
\newcommand{\PP}{\mathbb{P}}
\newcommand{\N}{\mathbb{N}}
\newcommand{\Q}{\mathbb{Q}}
\newcommand{\Z}{\mathbb{Z}}
\newcommand{\B}{\mathbb{B}}
\newcommand{\A}{\mathbb{A}}
\newcommand{\Hom}{\rm{Hom}}
\newcommand{\F}{\mathbb{F}}
\newcommand{\ii}{\sqrt{-1}}
\newcommand{\ph}{\varphi}
\newcommand{\rd}{\partial}
\newcommand{\dux}{\frac{\partial u}{\partial x}}
\newcommand{\duy}{\frac{\partial u}{\partial y}}
\newcommand{\dvx}{\frac{\partial v}{\partial x}}
\newcommand{\dvy}{\frac{\partial v}{\partial y}}
\newcommand{\ulr}{\underline{r}}
\newcommand{\ulu}{\underline{u}}
\newcommand{\ulv}{\underline{v}}
\newcommand{\ulg}{\underline{g}}
\newcommand{\ulga}{\underline{\gamma}}
\newcommand{\olb}{\overline{\B}}
\newcommand{\dsp}{\displaystyle}
\newcommand{\gdg}{\emph{Grundlagen der Geometrie}}
\newcommand{\n }{^{-1}}
\newcommand{\rr}{\mathbf{r}}
\newcommand{\col}{\textcolor}
\newcommand{\ci}{\mathcal{I}}
\newcommand{\cz}{\mathcal{Z}}
\newlength{\wdth}
\newcommand{\strike}[1]{\settowidth{\wdth}{#1}\rlap{\rule[.5ex]{\wdth}{.4pt}}#1}


\newcommand{\mm}{$\mathbf{m}$}
\newcommand{\rrt}{Riemann--Roch Theorem}
\newcommand{\dieq}{dialytic equation}
\newcommand{\chnr}{characteristic number}
\newcommand{\moeq}{modular equation}
\newcommand{\nft}{Noether's Fundamental Theorem}
\newcommand{\cbt}{Cayley--Bacharach theorem}

\def\PP{{\mathbb P}}

\newtheorem{theorem}{Theorem}
\newtheorem{lemma}{Lemma}
\newtheorem{corollary}[theorem]{Corollary}
\newtheorem{definition}{Definition}
\newtheorem{example}{Example}
\raggedbottom

\def\de#1{{{\bf \col{red}{David says [* }\col{red}{#1{\bf\ *]} }}}}
\def\jg#1{{{\bf \col{red}{Jeremy says [* }\col{red}{#1{\bf\ *]} }}}}



\begin{document}
\section{The shopping list}
1) The greeks studied conics -- who?, when?

2) Descartes coordinatized the plane; did people then already write graphs of functions?? or other curves?

3) I gather that the study of conics remained synthetic for a long time; how did that change?

4) sometime in the 19th C somebody -- Pl\"ucker (?) and Monge (?) -- developed coordinates
for $P^2$; I've read that there were two systems. Equivalent? the same as now?

5. How do Puiseux and B\'ezout fit into this?

6. Are there others in the pre-history who should be mentioned (I know, of course, about Gauss and the complex numbers).


\section{Greek mathematicians and conic sections}
Surviving sources suggest that the study of conic sections, literally sections of a cone, may well have arisen with Menaechmus's work on doubling the cube around 350 BCE. This was a long-standing problem that was given a theological spin -- double the size of an altar (the Delian problem) -- that may well also have stood out because Plato in the \emph{Meno} dialogue made such a fuss of doubling the square. Earlier, Hippocrates of Chios had reduced the problem to that finding two mean proportionals between 1 and 2; i.e. the numbers $x$ and $y$ such that 
\begin{equation}~\label{Menmus}
1:x= x:y = y: 2.
\end{equation}

For Greek mathematicians, these would have been line segments of those lengths. Menaechmus may have been the first to connect the problem of doubling the cube to the idea of conic sections, or perhaps we should say quadratic curves, because it has been argued that he considered equation \eqref{Menmus} as defining or generating curves that would have been drawn and understood pointwise. I'll return to this when discussing graphing of curves. At all events, he expressed the solution to the Delian problem in terms of the intersection of a hyperbola and a parabola. 

About a century later, Diocles, and soon after him Apollonius, made significant progress with the conic sections, and the transformation from what Manaechmus did in solving problems to identifying a family of curves may be due to one Aristaeus, whose work is now lost. Diocles identified the curve that would focus the sun's rays to a point as the parabola, and knew precisely how to cut a cone so as to obtain it. 

Apollonius produced the first \emph{theory} of conic sections as sections of a cone. The books are very dry. Van der Waerden called him a virtuoso ``in dealing with geometric algebra, and also a virtuoso in hiding his original line of thought'', while going on to say that ``his reasoning was crystal clear and elegant.'' It's hard to summarise Apollonius's work. He began with basic definitions of a cone (on a circle) and its three types of section: the hyperbola, parabola, and ellipse; the names are due to him. He showed that all the known sections that had been studied as sections of a cone with vertical angle a right angle could be obtained as sections of a suitable but otherwise arbitrary cone.  

It's entirely arguable that a system of coordinates was present in his work, inasmuch as he referred everything to a pair of distinguished lines in the plane of section, but it is buried in the heavy use of proportion theory between different line segments, and you have to be very smart, as Apollonius undoubtedly was, to handle it. That said, he produced a theory of the principal diameters of conics, and studied such problems as finding the tangents to a conic from an exterior point, and came close to observing the properties of cross-ratio that modern writers detected. He also had a theory about normals to conics from which, apparently, it's a short step to obtaining their evolutes. 

We know this because four books of his conics survive in Greek and three more in Arabic; the eighth and final volume is lost. He also studied several particular problems, the solution to which required knowing properties of conic sections.  

A number of other Greek mathematicians pitched in. Sometimes their work survives directly, more often in the work of later writes such as Pappus; much of it, including a book by Euclid, who lived a little before Apollonius, is lost. 

The biggest weakness in Apollonius's theory was that very often he had to treat the three kinds of conic section separately, although you can see him trying for greater generality.

\section{Descartes,  coordinatising the plane,  graphing functions and curves}
A common way to think of curves in antiquity was pointwise: some length depends in a given way on some other length. Accordingly, at least in principle, if you know the independent length
(the ordinate, or $x$ coordinate) you know the dependent length (the abscissa, or $y$ coordinate).  

By Descartes' time there was already some sophisticated algebra expressed in a formalism that hadn't quite shaken off   the Greek insistence on seeing everything as geometrical magnitudes: lengths, areas, volumes, and, well, what exactly? It was possible to write polynomial equations in this language. First Fermat in 1636, and then much more boldly Descartes in 1637, realised that you could extend the language to two variables and so describe curves in the plane.

Descartes' s first achievement was to eliminate the dimensional aspect. A simple use of similar triangles allowed him to show that the product of two lengths could be seen as another length (not an area) so all geometrical quantities could be regarded as one-dimensional and the idea of dimension quietly dropped. Then came the real work. Almost all mathematical problems in his day were expressed in the language of geometry, except for some problems we would call diophantine and were implicitly about integers and rational numbers. Accordingly, the answer had to be expressed geometrically. Descartes's idea was to give letters to all the lengths involved in a problem, use the statement of the problem to express relationships between the letters, and reduce the equations to a single equation. Then solve the equation and express the answer again in geometrical terms.  

His achievements include replacing the cumbersome algebra of his day, which was written in capital letters with abbreviations for the algebraic operations, with something much more like   what we use today. He wrote $x$ and $y$ for the key variables (Not $A$ and $E$ as hitherto). He was clear that he was using coordinates, although his $x$ and $y$ coordinates could have oblique axes. He could find normals to a curve at a point if the curve had an algebraic equation, and described a system of sliding rulers that he said could be adapted to draw any such curve. (Curves like the cycloid that were patently not algebraic he sought to exclude from geometry.) In this way he solved the famous Pappus problem: given four lines and four angles (nothing is lost if take these angles to be right angles), find the locus of points $P$ such that the product of the distances of $P$ from the first two lines is proportional to the product of the distances of $P$ from the last two lines. As he showed, and Pappus had known, the answer is a conic section, but 
Descartes went further and claimed that in this way he could solve the Pappus problem for any number of lines. 

Digression. This was to infuriate Newton, who showed that Greek methods were indeed adequate to the problem. The background here is that Newton was also engaged in demolishing Descartes's theory of planetary motion in favour of his own, which led him into the theory of conic sections and the problem of finding the conic (or conics) through $n$ points and tangent to $5-n$ lines.

As for graphing functions, I could simply say that before Euler in the late 1840s there was no concept of a function, everything was geometry, a relationship between two varying lengths.

\section{How did the study of conics cease to be synthetic?}
Recent historical work suggests that it was all a bit murky, and algebraic methods were also often used. Several things promoted the use of synthetic methods. They can be elegant when algebraic methods are blunt; they correspond to the visual form of the conics; they provide a language for describing what is apparent or to be found in a problem. Against them is the obstinate fact that algebra is more general: it does not care if some quantities become negative, but what is a negative length? Once ways round that were found (by Poncelet and then Chasles) the way was open to a truly systematic synthetic theory of conics.   

What would that be? The key idea is due to Desargues in 1639, who wrote a short, difficult essay on the projective theory of conics. In it, all non-degenerate conics are treated on a par, the key idea is the idea of four harmonic points ($A,  B, C, D$ are four points on a line and $AC/CB=AD/DB.$) Sometimes people said $B$ and $D$ separate $A$ and $C$ in the same ratio. He also had a more complicated property of six points, which, like the four-point property was invariant under a projection. He connected all this to the study of the ``complete quadrilateral'' -- take four points, no three collinear, and join them in all possible ways, and thence to the study of conics through those four points. 

The book is fiercely unreadable, and although I am known for a book about it I am no longer an expert. If you want to know more I can refer you to three or four good recent papers. The book was also lost for a long time and known only through commentaries by later authors until Michel Chasles found a copy in the 1820s (I think). Thus de la Hire, a generation after Desargues, wrote some much more readable, and longer, works in Desargues's spirit, illuminating the role of cross-ratio in the theory of tangents that came close to a theory of duality. Desargues's famous theorem on two triangles in perspective was published separately.

But the best attention Desargues's little book got was from his younger contemporary Blaise Pascal, who evidently produced a virtually complete theory around what he called the ``mystical hexagram''. Unhappily much of it is lost, and known to us only from some notes made about by Leibniz, but the idea is that while there is always a conic through five points something happens if you want a conic through six points: we call it Pascal's theorem. Then, if you let the sixth point collapse onto one of the other five you get a tangent to the conic through those five points. One way or another all the key properties of conics are wrapped up in this idea, or so Pascal seems to have shown, and much of the early 19th century work in France can be seen as attempts to recover such a theory. It includes such topics as duality, in the form of the pole and polar relationship with respect to a conic, more or less known earlier to de la Hire.

How then did this change? I think the answer goes like this. Very few of the original protagonists disdained algebra outright, and as the (projective) theory of conics reached completion and established its fundamental character, being more general than metrical Euclidean geometry, it also had its baroque aspects. But worse, it did not generalise at all to the study of curves of higher degree. For that, as even Newton had recognised, a hefty dose of algebra was required.

The key figures here is Pl\"ucker. Although Euler and Cramer had written on cubic and quartic curves, Pl\"ucker saw that much needed to be done. His key idea to study families of curves, using the symbolic notation he devised. If $S_1$ and $S_2$ stand for the equations of two curves of the same degree, then $S_1 + \lambda S_2$ is the equation of another degree of that kind. This allowed him to pull out geometrical properties of cubic and quartic plane curves while avoiding the algebraic complexities that had defeated even Euler.


\section{Coordinates for $P^2$}
By and large, Pl\"ucker did not have them. In his study of curves he would first discuss them in  the plane, and then as they went off to infinity; he didn't say that the line at infinity could be mapped by a projective transformation into the finite part of the plane. 

The way forward was indicated by M\"obius, who introduced barycentric coordinates. As you surely know, that goes like this. Pick three points forming a triangle, say $ABC$, and attach weights, positive, zero, or negative (not all zero) to these points. The barycentre or centre of gravity of these three weighted points is a point $P$ which can be said to have those three weights (or, better, their ratios) as its barycentric coordinates. If you put the points $A, B, C$ at, say,  $(0, 0), (1, 0), (0, 1)$ you get an easy way to relate points in what could be called the Cartesian and barycentric coordinate planes. The big plus is that the line at infinity, which is invisible in Cartesian coordinates, is a perfectly sensible line in barycentric coordinates. In this way M\"obius obtained a simple theory of conics and duality in the plane. (By the way, he also showed that there are dualities in $p^3$ that are not pole-polar dualities.)

If you drop the talk about weights, and keep the idea that barycentric coordinates are best thought of as ratios of three numbers, you have projective coordinates, and speaking from memory this was one of the contributions of Otto Hesse. But we must note a reluctance to decide if this was $\R P^2$ or, less likely $\C P^2$.


\section{B\'ezout and Puiseux}
I know very little about B\'ezout. Apparently he lived from 1739 to 1783, and made his living teaching mathematics at the French military and naval academies. He published the theorem that bears his name in a book of 1779; it is based on his theory of the resultant of two polynomial equations that he developed in a paper of 1764. His proof of B\'ezout's theorem is gappy and intuitive by any standards, but so much better than what had done before that his immediate successors were willing to give him real credit for doing as much as he did. His results inspired later work by Cauchy and Sylvester.

Puiseux was one of a number of mathematicians in the circle around Cauchy. Cauchy had spent the 1830 and early 1840s following the Bourbon Court around Europe from a strange belief that the oath of allegiance he had sworn to crown on becoming a professor compelled him to do so. As a result, few people knew the work he had done in those years, and even he seems to have forgotten what he had done in complex variable theory back in the 1820s, which included a limited version of what we call the Cauchy integral theorem. One novelty of the work he did on his return was to think much more geometrically; previously his attitude to a many-valued `function' was to cut the plane and study just a branch of it on what remains. His understanding of branch points was quite limited. Puiseux, in his paper of 1850-51, dived in, studied integrals on arbitrary contours (Cauchy had mostly used rectangular ones) and studied what happens to integrals taken around branch points.                                                                                                                                                                                                                                                                                                                                                                                                                                                                                                                                                                                                                                                                                                                                                                                                                                                                                                                                                                                                                                                                                                            


\section{Others in the pre-history}
Well, that's quite  lot. There's a projective geometry story that goes something like this. Gaspard Monge was for a time interested in how to depict three dimensions on two for military purposes, and devised a method of plan and elevation (projections onto a horizontal and a vertical plane) that he could couple to some simple algebra. He was influential in setting up the Ecole Polytechnique, and an inspiring teacher of geometry, and did much to revive the subject. Among those so inspired there was Jean Victor Poncelet, who promoted a much more general theory of transformations with a view to unifying the theory of conics, and Michel Chasles, who used the projective invariance of the cross-ratio of four points to eliminate much of the weirdness of Poncelet's ideas. Poncelet had a dispute with Gergonne about what duality in the plane actually is. Poncelet always saw it as pole and polar with respect to a conic, Gergonne saw it as a new and fundamental feature of projective geometry. This led into confusion when applying it to curves of degree three of more, a matter that began to be sorted only with Pl\"ucker's work, as we discuss.

A truly foundational work on real and complex projective geometry that avoided deriving it from Euclidean geometry was the achievement of von Staudt in the 1850s; Robin Hartshorne knows more about him than anyone else does.

On the vexed history of complex numbers I like Euler's attitude that there was nothing to explain. We have these expressions of the form $a+bi$ and they behave arithmetically like numbers, so let's call them numbers, and as for what the symbol $i$ means well, to sure, you'll never meet a length of $i$ but we can imagine these expressions in our mind and control them, so what more do you want? Euler, and Cauchy after him, was not one for definitions of a philosophical nature, but what he was rejecting was the idea that ultimately mathematical quantities must be exhibited in nature: three sheep, a length of $\sqrt{2}$, and so on. (There were even Brits denying that negative numbers made no sense without more work; negative three sheep are indeed not like positive three sheep, which have specific properties such as wool colour, etc.) Cauchy's take was more explicit and very close to saying that you can think of the field of complex numbers as $\R [x]/(x^2 + 1)$. This works for finding a proof of the fundamental theorem of algebra, which he gave a proof of in 1816, but was not a productive way to think about contour integration. 

One should also mention the interpretations of complex numbers due to Argand and Wessel, but we shouldn't let everything sprawl. 


Then we have to the marvellous work of Abel and Jacobi on elliptic functions. They transformed Legendre's elaborate theory of elliptic integrals by allowing everything to be complex and inverting the integral (the analogy is passing from the integral for $\arcsin (x)$ to the study of $\sin (x)$). But in many ways both men were algebraists at heart -- formidable algebraists -- and quite what the plane of complex numbers is did not really interest them. Nor, strictly, did it interest Gauss when he gave his first proof of the fundamental theorem of algebra in 1799. One of his later proofs is a neat complex integration argument, but his fullest discussion of what complex numbers are, as opposed to how they are represented in the plane, came in the paper of 1831 in which he introduced what we now call the Gaussian integers. By then William Rowan Hamilton had published his rigorous theory of order pairs of real numbers (1826).


I think of Riemann as one of those mathematicians who rewrite what they have been told in his own way, in his case when appropriate a more geometrical way. He certainly discussed some geometrical topics with Gauss, but his real teacher was Dirichlet, who was more rigorous than Gauss and much more rigorous than Cauchy. (Jacobi said ``When Gauss says something, it is probably true; when Cauchy says something the chances are 50 percent; when Dirichlet says something, it is certain. For myself, I have no taste for such delicacies.") 






\end{document}
