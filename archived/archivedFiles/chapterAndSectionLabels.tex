\chapter{Introduction}
\label{IntroChapter}

\chapter{Linear series and morphisms to projective space}
\label{linear series}

\chapter{The Riemann-Roch Theorem}
\label{RiemannRochChapter}

\chapter{Curves of genus 0 and 1}
\label{genus 0 and 1 chapter}
\section{Rational Normal Curves}\label{rational normal curves section}

\chapter{Jacobians}
\label{Jacobians chapter}\label{new Jacobians chapter}\label{JacobianChapter}
\section{Symmetric products and the universal divisor}\label{symmetric section}
\section{The Picard varieties}\label{Picard section}
\section{The $g+3$ theorem}\label{g+3 section}

\chapter{Hyperelliptic curves and curves of genus 2 and 3}
\label{genus 2 and 3 chapter}
\section{Interlude: branched covers with specified branch divisor}\label{branched covers}
\section{Curves of genus 2}\label{genus 2 section}


\chapter{Fine Moduli spaces} 
\label{Moduli chapter}\label{ModuliChapter}
\section{Hilbert schemes}\label{hilbert scheme section}
\begin{theorem}\label{tangent space of Hilb}
\subsection{Construction of the Hilbert scheme in general}\label{hilb construction}
\subsection{Equations defining the Hilbert scheme}\label{eqns of Hilb}
\section{Hurwitz spaces}\label{Hurwitz spaces}
\section{The Severi variety}\label{severi variety}

\chapter{Moduli of curves} 
\label{CurvesModuli chapter}\label{CurvesModuliChapter}
\subsection{$M_g$ and $\overline M_g$ are almost fine}\label{almost fine}
\subsection{Can one write down a general curve of genus $g$?}\label{mgunirational}
\subsection{$M_g$ is not a fine moduli space}\label{coarse moduli}

\chapter{Curves of genus 4 and 5}
\label{genus 4, 5 Chapter}
\subsection{The canonical model}\label{canonical genus 4}
\section{Canonical curves are projectively normal}\label{Noether theorem section} %% is this the right place for it?

\chapter{Hyperplane sections of a curve}
\label{linear general position chapter}
\subsection{Existence of good projections}\label{projection section}\label{good projections}
\subsection{The $g+2$ theorem}\label{g+2 section}
\section{Flexes and bitangents are isolated}\label{isolated flexes and bitangents}

\chapter{Monodromy of Hyperplane Sections}
\label{uniform position}

\chapter{Linear series on general curves, and curves of genus 6}
\label{Brill-Noether}\label{BNChapter}
\section{Curves of genus 6}\label{genus 6 section}

\chapter{Inflection points}\label{inflections chapter}
\label{InflectionsChapter}
\subsection{Flexes of plane curves}\label{plane curve pluecker}
\subsection{Weierstrass points}\label{Weierstrass points}
\section{Finiteness of the automorphism group}\label{finiteness section}
\subsection{Schubert cycles}\label{Schubert1}

\chapter{Proof of the Brill Noether Theorem}
\label{Brill Noether proof chapter}!!
\label{BrillNoetherproofChapter}

\chapter{Plane Curves}
\label{PlaneCurvesChapter}
\section{Using plane models to compute linear series} \label{computing linear series}
\section{Differentials on a smooth plane curve}\label{canonical series on smooth plane curves}
\section{Linear series on a nodal plane curve}\label{linear series on nodal plane curves}
\subsection{Adjoint and deficiency}\label{adjoint ideal}

\chapter{Linkage and the canonical sheaves of singular curves}
\label{LiaisonChapter}\label{linkageChapter}\label{LinkageChapter}
\section{Linkage of smooth curves in $\PP^3$}\label{SLinkage}\label{linkage section}
\subsection{Dualizing sheaves for singular curves}\label{duality}

\chapter{Scrolls and the Curves They Contain}
\label{ScrollsChapter}
\section{Some classical geometry}\label{daily name}
\section{1-generic matrices and the equations of scrolls}\label{particular name}
\section{Scrolls as Images of Projective Bundles}\label{inscrutable name}
\section{Curves on a 2-dimensional scroll}\label{curves on scrolls}

\chapter{Free resolutions and canonical curves}
\label{SyzygiesChapter}
\section{The Eagon-Northcott Complex}\label{EN section}

\chapter{Hilbert Schemes I: Examples}
\label{HilbertSchemesChapter}
\subsection{Genus 0}\label{degree 4 genus 0}
\section{Why  $4d$?}\label{estimating dim hilb}

\chapter{Hilbert Schemes II: Counterexamples} 
\label{HilbertSchemesCounterexamplesChapter}
\section{Degree 9, genus 10}\label{deg9 section}
\section{Open problems}\label{open problems}
\section{Degree 14, genus 24}\label{mumford example}
