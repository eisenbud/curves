%header and footer for separate chapter files

\ifx\whole\undefined
\documentclass[12pt, leqno]{book}
\usepackage{graphicx}
\usepackage{eps-to-pdf}
\input style-for-curves.sty
%\input sl-macros.sty
\usepackage{hyperref}
\usepackage{showkeys} %This shows the labels.
\usepackage{msribib}
\usepackage{pdfpages}
\usepackage{draftwatermark}
\SetWatermarkText{DRAFT:\ \today}
\SetWatermarkScale{2}
\SetWatermarkColor[gray]{0.9}

%\usepackage{SLAG,msribib,local}
%\usepackage{amsmath,amscd,amsthm,amssymb,amsxtra,latexsym,epsfig,epic,graphics}
%\usepackage[matrix,arrow,curve]{xy}
%\usepackage{graphicx}
%\usepackage{diagrams}
%
%%\usepackage{amsrefs}
%%%%%%%%%%%%%%%%%%%%%%%%%%%%%%%%%%%%%%%%%%
%%\textwidth16cm
%%\textheight20cm
%%\topmargin-2cm
%\oddsidemargin.8cm
%\evensidemargin1cm
%
%%%%%%Definitions
%\input preamble.tex
%\input style-for-curves.sty
%\def\TU{{\bf U}}
%\def\AA{{\mathbb A}}
%\def\BB{{\mathbb B}}
%\def\CC{{\mathbb C}}
%\def\QQ{{\mathbb Q}}
%\def\RR{{\mathbb R}}
%\def\facet{{\bf facet}}
%\def\image{{\rm image}}
%\def\cE{{\cal E}}
%\def\cF{{\cal F}}
%\def\cG{{\cal G}}
%\def\cH{{\cal H}}
%\def\cHom{{{\cal H}om}}
%\def\h{{\rm h}}
% \def\bs{{Boij-S\"oderberg{} }}
%
%\makeatletter
%\def\Ddots{\mathinner{\mkern1mu\raise\p@
%\vbox{\kern7\p@\hbox{.}}\mkern2mu
%\raise4\p@\hbox{.}\mkern2mu\raise7\p@\hbox{.}\mkern1mu}}
%\makeatother

%%
%\pagestyle{myheadings}

%\input style-for-curves.tex
%\documentclass{cambridge7A}
%\usepackage{hatcher_revised} 
%\usepackage{3264}
   
\errorcontextlines=1000
%\usepackage{makeidx}
\let\see\relax
\usepackage{makeidx}
\makeindex
% \index{word} in the doc; \index{variety!algebraic} gives variety, algebraic
% PUT a % after each \index{***}

\overfullrule=5pt
\catcode`\@\active
\def@{\mskip1.5mu} %produce a small space in math with an @

\title{A Chapter from ``The Practice of Algebraic Curves"}
\author{\copyright David Eisenbud and Joe Harris}
%%\includeonly{%
%0-intro,01-ChowRingDogma,02-FirstExamples,03-Grassmannians,04-GeneralGrassmannians
%,05-VectorBundlesAndChernClasses,06-LinesOnHypersurfaces,07-SingularElementsOfLinearSeries,
%08-ParameterSpaces,
%bib
%}

\date{\today}
%%\date{}
%\title{Curves}
%%{\normalsize ***Preliminary Version***}} 
%\author{David Eisenbud and Joe Harris }
%
%\begin{document}

\begin{document}
\maketitle

\pagenumbering{roman}
\setcounter{page}{5}
%\begin{5}
%\end{5}
\pagenumbering{arabic}
\tableofcontents
\fi


\chapter{Brill-Noether theory and applications to genus 6}\label{Brill-Noether}\label{BNChapter}

\section{Linear series on general curves of genus 6}\label{genus 6 section}\label{general genus 6}
%\fix{This will come after the plane curves: we get the 5-ic del Pezzo for free, given the "conditions of adjunction", and then
%the special cases will give us exercise in what the conditions of adjunction are.}


Throughout our analyses of curves of genus $g \leq 5$, we have been able to analyze the geometry of the canonical model to verify the statement of the Brill-Noether theorem in each case. In genus 6, by contrast, we cannot readily deduce the Brill-Noether theorem from studying the geometry of the canonical curve---we can determine that a canonical curve $C \subset \PP^5$ of genus 6 lies on a 6-dimensional vector space of quadrics, for example, but that doesn't tell us much about its geometry. Rather, we need to use the Brill-Noether Theorem, in the following form:

\begin{theorem}\label{BN consequences}
Every smooth curve $C$ of genus 6 has at least one $g^{2}_{6}$. If $C$ is general, then
$W^{2}_{6}(C)$ and $W^{1}_{4}(C)$ each consist of 5 reduced points, while $W^{2}_{5} = W^{1}_{3} = \emptyset$.  Less formally, C has precisely 5 $g^{2}_{6}$s and 5 $g^{1}_{4}$s, but no $g^{2}_{5}$s and no $g^{1}_{3}$s. The image of each $g^{2}_{5}$ is a nodal plane curve and its nodes are linearly independent.
\end{theorem}

All these assertions except for the linear independence of the nodes follow immediately from 
Theorem~\ref{basic BN} and
Theorem~\ref{Wrd omnibus}; we will deduce the independence from the relationship of the different
linear series on a general curve that are given by these theorems.

%\subsection{Linear series on general curves of genus 6}\label{general genus 6}

\emph{We suppose for the rest of this section that $C$ is a general smooth curve of genus 6.}

Theorem~\ref{Wrd omnibus} tells that the image of $C$ under a $g^{2}_{6}$, which is a plane curve $C_{0}$ of degree 6 and
geometric genus 6, has only nodes
as singularities. Since a plane sextic has arithmetic genus $\binom{6-1}{2} = 10$, the curve $C_{0}$
must have exactly 4 nodes.

Once we have exhibited one birational map of $C$ to a plane sextic with 4 nodes, we can describe all five $g^2_6$s and all five $g^1_4$s in terms of this plane model. For example, composing a $g^1_6$ corresponding to $f: C\to \PP^1$ with the projections from the 4 nodes gives four $g^1_4$s. To see the 5th $g^{1}_{4}$ we introduce some terminology:

Suppose $f : X \to S$ is a regular map from any smooth curve $X$ to a surface $S$. If $\sL $ is a line bundle on $S$ and $V \subset H^0(\sL )$ a vector space of sections, we can associate to them a linear system on $X$ by taking the pullback linear system $f^*V \subset H^0(f^*\sL )$ on $X$ and subtracting the base points; this is called the \emph{linear series cut out on $X$ by $V$}. 

To see the $g^{1}_{4}$s on $C$ in this way,  suppose again that $C$ is a general curve of genus 6 as above and $f : C \to \PP^2$ is a birational map onto a sextic curve $C_0$ with four nodes; let $p \in C_0$ be one of the nodes and consider the linear system $(\cO_{\PP^2}(1),V)$ of lines in $\PP^2$ through $p$. The pullback $f^*\cO_{\PP^2}(1)$ of course has degree 6, but the pullback linear series $f^*V$ has two base points, at the points $q, r \in C$ lying over $p$. The linear series cut on $C$ by $V$ is thus a $g^1_4$. 

To produce the fifth $g^1_4$, consider the linear series cut on $C$ by conic plane curves passing through all four nodes of $C_0$. There is a pencil of such conics, and the pullback $f^*\cO_{\PP^2}(2)$ has degree 12.
If the nodes are linearly independent then the pullback series has eight base points; thus we arrive at another $g^1_4$ on $C$.  Not all the nodes can be contained in a line, since then, by B\'ezout's theorem, the line would be a component of $C_0$. Thus if the nodes are linearly dependent, then exactly 3 lie on a line
so the linear series cut by the conics containing the nodes coincides with the projection from the 4th node. 
This would represent a non-reduced point of the scheme $W^1_4(C)$, the subject of Exercise~\ref{nonreduced Wrd} below. Thus the nodes are independent.

For another example, consider the linear system cut on $C$ by cubics passing through all four nodes. This has degree $3\cdot 6 - 8 = 10$ and dimension 6. It follows that this is the complete canonical series on $C$. (In Chapter~\ref{PlaneCurvesChapter} we will see directly that this is the case.)

Given the degree six map $f : C \to C_0 \subset \PP^2$ corresponding to one $g^2_6$ we can use the fact that the five $g^2_6$s on $C$ are residual to the five $g^1_4$s in the canonical series to construct the other four $g^2_6$s: they are cut out on $C$ by the linear system of plane conics passing through three of the four nodes of $C_0$. Equivalently, their images are the curves obtained from $C_{0}$ by the quadratic transformation
of $\PP^{2}$ centered at 3 of the 4 nodes, which blows up these 3 nodes and blows down the three lines
 joining them.
 
 In previous chapters we have seen that in genus $\leq 5$ a general canonical curve is  a complete intersection, but this fails for a canonical curve $C$ of genus 6. There is a 21-dimensional vector space of
quadratic forms on $\PP^5$, and $h^0(\sO_C(2)) = 2(2g-2)-g+1 = 15$, so $C$ lies on at least 6 quadrics, and we will show that its ideal sheaf is generated by exactly 6 quadrics. Since $6>\codim C$, the canonical curve of genus 6 is not a complete intersection. However, such curves lie on a quintic del Pezzo surface, which may be described as follows.


%There is a further consequence of this description: the four nodes of $C_0$ are in linear general position; that is, no three are collinear. 
%By parts~(\ref{rho=0}) and~\ref{Petri} of Theorem~\ref{Wrd omnibus}, $C$ must have 5 distinct $g^1_4$s, and if three of the four nodes of $C_0$ were collinear, the $g^1_4$ cut on $C$ by lines through the fourth node would coincide with the $g^1_4$ cut on $C$ by conics through all four.  
%
 
\begin{figure}
\centerline {\includegraphics[height=2in]{"main/Fig11-2"}}
\caption{A $g^1_4$ as the projection from a node of a plane sextic.}
\label{default}
\end{figure}

\begin{figure}
\centerline {\includegraphics[height=2in]{"main/Fig11-3"}}
\caption{A $g^1_4$ as a pencil of hyperbolas through the four nodes of a plane sextic.}
\label{default}
\end{figure}
 
 
\begin{figure}
\centerline {\includegraphics[height=2in]{"main/Fig11-4"}}
\caption{A sextic with 4 nodes and the fundamental triangle of the quadratic transformation giving
a different $g^{2}_{6}$.}
\label{default}
\end{figure}


\subsection{Del Pezzo surfaces}

We briefly sketch, without proofs, a little of the 
 rich classical theory of del Pezzo surfaces. The basics are well treated in \cite[pp. 45--50]{Beauville}; for more, see the
beautiful book \cite{Manin}, which also goes into some of the arithmetic theory. We will use only the case of the del Pezzo surface of degree 5, which lies in $\PP^{5}$

By definition,
a \emph{del Pezzo} surface is a smooth surface embedded in $\PP^n$ space by its complete anticanonical series $-K_S$. These exist only for $3\leq n\leq 9$. The best-known example is a smooth cubic surface in $\PP^3$. That it is a del Pezzo surface follows from the adjunction formula.

A del Pezzo surface in $\PP^n$ has degree $n$, and is isomorphic to the blow-up of $\PP^2$ at $9-n$ points of which no 3 lie on a line and no 6 lie on a conic, embedded by the linear series on $\PP^2$
consisting of the cubics passing through the $9-n$ points---except when $n=8$, in which case the
linear series of curves of type $(2,2)$ on $\PP^1\times \PP^1$ provides another example.

Comparing the linear series  of cubic forms containing $p_1,\dots,p_4$ with the linear series  of sextic forms vanishing to order 2 at $p_1,\dots,p_4$, we see that a quintic del Pezzo surface $S \subset \PP^5$ lies on at least $5$ quadrics. In fact, its homogeneous ideal is generated by exactly 5 quadrics.


%$\bullet$ If $S \subset \PP^n$ is a del Pezzo surface, then by adjunction its hyperplane sections are curves of genus 0, embedded in $\PP^{n-1}$ by a complete linear system of degree $n$. 
%
%\
%
%$\bullet$ A quartic del Pezzo surface $S \subset \PP^n$ is the complete intersection of two quadrics. It contains exactly 16 lines, which (in terms of the description of $S$ as the blow up of $\PP^2$ at five points $p_1,\dots,p_5 \in \PP^2$) are the 5 exceptional divisors, the 10 proper transforms of the lines joining the $p_i$ pairwise and the proper transform of the conic through all five $p_i$.
%


A quintic del Pezzo surface $S \subset \PP^5$ contains exactly 10 lines, which (in terms of the description of $S$ as the blow up of $\PP^2$ at four points $p_1,\dots,p_4 \in \PP^2$) are the 4 exceptional divisors and the 6 proper transforms of the lines joining the $p_i$ pairwise. 
It is the intersection of a $\PP^5$ with the Grassmannian $G(2,5) \subset \PP^9$, and correspondingly the five quadrics containing $S$ can be realized as the Pfaffians of a  $5\times 5$ skew-symmetric matrix of linear forms on $\PP^5$.

\begin{figure}
\centerline {\includegraphics[height=1.6in]{"main/Fig11-5"}}
\caption{Dual graph of the configuration of 10 lines on a quintic del Pezzo surface, the plane blown up
at 4 points showing 4 pairwise disjoint exceptional divisors.}
\label{dual graph of the configuration of 10 lines on a quintic del Pezzo surface}
\end{figure}

There is also a notion of a \emph{weak del Pezzo} surface; this is a smooth surface whose anticanonical bundle is nef but not necessarily ample. We get such a surface if we blow up $\PP^2$ at a configuration of points of which three are collinear; in this circumstance the anticanonical bundle on the blow-up $S$ has degree 0 on the proper transform of the line containing the three points, and this proper transform is correspondingly collapsed to a rational double point of the image $\phi_{-K}(S)$. In general, the description of del Pezzo surfaces as blow-ups of the plane extends to the case of weak del Pezzos.



\subsection{The canonical image of a general curve of genus 6}

Using the Brill-Noether theorem, we have seen that a general curve $C$ of genus 6 is the normalization of a plane sextic $C_0$ with four nodes, and that the canonical series on $C$ is cut out by cubics in the plane passing through the four nodes. Thus the canonical model lies on the surface $S \subset \PP^5$ that is the image of the plane under the (rational) map given by cubics through these four points, which we now recognize as a quintic del Pezzo surface.

\begin{theorem}
A general canonical curve $C$ of genus 6 is the intersection of a quintic del Pezzo surface and a quadric. 
\end{theorem}

\begin{proof}
We have seen that $C \subset \PP^5$ lies on a quintic del Pezzo surface $S \subset \PP^5$. The surface is cut out by 5 quadrics, and we know that $C$ lies on 6 independent quadrics,
so $C$ is contained in the complete intersection of $S$ with a quadric. Since this scheme has degree 10, which is the degree of $C$, they are equal.
\end{proof}


\section{Linear series on other curves of genus 6}

From Theorem~\ref{BN consequences}  we know that every curve $C$ of genus 6 has a $g^{2}_{6}$. The organizing principle of our classification will be to ask for the geometry of the associated map $\phi = \phi_D : C \to \PP^2$. For a general curve  $\phi$ maps $C$ birationally onto a plane sextic with 4 nodes; here we'll ask, ``what could go wrong?", and classify curves of genus 6 accordingly. It is possible to go quite far 
in this direction, and this section consists partly of exercises inviting the readers to explore the terrain for themselves.

Since we already have a complete picture in the hyperelliptic case, we'll assume that $C$ is non-hyperelliptic. This implies that  the $g^2_6$ is complete: Clifford's theorem rules out
the existence of a $g^3_6$. Thus we may write the $g^{2}_{6}$ as the complete linear series
$|D|$ for some divisor $D$ of degree 6.

%\fix{add table of the different configurations of $W^1_4(C)$.}

Our analysis  divides into cases as in the following outline:
\begin{enumerate}
 \item $|D|$ has a base point;
 \item $|D|$ is base-point free, but the map $\phi_{D}$
 
\begin{enumerate}
\item has degree 3, with image a conic $Q \subset \PP^2$. Then $D= 2F$, where $F$ is a $g^{1}_{3}$ and either
\begin{enumerate}
 \item 3F is nonspecial; or
 \item 3F is special.
\end{enumerate}
\item has degree 2 with image a cubic  $E \subset \PP^2$; or
\item is birational onto a plane sextic $C_0\subset \PP^2$ but with 
\begin{enumerate}
 \item a triple point
 \item only double points
\end{enumerate}
\end{enumerate}
\end{enumerate}


\subsection{$|D|$ has a base point}\label{g26 has a base point}
Clifford's theorem shows that a nonhyperelliptic curve of genus 6 cannot have a $g^2_4$, so that  
$|D|$ has one base point.
When we subtract the base point we get a base-point-free $g^2_5$. Since 5 is prime, the associated map $\phi_D : C \to \PP^2$ must be birational onto a quintic curve (in general, if $C \to C_0 \subset \PP^r$ is the map given by a base-point free $g^r_d$, the degree $d$ of the linear series is the degree of the image curve $C_0$ times the degree of the map $C \to C_0$). Moreover, since plane quintic curves have arithmetic genus 6, the image $\phi_D(C)$ is smooth; that is, in this case the curve $C$ is isomorphic to a smooth plane quintic.

To describe the special linear series on such a curve, suppose that $C \subset \PP^2$ is a smooth plane quintic. By adjunction, the canonical bundle is $K_C = \cO_C(2)$, so the canonical series $|K_C|$ is cut on $C$ by conics in the plane. Now, any three points in the plane impose independent conditions on conics (as does more generally any subscheme $\Gamma \subset \PP^2$ of dimension 0 and degree 3), so we  conclude that $C$ is not trigonal. Similarly, four points fail to impose independent conditions on conics only if they are colinear, so the $g^1_4$s on $C$ are cut out by lines through a fixed point $p \in C$; thus we get an isomorphism $C \cong W^1_4(C)$.

The canonical model of such a curve $C$, like any canonical curve of genus 6, lies on six quadrics in $\PP^5$; but these quadrics do not intersect in $C$. Rather, as the following exercise suggests, their intersection is a Veronese surface:

\begin{exercise}
Let $C \subset \PP^2$ be a smooth plane quintic curve.
\begin{enumerate}
\item Show that the canonical model of $C$ lies on a Veronese surface; and
\item Show that this Veronese surface is the intersection of the quadrics containing the canonical curve.
\end{enumerate}
\end{exercise}

Hint: For the first part, we observe that since $K_C \cong \cO_C(2)$, the linear system of conics in $\PP^2$ simultaneously maps $\PP^2$ to the Veronese surface and $C$ to its canonical image. For the second, show that the images of lines $\sL  \subset \PP^2$ under this map are plane conics in $\PP^5$ meeting the canonical curve $C$ in five points.

Next we examine the cases where $|D|$ is base-point free.

\subsection{$\phi_{D}$ is a degree 3 map onto a conic $Q \subset \PP^{2}$}

Since $Q  \cong \PP^1$, the curve $C$ is trigonal, and  $D = 2F$ where $|F|$ is a $g^{1}_{3}$ on $C$. This $g^1_3$ is unique: if there were two we would get a birational map $C \to \PP^1\times \PP^1$ whose image would have class $(3,3)$, and thus
arithmetic genus 4, a contradiction.

In this case, the residual linear series $K_C - F$ of the $g^1_3$ is a $g^3_7$, so in addition to the $g^2_6$ $|D|$ we started with, we have a one-parameter family of $g^2_6$s of the form  $K_C - F - p$; thus the variety $W^2_6(C)$ is the locus of the invertible sheaves corresponding to these linear series.

As with any nonhyperelliptic curve of genus 6, the space of quadratic forms in the ideal of the canonical model $C \subset \PP^5$ is 6-dimensional. However, by the geometric Riemann-Roch theorem, the triples of points in the divisors of the $g^1_3$ are colinear,
and the lines they span must all be contained in the quadrics containing $C$; thus the intersection of the 6 quadratic
form contains the union of these lines. It is a \emph{rational normal scroll},
and we shall study it further in Chapters \ref{ScrollsChapter} and~\ref{SyzygiesChapter}.

This case naturally breaks down into two subcases. To begin with, let $|F|$ be the (unique) $g^1_3$ on $C$. By what we've said, the double $|2F|$ is a $g^2_6$, meaning in particular that $2F$ is special. On the other hand, the divisor $4F$ has degree 12, and so is  nonspecial. This leaves open the question: is $3F$ special? The answer to this question governs the behavior of linear series on the curve $C$, and we'll consider both cases in turn.

\subsubsection{Case 1: $3F$ is nonspecial} This means that $K-3F$ is not effective, which in turn means that the residual series $K-2F$ does not have a base point: if $|K-2F|$ had a base point $p$, we would have $K - 2F \sim F + p$\, and so $K-3F \sim p$ would be effective. Thus we have a base-point-free $g^1_4$ on $C$, which we'll write as $|G|$.

Now consider the map
$$
\phi_F \times \phi_G : C \rTo \PP^1 \times \PP^1.
$$
Since 3 and 4 are relatively prime, this map is birational onto its image. The arithmetic genus of a curve of type $(3,4)$ in $\PP^1 \times \PP^1$ is 6, so the image is smooth; thus in this case $C$ is isomorphic to a smooth curve of type $(3,4)$ on $\PP^1 \times \PP^1$.

In this case, we see that the scheme $W^1_4(C)$ is the union of a curve isomorphic to $C$ (the linear series of the form $F + q$, which have base points) and an isolated point corresponding to the linear series $|K - 2F|$ (which does not have a base point).

To describe the canonical model of $C$, note that by adjunction the canonical series on the curve $C \subset \PP^1 \times \PP^1$ is cut out by curves of type $(1,2)$. Now, the linear series $|\cO_{\PP^1 \times \PP^1}(1,2)|$ embeds $\PP^1 \times \PP^1$ in $\PP^5$ as a rational normal scroll.

\begin{exercise}
To see the embedding $C \subset \PP^1 \times \PP^1$ another way, consider the map $\phi_{K-F} : C \to \PP^3$. Show that this embeds the curve $C$ as a septic curve in $\PP^3$, and this curve lies on a smooth quadric surface.
\end{exercise}

\subsubsection{Case 2: $3F$ is special}  Here we have $K - 3F \sim p$ for some $p \in C$. We claim that in this case, $C$ does not have a base-point-free $g^1_4$; in other words, every $g^1_4$ on $C$ is of the form $|F| + q$, where $|F|$ is the $g^1_3$. To see this, observe that if $|G|$ is a base-point-free $g^1_4$ on $C$, the sum $|F+G|$ is be a $g^3_7$; by the uniqueness of the $g^1_3$, we have $F+G \sim K - F$, so that $G = K-2F$---which by hypothesis has base point $p$. Thus the variety $W^1_4(C)$ consists just of divisor classes of the form $F+q$, though the following exercise shows that as a scheme, $W^1_4$ has an embedded point at the point $F+p$.

In fact  the two subcases of trigonal curves of genus 6 are distinguished by the isomorphism classes of the scrolls containing their canonical models: in the first case the lines making up the scroll join corresponding points of a twisted
cubic and a line; in the second they join corresponding points of two conics.
See Chapter~\ref{ScrollsChapter}, where we discuss rational normal scrolls in general.

\begin{exercise}
Use Part~\ref{Petri} of Theorem~\ref{Wrd omnibus} to show that in this case the point $\mu(F+p)$ is an embedded point of the scheme $W^1_4(C)$.
\end{exercise} 

Thus, if we have a family $\{C_t\}$ of trigonal curves of genus 6, with $3F$ nonspecial for $t \neq 0$ and $3F$ special on $Q$, we see the corresponding family of schemes $W^1_4(C_t)$: for $t \neq 0$, this is the disjoint union of a curve $\{F+q\}_{q \in C_t}$ and a point $\{K-2F\}$; as $t \to 0$, the point moves onto the curve, creating an embedded point.

\begin{exercise}
Show that in the case $3F$ special, the linear series $|K-F|$ embeds $C$ as a septic curve in $\PP^3$ lying on a quadric cone, with $p$ at the vertex of the cone.
\end{exercise}

More generally, the locus of trigonal curves is stratified by  the \emph{Maroni invariant}, the smallest multiple  of the $g^1_3$ that is nonspecial; this determines the isomorphism class of the unique rational normal scroll containing the canonical model of $C$. See Section~\ref{curves on scrolls}. 

\subsection{$\phi_{D} $ is a degree 2 map onto a cubic $E\subset \PP^{2}$}

The cubic curve $E$ is smooth since otherwise it would have geometric genus 0 and $C$ would be  hyperelliptic. In general, a curve expressible as a 2-sheeted cover $\pi : C \to E$ of a curve of genus 1 is called \emph{bielliptic}.

To describe such a curve, we note first that the canonical divisor class $K_C$ is the pullback of an invertible sheaf $\cO_E(F)$ for some divisor class of degree 5 on $E$. But it's not the case that the canonical series $|K_C|$ is the pullback of the linear series $|\cO_E(F)|$: by Riemann-Roch, the latter has dimension 4, rather than 5. Indeed, if we recall that the target of the canonical map $\phi_K : C \to \PP^5$ is the projective space $\PP H^0(K_C)$, there is be a point $X \in \PP^5$ corresponding to the hyperplane $\pi^*H^0(F) \hookrightarrow H^0(K_C)$, and projection of the canonical curve from this point maps $C$ 2-to-1 onto the image $\phi_F(E) \subset \PP^4$. In other words, the canonical model of $C$ lies on a cone $S = \overline{X, E}$ over an elliptic normal quintic curve $E \subset \PP^4$.

\begin{exercise}
First, show that an elliptic normal quintic curve $\phi_F(E) \subset \PP^4$ lies on 5 quadrics, as does the cone $S \subset \PP^5$ over it; deduce that there is a quadric $Q \subset \PP^5$ containing $C$ but not containing $S$. Now invoke B\'ezout's theorem to deduce that in this case, the canonical model of a bielliptic curve of genus 6 is the intersection of the cone over an elliptic quintic curve with a quadric.
\end{exercise}


\begin{exercise}
Use the preceding exercise to show that if $C$ is a bielliptic curve of genus 6---that is, a 2-sheeted cover of an elliptic curve $E$---then every $g^1_4$ on $C$ is the pullback of a $g^1_2$ on $E$, and likewise  every $g^2_6$ on $C$ is the pullback of a $g^2_3$ on $E$. Deduce that in this case, $W^1_4(C)$ and $W^2_6(C)$ are each isomorphic to $E$.
\end{exercise}

\subsection{$\phi_{D}$ is a birational map onto a plane sextic curve $C_{0}$ with a triple point}

By the adjunction formula $C_0$ must be singular. If the singular point $q$ were a quadruple point then projection from
this point would show that  $C$ is hyperelliptic contradicting our hypothesis. Thus the multiplicity can
only be 3. 
Projection from the triple point $q$ expresses $C$ as a 3-sheeted cover of $\PP^1$, so $C$ is trigonal. We saw in the trigonal case that the variety $W^2_6(C)$ typically has two components: a point, corresponding to the double of the $g^1_3$, and a curve, corresponding to the locus of $g^2_6$s of the form $K_C - E - p$, where $E$ is the $g^1_3$ and $p \in C$ is any point; this case is what we get if we start with a trigonal curve $C$ and choose a $g^2_6$ of the latter type, as the following exercise asks you to verify.


\begin{exercise}
Let $C$ be a trigonal curve of genus 6 with $g^1_3$ $|E|$, and $p \in C$ a general point. Show that the linear series $|K_C - E-p|$ is a $g^2_6$, and that the corresponding map $C \to \PP^2$ maps $C$ birationally onto a plane sextic curve with a triple point.
\end{exercise}

\subsection{$\phi_{D}$ is a birational map onto a plane sextic curve $C_{0}$ with no triple point}

Finally, suppose that $\phi : C \to \PP^2$ is birational onto a plane sextic curve $C_0$ with only double points. As we saw, if $C$ is a general curve of genus 6, then $C_0$ has exactly four nodes, and they are in general position in the plane $\PP^2$ (i.e., no three colinear). But there are other possibilities: instead of four nodes, we could have two nodes and a tacnode, two tacnodes, an oscnode (two smooth branches with contact of order 3) and a node, or one hyper-oscnode (two smooth branches with contact of order 4). And, within each of these types, the configuration of the singularities may be different: in the four-node case, three can be colinear; in the two nodes and a tacnode case, they can be colinear; or the tangent line to the tacnode can pass through one of the nodes, etc. And, of course, the same curve $C$ can show up in multiple cases, depending on which $g^2_6$ on $C$ you chose to begin with, as in Exercise~\ref{plane models}. 

\begin{exercise}\label{plane models}
Let $C$ be the normalization of a plane sextic $C_0$ with four nodes, three of which are colinear. Show that by choosing a different $g^2_6$ on $C$, we can express it as the normalization of a plane sextic with two nodes and a tacnode.
\end{exercise}


\begin{exercise}
Show that if $C$ is the normalization of a plane sextic $C_0$ with only double points, then $W^1_4(C) \cong W^2_6(C)$ is zero-dimensional (so in particular this case does not overlap with any of the previous cases)
\end{exercise}

In fact, in all the cases where $C$ is the normalization of a plane sextic $C_0$ with only double points, the scheme $W^1_4(C)$ has degree 5; in the  case of a general curve of genus 6, it  has 5 reduced points; but more generally the multiplicities of the points of $W^1_4(C)$ can be any partition of 5.


\begin{exercise}
Find an example of a curve $C$ of genus 6 such that $W^1_4(C)$ consists of one point of degree 5.
\end{exercise}

%\begin{exercise}
%Show that if $C$ is a smooth plane quintic, then the $g^2_6$s on $C$ all have a base point; that is, they are all of the form $|K_C| + p$ for $p \in C$. 
%
%Furthermore, the canonical model of $C$ will lie on a quadratic Veronese surface $S$; and the six
%quadrics containing the canonical curve $C$ are the six quadrics containing $S$ (in particular, the intersection of the quadrics containing $C$ will be $S$, so
%the ideal of $C$ requires generators of degree $>2$
%\end{exercise} 

\begin{exercise}
Prove a slightly stronger version of Theorem~\ref{arbitrary linear series} in the range $d \leq g-1$: that under the hypotheses of Theorem~\ref{arbitrary linear series} there exists a \emph{complete} linear series of degree $d$ and dimension $r$ for any $r \leq d/2$.
\end{exercise}

\begin{exercise}\label{rarity of Castelnuovo}
We have seen that complete intersections $C = Q \cap S \subset \PP^3$ of a quadric surface $Q$ and a surface $S$ of degree $k$ achieve Castelnuovo's bound $g = \pi(2k, 3)$ on the genus of curves of degree $2k$ in $\PP^3$. In fact, we will see in Chapter~\ref{ScrollsChapter} that any curve $C \subset \PP^3$ of degree $2k$ and genus $g = \pi(2k, 3) = (k-1)^2$ is of this form.
\begin{enumerate}
\item Find the dimension of the subvariety $\Gamma \subset M_g$ consisting of Castelnuovo curves.
\item Find the dimension of the subvariety $H \subset M_g$ of hyperelliptic curves, and compare this to the result of the first part.
\end{enumerate}
\end{exercise}


\begin{exercise}\label{nonreduced Wrd}
Show that if the nodes of the curve $C_0$ are in linear general position---that is, no three collinear---then indeed the map $\mu : H^0(D) \otimes H^0(K-D) \to H^0(K)$ is an isomorphism for each of the five $g^1_4$s on $C$.
\end{exercise}

We can describe similarly curves of genus 6 with only three, two or even one $g^1_4$. The most special is the case where $C$ has only one $g^1_4$; this is the normalization of a plane sextic with a \emph{flexed hyperoscnode}---that is, a double point consisting of two smooth branches with contact of order 4 with each other, and such that both branches have contact of order 3 with their common tangent line.

In general, we see that if $C$ is a non-trigonal curve of genus 6, the variety $W^1_4(C)$ is finite, and curvilinear (Zariski tangent space of dimension at most 1 at each point). There are 7 such schemes, corresponding to the number of partitions of 5, and indeed all occur.

\begin{exercise}
Find an example of a non-trigonal curve of genus 6 whose scheme $W^1_4(C)$ is isomorphic to each of the curvilinear schemes of degree 5 and dimension 0.
\end{exercise}

\begin{fact}
The seven possibilities here correspond exactly to the seven isomorphism classes of possibly singular del Pezzo quintic surfaces.
\end{fact}




\input footer.tex