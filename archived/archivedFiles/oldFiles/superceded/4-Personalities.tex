%header and footer for separate chapter files

\ifx\whole\undefined
\documentclass[12pt, leqno]{book}
\usepackage{graphicx}
\input style-for-curves.sty
\usepackage{hyperref}
\usepackage{showkeys} %This shows the labels.
%\usepackage{SLAG,msribib,local}
%\usepackage{amsmath,amscd,amsthm,amssymb,amsxtra,latexsym,epsfig,epic,graphics}
%\usepackage[matrix,arrow,curve]{xy}
%\usepackage{graphicx}
%\usepackage{diagrams}
%
%%\usepackage{amsrefs}
%%%%%%%%%%%%%%%%%%%%%%%%%%%%%%%%%%%%%%%%%%
%%\textwidth16cm
%%\textheight20cm
%%\topmargin-2cm
%\oddsidemargin.8cm
%\evensidemargin1cm
%
%%%%%%Definitions
%\input preamble.tex
%\input style-for-curves.sty
%\def\TU{{\bf U}}
%\def\AA{{\mathbb A}}
%\def\BB{{\mathbb B}}
%\def\CC{{\mathbb C}}
%\def\QQ{{\mathbb Q}}
%\def\RR{{\mathbb R}}
%\def\facet{{\bf facet}}
%\def\image{{\rm image}}
%\def\cE{{\cal E}}
%\def\cF{{\cal F}}
%\def\cG{{\cal G}}
%\def\cH{{\cal H}}
%\def\cHom{{{\cal H}om}}
%\def\h{{\rm h}}
% \def\bs{{Boij-S\"oderberg{} }}
%
%\makeatletter
%\def\Ddots{\mathinner{\mkern1mu\raise\p@
%\vbox{\kern7\p@\hbox{.}}\mkern2mu
%\raise4\p@\hbox{.}\mkern2mu\raise7\p@\hbox{.}\mkern1mu}}
%\makeatother

%%
%\pagestyle{myheadings}

%\input style-for-curves.tex
%\documentclass{cambridge7A}
%\usepackage{hatcher_revised} 
%\usepackage{3264}
   
\errorcontextlines=1000
%\usepackage{makeidx}
\let\see\relax
\usepackage{makeidx}
\makeindex
% \index{word} in the doc; \index{variety!algebraic} gives variety, algebraic
% PUT a % after each \index{***}

\overfullrule=5pt
\catcode`\@\active
\def@{\mskip1.5mu} %produce a small space in math with an @

\title{Personalities of Curves}
\author{\copyright David Eisenbud and Joe Harris}
%%\includeonly{%
%0-intro,01-ChowRingDogma,02-FirstExamples,03-Grassmannians,04-GeneralGrassmannians
%,05-VectorBundlesAndChernClasses,06-LinesOnHypersurfaces,07-SingularElementsOfLinearSeries,
%08-ParameterSpaces,
%bib
%}

\date{\today}
%%\date{}
%\title{Curves}
%%{\normalsize ***Preliminary Version***}} 
%\author{David Eisenbud and Joe Harris }
%
%\begin{document}

\begin{document}
\maketitle

\pagenumbering{roman}
\setcounter{page}{5}
%\begin{5}
%\end{5}
\pagenumbering{arabic}
\tableofcontents
\fi


\chapter{Personalities of Curves of low Genus}\label{personalities chapter}

The subject of algebraic curves abounds with examples amenable to explicit construction and analysis. In this chapter, we will survey the basic geometry and embeddings of the curves of genus 0 to  6. Our knowledge of the geometry of curves becomes increasingly less complete as the genus increases, and 6, as we shall see, is a natural turning point. 
%At the end of this chapter, the reader will be able to say with some confidence that he or she has seen every curve of genus $g \leq 6$, and understands its geometry.
\section{pre-requisites and conventions}


Basic results used in this section: B\'ezout, Riemann-Roch, Lasker (aka AF+BG), Clifford, Adjunction.
To write: an appendix on cohomology covering RR, exact sequences.
Section on Families: define family, define Hilbert Scheme and Chow variety;  but say we're not going to treat them formally very much. Flatness referred to ``Geom Schemes''; our families are smooth.


Would it be more confusing or less to use the same letter for a polynomial vanishing on $C$ and the surface it defines?

\




\section{Curves of genus 0} 

rational curves as projections of rational normal curves. Rational quartic in $\PP^3$ as curve of type 1,3 on quadric do dimension count. Branch points can be chosen. $g^3_4$ is sum of $g^1_1$ and a $g^1_3$. Cheerful fact: Set theor comp int problem.. Maximal rank for forms of degree d. Open questions: Hilbert functions? generators of the ideal? mention "secant conjecture"?

\subsubsection{Rational normal curves}

The first thing to observe about curves of genus 0 is that \emph{there is only one}: any curve $C$ of genus 0 is isomorphic to $\PP^1$. This follows immediately from the statement (\ref{degree 2g+1 embedding}) that any line bundle of degree $2g+1$ or greater on a curve of genus $g$ is very ample: if $p \in C$ is any point, by Riemann-Roch we have $h^0(\cO_C(p)) = 2$, and so the linear series $|\cO_C(p)|$ gives an ``embedding" of $C$ in $\PP^1$. Note that this works only because we are working over an algebraically closed field $K$: without that assumption, $C$ may not have any $K$-rational points at all, and indeed the classification of curves of genus 0 over non-algebraically closed fields is a subject that goes back to Gauss.

Another key fact about $\PP^1$ is that \emph{there is only one line bundle of degree $d$ on $\PP^1$} for any $d$; this is the bundle $\cO_{\PP^1}(d)$. This follows from direct observation: if $D = z_1+z_2+\dots+z_d$ and $E = w_1+\dots+w_d$ are two divisors of degree $d$, the rational function
$$
f(z) \; = \; \frac{(z-z_1)(z-z_2)\cdots(z-z_d)}{(z-w_1)(z-w_2)\cdots(z-w_d)}
$$
gives a rational equivalence between $D$ and $E$. Note that $h^0(\cO_{\PP^1}(d)) = d+1$; this follows from Riemann-Roch, or we can see it directly either by writing out explicitly the vector space of rational functions with poles along a divisor $D = z_1+z_2+\dots+z_d$:
$$
L(D) \; = \; \left\{ \frac{g(z)}{(z-z_1)(z-z_2)\cdots(z-z_d)} \mid \deg(g) \leq d \right\}
$$

The image $C \subset \PP^d$ of $\PP^1$ under the map $\phi_d : \PP^1 \to \PP^d$ associated to the complete linear series $|\cO_{\PP^1}(d)|$ is called the \emph{rational normal curve} of degree $d$. In case $d=2$, this is simply a plane conic (as we'll see, it is the zero locus of a single quadratic polynomial on $\PP^2$); in case $d=3$ it's called the \emph{twisted cubic}.

It's easy to write down the equations that define a rational normal curve. First, in coordinates, we can realize the map $\phi_d$ as
$$
\phi_d : z \mapsto [1, z, z^2,\dots,z^d],
$$
from which we see that $C$ lies in the zero locus of the homogeneous quadratic polynomial $W_iW_j - W_kW_l$ for every $i+j=k+l$. As a convenient way to package these, we can realize their span as the span of the $2\times 2$ minors of the matrix
$$
M \; = \; \begin{pmatrix}
W_0 & W_1 & \dots & W_{d-1} \\
W_1 & W_2 & \dots & W_d
\end{pmatrix}.
$$

In fact, these are all the quadratic polynomials on $\PP^d$ vanishing on $C$. To see this, consider the restriction map
$$
H^0(\cO_{\PP^d}(2)) \; \to \; H^0(\cO_{C}(2)) = H^0(\cO_{\PP^1}(2d)).
$$
This map is surjective (every monomial of degree $2d$ on $\PP^1$ is a product of two monomials of degree $d$); comparing dimensions, we see that the dimension of the kernel---that is, the space of quadratic polynomials on $\PP^d$ vanishing on $C$---has dimension
$$
\binom{d+2}{2} - (2d+1) \; = \; \binom{d}{2},
$$
which is exactly the dimension of the span of the minors of $M$. It's also easy to see that $C$ is exactly the zero locus of these quadratic polynomials and in fact they generate the homogeneous ideal of the curve $C \subset \PP^d$.

There is one other property of rational normal curves we should mention. In general, we say that a smooth curve $C \subset \PP^d$ is \emph{projectively normal} if the restriction map
$$
H^0(\cO_{\PP^d}(m)) \; \to \; H^0(\cO_{C}(m)) 
$$
is surjective for every $m$. By the same logic as above (every monomial of degree $md$ on $\PP^1$ is a product of $m$ monomials of degree $d$), we see that the rational normal curve is projectively normal. 

Projective normality is a significant geometric property of a curve. We'll see it in many settings, in particular the discussion of \emph{liaison} in Chapter~\ref{**}.


The rational normal curve of degree $d$ can also be characterized as the unique irreducible, nondegenerate curve of minimal degree $d$ in $\PP^d$. To see this, suppose that $C \subset \PP^d$ is any irreducible, nondegenerate curve. If $p_1,p_2,\dots,p_{d}$ are any $d-1$ points of $C$, they lie in a hyperplane which meets $C$ in at least those points, whence $\deg(C) \geq d$; and if we have equality then the projection $\pi_\Lambda : C \to \PP^1$ from the plane $\Lambda = \overline{p_1,p_2,\dots,p_{d-1}}$ has degree 1, from which we see that $C \cong \PP^1$.  Note that by degree considerations, \emph{any $m \leq d+1$ points on a rational normal curve $C \subset \PP^d$ are linearly independent}---in case $m=d+1$, Bezout tells us that the points cannot lie in a hyperplane, and the case $m < d+1$ follows. More generally, by the same argument \emph{any subscheme $\Gamma \subset C$ of degree $m \leq d+1$ spans an $(m-1)$-plane in $\PP^d$}.

(Having established that the smallest possible degree of an irreducible, nondegenerate curve in $\PP^d$ is $d$, it's natural to ask what is the minimal degree of an irreducible, nondegenerate variety $X \subset \PP^d$ of dimension $k$. We'll see the answer ($\deg(X) \geq d-k+1$) and describe varieties of minimal degree in Section~\ref{**}.)

MCF: the rational normal curve $C \subset \PP^d$ can also be characterized as the unique \emph{homogeneous} curve in $\PP^d$: that is, such that the automorphisms of $\PP^d$ carrying $C$ to itself act transitively on $C$.


\subsubsection{Other rational curves}

What about other rational curves in projective space? Since any linear series $\cD$ of degree $d$ on $\PP^1$ is a subseries of the complete series $|\cO_{\PP^1}(d)|$, we see that \emph{any rational curve $C \subset \PP^r$ of degree $d$ is a projection of a rational normal curve in $\PP^d$}. Slightly more generally, any map $\phi : \PP^1 \to \PP^r$ of degree $d$ is given as
$$
z \; \mapsto \; [f_0(z), \dots, f_r(z)]
$$
for some $(r+1)$-tuple of polynomials $f_\alpha$ of degree $d$ on $\PP^1$, which is to say it's the composition of the embedding $\phi_d : \PP^1 \to \PP^d$ of $\PP^1$ as a rational normal curve with a linear projection $\pi : \PP^d \to \PP^r$. 

Given how easy it is to describe rational curves in projective space in this way, it is in some ways surprising how many open questions there are about such curves: for example, it is not known what are the possible Hilbert functions of such curves; and even the Hilbert function of a general such curve was found only recently (it's now a special case of the \emph{maximal rank conjecture}; see~\ref{**}).  To give a sense of what we can say about such curves, we'll consider one of the first and simplest cases: smooth rational curves of degree $4$ in $\PP^3$.

So: let $C \subset \PP^3$ be a smooth, nondegenerate curve of degree 4 and genus 0 in $\PP^3$. To describe the geometry of $C$, the first thing to determine is what surfaces it lies on---that is, what degree polynomials on $\PP^3$ vanish on $C$. To start with, we can ask: does $C$ lie on a quadric surface? To answer this, we consider again the restriction map
$$
H^0(\cO_{\PP^3}(2)) \; \to \; H^0(\cO_{C}(2)) = H^0(\cO_{\PP^1}(8)).
$$
Here the vector space on the left---homogeneous quadratic polynomials on $\PP^3$---has dimension 10, while the one on the right, either by Riemann-Roch or by direct examination, has dimension 9. We conclude that \emph{the curve $C$ must lie on at least one quadric surface $Q \subset \PP^3$}.

Since $C$ is irreducible and nondegenerate, it can't lie on a union of planes, so the quadric $Q$ must either be smooth or a cone over a conic curve. We'll see in a moment that the latter case can't occur, so let's assume for now that $Q$ is smooth. 

The natural follow-up question is, what is the class of $C$ in the Picard group of $Q$? We know that $Q \cong \PP^1 \times \PP^1$, with the fibers of the two projections appearing as lines of the two rulings of $Q$. Lines $L$ and $M$ of the two rulings generate the Picard group, so that we must have $C \sim aL + bM$ for some $a, b$ (in other words, in terms of the isomorphism $Q \cong \PP^1 \times \PP^1$, $C$ is the zero locus of a bihomogeneous polynomial of bidegree $(a,b)$), and we ask what $a$ and $b$ are. The choices are limited: since $C$ is a quartic curve, we must have $a+b = 4$. Adjunction tells us which must be the case: the genus formula for curves on $Q$ tells us that the genus of a smooth curve of class $(a,b)$ on $Q$ has genus $(a-1)(b-1)$, whence the class of our curve $C$ must be $(1,3)$ (for a suitable ordering of the two rulings).

It follows in particular that \emph{$Q$ is the unique quadric containing $C$}. One way to see this is that since $C$ has class $(1,3)$ it meets the lines of the first ruling three times; if $Q'$ is any quadric containing $C$, then, it must contain all these lines and hence must equal $Q$. Alternatively, we may consider the exact sequence
$$
0 \to \cI_{C/Q}(2) \to \cO_Q(2)  \to \cO_C(2) \to 0.
$$
If $C$ has class $L+3M$, we have $\cI_{C/Q}(2) = \cO_{Q}(L-M)$. Since this bundle has negative degree of every line of the first ruling, it has no sections; hence the restriction map $H^0(\cO_Q(2))  \to H^0(\cO_C(2))$ is injective and so there are no  quadrics in $\PP^3$ containing $C$ other than $Q$.

(It is interesting to compare the two arguments above: they are exactly the same argument, expressed first in 19th century language and then in the language of the 20th century.)

We can also describe the rest of the ideal of $C$ similarly. For example, to find the cubic polynomials vanishing on $C$ we consider the restriction map
$$
H^0(\cO_{\PP^3}(3)) \; \to \; H^0(\cO_{C}(3)) = H^0(\cO_{\PP^1}(12)).
$$
The dimensions of these two vector spaces being 20 and 13 respectively, we see that $C$ must lie on at least 7 cubics; four of these are simply products of $Q$ with linear forms, and so we see that $C$ must lie on at least three cubics modulo those containing $Q$. Indeed, these are easy to spot: if $L$ and $L'$ are any two lines of the first ruling, the divisor $C + L + L'$ has class $(3,3)$ on $Q$ and hence is the intersection of $Q$ with a cubic surface. As $L+L'$ varies in a two-dimensional linear series, we get three cubics containing $C$ modulo those containing $Q$. Conversely, any cubic containing $C$ (but not containing $Q$) will intersect $Q$ in the union of $C$ with a curve of type $(2,0)$ on $Q$, which is to say the sum of two lines of the first ruling, so these are all the cubics containing $C$.

Finally, we have to show that the quadric containing the curve $C$ cannot be a cone over a conic plane curve. The key question here is whether or not $C$ contains the vertex $p$ of the cone: if not, the same adjunction-based calculation shows that $C$ must have genus 1; while a parity argument (how many times does $C$ meet a line of the ruling of $Q$?) shows that if a curve $C \subset Q$ of even degree contains $p$ it must be singular there.

Before moving on, we should remark that this one example of a non-linearly normal rational curve in projective space is misleading in that we can give such a complete description. For general $d$ and $r$, we have no idea what may be the Hilbert function of a rational curve of degree $d$ in $\PP^r$, let alone what its resolution might look like.


\begin{exercise}
Find all possible Hilbert functions of smooth rational quintic  curves $C \subset \PP^3$. (There are only two, depending on whether or not $C$ lies on a quadric, so this isn't so bad.)
\end{exercise}

\begin{exercise}
Every $g^3_4$ on $\PP^1$ is uniquely expressible as a sum of the $g_1^1$ and a $g^1_3$
\end{exercise}

\begin{exercise}
There is a 1-parameter family of rational quartic curves in $\PP^3$ up to projective equivalence. (Finding the invariants is a nice problem, which we should talk about.)
\end{exercise}


\section{Curves of genus 1}

Wonderful subject; refer to somewhere else. Double cover of $\PP^1$, leading to $y^2 - f(x)$. Plane cubic, quartic in $\PP^3$. Cheerful fact:  elliptic quintic is Pfaffian. Cheerful fact: any $g^5_6$ is the product of two $g^2_3$s. Get a $3\times 3$ matrix of linear forms. The image of the matrix and its transpose are $g^2_3$'s. Prove this by going to the Segre embedding $\PP^2\times \PP^2 \subset\PP^8$.

The subject of curves of genus 1, a.k.a. elliptic curves\footnote{Technically, an elliptic curve is a smooth curve of genus 1 with a distinguished point, called the \emph{origin}.} is a wonderful one. They appeared, in the second half of the 19th century, as key objects in the developing subjects of geometry, number theory and complex analysis, and the literature is correspondingly rich---the total number of journal and book pages devoted to the topic probably exceeds a million. Here we'll focus on the geometric side, and try to describe maps of genus 1 curves to projective space.

Two remarks are in order before we get underway. To begin with, unlike the case of genus 0 there are many different isomorphism classes of curves of genus 1; as we remarked in Section~\ref{**} and as we'll see shortly, there is a one-parameter family of them. Secondly, while there are many different line bundles of a given degree $d$ on a curve $E$ of genus 1---they are parametrized by the Jacobian, which is one-dimensional in this case---if $d \neq 0$ \emph{the automorphism group of $E$ acts transitively on them}. In other words, if $\phi, \phi' : E \to \PP^r$ are two maps given by complete linear series $|L|$ and $|L'|$ of degree $d$ on $E$, then there exists  automorphisms $\alpha : \PP^r \to \PP^r$ and $\beta : E \to E$ such that $\phi' \circ \beta= \alpha \circ \phi$. In particular, if $\phi$ and $\phi'$ are embeddings---as will be the case when $d \geq 3$---then their images are projectively equivalent.

\subsubsection{Double covers of $\PP^1$}

Let $E$ be a smooth projective curve of genus 1. If $L$ is any line bundle of degree 1 on $E$, Riemann-Roch says that $h^0(L) = 1$, so if we're looking for nonconstant maps to projective space we have to go to degree 2 and higher.

To start with, suppose $L$ is a line bundle of degree 2 on $E$. By Riemann-Roch, $h^0(L) = 2$ and the linear series $|L|$ is base point free, so we get a map $\phi : E \to \PP^1$ of degree 2. By Riemann-Hurwitz, the map $\phi$ will have 4 branch points; by the remark above, these four points are determined, up to automorphisms of $\PP^1$ by the curve $E$, and are independent of the choice of $L$.
After composing with an automorphism of $\PP^1$ we can take these four points to be $0, 1, \infty$ and $\lambda$ for some $\lambda \neq 0, 1 \in \CC$. Since there is a unique double cover of $\PP^1$ with given branch divisor (see~\ref{**}) it follows that $E \cong E_\lambda$, where $E_\lambda$ is the curve given by the affine equation
$$
y^2 = x(x-1)(x-\lambda).
$$

When are two curves $E_\lambda$ and $E_{\lambda'}$ isomorphic? By what we've said, this will be the case if and only if there is an automorphism of $\PP^1$ carrying the points $\{0,1,\infty,\lambda\}$ to $\{0,1,\infty,\lambda'\}$, in any order. This will be the case if and only if $\lambda$ and $\lambda'$ belong to the same orbit under the action of the group $G \cong S_3$ of automorphisms of $\PP^1$ permuting the three points $0, 1$ and $\infty$; that is, if
$$
\lambda' \in \{\lambda, \; 1-\lambda, \; \frac{1}{\lambda},\;  \frac{1}{1-\lambda}, \; \frac{\lambda - 1}{\lambda}, \; \frac{\lambda}{\lambda - 1} \}.
$$
Now, the quotient of $\PP^1$ by the action of $G$ is again isomorphic to $\PP^1$ by Luroth's theorem, which means that the field of rational functions on $\PP^1$ invariant under $G$ is again a purely transcendental extension $K(j)$; explicitly, we can take
$$
j \; = \; 256\cdot \frac{\lambda^2 - \lambda + 1}{\lambda^2(\lambda - 1)^2}.
$$
(the factor of 256 is there for arithmetic reasons). In any case, we see explicitly that there is a unique smooth projective curve of genus 1 for each value of $j$; in particular, the family of all such curves is parametrized by a curve.

\subsubsection{Plane cubics}

Moving from degree 2 to degree 3, let $L$ be a line bundle of degree 3 on $E$. We see from Corollary~\ref{degree 2g+1 embedding} that the sections of $L$ give an embedding of $E$ as a smooth plane cubic curve; conversely, the genus formula tells us that a smooth plane cubic curve indeed has genus 1. 

We won't delve into the geometry of plane cubics, except to point out that once more we can use this representation to argue that the isomorphism classes of elliptic curves form a 1-dimensional family. To see this, observe that the space of homogeneous polynomials of degree 3 in three variables is 10-dimensional, and the space of plane cubic curves is correspondingly parametrized by  $\PP^9$; the locus of smooth curves is a Zriski open subset of this $\PP^9$. On the other hand, by what we've said, two plane cubics are isomorphic iff they are congruent under the group $PGL_3$ of automorphisms of $\PP^2$. Since the group $PGL_3$ has dimension 8, we would expect that the family of such curves up to isomorphism has dimension 1.

\subsubsection{Quartics in $\PP^3$} 

Onwards! Let $E$ again be a smooth projective curve of genus 1, and consider now the embedding of $E$ into $\PP^3$ given by the sections of a line bundle $L$ of degree 4. The first question we might ask is what polynomial equations in $\PP^3$ cut out the image, and as before we'll do this by looking at the restriction map
$$
\rho_2 \;  : \; H^0(\cO_{\PP^3}(2)) \; \to \; H^0(\cO_{E}(2)) = H^0(L^2).
$$
The space on the right---the space of homogeneous polynomials of degree 2 in four variables---has dimension 10, while by Riemann-Roch the space $H^0(L^2)$ has dimension 8. It follows that $E$ lies on at least two linearly independent quadrics $Q$ and $Q'$. Since $E$ does not lie in any plane, neither $Q$ nor $Q'$ can be reducible; thus by Bezout we see that
$$
E = Q \cap Q'
$$
is the complete intersection of two quadrics in $\PP^3$. Moreover, we also see from the Noether AF+BG theorem that the kernel of $\rho_2$ is exactly the span of $Q$ and $Q'$. Thus $E$ determines a point in the Grassmannian $G(2, H^0(\cO_{\PP^3}(2))) = G(2, 10)$ of pencils of quadrics; and by Bertini a Zariski open subset of that Grassmannian correspond to smooth quartic curves of genus 1. We can use this to once more calculate the dimension of the family of curves of genus 1: the Grassmannian $G(2,10)$ has dimension 16, while the group $PGL_4$ of automorphisms of $\PP^3$ has dimension 15, so we may conclude that the family of curves of genus 1 up to isomorphism has dimension 1.

\subsubsection{Projective normality II}

Observe that last two cases (cubic and quartic genus 1 curves) are projectively normal; extend this to arbitrary smooth complete intersections.

Exercise: $C \subset Q \subset \PP^3$ of class $(a,b)$ is projectively normal iff $|a-b| \leq 1$.

\section{Curves of genus 2}

Canonical map to $\PP^1$. Embedding in $\PP^3$ as $(2,3)$ on a quadric, via any degree 5 line bundle. Ideal is 1 quadric, 2 cubics.
Plane model of degree 4 with node or cusp.

\subsubsection{Representations as double covers of $\PP^1$}

As with curves of genus 1, there are no nontrivial linear series of degree 0 or 1 on a curve of genus 2; the first positive-dimensional linear series occurs in degree 2. Unlike the case of genus 1, however, this series is unique: by Riemann-Roch, if $D$ is any divisor of degree 2 on a curve $C$ of genus 2, we have
$$
h^0(D) = 1 + h^1(D) = 1 + h^0(K-D);
$$
since $K-D$ has degree 0, this says that $h^0(D) > 1$ if and only if $D=K$, in which case $|D| = |K|$ is the canonical $g^1_2$ on $C$.

The canonical series gives a map $\phi_K : C \to \PP^1$ expressing $C$ as a double cover of $\PP^1$; as in the case of genus 1, this means we can realize $C$ as the smooth projective compactification of the affine curve given by
$$
y^2 = x(x-1)(x - \alpha)(x - \beta)(x - \gamma)
$$
for some triple $\alpha,\beta,\gamma \in \CC$ distinct from each other and from 0 and 1. This representation shows us that the moduli space $M_2$ is the space of 6-tuples of distinct points in $\PP^1$ modulo the action of $PGL_2$. This tells us immediately that $M_2$ is irreducible of dimension 3; with a fair amount of additional work, we can also use this to describe the coordinate ring of $M_2$ (\ref{**}).

\subsubsection{Embeddings in $\PP^3$}

For line bundles $L$ of degree $d \geq 3$ on $C$, Riemann-Roch tells us simply that $h^0(D) = d - 1$; if we want to embed our curve $C$ in projective space, accordingly, we had better take $d \geq 5$. Conversely, Corollary~(\ref{degree 2g+1 embedding}) tells us that any line bundle of degree 5 on $C$ is very ample, so we'll consider first the embeddings of $C$ given by those.

So: for the following, let $L$ be any line bundle of degree 5 on our curve $C$, and $\phi_L : C \to \PP^3$ the embedding given by the complete linear system $|L|$. By a mild abuse of language, we'll also denote the image $\phi_L(C) \subset \PP^3$ by $C$.

The first question to ask is once more, what degree surfaces in $\PP^3$ contain the curve $C$? We start with degree 2, where we consider the restriction map
$$
H^0(\cO_{\PP^3}(2)) \to H^0(\cO_C(2)) = H^0(L^2).
$$
The space on the left has dimension 10 as always; on the right, Riemann-Roch tells us that $h^0(L^2) = 2\cdot5 - 2 + 1 = 9$. It follows that $C$ must lie on a quadric surface $Q$; and by Bezout that $Q$ is unique (since $C$ can't lie on a union of planes, any quadric containing $C$ must be irreducible; if there were more than one such, Bezout would imply that $\deg(C) \leq 4$).

We might ask at this point: is $Q$ smooth or a quadric cone? The answer depends on the choice of line bundle $L$:

\begin{proposition}
Let $C \subset \PP^3$ be a smooth curve of degree 5 and genus 2 and $Q \subset \PP^3$ the unique quadric containing $C$. If $L = \cO_C(1) \in \pic^5(C)$, then $Q$ is singular if and only if we have
$$
L \cong K^2(p)
$$
for some point $p \in C$.
\end{proposition}

(Note that there is a 2-parameter family of line bundles of degree 5 on $C$ **we don't know this yet, unless we want to state it somewhere**, of which a one-dimensional subfamily are of the form $K^2(p)$, conforming to our naive expectation that ``in general" $Q$ should be smooth, and that it should become singular in codimension 1.)

\begin{proof}
First, suppose that the line bundle $L \cong K^2(p)$ for some $p \in C$. Then $L(-p) \cong K^2$, meaning that the map $\pi : C \to \PP^2$ given by projection from $p$ is the map $\phi_{K^2} : C \to \PP^2$ given by the square of the canonical bundle.

What does this map look like?
\end{proof}

Whether the quadric $Q$ is smooth or not, we can describe a minimal set of generators of the homogeneous ideal $I(C) \subset \CC[x_0, x_1, x_2, x_3]$ similarly. First, we look at the restriction map
$$
H^0(\cO_{\PP^3}(3)) \to H^0(\cO_C(3));
$$
since the dimensions of these spaces are 20 and $15-2+1 = 14$ respectively, we see that  vector space of cubics vanishing on $C$ has dimension at least 6. Four of these are already accounted for: we can take the defining equation of $Q$ and multiply it by any of the linear forms on $\PP^3$; we conclude, accordingly, that \emph{there are at least two cubics vanishing on $C$ linearly independent modulo those vanishing on $Q$}.

In fact, we can prove the existence of these cubics geometrically, and show that there are no more than 2 linearly independent modulo the ideal of $Q$. Suppose first that $Q$ is smooth, so that $C$ is a curve of type $(2,3)$ on $Q$. In that case, if $L \subset Q$ is any line of the first ruling, the sum $C+L$ is the complete intersection of $Q$ with a cubic $S_L$, unique modulo the ideal of $Q$; conversely, if $S$ is any cubic containing $C$ but not containing $S$, the intersection $S \cap Q$ will be the union of $C$ and a line $L$ of the first ruling; thus, mod $I(Q)$, $S = S_L$. A similar argument applies in case $Q$ is a cone, and $L$ is any line of the (unique) ruling of $Q$.

\begin{exercise}
Show that for any pair of lines $L, L'$ of the appropriate ruling of $Q$, the three polynomials $Q$, $S_L$ and $S_{L'}$ generate the homogeneous ideal $I(C)$. Find relations among them. Write out the minimal resolution of $I(C)$.
\end{exercise}

\subsubsection{Projective normality III}

\begin{theorem}
 Let $C$ be a smooth (is reduced, irreducible enough?) curve of arithmetic genus $g$, and let $\cL$ be a line bundle on $C$ of degree $\geq 2g+1$. The image of 
 $C$ under the complete linear series $|\cL|$ is projectively normal ( when $C$ is singular, aritmetically Cohen-Macaulay).
\end{theorem}

\begin{proof}
 The line bundle $\cL$ is very ample by \ref{?}. Thus it suffices We must show that the multiplication map $H^0(\sL)\otimes H^0(\cL^{m}) \to H^0(\cL^{m+1})$ is surjective for all $m\geq 1$.
 For $m=1$ do it by number of quadrics, uniform position. For m>1 the bpf pencil trick.
\end{proof}

\section{Curves of genus 3}

This will be, perhaps somewhat counter-intuitively, the shortest of the sections in this chapter. The reason is simple: for a non-hyperelliptic curve of genus 3, the canonical model is virtually the only one we will deal with. For curves $C$ of other genera, different representations of $C$---as a branched cover of $\PP^1$, as the normalization of a plane curve $C_0 \subset \PP^2$, as embedded in $\PP^3$ and higher-dimensional projective spaces---display different aspects of the geometry of the curve; and it's correspondingly valuable to understand all these different models of $C$ and their relation to one another. For a non-hyperelliptic curve of genus 3, by contrast, the canonical embedding is  the only one we deal with; virtually all the aspects of the geometry of $C$ are best seen in this model.

So: let $C$ be a smooth projective curve of genus 3. The is an immediate bifurcation into two cases, hyperelliptic and non-hyperelliptic curves; we will discuss hyperelliptic curves of any genus in Section~\ref{**}, and so for the following we'll assume $C$ is nonhyperellitic. By our general theorem~\ref{**}, this means that the canonical map $\phi_K : C \to \PP^2$ embeds $C$ as a smooth plane quartic curve; and conversely, by adjunction any smooth plane of degree 4 has genus 3 and is canonical (that is, $\cO_C(1) \cong K_C$).

Note that this gives us a way to determine the dimension of the moduli space $M_3$ of smooth curves of genus $3$: if $\PP^{14}$ is the space of all plane quartic curves, and $U \subset \PP^{14}$ the open subset corresponding to smooth curves, we have a dominant map $U \to M_3$ whose fibers are isomorphic to the 8-dimensional affine group $PGL_3$. (Actually, the fiber over a point $[C] \in M_3$ is isomorphic to the quotient of $PGL_3$ by the automorphism group of $C$; but since $Aut(C)$ is finite this is still 8-dimensional.) We conclude, therefore, that
$$
\dim M_3 = 14 - 8 = 6.
$$

What about other linear series on $C$, and the corresponding models of $C$? To start with, by hypothesis $C$ has no $g^1_2$s; that is, it is not expressible as a 2-sheeted cover of $\PP^1$. On the other hand, it is expressible as a 3-sheeted cover: if $L \in \pic^3(C)$ is a line bundle of degree 3, by Riemann-Roch we have
$$
h^0(L) = 
\begin{cases}
2, &\text{if $L \cong K-p$ for some point $p \in C$; and} \\
1 &\text{otherwise.}
\end{cases}
$$
There are thus a 1-dimensional family of representations of $C$ as a 3-sheeted cover of $\PP^1$. In fact, these are plainly visible from the canonical model: the degree 3 map $\phi_{K-p} : C \to \PP^1$ is just the composition of the canonical embedding $\phi_K : C \to \PP^2$ with the projection from the point $p$.

There are of course other representations of $C$ as the normalization of a plane curve. By Riemann-Roch, $C$ will have no $g^2_3$s and the canonical series is the only $g^2_4$, but there are plenty of models as plane quintic curves: by Proposition~\ref{**}, if $L$ is any line bundle of degree 5, the linear series $|L|$ will be a base-point-free $g^2_5$ as long as $L$ is not of the form $K+p$, so that $\phi_L$ maps $C$ birationally onto a plane quintic curve $C_0 \subset \PP^2$. But these can also be described geometrically in terms of the canonical model: any such line bundle $L$ is of the form $2K-p-q-r$ for some trio of  points $p, q, r \in C$ that are not colinear in the canonical model, and we see correspondingly that $C_0$ is obtained from the canonical model of $C$ by applying a Cremona transform with respect to the points $p, q$ and $r$. 

We can also embed $C$ in $\PP^3$ as a smooth sextic curve by Proposition~\ref{**}; in fact, a line bundle $L \in \pic^6(C)$ of degree 6 will be very ample if and only if it is not of the form $K+p+q$ for any $p, q \in C$. One cheerful fact in this connection is that these curves are determinantal:

\begin{exercise}
Let $C \subset \PP^3$ be a smooth non-hyperelliptic curve of degree 3 and genus 6. Show that there exists a $3 \times 4$ matrix $M$ of linear forms on $\PP^3$ such that 
$$
C = \{ p \in \PP^3 \mid \rank(M(p)) \leq 2 \}.
$$
\end{exercise}

\section{Curves of genus 4}

As in the case of curves of genus 3, the study of curves of genus 4 bifurcates immediately into two cases: hyperelliptic and non-hyperelliptic; again, we will study the geometry of hyperelliptic curves in Chapter~\ref{****} and focus here on the nonhyperelliptic case.

In genus 4 we have a question that the elementary theory based on the Riemann-Roch formula cannot answer: are nonhyperelliptic curves of genus 4 expressible as three-sheeted covers of $\PP^1$? The answer will emerge from our analysis in Proposition~\ref{genus 4 trigonal} below.

Let $C$ be a non-hyperelliptic curve of genus 4. We start by considering the canonical map $\phi_K : C \hookrightarrow \PP^3$, which embeds $C$ as a curve of degree 6 in $\PP^3$. We identify $C$ with its image, and investigate the homogeneous ideal $I = I_C$ of equations it satisfies. As in previous cases we may try to answer this by considering the restriction maps
\fix{replaced $K_C^m$ with $mK_C$.}
$$
r_m : \HH^0(\cO_{\PP^3}(m)) \; \to \; \HH^0(\cO_{C}(m)) = \HH^0(mK_C).
$$

For $m=1$, this is by construction an isomorphism; that is, the image of $C$ is non-degenerate (not contained in any plane).

For $m=2$ we know that $\h^0(\cO_{\PP^3}(2)) = \binom{5}{3} = 10$, while by the Riemann-Roch
Theorem we have
$$
\h^0(\cO_C(2)) = 12 - 4 + 1 = 9.
$$
This shows that the curve $C \subset \PP^3$ must lie on at least one quadric surface $Q$. The quadric $Q$ must be irreducible, since any any reducible and/or non-reduced quadric must be a union of planes, and thus cannot contain an irreducible non-degenerate curve.
If $Q'\neq Q$ is any other quadric then, by B\'ezout's Theorem, $Q\cap Q'$ is a curve of degree 4 and thus could not contain $C$. From this we see that $Q$ is unique, and it follows that $r_2$ is surjective.

What about cubics? Again we consider the restriction map
$$
r_3 : \HH^0(\cO_{\PP^3}(3)) \; \to \; \HH^0(\cO_{C}(3)) = \HH^0(3K_C).
$$
The space $\HH^0(\cO_{\PP^3}(3))$ has dimension $\binom{6}{3} = 20$, while  the Riemann-Roch Theorem shows that
$$
\h^0(\cO_C(3)) = 18 - 4 + 1 = 15.
$$
It follows that the ideal of $C$ contains at least a 5-dimensional vector space of cubic polynomials. We can get a 4-dimensional subspace as products of the unique quadratic polynomial $F$ vanishing on $C$ with linear forms---these define the cubic surfaces containing $Q$. Since $5 > 4$ we  conclude that the curve $C$ lies on at least one cubic surface $S$  not containing $Q$. 
B\'ezout's Theorem shows that the curve $Q \cap S$ has degree 6; thus it must be equal to $C$. 

Let $G=0$ be the cubic form defining the surface $S$. By Lasker's Theorem the ideal $(F,G)$ is unmixed, and thus is equal to the homogeneous ideal of $C$. Putting this together, we have proven the first statement of the following result:

\begin{theorem}
The canonical model of any nonhyperelliptic curve of genus 4 is a complete intersection of a quadric $Q = V(F)$ and a cubic surface $S = V(G)$ meeting along nonsingular points of each. Conversely, any smooth curve that is the intersection of a quadric and a cubic surface in $\PP^3$ is the canonical model of a nonhyperelliptic curve of genus 4.
\end{theorem}
 
\begin{proof}
Let $C = Q\cap S$ with $Q$ a quadric and $S$ a cubic. Because $C$ is nonsingular and a complete intersection, both $S$ and $Q$ must be nonsingular at every point of their intersection Applying the Adjunction Formula to $Q\subset \PP^3$ we get
$$
\omega_Q = (\omega_{\PP^3} \otimes \cO_{\PP^3}(2))|_Q = \cO_Q(-4+2) = \cO_Q(-2).
$$
Applying it again to $C$ on $Q$, and noting that $\O_Q(C) = \O_Q(3)$, we get
$$
\omega_C = ((\omega_{Q} \otimes \cO_{3}(3))|_C = \cO_C(-2+3) = \cO_C(1)
$$
as required. 
\end{proof}

We can now answer the question we asked at the outset, whether a nonhyperelliptic curve of genus 4 can be expressed as a three-sheeted cover of $\PP^1$. This amounts to asking if there are any divisors $D$ on $C$ of degree 3 with $r(D) \geq 1$; since we can take $D$ to be a general fiber of a map $\pi : C \to \PP^1$, we can for simplicity assume $D = p+q+r$ is the sum of three distinct points.

By the geometric Riemann-Roch theorem, a divisor $D = p+q+r$ on a canonical curve $C \subset \PP^{g-1}$ has $r(D) \geq 1$ if and only if the three points $p,q,r \in C$ are colinear. If three points $p,q,r \in C$ lie on a line $L \subset \PP^3$ then the quadric $Q$ would meet $L$ in at least three points, and hence would contain $L$. Conversely,  if $L$ is a line contained in $Q$, then the divisor $D = C \cap L = S \cap L$ on $C$ has degree  3. Thus we can answer our question in terms of the family of lines contained in $Q$.

Any smooth quadric is isomorphic to $\PP^1\times \PP^1$, and contains two families of lines, or \emph{rulings}. On the other hand, any singular quadric is a cone over a plane conic, and thus has just one ruling. By the argument above, the pencils of divisors on $C$ cut out by the lines of these rulings are the $g^1_3$s on $C$. This proves:

\begin{proposition}\label{genus 4 trigonal}
A nonhyperelliptic curve of genus 4 may be expressed as a 3-sheeted cover of $\PP^1$ in either one or two ways, depending on whether the unique quadric containing the canonical model of the curve is singular or smooth.
\end{proposition}

\fix{include this?} (One might ask why the non-singularity of the cubic surface $S$ plays no role. However, $G$ is determined only up to a multiple of $F$, and it follows that the linear series of cubics in the ideal
$I_C$ has only base points along $C$. Bertini's Theorem says that a general element of this series will be nonsingular away from $C$; and since any every irreducible cubic in the family must be nonsingular along $C$, it follows that the general such cubic is nonsingular.)

A curve expressible as a 3-sheeted cover of $\PP^1$ is called \emph{trigonal}; by the analyses of the preceding sections, we have shown that \emph{every curve of genus $g \leq 4$ is either hyperelliptic or trigonal}. 

We can also describe the lowest degree plane models of nonhyperelliptic curves $C$ of genus 4. 
We can always get a plane model of degree 5 by projecting $C$ from a point $p$ of the canonical model of $C$. Moreover, the Riemann-Roch Theorem shows that if $D$ is a divisor of degree 5 with $r(D)=2$ then,  $\h^0(K-D) = 1$. Thus $D$ is of the form $K-p$ for some point $p \in C$, and the map to $\PP^2$ corresponding to $D$ is $\pi_p$. These  maps $\pi_p: C\to \PP^2$ have the lowest possible degree (except for those whose image is  contained in a line) because, by Clifford's Theorem a nonhyperelliptic curve of genus 4 cannot have a $g^2_4$.

We now consider the singularities of the plane quintic $\pi_p(C)$. Suppose as above that $C = Q\cap S$, with $Q$ a quadric. If a line $L$ through $p$ meets $C$ in $p$ plus a divisor of degree $\geq 2$ then, as we have seen, $L$ must lie in $Q$.  All other lines through $p$ meet $C$ in at most a single points, so $\pi_p$ whose images are thus nonsingular points of $\pi(C)$, and $\pi_C$ is one-to-one there. Moreover, a line that met $C$ in $>3$ points would have to lie in both the quadric and the cubic containing $C$, and therefore would be contained in $C$. Since $C$ is irreducible there can be no such line.

We distinguish two cases:

\begin{enumerate}
\item $Q$ is nonsingular:
In this case there are two lines $L_1, L_2$ on $Q$ that pass through $p$; they meet $C$ in $p$ plus divisors $E_1$ and $E_2$ of degree 2. If $E_i$ consists of distinct points, then, since the tangent planes to the quadric along $L_i$ are all distinct $\pi(C)$ will have a node at their common image. 

\fix{do we expect the reader to know this about quadrics, or should we prove it? or should we argue that since there are two distinct branches and a plane quintic of genus 2 can have only the equivalent of two double points, these must be simple?? The first option is probably better.}

On the other hand, if $E_i$ consists of a double point $2q$ (that is, $L_i$ is tangent to $C$ at $q\neq p$, or meets $C$ 3 times at $q = p$), then $\pi(C)$ will have a cusp at the corresponding image point. 
In either case, $\pi(C)$ has two distinct singular points, each either a node or a cusp. The two $g^1_3$s on $C$ correspond to the projections from these singular points.

\item $Q$ is a cone:
In this case, since the curve cannot pass through the singular point of $Q$ there is a unique line $L\subset Q$ that passes through $p$. Let $p+E$ be the divisor on $C$ in which this line meets $C$. The tangent planes to $Q$ along $L$ are all the same. Thus if $E = q_1+q_2$ consists of two distinct points, the image $\pi_p(C)$ will have two smooth branches sharing a common tangent line at
$\pi_p(q_1) = \pi_p(q_2)$. Such a point is called a \emph{tacnode} of $\pi_p(C)$. On the other hand, if $E= 2q$, that is, if $L$ meets $C$ tangentially at one point $q\neq p$ (or meets $C$ 3 times at $p$) then the image curve will have a higher order cusp, called a \emph{ramphoid cusp}. In either case, the one $g^1_3$ on $C$ is the projection from the unique singular point of $\pi(C)$.
\end{enumerate}

\fix{add pictures illustrating some of the possibilities above.}


\section{Curves of genus 5}

We consider now nonhyperelliptic curves of genus 5. There are now two questions that cannot be answered by simple application of the Riemann-Roch Theorem:

\begin{enumerate}
\item Is $C$ expressible as a 3-sheeted cover of $\PP^1$? In other words, does $C$ have a $g^1_3$?
\item Is $C$ expressible as a 4-sheeted cover of $\PP^1$? In other words, does $C$ have a $g^1_4$?
\end{enumerate}

As we'll see, all other questions about the existence or nonexistence of linear series on $C$ can be answered by the Riemann-Roch Theorem.

As in the preceding case, the answers can be found through an investigation of the geometry of the canonical model $C \subset \PP^4$ of $C$. This is an octic curve in $\PP^4$, and as before the first question to ask is what sort of polynomial equations define $C$. We start with quadrics, by considering the restriction map
$$
r_2 : \HH^0(\cO_{\PP^4}(2)) \; \to \; \HH^0(\cO_{C}(2)).
$$
On the left, we have the space of homogeneous quadratic polynomials on $\PP^4$, which has dimension $\binom{6}{4} = 15$, while by the Riemann-Roch Theorem the target is a vector space of dimension
$$
2\cdot8 - 5 + 1 = 12.
$$
We deduce that $C$ lies on at least 3 independent quadrics. We will see in the course of the following analysis that it is exactly 3; that is, $r_2$ is surjective.) Since $C$ is irreducible and, by construction, does not lie on a hyperplane, each of the quadrics containing $C$ is irreducible, and thus the intersection of any two is a surface of degree 4. There are now two possibilities:  The intersection of (some) three quadrics $Q_1 \cap Q_2 \cap Q_3$ containing the curve is 1-dimensional; or every such intersection is two dimensional. 

We first consider the case where $Q_1 \cap Q_2 \cap Q_3$ is 1-dimensional. By the principal ideal theorem the intersection has no 0-dimensional components. By B\'ezout's Theorem the intersection is a curve of degree 8, and since $C$ also has degree 8 we must have $C=Q_1 \cap Q_2 \cap Q_3$. Lasker's Theorem then shows that the three quadrics $Q_i$ generate the whole homogeneous ideal of $C$.

We can now answer the first of our two questions for curves of this type. As in the genus 4 case the geometric Riemann-Roch Theorem implies that $C$ has a $g^1_3$ if and only if the canonical model of $C$ contains 3 colinear points or, more generally, meets a line $L$ in a divisor of 3 points. When $C$ is the intersection of quadrics, this cannot happen, since the line $L$ would have to be contained in all the quadrics that contain $C$ and $L\subset C$, which is absurd. Thus, in this case, 
$C$ has no $g^1_3$.

What about $g^1_4$s? Again invoking the geometric Riemann-Roch Theorem, a divisor of degree 4 moving in a pencil lies in a 2-plane; so the question is, does $C \subset \PP^4$ contain a divisor of degree 4, say $D = p_1+\dots +p_4 \subset C$, that lies in a plane $\Lambda$? Supposing this is so, we consider the restriction map
$$
\HH^0(\cI_{C/\PP^4}(2)) \; \to \; \HH^0(\cI_{D/\Lambda}(2)).
$$
By hypothesis, the left hand space is 3-dimensional; but any four noncolinear points in the plane  impose independent conditions on quadrics, \fix{this is a scheme of length 4; how is the reader supposed to cope with this if we don't assume the notion of a scheme, at least a finite one? And does the reader really know this fact about schemes of length 4 in the plane?} so that the right hand space is 2-dimensional. It follows that \emph{$\Lambda$ must be contained in one of the quadrics $Q$ containing $C$}. 

The quadrics in $\P^4$ that contain 2-planes are exactly the singular quadrics: such a quadric is a cone over a quadric in $\P^3$, and it is ruled by the (one or two) families of 2-planes it contains, which are the cones over the (one or two) rulings of the quadric in $\P^3$. The argument above shows that the existence of a $g_4^1$s on $C$ in this case implies the existence of a singular quadric containing $C$.

Conversely, suppose that $Q \subset \PP^4$ is a singular quadric containing $C = Q_1 \cap Q_2 \cap Q_3$. Now say $\Lambda \subset Q$ is  a 2-plane. If $Q'$ and $Q''$ are ``the other two quadrics" containing $C$, we can write
$$
\Lambda \cap C = \Lambda \cap Q' \cap Q'', 
$$ 
from which we see that $D = \Lambda \cap C$ is a divisor of degree 4 on $C$, and so has $r(D) = 1$ by the geometric Riemann-Roch Theorem. Thus, the rulings of  singular quadrics containing $C$ cut out on $C$ pencils of degree 4; and every pencil of degree 4 on $C$ arises in this way.

Does $C$ lie on singular quadrics? There is a $\PP^2$ of quadrics containing $C$---a 2-plane in the space $\PP^{14}$ of quadrics in $\PP^4$---and the family of singular quadrics  consists of a  hypersurface of degree 5 in $\PP^{14}$--called the \emph{discriminant} hypersurface. By Bertini's Theorem, not every quadric containing $C$ is singular. Thus the set of singular quadrics containing $C$ is a plane curve $B$ cut out by a quintic equation. So $C$ does indeed have a $g^1_4$, and is expressible as a 4-sheeted cover of $\PP^1$. In sum, we have proven:

\begin{proposition}
Let $C \subset \PP^4$ be a canonical curve, and assume $C$ is the complete intersection of three quadrics in $\PP^4$. Then $C$ may be expressed as a 4-sheeted cover of $\PP^1$ in a one-dimensional family of ways, and there is a map from the set of $g^1_4$s on $C$ to a plane quintic curve $B$, whose fibers have cardinality 1 or 2.
\end{proposition}

\fix{could the "quintic curve" be reducible/multiple? Just a line?}
Of course, we can go further and ask about the geometry of the plane curve $B$ and how it relates to the geometry of $C$; a fairly exhaustive list of possibilities is given in \cite{****} [ACGH]. But that's enough for now.

In the second possibility above, that the canonical curve $C \subset \PP^4$ is not a complete intersection; we will see in *** that the
 the intersection of the quadrics containing $C$ is two-dimensional: a rational normalscroll;  and  $C$ is trigonal, that is, a 3-sheeted cover of $\PP^1$. 

\section{Curves of genus 6}
Canonical model lies on at least 6 quadrics. 

To prove projective quadratic normality,  use general position: the general hyperplane section is 10 points in $\PP^4$ 8 of them lie on the union of two hyperplanes -- which won't contain the rest -- so they impose exactly 9 conditions. 

Prove monodromy of hyperplane sections is the symmetric group. Do this carefully. Explain the correspondence between monodromy and Galois theory. 

Deduce projective normality from quadratic normality.

At this point, we're stuck: we still don't know what linear series exist on our curve, or much about the geometry of the canonical model. But if we invoke Brill-Noether, we have both: the curve has a $g^2_6$, which gives us a plane model as a sextic (with only double points, since no $g^1_3$s); the canonical series on the curve is cut out by cubics passing through the double points, which embeds the (blow-up of the) plane as a del Pezzo surface in $\P^5$, of which the canonical curve is a quadric section. Also, use the count of $g^2_6$s on $C$ to deduce the uniqueness of the del Pezzo.

\input footer.tex


