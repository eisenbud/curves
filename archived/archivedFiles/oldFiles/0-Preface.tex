\input header.tex

\setcounter{chapter} {-1}
\chapter{Basic Questions}

\fix{The following is more material for a preface than a preface...}
\begin{center}
\emph{I'm very well acquainted, too, with matters mathematical,\\
I understand equations, both the simple and quadratical,\\
About binomial theorem I am teeming with a lot o' news,\\
With many cheerful facts about the square of the hypotenuse.}

---Gilbert and Sullivan, Pirates of Penzance, Major General's Song


\emph{Be simple by being concrete. Listeners are prepared to
accept unstated (but hinted) generalizations much more than they are able, on the spur of the moment, to
decode a precisely stated abstraction and to re-invent the special cases that motivated it in the first place. }

--Paul Halmos, How to Talk Mathematics

\emph{Another damned thick book! Always scribble, scribble, scribble! Eh, Mr. Gibbon?} --- \scriptsize{Prince William Henry, upon receiving the second  volume of The History of the Decline and Fall of the Roman Empire from the author.}
\end{center}


The most primitive objects of algebraic geometry are affine algebraic sets---subsets of $\RR^{n}$ or $\CC^{n}$ defined by the vanishing of polynomial functions---and the maps between them. But already in the first half of the 19th century geometers realized that there was a great advantage in working with varieties in complex projective space, treating affine varieties as projective varieties minus the intersection with the plane at infinity and real varieties as the fixed points of the complex involution. One sees this in the simplest examples: the ellipses, hyperbolas and parabolas in the real affine plane are all the same in the complex projective plane; the difference is only in how they intersect the line at infinity. A difficulty with the projective point of view is that on a connected projective variety there are no non-constant functions at all (reason: a function on a projective variety is a map to the affine line; since the image of a projective variety is again projective, the image would be a single point.) 

Starting with Riemann in the 1860s and culminating in the scheme theory of Grothendieck in the 1950s, algebraic varieties were treated in a way independent of any embedding: An algebraic variety is is a topological space with a sheaf of locally
defined polynomial functions. Many interesting aspects of geometry have to do not with single abstract varieties, but with maps between them, and in particular with embeddings in projective spaces. In general, maps between varieties can be described by their graphs, which are again varieties.  But for the special case of maps to projective spaces, the theory of \emph{linear series} is usually a more convenient description. The collection of all linear series on a variety reflects some of its best understood invariants. 

The basic objects of study in this book are smooth, connected projective algebraic curves over an algebraically closed field of characteristic 0, which we take to be the complex numbers $\CC$. Though we assume that the reader has been exposed to this theory in some form before, perhaps from Chapter IV of Hartshorne's {\it Algebraic Geometry}, we will review the elements  in the form we will use. 

\section{Algebraic Curves and Riemann Surfaces}

These objects can be viewed in two distinct but equivalent ways: as \emph{compact Riemann surfaces}, or compact complex manifolds of dimension 1; and as \emph{smooth projective algebraic curves over $\CC$}. (Here, when we use the term projective variety, we mean a variety isomorphic to a closed subset of projective space, not a variety with a specified embedding in $\PP^n$.) There are advantages to each point of view---the complex analytic point of view is more concrete, and requires relatively minimal preliminaries; the algebraic point of view is substantially broader. 

First, if $C \subset \PP^n$ is a smooth, projective curve over $\CC$, then it is a submanifold of complex projective space, and so a Riemann surface. 
\fix{discuss geometric genus vs. arithmetic genus here?}

The other direction---going from a compact Riemann surface $C$ to a smooth projective curve over $\CC$, or equivalently embedding $C$ as a complex submanifold of $\PP^n$, after which Chow's theorem says that it is in fact a projective variety---is much deeper. The first, and hardest step is to show that a compact Riemann surface admits a nonconstant meromorphic function $f:C \to \CC$, and the corresponding statement is not true in higher dimensions. The function $f$ can be viewed as
a rational map $f': C\to \PP^{1}$. The next step is to see that the field $K(C)$ of all meromorphic functions
on $\CC$ is a finite extension of the field of rational functions on $\PP^{1}$; the sheaf of regular functions on $C$ is then the integral closure of the sheaf of regular functions on $\PP^{1}$ in $K(C)$.

Though equivalent for curves defined over $\CC$, these approaches have a very different flavors. For example,
given a map  $f: C\to C'$ from a smooth curve $C$ to a possibly singular curve $C'$ that is generically one-to-one, we can reconstruct $C$.
From the algebraic point of view this can be done by \emph{normalization}, or more concretely by blowing up the singular points of $C'$. From the analytic point of view, we can use the Weierstrass preparation theorem, which implies that there is the neighborhood $U$ of any point $p\in C'$ such that the punctured neighborhood $U \setminus p$ is isomorphic to a disjoint union of punctured discs; and $C$ is obtained by completing this to the corresponding disjoint union of discs.

\section{Families of varieties}

\subsection{Hilbert schemes}

\subsubsection{Definition, universal property; construction}

\subsubsection{examples of hypersurfaces and linear spaces}

\subsubsection{tangent space}

\subsubsection{Fundamental problem: irreducible components of Hilb parametrizing smooth curves and their dimensions}

\begin{example}
conics in $\PP^3$ (refer to 3264)
\end{example}

\subsection{Moduli spaces of curves}

\subsubsection{basic properties of $M_g$ (coarse rather than fine; fine over automorphism-free curves)}

\subsubsection{dimension $3g-3$, irreducible}  (just statements, w/ref to Harris-Morrison)



\section{Moduli problems}

It is a fundamental aspect of algebraic geometry that the objects we deal with often vary in families, and can often be parametrized by a ``universal" such family. For example, the family of plane curves of degree $d$ may be thought of as the projective
space $\PP(H^{0} \sO_{\PP^{2}}(d))$, and similarly with hypersurfaces in any projective space. This notion of objects varying with parameters underlies many of the constructions and theorems we will discuss. 

\subsection{What is a moduli problem?}

Briefly, a \emph{moduli problem} consists of two things: a class of objects, or isomorphism classes of objects; and a notion of what it means to have a \emph{family} of these objects parametrized by a given scheme $B$. To make this relatively explicit, the four main examples of moduli problems we'll be discussing here are:

\begin{enumerate}
\item  smooth curves: objects are isomorphism classes  of smooth, projective curves $C$ of a given genus $g$. A family over $B$ is a subscheme $\cX \subset B \times \PP^r$, smooth, over $B$, whose fibers are curves of genus $g$.

\item the Hilbert scheme: objects are subchemes of $ \PP^r$ with a given Hilbert polynomials. A family  is a subscheme $\cX \subset B \times \PP^r$, with $cX$ flat over $B$, whose fibers have the given Hilbert polynomial. We will be interested in the case of Hilbert polynomial $p(m) = dm-g+1$ and the open subscheme corresponding to smooth projective curves $C \subset \PP^r$ of degree $d$ and genus $g$.

\item effective divisors on a given curve: objects are effective divisors of a given degree $d$ on a given smooth, projective curve $C$. A family over $B$ will be a subscheme $\cD \subset B \times C$ flat over $B$, with fibers of degree $d$

\item invertible sheaves on a given curve $C$: objects are invertible sheaves of a given degree $d$ on $C$. A family over $B$ is an invertible sheaf on the product $B \times C$ whose restriction to each fiber over $B$ has degree $d$. We identify two such sheaves if they differ by tensor product with an invertible sheaf pulled back from $B$.
\end{enumerate}

Given a moduli problem, our goal will be to describe a corresponding \emph{moduli space}. By this we mean a scheme $M$ whose points are in \emph{natural} one-to-one correspondence with the objects in our moduli problem. This will realize the objects of the moduli problem as the points of the underlying set of the scheme $M$.

If the moduli space in question and the base of the family are varieties, then the crucial condition that the correspondence be \emph{natural} is simple to express: that given a family of the objects in our moduli problem over a variety $B$, the map from underlying set of $B$ to the underlying set of $M$ taking each fiber
to the corresponding point of $M$ should be a morphism of varieties.  But in the world of schemes the set-theoretic mapping does not determine the morphism of schemes (think, for example, of the morphisms from $\Spec(\CC[x]/x^{2})$ into the plane with the closed point mapping to the origin. The situation is even worse when the moduli space itself is not a variety.)

To deal with the general case, we recast the naturality condition in functorial terms. We observe first that a moduli problem defines a functor $\cM$ from the category of schemes to the category of sets: the value of the functor at a scheme $B$ is the set of families of objects parametrized by $B$; a morphism $B' \to B$ of schemes gives rise, via pullback, to a map of sets $\cM(B) \to \cM(B')$. We define a \emph{fine moduli space} for the moduli problem to be a scheme $M$ that represents this functor, in the sense that there is an isomorphism of functors
$$
\cM \to {\rm Mor}(\bullet, M)
$$
In other words, for every scheme $B$ we have a bijection between families of our objects over $B$ and morphisms from $B$ to $M$. In particular, applying this to $B = \Spec \CC$, we have a bijection between the set of objects and the closed points of $M$; and for any family over an arbitrary scheme $B$, the map from $B(\CC)$ to $M(\CC)$ sending each closed point  $b \in B$ to the point in $M(\CC)$ corresponding to the fiber over $b$ is the underlying map of a morphism $B \to M$ of schemes.

If a fine moduli space for a given problem exists at all, then Yoneda's Lemma shows that it is unique up to a unique isomorphsm. This is a real problem: there is no fine moduli space for the first and most important of the examples above---the isomorphism classes of smooth curves---though there is for the others. We'll defer the discussion of why this is, and what we can do about it, until Chapter~\ref{Moduli chapter}.

Looking ahead, we'll discuss the third and fourth example in Chapter~\ref{}, where we'll describe the moduli spaces for effective divisors of given degree $d$ on a given curve $C$ (the symmetric powers of the curve) and for invertible sheaves of a given degree on $C$ (the \emph{Jacobian} and \emph{Picard variety} of $C$). These, as we'll see, are smooth, irreducible projective varieties of dimensions $d$ and $g$ respectively.

We'll take up the moduli space $M_g$ of smooth curves in Chapter~\ref{Moduli chapter}, where we'll see that this space (or rather the closest approximation to it we can cook up) is irreducible of dimension $3g-3$ for $g \geq 2$, though not smooth or projective.

Finally, the Hilbert scheme will be described (to the extent that we can!) in Chapter~\ref{HilbertSchemesChapter}; this will turn out to be much wilder and more varied in its behavior than any of the above.




%\section{Outline}
% \begin{enumerate}
%
%\item Background \begin{enumerate}
%
%\item Lasker's Theorem (complete intersections are unmixed) -- sometimes incorrectly called ``$AF+BG$''
%state ci implies unmixed; prove by $H^1$ of line bundle.
%
%\item B\'ezout and the  weak B\'ezout (ex 8.4.6 in Fulton).
%B\'ezout via Koszul complex, at least in codim 2.
%
%
%
%
%\end{enumerate}
%\item Basics of curves (corresponds to material in Hartshorne Ch 4, sects 1,2,4(?)---all except elliptic curves and ``classification")
%
%\begin{enumerate}
% \item Discussion of genus of smooth curves. Riemann-Roch. Hilbert coefficient.
%
%\item Divisors and maps; canonical divisor -- cotangent bundle; canonical map as example
%Exercise: Projections from on and off a curve
%
%\item canonical series is bpf, $2g+1$ is very ample. Canonical Curves and geometric RR
%
%\item Adjunction formula for curves in a surface (Quote from Hartshorne)
%
%\item Clifford, including strong form, canonical series is va except in hyperelliptic case.
%
%\item Riemann-Hurwitz (pull back a differential form)
%
%\end{enumerate}
%%\end{enumerate}
%
%\item Families of curves and Brill-Noether theory
%\begin{enumerate}
%\item Families. Define families. discuss "good and bad" families. example: symmetric product. example: plane curves of degree d. family of line bundles on a curve (or a family of curves.)
%
%\item How many curves of genus g are there? Informal discussion of moduli. 
%\item Cheerful fact about the existence of  Hilbert scheme, $M_g$, and $\overline M_g$, meaning of the phrase ``coarse moduli space''.
%
% \item Lowest degree of a covering of $\PP^1$?
% \item Lowest degree of a plane model?
% \item Lowest degree of an embedding?
% \item Number of forms of of degree d vanishing on the curve---ideals of the embeddings, Hilbert Functions.
%\end{enumerate}
%
%\item Personalities
%\begin{enumerate}
%\item $g=0$: rational curves as projections of rational normal curves. Rational quartic in $\PP^3$. Mention Set-Theoretic Complete Intersection problem.
%\item $g=1$. Wonderful subject; refer to somewhere else. Plane cubic, quartic in $\PP^3$, state that an elliptic quintic is Pfaffian (give proof??).
%\item $g=2$. Canonical map to $\PP^1$. Embedding in $\PP^3$ as $(2,3)$ on a quadric, ideal is 1 quadric, 2 cubics.
%\item $g=3$. Plane quartics. Sextic curves in $\PP^3$: general $g^3_6$ is an embedding, det variety.  $g\times g+1$ matrix of linear forms in $\PP^3$ gives curve of genus $g$, degree ${g\choose 2}$
%\item $g=4$. Canonical model is ci. Therefore $C$ is a 3-sheeted cover of $\PP^1$, in 1 or 2 ways, depending on the singularity of the unique quadric in the ideal.
%\item $g=5$. Canonical model lies on 3 quadrics. Either a CI, or trigonal. In latter case, Pfaffians. Introduce Green's conjecture?
%\item $g=6$. Canonical model lies on at least 6 quadrics. We will see exactly 6, and what they are...(projective normality of canonical embedding; and Brill-Noether for the plane model with 4 nodes.) Talk about the del Pezzo.
%\end{enumerate}
%
%\item Abel map and Jacobian
%\begin{enumerate}
%\item $J = H^0(\omega)^\vee/\hbox{lattice of periods}$. $\dim J = g$ general $g^3_{g+3}$ is very ample. Do this as an example of Hodge Theory.
%
%\item Symmetric powers, Abel-Jacobi map, Abel's Theorem (statement), differential of the Abel map.
%\item connection with addition formulas for integrals
%\end{enumerate}
%
%\item Scrolls, Hyperelliptic curves, trigonal curves 
%\begin{enumerate}
%\item general emb of degree $g+3$, in $\PP^3$ as divisor on a quadric of type $(2,g+1)$
%\item Other linear series on hyperell curve: 1) special linear series are mult $g^1_2$+basepoints. 2) Given an embedding, there's a union of lines. If the embedding is complete, we get a matrix...that defines the union of lines. Scrolls in all dimensions as unions of spans of divisors. 
%\item scrolls, starting with the matrix. Map to $\PP^1$ is the line bundle defined by the cokernel. Get the VB as the pushforward of $\cO(1)$.
%Mention higher-dim scrolls, but do surface scrolls in more detail: dim and genus of curves in each linear series. Condition for the existence of integral curves in each class iff the int number with the directrix is >0 and the curve is not a multiple of the fiber. 
%\item Minimal degree varieties; as the varieties of given degree lying on the maximal number of quadrics. 
%\item Type of scroll containing a hyperelliptic curve (Maroni invariant) --- invariant of the line bundle
%\item canonical image of a trigonal curve lies on a 2-dim scroll (non -subcanonical embedding only on 3-dim scrolls).  embedding of a trigonal curve lies on the same scroll.Stratification of trigonal curves by Maroni invariants. Dimensions via automorphism groups of scrolls.
%\item for a curve of genus 5, not cut out by quadrics:
% the intersection of the quadrics containing $C$ is two-dimensional: a rational normalscroll;  and  $C$ is trigonal, that is, a 3-sheeted cover of $\PP^1$. 
%\item Statement of Castelnuovo theory
%
%\item Old list: Scrolls and their divisors
%\begin{enumerate}
% \item Hirzebruch surfaces embedded by complete series
% \item lines joining 2 rational normal curves
% \item determinantal varieties
%\end{enumerate}
%
%\end{enumerate}
%
%\bigbreak

%\centerline {\bf Appendices?}
%\item Castelnuovo Theory
%\begin{enumerate}
%\item Ineq on Hilbert functions, leading to bound on $g,d$. 
%\item Give the scroll examples
%\item Castelnuovo's lemma (2n+3 points in $\PP^n$ by determinantal proof) 
%\item Characterization of Castelnuovo curves
%\item Projective normality of the canonical curve.
%\end{enumerate}

%\item Parameter space (open problems to be discussed in BN chapter or following.
%\begin{enumerate}
%\item Survey: Hilbert Scheme, Hurwitz Variety, Moduli
%\item Irreducibility (or not) and dimension. 
%\end{enumerate}
%
%\item Projective Normality
%\begin{enumerate}
%\item Castelnuovo's theorem: $2g+1$ is projectively normal.
%\item Liaison of curves in $\PP^3$.
% \end{enumerate}
%
%
%\item Inflectionary behavior and Brill-Noether
%\begin{enumerate}
% \item Pl\"ucker formulas
% \item Weierstrass Points
% \item finiteness of the automorphism group
%\item Proof of 1/2 BN by cuspidal curves
%\item Reference to the appendix of 3264 for the existence half
%\item generic $g^2_*$ is birational; $g^3_*$ is very ample (on a general curve)
%\end{enumerate}
%
%
%
%
%%\item
%%\begin{enumerate}
%%\item Broader View: 
%%\end{enumerate}
%
%
%
%\end{enumerate}
%


\begin{thebibliography}{ABC99}

%\bibitem[1965]{BR} D. Buchsbaum and D.S.Rim.
%A generalized Koszul complex. III. A remark on generic acyclicity.
%Proc. Amer. Math. Soc. 16 (1965) 555--558. 

\bibitem[Walker]{Walker} Walker.
\bibitem[Hartshorne]{Hartshorne} Hartshorne Ch 4
\bibitem[Fulton]{Fulton} Fulton Alg curves
\bibitem[Schemes]{Eisenbud-Harris} Schemes
\bibitem[3264]{Eisenbud-Harris} 3264
\bibitem[Griffiths-Harris]{Griffiths-Harris} (for the Abel-Jacobi stuff)
\bibitem[Griffiths]{Griffiths-Chinese}(for the Abel-Jacobi stuff)
\bibitem[Mumford]{Mumford}-Curves and their Jacobians
\bibitem[Voisin]{Voisin} Hodge Theory
%\bibitem[]{Smith} Smith, K.

\end{thebibliography}
\bigskip

\vbox{\noindent Author Addresses:\par
\smallskip
\noindent{David Eisenbud}\par
\noindent{Department of Mathematics, University of California, Berkeley,
Berkeley CA 94720}\par
\noindent{eisenbud@math.berkeley.edu}\par
}

\input footer.tex