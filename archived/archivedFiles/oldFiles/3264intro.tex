%header and footer for separate chapter files

\ifx\whole\undefined
\documentclass[12pt, leqno]{book}
\usepackage{graphicx}
\input style-for-curves.sty
\usepackage{hyperref}
\usepackage{showkeys} %This shows the labels.
%\usepackage{SLAG,msribib,local}
%\usepackage{amsmath,amscd,amsthm,amssymb,amsxtra,latexsym,epsfig,epic,graphics}
%\usepackage[matrix,arrow,curve]{xy}
%\usepackage{graphicx}
%\usepackage{diagrams}
%
%%\usepackage{amsrefs}
%%%%%%%%%%%%%%%%%%%%%%%%%%%%%%%%%%%%%%%%%%
%%\textwidth16cm
%%\textheight20cm
%%\topmargin-2cm
%\oddsidemargin.8cm
%\evensidemargin1cm
%
%%%%%%Definitions
%\input preamble.tex
%\input style-for-curves.sty
%\def\TU{{\bf U}}
%\def\AA{{\mathbb A}}
%\def\BB{{\mathbb B}}
%\def\CC{{\mathbb C}}
%\def\QQ{{\mathbb Q}}
%\def\RR{{\mathbb R}}
%\def\facet{{\bf facet}}
%\def\image{{\rm image}}
%\def\cE{{\cal E}}
%\def\cF{{\cal F}}
%\def\cG{{\cal G}}
%\def\cH{{\cal H}}
%\def\cHom{{{\cal H}om}}
%\def\h{{\rm h}}
% \def\bs{{Boij-S\"oderberg{} }}
%
%\makeatletter
%\def\Ddots{\mathinner{\mkern1mu\raise\p@
%\vbox{\kern7\p@\hbox{.}}\mkern2mu
%\raise4\p@\hbox{.}\mkern2mu\raise7\p@\hbox{.}\mkern1mu}}
%\makeatother

%%
%\pagestyle{myheadings}

%\input style-for-curves.tex
%\documentclass{cambridge7A}
%\usepackage{hatcher_revised} 
%\usepackage{3264}
   
\errorcontextlines=1000
%\usepackage{makeidx}
\let\see\relax
\usepackage{makeidx}
\makeindex
% \index{word} in the doc; \index{variety!algebraic} gives variety, algebraic
% PUT a % after each \index{***}

\overfullrule=5pt
\catcode`\@\active
\def@{\mskip1.5mu} %produce a small space in math with an @

\title{Personalities of Curves}
\author{\copyright David Eisenbud and Joe Harris}
%%\includeonly{%
%0-intro,01-ChowRingDogma,02-FirstExamples,03-Grassmannians,04-GeneralGrassmannians
%,05-VectorBundlesAndChernClasses,06-LinesOnHypersurfaces,07-SingularElementsOfLinearSeries,
%08-ParameterSpaces,
%bib
%}

\date{\today}
%%\date{}
%\title{Curves}
%%{\normalsize ***Preliminary Version***}} 
%\author{David Eisenbud and Joe Harris }
%
%\begin{document}

\begin{document}
\maketitle

\pagenumbering{roman}
\setcounter{page}{5}
%\begin{5}
%\end{5}
\pagenumbering{arabic}
\tableofcontents
\fi


\setlength{\parskip}{5pt}

\addtocounter{chapter}{-1}
\chapter{Introduction}
\label{IntroChapter}

\begin{quote}
\small\sf
``Es gibt nach des Verf. Erfarhrung kein besseres Mittel, Geometrie zu lernen, als
das Studium des Schubertschen `Kalk\"uls der abz\"ahlenden Geometrie'.''

(There is, in the author's experience, no better means of learning geometry than
the study of Schubert's ``Calculus of Enumerative Geometry.")

--B. L. van der Waerden (in a Zentralblatt review of an introduction to enumerative geometry
by Hendrik de Vries).
\bigskip

\end{quote}

%\noindent
%{\bf 1066 \& All That} (\cite{1066}) is ``A memorable history of England, comprising all the parts you can remember, including one hundred and three \emph{good} things, five \emph{bad} kings, and two \emph{genuine} dates\dots. History is not what you thought. \emph{It is what you can remember.} 

%\dots
%
%``In the year 1066 occurred the other memorable date in English History, viz. \emph{William the Conquereor, Ten Sixty-six.} 
%This is also called \emph{The Battle of Hastings,} and was when William I (1066) conquered England at the Battle of Senlac (\emph{Ten Sixty-six})\dots 
%The Norman Conquest was a Good Thing, as from this time onwards England stopped being conquered and thus was able to become top nation.''


\section{Why you want to read this book}

Algebraic geometry is one of the central subjects of mathematics. All but the most analytic of number theorists speak our language, as do mathematical physicists, homotopy theorists, complex analysts, symplectic geometers, representation theorists\dots. How else could you get between such apparently disparate fields as topology and number theory in one hop, except via algebraic geometry?

And intersection theory is at the heart of algebraic geometry. From the very beginnings of the subject, the fact that the number of solutions to a system of polynomial equations is, in many circumstances, constant as we vary the coefficients of those polynomials has fascinated algebraic geometers. The distant extensions of this idea still drive the field forward.

At the outset of the 19th century, it was to extend ``preservation of number" that algebraic geometers made two important choices: to work over the complex numbers rather than the real numbers, and to work in projective space rather than affine space. (With these choices the two points of intersection of a line and an ellipse have somewhere to go as the ellipse moves away from the real points of the line, and the same for 
the point of intersection of two lines as the lines become parallel.) Over the course of the century, geometers refined the art of counting solutions to geometric problems---introducing the central notion of a parameter space; proposing the notions of an equivalence relation on cycles and a product on the equivalence classes and using these in many subtle calculations. These constructions were fundamental to the developing study of algebraic curves and surfaces. 
%A landmark along this path was Chasle's computation of the number 3264 of plane conics tangent to five given conics---the computation that inspired the number in the title of this book.

In a different field, it was the search for a mathematically precise way of describing intersections that underlay Poincar\'e's study of what became algebraic topology. We owe Poincar\'e duality and a great deal more in algebraic topology directly to this search.
The difficulties Poincar\'e encountered in working with continuous spaces (now called manifolds) led him to develop the idea of a simplicial complex, too. 

Despite the lack of precise foundations, nineteenth century enumerative geometry rose to impressive heights: for example Schubert, 
whose \emph{Kalk\"ul der abz\"ahlenden Geometrie} (\cite{MR555576}) represents the summit of intersection theory at the time of its writing, calculated the number of twisted cubics tangent to 12 quadrics---and got the right answer (5,819,539,783,680). Imagine landing a jumbo jet blindfolded!

At the outset of the 20th century, Hilbert made finding rigorous foundations for Schubert calculus one of his celebrated Problems, and the quest to put intersection theory on a sound footing drove much of algebraic geometry for the following century;  the search for a definition of multiplicity fueled the subject of commutative algebra in work of van der Waerden, Zariski, Samuel, Weil and Serre. This progress culminated, towards the end of the century,  in the work of Fulton and MacPherson and then in Fulton's landmark book \emph{Intersection Theory} (\cite{Fulton1984}), which both greatly extended the range of intersection theory and for the first time put the subject on a precise and rigorous foundation.

The development of intersection theory is far from finished. Today the focus is on things like virtual fundamental cycles, quantum intersection rings, Gromov-Witten theory and the extension of intersection theory from schemes to stacks. 

A central part of a central subject of mathematics---of course you want to read this book! 

\section{Why we wrote this book}

Given the centrality of the subject, it is not surprising how much of algebraic geometry one encounters in learning enumerative geometry. And that's how this book came to be written, and why: like van der Waerden, we found that intersection theory makes for a great ``second course" in algebraic geometry, weaving together threads from all over the subject. Moreover, the new ideas one encounters in this setting have a context in which they're not merely more abstract definitions for the student to memorize, but tools that help answer concrete questions. 

\section{What's with the title?}


The number in the title of this book is a reference to the solution of a classic problem in enumerative geometry: the determination, by Chasles, of the number of smooth conic plane curves tangent to five given general conics. The problem is emblematic of the dual nature of the subject. On the one hand, the number itself is of little significance: life would not be materially different if there were more or fewer. But the fact that the problem is well-posed---that there is a Zariski open subset of the space of 5-tuples of conics $(C_1,\dots,C_5)$ for which the number of conics tangent to all five is constant, and that we can in fact determine that number---is at the heart of algebraic geometry. And the insights developed in the pursuit of a rigorous derivation of the number---the recognition of the need for, and introduction of, a new parameter space for plane conics, and the understanding of why intersection products are well defined for this space---are landmarks in the development of algebraic geometry.

The rest of the title is from ``1066 \& All That" by W. C. Sellar and R. J. Yeatman, a parody of English history textbooks; in many ways the number 3264 of conics tangent to five general conics is as emblematic of enumerative geometry as the date 1066 of the Battle of Hastings is of English history.


\section{What's in this book}


\begin{quote}
\small\sf
We are dealing here with a fundamental and almost paradoxical difficulty. Stated briefly, it is that learning is sequential but knowledge is not. A branch of mathematics... consists of an intricate network network of interrelated facts, each of which contributes to the understanding of those around it. When confronted with this network for the first time, we are forced to follow a particular path, which involves a somewhat arbitrary ordering of the facts.

--Robert Osserman.

\end{quote}



Where to begin? To start with the technical underpinnings of a subject risks losing the reader before the point of all that preliminary work is made clear; but to defer the logical foundations carries its own dangers---as the unproved assertions mount up, the reader may well feel adrift.

Intersection theory poses a particular challenge in this regard, since the development of its foundations is so demanding. It is possible, however, to state fairly simply and precisely the main foundational results of the subject, at least in the limited context of intersections on smooth projective varieties. The reader who is willing to take these results on faith for a little while, and accept this restriction, can then be shown ``what the subject is good for," in the form of examples and applications. This is the path we've chosen in this book, as we'll now describe.

\subsection{Overture}

The first two chapters may be thought of as an overture to the subject, introducing the central themes that will play out in the remainder of the book. 
In the first chapter, we introduce rational equivalence, the Chow ring, the pull-back and push-forward maps---the ``Dogma'' of the subject. We follow this in the second chapter with a range of simple examples to give the reader a sense of the themes to come: the computation of Chow rings of affine and projective spaces, their products and (some) blowups. To illustrate how intersection theory is used in algebraic geometry, we examine loci of various types of singular cubic plane curves, thought of as subvarieties of the projective space
$\P^9$ parametrizing plane cubics. Finally, we discuss briefly intersection products of curves on surfaces, an important early example of the subject.

%; and to illustrate how the language of intersection theory has become pervasive in the subject of algebraic geometry, we give a quick run-through of the theory of algebraic surfaces. 

%There are many other possible cycle theories that allow us to carry out the basic constructions of intersection theory. These, and the relations among them, are discussed in Appendix~\ref{topology appendix}.

%; these are optional at this point, but readers---especially those interested in the geometry and topology of complex varieties---may find this useful.

\subsection{Grassmannians}

The intersection rings of the Grassmannians are archetypal examples of intersection theory.  Chapters~\ref{GrassmannianChapter} and~\ref{GeneralGrassmannianChapter} are
devoted to them and the geometry that underlies them. Here we introduce the Schubert cycles, which form a basis for the Chow ring, and use them to solve a number of geometric problems, illustrating again how intersection theory is used to solve enumerative problems.



%{\bf --- Eppur, se muove!} (And yet, it moves!---Attributed to Galileo)
%\smallskip
%
%\subsection{What's under the hood: foundational results}
%
%The next two chapters,~\ref{ChowGroupsChapter} and~\ref{ChowRingsChapter}, are devoted to proofs and substantial discussion of the results stated in the Overture. Chapter~\ref{ChowGroupsChapter} introduces the foundational ideas of commutative algebra, centered around finite extensions of one-dimensional rings, that are necessary in establishing the properties of the pushforward map. 
%
%We have chosen to base our development of the theory on the \emph{moving lemma}, a central result that allows us to define intersection products in a relatively intuitive way, albeit in the limited context of intersections on smooth projective varieties. Chapter~\ref{ChowRingsChapter} introduces the beautiful geometry, first described by Severi and refined by Chow and others, of its proof.
%
%This is not the only way to develop intersection products. The ``deformation to the normal cone" (explained in Chapter \ref{ExcessChapter}), the foundational idea used in \cite{Fulton1984}, has many advantages, and leads ultimately to a sharper and more general theory. As the student becomes an expert the transition to the new point of view will be natural. But we felt that, in its more limited sphere, the approach via the moving lemma gives a direct intuition for why the basic results are true that is difficult to gain from the  newer approach. One might compare the situation to that encountered in learning algebraic geometry. No one would doubt that the theory of schemes is a more natural and supple language for the subject; but a beginner may be better served by learning the more concrete case of  varieties first.
%


%To explain why we have nevertheless based our development on
%the old-fashioned moving lemma, we propose an analogy: The theory of schemes is a natural and flexible language for algebraic geometry. But usually a student encountering the subject for the first time starts with varieties, and only goes on to learn about schemes after gaining some familiarity with the subject. 
%
%In much the same way, there is no question in our minds that the classical, 19th century-and-first-three-quarters-of-the-20th approach to intersection theory is inferior to the theory developed and laid out in Fulton. The latter is much more generally applicable and yields deeper insights into what's going on, much as the theory of schemes is and does relative to varieties. 

%But the classical theory is still, I think, a useful pedagogical tool for introducing students to the subject: as we said, it allows the reader to understand the basic constructions of intersection theory, and many of their applications, right away.

\subsection{Chern classes}

We then come to a watershed in the subject.  Chapter~\ref{ChernClassChapter} takes up in earnest a notion that is at the center of modern intersection theory, and indeed of modern algebraic geometry: Chern classes. As with the development of intersection theory we focus on the classical characterization of Chern classes, as degeneracy loci of collections of sections. This interpretation provides useful intuition and is basic to many applications of the theory.

%In a brief interlude (Chapter~\ref{DirectImageChapter}) we review the basics of the theory of vector bundles (also known as locally free sheaves) and their direct images, leading up to the theorem on cohomology and base change. With these ideas in hand,

\subsection{Applications I: using the tools} 

We illustrate the use of Chern classes by taking up two classical problems: Chapter~\ref{LinesOnHypersurfacesChapter} deals with the question of how many lines lie on a hypersurface (for example there are exactly 27 lines on each smooth cubic surface), and Chapter~\ref{SingularElementsChapter} looks at the singular hypersurfaces in a one-dimensional family  (for example, what is the degree of the discriminant of a polynomial in several variables). Using the basic technique of \emph{linearization}, these problems can be translated into problems of computing Chern classes. These and the next few chapters are organized around geometric problems involving  constructions of useful vector bundles and the calculation of their Chern classes. 

\subsection{Parameter spaces}

Chapter~\ref{CompactifyingChapter}
deals with an area in which intersection theory has had a profound influence on modern algebraic geometry: \emph{parameter spaces} and their compactifications. This is illustrated with the five-conic problem; there is also a discussion of the modern example of Kontsevich spaces, and an application of those. 

\subsection{Applications II: further developments}

The remainder of the book introduces a series of increasingly advanced topics. Chapters~\ref{ChowRingsProjectiveBundlesChapter}, \ref{SegreChapter}
 and \ref{TangentBundleChapter}
deal with a situation ubiquitous in the subject, the intersection theory on projective bundles, and its applications to subjects such as projective duality and the enumerative geometry of contact conditions. 

Chern classes are defined in terms of the loci where  collections of sections of a vector bundle become dependent. These can be interpreted as loci where maps from  trivial vector bundles drop rank. The Porteous formula, proved and applied in Chapter~\ref{PorteousChapter}, generalizes this, 
expressing the classes of the loci where a map between two general vector bundles has a given rank or less in terms of the Chern classes of the two bundles involved.

\subsection{Advanced topics}

Next, we come to some of the developments of the modern theory of intersections. In Chapter~\ref{ExcessChapter}, we introduce the notion of ``excess'' intersections and the \emph{excess intersection formula}, one of the subjects that was particularly mysterious in the nineteenth century but that was elucidated by Fulton and MacPherson. This theory makes it possible to describe the intersection class of two cycles even if their intersection has ``too large'' dimension.  Central to this development is the idea of \emph{deformation to the normal cone}, a construction fundamental to the work of Fulton and Macpherson; we use this to prove the famous ``key formula" comparing intersections of cycles in a subvariety $Z \subset X$ to the intersections of those cycles in $X$, and use this in turn to give a description of the Chow ring of a blow-up.

 
Chapter \ref{GRRChapter}
contains an account of Riemann-Roch formulas, leading up to a description of Grothendieck's version.
The chapter concludes with a number of examples and applications showing how Grothendieck's formula can be used.

\subsection{Applications III: the Brill-Noether theorem}

The last chapter of the book, Chapter~\ref{Brill-NoetherChapter}, explains an application of enumerative geometry to a  problem  that is central in the study of algebraic curves and their moduli spaces: the existence of special linear series on curves. We give the proof of this theorem by Kempf and Kleiman-Laksov, which draws upon many of the ideas and techniques of the book, plus one new one: the use of topological cohomology in the context of intersection theory. This is also a wonderful illustration of the way in which enumerative geometry can be the essential ingredient in the proof of a purely qualitative result.

\subsection{Relation of this book to ``Intersection Theory''} 

Fulton's book on intersection theory  (\cite{Fulton1984}) is a great work. It sets up a rigorous framework for intersection theory in a generality significantly extending and refining what was known before and laying out an enormous number of applications. It is a work that can serve as an encyclopedic reference to the subject.

By contrast, the present volume is intended as a textbook in algebraic geometry, a second course in which the classical side of intersection theory is a starting point for exploring many topics in geometry. We introduce the intersection product at the outset,  and focus on basic examples. We use concrete problems to motivate the introduction of new tools from all over algebraic geometry. Our book is not a substitute for Fulton's: it has a different aim. We hope that it will provide the reader with intuition and motivation that will make reading Fulton's book easier.

This trade-off---making the restrictive hypothesis of smoothness, in exchange for relative ease of intuition and immediate application---does obscure some important aspects of cycle theory. By way of analogy, imagine that, as a topologist, you could define and work with homology---but only for compact, oriented manifolds. Even with this restriction, you could develop a pretty good sense of what homology groups represent, and be able to carry out most of the standard applications of homology; if you were primarily interested in applications to manifolds, this would be enough.
You might even think you understood cohomology and cup products as well: after all, in this limited context (rational) cohomology is isomorphic to homology, and via this isomorphism the cup product is simply the intersection of cycles. But this would be a mistake: to gain a real understanding of homology and cohomology, you have to develop both in a broader context. In the same way, to really understand the cycle theory of algebraic varieties, you have to read Fulton. 

\subsection{Keynote problems} To highlight the sort of problems we'll  learn to solve, and to motivate the material we present, we'll begin each chapter with some {\it keynote questions}. 
%We urge the reader to pause for a moment and think about each one before diving into the body of the chapter.


\section{Prerequisites, notation and conventions}

\subsection{What you need to know before starting}
When it comes to prerequisites, there are two distinct questions: what you should know to start reading this book; and what you should be prepared to learn along the way. 

Of these, the second is by far the more important.
In the course of developing and applying intersection theory we introduce many key techniques of algebraic geometry, such as deformation theory, specialization methods, characteristic classes, Hilbert schemes, commutative and homological algebra and topological methods. That's not to say that you need to know these things going in. Just the opposite, in fact: reading this book is an occasion to learn them.

%To aid you we try to provide a guide to the relevant literature.
%%hmmm, we better add some!
% In each chapter, we indicate what new ideas or techniques from algebraic geometry are being used, and give suggestions of where the reader might go for background or a further development of the subject. For example, you certainly need not have read the first author's \emph{Commutative Algebra With a View Toward Algebraic Geometry} (\cite{Eisenbud1995}), but there are many specific references to it. 

So what do you need before starting? 

\begin{enumerate}

\item An undergraduate course in classical algebraic geometry or its equivalent, comprising the elementary theory of affine and projective varieties.  \emph{An Invitation to Algebraic Geometry} (\cite{MR1788561}) %Smith et al
 contains almost everything required. Other books that cover this material include 
\emph{Undergraduate Algebraic Geometry} (\cite{MR982494}), %Reid
\emph{Introduction to Algebraic Geometry} (\cite{MR2324354}) %Hassett
and, at a somewhat more advanced level 
 \emph{Algebraic Geometry I: Complex Projective Varieties}  (\cite{Mumford1976}), %Mumford
\emph{Basic Algebraic Geometry, Volume I} (\cite{Shafarevich1974}) and \emph{Algebraic Geometry: A First Course} (\cite{Harris1992}). The last three include much more than we'll use here.

\item An acquaintance with the language of schemes. This would be amply covered by the first three chapters of  \emph{The Geometry of Schemes}  (\cite{MR1730819}). %Eisenbud-Harris. 

\item An acquaintance with coherent sheaves and their cohomology. For this, \emph{Faisceax Alg\'ebriques Coh\'erents}  (\cite{Serre1955}) remains an excellent source (it's written in the language of varieties, but applies nearly word for word to projective schemes over a field, the context in which this book is written). 

\end{enumerate}

In particular, \emph{Algebraic Geometry} (\cite{Hartshorne1977}) contains much more than you need to know to get started.

\subsection{Language}
 

Throughout this book, a \emph{scheme} $X$ will be a scheme of finite type over an algebraically closed field
$K$. We use the term \emph{integral} to mean reduced and irreducible; by a \emph{variety} we will mean an integral scheme. (The terms ``curve" and ``surface," however, refer to one-dimensional and two-dimensional schemes; in particular, they are not presumed to be integral.) A subvariety $Y \subset X$ will be presumed closed unless otherwise specified.
 If $X$ is a variety we write $K(X)$ for the field of rational functions on $X$. A \emph{sheaf} on $X$ will be a coherent sheaf unless otherwise noted.
 
By a \emph{point}
we mean a closed point. 
Recall that a \emph{locally closed} subscheme $U$ of a scheme $X$ is 
a scheme that is an open subset of a closed subscheme of $X$. We generally use the term
``subscheme'' (without any modifier) to mean a closed subscheme, and similarly for ``subvariety."

A consequence of the finite type hypothesis
is that to any subscheme $Y$ of 
$X$ has a \emph{primary decomposition}: locally, we can write the ideal of $Y$ as an irredundant intersection of primary ideals with distinct associated primes. We can correspondingly write $Y$ globally as an irredundant union of closed subschemes $Y_i$ whose supports are distinct subvarieties of $X$. In this expression, the subschemes $Y_i$ whose supports are maximal---corresponding to the minimal primes in the primary decomposition---are uniquely determined by $Y$; they are called the \emph{irreducible components} of $Y$. The remaining subschemes are called \emph{embedded components}; they are not determined by $Y$, though their supports are.

If a family of objects is parametrized by a scheme $B$, we will say that a ``general" member of the family has a given property $P$ if the set $U(P) \subset B$ of members of the family with that property contains an open dense subset of $B$. When we say that a ``very general" member has this property  we will mean that $U(P)$ contains the complement of a countable union of proper subvarieties of $B$.


By the \emph{projectivization} $\P V$ of a vector space $V$ we'll mean the scheme $\proj(\Sym^* V^*)$; this is the space whose closed points correspond to one-dimensional subspaces of $V$.

If $X$ and $Y \subset \P^n$ are subvarieties of projective space, we define the \emph{join} of $X$ and $Y$, denoted $\overline{X\,Y}$, to be the closure of the union of lines meeting $X$ and $Y$ at distinct points. If $X = \Gamma \subset \P^n$ is a linear space, this is just the cone over $Y$ with vertex $\Gamma$; if $X$ and $Y$ are both linear subspaces, this is simply their span.

There is a one-to-one correspondence between vector bundles on a scheme $X$ and locally free sheaves on $X$. We will use the terms interchangeably, generally preferring ``line bundle'' and ``vector bundle'' to ``invertible sheaf'' and ``locally free sheaf''.

By a \emph{linear system}, or \emph{linear series}, on a scheme $X$ we will mean a pair $(\CL, V)$ where $\CL$ is a line bundle on $X$ and $V \subset H^0(\CL)$ a vector space of sections. Associating to a section $\s \in H^0(\CL)$ its zero locus $V(\s)$, we can also think of a linear system as a family $\CD = \{ V(\s) \mid \s \in V\}$ of subschemes parametrized by the projective space $\P V$; in this setting, we will sometimes refer to the linear system $\CD$. By the \emph{dimension} of the linear series we mean the dimension of the projective space $\P V$ parametrizing it; that is, $\dim V - 1$. Specifically, a one-dimensional linear system is called a \emph{pencil}; a two-dimensional system is called a \emph{net} and a three-dimensional linear system is called a \emph{web}.

We write $\CO_{X,Y}$
for the local ring of $X$ along $Y$, and more generally, if $\CF$ is a sheaf of
$\CO_{X}$-modules then we write $\CF_Y$ for the 
corresponding $\CO_{X,Y}$-module.

We can identify the Zariski tangent space to affine space $\A^n$ with $\A^n$ itself. If $X \subset \A^n$ is a subscheme, by the \emph{affine tangent space} to $X$ at a point $p$ we will mean the affine linear subspace $p + T_pX \subset \A^n$. If $X \subset \P^n$ is a subscheme, by the \emph{projective tangent space} to $X$ at $p \in X$, denoted $\T_pX \subset \P^n$, we will mean the closure in $\P^n$ of the affine tangent space to $X \cap \A^n$ for any open subset $\A^n \subset \P^n$ containing $p$. Concretely, if $X$ is the zero locus of polynomials $F_\a$ (that is, $X = V(I) \subset \P^n$ is the subscheme defined by the ideal $I = (\{F_\a\}) \subset K[Z_0,\dots,Z_n]$), the projective tangent space is the common zero locus of the linear forms
$$
L_\a(Z) = \frac{\partial F_\a}{\partial Z_0}(p)Z_0 + \dots + \frac{\partial F_\a}{\partial Z_n}(p)Z_n.
$$

By a ``one-parameter family" we will always mean a family $X \to B$ with $B$ smooth and one-dimensional (an open subset of a smooth curve, or spec of a DVR or power series ring in one variable), with marked point $0 \in B$. In this context, ``with parameter $t$" means $t$ is a local coordinate on the curve, or a generator of the maximal ideal of the DVR or power series ring.


\subsection{Basic results on dimension and smoothness}

There are a number of theorems in algebraic geometry that we'll use repeatedly; we give the statements and references here.

To start with, we will often use  the following basic results of commutative algebra:

\begin{thm}[Krull's Principal Ideal Theorem] \label{Krull}
An ideal generated by $n$ elements in a Noetherian ring has codimension
$\leq n$. 
\end{thm}


See \cite{Eisenbud1995} Theorem 10.2 for a discussion and proof.
We will also use the following important extension of the Principal Ideal Theorem:

\begin{thm}[Generalized Principal Ideal Theorem]\label{Serre}
If  $f:Y\to X$ is a morphism of varieties, and $X$ is smooth, then for any subvariety $A\subset X$, 
$$
\codim f^{-1}A \leq \codim A.
$$
In particular, if $A,B$ are subvarieties of $X$, and $C$ is an irreducible component of $A\cap B$,
 then $\codim C \leq \codim A + \codim B$.
\end{thm}
The proof of this result can be reduced to the case of an intersection of two subvarieties, one of which is locally a complete intersection, by expressing the inverse image $f^{-1}A$ as an intersection with the graph $\Gamma_f \subset X \times Y$ of $f$. In this form it follows from Krull's 
Theorem. The result holds in greater generality; see %Serre Alg Loc
\cite{MR1771925} Theorem V.3. Smoothness is necessary for this (Exercise \ref{sharpness of Serre PIT}).

\begin{thm}[Jordan-H\"older Theorem]\label{J-H}
A module $M$ of finite length over a commutative local ring $R$ with maximal ideal $\gm$ 
has a maximal sequence of submodules (called its composition series)
$M\supsetneq \gm M\cdots\supsetneq \gm^{k}M = 0$
whose length $k$ is called the length of $M$.
\end{thm}

\begin{thm}[Chinese Remainder Theorem]\label{chinese}
A module of finite length over a commutative ring is the direct sum of its localizations at 
finitely many maximal
ideals.
\end{thm}


For a discussion and proof see \cite{Eisenbud1995} Chapter 2, and especially Theorem 2.13. 


\begin{thm}[Bertini]\label{bertini}
If $\CD$ is a linear system on a variety $X$ in characteristic 0, the general member of $\CD$ is smooth outside the base locus of $\CD$ and the singular locus of $X$
\end{thm}

This is the form in which we'll usually apply Bertini. But there is another version that is equivalent in characteristic 0 but allows for an extension to positive characteristic:


\begin{thm}[Bertini]\label{bertini2}
If $f : X \to \P^n$ is any generically separated morphism from a smooth, quasiprojective variety $X$ to projective space, then the preimage $f^{-1}(H)$ of a general hyperplane $H \subset \P^n$ is smooth.
\end{thm}


\newpage

\

\begin{quote}
\small\sf
We are all familiar with the after-the-fact tone---weary, self-justificatory, aggrieved, apologetic---shared by ship captains appearing before boards of inquiry to explain how they came to run their vessels aground, and by authors composing forewords.

--John Lanchester 
\bigskip

\end{quote}



\

%footer for separate chapter files

\ifx\whole\undefined
%\makeatletter\def\@biblabel#1{#1]}\makeatother
\makeatletter \def\@biblabel#1{\ignorespaces} \makeatother
\bibliographystyle{msribib}
\bibliography{slag}

%%%% EXPLANATIONS:

% f and n
% some authors have all works collected at the end

\begingroup
%\catcode`\^\active
%if ^ is followed by 
% 1:  print f, gobble the following ^ and the next character
% 0:  print n, gobble the following ^
% any other letter: normal subscript
%\makeatletter
%\def^#1{\ifx1#1f\expandafter\@gobbletwo\else
%        \ifx0#1n\expandafter\expandafter\expandafter\@gobble
%        \else\sp{#1}\fi\fi}
%\makeatother
\let\moreadhoc\relax
\def\indexintro{%An author's cited works appear at the end of the
%author's entry; for conventions
%see the List of Citations on page~\pageref{loc}.  
%\smallbreak\noindent
%The letter `f' after a page number indicates a figure, `n' a footnote.
}
\printindex[gen]
\endgroup % end of \catcode
%requires makeindex
\end{document}
\else
\fi


