%header and footer for separate chapter files

\ifx\whole\undefined
\documentclass[12pt, leqno]{book}
\usepackage{graphicx}
\input style-for-curves.sty
\usepackage{hyperref}
\usepackage{showkeys} %This shows the labels.
%\usepackage{SLAG,msribib,local}
%\usepackage{amsmath,amscd,amsthm,amssymb,amsxtra,latexsym,epsfig,epic,graphics}
%\usepackage[matrix,arrow,curve]{xy}
%\usepackage{graphicx}
%\usepackage{diagrams}
%
%%\usepackage{amsrefs}
%%%%%%%%%%%%%%%%%%%%%%%%%%%%%%%%%%%%%%%%%%
%%\textwidth16cm
%%\textheight20cm
%%\topmargin-2cm
%\oddsidemargin.8cm
%\evensidemargin1cm
%
%%%%%%Definitions
%\input preamble.tex
%\input style-for-curves.sty
%\def\TU{{\bf U}}
%\def\AA{{\mathbb A}}
%\def\BB{{\mathbb B}}
%\def\CC{{\mathbb C}}
%\def\QQ{{\mathbb Q}}
%\def\RR{{\mathbb R}}
%\def\facet{{\bf facet}}
%\def\image{{\rm image}}
%\def\cE{{\cal E}}
%\def\cF{{\cal F}}
%\def\cG{{\cal G}}
%\def\cH{{\cal H}}
%\def\cHom{{{\cal H}om}}
%\def\h{{\rm h}}
% \def\bs{{Boij-S\"oderberg{} }}
%
%\makeatletter
%\def\Ddots{\mathinner{\mkern1mu\raise\p@
%\vbox{\kern7\p@\hbox{.}}\mkern2mu
%\raise4\p@\hbox{.}\mkern2mu\raise7\p@\hbox{.}\mkern1mu}}
%\makeatother

%%
%\pagestyle{myheadings}

%\input style-for-curves.tex
%\documentclass{cambridge7A}
%\usepackage{hatcher_revised} 
%\usepackage{3264}
   
\errorcontextlines=1000
%\usepackage{makeidx}
\let\see\relax
\usepackage{makeidx}
\makeindex
% \index{word} in the doc; \index{variety!algebraic} gives variety, algebraic
% PUT a % after each \index{***}

\overfullrule=5pt
\catcode`\@\active
\def@{\mskip1.5mu} %produce a small space in math with an @

\title{Personalities of Curves}
\author{\copyright David Eisenbud and Joe Harris}
%%\includeonly{%
%0-intro,01-ChowRingDogma,02-FirstExamples,03-Grassmannians,04-GeneralGrassmannians
%,05-VectorBundlesAndChernClasses,06-LinesOnHypersurfaces,07-SingularElementsOfLinearSeries,
%08-ParameterSpaces,
%bib
%}

\date{\today}
%%\date{}
%\title{Curves}
%%{\normalsize ***Preliminary Version***}} 
%\author{David Eisenbud and Joe Harris }
%
%\begin{document}

\begin{document}
\maketitle

\pagenumbering{roman}
\setcounter{page}{5}
%\begin{5}
%\end{5}
\pagenumbering{arabic}
\tableofcontents
\fi


\chapter{Curves on Scrolls}
\label{ScrollsChapter}
\section{Hyperelliptic curves}
\fix{maybe start a new Chapter?}

In our early encounters with curves, we frequently assumed that the curve we were considering was non-hyperelliptic, since the behavior of hyperelliptic curves is so atypical. In this section, we'll describe the geometry of hyperelliptic curves.

\subsection{Basic models of hyperelliptic curves}\fix{move this section to ch 2; add discussion of adjoints--- perhaps as exercises?}

We start by establishing some basic facts about hyperelliptic curves. Many of these follow from general theorems like Riemann-Roch; but since they can be established by direct examination we will carry that out here.

Suppose $C$ is a smooth, projective hyperelliptic curve of genus $g \geq 2$. By definition, $C$ admits a degree 2 map $\pi : C \to \PP^1$; and as we've observed (\ref{**}) this map is unique.

By Riemann-Hurwitz, \fix{attibution?} the map $\pi : C \to \PP^1$ will have $2g+2$ distinct simple branch points, say $\lambda_1,\dots,\lambda_{2g-2} \in \PP^1$. An open subsect $C^\circ$ of $C$ can then be realized as the smooth projective completion of the affine curve given as
$$
C^\circ = \big\{ (x,y) \in \AA^2 \; \mid \; y^2 = \prod_{i=1}^{2g+2} (x - \lambda_i) \big\}.
$$ 
\fix{if two of the $\lambda_i$ coinncide, then the curve develops a singular point. Much of what we will do carries over to the singular case.} \fix{say the smooth model has 2 points at $\infty$.} Note that if we simply take the closure of this locus in $\PP^2$, the resulting curve will be highly singular at the point $[1,0,0]$, as can be seen either  directly by making an appropriate change of variables, or by invoking the genus formula for plane curves: if the closure were smooth, it would have genus $\binom{2g+1}{2}$. We can, however, complete the curve simply in $\PP^1 \times \PP^1$, for example by setting \fix{this is a rabbit from a hat. Consider either saying that by the previous section, if there's an emb in P3 then its on P1 x P1 as a divisor of type
2,g+1; and then "finding" this embedding as below; or moving this page to the early place where hyperelliptic curves are first mentioned.}
$$
y' = \frac{y}{\prod_{i=1}^{g+1} (x - \lambda_i)};
$$
we can then write the equation of a still smaller open subset of $C$ as
$$
{y'}^2 \cdot \prod_{i=1}^{g+1} (x - \lambda_i) \; = \; \prod_{i=g+2}^{2g+2} (x - \lambda_i).
$$
If we now take the closure of this locus in $\PP^1 \times \PP^1$, we get a curve of type $(2,g+1)$ on $\PP^1 \times \PP^1$; this curve is smooth, as can be seen again either directly in coordinates or by invoking the genus formula for curves on $\PP^1 \times \PP^1$. In other words,
$$
C \; = \; V\Big(Y_0^2\cdot \prod_{i=1}^{g+1} (X_1 - \lambda_iX_0) - Y_1^2 \cdot \prod_{i=g+2}^{2g+2} (X_1 - \lambda_iX_0) \Big)
$$

Next, let's describe the space of regular differentials on $C$. For this, it's convenient to work with the affine model $C^\circ = V(f) \subset \AA^2$, where
$$
f(x,y) = y^2 - \prod_{i=1}^{2g-2} (x - \lambda_i).
$$

We'll denote the two points at infinity---that is, the two points of $C \setminus C^\circ$---as $p$ and $q$.

To start, consider the simple differential $dx\in \Omega_{C^\circ/k}$. This is clearly regular on $C^\circ$, with zeros at the ramification points $r_i = (\lambda_i, 0)$. But it does not extend to a regular differential on all of $C$: it will have double poles at $p$ and $q$, as can be seen either directly or by degree considerations: as we said, $dx$ has $2g+2$ zeros, while the degree of $K_C$ is $2g-2$, meaning that there must be poles at the points $p$ and $q$.

To kill these poles, we can of course divide by $x^2$ (or any quadratic polynomial in $x$). But that just introduces new poles in the finite part $C^\circ$ of $C$. Instead, we want to multiply $dx$ by a rational function with zeros at $p$ and $q$, but \emph{whose poles occur only at the points where $dx$ has zeroes}---that is, the points $r_i$.  A natural choice is simply the reciprocal of the partial derivative $f_y = \partial f/ \partial y = 2y$, which vanishes exactly at the points $r_i$, and has correspondingly a pole of order $g+1$ at each of the points $p$ and $q$ (reason: the involution $y\to -y$ fixes $C^\circ$ and $x$), and exchanges the points $p,q$. In other words, the differential
$$
\omega = \frac{dx}{f_y}
$$
is regular, with divisor
$$
(\omega) = (g-1)p + (g-1)q.
$$
The remaining regular differentials on $C$ are now easy to find: Since $x$ has only a simple pole
at the two points at infinity \fix{say why}. we can  multiply $\omega$ by any $x^k$ with $k = 0, 1, \dots, g-1$. Since this gives us $g$ independent differentials, these  form a basis for $H^0(K_C)$.


 1) special linear series are mult $g^1_2$+basepoints. 2) Given an embedding, there's a union of lines. If the embedding is complete, we get a matrix...that defines the union of lines. Scrolls in all dimensions as unions of spans of divisors.
 
 
\subsection{General embeddings of degree
genus$+3$} 

It's a divisor on a quadric in $\PP^{3}$ of type $(2,g+1)$


\section{Trigonal curves}

\subsection{Trigonal curves lie on scrolls}

The key to our analysis of linear systems on trigonal curves will be the fact that \emph{trigonal curves lie on scrolls}; we'll start by establishing that fact.

There are two ways we might do this: concretely and abstractly. To start with the former, let $C$ be a smooth, projective trigonal curve of genus $g > 2$. By the exercise below, $C$ cannot be hyperelliptic, and so its canonical map embeds $C$ as a canonical curve in $\PP^{g-1}$.

\begin{exercise}
\begin{enumerate}
\item Show that a curve of genus $g > 2$ cannot be both hyperelliptic and trigonal.
\item Show that a trigonal curve of genus $g > 4$ has a unique $g^1_3$.
\end{enumerate}
\end{exercise}

Consider the divisors $\{D_\lambda\}_{\lambda \in \PP^1}$ of the $g^1_3$ on $C$. By the geometric Riemann-Roch theorem, each consists of three colinear points; and hence any quadric hypersurface $Q$ containing $C$ will contain each of the lines $L_\lambda$ spanned by these divisors. 

Now, let $S \subset \PP^{g-1}$ be the surface swept out by these lines. $S$ is clearly an irreducible, nondegenerate surface in $\PP^{g-1}$, lying on each of the $\binom{g-2}{2}$ quadrics containing the curve $C$. But we've seen that the maximum possible number of quadrics containing an irreducible, nondegenerate surface $T \subset \PP^n$ is $\binom{n-1}{2}$, and any such surface lying on that many quadrics must be either a rational normal scroll or a Veronese surface. Since $S$ is swept out by lines, and the Veronese surface contains no lines, we conclude that \emph{the canonical model of a trigonal curve lies on a rational normal scroll}.

In fact, we can describe the scroll $S$ directly: if we consider the product map
$$
H^0(\cO_C(D)) \otimes H^0(K_C(-D)) \to H^0(K),
$$
the surface $S$ is simply the rank 1 locus of the transpose map
$$
H^0(K)^* \to \Hom\left((H^0(\cO_C(D)), H^0(K_C(-D))^*\right).
$$

There is another, more abstract way of describing the scroll $S$. Let $\pi : C \to \PP^1$ be the map associated to the $g^1_3$ on $C$. We take the direct image $E = \pi_*\cO_C$ of the structure sheaf of $C$; this is a vector bundle of rank 3 on $\PP^1$. There is a natural inclusion of the structure sheaf $\cO_{\PP^1}$ in $E$, and if we let $F = E/\cO_{\PP^1}$ be the quotient, we have a natural embedding 
$$
C \hookrightarrow \PP F;
$$
the projectivization of $\PP F$ is the scroll $S$.

 \subsection{Which scrolls?}

There is a natural follow-up question: given that a trigonal curve $C$ has a natural inclusion in a rational normal scroll, we can, ``which one?" The following lemma gives the answer:

\begin{lemma}
Let $C$ be a trigonal curve of genus $g$, and let $S \cong \FF_n$ be the scroll associated to $C$ as above. 
\begin{enumerate}
\item $n \equiv g \; mod (2)$; and
\item If $C$ is general, then  either  $n=0$ (if $g$ is even) or $n=1$ (if $g$ is odd).
\end{enumerate}\end{lemma}

\subsection{The Maroni inviariant}

\begin{theorem}
 There is a smooth canonical curve on a rational normal scroll $S$ in $\PP^{g-1}$ iff the self-intersection of the directrix on $S$ is equivalent to 
$H-mR$ where $m\leq (2g-2)/3$.
\end{theorem}

\subsection{Special linear series on trigonal curves}

In analyzing special linear series on a hyperelliptic curve, we made crucial use of the facts that the canonical image of a hyperelliptic curve is a rational normal curve, and that any collection of points on a rational normal curve $C \subset \PP^n$ either are linearly independent or span $\PP^n$. In a similar (though necessarily less complete) way, we can use the fact that the canonical image of a trigonal curve lies on a rational normal surface scroll to describe special linear series on it.

\begin{theorem}
Let $L$ be a special line bundle on a trigonal, non-hyperelliptic curve, and suppose that $h^0(L) \geq 2$ and $h^1(L)\geq 2$.  Then either contains a $g^1_3$ or is contained in $K$ minus a $g^1_3$. If $g\geq 5$ then both conditions must hold.
\end{theorem}

\fix{ can we give a necessary and sufficient condition?}


\begin{proof}
Consider the canonical embedding of a trigonal curve $C$, and a general divisor $D$ of a base-point free special linear series. Since $D$ is special, it lies in. a hyperplane, and since $D$ moves, the Geometric Riemann-Roch Theorem~\ref{} shows that  $D$ spans a space of dimension $<\deg D-1$.  

First suppose that $g\geq 4$. Since $C$ lies on a rational normal scroll, $D$ lies on a hyperplane section $C'$ of the scroll. If $C;$  is irreducible, then it is a rational normal curve; but divisors on rational normal curves are always linearly independent. Thus $C'$ must be reducible. By Lemma~\ref{reducible sections}, $C'$ consists of lines of the ruling together with a rational normal curve $C''$ of lower degree, and thus embedded in a lower-dimensional plane. Since the points of $D$ are dependent, $D$ must have at least 3 points on one of the rulings (and thus contains a $g^1_3$) or at least 3 points on $C''$, in which case it is contained in $K$ minus a $g^1_3.$

If $g\geq 5$ then 

In the case $g =3$, $C$ is a smooth plane curve, and the points of $D$ must lie on a line. Since the lines through any point of $C$ cut out a $g^1_3$, we see that $D$ is contained in a $g^1_3$. 
\end{proof}

\begin{lemma}\label{reducible sections}
Let $S = S_{a,b} \subset \PP^n$ be a rational normal surface scroll. Any hyperplane section $H \cap S$ consists of the union of a rational normal curve $E$, which is a section of the scroll, and a union of lines of the ruling of the scroll.
\end{lemma}

Note that the curve $E$ must be a reduced component of $S \cap H$, but the lines $L_i$ may coincide, i.e., may be non-reduced components of the intersection. In the following proof, we'll assume for clarity that the lines $L_i$ are distinct (that is, $S \cap H$ is reduced); we leave it as an exercise to rewrite the proof to accommodate the remaining cases.

\begin{proof}
Let $F \in \Pic(S)$ be the class of a line of the ruling. Since $F^2 = 0$ and $H\cdot F = 1$, exactly one of the components of $S \cap H$ must have intersection number 1 with $F$; all other components must have intersection number 0 with $F$ and so must be lines of the ruling.

It remains to show that the unique component $E$ of $H \cap S$ having intersection number 1 with $F$ is a rational normal curve. This can be seen directly, but there's a shortcut. Suppose that we have
$$
S \cap H = E \cup L_1 + \dots + L_k,
$$
so that in particular $\deg(E) = n-1-k$. Since each of the lines $L_i$ of the ruling must meet $C$, we have that
\begin{align*}
n-1 &= \dim(\overline{S \cap H}) \\
&\leq \dim(\overline {E}) + k\\
&\leq (n-1-k) + k \\
&= n-1.
\end{align*}
We conclude that $\dim(\overline E) = n-k-1$, and hence that $E$ is a rational normal curve.
\end{proof}

Note that if $S = S_{a,b}$ with $a \leq b$, we must have either $0 \leq k \leq a$ or $k = b$: as soon as $k > a$, the span of the lines $L_i$ will contain the directrix of the scroll, and so must consist of the union of the directrix with $n-1-a = b$ lines.

Now let $C$ be a trigonal curve of genus $g \geq 5$, embedded in $\\P^{g-1}$ as a canonical curve, and let $S$ be the scroll containing $C$. We want to describe special linear series $\cD = |D|$. If our linear series has base points, we can delete them; so we'll assume that $|D|$ and $|K-D|$ are base point free. Note that this implies that   both  $r(D) \geq 1$ and $r(K-D) \geq 1$. In addition, it follows by Bertini that a general divisor $D \in \cD$ is reduced, that is, consists of distinct points $p_1,\dots,p_d$.

Now, the first hypothesis, that $r(D) \geq 1$, says that the points $p_1,\dots,p_d$ are linearly dependent. The second hypothesis, that $r(K-D) \geq 1$, says that the points $p_i$ span a subspace of codimension at least 2 in $\PP^{g-1}$ They therefore lie on at least a pencil of hyperplanes; let $H$ be a general hyperplane containing $D$.




. is effective, says that the divisor $D$ lies in a hyperplane section $C \cap H$; let $H$ be a general such hyperplane. At the same time 

canonical image lies on a 2-dim scroll (non -subcanonical embedding only on 3-dim scrolls).  embedding of a trigonal curve lies on the same scroll.Stratification of trigonal curves by Maroni invariants. Dimensions via automorphism groups of scrolls.

\section{Castelnuovo's Theorem}
(Statement only) \fix{we'll need the existence of smooth curves in given classes --- base point freeness of certain divisor classes on the scroll. Theorem: bpf iff they meet both a,b rational normal curves positively. reference to Montreal? better to make a tex file of the essential bit and put it in, as appendix. or ACGH?}

%footer for separate chapter files

\ifx\whole\undefined
%\makeatletter\def\@biblabel#1{#1]}\makeatother
\makeatletter \def\@biblabel#1{\ignorespaces} \makeatother
\bibliographystyle{msribib}
\bibliography{slag}

%%%% EXPLANATIONS:

% f and n
% some authors have all works collected at the end

\begingroup
%\catcode`\^\active
%if ^ is followed by 
% 1:  print f, gobble the following ^ and the next character
% 0:  print n, gobble the following ^
% any other letter: normal subscript
%\makeatletter
%\def^#1{\ifx1#1f\expandafter\@gobbletwo\else
%        \ifx0#1n\expandafter\expandafter\expandafter\@gobble
%        \else\sp{#1}\fi\fi}
%\makeatother
\let\moreadhoc\relax
\def\indexintro{%An author's cited works appear at the end of the
%author's entry; for conventions
%see the List of Citations on page~\pageref{loc}.  
%\smallbreak\noindent
%The letter `f' after a page number indicates a figure, `n' a footnote.
}
\printindex[gen]
\endgroup % end of \catcode
%requires makeindex
\end{document}
\else
\fi
