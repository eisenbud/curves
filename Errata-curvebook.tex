\documentclass[twoside,12pt, leqno]{amsart}
\usepackage{amsmath,amscd,amsthm,amssymb,amsxtra,latexsym,epsfig,epic,graphics}
\usepackage[matrix,arrow,curve]{xy}
\usepackage{graphicx}
\usepackage{diagrams}
\usepackage{MnSymbol}
%\usepackage{pgfplots}
\usepackage{tikz}  %TikZ
\usepackage{color} 
\usetikzlibrary{arrows,calc} 
\usetikzlibrary{decorations.pathmorphing}
\usetikzlibrary{decorations.markings} 
\usetikzlibrary{decorations.pathreplacing} 
\usetikzlibrary{plothandlers}

%\usepackage{amsrefs}
%%%%%%%%%%%%%%%%%%%%%%%%%%%%%%%%%%%%%%%%%
%\textwidth16cm
%\textheig\codim20cm
%\topmargin-2cm
\oddsidemargin.8cm
\evensidemargin1cm


%%%%%Definitions
%%%%%%%%%%%%%%%%%%%%%%%%%%%%
%%%The black board font
%%%%%%%%%%%%%%%%%%%%%%%%%%%
%\newcommand{\bK}{{\bf K}}
\newcommand{\bK}{{\Bbbk}}
\newcommand{\BB}{{\mathbb B}}
\newcommand{\CC}{{\mathbb C}}
\newcommand{\DD}{{\mathbb D}}
\newcommand{\EE}{{\mathbb E}}
\newcommand{\FF}{{\mathbb F}}
\newcommand{\GG}{{\mathbb G}}
\newcommand{\NN}{{\mathbb N}}
\newcommand{\PP}{{\mathbb P}}
\newcommand{\QQ}{{\mathbb Q}}
\newcommand{\RR}{{\mathbb R}}
\newcommand{\TT}{{\mathbb T}}
\newcommand{\XX}{{\mathbb X}}
\newcommand{\YY}{{\mathbb Y}}
\newcommand{\ZZ}{{\mathbb Z}}

\newcommand{\HH}{{\mathrm{H}}}
\newcommand{\ms}{{\mathrm{m}}}
\newcommand{\lc}{{\mathrm{lc}}}
\newcommand{\lm}{{\mathrm{lm}}}
\newcommand{\ec}{{\mathrm{ec}}}
\newcommand{\Ext}{{\mathrm{Ext}}}
\newcommand{\rad}{{\mathrm{rad\;}}}
\DeclareMathOperator{\mult}{mult}
\DeclareMathOperator{\Hilb}{Hilb}
\DeclareMathOperator{\Isom}{Isom}
\newcommand{\V}{{\mathrm V}}
\newcommand{\Id}{{\mathrm I}}
\newcommand{\id}{{\mathrm{id}}}
\newcommand{\coker}{{\mathrm{coker}\,}}

%%%%%%%%%%%%%%%%%%%%%%%%%%%%%%
%%%The mathscript for sheaves
%%%%%%%%%%%%%%%%%%%%% %%%%%%%%%
\newcommand{\s}{\mathcal}
\newcommand{\sA}{\s A}
\newcommand{\sB}{\s B}
\newcommand{\sC}{\s C}
\newcommand{\sD}{\s D}
\newcommand{\sE}{\s E}
\newcommand{\sF}{\s F}
\newcommand{\sG}{\s G}
\newcommand{\sH}{\s H}
\newcommand{\sI}{\s I}
\newcommand{\sJ}{\s J}
\newcommand{\sK}{\s K}
\newcommand{\sL}{{\s L}} %fix
\newcommand{\sM}{\s M}
\newcommand{\sN}{\s N}
\newcommand{\sO}{{\s O}} %fix
\newcommand{\sP}{\s P}
\newcommand{\sQ}{\s Q}
\newcommand{\sR}{\s R}
\newcommand{\sS}{\s S}
\newcommand{\sT}{\s T}
\newcommand{\sU}{\s U}
\newcommand{\sV}{{\s V}}
\newcommand{\sW}{\s W}
\newcommand{\sX}{\s X}
\newcommand{\sY}{\s Y}
\newcommand{\sZ}{\s Z}

\newcommand{\cO}{\s O}
\newcommand{\cF}{\s F}
\newcommand{\cG}{\s G}
\newcommand{\cI}{\s I}
\newcommand{\cL}{{\s L}} %fix
\newcommand{\cM}{\s M}
\newcommand{\cR}{\s R}
\newcommand{\cN}{\s N}
\newcommand{\cT}{\s T}
\newcommand{\cX}{{\s X}}
\newcommand{\lh}{\ell }
\DeclareMathOperator{\Gal}{Gal}
\DeclareMathOperator{\Hess}{Hess}
%%%%%%%%%%%%%%%%%%%%%%%%%%%%%%%%
%% Arrows
%%%%%%%%%%%%%%%%%%%%%%%%%%%%%%%
\newcommand{\inj}{\hookrightarrow}
\newcommand{\surj}{\lra}
\newcommand{\lra}{\longrightarrow}
\newcommand{\lla}{\longleftarrow}
%%%%%%%%%%%%%%%%%%%%%%%%%%%%%%%%%%%%
%\newcommand{\C}{\C}
\newcommand{\openP}{\P}
\newcommand{\uf}{{\bf F}}
\newcommand{\uc}{{\bf C}}
\newcommand{\tensor}{\otimes}
\newcommand{\mi}{{\bf m}}
\newcommand{\tX}{\widetilde{X}}

\def \fix#1 {{\hfill\break \bf (( #1 ))\hfill\break}}
\def \pict#1 {{\hfill\break \bf ((Possible picture: #1 ))\hfill\break}}

\newcommand{\later}{{\bf ** add reference later **}}
\newcommand{\locring}{\mathcal O_{A,p}}
\newlength{\br}
\DeclareMathOperator{\Pic}{Pic}
\newlength{\ho}

\DeclareMathOperator{\Ann}{Ann}
\newcommand{\ann}{{\mathrm{ann}}}
\DeclareMathOperator{\GL}{GL}
\DeclareMathOperator{\Aut}{Aut}
\DeclareMathOperator{\Sym}{Sym}
\DeclareMathOperator{\Spec}{Spec}
\DeclareMathOperator{\Proj}{Proj}
\DeclareMathOperator{\Hom}{Hom}
\DeclareMathOperator{\sHom}{\sH`om}
\DeclareMathOperator{\ord}{ord}
\DeclareMathOperator{\supp}{supp}
\DeclareMathOperator{\im}{im}
\DeclareMathOperator{\depth}{depth}
\DeclareMathOperator{\length}{length}
\DeclareMathOperator{\pd}{pd}
\DeclareMathOperator{\SL}{SL}
\DeclareMathOperator{\SO}{SO}
\DeclareMathOperator{\PSL}{PSL}
\DeclareMathOperator{\PGL}{PGL}
\DeclareMathOperator{\Tor}{Tor}
\AtBeginDocument{\def\div{{\operatorname{div}}}}
\DeclareMathOperator{\Div}{Div}
\DeclareMathOperator{\wdim}{wdim}
\DeclareMathOperator{\cdim}{cdim}
\DeclareMathOperator{\codim}{codim}
\DeclareMathOperator{\rank}{rank}
\DeclareMathOperator{\height}{height}
\DeclareMathOperator{\End}{End}
%\renewcommand{\labelenumi}{(\arabic{enumi})}
%\newcommand{\Ndash}{\nobreakdash--}% for pages 1\Ndash 9

%%%theosdefinitionen
\newcommand{\gm}{\mathfrak m}
\newcommand{\gn}{\mathfrak n}
\def\gr{{\mathfrak {gr}}}
\newcommand{\gp}{\mathfrak p}
\newcommand{\ga}{\mathfrak a}
\newcommand{\gq}{\mathfrak q}
\newcommand{\gP}{\mathfrak P}
\newcommand{\gQ}{\mathfrak Q}

\newcommand{\bx}{\boldsymbol{x}}
\newcommand{\by}{\boldsymbol{y}}

\def\e{{\epsilon}}
\def\TU{{\bf U}}
\def\AA{{\mathbb A}}
\def\BB{{\mathbb B}}
\def\bB{{\mathbb B}}
\def\PP{{\mathbb P}}
\def\P{{\mathbb P}}
\def\QQ{{\mathbb Q}}
\def\FF{{\mathbb F}}
\def\TT{{\mathbb T}}
\def\facet{{\bf facet}}
\def\image{{\rm image}}
\def\name{{\rm name}}
\def\cE{{\cal E}}
\def\cF{{\cal F}}
\def\cG{{\cal G}}
\def\cH{{\cal H}}

\def\sEnd{{\mathcal End}}

\def\cHom{{{\cal H}om}}
\def\fix#1{{\bf ***Fix:} #1 {\bf ***}}
\def\david#1{{\bf *** David:} #1 {\bf ***}}
\DeclareMathOperator{\rH}{{\rm H}}
%\DeclareMathOperator{\Ext}{{\rm Ext}}
\DeclareMathOperator{\Cliff}{{\rm Cliff}}
\DeclareMathOperator{\chara}{{char}}
\DeclareMathOperator{\hess}{{hess}}

\def\fC{{\mathfrak C}}
\def\Tr{{\rm Tr}}
\def\bC{{\mathbb C}}
\def\Gr{{\rm Gr}}
\def\CI{{\mathcal I}}
\def\CH{{\mathcal H}}
%\def\CCH{{\mathcal {CNT}}}
\def\CCH{{\mathcal {HC}}}
\def\rH{{\rm H}}

\def\all{{\{1,\ldots,2g+2\}}}
\def\soc{{\rm soc\,}}
\def\jacobian{{\rm Jac}}
\def\Rbar{{\overline R}}
\def\Ibar{{\overline I}}
\def\mm{{\frak m}}
\def\RR{{\mathcal R}}
\def\Trace{{\rm Tr}}

\def\CO{{\mathcal O}}
\def\CT{{\mathcal T}}
\def\CHom{{\mathcal Hom}}
\def\Spec{{{\rm Spec}\,}}
\def\cone{{{\rm cone}\,}}

\def\tR{{\widetilde R}}
\def\tI{{\widetilde I}}
\def\tJ{{\widetilde J}}
\def\tK{{\widetilde K}}
\def\tH{{\widetilde H}}
\def\tF{{\widetilde F}}

\newarrow{Iso} -----

\def\Abar{{\overline A}}
\def\Rbar{{\overline R}}
\def\Ibar{{\overline I}}
\def\Jbar{{\overline J}}
\def\Kbar{{\overline K}}
\def\abar{{\overline \alpha}}
\def\bbar{{\overline \beta}}
\def\m{{\frak m}}
\def\Rbar{{\overline R}}

\def\gr{{\rm gr}}
\def\init{{\rm in}}

\def\frank#1{{\bf *** Frank:} #1 {\bf ***}}
\def\david#1{{\bf *** David:} #1 {\bf ***}}
\def\lbracket{{[\kern-1.5pt[}}
\def\rbracket{{]\kern-1.5pt]}}

\def\seq#1#2{{#1_{1},\dots,#1_{#2}}}
\def\ff#1{{f_{1},\dots, f_{#1}}}

\makeatletter
\def\Ddots{\mathinner{\mkern1mu\raise\p@
\vbox{\kern7\p@\hbox{.}}\mkern2mu
\raise4\p@\hbox{.}\mkern2mu\raise7\p@\hbox{.}\mkern1mu}}
\makeatother


%%%%%%%%%%%%%%%%%%Silvio's macros for the diagrams
\usepackage{times}
\newdimen\x \x=12pt

%\usepackage{mat\codimime}
\usepackage{color}

%\usepackage{color}
%\usepackage[usenames,dvipsnames,svgnames,table]{xcolor}

\usepackage[breaklinks,bookmarksopen,bookmarksnumbered,urlcolor=blue]{hyperref}
\hypersetup{colorlinks=true,backref=true,citecolor=blue}

%\pagestyle{MYheadings}
%\date{April 2013-December 2015}
\author[David Eisenbud]{David Eisenbud}
\address{Department of Mathematics, University of California at Berkeley, Berkeley, CA 94720, USA}
\email{de@msri.org}

\author[Joe Harris ]{Joe Harris}
\address{Department of Mathematics, Harvard University, Cambridge MA 02138}
\email{harris@math.harvard.edu}

\title{Errata for The Practice of Algebraic Curves}


\begin{document}


\
\maketitle


\renewcommand{\thefootnote}{}%Removing the footnote symbol.
%
p. 9, Theorem 0.3: We should have added the hypothesis that $X\cap Y$
is generically smooth. 

As stated, this version of B\'ezout's theorem can fail when the intersection
scheme $X\cap Y$ is not generically smooth. For example If X is the union of two 2-planes meeting in 
a point in $\PP^{4}$ and $Y$ is a generic 2-plane passing through this point, then
the degree of $X\cap Y$ is 3, not 2 as asserted. (This example can be massaged to
make $X$ into a variety).

However the generically smooth case implies the correct result
$\deg( [X][Y]) = \deg (X) \deg(Y)$, where $[X][Y]$ is the product in the Chow ring.

p. 17 line -11: "it is a" should be "it corresponds to a"

p. 25, line 16: "the map $C \to \phi(C)$ is finite". Correct, but we should have mentioned "quasifinite \& projective implies finite" in the prerequisites section.

p. 30, line 4:  $\div$ prints as the sign for dividing one number by another. Should print as $\rm Div$ . 

p. 31, line -3 : after the summation sign, $dx_1$ should be $dx_0$

p. 36, line -3: $(\sO_{\PP^1}, W)$  -> $(\sO_{\PP^1}(d), W)$

p. 49: line -4, formula for $p_a$: $K_X$ should be $K_S$

p. 50, Theorem 2.8: Delete the $+1$ at the end of the formula.

p. 53, This is not correct. The correct version is given in exercise 2.14. Delete exercise 2.2 (2). 

p. 61, line -1,-2: delete the (incorrect) phrase beginning ", so $\Gamma$ imposes... ."

p. 65, line -9: $H^0_*(I_C/Q)$ should be $H^0_*(\sI_C/Q)$

p. 65, line -5 : $I_C/Q$ should be $\sI_C/Q$

p.72: ``sum of $\Sym^{e_i}V$'' . It would be better to say "a sum of representations $\Sym^dV$
for various $d$ (the symbols $e_i$ were used for the basis of $V$).

p. 78, line 17: "normal sheaf" -> "conormal sheaf"

p. 78, line -10: "is a globally on $C$" -> "is a global section of $\omega_C$."

p. 82, part (2) of the proof of Corollary 4.8. Replace this paragraph (which is nonsense) by:

(2) In this case a divisor $\Gamma$ in the series $\mathcal V$ fails by 2 to impose independent conditions
on forms of degree $d-3$. It follows that if we subtract a point from $\Gamma$ the resulting divisor $\Gamma'$ fails
by at least one to impose independent conditions on forms of degree $d-3$. Thus $\Gamma$ has degree
at least $d$, and by part (1), $\Gamma'$ is contained in a line. Repeating this argument with another point of $\Gamma$,
we see that $\Gamma$ must be the intersection of $C$ with a line; that is, $\mathcal V$ is the series $|\sO_C(1)|$.

p. 83, paragraph beginning on line -7. This is nonsense. Here is a characteristic 0 argument:
We may assume n>1. If the map p-> np were not surjective, it would be constant; equivalently, np~nq for every
pair of points p,q. But then every point would be a ramification point of the map
corresponding to the complete series |np|, so this map would be inseparable.

p. 90: The result of Exercise 4.1 is the same as Corollary 4.8.


p. 90: Exercise 4.3. the $x_0^2$ is multiplying the wrong side of the equation.

p. 91: The statement of the Sylvester Gallai theorem should be:
In any finite set $\Gamma\subset \PP^2_\RR$ not contained in a line
there is a line containing just 2 points of $\Gamma$.

p. 97, last sentence of 2nd para of 5.6: "$\Pic_d(C)$ is a for $\Pic_0(C)$ should say
``$\Pic_d(C)$ is a torsor for $\Pic_0(C)$, which means that if $\sL_0$ is an invertible sheaf of
degree $d$ on $C$, then the map $\Pic_0(C)\to \Pic_d(C)$ sending $\sL to \sL\otimes \sL_0$
is an isomorphism.''

p. 104, line -15: end of line, should be $\geq 4$. Better argument for this para: If $D$ is a special divisor ,
then $|D|$ is a subseries of the canonical series, which itself is not very ample.

Theorem 10.1: $n$ should be $r$.

p. 106, third line before subsection 5.8: Should be $\omega_C \otimes \mathcal L'^{-1}$, where $\mathcal L'$ has degree 1

p. 107: Lemma 5.17: change $\PP^r$ to $\PP^n$ to avoid the conflict of notation between that $r$ and
the superscript on $\Sigma^r_d$. 

p. 107, next to last sentence before the Proof of Martens Theorem: Should say: we see that
$\Sigma^{r-1}_d \setminus \Sigma^{r-1}_d$ is dense in $\Sigma^{r-1}_d$.


p. 107 line -7 Marten's should be Martens'

p. 108 Exc 5.2: This is the wrong defn of ``freely''. Should be: $gx = x$ only when $g$
is the identity.

p. 108 Exc 5.3.2 : $A^2$ should be $\mathbb A.^2$.

p. 114. This whole discussion assumes that the "topological covering space" is a finite covering.

p. 142, line 4: $m$ should be $m_{0}$.

p. 143, Example 7.14: the wording is confusing. Better: 
In the Grassmannian of lines $\GG(1,3)$, the Pl\"ucker coordinates of the line L that is the span of the points $q,r$ 
are the $2\times 2$ minors of the matrix...``

p. 149, line 2: "locally" should be "locally analyitically" (or \'etale locally?) on B in order to find these sections. So in the end, you are using descent to "glue" together the morphism. 

150: Second para of the proof: It might be necessary to make a base change before there are sections sections of your family of genus 1 curvesThis is OK  since to construct an isomorphism of the pullbacks of the families after finite base change.

p. 153 In constructing $Hilb^\circ$,  the curves should be smooth, irreducible and non-degenerate.


p. 187: The proof given for the case of equality in Castelnuovo's theorem is not correct: the top line of the first displayed diagram is NOT left exact.

To fix it, redraw the diagram, eliminating the top row, and replacing $m$ by $k$.

Let $a_{k}$ denote the right-hand map in the  middle row of the 
diagram, and let $b_{k}$ denote the right hand map in the bottom row of the diagram. Equality in Castelnuovo's theorem implies that for all $k$ the
rank of $a_{k}$, which is the number of conditions imposed
on $V_{k}$ by $\Gamma$, is equal to the rank of $b$, 
which is the number of conditions imposed
on $H^{0}(\sO_{C}(k))$ by $\Gamma$

Since the bottom row of the diagram is left exact, the rank of $b$
is $h^{0}(\sO_{C}(k)) - h^{0}(\sO_{C}(k-1))$. Since $V_{k-1}$
is contained in the kernel of $a$, the 
rank of $a$ is $\leq \dim V_{k} -\dim V_{k-1}$, whence
$$
\dim V_{k} -\dim V_{k-1} \geq  h^{0}(\sO_{C}(k)) - h^{0}(\sO_{C}(k-1))
$$
for every $k$. The rest of the proof can be taken starting with
``For large $m$ the restriction map...''

p. 201: last para before 11.3. can intersect... should be cannot intersect....

p. 202: We are rather cavalier about the fiber product, and this can be confusing.

p. 203, statement of Thm 11. delete first half of 2nd line (it's a broken, repeated phrase).

p. 212: Cor 12.5. Hypothesis should be $r\geq 1$.

p. 213: $C_{d-1} = \binom{2d-2}{d-1}/d$ is the correct number.

page 224, between the first two displayed equations: "and the on $C$ to be" presumably should be "and the \emph{ramification divisor} on $C$ to be"

page 242, line 7: "an" should be "on"

page 242, last displayed equation: $d+1$ should be $d$


\bibliographystyle{plain}
\end{document}
