%\documentclass[12pt, leqno]{article}
%\usepackage{amsmath,amscd,amsthm,amssymb,amsxtra,latexsym,epsfig,epic,graphics}
%\usepackage[matrix,arrow,curve]{xy}
%\usepackage{graphicx}
%\usepackage{diagrams}
%%\usepackage{amsrefs}
%%%%%%%%%%%%%%%%%%%%%%%%%%%%%%%%%%%%%%%%%%
%%\textwidth16cm
%%\textheight20cm
%%\topmargin-2cm
%\oddsidemargin.8cm
%\evensidemargin1cm


\ifx\whole\undefined
\documentclass[12pt, leqno]{book}
\input style-for-curves.sty

%%%%%Definitions
%\input preamble.tex
%\input style-for-curves.sty
%\def\TU{{\bf U}}
%\def\AA{{\mathbb A}}
%\def\BB{{\mathbb B}}
%\def\CC{{\mathbb C}}
%\def\QQ{{\mathbb Q}}
%\def\RR{{\mathbb R}}
%\def\facet{{\bf facet}}
%\def\image{{\rm image}}
%\def\cE{{\cal E}}
%\def\cF{{\cal F}}
%\def\cG{{\cal G}}
%\def\cH{{\cal H}}
%\def\cHom{{{\cal H}om}}
%\def\h{{\rm h}}
% \def\bs{{Boij-S\"oderberg{} }}

\makeatletter
\def\Ddots{\mathinner{\mkern1mu\raise\p@
\vbox{\kern7\p@\hbox{.}}\mkern2mu
\raise4\p@\hbox{.}\mkern2mu\raise7\p@\hbox{.}\mkern1mu}}
\makeatother

%%
%\pagestyle{myheadings}
\date{August 4, 2017}
%\date{}
\title{To Fix}
%{\normalsize ***Preliminary Version***}} 
\author{David Eisenbud and Joe Harris }

\begin{document}

\maketitle

\setlength{\parskip}{5pt}

 
 \section{Items to discuss with Silvio}
 
 \begin{enumerate}
 
 \item The exercises, with whatever hints there are, should be copied to a separate document, and the format of hints regularized: 
 exercise/blank line or hrule/hing. The hints should then be deleted from the original document. The numbering in the new document should of course correspond to that in the book. Watch out for references to the text from the hints.
 
 This will be a document on the book's website at the AMS, 
 mentioned in but not distributed with the book.
 
 We don't yet know the ams website address: it needs to be filled in a line in the intro when it is knownl
   
 \item Capitalization: named chapters, sections etc get a capital; not generic references. First word of a title only gets a cap.
 
 \item statements sometime have a superfluous future tense (``the point will be smooth'',``this will be proved..." and the like.) Make present tense.
 
 \end{enumerate}
\end{document}



