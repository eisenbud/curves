%header and footer for separate chapter files

\ifx\whole\undefined
\documentclass[12pt, leqno]{book}
\usepackage{graphicx}
\input style-for-curves.sty
\usepackage{hyperref}
\usepackage{showkeys} %This shows the labels.
%\usepackage{SLAG,msribib,local}
%\usepackage{amsmath,amscd,amsthm,amssymb,amsxtra,latexsym,epsfig,epic,graphics}
%\usepackage[matrix,arrow,curve]{xy}
%\usepackage{graphicx}
%\usepackage{diagrams}
%
%%\usepackage{amsrefs}
%%%%%%%%%%%%%%%%%%%%%%%%%%%%%%%%%%%%%%%%%%
%%\textwidth16cm
%%\textheight20cm
%%\topmargin-2cm
%\oddsidemargin.8cm
%\evensidemargin1cm
%
%%%%%%Definitions
%\input preamble.tex
%\input style-for-curves.sty
%\def\TU{{\bf U}}
%\def\AA{{\mathbb A}}
%\def\BB{{\mathbb B}}
%\def\CC{{\mathbb C}}
%\def\QQ{{\mathbb Q}}
%\def\RR{{\mathbb R}}
%\def\facet{{\bf facet}}
%\def\image{{\rm image}}
%\def\cE{{\cal E}}
%\def\cF{{\cal F}}
%\def\cG{{\cal G}}
%\def\cH{{\cal H}}
%\def\cHom{{{\cal H}om}}
%\def\h{{\rm h}}
% \def\bs{{Boij-S\"oderberg{} }}
%
%\makeatletter
%\def\Ddots{\mathinner{\mkern1mu\raise\p@
%\vbox{\kern7\p@\hbox{.}}\mkern2mu
%\raise4\p@\hbox{.}\mkern2mu\raise7\p@\hbox{.}\mkern1mu}}
%\makeatother

%%
%\pagestyle{myheadings}

%\input style-for-curves.tex
%\documentclass{cambridge7A}
%\usepackage{hatcher_revised} 
%\usepackage{3264}
   
\errorcontextlines=1000
%\usepackage{makeidx}
\let\see\relax
\usepackage{makeidx}
\makeindex
% \index{word} in the doc; \index{variety!algebraic} gives variety, algebraic
% PUT a % after each \index{***}

\overfullrule=5pt
\catcode`\@\active
\def@{\mskip1.5mu} %produce a small space in math with an @

\title{Personalities of Curves}
\author{\copyright David Eisenbud and Joe Harris}
%%\includeonly{%
%0-intro,01-ChowRingDogma,02-FirstExamples,03-Grassmannians,04-GeneralGrassmannians
%,05-VectorBundlesAndChernClasses,06-LinesOnHypersurfaces,07-SingularElementsOfLinearSeries,
%08-ParameterSpaces,
%bib
%}

\date{\today}
%%\date{}
%\title{Curves}
%%{\normalsize ***Preliminary Version***}} 
%\author{David Eisenbud and Joe Harris }
%
%\begin{document}

\begin{document}
\maketitle

\pagenumbering{roman}
\setcounter{page}{5}
%\begin{5}
%\end{5}
\pagenumbering{arabic}
\tableofcontents
\fi



\chapter{Curves of genus 0 and 1 exercises}\label{genus 0 and 1 exercises}

\begin{exercise}
 Let $\nu_d: \PP^r \to \PP^{\binom{r+d}{r}-1}$ be the $d$-Veronese map, and let $C\subset \PP^r$ be the rational normal curve of degree $r$. Is $\nu_d(C)$ nondegenerate? If not, what is the dimension of its linear span (that is, of the smallest linear
 space that contains it?
\end{exercise}

\begin{exercise}
Establish the analog of Proposition~\ref{rnc on most quadrics} for hypersurfaces of any degree $m$, that is to say no irreducible, nondegenerate curve in $\PP^r$ lies on more hypersurfaces of degree $m$ than the rational normal curve.
To do this, let $C\subset \PP^d$ be any irreducible nondegenerate curve, and use the exact sequences
$$
0 \to \cI_{C/\PP^d}(l-1) \to \cI_{C/\PP^d}(l) \to \cI_{\Gamma/\PP^{d-1}}(l) \to 0.
$$ 
with $2 \leq l \leq m$ to show that
$$
h^0(\cI_{C/\PP^d}(m)) \leq  \binom{d+m}{m} - (md+1)
$$
with equality only if $C$ is a rational normal curve.
\end{exercise}

\begin{exercise}
Prove directly  the special case $r=3$: that the twisted cubic is the unique irreducible, nondegenerate space curve lying on three quadrics. (Hint: if $C \subset \PP^3$ is such a curve lying on three quadrics, what must be the intersection of two of the quadrics containing $C$?)
\end{exercise}


\begin{exercise}\label{F2}
Let $Q \subset \PP^3$ be a cone over a smooth conic curve in $\PP^2$; let $\pi : S \to Q$ be the blow-up of $Q$ at the vertex and $E \subset S$ the exceptional divisor of the blow-up.
\begin{enumerate}
\item Show that $S$ is smooth.
\item Show that the Picard group $\Pic(S)$ is freely generated by two classes, the class $f$ of the proper transform of a line in $Q$ and the class $e$ of the exceptional divisor.
\item\label{intersections on a quadric} Show that the intersection pairing on $\Pic(S)$ is given by
$$
f \cdot f = 0; \quad f \cdot e = 1 \quad \text{and} \quad e \cdot e = -2
$$
(Hint: one way to do the last of these is to show that a hyperplane section of $Q$ has class $h = e + 2f$ and use $h^2 = 2$.)
\item Show that the canonical class $K_S = -2e-4f$.
\item Show that if $\tilde C \subset S$ is a curve with class $ae + bf$, and
$C = \pi(\tilde C) \subset Q \subset \PP^3$, then
$$
\deg(C) = b \quad \text{and} \quad g(\tilde C) = a(b-a) -b +1
$$
\item Deduce that there does not exist a smooth rational quartic curve on a quadric cone.
\end{enumerate}
\end{exercise}

\begin{exercise}
As a consequence of our description of rational quartic curves, show that a general $g^3_4$ on $\PP^1$ is uniquely expressible as a sum of the $g_1^1$ and a $g^1_3$
(in other words, a general 4-dimensional vector space of quartic polynomials on $\PP^1$ is uniquely expressible as the product of a 2-dimensional vector space of cubics and the 2-dimensional space of linear forms.
\end{exercise}

\begin{exercise}
Show that, up to projective equivalence, there is a 1-parameter family of rational quartic curves in $\PP^3$ 
by constructing an invariant that distinguishes them. (Hint: think of a curve of type $(1,3)$ on a quadric as a the graph of a degree 3 map $\PP^1 \to \PP^1$, and use Hurwitz' Theorem.)
\end{exercise}






\begin{exercise}\label{Castelnuovo uniqueness}
Complete the proof of Proposition~\ref{points on rnc} by showing that if $C, C' \subset \PP^n$ are two rational normal curves and $\#(C \cap C') \geq n+3$, then $C = C'$. (Hint: use induction on $n$.)
\end{exercise}


\begin{exercise}\label{rnc and representations}
Let $V = \CC\cdot e_1\oplus \CC\cdot e_2$ be a 2-dimensional vector space. 

The group $SL_2= SL(V)$ acts on the rational normal curve of degree $d$ through automorphisms induced from its action on
 on the ambient space $\PP^d$ of the rational normal curve, which may be identified with $\PP(\Sym^d(V))$.

In~\cite[pp. 146--150]{Fulton-Harris} it is shown that
 every finite dimensional rational 
representation of $V$ is a direct sum of representations of the form $\Sym^e(V)$ for various $e\geq 0$. Moreover, it is often easy to understand
how a given representation decomposes by looking at the action of
$$
\alpha := \begin{pmatrix}
t&0\\
0&t^{-1}
\end{pmatrix}
\in SL(V).
$$
Note that $\Sym^e(V)$ is spanned by ``weight vectors" ($\equiv$ eigenvectors of $\alpha$) $w_s := e_1^{e-s} e_2^{s}$ 
which satisfy $\alpha w_s = t^{e-2s}$ for $s = 0, \dots e$.
To decompose an arbitrary representation $W$, knowing that $W$ is a direct sum of $\Sym^{e_i}V$, it is enough to know the 
eigenvalues for the action of $\alpha$: We begin by finding an element $w\in W$ that
is an eigenvector of $\alpha$ and transforms by $\alpha$ as
as $\alpha w = t^mw$ with the highest possible $m$ (this is called a ``highest weight vector''). This element $w$ must be contained
in a summand $\Sym^m(V)$, and after removing the eigenvalues of the action of $SL_2$ on $\Sym^m(V)$, we continue. 
\begin{enumerate}
 \item Use this method to show that 
\begin{align*}
&\Sym^d(V)\otimes Sym^d(V)= \Sym^{2d}(V) \oplus  \Sym^{2d-2}(V) \oplus \Sym^{2d-4}(V) \cdots\\
 &\Sym^2(\Sym^d(V))= \Sym^{2d}(V) \oplus \Sym^{2d-4}(V)\oplus \Sym^{2d-8}(V) \cdots\\
 &\bigwedge^2(\Sym^d(V))= \Sym^{2d-2}(V) \oplus \Sym^{2d-6}(V)\oplus \Sym^{2d-10}(V) \cdots
\end{align*}
  (where we take $\Sym^{m}(V)=0$ when $m<0$
 \item Show that the space of quadrics containing the rational normal curve is a representation of $SL_2$ of the form
 $$
 \Sym^{2d-4}(V)\oplus \Sym^{2d-8}(V) \cdots
 $$
  \item Show  there is a distinguished nonsingular skew-symmetric form (up to scalars) on the ambient space of the twisted cubic; in particular
  is, given a twisted cubic in $\PP^3$ there is a distinguished plane containing each point of $\PP^3$.
 \item Show that if $d$ is divisible by 4 there is a distinguished quadric in the ideal of the rational normal curve.
\end{enumerate}
\end{exercise}

\begin{exercise}
Let $\PP^1 \hookrightarrow C \subset \PP^3$ be a twisted cubic. Show that the normal bundle $\cN_{C/\PP^3}$ (defined to be the quotient of the restriction $T_{\PP^3}|_C$ to $C$ of the tangent bundle  of $\PP^3$  by the tangent bundle $T_C$) is 
$$
\cN_{C/\PP^3} \cong \cO_{\PP^1}(5) \oplus  \cO_{\PP^1}(5).
$$
Hint: for any point $p \in C$, let $L_p \subset \cN_{C/\PP^3}$ be the sub-line bundle of $\cN_{C/\PP^3}$ whose fiber over any point $q \neq p \in C$ is the one-dimensional subspace of $(\cN_{C/\PP^3})_q$ spanned by the line $\overline{p,q}$. (This of course only defines a sub-line bundle of $\cN_{C/\PP^3}$ over $C \setminus \{p\}$, but there is a unique extension to a sub-line bundle of $\cN_{C/\PP^3}$ over all of $C$.) Show that for $p \neq p'$ we have
$$
\cN_{C/\PP^3} = L_p \oplus L_{p'}.
$$
\end{exercise}

\begin{exercise}
Let $\PP^1 \hookrightarrow C \subset \PP^d$ be a rational normal curve. Show that the normal bundle $\cN_{C/\PP^d}$  is 
$$
\cN_{C/\PP^d} \cong \bigoplus_{i=1}^{d-1} \cO_{\PP^1}(d+2).
$$
\end{exercise}

\begin{exercise}
In the situation of the preceding problem, the set  of direct summands of $\cN_{C/\PP^d} $ is a projective space $\PP^{d-2}$. How does the  group of automorphisms of $\PP^d$ carrying $C$ to itself act on this $\PP^{d-2}$?

\end{exercise}

\begin{exercise}
 Consider the matrix
 $$
M \; = \; \begin{pmatrix}
x_0 & x_1 & \dots & x_{d-1} \\
x_1 & x_2 & \dots & x_d
\end{pmatrix}
$$
over the polynomial ring $S:=\CC[x_0,\dots, x_d]$. Note that the minors vanish at a point if and only if
a ``generalized row'' of $M$---that is, a linear combination of the two rows---vanishes; these are the points
of $C$.
In the text we showed that $2\times 2$ minors of $M$ set-theoretically define the ideal of the rational normal curve
$C$ of degree $d$. Regarding $M$ as a homomorphism $M: \sO_{\PP^d}^d(-1) \to \sO_{\PP^d}^2$, show that the  cokernel of
$M$ is an invertible sheaf on $C$ with (at least) two sections. Use it to define a one-to-one morphism from the scheme defined by the minors of $M$ to $\PP^1$, whose fibers are reduced points.
\end{exercise}




\input footer.tex