%header and footer for separate chapter files

\ifx\whole\undefined
\documentclass[12pt, leqno]{book}
\usepackage{graphicx}
\input style-for-curves.sty
\usepackage{hyperref}
\usepackage{showkeys} %This shows the labels.
%\usepackage{SLAG,msribib,local}
%\usepackage{amsmath,amscd,amsthm,amssymb,amsxtra,latexsym,epsfig,epic,graphics}
%\usepackage[matrix,arrow,curve]{xy}
%\usepackage{graphicx}
%\usepackage{diagrams}
%
%%\usepackage{amsrefs}
%%%%%%%%%%%%%%%%%%%%%%%%%%%%%%%%%%%%%%%%%%
%%\textwidth16cm
%%\textheight20cm
%%\topmargin-2cm
%\oddsidemargin.8cm
%\evensidemargin1cm
%
%%%%%%Definitions
%\input preamble.tex
%\input style-for-curves.sty
%\def\TU{{\bf U}}
%\def\AA{{\mathbb A}}
%\def\BB{{\mathbb B}}
%\def\CC{{\mathbb C}}
%\def\QQ{{\mathbb Q}}
%\def\RR{{\mathbb R}}
%\def\facet{{\bf facet}}
%\def\image{{\rm image}}
%\def\cE{{\cal E}}
%\def\cF{{\cal F}}
%\def\cG{{\cal G}}
%\def\cH{{\cal H}}
%\def\cHom{{{\cal H}om}}
%\def\h{{\rm h}}
% \def\bs{{Boij-S\"oderberg{} }}
%
%\makeatletter
%\def\Ddots{\mathinner{\mkern1mu\raise\p@
%\vbox{\kern7\p@\hbox{.}}\mkern2mu
%\raise4\p@\hbox{.}\mkern2mu\raise7\p@\hbox{.}\mkern1mu}}
%\makeatother

%%
%\pagestyle{myheadings}

%\input style-for-curves.tex
%\documentclass{cambridge7A}
%\usepackage{hatcher_revised} 
%\usepackage{3264}
   
\errorcontextlines=1000
%\usepackage{makeidx}
\let\see\relax
\usepackage{makeidx}
\makeindex
% \index{word} in the doc; \index{variety!algebraic} gives variety, algebraic
% PUT a % after each \index{***}

\overfullrule=5pt
\catcode`\@\active
\def@{\mskip1.5mu} %produce a small space in math with an @

\title{Personalities of Curves}
\author{\copyright David Eisenbud and Joe Harris}
%%\includeonly{%
%0-intro,01-ChowRingDogma,02-FirstExamples,03-Grassmannians,04-GeneralGrassmannians
%,05-VectorBundlesAndChernClasses,06-LinesOnHypersurfaces,07-SingularElementsOfLinearSeries,
%08-ParameterSpaces,
%bib
%}

\date{\today}
%%\date{}
%\title{Curves}
%%{\normalsize ***Preliminary Version***}} 
%\author{David Eisenbud and Joe Harris }
%
%\begin{document}

\begin{document}
\maketitle

\pagenumbering{roman}
\setcounter{page}{5}
%\begin{5}
%\end{5}
\pagenumbering{arabic}
\tableofcontents
\fi


\chapter{Free resolutions and canonical curves}
\label{SyzygiesChapter}

\def\length{{\rm length}}

In Chapter~\ref{LinkageChapter} we related the resolutions of curves in $\PP^3$ to their Hartshorne-Rao modules. The
simplest case was that in which the Hartshorne-Rao module vanishes, that is, the case of arithmetically Cohen-Macaulay curves.
In this chapter we first quote a result that is a step toward understanding the structure of free resolutions, and use it to explain how the condition that a curve in $\PP^r$ is arithmetically Cohen-Macaulay manifests itself in the 
free resolution of the homogeneous ideal of the curve.  We will then introduce the Eagon-Northcott complexes, and explain their relation to the free resolutions of canonical curves. We close the chapter with an explanation of Green's conjecture, which proposes a way in which the intrinsic geometry
of a curve may be connected to the shape of its minimal free resolution of its ideal in the canonical embedding.

\begin{remark}
 There are two related contexts for the results of this section:  modules over a local ring, and graded modules over a polynomial ring whose variables have positive degree. These are parallel, essentially because Nakayama's lemma works in both cases. We will simply work with local rings or polynomial rings over a field with variables in degree 1 and leave the translations to the reader.\end{remark}

\section{Free resolutions}
A more complete presentation of the material in this section may be found in~\cite[Chapter 19]{Eisenbud1995}.


Let $M$ be a finitely generated graded module over $S := \CC[x_0, \dots x_r]$. A \emph{free resolution} of $M$ is an exact complex
of graded free modules, with maps of degree 0:
$$
(\FF, \phi):\quad 0\lTo N \lTo^\epsilon \bigoplus_jS(-j)^{\beta_{0,j}} \lTo\cdots
 \lTo^{\phi_t}\bigoplus_jS(-j)^{\beta_{t,j}}\lTo 0.
$$
Here $S(-j)$ denotes the graded free module of rank 1 with generator in degree $j$.
%$$
%\FF: \cdots \rTo F_s \rTo^{\phi_s} F_{s-1} \rTo^{\phi_{s-1}} \cdots \rTo F_1 \rTo^{\phi_1}  F_0 \rTo^\epsilon M \rTo 0.
%$$
(The map to $N$ is not considered part of the free resolution). The resolution is \emph{minimal} if a minimal set of generators of $F_i$ maps to a minimal set of generators of the kernel of the following map,
or equivalently (by Nakayama's lemma) the maps in $\FF\otimes_S \CC$ are all 0.

The first examples of a minimal free resolution are the Koszul complexes
(first defined, despite the name, in \cite{Cayley}, and repeated, as examples, 
in \cite{Hilbert1890}), which resolve $S/I$ when
 $I = (f_1,\dots, f_t)$
is a complete intersection, that is, $f_1,\dots, f_t$ is a regular sequence, with $\deg f_i = d_i$.  For $t = 2,3$, these look like

$$
%0\lTo S/I 
%\lTo^\epsilon 
S
\lTo^{ \begin{pmatrix}
f_1&f_2
\end{pmatrix}} 
S(-d_1)\oplus S(-d_2) 
\lTo^{\begin{pmatrix}
-f_2\\f_1
\end{pmatrix}}
 S(-d_1-d_2)\lTo 0,
$$
and
%\begin{tiny}
$$
%0\lTo S/I 
%\lTo^\epsilon 
S
\lTo^{
 \begin{pmatrix}
f_1&f_2&f_3
\end{pmatrix}} 
F_1\lTo^{
\begin{pmatrix}
 0&f_3&-f_2\\
 -f_3&0&f_1\\
 f_2&-f_1&0
\end{pmatrix}}
F_2
\lTo^{\begin{pmatrix}
f_1\\f_2\\f_3
\end{pmatrix}}
 F_3 \lTo 0,
$$
%\end{tiny}

where
$$
F_1 = \bigoplus_{j=1}^3S(-d_j),\
 F_2=
\bigoplus_{1\leq i <j\leq 3} S(-d_i-d_j) \text{ and }
F_3 = 
S(-d_1-d_2-d_3).
$$
%%

%and
%
%$$
%\begin{aligned}
%0\rTo S \rTo^{
%\begin{pmatrix}
%f_1\\f_2\\f_3
%\end{pmatrix}} S^3&\rTo{
%\begin{pmatrix}
% 0&f_3&-f_2\\
% -f_3&0&f_1\\
% f_2&-f_1&0
%\end{pmatrix}}
% S^3\rTo^{
% \begin{pmatrix}
%f_1&f_2&f_3
%\end{pmatrix}}
%S\rTo S/I \rTo 0
%\end{aligned}
%$$

Hilbert's Basis and Syzygy theorems together imply that every finitely generated $S$-module has a finite free resolution
of length no greater than the number of variables, $r+1$. (Fundamental
results of Auslander, Buchsbaum and Serre say that a local ring $R$ is \emph{regular}---that is, the Krull dimension of $R$ is equal to the minimal number
of generators of $\gm$---if and only if the minimal free resolution of the residue field is finite, in which case
the minimal free resolution of every module is finite.)

To construct this, suppose that a minimal homogeneous set of generators of $N$
contains $\beta_{0,j}$ generators of degree $j$ for each $j$; this defines a degree 0 map $\epsilon$
from 
$
\oplus_jS(-j)^{\beta_{0,j}}
$
onto $N$. We proceed similarly with the kernel of $\epsilon$, and continue to construct the whole resolution.
Hilbert's syzygy theorem \cite[Corollary 19.7]{Eisenbud1995} implies that the process ends after $t\leq r$ steps with a map $\phi_t$ whose
kernel is 0. 

The number $t$ is then the projective dimension of $N$. 
Equivalently, 
$$
\pd\ M = \max\{t \mid \Ext_S^t(M,k) \neq 0 \}.
$$
It follows from the Auslander-Buchsbaum
theorem~\cite[Theorem 19.9]{Eisenbud1995} that $t \geq \codim \ann_S(N)$, the codimension of the support of $N$.
Minimal free resolutions of a given module $M$ are all isomorphic.

\section{Classification of 1-generic $2\times f$ matrices}\label{Kronecker}

We can use the uniqueness of minimal free resolutions to give a simple proof of Kronecker's classification of 1-generic $2\times f$
 matrices~\cite{Gantmacher}, which was announced in Chapter~\ref{ScrollsChapter}.
 
\begin{theorem}\label{matrix pencils}
 Every
 1-generic $2 \times f$ matrix of linear forms can be transformed by row and column operations and a linear change
 of variables to one of the type shown in
Corollary~\ref{equations of scrolls}, and thus the minors of any 1-generic matrix define a scroll. 
\end{theorem}


To prove this result we first reinterpret the 1-generic condition:
We have observed that an
$a\times b$ matrix of linear forms in $c$ variables is the same as a $\CC$-linear map of vector spaces
$A \otimes B \to C$, where $A, B$ and $C$ have dimensions $a,b$ and $c$ respectively. Such a map
can be viewed in several ways, for example as a map $C^{*} \otimes B\to A^{*}$---in other words, a $c\times b$ matrix in $a$ variables---or equivalently an $a$-dimensional family of $c\times b$ matrices (and similarly for other permutations of $A,B,C$). 
This bit of trivial formalism pays off in the following observation:

\begin{proposition}\label{reinterpretation of 1-generic}
An  $a\times b$ matrix of linear forms in $c$ variables $M$ corresponding to $A\otimes B \to C$ is 1-generic if and only if the $c \times b$ matrix $N$ of linear forms in $a$ variables corresponding to $C^{*}\otimes B \to A$ has constant rank $b$; that is, 
for point $x$ in $\PP^{a-1}$, the rank of $N$ evaluated at $x$ is $b$.
\end{proposition}

\begin{proof}
The $b$ columns of $N$ correspond to the $b$ columns of $M$, while the rows of $N$ are indexed
by the $c$ variables in $M$ and the variables in $N$ are indexed by the $a$ rows of $M$. Thus the
evaluation of $N$ at a point $x$ corresponds to a generalized row of $M$; and the image of $N(x)$
is the span of the variables in that generalized row. The matrix $M$ is 1-generic if the dimension
of that span is $b$ for every generalized row.
\end{proof}

Thus the classification of 1-generic $2\times f$ matrices of linear forms in $r+1$ variables of Theorem~\ref{matrix pencils} is equivalent to the classification
of \emph{matrix pencils} of constant maximal rank---that is $f \times (r+1)$ matrices of linear forms over $\PP^{1}$ with constant rank $r$.
Such ``matrix pencils'' were first classified by Kronecker; see 
\cite[Vol. II, Chapter 12]{Gantmacher} for an exposition, and ~\cite{Eisenbud-Harris-Centennial} for a geometric approach.
We will give a still simpler proof:

\begin{proof}[Proof of Theorem~\ref{matrix pencils}]
Let $M$ be a 1-generic $2\times b$ matrix in $r+1$ variables. We may assume that the span of the entries
is equal to the vector space of linear forms in $\PP^{r}$. The associated 
$(r+1)\times b$ matrix
$$
N: \sO_{\PP^{1}}(-1)^{b} \to \sO_{\PP^{1}}^{r+1}
$$
 of linear forms over $\PP^{1}$ has constant rank $b$.
For every point $x\in \PP^{1}$ the scalar matrix $N(x)$ is a split inclusion, and thus
$\coker N$ is a vector bundle, necessarily isomorphic to $\sum_{i =1}^{d}\sO_{\PP^{1}}(a_{i})$ for some
integers $a_{i}\geq 0$. 

Regarded as a matrix over $\CC[s,t]$ it follows that the exact sequence
$$
0\to \sO_{\PP^{1}}(-1)^{b} \rTo^{N} \sO_{\PP^{1}}^{r+1}  \rTo \bigoplus_{i = 1}^{d}\bigl( \oplus_{j=0}^{\infty}H^{0}(\sO_{\PP^{1}}(j-a_{i}))\bigr)\rTo 0
$$
is a minimal free resolution. It follows that the map $N$ is the direct sum of the minimal free resolutions over $\CC[s,t]$
of the modules 
$$
\bigoplus_{j=0}^{\infty}H^{0}(\sO_{\PP^{1}}(j-a_{i})) = (s,t)^{a_{i}}.
$$
As we will explain in Example~\ref{res of max ideal power}, the minimal presentation of $(s,t)^{a_{i}}$ is the $(a_{i}+1)\times a_{i}$ matrix
$$
\begin{pmatrix}
s&0&0&\dots&0\\
-t&s&0&\dots&0\\
0&-t&s&\dots&0\\
0&0&-t&\dots&0\\
\vdots&&\cdots&&\vdots\\
0&&\dots&\ddots&s\\
0&&\dots&\dots&-t\\
\end{pmatrix}
$$
The direct sum decomposition of $N$ corresponds to a block decomposition of $M$,
and thus after a change of bases and variables the matrix $N$ is transformed into the direct sum of these matrices.
Translating this form back to a $2\times (r+d-1)$ matrix $M$, we have transformed $M$ into a matrix of 
the desired form.
\end{proof}

\subsection{How to look at a resolution}
As is apparent even in the example $t=3$ above, free resolutions can be bulky to describe; the 
\emph{Betti table} is a compact representation of the numerical information in the resolution.
Suppose that 
$F$ is a minimal free resolution of a module $M$ as illustrated in the beginning of the previous section. Since we choose a minimal set of generators at each stage, the matrices of the $\phi_i$ have entries in the maximal
ideal $(x_0,\dots x_r)$, and thus each $\beta_{i+1, j}$ must be strictly greater than some $\beta_{i,j}$. For this reason it
is convenient to tabulate the numbers so that $\beta_{i,j}$ is in the $i$-th column and $(j-i)$-th row:
$$
\scriptsize{
\begin{matrix} 
j\backslash i&\vline&0   &  1    & \cdots & n    \cr\hline
\vdots&\vline&\vdots&\vdots & \cdots    &\vdots     \cr 
       0&\vline&\beta_{0,0}&\beta_{1,1}&\cdots&\beta_{n,n}\cr
       1&\vline&\beta_{0,1}&\beta_{1,2}&\cdots&\beta_{n,n+1}\cr
\vdots&\vline&\vdots&\vdots & \cdots    &\vdots     \cr 
\end{matrix}.
}
$$         

\begin{example}
 The Koszul complex that resolves the homogeneous coordinate ring $S/(Q, F_1, F_2)$ of the complete intersection of  2 quadrics and a cubic in $\PP^3$ has the form
$$
S \lTo S(-2)\oplus S(-3)^2 \lTo S(-5)^2\oplus S(-6) \lTo S(-8) \lTo 0,
$$
which has Betti table:

\centerline{\scriptsize
\begin{tabular}{r|ccccc} 
$j\backslash i$&0&1&2&3\\ 
\hline 
0&1&$-$&$-$&$-$\\ 
1&$-$&1&$-$&$-$\\
2&$-$&2&$-$&$-$\\
3&$-$&$-$&$2$&$-$\\
4&$-$&$-$&$1$&$-$\\
5&$-$&$-$&$-$&$1$\\
\end{tabular}.}
\end{example}

 To simplify our language, we will speak of the \emph{Betti table of a scheme $X$} rather than the ``Betti table of the minimal free resolution of the ideal of $X$".

\subsection{When is a finite free complex a resolution?}
How does a free resolution over $S := \CC[x_0, \dots x_r]$ ``know'' to end no later than the $(r+1)$-st step? 
The following result describes a sense in which the maps in the resolution change as the resolution continues.
The result is proven in~\cite[Theorem 20.9]{Eisenbud1995}. 

A central role in the theorem is played by the ideals of minors of the differentials in the complex: If $\phi: F\to G$ is a map of finitely generated free $R$-modules with cokernel $M$, then
$I_t(\phi)$ denotes the ideal generated by all the $t\times t$ minors (=subdeterminants) of a matrix representing $\phi$; this is independent of the choice of bases used to represent $\phi$ as a matrix.
The ideal $I_{\rank G -j}(\phi)$ depends only on $M$; it called the $j$-th Fitting ideal of $\coker \phi$  and 
usually written $\Fitt_{j}(M)$. Some basic properties of these ideals are
given in Exercise~\ref{Fitt}, where the reader may show that the annihilator of $M$ has the
same radical as $\Fitt_{0}(M)$. Actually more is true: 

\begin{fact}
If $F\to G \to M \to 0$ is an exact sequence of finitely generated $R$-modules, then 
$$
\ann_{R}(\coker \phi)^{\rank G}\subset Fitt_{0}(\phi) \subset \ann_{R}(\coker \phi).
$$
For this and other such inequalities, see~\cite{MR476736}.
\end{fact}

The \emph{rank} of $\phi: F\to G$ is the largest size of a nonvanishing minor of a matrix for $\phi$,
or equivalently the largest $k$ such that the exterior power $\wedge^{k}\phi : \wedge^{k}F \to \wedge^{k}G$
is nonzero. If $r:= \rank \phi = \rank G$, then the ideal $I(\phi): = I_{\rank \phi}(\phi)$ plays a special role: the cokernel of $\phi$
is projective (= locally free) if and only if $I(\phi) = R$. If $I(\phi)$ contains a non-zerodivisor, then
the construction localizes, and we see that $I(\phi)$ defines the locus  $P\in \Spec R$ where $(\coker \phi)_{P}$
is not free. 


Recall that the grade of an ideal is the length of a maximal regular sequence
contained in it, or $\infty$ if $I=R$. If $R$ is a Cohen-Macaulay ring---for example $\CC[x_{0},\dots, x_{r}]$---then the grade of any proper ideal is equal to its codimension, so grade becomes a geometric notion.
Readers less familiar with commutative algebra will lose little if they stick with the case when $R$ is
regular, or even the case when $R$ is $\CC[x_{0},\dots, x_{r}]$. This suffices, for example for the applications
of the Eagon-Northcott complex described below.


\begin{theorem}\label{WMACE}
 Let $R$ be a Noetherian ring and let
 $$ 
\FF:  F_0\lTo^{\phi_1}F_1 \lTo \cdots \lTo F_{n-1}\lTo^{\phi_n} F_n\lTo 0
 $$
be a finite complex of free $S$-modules. Set $r_i := \rank \phi_i$. 
The complex $\FF$ is \emph{acyclic} (that is, $H_i(\FF) = 0$ for all $i>0$) if and only if
\begin{enumerate}
 \item $\rank F_i = r_i+r_{i+1}$; and
 \item $\grade I_{r_{i}}(\phi_i) \geq i$.
\end{enumerate}
for all $i$.
\qed
\end{theorem}

In particular, if $R$ is Noetherian, then a map of finitely generated free modules  $G\lTo^{\phi}F$ is injective 
if and only if $I_{\rank F}(\phi)$ contains a non-zerodivisor.

A familiar case occurs when  $r=1$ and $R$ is a domain. In this case the theorem says that a map $F_1\to F_0$ is a monomorphism iff it becomes a monomorphism after tensoring with the field of rational functions $K$, which follows from the flatness of
localization and the fact that $F_1$ is torsion-free, so that
$F_1 \subset F_1 \otimes K$. 

To see the relevance of the second hypothesis to the conclusion, suppose for a moment that $R$ is
a regular local ring of dimension $r$, and suppose that the complex $\FF$ is acyclic. The hypothesis $\grade I(\phi_{d+1}) \geq d+1$ can only be satisfied if $I(\phi_{d+1}) = R$ (so that its grade is $\infty$ by convention). This  is equivalent to the cokernel of $\phi_{d+1}$ being free. Thus the theorem ``explains'' why a minimal free resolution
has length $\leq r+1$.

\section{Depth and the Cohen-Macaulay property}

If $M$ is a graded  $\CC[x_0, \dots x_r]$-module then an \emph{$M$-regular sequence} is a sequence of homogeneous polynomials
$f_1,\dots,f_m \in (x_0,\dots, x_r)$ such that $f_1$ is a non-zerodivisor on $M$, $f_2$ is a non-zerodivisor on $M/(f_1M)$, and so on. 
The maximal length of such a sequence is called the \emph{depth} of $M$, or more properly the depth of $(x_0,\dots, x_r)$ on $M$.
The length of any all maximal $M$-regular sequences are the same, as one shows by proving

\begin{theorem} (Auslander-Buchsbaum)\label{Auslander-Buchsbaum}
If $M$ is a finitely generated $S := \CC[x_0, \dots x_r]$-module, then the length of every $M$-regular sequence is
the smallest integer $m$ such that $\Ext_S^m(S/(x_0, \dots x_r), M) \neq 0$ and is also $r+1 - \pd\  M$.
\end{theorem}
 
 The depth of a module $M$ is bounded above by $\dim M$, the Krull dimension. The reason is that if the dimension of $M$
 is $d$, and $f_1 \in (x_0, \dots x_r) $ is a non-zerodivisor on $M$, then $\dim M/(f_1)M= \dim M-1$. Thus by induction, if
  $f_1,\dots, f_d$ is $M$-regular then $M/(f_1, \dots, f_d)M$ has dimension 0, which is equivalent to its being Artinian. Thus any 
$ f_{d+1} \in(x_0, \dots x_r) $ acts as a nilpotent endomorphism of $M/(f_1, \dots, f_d)M$.

It follows from these facts that the depth of an $S$-module $M$ is equal to the dimension of $M$ if and only if the projective dimension
of $M$ is equal to the codimension of $M$; in this case we say that $M$ is a
\emph{Cohen-Macaulay module}. 

As we showed in Chapter~\ref{linkageChapter}, a curve $C\subset \PP^3$ is linked to a complete intersection
if and only if  $H^1_*(\sI_C) := \oplus_{m\in \ZZ} H^1(\sI_C(m)) = 0$, in which case $C$ is said to be is arithmetically Cohen-Macaulay.
From the Auslander-Buchsbaum theorem and Theorem~\ref{ACM basics} we see that $C$ is arithmetically Cohen-Macaulay if
and only if the homogeneous coordinate ring $R_C$ is Cohen-Macaulay, and this is true
if and only if $\pd\  R_C = \codim C$.


\subsection{The Gorenstein property} 
Another important homological condition is the condition that $\omega_X$ is an invertible sheaf; when this holds, we say that $X$ is \emph{quasi-Gorenstein}. When, in addition, $X$ is Cohen-Macaulay we say that $X$ is \emph{Gorenstein}. Any scheme that is locally a complete intersection, such as any smooth scheme, is Gorenstein. Since the restriction
of $\sO_{\PP^r}(1)$ to a subvariety is always invertible, saying that a scheme $X$ is canonically embedded implies that
$X$ is at least quasi-Gorenstein. As with the Cohen-Macaulay property, the Gorenstein property is interpreted locally
on a scheme. We say that a projective scheme $X$ is \emph{arithmetically Gorenstein}
if its homogeneous coordinate ring is Gorenstein, and it follows that $\omega_{S/I} \cong S/I(a)$ for some integer $a = a(X)$.

In Chapter~\ref{LinkageChapter},  we expressed $\omega_X$
for a subscheme $X\subset Y$ of a scheme $Y$ as
 $\Ext^{\codim X}_{\sO_Y}(\sO_X, \omega_Y)$. Slightly extending this idea, if $C\subset \PP^r$ is a curve
with homogeneous coordinate ring $R_{C}$,
 we define $\omega_{R_C}$ to be $\Ext^{r-1}_S(R_C, S(-r-1))$, where $S$ is the homogeneous coordinate ring of $\PP^r$.
Since $\omega_{\PP^r} = \sO_{\PP^r}(-r-1)$, the sheafification of this module is  $\omega_C$.

If $C$ is arithmetically Cohen-Macaulay, so that
$\pd (C) = \codim(C) = r-1$,  then 
computing $\Ext^{r-1}_S(R_C, S(-r-1))$ from the minimal free resolution 
$$
(\FF, \phi):\quad 0\lTo R_C \lTo S\lTo^{\phi_1} F_1 \lTo \cdots \lTo F_{r-2} \lTo^{\phi_{r-1}} F_r\lTo 0
$$
we see that $\omega_{R_C} = \coker \phi_{r-1}^*$. Theorem~\ref{WMACE} implies that the complex $(\FF^*, \phi^*)$ which is the dual
of the  resolution $(\FF, \phi)$ is again acyclic, so it is the minimal free resolution of $\omega_{R_C}$. Thus 
$\omega_{R_C}$ is a Cohen-Macaulay module. Just as the Cohen-Macaulay property of
$R_C$ implies that $R_C = H^0_*(\sO_C)$, it follows that $\omega_{R_C} = H^0_*(\omega_C)$.

If, in addition, $C$ is a canonical curve, so that $\omega_C = \sO_C(1)$, then we derive:
$$
\omega_{R_C} = \coker(\phi_{r-2}^*)(-r-1) = R_C(1)
$$
so $\coker \phi_{r-2}^* = R_C(r)$. Thus $(\FF^*(-r), \phi^*)$ is a minimal free resolution of $R_C$, and is therefore isomorphic to
$(\FF^*, \phi^*)$; that is, $(\FF^*, \phi^*)$ is self-dual.
 We have seen an example already
in the Koszul complex (a complete intersection is arithmetically Gorenstein).

Taking into account that in a resolution $(\FF, \phi)$ each summand $S(-j)$ of $F_{i+1}$ can only
map to summands $S(-l)$ of $F_i$ with $\ell < j$, and similarly for the dual, we see that the 
Betti table of the minimal free resolution of a canonical curve of genus $g$ must have the form:


% \centerline{\scriptsize
%\begin{tabular}{r|cccccc} 
%$j\backslash i$&0&1&2&$\cdots$&$r-2$&$r-1$\\ 
%\hline 
%0&1&$-$&$-$&$\cdots$&$-$&$-$\\ 
%1&$-$&$\binom{g-2}{2}$&$?$&$\cdots$&$?$&$-$\\
%2&$-$&?&$?$&$\cdots$&$\binom{g-2}{2}$&$-$\\
%3&$-$&$-$&$-$&$\cdots$&$-$&$1$\\
%\end{tabular}.}
%
%\centerline{\scriptsize
%\begin{tabular}{r|cccccc} 
%$j\backslash i$&0&1&2&$\cdots$&$r-2$&$r-1$\\ 
%\hline 
%0&1&$-$&$-$&$\cdots$&$-$&$-$\\ 
%1&$-$&$\binom{g-2}{2}$&$b_2$&$\cdots$&$b_{r-2}$&$-$\\
%2&$-$&$b_{r-2}$&$b_{r-3}$&$\cdots$&$\binom{g-2}{2}$&$-$\\
%3&$-$&$-$&$-$&$\cdots$&$-$&$1$\\
%\end{tabular}.}

\centerline{\scriptsize
\begin{tabular}{r|cccccc} 
$j\backslash i$&0&1&2&$\cdots$&$g-3$&$g-2$\\ 
\hline 
0&1&$-$&$-$&$\cdots$&$-$&$-$\\ 
1&$-$&$b_1$&$b_2$&$\cdots$&$b_{g-3}$&$-$\\
2&$-$&$b_{g-3}$&$b_{g-4}$&$\cdots$&$b_1$&$-$\\
3&$-$&$-$&$-$&$\cdots$&$-$&$1$\\
\end{tabular}.}
\noindent where  $-$ represents 0.  It turns out that the $b_i$ depend on the particular canonical curve,
but since the Hilbert function of the curve is the alternating sum of the Hilbert functions in the resolution,
the differences $b_i- b_{g-i-1}$ are independent of the curve.

To go further, we can make use of the invertible sheaves $\sL$ on $C$ with $h^0(\sL)=2$ and $h^1(\sL)\geq 2$. 
The existence of such a series guarantees that the ideal of $C$ contains the ideal of $2\times 2$ minors of the $2\times h^1(\sL)$
matrix
corresponding to the multiplication map 
$H^0(\sL) \otimes H^0(\sL^{-1}\otimes \omega_C) \to H^0(\omega_C) = H^0(\sO_C(1))$
as in Chapter~\ref{ScrollsChapter}.
The mechanism is a resolution of
the ideal generated by the minors of this matrix.
 
\section{The Eagon-Northcott complex}\label{EN section}

The Eagon-Northcott complex $EN(\phi)$~\cite{MR0142592} associated with a matrix, or a map of free modules $\phi: F\to G$,
is a generalization of the Koszul complex, which is the case $\rank G = 1$. Like the Koszul complex,
it is tautological: its existence depends only on the properties of commutative rings; and like the Koszul complex it is exact or not depending on a property of the matrix $\phi$ related to regular sequences. It is part of a family of complexes described in
\cite[Appendix A2]{Eisenbud1995}, and, from a more conceptual and general point of view, in \cite{Weyman-book}. 

We
are interested in $EN(\phi)$ because its shape
influences the shape of the free resolutions of canonical curves in an interesting way, 
leading to Green's conjecture. This conjecture, one of the central open problems in the theory of algebraic curves, is described in the last
section of this chapter. We will also use the the Eagon-Northcott complex, in a special case, to give a proof of the classification of matrix pencils and an analysis of the ideals of ACM curves in $\PP^{3}$.

To prepare for the description of the Eagon-Northcott complex we revisit the Koszul complex, which is equal
to the Eagon-Northcott complex in the case $\rank G = 1$, and then consider the case $\rank F = \rank G + 1$.

\subsubsection{$\mathbf {\bf rank\ }  \mathbf{G = 1}$}

In this subsection we let $\phi:F = R^{f}\to R$ be a homomorphism from a free module to a ring $R$, and we
we will analyze the Koszul complex $K(\phi)$. 

We may write
$K(\phi) = EN(\phi)$ in the form
$$
S \lTo^{\delta_{1}} F \lTo^{\delta_{2}} \wedge^{2}F \lTo^{\delta_{3}} \cdots \lTo^{\delta_{f}} \wedge^{f}F \lTo 0
$$
where $\delta_{1} = \phi$.

To define the complex, we must construct the differentials $\delta_{i}$ and prove that
$\delta_{i}\delta_{i+1} = 0$. Since the modules are free, it suffices to do this for the 
dual maps 
$$
\partial_{i}: \wedge^{i}F^{*} \to \wedge^{i+1}F^{*},
$$
and it turns out that this is in a sense even more natural. 

It is convenient to think of $R$ as an $S := \ZZ[x_{1},\dots, z_{f}]$-algebra by the map sending 
$x_{i}$ to $\phi_{i}$; we  define the Koszul complex of $\phi$ over $R$ by tensoring
the Koszul complex of $(x_{1}, \dots, x_{f})$ with $R$.

Thus for the definition we take the map $\phi$ to be 
$$
\phi: S^{f}\rTo^{
\begin{pmatrix}
x_{0} &\dots & x_{f}
\end{pmatrix}
} S.
$$

First of all, the map $\partial_{i}$ (like the map $\delta_{i}$) is \emph{linear}: the image of a basis vector of $\wedge^{i}F^{*} $ is a sum of variables times basis vectors
of $\wedge^{i+1}F^{*}$. We may write $S$ as $\Sym(V)$, where $V$ is the free $\ZZ$-module generated by $x_{1}, \dots, x_{f}$, and we may think of $F$ as the module $V\otimes S$ with the map
$\phi$ sending $V\otimes 1\subset F$ by the identity to $V = S_{1}\subset S$---the ``tautological map''. 
Let $t\in V\otimes V^{*}\subset S\otimes \wedge V^{*}$ be the ``trace element'' represented in terms of any basis $\{x_{i}\}$ of $V$
and dual basis $\{\hat e_{i}\}$ of $V^{*}$ as $t = \sum x_{i}\otimes \hat e_{i}$. Because $\wedge V^{*}$ is 
an anti-commutative algebra, we have $t^{2} = 0$.

We define the map 
$$
\partial_{i}: S\otimes_{\CC} \wedge^{i}V^{*} = \wedge^{i}F^{*}  \to \wedge^{i+1}F^{*} = S\otimes_{\CC} \wedge^{i+1}V^{*}
$$
to be multiplcation by $t$, and thus $\partial_{i+1}\partial_{i}$ is multiplication by $t^{2} = 0$.

Having defined the complex $K(\phi) = EN(\phi)$ in the case $\rank G = 1$, we next ask what conditions on $\phi$
make it acyclic (that is, a free resolution of $\coker (\delta_{1})$). 

\begin{theorem}\label{rankG1}
 Suppose that $R$ is a ring and $\phi: F\to R$ is a map from a free $R$-module of rank $f$.
 The complex $K(\phi)$ is acyclic if and only if the ideal $I := I_{1}(\phi)$ has grade $\geq f$.
 \end{theorem}

\begin{proof}
Theorem~\ref{WMACE} immediately implies that if $K(\phi)$ is acyclic then
$\rank \delta_{f} = 1$ and $\grade I(\delta_{f}) \geq f$. Since $I(\delta_{f}) = I$, this proves one implication.

Now assume that $\grade I \geq f$. We first prove that 
$K(\phi)$ is split exact when $I = R$. The condition $I=R$ implies that
 $\partial_{1}$ is a split injection. In this case we may write $F^{*} = R\oplus F'^{*}$ in such a way that $\partial_{1}$ is the injection into the first summand, and we may
choose a basis $\{\hat e_{i}\}$ of $F^{*}$ so that the last $f-1$ basis elements are a basis for $F'^{*}$. 
Specializing the sequence $x_{1},x_{2}, \dots, x_{f}$ to the sequence $1, 0,\dots, 0$, the differential of $K(\phi)^{*}$
becomes the multiplication by  $1\otimes e_{1}$.

The module
$\wedge^{i}F$  decomposes as 
$$
\wedge^{i}F = \bigl(Re_{1}\otimes_{R} \wedge^{i-1}F^{*} \bigr) \oplus \wedge^{i}F'^{*}.
$$
Because $e_{1}\wedge e_{1} = 0$ the differential $\partial_{i}$ has the form
$$
\begin{diagram}
&&Re_{1}\otimes \wedge^{i-2}F'^{*} &\rTo^{0}&  Re_{1}\otimes \wedge^{i-1}F'^{*}\\
\wedge^{i-1}F^{*}&=& \bigoplus&\ruTo^{\cong}&\bigoplus&=&\wedge^{i}F^{*}\\
 &&\wedge^{i-1}F'^{*}&\rTo^{0}& \wedge^{i}F'^{*}
\end{diagram}.
$$
Thus we see that $K(\phi)$ is split exact when $\phi$ is a split surjection.

We now assume only that $\grade I\geq f$. From what we just proved we see that if we localize
$R$ by inverting any element of $I$ the complex $K(\phi)$ becomes split exact. Since $\grade f\geq 1$,
we can find such  a non-zerodivisor in $I$, and inverting it does not change the ranks of the 
maps $\phi_{i}$. Because ranks of free modules are additive in direct sums, it is obvious that
in the split exact case the condition on the ranks of the $\phi_{i}$ is satisfied; more precisely,
$\rank(\delta_{i}) = \binom{i-1}{f-1}$. We also see that after inverting
a non-zerodivisor in $I(\delta_{1})$ we have $I(\delta_{i}) = R$; equivalently, 
$$
I  \subset \sqrt {I(\delta_{i})}.
$$
(In fact $I(\delta_{i}) = I^{\binom{i-1}{f-1}}$, though this requires a separate argument.) Thus if $\grade I = f$ then  $\grade I(\delta_{i}) \geq i$ for all $i$, so $K(\phi)$ is acyclic.
\end{proof}


\subsubsection{ $\mathbf{{\bf rank\ } F = {\bf rank\ }G + 1}$}

We set $g=\rank G$ and $f = \rank F = g+1$. In this case the Eagon-Northcott complex has the form:
$$
EN(\phi):\quad 0\rTo G^{*}\otimes \wedge^{f}F \rTo^{\delta_{2}} \wedge^{f-1}F \rTo^{\delta_{1}} \wedge^gG.
$$
Here $\delta_{1} = \wedge^{g}\phi$, so that the entries of a matrix for $\delta_{1}$ are the $g\times g$ minors of 
$\phi$.

We choose an identification $\wedge^{f}F = S$, called an \emph{orientation} of $F$, and get a perfect pairing 
$$
\wedge^{g}F \times F \to \wedge^{f}F = S
$$
so that we may identify
 $\wedge^{g}F$ with $F^{*}$. With this identification, we define $\delta_{2}$ as
 $$
\delta_{2}:  G^{*}\rTo^{\phi^{*}}  F^{*} = \wedge^{g}F.
 $$
We also choose an orientation $\wedge^{g}G = S$, in terms of which the image of $\delta_{1}$ is
 the ideal generated by the $(f-1)\times (f-1)$  minors of $\phi$.
 
We first claim that $EN(\phi)$ is a complex; that is,  $\delta_{1}\delta_{2} = 0$.  As with the Koszul complex,
it is convenient to dualize and consider the maps 
$$
EN(\phi)^{*}:\quad 0\rTo \wedge^{g} G \to \wedge^{g}F^{*} = F \rTo^{\phi} G
$$
The fact that this composition is 0 is often taught as Cramer's rule for solving a system of
homogeneous equations represented by a $g\times (g+1)$ matrix of rank $g$.
The solutions---that is, the elements of $\ker \phi$---are multiples of the column
$\Delta_{1}, \dots, \Delta_{g}$ where the $\Delta_{j}$ is $-1^{j}$ times the determinant
of the matrix obtained from $\phi$ by leaving out the $j$-th column. This works
because the composition of the two maps is a column matrix whose $i$-th entry is the
expansion of the $(g+1)\times (g+1)$ determinant of the matrix obtained from $\phi$ by
repeating the $i$-th row.

\begin{theorem}\label{EN grade 2}
 Suppose that $R$ is a ring and $\phi: F\to G$ is a map of  free $R$-modules, 
 where $G$ has rank $g$ and $F$ has rank $f = g+1$.
 The complex $EN(\phi)$ is acyclic if and only if the ideal $I := I_{g}(\phi)$ has grade $\geq 2$.
 \end{theorem}

%\begin{theorem}
%If $\phi$ is a $g\times (g+1)$ homogeneous matrix of forms of positive degree in $S$ then the Eagon-Northcott
%complex $EN(\phi)$ is acyclic if and only if $\grade I_{g}(\phi) \geq 2$.
%\end{theorem}

\begin{example}
We have seen that the ideal of the twisted cubic is generated by the $2\times 2$ minors of the matrix
$$
\phi :=  
\begin{pmatrix}
 x_{0}&x_{1}&x_{2}\\
 x_{1}&x_{2}&x_{3}
\end{pmatrix}
$$
and it follows that the free resolution of its homogeneous coordinate ring is the Eagon-Northcott complex
$$
0\rTo S^{2}(-3) 
\rTo^{\begin{pmatrix}
 x_{0}&x_{1}\\
  x_{1}&x_{2}\\
x_{2}&x_{3}
\end{pmatrix}
}
S^{3}(-2)
\rTo^{\wedge^{2}\phi}
S
$$
\end{example}

\begin{example}\label{res of max ideal power}
In Section~\ref{Kronecker} we asserted that  the $(a+1)\times a$  matrix
$$
\phi_{a} :=\begin{pmatrix}
s&0&0&\dots&0\\
-t&s&0&\dots&0\\
0&-t&s&\dots&0\\
0&0&-t&\dots&0\\
\vdots&&\cdots&&\vdots\\
0&&\dots&\ddots&s\\
0&&\dots&\dots&-t\\
\end{pmatrix}
$$
is the minimal presentation of the ideal $(s^{a},s^{a-1}t, \dots, t^{a}) \subset R:=\CC[s,t]$. It is not hard to check this
directly, but in any case it's easy to see that its $a\times a$ minors of $\phi_{a}$ generate this ideal, which has grade 2, so the Eagon-Northcott resolution $EN(\phi)$
has the form
$$
R\lTo^{\wedge^{a}\phi_{a}} R^{a+1} \lTo^{ \phi_{a}} R^{a} \lTo 0
$$
verifying the assertion.
\end{example}

\begin{proof}
Once having shown that $EN(\phi)$ is a complex, as we did above,  the proof of the equivalence in the theorem follows the same pattern as the
proof given above for the Koszul complex.

If $EN(\phi)$ is acyclic, then by Theorem~\ref{WMACE} the $g\times g$ minors of $\phi = \delta_{2}$ must
have grade $\geq 2$.
For the converse, suppose first that
$I_{g}(\phi)$ is the
unit ideal. We may split  $F$ as  $S\oplus G$ with $\Delta_{1} = 1$ and $\Delta_{j} = 0$
for $j>1$, and then $EN(\phi)^{*}$ has the form:
$$
\begin{diagram}
&&0&\rTo&  G&\rTo^{0}&S\\
0&\rTo& \bigoplus&\ruTo^{\cong}&\bigoplus&\ruTo^{\cong}&\bigoplus\\
 &&G&\rTo^{0}&S&\rTo&0
\end{diagram}.
$$
Thus we see that $EN(\phi)$ is split exact in this case.

We now apply Theorem~\ref{WMACE}: From what we just proved we see that if we localize
$S$ by inverting any element of $I$ then the complex $EN(\phi)$ becomes split exact,
and therefore, before localizing,
$\rank(\delta_{1}) = 1$ and $\rank \delta_{2} = g$ . In this
case it follows from the definition that $I_{1}(\delta_{1}) = I_{g}(\phi) = I_{g}(\delta_{2})$
so if $I_{g}(\phi)$ has grade 2 then both conditions of Theorem~\ref{WMACE} are
satisfied.
\end{proof}

\subsection{The Hilbert-Burch theorem}

In a regular local ring any ideal of codimension 1 is principal (divisors are all Cartier). What about
ideals of codimension 2? The answer is the content of the \emph{Hilbert-Burch} theorem, proven
in 1890 by David Hilbert in the case of homogeneous ideals in $\CC[x_{0},x_{1}]$ and in general by
Lindsay Burch~\cite{MR212008}. We can deduce it as an application of the Eagon-Northcott complex
in the case $f =g+1$:

\begin{corollary}[Hilbert-Burch theorem]\label{Hilbert-Burch}
Suppose that $R$ is a local ring. Any ideal $I\subset R$ of projective dimension 1 and grade 2 has the form
$aI'$ where $I'$ is an ideal of grade 2 generated by the $g\times g$ minors
of a $g \times (g+1)$ matrix and $a$ is a non-zerodivisor of $R$; and conversely any ideal of this form
has projective dimension 1. 

In particular, if $C\subset \PP^{3}$ is an ACM curve whose homogeneous ideal $I$ is generated by
$f$ elements, then $I$ is minimally generated by the $(f-1)\times (f-1)$ minors of the syzygy matrix of $I$.
\end{corollary}

\begin{proof}
If $C\subset \PP^{3}$ is ACM, then the projective dimension of the homogeneous coordinate ring $R_{C}$
is 2  by the Auslander-Buchsbaum theorem, and thus the ideal of $C$ has projective dimension 1.

Now suppose that $I\subset R$ is an ideal of projective dimension 1 in any local ring, and suppose
that $I$ is generated by $f$ elements, so that we have a surjection $F:= R^{f} \to I$.  The module $R/I$
has free resolution of the form
$$
\FF: R\lTo^{\alpha} R^{f}\lTo^{\phi} G\lTo 0
$$
where $I = I_{1(\alpha})$, so by Theorem~\ref{WMACE} the free module $G$ has rank $g = f-1$, the $g\times g$
minors of $\phi$ generate an ideal $I'$ of grade $\geq 2$, and the ideal $I$ has grade $\geq 1$. Theorem~\ref{WMACE} implies
that both the Eagon-Northcott complex $EN(\phi)$
and its dual are acyclic. 

The dual of the complex $\FF$ will not be acyclic unless $\grade I = 2$, but there is at least a comparison map
$$
\begin{diagram}
\FF^{*}:&&G^{*}& \lTo^{\phi^{*}} & G^{*}&\lTo^{\alpha^{*}} &R&\lTo &0\\
&&\dTo_{=}&&\dTo_{=}&&\dTo_{a}\\
EN(\phi)^{*}:&& G^{*}&\lTo^{\phi^{*}}&F^{*}&\lTo^{\wedge^{g}\phi^{*}} &R&\lTo &0\\
\end{diagram}
$$
It follows that $I = aI'$, and since $I$ has grade 1, $a$ must be a non-zerodivisor.

Conversely, if $\phi: R^{f}\to R^{g}$ is a map with $f = g+1$ and $\grade I_{g}(\phi)\geq 2$,
then the acyclicity of $EN(\phi)$ shows that $I_{g}(\phi)$ has projective dimension 1; and if
$a$ is a non-zerodivisor, then $I := aI_{g}(\phi) \cong I_{g}(\phi)$ as $R$-modules, so
$I$ has projective dimension 1 as well.
\end{proof}

This argument applies, for example to the case of a non-hyperelliptic curve of genus 3 and 
degree 6 in $\PP^{3}$, discussed in Section~\ref{other genus 3}.


\subsection{The general case of the Eagon-Northcott complex}
With these two special cases in mind, we are ready for the general case. First the definition:
If $\phi: F\to G$ is a map of free $S$-modules with $f:=\rank F\geq  g:= \rank G$ then there is
a complex of free $S$-modules
\begin{align*}
EN(M) := 
S \lTo{\delta_{1}:= \bigwedge^g \phi} 
 \bigwedge^g F&
 \lTo^{\delta_{2}}
 G^*\otimes \bigwedge^{g+1} F  \lTo^{\delta_{3}}
  (\Sym^2G)^*\otimes\bigwedge^{g+2}F  \\
 &\lTo^{\delta_{4}}\cdots\lTo^{\delta_{f-g+1}} 
(\Sym^{f-g}G)^*\otimes\bigwedge^fF 
 \lTo 0
\end{align*}
where:
\begin{enumerate}
 
\item After identifying $\wedge^{g}G$ with $S$, the map $\delta_{1}$ is identified with $\wedge^{g}\phi$.

\item It is convenient to give a formula for $\partial_{i} = \delta_{i}^{*}$ by taking advantage of the 
algebra structures of $\Sym(G)$ and $\wedge F^{*}$. To do this, choose dual bases $\{e_{i}\}$ and $\{\hat e_{i}\}$ for $F$ and $F^{*}$. In these terms
$$
\delta_{i^{*}} = \partial_{i}: 
\Sym^{i-2}G \otimes \wedge^{g+i-2}F^{*} \to 
\Sym^{i-1}G \otimes \wedge^{g+i-1}F^{*}
$$
 is multiplication by the element
$\sum_{i = 1}^{f} \phi(e_{i}) \otimes \hat e_{i}$.
\end{enumerate}

To show that $\delta_{1}\delta_{2} = 0$ is almost the same as in the case $f = g+1$ because
a basis element 
$$
b: = e_{i_{1}}\wedge \cdots \wedge e_{i_{g+1}} \in \wedge^{g+1}F
$$
can be thought of as coming from a rank $g+1$ summand of $F$, and the value of $\delta_{2}\delta_{1}b$
is the same as it would be if $F$ were replaced by this summand. Thus from the case $f=g+1$ we see
that $\delta_{1}\delta_{2}b = 0$, and thus $\delta_{2}\delta_{1} = 0$.

On the other hand, for $i\geq 1$ the map $\delta_{i}\delta_{i+1}$ is multiplication by
$$
\left(\sum_{i = 1}^{f} \phi(e_{i}) \otimes \hat e_{i}\right)^{2},
$$
which we may think of as the square of an element of degree 1 in the exterior algebra
of the free $\Sym(G)$-module $\wedge(\Sym(G)\otimes F)$, and hence this square is 0.
Thus the given maps do define a complex.

\begin{theorem}\label{ENgeneral}
 Suppose that $S$ is a ring and $\phi: F\to G$ is a map of  free $R$-modules, 
 where $G$ has rank $g$ and $F$ has rank $f \geq g$.
 The complex $EN(\phi)$ is acyclic if and only if the ideal $I := I_{g}(\phi)$ has grade $\geq f-g+1$.
 \end{theorem}

\begin{example}
If $\phi$ is a matrix of linear forms, then the first map of $EN(\phi)$  is represented by the
row of $g\times g$ minors of $\phi$, which are forms of degree $g$, but all the rest of the maps
are represented by matrices of linear forms. Thus, for example, the Betti table of the Eagon-Northcott complex of 
a $2\times f$ matrix of linear forms is:

\

\centerline{\scriptsize
\begin{tabular}{r|ccccc} 
$j\backslash i$&0&1&2&3&$f-1$\\ 
\hline 
0&1&$-$&$-$&$\cdots$&$-$\\ 
1&$-$&$\binom{f}{2}$&$2\binom{f}{3}$&$\cdots$&$(f-1)\binom{f}{f}$\\ 
\end{tabular}}
\end{example}

\begin{proof}
The dual of the last differential of $EN(\phi)$ is
$$
\partial_{f-g+1}: \Sym^{f-g-1}G \otimes \wedge^{f-1}F^{*} \to \Sym^{f-g}(G) \otimes \wedge^{f}F^{*}.
$$ 
With our usual identifications $\wedge^{f}F^{*} = S$ and $\wedge^{f-1}F^{*} = F$ this becomes the map
$$
\Sym^{f-g-1}G \otimes F \rTo^{1\cdot \phi} \Sym^{f-g}G
$$
whose cokernel is $\Sym^{f-g}(\coker \phi)$ by the right exactness of the symmetric algebra functor~\cite[Proposition A2.2]{Eisenbud1995}. The support of $\Sym^{f-g}(\coker \phi)$ is obviously contained in the support
of $\coker \phi$, but in fact they are equal: if a localization of $\coker \phi$, over a local ring $S_{P}$,
 is nonzero, then by
Nakayama's lemma it surjects onto $S_{P}/P_{P} = \kappa(P)$, and again by the right exactness
of the symmetric algebra functor $\Sym^{f-g}(\coker \phi)_{P}$ surjects onto $\Sym^{f-g}(\kappa(P)) = \kappa(P)$.

By Theorem~\ref{WMACE} we see from this that if $EN(\phi)$ is acyclic, then the support of $\coker \phi$
has grade $\geq f-g+1$. This support is defined by the radical of $I_{g}(\phi)$, so $\codim I_{g}(\phi)\geq f-g+1$ as required.

Conversely, to show that $EN(\phi)$ is acyclic under the given hypothesis we first treat the case
$I_{g}(\phi) = S$, and prove that $EN(\phi)^{*}$ is split exact. This is the most complicated part of the proof,
but it is purely formal:

As before we may split $F$ and
assume that $F = G\oplus F'$, the map $\phi$ being the projection onto the first summand.
Assuming that the first summand corresponds to the basis elements $e_{1}, \dots, e_{g}\in G\subset F$
the dual differential $\partial_{i}$ takes the form $\sum_{i=1}^{g} e_{i}\otimes \hat e_{i}$.

The map 
$$
\wedge^{g}G\lTo^{\delta_{1}} 
\wedge^{g}F =\bigoplus_{j=0}^{g}\wedge^{j}G\otimes \wedge^{g-j}F'
$$ 
is the projection onto the $j=g$ summand, so we must show that the rest of $EN(\phi)$ is a split surjection
ending with the terms of the source of $\delta_{1}$ other than $\wedge^{g}G$.

Once again, it will be convenient to treat the dual complex. Using the splitting
%and the identification of $\wedge^{g-j}G^{*}$ with $\wedge^{j}G$ 
we may write the terms of the dual as
$$
EN_{i}(\phi)^{*} = \Sym^{i-2}G \otimes  \wedge^{g+i-2}F^{*}  = 
\bigoplus_{j} \Sym^{i-2}G \otimes  \wedge^{j}G^{*} \otimes \wedge^{g+i-2-j}F'^{*}
$$
for $i\geq 1$.
The map $\partial_{i}= \delta_{i}^{*}$ is a direct sum from $j=0$ to $g$ of the maps 
$$
\Sym^{i-2}G \otimes  \wedge^{j}G^{*} \otimes \wedge^{g+i-2-j}F'^{*}
\rTo
\Sym^{i-1}G \otimes  \wedge^{j+1}G^{*} \otimes \wedge^{g+i-2-j}F'^{*}
$$
that are all equal to  multiplication by $\sum_{k=1}^{g} e_{k}\otimes \hat e_{k}\otimes 1$
where the last tensor factor is the identity map of $\wedge^{g+i-2-j}F'^{*}$.

Thus it suffices to show that the complexes
$$
\begin{aligned}
 (*_{j}) \quad \Sym^{0}G\otimes\wedge^{j}G^{*}\to\cdots \to \Sym^{i}G \otimes  \wedge^{i+j}G^{*}  \to \cdots
\end{aligned}
$$
are split exact for $0\leq j<g$.

Let $R = \Sym G = S[e_{1}, \dots, e_{g}]$. The Koszul complex over $R$ of the sequence $\phi_{i} = e_{i}$
may be written as
$$
R\otimes_{S} \wedge G^* = \bigoplus_{p,q}\Sym^{p}G\otimes\wedge^{q}G^*
$$
and we have proven in Theorem~\ref{rankG1} that it is a free resolution of $R/(e_1, \dots, e_g)=S$, which appears
as $\Sym^{0}G\otimes \wedge^{0}G^*$. The $R$-dual of this complex has terms
$(\Sym G)^*\otimes\wedge^{j}G$
and the complexes $(*_{j})$ above are summands as $S$-modules. It follows that these finite
complexes have no homology at all, and since the modules are free over $S$, they are split exact.

Since the 
complex $EN(\phi)$ is split exact after inverting any element of $I = (\phi_{1}, \dots, \phi_{f})$, it follows that 
the rank condition of Theorem~\ref{WMACE} is satisfied, and
$$
I \subset \sqrt{I_{\rank \delta_{i}}(\delta_{i})}. 
$$
Since the length of $EN(\phi)$  is $f-g+1$, Theorem~\ref{WMACE} implies that
it is acyclic when $\grade I\geq f-g+1$, completing the proof.
\end{proof}

\begin{corollary}\label{E-N cor}
With notation as in Theorem~\ref{ENgeneral}, if the ideal $I_g(\phi)$ has codimension $\geq f-g+1$ then it has
codimension exactly $f-g+1$, the ring $S/I_g(M)$ is Cohen-Macaulay, and the $\binom{f}{g}$ forms
that are the $g\times g$ minors of a matrix for $\phi$ are linearly independent over $\CC$.
\end{corollary}

\begin{proof}
From the resolution $EN(M)$ we see that the projective dimension of $S/I_2(M)$ is $n-1$. Since the projective dimension of a module
is at least the codimension of its annihilator, the equality follows, and the Auslander-Buchsbaum formula implies that $S/I_2(M)$ is 
Cohen-Macaulay. The linear independence of the minors of $M$ follows because $EN(M)$ is a resolution and there
$EN(M)_2$ is generated in degree 3, so all the relations on the minors have coefficients of degree 1.
\end{proof}


In general when $X\subset Y\subset \PP^r$, so that $I_X \supset I_Y$, it may be hard to see which syzygies of $X$ come
from syzygies of $Y$. But when the degrees of the syzygies of $Y$ are smaller than those from $X$, the situation is simpler.
Here is the special case we will use:

\begin{proposition}
Suppose that $C\subset \PP^r$ is a nondegenerate curve. If $C\subset X \subset \PP^r$, where $X$ is a rational
normal scroll, then the Eagon-Northcott complex that is  the minimal free resolution of $I_X$ is termwise a direct summand
of the minimal free resolution of $I_C$. Thus the Betti table of the resolution of $I_C$ is termwise $\geq$ that of $I_X$.
\end{proposition}

\begin{proof}
Let $EN$ be the minimal resolution of $I_X$, and let $\FF$ be the minimal resolution of $I_C$.
The inclusion $I_X \subset I_C$ induces a map $\phi: EN\to \FF$, unique up to homotopy. Since the minimal generators of $I_X$ are quadratic, and $I_C$ contains no linear forms, $\phi_0: EN_0\to \FF_0$ is a split monomorphism.

By induction, we may assume that $\phi_{i-1}$ is a split monomorphism. The free module $EN_{i}$ is generated in 
degree $i+1$, while the free module $\FF_i$ is generated in degrees $\geq i+1$. It follows that the relations
represented by $EN_i$, extended by 0, are among the minimal generators of the relations represented by $\FF_i$,
completing the proof.
 \end{proof}
 
\section{Green's Conjecture}

Corollary~\ref{canonical hilbert function} implies that the dimension of the vector space of forms of degree $d$
vanishing on a canonical curve is independent of the curve; for example, for $d=2$ we get
$
\dim ({I_{C}})_{2} = {g-2\choose 2}.
$
The Hilbert function of $I_C$ is determined by the Betti table of its resolution, so that the table generally has more information.

 For example
when $C$ is trigonal then by the geometric Riemann-Roch theorem, $C$ has a 1-dimensional family of trisecant lines, and any quadric containing $C$ must contain all these. As we have seen in Chapter~\ref{ScrollsChapter}, these lines sweep
out the 2-dimensional rational normal scroll defined by the 1-generic $2\times (g-2)$ matrix $M$ corresponding to the decomposition of $\sO_C(1)$
into a tensor product of the  line bundle $\sL$ associated to the $g^{1}_{3}$ and the residual line bundle $\omega_{C}\otimes \sL^{-1}$. The latter has $g-2$ sections, and we see from Section 16.2
that the scroll itself lies on the ${g-2\choose 2}$ quadrics defined by the minors of $M$. The exactness of the Eagon-Northcott complex associated to this matrix shows that there are no relations of degree 0 on these minors---that is, they are linearly independent over the ground field. It follows that they generate the vector space of all quadrics containing $C$. 



Furthermore, if $g = 6$ and $C$ is isomorphic to a plane quintic curve, then the canonical series of the plane quintic is $5-3 = 2$ times the hyperplane series, and it follows that the canonical image of $C$ lies on the Veronese surface in $\PP^{5}$. Thus the Veronese surface is contained in (in fact, equal to) the intersection of the quadrics defined by the $2\times 2$ minors of a generic symmetric matrix, coming from the 
multiplication map 
$$
H^{0}(\sO_{\PP^{2}}(1))\otimes H^{0}(\sO_{\PP^{2}}(1)) \to H^{0}(\sO_{\PP^{2}}(2)) = H^{0}(\sO_{\PP^{5}}(1))
$$
and there are $6 = {g-2\choose 2}$ independent quadrics in this ideal. Again in this case, they cannot generate the ideal of the curve.

One might fear that this is the beginning of some long series of examples, but in fact it is not: 

\begin{theorem} [Petri]
The ideal of a canonical curve of genus $\geq 5$ is generated by the $\g-2\choose 2$-dimensional space of quadrics it contains unless the curve is either trigonal or isomorphic to a plane quintic; in the latter cases, the ideal of the curve is generated by quadrics and cubics.
\end{theorem}

For a modern treatment of Petri's theorem in this level of generality see \cite{Schreyer}; for a different treatment see \cite{Arbarello-Sernesi}.

The two exceptions can be described simultaneously by using the Clifford index:

\begin{definition}
 The \emph{Clifford index} Cliff $\sL$ of a line bundle $\sL$ on a curve $C$ is $d-2r$, where $d := \deg \sL$ and $r :=  h^0(\sL)-1$. The Clifford index Cliff $C$ of
 a curve $C$ of genus $\geq 2$ is the minimum of the Clifford indices of special line bundles with at least 2 sections.
\end{definition}

Clifford's theorem (Corollaries \ref{Clifford bound} and ~\ref{equality in Clifford from Martens}) says that Cliff $C \geq 0$, and that Cliff $C = 0$ if and only if $C$ is hyperelliptic. If $C$ is not hyperelliptic, then it turns out that Cliff $C=1$ if and only if $C$ is either trigonal or isomorphic to a plane quintic. The Clifford index of any smooth curve of genus $g\geq 2$ is $\leq \lceil g/2\rceil+1$, with equality for a general curve, as one sees from the Brill-Noether Theorem~\ref{basic BN}, and for ``most'' curves the line bundle $\sL$ of maximal Clifford index has only 2 sections, though there is an infinite sequence of examples where this
``Clifford dimension'' is greater.

Moving to cubic forms, we see that $\dim ({I_C})_3 = {g+2\choose 3}-(5g-5)$. Comparing this number with the number of (possibly linearly dependent)
cubics obtained by multiplying $g$ linear forms and ${g-2\choose 2}$ quadrics, we see that the ideal of the curve has at least
$$
{g-2\choose 2} - {g+2\choose 3}-(5g-5) 
$$
independent syzygies of total degree 3 (that is, linear syzygies on the quadrics). For example when $g=4$ so that $C\subset \PP^3$ there is one quadric and 5 independent
cubics, at most 4 of which are multiples of the quadric. Since the curve has degree $6 = 2\times 3$, the ideal of the curve must be generated by
the quadric and one cubic. When $g=5$ there are genuinely two possibilities: the three quadrics in the ideal might be a complete intersection
(then they generate the ideal), so the Betti table would be

\

\centerline{\small %\scriptsize
\begin{tabular}{r|ccc} 
$j\backslash i$&0&1&2\\ 
\hline 
0&1&$-$&$-$\\ 
1&$-$&2&$-$\\ 
2&$-$&$-$&$1$\\ 
\end{tabular}}

\noindent or the curve could be trigonal, in which case the 3 quadrics generate the ideal of a surface scroll $F$. In the latter
case, the Eagon-Northcott complex resolves the homogeneous coordinate  ring  $S_F$ of the scroll,
$$
0\to S^2(-3) \to S^3(-2) \to S \to S_F \to 0
$$
which has Betti table

\centerline{\small %\scriptsize
\begin{tabular}{r|ccc} 
$j\backslash i$&0&1&2\\ 
\hline 
0&1&$-$&$-$\\ 
1&$-$&3&$2$\\ 
\end{tabular}}
\noindent and we see that there are 2 linear relations among the quadrics. Thus the minimal generators of $I_C$ must include exactly 2 cubics as well as the 3 quadrics. Since the homogeneous ring of a canonical curve is Gorenstein, its minimal free resolution is symmetric, and this is enough for us to fill in its Betti table:

\centerline{\small %\scriptsize
\begin{tabular}{r|cccc} 
$j\backslash i$&0&1&2&3\\ 
\hline 
0&1&$-$&$-$&$-$\\ 
1&$-$&3&$2$&$-$\\ 
2&$-$&$2$&$3$&$-$\\ 
3&$-$&$-$&$-$&$1$\\ 
\end{tabular}}
\noindent Note that we can ``see'' the scroll reflected in the top two lines of the table.

From the analogue of the Hilbert-Burch theorem for Gorenstein rings of codimension 3 one can show that the 5 generators can be written as the 
Pfaffians of a skew symmetric $5\times 5$ matrix whose entries are of degrees 1 and 2, in the following pattern (we give just the degrees, and put --- in the places that are 0):
$$
\begin{pmatrix}
 -&-&1&1&1\\
-&-&1&1&1\\
1&1&-&2&2\\
1&1&2&-&2\\
1&1&2&2&-
\end{pmatrix}
$$
Here the $2\times 2$ minors of the upper $2\times 3$ block of linear forms generate the ideal of the scroll. 

Applying this logic more generally we get the following result about the canonical embedding of curves with low degree maps to $\PP^1$:

\begin{theorem}
 Let $C\subset \PP^{g-1}$ be a reduced, irreducible canonical curve. If $C$ has a line bundle $\sL$ of degree $d \leq g-1$ with $h^0(\sL) = 2$  then
 there is a 1-generic  $2\times (g+1-d)$ matrix of linear forms whose minors define a scroll of codimension $g-d$ containing $C$; and thus an
 Eagon-Northcott complex of length $g-d$ is a subcomplex of the minimal free resolution of $R_C$. In particular, the Betti table of $R_C$ is
 termwise $\geq$ that of the homogeneous coordinate ring of the scroll.
 \end{theorem}
 
 (We have stated this theorem for canonical curves, but in fact the construction applies much more generally to a linearly normal variety $X \subset \PP^n$ of any dimension: if $X$ has a divisor $D$ that moves in a pencil and is contained in a subspace $\PP^k$ with $k \leq n-2$, the planes spanned by the divisors of the pencil $|D|$ sweep out a rational normal scroll.)
 
 Thus the existence of the $g^1_d$ on $C$, together with the symmetry of the resolution of the Gorenstein ring $R_C$,
  implies that the Betti table of $R_C$ has the form
 
 \centerline{\small %\scriptsize
\begin{tabular}{r|cccccccccccc} 
$j\backslash i$&0&1&2&\dots&$d-3$&$d-2$&\dots&$g-d-1$&$g-d$&\dots&$g-3$&$g-2$\\ 
\hline 
0&1&$-$&$-$&$\cdots$&$-$&$-$&$-$&$-$&$-$&$-$&$-$&$-$\\ 
1&$-$&*&*&$\cdots$&*&*&$\cdots$&*&$?$&$\cdots$&?&?\\ 
2&$-$&?&?&$\cdots$&?&*&$\cdots$&*&*&$\cdots$&*&*\\ 
3&$-$&$-$&$-$&$\cdots$&$-$&$-$&$-$&$-$&$-$&$-$&$-$&1
\end{tabular}}
\noindent where we have assumed for illustration that $d-2<g-d-1$. The places marked $-$ are definitely 0 and those marked * are definitely nonzero. The entries of the rows marked 0 and 1 are $\geq$ the corresponding entries of the Betti table of the scroll.

We can summarize this by saying
that if the curve $C$ has a line bundle $\sL$ of degree $d$ with exactly 2 sections (which is thus of Clifford index $c = d-2$) the row labeled  2 
in the Betti diagram definitely has $\beta_{c, c-2} \neq 0$. As with the case of the plane quintics above, one can
make a similar argument for \emph{any} line bundle of Clifford index $c$. Thus:

\begin{corollary}
 If Cliff $C \leq c$ then $\beta_{c,c+2}(S/I_C)) \neq 0.$
\end{corollary}
 
Starting from examples such as the case of genus 6, Mark Green made a bold conjecture that is still open as of this writing:

\begin{conjecture}[Green's Conjecture]
If $C$ is a smooth canonical curve of genus $g$ and $S/I_C$ is the homogeneous coordinate ring of $C$ in its canonical embedding,
then the Clifford index of $C$ is $\leq c$ if and only if $\beta_{c,c+2}(S/I_C) \neq 0$. 
\end{conjecture}

The conjecture was made for curves over a field of characteristic 0, and is known in many cases, though it is also known to fail in small finite characteristics (see~\cite{Bopp-Schreyer} for an amended conjecture that may hold in all characteristics.)
In \cite{MR1941089} and \cite{MR2157134} the conjecture was proven for generic curves of each Clifford index.  See~\cite{MR4022070}, \cite{MR4213770} and~\cite{arXiv:2205.00266}  for various simpler proofs.  As of this writing the full conjecture is known up to genus 9,  for plane curves, and in a number of other special cases.
See for example \cite{Farkas-progress-on-syzygies} for a survey on this and related topics.

\subsection{Low genus canonical embeddings} 
In his 1983 Brandeis thesis \cite{Schreyer-canonical}, Frank Schreyer analyzed all the possibilities for resolutions of smooth canonical curves up to genus 8. The project was carried further to genus 9 in a Saarbr\"ucken thesis \cite{Sagraloff}  of a student of Schreyer.


%\fix{M2 notation is
%different; and the hyperelliptic cases are included. Maybe retype in normal notation?}

%\includepdf[pages=1, scale=.8]{"SyzygiesGupto8"}
%\includegraphics[scale = .35]{"genus 9 Sagraloff"}
%\includepdf[scale=.8]{"SyzygiesGupto8"}
%\section{Low degree}

\section{Exercises}

\begin{exercise}\label{WMACE corollary}
 Let $S = k[x_0,\dots, x_r]$, and let
 $$ 
\FF:  F_0\lTo^{\phi_1}F_1 \lTo \cdots \lTo F_{n-1}\lTo^{\phi_n} F_n\lTo 0
 $$
be a finite complex of free $S$-modules. Set
$$
X_i = \{p\in \AA^{n+1} \mid  H_i(\FF \otimes \kappa(p)) \neq 0\}
$$
Use Theorem~\ref{WMACE} to prove that the complex $\FF$ is \emph{acyclic} (that is, $H_i(\FF) = 0$ for all $i>0$) if and only if
$$
\codim X_i \geq i
$$
for all $i>0$. Moreover, $X_{0}\supseteq X_{1}\supseteq \cdots \supseteq X_{n}$

Hint: Elementary linear algebra shows that, if $k$ is a field, then a complex $k^p \lTo^{\phi} k^q \lTo^\psi k^r$ is exact at $k^q$ if and
only if $\rank \phi +\rank \psi = q$. 
\end{exercise}

\begin{exercise}
Prove that if $X\subset \PP^r$ is arithmetically Cohen-Macaulay then the dual of the minimal free resolution of $S/I_X$
is the minimal free resolution of $\omega_X$. Hint:Use Theorem~\ref{WMACE} and the characterization of $\omega_{S/I_X}$
as an Ext module.
\end{exercise}

\begin{exercise}
The \emph{depth lemma} states that if 
$$
0\to A\to B\to C \to 0
$$
is an exact sequence of nonzero finitely generated modules over a local ring $R$ then
$$
\begin{aligned}
 \depth C &\geq \min\{\depth B, \depth A-1\}\\
 \depth A & \geq \min\{depth B, \depth C +1\}
 \end{aligned}
 $$
Prove this in the special case when $R$ is regular using the characterization of depth
 via projective dimension.
\end{exercise}

\begin{exercise}
\begin{enumerate}
 \item Prove that if $\phi: F\to G$ is a free presentation of a finitely generated module $M$
then 
$$
\ann_{R}(\coker \phi)^{\rank G}\subset Fitt_{0}(\phi) \subset \ann_{R}(\coker \phi).
$$
\item Prove that over a local ring every projective module is free; and show that 
$I(\phi)$ defines the non-free locus of $\coker \phi$. 
\end{enumerate}
\end{exercise}

\begin{exercise}
In dealing with arithmetically Gorenstein schemes, we used the fact that if $\omega_{S/I}$ is an invertible
sheaf (over $\Spec(S/I)$) then it is isomorphic to $S/I(a)$ for some $a$. Why is this true?

Hint: Nakayama's Lemma can be used to prove that projectives are free in some cases.
\end{exercise}


\begin{exercise}
Find a degree 6 embedding of a curve of genus 3 that is not arithmetically Cohen-Macaulay, and another that is.

Hint: Show that the $3\times 3$ minors of a general $4\times 3$matrix of linear forms defines a Cohen-Macaulay curve
of genus 3. Show that a curve of type $(2,4)$ on a smooth quadric in $\PP^3$ is not arithmetically Cohen-Macaulay.
\end{exercise}

\begin{exercise}
Referring to the Betti table of a canonical curve just before Section~\ref{EN section}, give a formula
for the differences $b_i- b_{g-i-1}$ that depends only on $i$ and $g$.
\end{exercise}

\begin{exercise}
 Show that if $I \subset S := \CC[x_0,\dots, x_r]$ is a codimension 2 ideal, then $S/I$ is Cohen-Macaulay if and only
 if the minimal $S$-free resolution of $S/I$ has the form
 $$
 0\rTo S^{n-1} \to S^n \to S
 $$
 for some $n$. Show that $S/I$ is Gorenstein if and only if $I$ is a complete intersection. Hint: for the first part, tensor with
 the field of rational functions. 
\end{exercise}

\begin{exercise}
Which sets of 4 distinct points in $\PP^2$ are arithmetically Gorenstein? Which sets of 5 points? Which sets of 6 points?
\end{exercise}

\begin{exercise}
 Give an example of a set of points in $\PP^3$ that is arithmetically Gorenstein but not a complete intersection. 
 
 Hint: take the 
hyperplane section of a trigonal canonical curve of genus 5.
\end{exercise}

\begin{exercise}
 Let $Q\subset \PP^3$ be the quadric defined by the determinant of the $2\times 2$ matrix 
 $$
F \rTo^{q =\begin{pmatrix}
 x_0&x_1\\
 x_2&x_3
\end{pmatrix}}
G
$$
where $F = S^2(-1)$ and $G = S^2$.
\begin{enumerate}

\item Show that the sheafification of the graded module $M := \coker q$ is $\sO_Q(1,0)$ and the sheafification
of $\coker q^*$ is $\sO_Q(0,1)$ by computing the vanishing locus
of the two sections corresponding to the generators of the module.

\item Show that the sheafification of $\Sym^a(M)$ is $\sO_Q(a,0)$. Conclude that
 if $a\leq b$ then the relative ideal sheaf $\sI_{C/Q} = \sO(-a,-b)$ of a curve $C$ of type $(a,b)$
is the sheafification of the module $Sym^{b-a}(M)(-b)$.

\item Let $S = \CC[x_0,\dots, x_3]$. Show that the minimal $S$-free resolution of $\Sym^a(M)$ 
has the form 
\begin{small}
$$
0 \rTo \wedge^2 F \otimes \Sym^{a-2} G(-2)\rTo F\otimes \Sym^{a-1} G(-1)\rTo \Sym^a G.
%\rTo \Sym^a(M)\rTo 0
$$
\end{small}
Hint: use multilinear algebra (as in \cite{Eisenbud1995}) to define the maps, and use
Theorem~\ref{WMACE} to prove that this is a resolution.

\item Show that if $C\subset \PP^3$ is a curve of type $(a, b)$ with $a\leq b$ then
 there is a free resolution of a module
whose sheafification is $\sI_{C/\PP^3}$ that is a mapping cone of the map of complexes: 
\begin{tiny}
$$
\begin{diagram}[size=2em]
                                                       0&\rTo&0&\rTo& \wedge^2 F &\rTo^{\wedge^2 q}&\wedge^2 G\\
 &&\uTo&&\uTo&&\uTo\\
 0 &\rTo& \wedge^2 F \otimes \Sym^{b-a-2} G(-b-2) &\rTo &F\otimes \Sym^{b-a-1} G(-b-1) &\rTo& \Sym^{b-a} G(-b)
\end{diagram}
$$
\end{tiny}

\item Conclude that the deficiency module of a curve of type $(a, b)$ with $a\leq b$ is the cokernel of a map
$$
\wedge^2 F^* \otimes \Sym^{b-a-2} (G^*)(b+2) \lTo F^*\otimes \Sym^{b-a-1} G^*(b+1)
$$
and thus, after choosing a basis of $F = S^2$, may be identified as the sheafification of the cokernel of
$$
\Sym^{b-a-2} (G^*)(b+2) \lTo F\otimes \Sym^{b-a-1} G^*(b+1)
$$
where the map is the action of $F$ on $\wedge G^*$ via the map $q: F\to G$.
\end{enumerate}
 Hint: Show that the sheafification of the graded module $\coker q$ is $\sO_Q(1,0)$ by computing the vanishing locus
of the two sections corresponding to the generators of the module. Imitate the proof of Theorem~\ref{ENgeneral} to prove
the exactness of the given resolution of $\Sym^a(M)$. See also \cite[Appendix A2.6]{Eisenbud1995}. 
\end{exercise}


\begin{exercise}~\label{Fitt}
 
 Let $\phi: F\to G \to M\to 0$ be an exact sequence of finitely generated $R$-modules, 
with $F$ and $G$ free. 

\begin{enumerate}

\item Show that the annihilator of  $\coker \phi$ has the
same radical as the ideal $I_{\rank G}(\phi)$.

 \item Set $r:= \rank \phi$ and  $I(\phi): = I_{\rank \phi}(\phi)$. Show that the cokernel of $\phi$
is locally free if and only if $I(\phi) = R$. 

\end{enumerate}
\end{exercise}

%footer for separate chapter files

\ifx\whole\undefined
%\makeatletter\def\@biblabel#1{#1]}\makeatother
\makeatletter \def\@biblabel#1{\ignorespaces} \makeatother
\bibliographystyle{msribib}
\bibliography{slag}

%%%% EXPLANATIONS:

% f and n
% some authors have all works collected at the end

\begingroup
%\catcode`\^\active
%if ^ is followed by 
% 1:  print f, gobble the following ^ and the next character
% 0:  print n, gobble the following ^
% any other letter: normal subscript
%\makeatletter
%\def^#1{\ifx1#1f\expandafter\@gobbletwo\else
%        \ifx0#1n\expandafter\expandafter\expandafter\@gobble
%        \else\sp{#1}\fi\fi}
%\makeatother
\let\moreadhoc\relax
\def\indexintro{%An author's cited works appear at the end of the
%author's entry; for conventions
%see the List of Citations on page~\pageref{loc}.  
%\smallbreak\noindent
%The letter `f' after a page number indicates a figure, `n' a footnote.
}
\printindex[gen]
\endgroup % end of \catcode
%requires makeindex
\end{document}
\else
\fi





%$$
%\vbox{\offinterlineskip %\baselineskip=15pt
%\halign{\strut\hfil# \ \vrule\quad&# \ &# \ &# \ &# \ &# \ &# \ 
%&# \ &# \ &# \ &# \ &# \ 
%\cr
%degree&\cr
%\noalign {\hrule}
%0&1&--&--&--&--&--&--\cr
%1&--&17&46&45&4&--&--\cr
%2&--&--&--&--&25&18&4\cr
%\noalign{\bigskip}
%\omit&\multispan{8}{\bf Conjectural shape of $F_\bullet$}\cr
%\noalign{\smallskip}
%}}
%$$
%
%
%\centerline{\scriptsize
%\begin{tabular}{r|ccc} 
%$j\backslash i$&0&1&2\\ 
%\hline 
%0&1&$-$&$-$\\ 
%1&$-$&3&2\\ 
%\end{tabular}}
%
%
