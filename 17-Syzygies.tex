%header and footer for separate chapter files

\ifx\whole\undefined
\documentclass[12pt, leqno]{book}
\usepackage{graphicx}
\input style-for-curves.sty
\usepackage{hyperref}
\usepackage{showkeys} %This shows the labels.
%\usepackage{SLAG,msribib,local}
%\usepackage{amsmath,amscd,amsthm,amssymb,amsxtra,latexsym,epsfig,epic,graphics}
%\usepackage[matrix,arrow,curve]{xy}
%\usepackage{graphicx}
%\usepackage{diagrams}
%
%%\usepackage{amsrefs}
%%%%%%%%%%%%%%%%%%%%%%%%%%%%%%%%%%%%%%%%%%
%%\textwidth16cm
%%\textheight20cm
%%\topmargin-2cm
%\oddsidemargin.8cm
%\evensidemargin1cm
%
%%%%%%Definitions
%\input preamble.tex
%\input style-for-curves.sty
%\def\TU{{\bf U}}
%\def\AA{{\mathbb A}}
%\def\BB{{\mathbb B}}
%\def\CC{{\mathbb C}}
%\def\QQ{{\mathbb Q}}
%\def\RR{{\mathbb R}}
%\def\facet{{\bf facet}}
%\def\image{{\rm image}}
%\def\cE{{\cal E}}
%\def\cF{{\cal F}}
%\def\cG{{\cal G}}
%\def\cH{{\cal H}}
%\def\cHom{{{\cal H}om}}
%\def\h{{\rm h}}
% \def\bs{{Boij-S\"oderberg{} }}
%
%\makeatletter
%\def\Ddots{\mathinner{\mkern1mu\raise\p@
%\vbox{\kern7\p@\hbox{.}}\mkern2mu
%\raise4\p@\hbox{.}\mkern2mu\raise7\p@\hbox{.}\mkern1mu}}
%\makeatother

%%
%\pagestyle{myheadings}

%\input style-for-curves.tex
%\documentclass{cambridge7A}
%\usepackage{hatcher_revised} 
%\usepackage{3264}
   
\errorcontextlines=1000
%\usepackage{makeidx}
\let\see\relax
\usepackage{makeidx}
\makeindex
% \index{word} in the doc; \index{variety!algebraic} gives variety, algebraic
% PUT a % after each \index{***}

\overfullrule=5pt
\catcode`\@\active
\def@{\mskip1.5mu} %produce a small space in math with an @

\title{Personalities of Curves}
\author{\copyright David Eisenbud and Joe Harris}
%%\includeonly{%
%0-intro,01-ChowRingDogma,02-FirstExamples,03-Grassmannians,04-GeneralGrassmannians
%,05-VectorBundlesAndChernClasses,06-LinesOnHypersurfaces,07-SingularElementsOfLinearSeries,
%08-ParameterSpaces,
%bib
%}

\date{\today}
%%\date{}
%\title{Curves}
%%{\normalsize ***Preliminary Version***}} 
%\author{David Eisenbud and Joe Harris }
%
%\begin{document}

\begin{document}
\maketitle

\pagenumbering{roman}
\setcounter{page}{5}
%\begin{5}
%\end{5}
\pagenumbering{arabic}
\tableofcontents
\fi


\chapter{Free resolutions and canonical curves}
\label{SyzygiesChapter}

\def\length{{\rm length}}

In Chapter~\ref{LinkageChapter} we related the resolutions of curves in $\PP^3$ to their Hartshorne-Rao modules. The
simplest case was that in which the Hartshorne-Rao module vanishes, that is, the case of arithmetically Cohen-Macaulay curves.
In this chapter we first quote a result that is a step toward understanding the structure of free resolutions, and use it to explain how the condition that a curve in $\PP^r$ is arithmetically Cohen-Macaulay manifests itself in the 
free resolution of the homogeneous ideal of the curve.  We will then introduce the Eagon-Northcott complexes, and explain their relation to the free resolutions of canonical curves. We close the chapter with an explanation of Green's conjecture, which proposes a a way in which the intrinsic geometry
of a curve may be connected to the shape of its minimal free resolution of its ideal in the canonical embedding.

\begin{remark}
 There are two related contexts for the results of this section:  modules over a local ring and, graded modules over a polynomial ring whose variables have positive degree. These are parallel, essentially because Nakayama's Lemma works in both cases. We will simply work with local rings or polynomial rings over a field with variables in degree 1 and leave the translations to the reader.\end{remark}

\section{Free resolutions}
A more complete presentation of the material in this section may be found in~\cite[Chapter ****]{Eisenbud1995}.


Let $M$ be a finitely generated graded module over $S := \CC[x_0, \dots x_r]$. A \emph{free resolution} of $M$ is an exact complex
of graded free modules, with maps of degree 0:
$$
(\FF, \phi):\quad 0\lTo N \lTo^\epsilon \oplus_jS(-j)^{\beta_{0,j}} \lTo\cdots
 \lTo^{\phi_t}\oplus_jS(-j)^{\beta_{t,j}}\lTo 0,
$$
Here $S(-j)$ denotes the graded free module of rank 1 with generator in degree $j$.
%$$
%\FF: \cdots \rTo F_s \rTo^{\phi_s} F_{s-1} \rTo^{\phi_{s-1}} \cdots \rTo F_1 \rTo^{\phi_1}  F_0 \rTo^\epsilon M \rTo 0.
%$$
(the map to $N$ is not considered part of the free resolution). The resolution is \emph{minimal} if a minimal set of generators of $F_i$ maps to a minimal set of generators of the kernel of the following map,
or equivalently (by Nakayama's Lemma) the maps in $\FF\otimes_S \CC$ are all 0.

The first examples of a minimal free resolution are the Koszul complexes
(first defined, despite the name, in \cite{Cayley}, and repeated, as examples, 
in \cite{Hilbert}), which resolve $/I$ when
 $I = (f_1,\dots, f_t)$
is a complete intersection, that is, $f_1,\dots, f_t$ is a regular sequence, with $\deg f_i = d_i$.  For $t = 2,3$, these look like

$$
%0\lTo S/I 
%\lTo^\epsilon 
S
\lTo^{ \begin{pmatrix}
f_1&f_2
\end{pmatrix}} 
S(-d_1)\oplus S(-d_2) 
\lTo^{\begin{pmatrix}
-f_2\\f_1
\end{pmatrix}}
 S(-d_1-d_2)\lTo 0,
$$
and
%\begin{tiny}
$$
%0\lTo S/I 
%\lTo^\epsilon 
S
\lTo^{
 \begin{pmatrix}
f_1&f_2&f_3
\end{pmatrix}} 
F_1\lTo^{
\begin{pmatrix}
 0&f_3&-f_2\\
 -f_3&0&f_1\\
 f_2&-f_1&0
\end{pmatrix}}
F_2
\lTo^{\begin{pmatrix}
f_1\\f_2\\f_3
\end{pmatrix}}
 F_3 \lTo 0,
$$
%\end{tiny}

where
$$
F_1 = \oplus_{j=1}^3S(-d_j),\
 F_2=
\oplus_{1\leq i <j\leq 3} S(-d_i-d_j),\ 
F_3 = 
S(-d_1-d_2-d_3).
$$
%%

%and
%
%$$
%\begin{aligned}
%0\rTo S \rTo^{
%\begin{pmatrix}
%f_1\\f_2\\f_3
%\end{pmatrix}} S^3&\rTo{
%\begin{pmatrix}
% 0&f_3&-f_2\\
% -f_3&0&f_1\\
% f_2&-f_1&0
%\end{pmatrix}}
% S^3\rTo^{
% \begin{pmatrix}
%f_1&f_2&f_3
%\end{pmatrix}}
%S\rTo S/I \rTo 0
%\end{aligned}
%$$

Hilbert's Basis and Syzygy theorems together imply that every finitely generated $S$-module has a finite free resolution
of length no greater than the number of variables, $n+1$. (Fundamental
results of Auslander, Buchsbaum and Serre say that a local ring $R$ is \emph{regular}---that is, the Krull dimension of $R$ is equal to the minimal number
of generators of $\gm$---if and only if the minimal free resolution of the residue field is finite, in which case
the minimal free resolution of every module is finite.)

To construct this, suppose that a minimal homogeneous set of generators of $N$
contains $\beta_{1,j}$ generators of degree $j$ for each $j$; this defines a degree 0 map $\epsilon$
from 
$
\oplus_jS(-j)^{\beta_{0,j}}
$
onto $N$. We proceed similarly with the kernel of $\epsilon$, and continue to construct the whole resolution.
Hilbert's syzygy theorem \cite[***]{Eisenbud1995} implies that the process ends after $t\leq n$ steps with a map $\phi_t$ whose
kernel is 0. 

The number $t$ is then the projective dimension of $N$. 
Equivalently, 
$$
\pd\ M = \max\{t \mid \Ext_S^t(M,k) \neq 0 \}.
$$
It follows from the Auslander-Buchsbaum
theorem~\cite[***]{Eisenbud1995} that $t \geq \codim \ann_S(N)$, the codimension of the support of $N$.
Minimal free resolutions of a given module $M$ are all isomorphic.


\subsection{How to look at a resolution}
As is apparent even in the example $t=3$ above, free resolutions can be bulky to describe; the 
\emph{betti table} is a compact representation of the numerical information in the resolution.
Suppose that 
$F$ is a minimal free resolution of a module $M$ as illustrated in the beginning of the previous section. Since we choose a minimal set of generators at each stage, the matrices of the $\phi_i$ have entries in the maximal
ideal $(x_0,\dots x_r)$, and thus each $\beta_{i+1, j}$ must be strictly greater than some $\beta_{i,j}$. For this reason it
is convenient to tabulate the numbers so that $\beta_{i,j}$ is in the $i$-th column and $j-i$-th row:
$$
\scriptsize{
\begin{matrix} 
j\backslash i&\vline&0   &  1    & \cdots & n    \cr\hline
\vdots&\vline&\vdots&\vdots & \cdots    &\vdots     \cr 
       0&\vline&\beta_{0,0}&\beta_{1,1}&\cdots&\beta_{n,n}\cr
       1&\vline&\beta_{0,1}&\beta_{1,2}&\cdots&\beta_{n,n+1}\cr
\vdots&\vline&\vdots&\vdots & \cdots    &\vdots     \cr 
\end{matrix}.
}
$$         

\begin{example}
 The Koszul complex that resolves the homogeneous coordinate ring $S/(Q, F_1, F_2)$ of the complete intersection of  2 quadrics and  and a cubic in $\PP^3$ has the form
$$
S \lTo S(-2)\oplus S(-3)^2 \lTo S(-5)^2\oplus S(-6) \lTo S(-8) \lTo 0,
$$
which has betti table:

\centerline{\scriptsize
\begin{tabular}{r|ccccc} 
$j\backslash i$&0&1&2&3\\ 
\hline 
0&1&$-$&$-$&$-$\\ 
1&$-$&1&$-$&$-$\\
2&$-$&2&$-$&$-$\\
3&$-$&$-$&$2$&$-$\\
4&$-$&$-$&$1$&$-$\\
5&$-$&$-$&$-$&$1$\\
\end{tabular}.}
\end{example}

 To simplify our language, we will speak of the \emph{betti table of a scheme $X$} rather than the ``betti table of the minimal free resolution of the ideal of $X$.'

\subsection{When is a finite free complex a resolution?}
How does a free resolution over $S := \CC[x_0, \dots x_r]$ ``know'' to end no later than the $(r+1)$-st step? 
The following result describes a sense in which the maps in the resolution change as the resolution continues.
The result is proven in a more general form (and with a slightly different statement) 
in~\cite[Theorem 20.9]{Eisenbud1995}. We make the convention
that the codimension of the empty set is infinity. If $\phi: F\to G$ is a map of free modules then
$I_t(\phi)$ denotes the ideal generated by all the $t\times t$ minors (=subdeterminants) of a matrix representing $\phi$.


\begin{theorem}\label{WMACE}
 Let $S = k[x_0,\dots, x_r]$, and
 $$ 
\FF:  F_0\lTo^{\phi_1}F_1 \lTo \cdots \lTo F_{n-1}\lTo^{\phi_n} F_n\lTo 0
 $$
be a finite complex of free $S$-modules, and set $r_i := rank \phi_i$. 
The complex $\FF$ is \emph{acyclic} (that is, $H_i(\FF) = 0$ for all $i>0$) if and only if,
\begin{enumerate}
 \item $\rank F_i = \rank \phi_i+\rank \phi_{i+1}$; and
 \item $\codim I_{\rank \phi_i}(\phi_i) \geq i$.
\end{enumerate}
for all $i$.
\qed
\end{theorem}

A familiar case occurs when  $r=1$. In this case the theorem says that a map $F_1\to F_0$ is a monomorphism iff it becomes a monomorphism after tensoring with the field of rational functions $K$, which follows from the flatness of
localization and the fact that $F_1$ is torsion-free, so that
$F_1 \subset F_1 \otimes K$. 

For example, Nakayama's Lemma implies that $X_{0}$ is the support of $\coker \phi_{1}$; thus $X_{0}$ is the set defined by the $\rank F_{0}$-sized minors of $\phi_{1}$. Similarly, 
$X_{r+1}$ is the support of the cokernel of the dual of $\phi_{n}$. 

Note that the theorem would require $\codim X_{r+2}$ to be $\geq r+2$ -- that is, $X_{r+2}$ would have to be empty. 
This means that the minors of $\phi_{r+2}$ of size $\rank \phi_{r+2}$ would be the unit ideal, and it is not hard to show
that this is equivalent to the cokernel of $\phi_{r+2}$ being free. Thus the theorem ``explains'' why a minimal free resolution
has length $\leq r+1$.

\section{Depth and the Cohen-Macaulay property}

If $M$ is a graded module then a \emph{M-regular sequence} is a sequence of homogeneous polynomials
$f_1,\dots,f_m \in (x_0,\dots, x_r)$ such that $f_1$ is a non-zerodivisor on $M$, $f_2$ is a non-zerodivisor on $M/(f_1M)$, and so on. 
The maximal length of such a sequence is called the \emph{depth} of $M$, or more properly the depth of $(x_0,\dots, x_r)$ on $M$.
The length of any all maximal $M$-regular sequences are the same, as one shows by proving

\begin{theorem} (Auslander-Buchsbaum)
If $M$ is a finitely generated $S := \CC[x_0, \dots x_r]$-module, then the length of every $M$-regular sequence is
the smallest integer $m$ such that $\Ext_S^m(S/(x_0, \dots x_r), M) \neq 0$ and is also $r+1 - \pd\  M$.
\end{theorem}
 
 The depth of a module $M$ is bounded above by $\dim M$, the Krull dimension. The reason is that if the dimension of $M$
 is $d$, and $f_1 \in (x_0, \dots x_r) $ is a non-zerodivisor on $M$, then $\dim M/(f_1)M= \dim M-1$. Thus by induction, if
  $f_1,\dots, f_d$ is $M$-regular then $M/(f_1, \dots, f_d)M$ has dimension 0, which is equivalent to its being artinian. Thus any 
$ f_{d+1} \in(x_0, \dots x_r) $ acts as a nilpotent endomorphism of $M/(f_1, \dots, f_d)M$.

It follows from these facts that the depth of an $S$-module $M$ is equal to the dimension of $M$ if and only if the projective dimension
of $M$ is equal to the codimension of $M$; In this case we say that $M$ is a
\emph{Cohen-Macaulay module}. 

As we showed in Chapter~\ref{linkageChapter}, a curve $C\subset \PP^3$ is linked to a complete intersection
if and only if  $H^1_*(\sI_C) := \oplus_{m\in \ZZ} H^1(\sI_C(m)) = 0$, in which case $C$ is said to be is arithmetically Cohen-Macaulay,
From the Auslander-Buchsbaum theorem and Theorem~\ref{***} we see that $C$ is arithmetically Cohen-Macaulay if
and only if the homogeneous coordinate ring $S_C$ is Cohen-Macaulay as an $S$-module, and this is true
if and only if $\pd\  S_C = \codim C$.

It follows from the homological characterization of depth that any localization of a Cohen-Macaulay module is Cohen-Macaulay,
so any arithmetically Cohen-Macaulay scheme is locally Cohen-Macaulay; but this is a much weaker property: any smooth
scheme, and even any scheme that this locally a complete intersection is locally Cohen-Macaulay, a property that does
not depend on a projective embedding.


\subsection{The Gorenstein property} 
Another important homological condition is the condition that $\omega_X$ is an invertible sheaf; when this holds, $X$ is \emph{quasi-Gorenstein}. When, in addition, $X$ is Cohen-Macaulay we say that $X$ is \emph{Gorenstein}. Any scheme that is locally a complete intersection, such as any smooth scheme or locally complete intersection scheme, is Gorenstein. Since the restriction
of $\sO_{\PP^r}(1)$ to a subvariety is always invertible, saying that a scheme $X$ is canonically embedded implies that
$X$ is at least quasi-Gorenstein. As with the Cohen-Macaulay property, the Gorenstein property is interpreted locally
on a scheme. We say that a projective scheme $X$ is \emph{arithmetically Gorenstein}
if its homogeneous coordinate ring is Gorenstein, and it follows that $\omega_{S/I} \cong S/I(a)$ for some integer $a = a(X)$.

In Chapter~\ref{LinkageChapter}, where we expressed $\omega_X$
for a subscheme $X\subset Y$ of a scheme $Y$ as
as $\Ext^{\codim X}_{\sO_Y}(\sO_X, \omega_Y)$. Slightly extending this idea, if $C\subset \PP^r$ is a curve,
 we define $\omega_{S_C}$ to be $\Ext^{r-1}_S,(S_C, S(-r-1))$, where $S$ is the homogeneous coordinate ring of $\PP^r$.
Since $\omega_{\PP^r} = \sO_{\PP^r}(-r-1)$, the sheafification of this module is  $\omega_C$.

If $C$ is arithmetically Cohen-Macaulay, so that
$\pd\ (C) = codim(C) = r-1$,  then 
computing $\Ext^{r-1}_S,(S_C, S(-r-1))$ from the minimal free resolution 
$$
(\FF, \phi):\quad 0\lTo S_C \lTo S\lTo^{\phi_1} F_1 \lTo \cdots \lTo F_{r-2} \lTo^{\phi_{r-1}} F_r\lTo 0
$$
we see that $\omega_{S_C} = \coker \phi_{r-1}^*$. Theorem~\ref{WMACE} implies that the complex $(\FF^*, \phi^*)$ which is the dual
of the the resolution $(\FF, \phi)$ is again acyclic, so it is the minimal free resolution of $\omega_{S_C}$. Thus 
$\omega_{S_C}$ is a Cohen-Macaulay module. Just as the Cohen-Macaulay property of
$S_C$ implies that $S_C = H^0_*(\sO_C)$, it follows that $\omega_{S_C} = H^0_*(\omega_C)$.

If, in addition, $C$ is a canonical curve, so that $\omega_C = \sO_C(1)$, then we derive:
$$
\omega_{S_C} = \coker(\phi_{r-2}^*)(-r-1) = S_C(1)
$$
so $\coker \phi_{r-2}^* = S_C(r)$. Thus $(\FF^*(-r), \phi^*)$ is a minimal free resolution of $S_C$, and is thus isomorphic to
$(\FF^*, \phi^*)$; that is, $(\FF^*, \phi^*)$ is self-dual.
 We have seen an example already
in the Koszul complex (a complete intersection is arithmetically Gorenstein).

Taking into account that in a resolution $(\FF, \phi)$ each summand $S(-j)$ of $F_{i+1}$ can only
map to summands $S(-l)$ of $F_i$ with $\ell < j$, and similarly for the dual, we see that the 
betti table of the minimal free resolution of a canonical curve of genus $g$ must have the form:


% \centerline{\scriptsize
%\begin{tabular}{r|cccccc} 
%$j\backslash i$&0&1&2&$\cdots$&$r-2$&$r-1$\\ 
%\hline 
%0&1&$-$&$-$&$\cdots$&$-$&$-$\\ 
%1&$-$&$\binom{g-2}{2}$&$?$&$\cdots$&$?$&$-$\\
%2&$-$&?&$?$&$\cdots$&$\binom{g-2}{2}$&$-$\\
%3&$-$&$-$&$-$&$\cdots$&$-$&$1$\\
%\end{tabular}.}
%
%\centerline{\scriptsize
%\begin{tabular}{r|cccccc} 
%$j\backslash i$&0&1&2&$\cdots$&$r-2$&$r-1$\\ 
%\hline 
%0&1&$-$&$-$&$\cdots$&$-$&$-$\\ 
%1&$-$&$\binom{g-2}{2}$&$b_2$&$\cdots$&$b_{r-2}$&$-$\\
%2&$-$&$b_{r-2}$&$b_{r-3}$&$\cdots$&$\binom{g-2}{2}$&$-$\\
%3&$-$&$-$&$-$&$\cdots$&$-$&$1$\\
%\end{tabular}.}

\centerline{\scriptsize
\begin{tabular}{r|cccccc} 
$j\backslash i$&0&1&2&$\cdots$&$g-3$&$g-2$\\ 
\hline 
0&1&$-$&$-$&$\cdots$&$-$&$-$\\ 
1&$-$&$b_1$&$b_2$&$\cdots$&$b_{g-3}$&$-$\\
2&$-$&$b_{g-3}$&$b_{g-4}$&$\cdots$&$b_1$&$-$\\
3&$-$&$-$&$-$&$\cdots$&$-$&$1$\\
\end{tabular}.}
\noindent where  $-$ represents 0.  It turns out that the $b_i$ depend on the particular canonical curve,
but since the Hilbert function of the curve is the alternating sum of the Hilbert functions in the resolution,
the differences $b_i- b_{g-i-1}$ are independent of the curve.

To go further, we can make use of the invertible sheaves $\sL$ on $C$ with $h^0(\sL)=2$ and $h^1(\sL)\geq 2$. 
The existence of such a series guarantees that the ideal of $C$ contains the ideal of $2\times 2$ minors of the $2\times h^1(\sL)$
matrix
corresponding to the multiplication map 
$H^0(\sL) \otimes H^0(\sL^-1\otimes \omega_C) \to H^0(omega_C) = H^0(\sO_C(1)$
as in Chapter~\ref{ScrollsChapter}.
The mechanism is a resolution of
the ideal generated by the minors of this matrix.
 
\section{The Eagon-Northcott Complex}\label{EN section}

The Eagon-Northcott complex is a complex of free modules associated to any matrix over any commutative ring. The most familiar special case is the Koszul complex, which one may think of as the Eagon-Northcott complex of a $1\times n$ matrix, and  even in the general case the Eagon-Northcott complex is in a sense built out of the Koszul complexes. A full treatment of this and a family of related complexes can be found in 
\cite[Appendix A2]{Eisenbud1995}, and, from a more conceptual and general point of view, in \cite{Weyman-book}. 

We have seen in Chapter~\ref{ScrollsChapter} that if $X\subset \PP^r$ is linearly normal reduced and irreducible scheme, then any decomposition
of the hyperplane section as the sum of two Cartier divisors
 $H = D+E$ where both $r(D) = 1$ and $r(E)\geq 1$ gives a 1-generic $h^0(\sO_X(D)) \times h^0(\sO_X(E))$
 matrix of linear forms whose $2\times 2$ minors generate the ideal of a rational normal scroll of codimension $r(E)$
 containing $X$. In this section we will give an explicit minimal free resolution of such an ideal of minors, and show that 
 this resolution is a termwise a direct summand of the
 minimal free resolution of the ideal of $X$.

\begin{theorem}\label{Eagon-Northcott}\label{E-N}
 Let $S = k[x_0,\dots, x_r]$ be a polynomial ring,  and let 
 $M: F = S^n(-1) \to G= S^2$ be a homomorphism of degree 0. If $I_2(M)$ has codimension $\geq n-1$, then the minimal free resolution of $S/I_2(M)$ has the form:
\begin{align*}
EN(M) := 
S \lTo{\bigwedge^2 M} 
 \bigwedge^2 F&
 \lTo^{\delta_{2}}
 G^*\otimes \bigwedge^3 F  \lTo^{\delta_{3}}
  (\Sym^2G)^*\otimes\bigwedge^4F  \\
 &\lTo^{\delta_{4}}\cdots\lTo^{\delta_{n-1}} 
(\Sym^{n-2}G)^*\otimes\bigwedge^nF 
 \lTo 0.
\end{align*}
\end{theorem}

Note that the degree of the generators of the $i$-term $EN(M)_i$ is $i+1$ for every $i\geq 1$.

Essentially the same construction works over any local ring, with any map $M: F\to G$ of free modules
with $\rank(F) \geq \rank(G) \geq 1$
and yields a minimal resolution whenever the entries of a matrix for $M$ are in the maximal
ideal and the maximal minors of $M$ generate an ideal of grade $\rank(F) - \rank(G)+1$.

\begin{proof} We begin the discussion of $EN(M)$ by defining the maps $\delta_i$ and and explaining why the given sequence is  a complex---that is, consecutive maps compose to 0. We will then sketch the proof that this complex is a resolution,
which depends only on (multilinear) algebra and Theorem~\ref{WMACE}. 

For simplicity of notation, we choose a generator of $\wedge^2 S^2$
 and identify it with $S$, which gives a sense to the map labeled $\bigwedge^2M$.
 
  Although it is not hard to do this directly, the dual maps
 $$
 \partial_i: \Sym^{i-2} G \otimes \bigwedge^i F^* \rTo \Sym^{i-1} G \otimes \bigwedge^{i+1} F^*
 $$
 have a more familar-looking description, so we define these instead. Indeed, the map $M$ corresponds to an
 element $\mu\in G\otimes F^*$. We may think of $ \Sym^{i-2} G \otimes \bigwedge^i  F^*$
 as a (bigraded) component of the exterior algebra over $ \Sym G$ of 
 $$
  \Sym G \otimes \bigwedge_S  F^*= \bigwedge_{ \Sym G} (\Sym G \otimes  F^*).
 $$
We define $\partial_i$ to be  multiplication by $\mu$ in the sense of this exterior algebra. Since $\mu$ has degree 1
in this sense, its square is 0. 

To show that $(\bigwedge^2 M)\circ \delta_2$ is zero, it is simplest to choose a matrix representing $M$.
Direct computation using only the usual expansion of a determinant
along a row shows that, up to sign, the
pure basis vector $e\otimes f_i\wedge f_j\wedge f_k$ of $G^*\otimes \bigwedge^3 F$
maps under the composition $(\bigwedge^2 M)\circ \delta_2$ to the determinant
of the $3\times 3$ matrix obtained from $M$ by repeating the row corresponding to $e$ and
the columns $i,j,k$. This determinant is 0 because it has a repeated row.

To complete the proof, one in any case show that 
if a map $M': F\to G$ is surjective, then the complex $EN(M')$
is split exact. Such a map $M'$ induces a splitting $F = G\oplus F'$ in such a way that the map $M': G\oplus F' \to G$ is the projection onto the first factor. 

%It will be more convenient to prove the equivalent statement that 
%the dual sequence,  
%\begin{align*}
%EN(M')^* := 
%0\rTo S \rTo{\bigwedge^2 M'^*} 
% &\bigwedge^2 F^*
% \rTo^{\partial_{2}}
% G\otimes \bigwedge^3 F^*  
% \rTo^{\partial_{3}}\\
%  \Sym^2G\otimes\bigwedge^4F^*  
% &\rTo^{\partial_{4}}\cdots\rTo^{\partial_{n-1}} 
%\Sym^{n-2}G\otimes\bigwedge^nF^* 
% \rTo 0.
%\end{align*}
%is split exact.
%
%We begin by proving split exactness at  
% $\Sym^{i} G \otimes \bigwedge^{i+2}  F^*$ where $i\geq 1$.
%The rank of $G$ is 2, so the module
%$ \Sym^{i} G \otimes \bigwedge^{i+2}  F^*$
%decomposes as
%\begin{align*}
%&\Sym^{i} G \otimes \bigwedge^2 G^* \otimes \bigwedge^{i} F'^*\oplus \\
%&\Sym^{i} G \otimes  G^*\otimes \bigwedge^{i+1} F'^* \oplus \\
%&\Sym^{i} G \otimes  \bigwedge^{i+2} F'^* 
%\end{align*}
%Note that under our hypothesis, the element $\mu' \in G\otimes F^* = G\otimes G^* \oplus G\otimes F'^*$
%has the form $(\mu_G, 0)$. Choosing dual bases $g_1,g_2$ of $G$ and $\widehat g_1,\widehat g_2$ of $G^*$,
%$\mu_G$ is the element
% $\mu_G = \sum_i g_i\otimes \widehat g_i$
%representing the identity map $G \to G$. Thus the complex
%$EN(M')^*$ is a direct sum over $i$ of 3-term complexes of the form
%%$$
%%\Sym^{i+1} G \otimes \bigwedge^2 G^* 
%%\lTo^{-\wedge \mu'} 
%%\Sym^{i} G \otimes  G^*
%%\lTo^{-\wedge \mu'} 
%%\Sym^{i-1} G
%%$$
%$$
%\Sym^{i-1} G 
%\rTo^{-\wedge \mu'} 
%\Sym^{i} G \otimes  G^*
%\rTo^{-\wedge \mu'} 
%\Sym^{i+1} G \otimes \bigwedge^2 G^* 
%$$
%tensored with various $\bigwedge^j F'^*$, and it suffices to show that the former are split exact in the middle
%when
%$i\geq 1$. Now $\Sym G$ may be identified with $R:= S[x,y]$, where $x,y$ are a basis of $S^2$, and
%as such the three-term sequence above may be identified with the $(i-1)$-st homogeneous component 
%of the Koszul complex of $x,y$ over $R$,
%%$$
%%R\otimes \wedge^2 G\lTo R\otimes G \lTo R .
%%$$
%$$
%0\to R \rTo R\otimes G^* \rTo R\otimes \wedge^2 G^*
%$$
%This is an $R$-free resolution of  $R/(x,y)R = S$ so as sequences
%of $S$-modules every homogeneous component is split exact at the term $R_i\otimes G^*$,
%as required.
%
%It remains to treat the beginning of the complex $EN(M')^*$,
%$$
%S \rTo{\bigwedge^2 M'^*} 
% \bigwedge^2 F^*
% \rTo^{-\wedge \mu'}
%G\otimes \bigwedge^3 F^*,
%$$
%which in our case may be written:
%%$$
%%\wedge^2 F^* = \bigwedge^2 G^* \oplus (G^*\otimes F'^*) \oplus \bigwedge^2 F'*
%%$$
%%Consider the pair of maps
%\begin{align*}
%S \rTo{\bigwedge^2 M'^*} 
% &\bigwedge^2 G^* \ \oplus\ (G^*\otimes F'^*)\ \oplus\ \bigwedge^2 F'^*
% \rTo^{-\wedge \mu'} \\
% &G\otimes \bigwedge^2 G^*\otimes F'^*\ \oplus\ (G\otimes G^*\otimes \bigwedge^2 F'^*)\ \oplus\ G\otimes \bigwedge^3 F'^*
%\end{align*}
%The map $\bigwedge^2 M'$ is the projection to $\bigwedge^2 G$ composed with the chosen isomorphism
%$\bigwedge^2 G \cong S$, and is thus a split monomorphism. To complete the argument, we must show that
% the map marked $-\wedge \mu'$ is a monomorphism on $(G^*\otimes F'^*) \oplus \bigwedge^2 F'^*$.
% 
% Identifying $\wedge^2 G^*$ with $S$, this map is the direct sum of the two maps
%  $$
% (G^* \rTo^{-\wedge \mu_G} G\otimes \bigwedge^2 G^*)  \otimes F'^*
% $$
% and
% $$
%(S  \rTo^{\mu_G} G\otimes G^*) \otimes \bigwedge^2 F'^*.
% $$
% The first of these is the tensor product of $F'^*$ with
% something that becomes the identity map after the natural identification of $G^*$ with $G\otimes \wedge^2 G^*$,
% while the second is the tensor product of $\wedge^2 F'^*$ with
%  $$
% 1 \mapsto \sum_i(g_i\otimes \widehat g_i).
% $$
%Since these are split monomorphisms,
%the proof of split exactness of $EN(M')^*$ and thus of $EN(M')$ is complete.

To complete the proof of Theorem~\ref{Eagon-Northcott},
let $X_{i}\subset \AA^{r+1}$ be the variety defined from the complex $EN(M)$ as in 
Theorem~\ref{WMACE}. Since $EN(M)$ becomes split exact after inverting any $2\times 2$ minor of $M$
$X_{i}$ is
contained in the closed set defined by $I_{2}(M)$, for all $i$. Theorem~\ref{WMACE} now shows that
 if $I_{2}(M)$ has codimension $n-1$,
then $EN(M)$ is a resolution, completing the proof.
\end {proof}

\begin{corollary}\label{E-N cor}
With notation as in Theorem~\ref{E-N}, if the ideal $I_2(M)$ has codimension $\geq n-1$ then it has
codimension exactly $n-1$, the ring $S/I_2(M)$ is Cohen-Macaulay, and the $\binom{n}{2}$ quadratic forms
that are the $2\times 2$ minors of $M$ are linearly independent over $\CC$.
\end{corollary}

\begin{proof}
From the resolution $EN(M)$ we see that the projective dimension of $S/I_2(M)$ is $n-1$. Since the projective dimension of a module
is at least the codimension of its annihilator, the equality follows, and the Auslander-Buchsbaum formula implies that $S/I_2(M)$ is 
Cohen-Macaulay. The linear independence of the minors of $M$ follows because $EN(M)$ is a resolution and there
$EN(M)_2$ is generated in degree 3, so all the relations on the minors have coefficients of degree 1.
\end{proof}

\begin{example}
The betti table of the Eagon-Northcott complex of 
a $2\times n$ matrix of linear forms is:

\centerline{\scriptsize
\begin{tabular}{r|ccccc} 
$j\backslash i$&0&1&2&3&4\\ 
\hline 
0&1&$-$&$-$&$\cdots$&$-$\\ 
1&$-$&$\binom{n}{2}$&$2\binom{n}{3}$&$\cdots$&$(n-1)\binom{n}{n}$\\ 
\end{tabular}}
\end{example}

In general when $X\subset Y\subset \PP^r$, so that $I_X \supset I_Y$, it may be hard to see which syzygies of $X$ come
from syzygies of $Y$. But when the degrees of the syzygies of $Y$ are smaller than those from $X$, the situation is simpler.
Here is the special case we will use:

\begin{proposition}
Suppose that $C\subset \PP^r$ is a nondegenerate curve. If $C\subset X \subset \PP^r$, where $X$ is a rational
normal scroll, then the Eagon-Northcott complex that is  the minimal free resolution of $I_X$ is termwise a direct summand
of the minimal free resolution of $I_C$. Thus the betti table of the resolution of $I_C$ is termwise $\geq$ that of $I_X$.
\end{proposition}

\begin{proof}
Let $EN$ be the minimal resolution of $I_X$, and let $\FF$ be the minimal resolution of $I_C$.
The inclusion $I_X \subset I_C$ induces a map $\phi: EN\to \FF$, unique up to homotopy. Since the minimal generators of $I_X$ are quadratic, and $I_C$ contains no linear forms, $\phi_0: EN_0\to \FF_0$ is a spit monomorphism.

By induction, we may assume that $\phi_{i-1}$ is a split monomorphism. The free module $EN_{i}$ is generated in 
degree $i+1$, while the free module $\FF_i$ is generated in degrees $\geq i+1$. It follows that the relations
represented by $EN_i$, extended by 0, are among the minimal generators of the relations represented by $\FF_i$,
completing the proof.
 \end{proof}
 

%\fix{START COMMENTED OUT MATERIAL}
%We will also use a special case of the Auslander-Buchsbaum formula connecting projective dimension and depth:
%
%\begin{theorem}\label{Auslander-Buchsbaum}
%If $R$ is a regular local ring of dimension $d$, and $M$ is a finitely generated $R$-module, then the projective dimension of $M$ is $\leq d$ with equality only if
%$M$ contains a submodule of finite length. 
%\end{theorem}
%
%\begin{corollary}\label{associated primes}
%If $R$ is a regular local ring of dimension $d$, and $M$ is a finitely generated $R$-module, then the codimension of an associated prime of $A$ is at most the projective dimension of $A$. 
%\end{corollary}
%\begin{proof}[Proof of  Corollary~\ref{associated primes}]
% Projective dimension can only decrease under localization, and
% the associated primes $P$ of $A$ are those for which $A_{P}$ contains a submodule
% of finite length.
%\end{proof}
%
%With this and the multi-linear algebra above we can  prove the basic acyclicity result for an Eagon-Northcott complex:
%
%\fix{the Theorem as now stated doesn't need the following. I've copied the short proof
%of acyclicity into the end of the proof above.}
%\begin{proposition}\label{acyclicity}
%Let $S = k[x_0,\dots, x_r]$ be a polynomial ring,  and let $M: F\to G$ be a homomorphism with
% $F = S^n(-1), G= S^2$.
% $$
% S^n \cong F \rTo^M G \cong S^2
% $$
% is a (not necessarily homogeneous) map of free $S$-modules.
% The Eagon-Northcott complex $EN(M)$ is acyclic if and only if $\codim I_2(M) \geq n-1$, in which case the dual complex is also acyclic and
% the associated primes of $I_2(M)$ are all minimal and of codimension $n-1$.
% \end{proposition}
%
%\begin{proof}[Proof of Proposition~\ref{acyclicity}]
%Let $X_{i}\subset \AA^{r+1}$ be the variety defined from the complex $EN(M)$ as in 
%Theorem~\ref{WMACE}. Since $EN(M)$ becomes split exact after inverting any $2\times 2$ minor of $M$
%$X_{i}$ is
%contained in the closed set defined by $I_{2}(M)$, for all $i$. Thus if $I_{2}(M)$ has codimension $n-1$,
%then $EN(M)$ is acyclic. 
%
%In this case the projective dimension
%of $S/I_{2}(M)$ is $n-1$, so all the associated primes of $I_{2}(M)$ have codimension
%exactly $n-1$.
%
%If $EN(M)$ is  acylic then, by Theorem~\ref{WMACE}, the codimension of $X_{n-1}$ is at least $n-1$. Thus to prove that the acyclicity of $EN(M)$ implies  $\codim I_{2}(M) \geq n-1$ (and thus
%$\codim I_{2}(M))  n-1$, it suffices to show that $X_{n-1} = X_{0}$ as algebraic sets.
%
%To see this, note that
%the ideal of $2\times 2$ minors of $M$. By definition, $X_{n-1}$ 
%is the set of points $p$ where $\kappa(p)\otimes \delta_{n-1}$ is not an inclusion, 
%or equivalently, that 
%that 
%$$
%\kappa(p)\otimes F\otimes \Sym^{n-3}G 
%\cong 
%\kappa(p)\otimes \bigwedge^{n-1} F^{*} \otimes  \Sym^{n-3}G 
%\rTo^{\partial_{n-1}}
%\kappa(p)\otimes \bigwedge^{n} F^{*} \otimes  \Sym^{n-2}G
%\cong
%\kappa(p)\otimes \Sym^{n-2}G
%$$
%is not a split surjection, and it is easy to see that the composite map takes
%$a\otimes b$ to $\kappa(p) \otimes M(a)\cdot b$, so the cokernel is the $(n-2)$-nd symmetric power
%of the cokernel of the map $\kappa(p) \otimes M$. Thus $X_{n}$ is equal to the support
%of the cokernel of $M$ itself. 
%
%By Nakayama's Lemma, $X_{0}$ is the support of $M$; furthermore, the localization of $\coker M$ at $p$ is 0 if and only if
%one of the 
%$2\times 2$ minors of $M$ is a unit locally at $p$ so $X_{0}$, so this is defined
%set-theoretically by $I_{2}(M)$.
%
%It now follows from Theorem~\ref{WMACE} that all the $X_{i}$ are equal, so $EN(M)$ is 
%acyclic if and only if $EN(M)^{*}$ is acyclic. 
%
%Since $M$ is 1-generic the entries of of the second row of $M$ are linearly independent, and since the dimension of the span of all the linear forms is at least $n+1$, some element in the first row is outside the span of the the elements in the second. After a permutation of columns we may assume that $l_{1,1}, l_{2,1}, l_{2,2},\dots l_{2,n}$ are linearly independent, and we may take them to be a subset of the variables, say $x_{0},\dots x_{n+1}$
%
%We next show by induction on $n$ that  $I_{2}(M)$ is prime. In the case $n=2$ we have $I_{2}(M) =  (x_{0}x_{2}-x_{1}l_{1,2})$ which obviously does not factor. 
%
%Now suppose that $n>2$, and let $M'$ be the matrix $M$ with the first column omitted. we know by induction that $I_{2}(M')$ is prime of codimension $n-2$. Since $I:=I_{2}(M)$ does not have the maximal ideal as an associated prime, it is saturated. The ideal $I_{2}(M)+x_{0}$ properly contains $I_{2}(M')$ and thus has codimension $\geq n-1$ in $S/x_{0}$, whence we see that every component of $I_{2}(M)$ meets the open set
%$x_{0} = 1$. Restricting to this open set \fix{complete the proof}
%\end{proof}
%
%The first non-trivial example of a finite free resolution is the Koszul complex on 3 variables, which is the minimal $S = k[x,y,z]$-free resolution of the module $S/(x,y,z)$:
%$$
%0\to S(-3) \rTo^{
%\begin{pmatrix}
%x\\y\\z 
%\end{pmatrix}}
% S^3(-2) \rTo^{\begin{pmatrix}
%0&-z&y\\
%z&0&-x\\
%-y&x &0
%\end{pmatrix}}
%S^3(-1) \rTo^{
%\begin{pmatrix}
%x&y&z
%\end{pmatrix}}
%S
%$$
%In fact this is the first example that 
%Hilbert
% presented in his famous paper \cite{Hilbert1890}. 																											
%\fix{END COMMENTED OUT MATERIAL}

%\section{How to look at a resolution}
%In this section we collect some definitions and results from \cite{Eisenbud1995} that will help to understand the exposition of Green's conjecture, which occupies the last section of this chapter.
%
%If $N$ is a finitely generated graded module over $S = \CC[x_0,\dots, x_r]$, then there is a minimal graded free resolution $\FF$
%of $N$ that we can write in the form
%$$
%0\lTo N \lTo^\epsilon \bigoplus_jS(-j)^{\beta_{0,j}} \lTo^{\phi_1} \cdots \lTo^{\phi_t}\bigoplus_jS(-j)^{\beta_{t,j}}\lTo 0,
%$$
%where it is understood that each direct sum has only finitely many nonzero terms.
%To construct this, suppose that a minimal homogeneous set of generators of $N$
%contains $\beta_{1,j}$ generators of degree $j$ for each $j$; this defines a degree 0 map $\epsilon$
%from 
%$
%\oplus S(-j)^{\beta_{0,j}}
%$
%onto $N$. We proceed similarly with the kernel of $\epsilon$, and continue to construct the whole resolution.
%Hilbert's syzygy theorem \cite[***]{Eisenbud1995} implies that the process ends after $t\leq n$ steps with a map $\phi_t$ whose
%kernel is 0. The number $t$ is then the projective dimension of $N$. It follows from the Auslander-Buchsbaum
%theorem~\cite[***]{Eisenbud1995} that $t \geq \codim \ann_S(N)$, the codimension of the support of $N$.
%
%Since we choose a minimal set of generators at each stage, the matrices of the $\phi_i$ have entries in the maximal
%ideal $(x_0,\dots x_r)$, and thus each $\beta_{i+1, j}$ must be strictly greater than some $\beta_{i,j}$. For this reason it
%is more convenient to tabulate the numbers so that $\beta_{i,j}$ is in the $i$-th column and $j-i$-th row:
%
%$$
%\scriptsize{
%\begin{matrix} 
%j\backslash i&\vline&0   &  1    & \cdots & n    \cr\hline
%\vdots&\vline&\vdots&\vdots & \cdots    &\vdots     \cr 
%       0&\vline&\beta_{0,0}&\beta_{1,1}&\cdots&\beta_{n,n}\cr
%       1&\vline&\beta_{0,1}&\beta_{1,2}&\cdots&\beta_{n,n+1}\cr
%\vdots&\vline&\vdots&\vdots & \cdots    &\vdots     \cr 
%\end{matrix}.
%}
%$$         
%
%
%%$$
%%\vbox{\offinterlineskip %\baselineskip=15pt
%%\halign{\strut\hfil# \ \vrule\quad&# \ &# \ &# \ &# \ &# \ &# \ 
%%&# \ &# \ &# \ &# \ &# \ 
%%\cr
%%degree&\cr
%%\noalign {\hrule}
%%0&1&--&--&--&--&--&--\cr
%%1&--&17&46&45&4&--&--\cr
%%2&--&--&--&--&25&18&4\cr
%%\noalign{\bigskip}
%%\omit&\multispan{8}{\bf Conjectural shape of $F_\bullet$}\cr
%%\noalign{\smallskip}
%%}}
%%$$
%%
%%
%%\centerline{\scriptsize
%%\begin{tabular}{r|ccc} 
%%$j\backslash i$&0&1&2\\ 
%%\hline 
%%0&1&$-$&$-$\\ 
%%1&$-$&3&2\\ 
%%\end{tabular}}
%%
%%
%\begin{example}
% For example the Koszul complex that resolves the homogeneous coordinate ring $S/(Q, F_1, F_2)$ of the complete intersection of  2 quadrics and  and a cubic in $\PP^3$ has the form
%$$
%S \lTo S(-2)\oplus S(-3)^2 \lTo S(-5)^2\oplus S(-6) \lTo S(-8) \lTo 0,
%$$
%which has betti table:
%
%\centerline{\scriptsize
%\begin{tabular}{r|ccccc} 
%$j\backslash i$&0&1&2&3\\ 
%\hline 
%0&1&$-$&$-$&$-$\\ 
%1&$-$&1&$-$&$-$\\
%2&$-$&2&$-$&$-$\\
%3&$-$&$-$&$2$&$-$\\
%4&$-$&$-$&$1$&$-$\\
%5&$-$&$-$&$-$&$1$\\
%\end{tabular}.}
%\end{example}



%We will also use a special case of the Auslander-Buchsbaum formula connecting projective dimension and depth:
%
%\begin{theorem}\label{Auslander-Buchsbaum}
%If $R$ is a regular local ring of dimension $d$, and $M$ is a finitely generated $R$-module, then the projective dimension of $M$ is $\leq d$ with equality only if
%$M$ contains a submodule of finite length. 
%\end{theorem}
%
%\begin{corollary}\label{associated primes}
%If $R$ is a regular local ring of dimension $d$, and $M$ is a finitely generated $R$-module, then the codimension of an associated prime of $A$ is at most the projective dimension of $A$. 
%\end{corollary}
%\begin{proof}[Proof of  Corollary~\ref{associated primes}]
% Projective dimension can only decrease under localization, and
% the associated primes $P$ of $A$ are those for which $A_{P}$ contains a submodule
% of finite length.
%\end{proof}
%

\section{Green's Conjecture}

Corollary~\ref{canonical hilbert function} implies that the dimension of the vector space of forms of degree $d$
vanishing on a canonical curve is independent of the curve; for example, for $d=2$ we get
$
\dim ({I_{C}})_{2} = {g-2\choose 2}.
$
The Hilbert function of $I_C$ is determined by the betti table of its resolution, so that the table generally has more information.

 For example
when $C$ is trigonal then, by the geometric Riemann-Roch theorem, $C$ has a 1-dimensional family of trisecant lines; any quadric containing $C$ must contain all these. As we have seen in Chapter~\ref{ScrollsChapter}, these lines sweep
out a 2-dimensional scroll defined by the $2\times g-2$ matrix corresponding to the decompostion of $\sO_C(1)$
into a tensor product of the $g^{1}_{3}$ defined by a line bundle $\sL$, the complementary linear series,
defined by $\omega_{C}\otimes \sL^{-1}$ has $g-2$ sections, and we see from Theorem ****
that the scroll itself lies on the ${g-2\choose 2}$ quadrics defined by the minors of a $2\times g-2$, 1-generic matrix of linear forms. The exactness of the Eagon-Northcott complex associated to this matrix shows that there are no relations of degree 0 on these minors -- that is, they are linearly independent over the ground field. It follows that they generate the vector space of all quadrics containing $C$. This
ideal of minors defines a rational normal scroll of dimension 2.

Furthermore, if $g = 6$ and $C$ is isomorphic to a plane quintic curve, then the canonical series of the plane quintic is $5-3 = 2$ times the hyperplane series, and it follows that the canonical image of $C$ lies on the Veronese surface in $\PP^{5}$. Using Theorem *** again, we see that the Veronese is contained in (in fact, equal to) the intersection of the quadrics defined by the $2\times 2$ minors of a generic symmetric matrix, coming from the 
multiplication map 
$$
H^{0}(\sO_{\PP^{1}}(1))\otimes H^{0}(\sO_{\PP^{1}}(1)) \to H^{0}(\sO_{\PP^{1}}(2)) = H^{0}(\sO_{\PP^{5}}(1))
$$
and there are $6 = {g-2\choose 2}$ independent quadrics in this ideal. Again in this case, they cannot generate the ideal of the curve.

One might fear that this is the beginning of some long series of examples, but in fact it is not: 

\begin{theorem} [Petri]
The ideal of a canonical curve of genus $\geq 5$ is generated by the $\g-2\choose 2$-dimensional space of quadrics it contains unless the curve is either trigonal or isomorphic to a plane quintic; in the latter cases, the ideal of the curve is generated by quadrics and cubics.
\end{theorem}

For a modern treatment of Petri's Theorem in this level of generality see \cite{Schreyer}; for a different treatment see \cite{Arbarello-Sernesi}.

The two exceptions can be described simultaneously by using the Clifford index:

\begin{definition}
 The \emph{Clifford index} Cliff $\sL$ of a line bundle $\sL$ on a curve $C$ is $d-2r$, where $d := \deg \sL$ and $r :=  h^0(\sL)-1$. The Clifford index Cliff $C$ of
 a curve $C$ of genus $\geq 2$ is the minimum of the Clifford indices of special line bundles with at least 2 sections.
\end{definition}

Clifford's Theorem (Corollaries \ref{Clifford bound} and ~\ref{equality in Clifford from Martens}) says that Cliff $C \geq 0$, and that Cliff $C = 0$ if and only if $C$ is hyperelliptic. If $C$ is not hyperelliptic, then it turns out that Cliff $C=1$ if and only if $C$ is either trigonal or isomorphic to a plane quintic. The Clifford index of any smooth curve of genus $g\geq 2$ is $\leq \lceil g/2\rceil+1$, with equality for a general curve, as one sees from the Brill-Noether Theorem~\ref{basic BN}, and for ``most'' curves the line bundle $\sL$ of maximal Clifford index has only 2 sections, though there is an infinite sequence of examples where this
``Clifford dimension'' is greater.

Moving to cubic forms, we see that $\dim ({I_C})_3 = {g+2\choose 3}-(5g-5)$. Comparing this number with the number of (possibly linearly dependent)
cubics obtained by multiplying $g$ linear forms and ${g-2\choose 2}$ quadrics, we see that the ideal of the curve has at least
$$
{g-2\choose 2} - {g+2\choose 3}-(5g-5) 
$$
independent syzygies of total degree 3 (that is, linear syzygies on the quadrics). For example when $g=4$ so that $C\subset \PP^3$ there is one quadric and 5 independent
cubics, at most 4 of which are multiples of the quadric. Since the curve has degree $6 = 2\times 3$, the ideal of the curve must be generated by
the quadric and one cubic. When $g=5$ there are genuinely two possibilities: the three quadrics in the ideal might be a complete intersection
(then they generate the ideal), so the Betti table would be

\

\centerline{\small %\scriptsize
\begin{tabular}{r|ccc} 
$j\backslash i$&0&1&2\\ 
\hline 
0&1&$-$&$-$\\ 
1&$-$&2&$-$\\ 
2&$-$&$-$&$1$\\ 
\end{tabular}}

\noindent or the curve could be trigonal, in which case the 3 quadrics generate the ideal of a surface scroll $F$. In the latter
case, the Eagon-Northcott complex resolves the homogeneous coordinate  ring  $S_F$ of the scroll,
$$
0\to S^2(-3) \to S^3(-2) \to S \to S_F \to 0
$$
which has Betti table

\centerline{\small %\scriptsize
\begin{tabular}{r|ccc} 
$j\backslash i$&0&1&2\\ 
\hline 
0&1&$-$&$-$\\ 
1&$-$&3&$2$\\ 
\end{tabular}}
\noindent and we see that there are 2 linear relations among the quadrics. Thus the minimal generators of $I_C$ must include exactly 2 cubics as well as the 3 quadrics. Since the homogeneous ring of a canonical curve is Gorenstein, its minimal free resolution is symmetric, and this is enough for us to fill in its Betti table:

\centerline{\small %\scriptsize
\begin{tabular}{r|cccc} 
$j\backslash i$&0&1&2&3\\ 
\hline 
0&1&$-$&$-$&$-$\\ 
1&$-$&3&$2$&$-$\\ 
2&$-$&$2$&$3$&$-$\\ 
3&$-$&$-$&$-$&$1$\\ 
\end{tabular}}
\noindent Note that we can ``see'' the scroll reflected in the top two lines of the table.

From the analogue of the Hilbert-Burch Theorem for Gorenstein rings of codimension 3 one can show that the 5 generators can be written as the 
pfaffians of a skew symmetric $5\times 5$ matrix whose entries are of degrees 1 and 2, in the following pattern (we give just the degrees, and put - in the places that are 0):
$$
\begin{pmatrix}
 -&-&1&1&1\\
-&-&1&1&1\\
1&1&-&2&2\\
1&1&2&-&2\\
1&1&2&2&-
\end{pmatrix}
$$
Here the $2\times 2$ minors of the upper $2\times 3$ block of linear forms generate the ideal of the scroll. 

Applying this logic more generally we get the following result about the canonical embedding of curves with low degree maps to $\PP^1$:

\begin{theorem}
 Let $C\subset \PP^{g-1}$ be a reduced, irreducible canonical curve. If $C$ has a line bundle $\sL$ of degree $d \leq g-1$ with $h^0(\sL) = 2$  then
 there is a 1-generic  $2\times (g+1-d)$ matrix of linear forms whose minors define a scroll of codimension $g-d$ containing $C$; and thus an
 Eagon-Northcott complex of length $g-d$ is a subcomplex of the minimal free resolution of $S_C$. In particular, the Betti table of $S_C$ is
 termwise $\geq$ that of the homogeneous coordinate ring of the scroll.
 \end{theorem}
 
 (We have stated this theorem for canonical curves, but in fact the construction applies much more generally to a linearly normal variety $X \subset \PP^n$ of any dimension: if $X$ has a divisor $D$ that moves in a pencil and is contained in a subspace $\PP^k$ with $k \leq n-2$, the planes spanned by the divisors of the pencil $|D|$ sweep out a rational normal scroll.)
 
 Thus the existence of the $g^1_d$ on $C$, together with the symmetry of the resolution of the Gorenstein ring $S_C$,
  implies that the Betti table of $S_C$ has the form
 
 \centerline{\small %\scriptsize
\begin{tabular}{r|cccccccccccc} 
$j\backslash i$&0&1&2&\dots&d-3&d-2&\dots&g-d-1&g-d&\dots&g-3&g-2\\ 
\hline 
0&1&$-$&$-$&$\cdots$&$-$&$-$&$-$&$-$&$-$&$-$&$-$&$-$\\ 
1&$-$&*&*&$\cdots$&*&*&$\cdots$&*&$?$&$\cdots$&?&?\\ 
2&$-$&?&?&$\cdots$&?&*&$\cdots$&*&*&$\cdots$&*&*\\ 
3&$-$&$-$&$-$&$\cdots$&$-$&$-$&$-$&$-$&$-$&$-$&$-$&1
\end{tabular}}
\noindent where we have assumed for illustration that $d-2<g-d-1$. The places marked $-$ are definitely 0 and those marked * are definitely nonzero. The entries of the rows marked 0 and 1 are $\geq$ the corresponding entries of the Betti table of the scroll.

We can summarize this by saying
that if the curve $C$ has a line bundle $\sL$ of degree $d$ with exactly 2 sections, and thus of Clifford index $c = d-2$ the row labeled  2 
in the Betti diagram definitely has $\beta_c, c+2\neq 0$. As with the case of the plane quintics, above, one can
make a similar argument for \emph{any} line bundle of Clifford index $c$. Thus:

\begin{corollary}
 If Cliff $C \leq c$ then $\beta_{c,c-2}(S/I_C)) \neq 0.$
\end{corollary}
 

Starting from examples such as the case of genus 6, Mark Green made a bold conjecture that is still open as of this writing:

\begin{conjecture}[Green's Conjecture]
If $C$ is a smooth canonical curve of genus $g$ and $S/I_C$ is the homogeneous coordinate ring of $C$ in its canonical embedding,
then the Clifford index of $C$ is $\leq c$ if and only if $\beta_{c,c-2}(S/I_C) \neq 0$. 
\end{conjecture}

The conjecture was made for curves over a field of characteristic 0, and is known in many cases, though it is also known to fail in small finite characteristics (see~\cite{Bopp-Schreyer} for an amended conjecture that may hold in all characteristics.)
In \cite{MR1941089} and \cite{MR2157134} the conjecture was proven for generic curves of each Clifford index.  See~\cite{MR4022070} for the currently simplest
proof. As of this writing the conjecture is also is known up to genus 9,  for plane curves, and in a number of other special cases.
See for example \cite{Farkas-progress-on-syzygies} for a survey on this and related topics.

\subsection{Low genus canonical embeddings} 
In his 1983 Brandeis thesis \cite{Schreyer-canonical}, Frank Schreyer analyzed all the possibilities for resolutions of smooth canonical curves up to genus 8. The project was carried further to genus 9 in a Saarbr\"ucken thesis \cite{Sagraloff}  of a student of Schreyer. The results are summarized in the tables below.
\fix{M2 notation is
different; and the hyperelliptic cases are included. Maybe retype in normal notation?}

\includepdf[pages=1, scale=.8]{"SyzygiesGupto8"}
\includegraphics[scale = .35]{"genus 9 Sagraloff"}
%\includepdf[scale=.8]{"SyzygiesGupto8"}
%\section{Low degree}

\section{Exercises}

\begin{exercise}\label{WMACE corollary}
Use Theorem~\ref{WMACE} to prove that if
 Let $S = k[x_0,\dots, x_r]$, and
 $$ 
\FF:  F_0\lTo^{\phi_1}F_1 \lTo \cdots \lTo F_{n-1}\lTo^{\phi_n} F_n\lTo 0
 $$
be a finite complex of free $S$-modules. Set
$$
X_i = \{p\in \AA^{n+1} \mid  H_i(\FF \otimes \kappa(p)) \neq 0\}
$$
The complex $\FF$ is \emph{acyclic} (that is, $H_i(\FF) = 0$ for all $i>0$) if and only if
$$
\codim X_i \geq i
$$
for all $i>0$. Moreover, $X_{0}\supseteq X_{1}\supseteq \cdots \supseteq X_{n}$

Hint: Elementary linear algebra shows that, if $k$ is a field, then a complex $k^p \lTo^phi k^q \lTo^\psi k^r$ is exact at $k^q$ if and
only if $\rank \phi +\rank \psi = q$. 
\end{exercise}

\begin{exercise}
Prove that if $X\subset \PP^r$ is arithmetically Cohen-Macaulay then the dual of the minimal free resolution of $S/I_X$
is the minimal free resolution of $\omega_X$. Hint:Use Theorem~\ref{WMACE} and the characterization of $\omega_{S/I_X}$
as an Ext module.
\end{exercise}

\begin{exercise}
 Prove Lemma~\ref{depth lemma} for modules over a regular local ring.  Hint: use the characterization of depth
 via projective dimension.
\end{exercise}

\begin{exercise}
In dealing with arithmetically Gorenstein schemes, we used the fact that if $\omega_{S/I}$ is an invertible
sheaf (over $\Spec(S/I)$) then it is isomorphic to $S/I(a)$ for some $a$. Why is this true?
\end{exercise}

\begin{exercise}
Find a degree 6 embedding of a curve of genus 3 that is not arithmetically Cohen-Macaulay, and another that is.
Hint: Show that the $3\times 3$ minors of a general $4\times 3$matrix of linear forms defines a Cohen-Macaulay curve
of genus 3. Show that a curve of type $(2,4)$ on a smooth quadric in $\PP^3$ is not arithmetically Cohen-Macaulay.
\end{exercise}

\begin{exercise}
Referring to the betti table of a canonical curve just before Section~\ref{EN section}, give a formula
for the differences $b_i- b_{g-i-1}$ that depends only on $i$ and $g$.
\end{exercise}

\begin{exercise}
 Show that if $I \subset S := \CC[x_0,\dots, x_r]$ is a codimension 2 ideal, then $S/I$ is Cohen-Macaulay if and only
 if the minimal $S$-free resolution of $S/I$ has the form
 $$
 0\rTo S^{n-1} \to S^n \to S
 $$
 for some $n$. Show that $S/I$ is Gorenstein if and only if $I$ is a complete intersection. Hint: for the first part, tensor with
 the field of rational functions. 
\end{exercise}

\begin{exercise}
Which sets of 4 distinct points in $\PP^2$ are arithmetically Gorenstein? Which sets of 5 points? Which sets of 6 points?
\end{exercise}

\begin{exercise}
 Give an example of a set of points in $\PP^3$ that is arithmetically Gorenstein but not a complete intersection. Hint: take the 
hyperplane section of a trigonal canonical curve of genus 5.
\end{exercise}

\begin{exercise}
 Let $Q\subset \PP^3$ be the quadric defined by the determinant of the $2\times 2$ matrix 
 $$
q := F \rTo^{\begin{pmatrix}
 x_0&x_1\\
 x_2&x_3
\end{pmatrix}}
G
$$
where $F = S^2(-1)$ and $G = S^2$.
\begin{enumerate}

\item Show that the sheafification of the graded module $M := \coker q$ is $\sO_Q(1,0)$ and the sheafification
of $\coker transpose q$ is $\sO_Q(0,1)$ by computing the vanishing locus
of the two sections corresponding to the generators of the module.

\item Show that the sheafification of $\Sym^a(M)$ is $\sO_Q(a,0)$. Conclude that
 if $a\leq b$ then the relative ideal sheaf $\sI_{C/Q} = \sO(-a,-b)$ of a curve $C$ of type $(a,b)$
is the sheafification of the module $Sym^{b-a}(M)(-b)$.

\item Let $S = \CC[x_0,\dots, x_3]$. Show that the minimal $S$-free resolution of $\Sym^a(M)$ 
has the form 
\begin{small}
$$
0 \rTo \wedge^2 F \otimes \Sym^{a-2} G(-2)\rTo F\otimes \Sym^{a-1} G(-1)\rTo \Sym^a G%\rTo \Sym^a(M)\rTo 0
$$
\end{small}

\item Show that if $C\subset \PP^3$ is a curve of type $(a, b)$ with $a\leq b$ then
 there is a free resolution of a module
whose sheafification is $\sI_{C/\PP^3}$ that is a mapping cone of the map of complexes: 
\begin{tiny}
$$
\begin{diagram}[size=2em]
                                                       0&\rTo&0&\rTo& \wedge^2 F &\rTo^{\wedge^2 q}&\wedge^2 G\\
 &&\uTo&&\uTo&&\uTo\\
 0 &\rTo& \wedge^2 F \otimes \Sym^{b-a-2} G(-b-2) &\rTo &F\otimes \Sym^{b-a-1} G(-b-1) &\rTo& \Sym^{b-a} G(-b)
\end{diagram}
$$
\end{tiny}

\item Conclude that the deficiency module of a curve of type $(a, b)$ with $a\leq b$ is the cokernel of a map
$$
\wedge^2 F^* \otimes \Sym^{b-a-2} (G^*)(b+2) \lTo F^*\otimes \Sym^{b-a-1} G^*(b+1)
$$
and thus, after choosing a basis of $F = S^2$, may be identified as the sheafification of the cokernel of
$$
\Sym^{b-a-2} (G^*)(b+2) \lTo F\otimes \Sym^{b-a-1} G^*(b+1)
$$
where the map is the action of $F$ on $\wedge G^*$ via the map $q: F\to G$.
\end{enumerate}
 Hint: Show that the sheafification of the graded module $\coker q$ is $\sO_Q(1,0)$ by computing the vanishing locus
of the two sections corresponding to the generators of the module. Imitate the proof of Theorem~\ref{E-N} to prove
the exactness of the given resolution of $\Sym^a(M)$. See also \cite[Appendix ****]{Eisenbud1995}. \fix{citation}.
\end{exercise}
%footer for separate chapter files

\ifx\whole\undefined
%\makeatletter\def\@biblabel#1{#1]}\makeatother
\makeatletter \def\@biblabel#1{\ignorespaces} \makeatother
\bibliographystyle{msribib}
\bibliography{slag}

%%%% EXPLANATIONS:

% f and n
% some authors have all works collected at the end

\begingroup
%\catcode`\^\active
%if ^ is followed by 
% 1:  print f, gobble the following ^ and the next character
% 0:  print n, gobble the following ^
% any other letter: normal subscript
%\makeatletter
%\def^#1{\ifx1#1f\expandafter\@gobbletwo\else
%        \ifx0#1n\expandafter\expandafter\expandafter\@gobble
%        \else\sp{#1}\fi\fi}
%\makeatother
\let\moreadhoc\relax
\def\indexintro{%An author's cited works appear at the end of the
%author's entry; for conventions
%see the List of Citations on page~\pageref{loc}.  
%\smallbreak\noindent
%The letter `f' after a page number indicates a figure, `n' a footnote.
}
\printindex[gen]
\endgroup % end of \catcode
%requires makeindex
\end{document}
\else
\fi





%$$
%\vbox{\offinterlineskip %\baselineskip=15pt
%\halign{\strut\hfil# \ \vrule\quad&# \ &# \ &# \ &# \ &# \ &# \ 
%&# \ &# \ &# \ &# \ &# \ 
%\cr
%degree&\cr
%\noalign {\hrule}
%0&1&--&--&--&--&--&--\cr
%1&--&17&46&45&4&--&--\cr
%2&--&--&--&--&25&18&4\cr
%\noalign{\bigskip}
%\omit&\multispan{8}{\bf Conjectural shape of $F_\bullet$}\cr
%\noalign{\smallskip}
%}}
%$$
%
%
%\centerline{\scriptsize
%\begin{tabular}{r|ccc} 
%$j\backslash i$&0&1&2\\ 
%\hline 
%0&1&$-$&$-$\\ 
%1&$-$&3&2\\ 
%\end{tabular}}
%
%
