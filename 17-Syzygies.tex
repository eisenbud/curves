%header and footer for separate chapter files

\ifx\whole\undefined
\documentclass[12pt, leqno]{book}
\usepackage{graphicx}
\usepackage{eps-to-pdf}
\input style-for-curves.sty
%\input sl-macros.sty
\usepackage{hyperref}
\usepackage{showkeys} %This shows the labels.
\usepackage{msribib}
\usepackage{pdfpages}
\usepackage{draftwatermark}
\SetWatermarkText{DRAFT:\ \today}
\SetWatermarkScale{2}
\SetWatermarkColor[gray]{0.9}

%\usepackage{SLAG,msribib,local}
%\usepackage{amsmath,amscd,amsthm,amssymb,amsxtra,latexsym,epsfig,epic,graphics}
%\usepackage[matrix,arrow,curve]{xy}
%\usepackage{graphicx}
%\usepackage{diagrams}
%
%%\usepackage{amsrefs}
%%%%%%%%%%%%%%%%%%%%%%%%%%%%%%%%%%%%%%%%%%
%%\textwidth16cm
%%\textheight20cm
%%\topmargin-2cm
%\oddsidemargin.8cm
%\evensidemargin1cm
%
%%%%%%Definitions
%\input preamble.tex
%\input style-for-curves.sty
%\def\TU{{\bf U}}
%\def\AA{{\mathbb A}}
%\def\BB{{\mathbb B}}
%\def\CC{{\mathbb C}}
%\def\QQ{{\mathbb Q}}
%\def\RR{{\mathbb R}}
%\def\facet{{\bf facet}}
%\def\image{{\rm image}}
%\def\cE{{\cal E}}
%\def\cF{{\cal F}}
%\def\cG{{\cal G}}
%\def\cH{{\cal H}}
%\def\cHom{{{\cal H}om}}
%\def\h{{\rm h}}
% \def\bs{{Boij-S\"oderberg{} }}
%
%\makeatletter
%\def\Ddots{\mathinner{\mkern1mu\raise\p@
%\vbox{\kern7\p@\hbox{.}}\mkern2mu
%\raise4\p@\hbox{.}\mkern2mu\raise7\p@\hbox{.}\mkern1mu}}
%\makeatother

%%
%\pagestyle{myheadings}

%\input style-for-curves.tex
%\documentclass{cambridge7A}
%\usepackage{hatcher_revised} 
%\usepackage{3264}
   
\errorcontextlines=1000
%\usepackage{makeidx}
\let\see\relax
\usepackage{makeidx}
\makeindex
% \index{word} in the doc; \index{variety!algebraic} gives variety, algebraic
% PUT a % after each \index{***}

\overfullrule=5pt
\catcode`\@\active
\def@{\mskip1.5mu} %produce a small space in math with an @

\title{A Chapter from ``The Practice of Algebraic Curves"}
\author{\copyright David Eisenbud and Joe Harris}
%%\includeonly{%
%0-intro,01-ChowRingDogma,02-FirstExamples,03-Grassmannians,04-GeneralGrassmannians
%,05-VectorBundlesAndChernClasses,06-LinesOnHypersurfaces,07-SingularElementsOfLinearSeries,
%08-ParameterSpaces,
%bib
%}

\date{\today}
%%\date{}
%\title{Curves}
%%{\normalsize ***Preliminary Version***}} 
%\author{David Eisenbud and Joe Harris }
%
%\begin{document}

\begin{document}
\maketitle

\pagenumbering{roman}
\setcounter{page}{5}
%\begin{5}
%\end{5}
\pagenumbering{arabic}
\tableofcontents
\fi

\chapter{Appendix: Syzygies and Geometry}\label{syzygy chapter}

\section{Betti tables}

\fix{to do: regularity; meaning of Cohen-Macaulayness. characterization of Gorenstein; mention Boij-Soederberg. }

 In this Chapter we will return to the case of standard graded rings $R$; that is, $R$ is an algebra over the field $k := R_0$, and is generated by the finite dimensional vector space $R_1$. In particular, $R$ is a homomorphic image of 
$S = k[R_1]$, and thus $R$ is Noetherian. Throughout this section, $S$ will denote a standard graded polynomial ring.

A module over any ring can be specified by its generators and relations. But Hilbert saw that there was something to be gained by studying the relation on the relations, and the relations on those, and so on---in short, the syzygies of the module.
In the case of a graded module $M$ over a graded ring $R$, we may take a minimal free resolution
$$
0\lTo M \lTo F_0 \lTo^d F_1\lTo^d\cdots
$$
that is graded, with differential $d$ of degree 0. Recall that $R(-a)$ denotes a rank one free graded module generated in degree $a$ (so that $R(-a)_0 = R_a$ and more generally $R(a)_{b} = R_{a+b}$). We may identify each $F_i$ with a direct sum of the form 
$\bigoplus_j R(-a_j)^{\gb_{i,j}}$. 
Theorem~\ref{uniqueness} holds in this case as well, so that 
the numbers $\beta_{i,j}$ depend only on the module $M$, as is also obvious from the formula $\Tor_i(M,k) = \bigoplus_j k(-a_j)^{\gb_{i,j}}$.
We sometimes refer to $\beta_{i,j}$ informally as the \emph{number of $i$-th syzygies of degree $j$}; more properly, it is the number of minimal generators of degree $j$ required by the module of $i$-th syzygies of $M$.

For convenience, the graded Betti numbers are usually displayed in a compact \emph{Betti table}, with the nonzero $\beta_{i,j}$ appearing in the $i$-th column and the $(j-i)$-th row: \footnote{The reason for the initially non-intuitive choice $j-i$ instead of $j$ is that, for the resolution $\FF$ to be minimal, it is necessary that if $\beta_{i,j}\neq 0$, then some $\beta_{i-1,k}\neq 0$ for some $k<j$. Thus shift by $-j$ makes the diagram more compact.) This useful convention seems to go back to Bayer and Stillman and their work on the program Macaulay in the 1980s. In those days of small displays, efficiency was even more important than now.}
%\setcounter{MaxMatrixCols}{13}
%\begin{small}
%$$
%\begin{matrix}
%j \backslash i     &0&1&2&3&4&5&6&7&8&9&10&11\\ \hline
%%\text{total:}&1&55&359&1211&2602&3824&3954&2889&1466&493&99&9 \\
%\text{0: }\vline &1&.&\text{.}&\text{.}&\text{.}&\text{.}&\text{.}&\text{.}&\text{.}&\text{.}&\text{.}&\text{.}\\
%\text{1: }\vline &.&55&320&930&1688&2060&1728&987&368&81&8&\text{.}\\
%\text{2: }\vline &\text{.}&\text{.}&39&280&906&1736&2170&1832&1042&384&83&8\\
%\text{3: }\vline &\text{.}&\text{.}&\text{.}&1&8&28&56&70&56&28&8&1\\
%\end{matrix}
%$$
%\end{small}
\setcounter{MaxMatrixCols}{13}
\begin{small}
$$
\begin{matrix}
j \backslash i     &0&1&2\\ \hline
%\text{total:}&1&55&359&1211&2602&3824&3954&2889&1466&493&99&9 \\
\text{0 }\vline &\gb_{0,0}&\gb_{1,1}&\gb_{2,2}&\cdots&\\
\text{1 }\vline &\gb_{0,1}&\gb_{1,2}&\gb_{2,3}&\cdots&\\
\text{2 }\vline &\gb_{0,2}&\gb_{1,3}&\gb_{2,4}&\cdots&\\
\vdots&\vdots&\vdots&\cdots&\\
%\text{3: }\vline &\text{.}&\text{.}&\text{.}&1&8&28&56&70&56&28&8&1\\
\end{matrix}
$$
\end{small}
The rows of the betti table thus correspond to \emph{linear strands} of the resolution---sequences of the free summands where the maps between them are represented by matrices of linear forms. We shall soon see some examples and the length of the linear strands plays a central role in Green's conjecture. For the sake of simplicity we usually replace some 0 entries with $-$.

\noindent We sometimes speak of the \emph{Betti table of a variety $X\subset \PP^{n}$}, by which we will mean the Betti table of the minimal free resolution
of the homogeneous coordinate ring $S_{X}$ of $X$, as a module over the homogeneous coordinate ring of $\PP^{n}.$ 


\section{Regularity of modules and sheaves}
The number of the last nonzero column of the Betti table of the resolution of a graded $S$-module $M$ is the projective dimension of $M$. The number of the last row, on the other hand is an important invariant that we have not
yet introduced: the (Castelnuovo-Mumford) \emph {regularity} of $M$. More formally,
\begin{definition}
Let $t_{i}(M) = \max \{j \mid \Tor^{S}_{i}(k,M)_{j} \neq 0\}$. The regularity of $M$ is
$$
\max_{i \geq 0} t_{i}(M)-i.
$$
\end{definition}

As defined above, the regularity of $M$ is obviously an upper bound for the degrees of a minimal generating
set of $M$, which is simply $t_{0}(M)$. One reason the regularity is important is that there is a different expression for the regularity in terms that do not seem to involve the generators directly:

\begin{theorem}
 Let $s_{i}(M) = max\{j \mid H^{i}_{\gm}(M)_{j} \neq 0\}$. The regularity of $M$ is equal to
 $$
\max_{i \geq 0} t_{i}(M)+i.
$$
\end{theorem}

Note the change of sign; see **** for a proof using local duality. The notion of regularity was introduced by Mumford in the study of sheaves, where it has a slightly simpler expression:

\begin{definition}
A $\sF$ be a sheaf on $\PP^{n}$ is \emph{$m$-regular} if $H^{i}(\sF(m-i) = 0$ for all $i>0$. The \emph{regularity} of $\sF$ is the minimal $m$ for which $\sF$ is $m$-regular.
\end{definition}

Here is it automatic that if $\sF$ is $m$-regular, then it is $m'$-regular for $m'>m$. Note that the regularity
of the sheaf $\tilde M$ associated to a graded $S$-module $M$ ignores $H^{0}_{\gm}(M)$ and 
$H^{1}_{\gm}(M)$; but otherwise the definitions are related by the translation from local to global
cohomology, and indeed the regularity of
$M$ is equal to the the regularity of the associated sheaf $\tilde M$ if $\depth M \geq 2$. Most applications of
regularity involve the following result:
\begin{theorem}
 If $\sF$ is a sheaf on $\PP^{n}$ that is $m$-regular, then $\sF(m)$ is generated by its global sections.
\end{theorem}

The regularity of a projective variety $X\subset \PP^{n}$ is  defined to be the regularity of the ideal sheaf of $X$. A striking result of Gruson-Lazarsfeld-Peskine~\cite{} relates this to more elementary notions:

\begin{theorem}\label{GLP}
 Let $C$ be a reduced, irreducible nondegenerate curve of degree $d$ in $\PP^{n}$. The regularity of $C$ is $\leq d-n+2$, with equality if and only if $C$ is smooth and rational and either $d=n$ or $d=n+1$ or $C$ has a $d-n+3$-secant line. 
 \end{theorem}
 
Since the regularity is an upper bound for the degrees of the minimal generators of the ideal of $C$, it is clear that this result is best possible if $C$ has a $d-1$-secant line. For smooth curves, Gruson-Lazarsfelf-Peskine also prove the converse (and of course this condition implies that the curve is rational, since projetion from the line is an isomorphism with $\PP^{1}$.)
For a further treatment of these ideas, see for example~\cite{geomsyz}.

\section{Examples of Betti tables}
Two simple ideas sometimes help with the computation of Betti tables. First of all:

\begin{proposition}\label{reduction modulo a nzd}
Let $R$ be a local or standard graded polynomial ring, and let 
$M$ be an $R$-module with minimal $R$-free resolution $\FF$.
If $f\in R$ is a nonzerodivisor on both $R$ and $M$, then the minimal $\overline R:=R/(f)$-free resolution of
$M/fM = \overline R \otimes_{R}M$ is $\overline R \otimes_{R}\FF$, and the Hilbert function of $M/fM$ is
the first difference of that of $M$:  
$$
H_{M/fM}(t) = H_{M}(t) - H_{M}(t-1).
$$
\end{proposition}
\begin{proof}
 The homology of $\overline R \otimes_{R}\FF$ is $\Tor^{R}(\overline R,M)$, which can also be computed as the homology of the tensor product of $M$ with the $R$-free resolution of $R/(f)$. Since $f$ is a nonzerodivisor on $R$, this is
$$
0\to R(-1)\rTo^{f} R \rTo R/(f) \to 0.
$$
It follows that the homology of $\overline R \otimes_{R}\FF$ is the same as the homology of
$$
(0\to R(-1)\rTo^{f} R)\otimes_{R}M  = 0\to M(-1)\rTo^{f} M.
$$
Since $f$ is a nonzerodivisor on $M$, we see that $H_{i}( \overline R \otimes_{R}\FF) = 0$ for $i>0$, showing that $\overline R \otimes_{R}\FF$
is a free resolution of $M/(f)M$. The minimality is obvious. The formula for the Hilbert functions follows from the same exact sequences.
\end{proof}

The second idea is related to the step-by-step construction of a minimal resolution.
\begin{proposition}\label{zero implications}
If $\beta_{i,j}$ are the graded Betti numbers of an $S$-module, and $\beta_{i,j} = 0$ for all $j\leq j_{0}$ then
$\beta_{i+1, j+1} = 0$ for all $j\leq j_{0}$ as well.
\end{proposition}

Note that $\beta_{i,j}$ and $\beta_{i+1,j+1}$ are consecutive elements of the same row of the Betti
table, so the Proposition says that a zero entry in the $i$-position of all the rows of the table at and above
a certain spot implies that all successive entries
of those rows are zero as well.

\begin{proof}
The table represents the summands of a minimal free resolution, and minimality means that the maps in the resolution are represented by matrices whose entries have strictly positive degree. Thus if
$F_{i}$ has no summand isomorphic to $S(-j)$ for $j\leq j_{0}$ it follows that there can be no summand
of $F_{i+1}$ isomorphis to $S(-j-1)$ for the same range of $j$.
\end{proof}

The third idea is connected to Hilbert's original application of the Syzygy Theorem the computation of the Hilbert function of a module as a (finite!) alternating sum of the Hilbert functions of the (easy to compute) Hilbert functions of the free modules in the resolution. If the graded Betti numbers of $M$ are $\{\beta_{i,j}\}$, then clearly
\begin{align*}
 H_{M}(t) &= \sum_{i}(-1)^{i}\sum_{j}\beta_{i,j}H_{S(-j-i)}\\
 &= \sum_{i}(-1)^{i}\sum_{j}\beta_{i,j}{n-j+t\choose n}
\end{align*}
Reversing the order of summation, we note that $H_{M}(t)$ can be computed from the Betti table as the 
alternating sum, with appropriate binomial coefficients, of the $t$-th anti-diagonal of the Betti table. 

Reversing the flow of information, suppose that the Hilbert function $H_{M}(t)$ of a finitely generated
graded module $M$ is known. To simplify the notation, set $m_{t} := H_{M}(t)$. Since the Hilbert function of the residue field $k$ has values $(1,0,0,\dots)$, and the $S$-free resolution of $k$ is the
Koszul complex $\KK$, with Betti table
$$
\begin{matrix}
j \backslash i     &0&1&2\\ \hline
%\text{total:}&1&55&359&1211&2602&3824&3954&2889&1466&493&99&9 \\
\text{0 }\vline &1&n+1&{n+1\choose 2}&\dots\\
\text{1 }\vline &-&-&-&\dots\\
\end{matrix},
$$
we can produce a (generally not finitely generated) module 
$$
M' := \bigoplus_{-N}^{\infty}k(-t)^{m_{t}}
$$
with the same Hilbert function, and resolution a corresponding direct sum of shifted Koszul complexes. 
A priori this tells us nothing about the Betti table of $M$. However,
there is a flat deformation $\sM$ over $k[z]$ of the module $M$ to the trivial module 
$\oplus_{t}k(-t)^{m_{t}}$
obtained by pulling back $M$ along the map
$$
k[x_{0}\dots,x_{n}] \to k[x_{0},\dots,x_{n}]: \quad x_{i} \mapsto zx_{i}
$$
and taking the flat limit at $z=0$. 
The Betti table of the fiber of $\sM$ over $z=0$ is thus
\begin{align*}
B_{0} =  \begin{matrix}
j \backslash i     &0&1&2\\ \hline
%\text{total:}&1&55&359&1211&2602&3824&3954&2889&1466&493&99&9 \\
\text{\kern 5pt t-1}\vline &\cdots&\cdots&\cdots&\cdots\\
\text{ \kern 7pt t }\vline &m_{t}&m_{t}(n+1)&m_{t}{n+1\choose 2}&\dots\\
\text{t+1}\vline &\cdots&\cdots&\cdots&\cdots\\
\end{matrix}
\end{align*}
Such a flat deformation corresponds to a deformation of resolutions, as well, and one may deduce:

\begin{proposition}\cite{Peeva}\label{cancellation}
With notation as above, the Betti table $B$ of $M$ is obtained from the array $B_{0}$ by successively cancelling pairs of terms along the anti-diagonals; that is $B$ is derived from $B_{0}$ by repeated moves of the form
\begin{align*}
\begin{matrix}
j \backslash i     &j-1&j&j+1\\ \hline
%\text{total:}&1&55&359&1211&2602&3824&3954&2889&1466&493&99&9 \\
\text{\kern 19pt}\vline &\cdots&\cdots&\cdots&\cdots\\
\text{ \kern 7pt t }\vline &\dots&\dots&b&\dots\\
\text{t+1}\vline &\dots&c&\dots&\dots\\
\text{\kern 19pt}\vline &\cdots&\cdots&\cdots&\cdots\\
\end{matrix}
\quad \mapsto\quad
\begin{matrix}
j \backslash i     &j-1&j&j+1\\ \hline
%\text{total:}&1&55&359&1211&2602&3824&3954&2889&1466&493&99&9 \\
\text{\kern 19pt}\vline &\cdots&\cdots&\cdots&\cdots\\
\text{ \kern 7pt t }\vline &\dots&\dots&b-1&\dots\\
\text{t+1}\vline &\dots&c-1&\dots&\dots\\
\text{\kern 19pt}\vline &\cdots&\cdots&\cdots&\cdots\\
\end{matrix}
\end{align*}
\end{proposition}

 A couple of examples will help absorb these ideas.

%\begin{example}[A point in $\PP^3$]
%Any point in $\PP^3$ is the intersection of 3 hyperplanes, so it has Betti table:
%\begin{small}
%$$
%\begin{matrix}
%j \backslash i &0&1&2&3\\ \hline
%\text{0 }\vline &1&3&3&1\\
%\text{1 }\vline &-&-&-&-\\
%\end{matrix}
%$$
%\end{small}
%Note that the matrices of the differentials in the resolution---the Koszul complex on 3 linear forms---are all linear; again, this is reflect in the fact that all the 
%nonzero entries of the table are on one horizontal line.
%\end{example}

\begin{example}\label{3 points in P2}
As a first example, let $X = \{(1,0,0), (0,1,0), (0,0,1)$, a set of three non-collinear points in $\PP^2$. Since $X$ does not lie in a line, the Hilbert function of $S_{X}$ begins $1,3$. Since $S_{X}$ is reduced, the Hilbert function is non-decreasing and since $\deg X = 3$ it must continue $1,3,3,3,\dots$. The linear form 
$\ell :=x_{0}+x_{1}+x_{2}$ does not vanish on any of the points, so it is a nonzerodivisor on $S_{X}$. By Proposition~\ref{reduction modulo a nzd}, the Betti table of $X$ is the same as that of 
$S_{X}/(\ell)$, regarded as a module over $\overline S := k[x_{0}, x_{1}]$. The Hilbert function of $S_{X}/(\ell)$ is the first difference 
$$
1-0,\ 3-1,\ 3-3,\dots = 1,2,0,\dots
$$
of the Hilbert function of $S_{X}$. Thus the Betti table of $S_{X}$ is the same as that of the square of the maximal ideal in $\overline S$. By Proposition~\ref{cancellation} if must be obtained from the table
\begin{small}
$$
\begin{matrix}
j \backslash i &0&1&2&3\\ \hline
\text{0 }\vline &1&2&1\\
\text{1 }\vline &2&4&2&-\\
\end{matrix}
$$
\end{small}
by successive cancellations along anti-diagonals. Since $I_{X}$ does not contain a linear form, we see from Proposition~\ref{zero implications} that the first row
must become $1,0,0$ after cancellation, and thus the second row becomes
$0,3,2$; that is, the Betti table of $X$ is
\begin{small}
$$
\begin{matrix}
j \backslash i     &0&1&2\\ \hline
%\text{total:}&1&55&359&1211&2602&3824&3954&2889&1466&493&99&9 \\
\text{0 }\vline &1&-&-\\
\text{1 }\vline &-&3&2\\
\text{2 }\vline &-&-&-\\
\end{matrix}
$$
\end{small}
The fact that the 3 and the 2 are on the same line is reflects the fact that the 2 syzygies of the ideal $(x,y)^{2}$ are linear; this is a linear strand of the resolution. 

To check this, we observe that $I_{X}$ indeed contains the three quadrics $x_{0}x_{1}, x_{0}x_{2}, x_{1}x_{2}$.
Since $ \ell_i(\ell_j\ell_k) = \ell_j(\ell_k\ell_i) = \ell_k(\ell_i\ell_j)$ we also see the
 two linear syzygies among these forms. 
 
 Since we already know the Betti table, we see that the resolution
 of $S_{X}$ must be
$$
0\lTo S/I\lTo S
\lTo^{\begin{pmatrix}
\ell_2\ell_3&\ell_1\ell_3&\ell_1\ell_2
\end{pmatrix}}
 S(-1)^3
 \lTo^{\begin{pmatrix}
  \ell_1&0\\
  -\ell_2&\ell_2\\
 0&-\ell_3
 \end{pmatrix}}
 S(-2)^2
 \lTo 0
$$
Note that the
ideal $I$ can be written as the ideal of $2\times 2$ minors of the matrix
$$
 {\begin{pmatrix}
  \ell_1&0\\
  -\ell_2&\ell_2\\
 0&-\ell_3
 \end{pmatrix}}.
$$
The resolution above is thus a special case of the Eagon-Northcott complex, described more generally in Chapter~\ref{14- canonical curves}
\end{example}

\begin{example}
 Next, consider the twisted cubic curve $C\subset \PP^{3}$, the image of $\PP^{1}$ under the linear series
 $(s^{3}, s^{2}t, st^{2}, t^{3})\subset H^{0}(\sO_{\PP^{1}}(3))$. Write $S = k[x_{0},\dots,x_{3}]$ for the homogeneous coordinate ring of $\PP^{3}$ and $S_{C} = S/I_{C}$ for the homogeneous coordinate ring of $C$.

As we saw in Chapter ****, the ideal of forms vanishing on the twisted cubic is generated by three quadrics
$q_{1}, q_{2}, q_{3}$ that are the $2\times 2$ minors of the
$$
\phi := \begin{pmatrix}
 x_{0}&x_{1}\\
 x_{1}&x_{2}\\
 x_{2}&x_{3}
\end{pmatrix}.
$$
As discussed in Chapter~\ref{canonical curves} the $S$-free resolution of $S_{C}$ is an Eagon-Northcott complex, in this this case
$$
0 \rTo S^{2}(-3) \rTo^{\phi} S^{3}(-2) \rTo^{
\begin{pmatrix}
q_{1}&q_{2}&q_{3} 
\end{pmatrix}
} S.
$$

Thus the Betti table of the twisted cubic is the same as that of the 3 non-collinear points in $\PP^{2}.$

To understand this from another point of view, note that since $I_{C}$ is a prime ideal, $x_{3}$ (or any variable) is a nonzerodivisor on $S_{C}$. Proposition~\ref{reduction modulo a nzd} shows that the minimal free resolution of $C$ has the same Betti table as the minimal free resolution of $S_{C}/(x_{3}) = S/(I_{c}+(x_{3})$:


However, this is not quite all that needs to be said: certainly $I_{C}+(x_{3})$ is an ideal defining the hyperplane section $x_{3}= 0$ of $C$, but to conclude from  Proposition~\ref{reduction modulo a nzd} that the Betti table of $C$ is the same as that of the hyperplane section, we need to know that $I_{C}+(x_{3})$ is saturated.

By the Auslander-Buchsbaum formula, the depth of $S_{C}$ is 2---that is, $S_{C}$ is a Cohen-Macaulay ring. It follows that the depth of $S_{C}/(x_{3})$ is 1, and this shows that $I_{c}+(x_{3})$ is a saturated ideal completing the circle of ideas.
\end{example}

\begin{exercise}
 Let $C$ be the nonsingular rational quartic $\PP^{3}$, the image of $\PP^{1}$ under the linear series
 $(s^{4}, s^{3}t, st^{3}, t^{4})\subset H^{0}(\sO_{\PP^{1}}(3))$. Show that the general hyperplane section of $C$  is the intersection of two quadrics, while the ideal of $C$ has just one quadric generator. Conclude that
  that the Betti table of $C$ is not the same as the Betti table of the hyperplane section, and thus that
  $S_{C}$ is not Cohen-Macaulay. (in fact, the Betti table of $S_{C}$ is
\begin{small}
$$
\begin{matrix}
j \backslash i &0&1&2&3\\ \hline
\text{0 }\vline &1&-&-&-\\
\text{1 }\vline &-&1&-&-\\
\text{2 }\vline &-&3&4&1\\
\end{matrix}
$$
\end{small}
 which we see that the projective dimension of $S_{C}$ is 3, so by the Auslander-Buchsbaum theorem the depth is only 1.)
\end{exercise}

%\begin{example}\label{canonical in P3}
%For example, the minimal free resolution of the homogeneous coordinate ring $S_{C}$ of a canonical curve $C$ of genus 4 in $\PP^{3}$ whose ideal is generated by a quadric $q$ and a cubic $f$, as a module over the homogeneous coordinate ring $S = \CC[x_{0},\dots,x_{3}]$ of $\PP^{3}$, is the Koszul complex
%\small
%$$
%S \lTo^{
%\begin{pmatrix}
%  q& f
%\end{pmatrix}}
%S(-2) \oplus S(-3) \lTo^{
%\begin{pmatrix}
%f\\-q 
%\end{pmatrix}
%}
%S(-5)\lTo 0.
%$$
%\normalsize
%since
%$q$ and $f$ are relatively prime, and thus form a regular sequence
%Thus it has  Betti table:
%
%\setcounter{MaxMatrixCols}{13}
%\begin{small}
%$$
%\begin{matrix}
%j \backslash i     &0&1&2\\ \hline
%%\text{total:}&1&55&359&1211&2602&3824&3954&2889&1466&493&99&9 \\
%\text{0 }\vline &1&-&-\\
%\text{1 }\vline &-&1&-\\
%\text{2 }\vline &-&1&-\\
%\text{3 }\vline &-&-&1\\
%\end{matrix}
%$$
%\end{small}
%\end{example}

\begin{example}
 Consider a set $X$ of 7 points in linearly general position $\PP^3$. 
Castelnuovo's Theorem~\ref{castelnuovo} shows that any 6 points in linearly general position lie on a unique twisted cubic, so the first distinction one might make among sets of 7 points is whether the 7th point lies on the twisted cubic too.

As we saw in \ref{****}, any set of $2n+1$ points in linearly general position in $\PP^{n}$ imposes $2n+1$ conditions on quadrics, so in any case $I_{X}$ requires exactly 3 quadratic minimal generators, $q_{1},q_{2},q_{3}$ and the Hilbert function of $S_{X}$ begins $1,4,7,?$. Since $S_{X}$ is reduced, the Hilbert function is non-decreasing, and since the eventual value must be $\deg X = 7$, the function must be
$1,4,7,7,\dots$, and the ideal contains $20-7 = 13$ cubics. From this information we can work out the whole Betti table, as follows. 

First, by Proposition~\ref{reduction moduo nzd} the Betti table of $R := S_{X}/(x_{3})$ is the same as that of $S_{X}$. Write $R = \overline S/I$, where $\overline S = k[x_{0},x_{1},x_{2}]$. The Hilbert function of $R$ is determined by the exact sequence
$$
0\rTo S_{X}(-1)\rTo^{x_{3}} S_{X} \rTo R \rTo 0
$$
as the first difference function of that of $S_{X}$, namely
$$
1 -0, 4-1, 7-4, 7-7,\dots = 1,3,3,0,\dots;
$$
that is, $I$ is generated (not minimally) by the 3 quadrics and a total of 10 cubics. By Propositions~\ref{reduction modulo a nzd} and \ref{cancellation}, the Betti table of $X$ must be obtained by successive cancellations
from the table
\begin{small}
$$
\begin{matrix}
j \backslash i &0&1&2&3\\ \hline
\text{0 }\vline &1&3&3&1\\
\text{1 }\vline &3&9&9&3\\
\text{2 }\vline &3&9&9&3\\
\text{3 }\vline &-&-&-&-\\
\end{matrix}
$$
\end{small}
Moreover, since $X$ is not contained in a hyperplane, the zero-th row, after cancellation, must become
\begin{small}
 $
\begin{matrix}
\text{0 }\vline &1&0&0&0\\
\end{matrix},
$
\end{small}
so we can begin by making this cancellation. We also know that $I_{X}$ has exactly 3 quadratic generators, necessitating another cancellation, and we see that the actual Betti table of $X$ must be obtained by
cancelling from
\begin{small}
$$
\begin{matrix}
j \backslash i &0&1&2&3\\ \hline
\text{0 }\vline &1&0&0&0\\
\text{1 }\vline &0&3&8&3\\
\text{2 }\vline &0&9&9&3\\
\text{3 }\vline &-&-&-&-\\
\end{matrix}
$$
\end{small}


Now three general quadrics define a complete intersection of 8 points, and it follows that in the case of 7 general points $q_{1},q_{2},q_{3}$ form a complete intersection with resolution the Koszul complex, having Betti table:
\begin{small}
$$
\begin{matrix}
j \backslash i &0&1&2&3\\ \hline
\text{0 }\vline &1&-&-&-\\
\text{1 }\vline &-&3&-&-\\
\text{2 }\vline &-&-&3&-\\
\text{3 }\vline &-&-&-&1\\
\end{matrix}.
$$
\end{small}
In this case there are no linear relations among $q_{1},q_{2},q_{3}$, so the 8 and 3 in row 1 of this table must 
cancel completely, giving
\begin{small}
$$
\begin{matrix}
j \backslash i &0&1&2&3\\ \hline
\text{0 }\vline &1&0&0&0\\
\text{1 }\vline &0&3&0&0\\
\text{2 }\vline &0&1&6&3\\
\text{3 }\vline &-&-&-&-\\
\end{matrix}
$$
\end{small}
Since no further cancellation is possible, this must be the Betti table of a  set $X$ of 7 general points.

On the other hand, if the points lie on a twisted cubic, then $q_{1}, q_{2}, q_{3}$ generate the ideal of the twisted cubid, and as we  see from the Betti table in Example~\ref{}, these have 2 linear relations, which are themselves independent. It follows that row 1 of the Betti table of $X$ must cancel to become 
\begin{small}
 $
\begin{matrix}
\text{0 }\vline &1&0&0&0\\
\end{matrix}.
$
\end{small}
Thus the Betti table of $X$ in this case is
\begin{small}
$$
\begin{matrix}
j \backslash i &0&1&2&3\\ \hline
\text{0 }\vline &1&0&0&0\\
\text{1 }\vline &0&3&2&0\\
\text{2 }\vline &0&7&9&3\\
\text{3 }\vline &-&-&-&-\\
\end{matrix}
$$
\end{small}

\end{example}

%footer for separate chapter files

\ifx\whole\undefined
\makeatletter\def\@biblabel#1{#1]}\makeatother
\gdef\urlhook{\url}
\bibliography{slag}
\bibliographystyle{msribib}


%%%% EXPLANATIONS:

% f and n
% some authors have all works collected at the end

\catcode`\^\active
%if ^ is followed by 
% 1:  print f, gobble the following ^ and the next character
% 0:  print n, gobble the following ^
% any other letter: print letter
\makeatletter
\def^#1{\ifx1#1f\expandafter\@gobbletwo\else
        \ifx0#1n\expandafter\expandafter\expandafter\@gobble\else#1\fi\fi}
\makeatother
\let\moreadhoc\relax
\def\indexintro{%An author's cited works appear at the end of the
%author's entry; for conventions
%see the List of Citations on page~\pageref{loc}.  
%\smallbreak\noindent
The letter `f' after a page number indicates a figure, `n' a footnote.}
\printindex[gen]
%requires makeindex
\end{document}
\else
\fi

