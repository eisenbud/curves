%header and footer for separate chapter files

\ifx\whole\undefined
\documentclass[12pt, leqno]{book}
\usepackage{graphicx}
\input style-for-curves.sty
\usepackage{hyperref}
\usepackage{showkeys} %This shows the labels.
%\usepackage{SLAG,msribib,local}
%\usepackage{amsmath,amscd,amsthm,amssymb,amsxtra,latexsym,epsfig,epic,graphics}
%\usepackage[matrix,arrow,curve]{xy}
%\usepackage{graphicx}
%\usepackage{diagrams}
%
%%\usepackage{amsrefs}
%%%%%%%%%%%%%%%%%%%%%%%%%%%%%%%%%%%%%%%%%%
%%\textwidth16cm
%%\textheight20cm
%%\topmargin-2cm
%\oddsidemargin.8cm
%\evensidemargin1cm
%
%%%%%%Definitions
%\input preamble.tex
%\input style-for-curves.sty
%\def\TU{{\bf U}}
%\def\AA{{\mathbb A}}
%\def\BB{{\mathbb B}}
%\def\CC{{\mathbb C}}
%\def\QQ{{\mathbb Q}}
%\def\RR{{\mathbb R}}
%\def\facet{{\bf facet}}
%\def\image{{\rm image}}
%\def\cE{{\cal E}}
%\def\cF{{\cal F}}
%\def\cG{{\cal G}}
%\def\cH{{\cal H}}
%\def\cHom{{{\cal H}om}}
%\def\h{{\rm h}}
% \def\bs{{Boij-S\"oderberg{} }}
%
%\makeatletter
%\def\Ddots{\mathinner{\mkern1mu\raise\p@
%\vbox{\kern7\p@\hbox{.}}\mkern2mu
%\raise4\p@\hbox{.}\mkern2mu\raise7\p@\hbox{.}\mkern1mu}}
%\makeatother

%%
%\pagestyle{myheadings}

%\input style-for-curves.tex
%\documentclass{cambridge7A}
%\usepackage{hatcher_revised} 
%\usepackage{3264}
   
\errorcontextlines=1000
%\usepackage{makeidx}
\let\see\relax
\usepackage{makeidx}
\makeindex
% \index{word} in the doc; \index{variety!algebraic} gives variety, algebraic
% PUT a % after each \index{***}

\overfullrule=5pt
\catcode`\@\active
\def@{\mskip1.5mu} %produce a small space in math with an @

\title{Personalities of Curves}
\author{\copyright David Eisenbud and Joe Harris}
%%\includeonly{%
%0-intro,01-ChowRingDogma,02-FirstExamples,03-Grassmannians,04-GeneralGrassmannians
%,05-VectorBundlesAndChernClasses,06-LinesOnHypersurfaces,07-SingularElementsOfLinearSeries,
%08-ParameterSpaces,
%bib
%}

\date{\today}
%%\date{}
%\title{Curves}
%%{\normalsize ***Preliminary Version***}} 
%\author{David Eisenbud and Joe Harris }
%
%\begin{document}

\begin{document}
\maketitle

\pagenumbering{roman}
\setcounter{page}{5}
%\begin{5}
%\end{5}
\pagenumbering{arabic}
\tableofcontents
\fi

%\documentclass[12pt, leqno]{book}
%\usepackage{amsmath,amscd,amsthm,amssymb,amsxtra,latexsym,epsfig,epic,graphics}
%\usepackage[matrix,arrow,curve]{xy}
%\usepackage{graphicx}
%\usepackage{diagrams}
%%\usepackage{amsrefs}
%%%%%%%%%%%%%%%%%%%%%%%%%%%%%%%%%%%%%%%%%%
%%\textwidth16cm
%%\textheight20cm
%%\topmargin-2cm
%\oddsidemargin.8cm
%\evensidemargin1cm
%
%%%%%%Definitions
%\input preamble.tex
%\def\TU{{\bf U}}
%\def\AA{{\mathbb A}}
%\def\BB{{\mathbb B}}
%\def\CC{{\mathbb C}}
%\def\QQ{{\mathbb Q}}
%\def\RR{{\mathbb R}}
%\def\facet{{\bf facet}}
%\def\image{{\rm image}}
%\def\cE{{\cal E}}
%\def\cF{{\cal F}}
%\def\cG{{\cal G}}
%\def\cH{{\cal H}}
%\def\cHom{{{\cal H}om}}
%\def\h{{\rm h}}
% \def\bs{{Boij-S\"oderberg{} }}
%
%\makeatletter
%\def\Ddots{\mathinner{\mkern1mu\raise\p@
%\vbox{\kern7\p@\hbox{.}}\mkern2mu
%\raise4\p@\hbox{.}\mkern2mu\raise7\p@\hbox{.}\mkern1mu}}
%\makeatother
%
%%%
%%\pagestyle{myheadings}
%\date{April 30, 2018}
%%\date{}
%\title{Curves}
%%{\normalsize ***Preliminary Version***}} 
%\author{David Eisenbud and Joe Harris }
%
%\begin{document}

\chapter{Brill-Noether Theory}



\section{What linear series exist?}

In the last chapter, we established a basic correspondence between maps of curves to projective space and linear systems. The next question to ask, naturally, is ``What linear systems exist?"

There are various ways to interpret this question. Let's start by taking the question in its plain, unvarnished form---for which $g, r$ and $d$ does there exist a curve $C$ of genus $g$ and a linear system $(\cL,V)$ on $C$ of degree $d$ and dimension $r$? In this form, the answer is given for line bundles of large degree $d \geq 2g-1$ by the Riemann-Roch theorem: on any curve, there exists a linear series of degree $d \geq 2g-1$ and dimension $r$ iff $r \leq d-g$. 

\subsection{Clifford's theorem} 

Riemann-Roch still leaves open the question of what linear systems of degree $d \leq 2g-2$ may exist on a curve of genus $g$. The answer is given by the classical theorem of Clifford:

\begin{theorem}\label{Clifford}
Let $C$ be a curve of genus $g$ and $\cL$ a line bundle of degree $d \leq 2g-2$. Then
$$
r(\cL) \leq \frac{d}{2}.
$$
Moreover, if  equality holds then we must have either
\begin{enumerate}
\item $d=0$ and $\cL = \cO_C$;
\item $d = 2g-2$ and $\cL = K_C$; or
\item $C$ is hyperelliptic, and $|\cL|$ is a multiple of the $g^1_2$ on $C$.
\end{enumerate}
\end{theorem}

\begin{proof}
The proof of Clifford rests on a very basic construction and observation. 

To start, let $\cD = (\cL,V)$ and $\cE = (\cM, W)$ be two linear series on a curve $C$. By the \emph{sum} $\cD + \cE$ of $\cD$ and $\cE$, we will mean the pair 
$$
\cD + \cE = (\cL \otimes \cM, U) 
$$
where $U \subset H^0(\cL \otimes \cM)$ is the subspace generated by the image of $V \otimes W$, under the multiplication/cup product map $H^0(\cL) \otimes H^0(\cM) \to H^0(\cL \otimes \cM)$---in other words, it's the subspace of the complete linear series $|\cL\otimes \cM|$ spanned by divisors of the form $D+E$, with $D \in \cD$ and $E \in \cE$.

The observation is a simple one:
\begin{lemma}
If $\cD$ and $\cE$ are two nonempty linear series on a curve $C$, then
$$
\dim(\cD + \cE) \geq \dim \cD + \dim \cE.
$$
\end{lemma}
(To see this, we observe that to say $\dim \cD \geq m$ means exactly that we can find a divisor $D \in \cD$ containing any given $m$ points of $C$; since $\cD + \cE$ contains all pairwise sums $D + E$ with $D \in \cD$ and $E \in \cE$, we can certainly find a divisor $F \in cD + \cE$ containing any given $\dim \cD + \dim \cE$ points of $C$.)

Given this lemma, the proof of Clifford follows simply by applying it to the pair $|\cL|$ and $|K_C\otimes \cL^{-1}|$: by Riemann-Roch, we have
$$
r(K_C\otimes \cL^{-1}) = r(\cL) +g - d - 1
$$
and so we deduce that
$$
g = r(K_C) + 1 \geq r(\cL) + r(K_C\otimes \cL^{-1}) + 1 \geq 2r(\cL) +g - d;
$$
hence $r(\cL) \leq d/2$.

The proof of the second half of Clifford rests on a basic fact about the geometry of hyperplane sections of a curve in projective space; we'll defer it until we've established that fact.
\end{proof}

Combining Clifford with Riemann-Roch, we arrive at the answer to our initial question

\begin{theorem}\label{arbitrary linear series}
There exists a curve $C$ of genus $g$ and line bundle $\cL$ of degree $d$ on $C$ with $h^0(\cL) \geq r+1$ if and only if
$$
r \leq
\begin{cases}
d-g, \quad \text{if } d \geq 2g-1; \text{ and} \\
d/2,  \quad \text{if } 0 \leq d \leq 2g-2.
\end{cases}
$$
\end{theorem}

\begin{exercise}
Prove a slightly stronger version of Theorem~\ref{arbitrary linear series}: that under the hypotheses of Theorem~\ref{arbitrary linear series} there exists a \emph{complete} linear series of degree $d$ and dimension $r$.
\end{exercise}

\subsection{Castelnuovo's theorem}

Theorem~\ref{arbitrary linear series} gives a complete and sharp answer to the question originally posed: for which $d,r$ and $g$ does there exists a triple $(C,\cL,V)$ with $C$ a curve of genus $g$, $\cL$ a line bundle of degree $d$ on $C$ and $V \subset H^0(\cL)$ of dimension $r+1$. 

But maybe that wasn't the question we meant to ask! After all, we're interested in describing curves in projective space as images of abstract curves $C$ under maps given by linear systems on $C$. Observing that the linear series that achieve equality in Clifford's theorem give maps to $\PP^r$ that are 2 to 1 onto a rational curve, we might hope that we would have a different---and more meaningful---answer if we  restrict our attention to linear series $\cD = (\cL,V)$ for which the associated map $\phi_\cD$ is at least a birational embedding.  With this restriction, the question is tantamount to the

\begin{question}
What is the largest possible genus of an irreducible, nondegenerate curve $C \subset \PP^r$ of degree $d$?
\end{question}

The answer to this question is indeed quite different from the inequality provided by Theorem~\ref{arbitrary linear series}. It is the content of \emph{Castelnuovo's theorem}, which gives a sharp answer to this question. We'll sketch the derivation of the inequality here; we'll prove that it is in fact sharp and describe in detail  the curves that achieve it in Chapter~\ref{}.

To start, Castelnuovo's bound follows from a very straightforward approach: if $C$ is a curve of degree $d$ and genus $g$ in $\PP^r$, the idea is to prove successive lower bounds for the dimensions $h^0(\cO_C(m))$ of multiples of the $g^r_d$ cut on $C$ by hyperplanes. For large values of $m$, of course, the line bundle $\cO_C(m)$ is non-special, and so a lower bound on the dimension of its space of sections translates, via Riemann-Roch, into an upper bound on the genus $g$.

\begin{definition}
Let $\cL$ be any line bundle on a smooth projective variety $X$, and $D = \{p_1,\dots,p_d\}$ a collection of points of $X$. By the \emph{number of conditions imposed by $D$ on sections of $\cL$} we will mean simply the difference
$$
h^0(\cL) - h^0(\cL \otimes \cI_{D/X});
$$
that is, the codimension in $H^0(\cL)$ of the subspace of sections vanishing on $D$. More generally, if $V \subset H^0(\cL)$ is any linear system, by the number of conditions imposed by $D$ on $V$ we will mean the difference
$$
\dim(V) - \dim \left(V \cap H^0(\cL\otimes \cI_{D/X}) \right).
$$
\end{definition}
Thus, for example, if $X = \PP^r$, the number of conditions imposed by $D$ on $H^0(\cO_{\PP^r}(m))$ is the value $h_D(m)$ of the Hilbert function of $D$.
Note that the number of conditions imposed by $D$ on a linear system $V$ is necessarily less than or equal to the degree $d$ of $D$; if it is equal we say that $D$ \emph{imposes independent conditions on $V$}.

To apply this notion, suppose $C \subset \PP^r$ is an irreducible, nondegenerate curve. Let $\Gamma = C \cap H$ be a general hyperplane section of $C$. Let $V_m \subset H^0(\cO_C(m))$ be the linear series cut on $C$ by hypersurfaces of degree $m$ in $\PP^r$, that is, the image of the restriction map
$$
H^0(\cO_{\PP^r}(m)) \to H^0(\cO_C(m)).
$$
We have then a series of more or less trivial inequalities:
\begin{align*}
h^0(\cO_C(m)) - h^0(\cO_C(m-1)) & \geq \text{\# of conditions imposed by $\Gamma$ on $H^0(\cO_C(m))$} \\
&\geq \text{\# of conditions imposed by $\Gamma$ on $V_m$} \\
&\geq \text{\# of conditions imposed by $\Gamma$ on $H^0(\cO_{\PP^r}(m))$} ;
\end{align*}
in other words, the dimension $h^0(\cO_C(m))$ is bounded below by the sum
$$
h^0(\cO_C(m)) \geq \sum_{k=0}^m h_\Gamma(k).
$$

We need, in other words, a lower bound on the Hilbert function of a general hyperplane section $\Gamma$ of our curve $C$. This is turn requires that we have some knowledge of the geometry of $\Gamma$, but  in fact we don't need all that much: all we need is the basic

\begin{lemma}[general position lemma]
If $C \subset \PP^r$ is an irreducible, nondegenerate curve and $\Gamma = C \cap H$ a general hyperplane section of $C$, then the points of $\Gamma$ are in linear general position in $H \cong \PP^{r-1}$, meaning no $r$ points of $\Gamma$ lie in a hyperplane $\PP^{r-2} \subset H$.
\end{lemma}
Thus, for example, if $C \subset \PP^3$ is a space curve, no three points of $\Gamma = H \cap C$ will be collinear.

\begin{exercise}
Prove directly that a general plane section of an irreducible, nondegenerate space curve does not contain three collinear points. (Hint: estimate the dimension of the family of trisecant lines to the curve.)
\end{exercise}

The general position lemma was originally asserted by Castelnuovo. In more modern treatments, it is usually deduced as a special case of the more general

\begin{lemma}[uniform position lemma]
With $C \subset \PP^r$ and $\Gamma = C \cap H$ as above, any two subsets $\Gamma', \Gamma'' \subset \Gamma$ of the same cardinality have the same Hilbert function, i.e., impose the same number of conditions on $\cO_{\PP^{r-1}}(m)$ for all $m$.
\end{lemma}

The general position lemma is just the special case $m=1$ of the uniform position lemma. This may not seem like much information about $\Gamma$, but in fact it's all we need to prove a sharp bound! The basic (and completely elementary) statement is

\begin{proposition}
If $\Gamma \subset \PP^n$ is a collection of $d$ points in linear general position and spanning $\PP^n$, then 
$$
h_\Gamma(m) \geq \min\{d, mn+1\}
$$
\end{proposition}

\begin{proof}
Suppose first that $d \geq mn+1$, and let $p_1,\dots,p_{mn+1} \in \Gamma$ be any subset of $mn+1$ points. We want to show that $\Gamma' = \{p_1,\dots,p_{mn+1}\}$ imposes independent conditions of $H^0(\cO_{\PP^n}(m))$, that is, for any $p_i \in \Gamma'$ we can find a hypersurface $X \subset \PP^n$ of degree $m$ containing all the points $p_1,\dots, \hat{p_i},\dots,p_{mn+1}$ but not containing $p_i$.

This is easy: simply group the $mn$ points of $\Gamma' \setminus \{p_i\}$ into $m$ subsets $\Gamma_k$ of cardinality $n$; each set $\Gamma_k$ will span a hyperplane $H_k \subset \PP^n$, and we can take $X = H_1 \cup \dots \cup H_m$. 
\end{proof}

This may seem like a crude argument, but the bound derived is sharp: any collection of point $\Gamma \subset \PP^n$ lying on a rational normal curve $D \subset \PP^n$ has exactly this Hilbert function.

At this point, all that remains is to add up the lower bounds in the proposition. To this end, let $C \subset \PP^r$ be as above an irreducible, nondegenerate curve of degree $d$, and set $M = \lfloor{\frac{d-1}{r-1}}\rfloor$, so that we can write
$$
d = M(r-1) + 1 + \epsilon \quad \text{ with } \quad 0 \leq \epsilon \leq r-2.
$$
We have then
\begin{align*}
h^0(\cO_C(M)) &\geq \sum_{k=0}^M h^0(\cO_C(k)) - h^0(\cO_C(k-1)) \\
&\geq  \sum_{k=0}^M k(r-1)+1 \\
&= \frac{M(M+1)}{2}(r-1) + M + 1
\end{align*}
and similarly
$$
h^0(\cO_C(M+m)) \geq \frac{M(M+1)}{2}(r-1) + M + 1 + md.
$$
For sufficiently large $m$, the line bundle $\cO_C(M+m)$ will be nonspecial, so we can plug this in to Riemann-Roch to arrive at
\begin{align*}
g &= (M+m)d - h^0(\cO_C(M+m)) + 1 \\
&\leq (M+m)d - \bigl(  \frac{M(M+1)}{2}(r-1) + M + 1 + md \bigr) \\
& = M\bigl( M(r-1) + 1 + \epsilon \bigr) - \bigl(  \frac{M(M+1)}{2}(r-1) + M + 1 \bigr) \\
&= \frac{M(M-1)}{2}(r-1) + M\epsilon.
\end{align*}

To summarize our discussion: for positive integers $d$ and $r$, we write
$$
 d = M(r-1) + 1 + \epsilon \quad \text{ with } \quad 0 \leq \epsilon \leq r-2
$$
and set
$$
\pi(d,r) = \frac{M(M-1)}{2}(r-1) + M\epsilon.
$$
In these terms, we have proved the

\begin{theorem}[Castelnuovo's bound]
If $C \subset \PP^r$ is an irreducible, nondegenerate curve of degree $d$ and genus $g$, then
$$
g \leq \pi(d,r).
$$
\end{theorem}

We will see in Chapter~\ref{} that this is in fact sharp: for every $r$ and $d \geq r$, there do exist such curves with genus exactly $\pi(d,r)$. For now, we make a few observations:

\begin{enumerate}
\item In case $r=2$, all the inequalities used in the derivation of Castenuovo's bound are in fact equalities, and indeed we see that in this case $\pi(d,2) = \binom{d-1}{2}$ is the genus of a smooth plane curve of degree $d$.

\item In case $r=3$, we have
$$
\pi(d,3) =
\begin{cases}
\left( k - 1 \right)^2 &\text{ if $d=2k$ is even; and} \\
k(k-1) &\text{ if $d=2k+1$ is odd.}
\end{cases}
$$
In this case again, it's not hard to see the bound is sharp: these are exactly the genera of curves of bidegree $(k,k)$ and $(k+1,k)$ on a quadric surface $Q \cong \PP^1 \times \PP^1 \subset \PP^3$.
\item In general, we see that for fixed $r$ asymptotically
$$
\pi(d,r) \sim \frac{d^2}{2(r-1)}.
$$
\end{enumerate}


\begin{exercise}
Show that with $C$ as above, the line bundle $\cO_C(M)$ is nonspecial. (We will see in Section~\ref{} that this is sharp; that is, there exist such curves $C$ with $\cO_C(M-1)$ special).
\end{exercise}

\section{Brill-Noether theory}

\subsection{Basic questions addressed by Brill-Noether theory}

In the last section, we restricted our attention to the linear series most of interest to us: those corresponding to embeddings of our curve in projective space (or at any rate birational embeddings) and their limits. But there is one other respect in which Castelnuovo theory fails to address a basic concern: the curves with linear systems achieving Castelnuovo's bound are, like hyperelliptic curves, very special. (In fact, we'll see in Section~\ref{**} that in general they are even rarer than hyperelliptic curves.) That is, if we were to pick a curve $C$ of genus $g$ ``at random" (we'll make this notion more precise when we describe the moduli space of curves in Chapter~\ref{**}), we would still have no idea what linear systems existed on $C$ or how they behaved.

Brill-Noether theory addresses exactly this issue: it asks, ``what linear series exist on \emph{all} curves of a given genus?" To start with, we'll give the crudest form of the theorem:

\begin{theorem}\label{basic BN}
Fix non-negative integers $g, r$ and $d$. It is the case that every curve of genus $g$ possesses a linear series of degree $d$ and dimension $r$ if and only if
$$
\rho(g,r,d) := g - (r+1)(g-d+r) \geq 0.
$$
\end{theorem}

In the following sections, we'll see why we might naively expect this to be the case, and we'll also describe some of the many refinements and strengthenings of the theorem (we will be able to give better versions in later chapters, after we have, for example, introduced the schemes parametrizing linear systems on a given curve). A proof of the existence half of the theorem (the ``if" part of the statement) may be found in \cite{3264}; and we will give in the concluding chapter of this book a relatively simple proof of the nonexistence part (the ``only if"). In the meantime, we'll mention here the special case $r=1$:

\begin{corollary}
If $C$ is any curve of genus $g$, then $C$ admits a rational function of degree $d$ for some positive $d \leq \lceil \frac{g+2}{2}\rceil$.
\end{corollary}

Thus, for example, any curve of genus 2 is hyperelliptic, any curve of genus 3 or 4 is either hyperelliptic or trigonal, and so on.

\subsection{Heuristic argument leading to the statement of Brill-Noether}

The Brill-Noether theorem, as we'll see, is a far-reaching description of the linear series to be found on a general curve. It starts, though, with a relatively simple dimension count---one that was first carried out almost a century and a half ago.

To set this up, let $C$ be a smooth projective curve of genus $g$, and $D = p_1 + \dots + p_d$ a divisor on $C$. We'll assume here the points $p_i$ are distinct; the same argument (albeit with much more complicated notation) can be carried out in general.

When does the divisor $D$ move in an $r$-dimensional linear series? Riemann-Roch gives an answer: it says that $h^0(D) \geq r+1$ if and only if the vector space $H^0(K-D)$ of 1-forms vanishing on $D$ has dimension at least $g-d+r$---that is, if and only if the  evaluation map
$$
H^0(K) \to H^0(K|_D) = \oplus K_{p_i}
$$
has rank at most $d-r$. 

We can represent this map by a $g \times d$ matrix. Choose a basis $\omega_1,\dots,\omega_g$ for the space $H^0(K)$ of 1-forms on $C$; choose an analytic open neighborhood $U_j$ of each point $p_j \in D$ and choose a local coordinate $z_j$ in $U_j$ around each point $p_j$, and write
$$
\omega_i = f_{i,j}(z_j)dz_j
$$
in $U_j$. We will have $r(D) \geq r$ if and only if the  matrix-valued function
$$
A(z_1,\dots,z_d) = 
\begin{pmatrix}
f_{1,1}(z_1) & f_{2,1}(z_1) & \dots & f_{g,1}(z_1) \\
f_{1,2}(z_2) & f_{2,2}(z_2) & \dots & f_{g,2}(z_2) \\
\vdots & \vdots &  & \vdots \\
f_{1,d}(z_d) & f_{2,d}(z_d) & \dots & f_{g,d} (z_d)
\end{pmatrix}
$$
has rank $d-r$ or less at $(z_1,\dots,z_d) = (0,\dots,0)$.

The point is, we can think of $A$ as a matrix valued function in the open set $U = U_1 \times U_2 \times \dots \times U_d \subset C_d$; and for divisors $D \in U$, we have $r(D) \geq r$ if and only if $\rank(A(D)) \leq d-r$. Now, in the space $M_{d,g}$ of $d \times g$ matrices, the subset of matrices of rank $d-r$ or less has codimension $r(g-d+r)$, and so we might naively expect that the locus of divisors with $r(D) \geq r$ would have dimension $d - r(g-d+r)$. At the same time, if any divisor of degree $d$ with $h^0(D) \geq r+1$ exists, then there must be at least an $r$-dimensional family of them; so we'd suspect that such divisors exist only if
$$
d - r(g-d+r) \; \geq \; r,
$$
which is exactly the Brill-Noether statement.

As we indicated, Theorem~\ref{basic BN} represents only the most bare-bones version of Brill-Noether. The full statement describes as well the space parametrizing linear series $g^r_d$ on a general curve $C$---it says that it has dimension $\rho(g,r,d)$, is smooth and irreducible when $\rho > 0$---and also the geometry of $C$ as mapped to projective space by a general such $g^r_d$. The problem is, all these versions involve the existence of a parameter space for linear series on a given curve (which we'll see how to construct in Chapter~\ref{Jacobians chapter}), as well as the existence of a moduli space $M_g$ parameterizing abstract curves of genus $g$ (which we'll discuss further in Chapter~\ref{}). For this reason, we will have to defer the full statement of Brill-Noether to that chapter. In the meantime, though, we'll see in the next chapter how the theory plays out in the case of curves of low genus.


\input footer.tex