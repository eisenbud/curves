%header and footer for separate chapter files

\ifx\whole\undefined
\documentclass[12pt, leqno]{book}
\usepackage{graphicx}
\input style-for-curves.sty
\usepackage{hyperref}
\usepackage{showkeys} %This shows the labels.
%\usepackage{SLAG,msribib,local}
%\usepackage{amsmath,amscd,amsthm,amssymb,amsxtra,latexsym,epsfig,epic,graphics}
%\usepackage[matrix,arrow,curve]{xy}
%\usepackage{graphicx}
%\usepackage{diagrams}
%
%%\usepackage{amsrefs}
%%%%%%%%%%%%%%%%%%%%%%%%%%%%%%%%%%%%%%%%%%
%%\textwidth16cm
%%\textheight20cm
%%\topmargin-2cm
%\oddsidemargin.8cm
%\evensidemargin1cm
%
%%%%%%Definitions
%\input preamble.tex
%\input style-for-curves.sty
%\def\TU{{\bf U}}
%\def\AA{{\mathbb A}}
%\def\BB{{\mathbb B}}
%\def\CC{{\mathbb C}}
%\def\QQ{{\mathbb Q}}
%\def\RR{{\mathbb R}}
%\def\facet{{\bf facet}}
%\def\image{{\rm image}}
%\def\cE{{\cal E}}
%\def\cF{{\cal F}}
%\def\cG{{\cal G}}
%\def\cH{{\cal H}}
%\def\cHom{{{\cal H}om}}
%\def\h{{\rm h}}
% \def\bs{{Boij-S\"oderberg{} }}
%
%\makeatletter
%\def\Ddots{\mathinner{\mkern1mu\raise\p@
%\vbox{\kern7\p@\hbox{.}}\mkern2mu
%\raise4\p@\hbox{.}\mkern2mu\raise7\p@\hbox{.}\mkern1mu}}
%\makeatother

%%
%\pagestyle{myheadings}

%\input style-for-curves.tex
%\documentclass{cambridge7A}
%\usepackage{hatcher_revised} 
%\usepackage{3264}
   
\errorcontextlines=1000
%\usepackage{makeidx}
\let\see\relax
\usepackage{makeidx}
\makeindex
% \index{word} in the doc; \index{variety!algebraic} gives variety, algebraic
% PUT a % after each \index{***}

\overfullrule=5pt
\catcode`\@\active
\def@{\mskip1.5mu} %produce a small space in math with an @

\title{Personalities of Curves}
\author{\copyright David Eisenbud and Joe Harris}
%%\includeonly{%
%0-intro,01-ChowRingDogma,02-FirstExamples,03-Grassmannians,04-GeneralGrassmannians
%,05-VectorBundlesAndChernClasses,06-LinesOnHypersurfaces,07-SingularElementsOfLinearSeries,
%08-ParameterSpaces,
%bib
%}

\date{\today}
%%\date{}
%\title{Curves}
%%{\normalsize ***Preliminary Version***}} 
%\author{David Eisenbud and Joe Harris }
%
%\begin{document}

\begin{document}
\maketitle

\pagenumbering{roman}
\setcounter{page}{5}
%\begin{5}
%\end{5}
\pagenumbering{arabic}
\tableofcontents
\fi



\chapter{Linear Series}

\begin{center}
\emph{Another damned thick book! Always scribble, scribble, scribble! Eh, Mr. Gibbon?} --- \scriptsize{Prince William Henry, upon receiving the second  volume of The History of the Decline and Fall of the Roman Empire from the author.}
\end{center}



\section{Divisors, line bundles and linear systems}


The basic objects of study in this book are smooth, connected projective algebraic curves over an algebraically closed field $K$ of characteristic 0. Dropping any of these hypotheses leads us to many fascinating questions, some of which we'll discuss in \ref{****}; but before we can get into those we have to start with the basic case.

And before we can deal with even that, we have to say what we mean by an ``algebraic curve." There are two common meanings for that phrase, the 19th century and the 20th century (there may in the fullness of time be a 21st century viewpoint, but it has not yet emerged). 

In the 19th century, what mathematicians meant by the term ``algebraic curve," or ``algebraic variety" in general, was simply the zero locus of a collection of homogeneous polynomials in $\PP^n$. The 20th century, by contrast, introduced a new notion: that of an \emph{abstract variety} and later an \emph{abstract scheme}: a set with the additional structure of a topology and a sheaf of regular functions, locally isomorphic to the zero locus of polynomials, and independent of any particular choice of embedding in projective space.

There is a natural analogy with the development of group theory: in the 19th century, a group was a subset of $GL_n$ closed under matrix multiplication and inversion. The 20th century saw the introduction of the abstract group, again a set with additional structure and defined without regard to a particular embedding in $GL_n$. While the two notions are largely equivalent---in particular, the introduction of the notion of abstract group did not in itself prove any new theorems---the change in viewpoint represented a huge shift in how we thought about groups.

In particular, it placed a spotlight on the bridge between the two viewpoints; that is, the subject of representation theory: the problem of how a given abstract group can be embedded in (or more generally, mapped to) a matrix group.

In the same way, the introduction of the notion of abstract curve transformed the way we study algebraic curves, and in particular it put our focus on the problem: given a curve $C$, to describe all the ways in which $C$ can be embedded in (or more generally, mapped to) projective space. In this, a central role is played by the notion of a \emph{linear system}, or \emph{linear series} (we will use the two terms interchangeably) on a curve. In this book, accordingly, we start by introducing this fundamental notion.

In the remainder of this chapter, we'll introduce the basic notation and terminology of linear series on curves; then we'll summarize the main results---the Riemann-Roch theorem, and Serre duality---that govern their behavior. In the following chapter, we'll describe three more advanced theorems on linear series: Clifford's theorem, the Brill-Noether theorem and Castelnuovo's bound; then, in Chapter~\ref{personalities chapter}, we'll see how it all plays out for curves of low genus ($\leq 6$). We would encourage the reader to look ahead to Chapter~\ref{personalities chapter} to see examples of how we use the set-up to describe curves in practice.

\subsection{The history of linear series in four pages}\label{linear series 1}

One of the basic facts about projective varieties is that \emph{any regular function on a connected projective curve $C$ is constant} (such a function gives a regular map $C \to \AA^1$; since the image is again a projective variety, it can only be a point).  In order to have nonconstant functions, accordingly, we have to look at rational functions. But the field of all rational functions is too large for most purposes, and so from a very early stage geometers looked at rational functions with \emph{bounded singularities}: they specified a finite collection of points $p_1,\dots,p_n \in C$, a corresponding collection of integers $m_i$, and considered the space of all rational functions $f$ on $C$, regular on $C \setminus \{p_1,\dots,p_n\}$  and satisfying

\begin{equation}\label{divisor condition}
\ord_{p_i}(f) \geq -m_i.
\end{equation}

To formalize this, they defined a \emph{divisor} on $C$ to be a formal finite linear combination
$$
D = \sum m_i \cdot p_i
$$
of points. The \emph{degree} of a divisor $D = \sum m_i \cdot p_i$ is defined to be the sum $\sum m_i$ of its coefficients, and $D$ is said to be \emph{effective} if all $m_i \geq 0$. 

If $f$ is a rational function on $C$, we associate to $f$ its divisor $(f)$ of zeroes and poles: that is,
$$
(f) \; := \; \sum_{p \in C} \ord_p(f)\cdot p.
$$
One key observation is that \emph{the divisor of a rational function has degree 0}---in other words, a rational function has the same number of zeroes and poles, counting multiplicity. 

Note that, in modern language, an effective divisor can be viewed simply as a subscheme of the curve $C$---the subscheme defined by the ideal $\prod \cI_{p_i}^{m_i}$, or in other words the unique subscheme with support $\{p_i\}$ and multiplicity $m_i$ at $p_i$.

In any event, the key point is that  the vector space $\cL(D)$ of rational functions satisfying~(\ref{divisor condition}) is finite-dimensional; the dimension of $\cL(D)$ was denoted $\ell(D)$. 


But this is somewhat inefficient: for example, if $f$ is any rational function, we have an isomorphism $\cL(D+E) \cong \cL(D)$ obtained by multiplying by $f$. Accordingly, geometers defined an equivalence relation on the group of divisors, calling two divisors $D$ and $D'$ \emph{linearly equivalent} if their difference $E = D - D'$ was the divisor of a rational function; the group of equivalence classes of divisors is called the \emph{Picard group} $\Pic(C)$ of $C$. Note that since the degree of the divisor $(f)$ of a rational function $f$ on $C$ is 0, linearly equivalent divisors have the same degree.

As we headed into the 20th century, some further tweaking was called for in order to avoid the awkwardness of equivalence relations. To any divisor $D = \sum m_ip_i$ on $C$, geometers associated a \emph{coherent sheaf} $\cO_C(D)$, defined by
$$
\cO_C(D)(U) = \{ f \in K(U) \mid \ord_{p_i}(f) \geq -m_i \quad \forall p_i \in U \}.
$$
This is a locally free sheaf of rank 1, also called an \emph{invertible sheaf} or a \emph{line bundle}. Two divisors are linearly equivalent iff the associated sheaves are isomorphic, so that the Picard group $\Pic(C)$ can be thought of simply as the group of line bundles on $C$. In this setting, the vector space $\cL(D)$ is simply the space of global sections $H^0(\cO_C(D))$ and accordingly the notation $\ell(D)$ was replaced by $h^0(\cO_C(D))$ or, for brevity, $h^0(D)$. 

What this formalism is good for is describing maps of a curve to projective space. To start with the classical approach, suppose $f : C \to \PP^r$ is a map. We can choose a hyperplane $H \cong \PP^{r-1} \subset \PP^r$ with complement $\PP^r \setminus H \cong \AA^r$; the map 
$$
C \setminus f^{-1}(H) \to \AA^r
$$
is then given simply by an $r$-tuple $(f_1,\dots,f_r)$ of regular functions on $C \setminus f^{-1}(H)$. Moreover, if we consider $f^{-1}(H)$ as a divisor $D$ on $C$, we can think of the functions $f$ as elements of $\cL(D)$.

All this requires a choice of hyperplane $H$, as well as an identification of the complement $\PP^r \setminus H$ with $\AA^r$---an unnecessary evil. We can do better using the language of line bundles: if we let $\cL = f^*\cO_{\PP^r}(1) = \cO_C(D)$ be the pullback to $C$ of the line bundle on $\PP^r$ associated to the divisor $H$, we can think of the $f_i$ as sections of $\cL$, so that associated to the map $f : C \to \PP^r$ we have a line bundle $\cL$ on $C$ and an $(r+1)$-dimensional vector space $V \subset H^0(\cL)$ of global sections of $\cL$ (including the constant function 1). Conversely, if we are given a line bundle $\cL$ on $C$ and an $(r+1)$-dimensional vector space $V \subset H^0(\cL)$ of global sections of $\cL$ without common zeroes, we can choose a basis $\sigma_0,\dots,\sigma_r$ for $V$; we then get a map 

\begin{align*}
C &\to \PP^r \\
p &\mapsto [\sigma_0(p),\dots,\sigma_r(p)].
\end{align*}
The point is,  the values $\sigma_i(p)$ are elements of a one-dimensional vector space---the fiber of $\cL$ at $p$---so that the vector $[\sigma_0(p),\dots,\sigma_r(p)]$ is well-defined up to scalars.

If we were even more choice-averse and wanted to describe the map $f$ associated to a pair $(\cL, V)$ without choosing coordinates on $\PP^r$, we could describe $f$ as the map

\begin{align*}
C &\to \PP(V^*) \\
p &\mapsto H_p
\end{align*}
where $H_p = \{\sigma \in V \mid \sigma(p) = 0\}$ is the hyperplane in $V$ of sections $\sigma \in V$ vanishing at $p$.

All of this leads us to the fundamental

\begin{definition}
A \emph{linear system} of degree $d$ and dimension $r$ on a curve $C$ is a pair $(\cL, V)$ with $\cL \in \Pic(C)$ a line bundle of degree $d$ on $C$ and $V \subset H^0(\cL)$ an $(r+1)$-dimensional vector space of sections of $\cL$.
\end{definition}

By way of language, a one-dimensional linear system is called a \emph{pencil}; a two-dimensional linear system is a \emph{net}, and (less commonly) a three-dimensional linear system is called a \emph{web}.

A point $p \in C$ is called a \emph{basepoint} of the linear system $(\cL,V)$ if $\sigma(p)=0$ for all $\sigma \in V$; the linear system $(\cL,V)$ is called \emph{basepoint-free} is the sections $\sigma \in V \subset H^0(\cL)$ have no common zeroes. The linear system is said to be \emph{complete} if $V = H^0(\cL)$; in this case it is sometimes denoted $|\cL|$ (or, if $\cL \cong \cO_C(D))$, it may be denoted $|D|$). In view of this, if $D$ is any divisor on $C$ we write $r(D)$ for the dimension of the complete linear series $|D|$; that is, $r(D) = h^0(\cO_C(D)) - 1$. Finally, a linear system of dimension 1 is called a \emph{pencil}, and a linear system of dimension 2 is called a \emph{net}.

With all this said, we have the 

\begin{proposition}
There is a natural bijection between the set of nondegenerate maps $\phi : C \to \PP^r$ modulo $PGL_{r+1}$, and base-point-free linear systems of dimension $r$ on $C$.
\end{proposition}

Here ``nondegenerate" means the image of the map $\phi$ is not contained in any hyperplane. In this correspondence, the degree of the linear system corresponds to the \emph{projective degree} of the map $\phi$; that is, the cardinality of the preimage $\phi^{-1}(H)$ of a general hyperplane $H \subset \PP^r$.

Via this correspondence, the statements about the geometry of a map $\phi : C \to \PP^r$ can be formulated as statements about the relevant linear systems. We will see this in many instances throughout this book; for a first example, we have the

\begin{proposition}\label{very ample}
Let $\cL$ be a line bundle on $C$. The complete linear system $|\cL|$ is base-point-free iff
$$
h^0(\cL(-p)) = h^0(\cL) - 1 \quad \forall p \in C;
$$
and in this case the associated map $\phi_\cL$ is an embedding iff
$$
h^0(\cL(-p-q)) = h^0(\cL) - 2 \quad \forall p, q \in C.
$$
\end{proposition} 

\begin{proof}[``Proof"]
The statement $h^0(\cL(-p-q)) = h^0(\cL) - 2$ for $p \neq q$ implies that $\phi_\cL(p) \neq \phi_\cL(q)$; in case $p=q$ it implies that the differential $d\phi_\cL$ is injective at $p$.
\end{proof}

We can also relate the geometry of the map associated to a incomplete linear system $V \subset H^0(\cL)$ to the geometry of the map associated to the complete linear system $|\cL|$. In general, if $V \subset W \subset H^0(\cL)$ are a pair of nested linear systems, we have a linear map $W^* \to V^*$ dual to the inclusion $V \hookrightarrow W$, and a corresponding linear projection $\pi : \PP W^* \dashrightarrow \PP V^*$, with indeterminacy locus the subspace $\PP(Ann(V)) \subset \PP W^*$. In this case, we have 
$$
\phi_V = \pi \circ \phi_W;
$$
that is, we have the diagram 

\begin{diagram}
& & \PP W^* \\
& \ruTo^{\phi_W} & \dDashto_\pi \\
C & \rTo^{\phi_V} & \PP V^*.
\end{diagram}

Note that in this case, given that $W$ is base-point-free, the condition that $V$ be base-point-free is equivalent to saying that the center $\PP(Ann(V))$ of the projection map $\pi$ is disjoint from $\phi_W(C)$.

By way of language, we will say that a curve $C \subset \PP^r$ embedded by a complete linear series is \emph{linearly normal}; this is equivalent to saying that the restriction map
$$
H^0(\cO_{\PP^r}(1)) \to H^0(\cO_{C}(1))
$$
is surjective, which is in turn equivalent to saying that $C$ is not the regular  projection of a nondegenerate curve $\tilde C \subset \PP^{r+1}$.

\begin{exercise}
Extend the statement of Proposition~\ref{very ample} to incomplete linear systems; that is, prove that the map associated to a linear system $(\cL, V)$ is an embedding iff
$$
\dim\big( V \cap H^0(\cL(-p-q))\big) = \dim V - 2 \quad \forall p, q \in C.
$$
\end{exercise}

For another example of the relationship between linear series on curves and maps of curves to projective space, consider a smooth curve $C \subset \PP^r$ embedded in projective space, and assume that $C$ is linearly normal. If $\phi : C \to C$ is any automorphism, we can ask whether $\phi$ is induced by an automorphism of $\PP^r$; in other words, does there exist an automorphism $\Phi : \PP^r \to \PP^r$ such that $\Phi(C) = C$ and $\Phi|_C = \phi$? The answer is expressed in the following exercise.

\begin{exercise}\label{projective automorphism}
In the circumstances above, the automorphism $\phi$ is induced by an automorphism of $\PP^r$ if and only if $\phi$ carries the line bundle $\cO_{C}(1)$ to itself; that is, $\phi^*(\cO_{C}(1)) = \cO_{C}(1)$.
\end{exercise}

For example, consider the image of $\PP^1$ in $\PP^n$ under the map given by the complete linear system $|\cO_{\PP^1}(n)|$; this is called the \emph{rational normal curve}. Since there is a unique line bundle of each degree $n$ on $C$, and the curve is linearly normal, we see that \emph{every automorphism of a rational normal curve $C \subset \PP^n$  is projective}.


\subsection{The canonical divisor class/line bundle}

There is one divisor class/line bundle on an arbitrary curve $C$ that plays a particularly important role in the geometry of $C$: the \emph{canonical} divisor class $K_C$. This is the class of the divisor $(\eta)$ of any rational 1-form $\eta$ on $C$; that is,
$$
(\eta) = \sum_{p \in C} \ord_p(\eta)\cdot p.
$$
Since the ratio of any two (nonzero) 1-forms is a rational function on $C$, this is a well-defined divisor class. Alternatively, in the world of line bundles, the canonical divisor class corresponds simply to the cotangent bundle of $C$, which is also called the \emph{canonical line bundle} and denoted $\omega = \omega_C$.

\subsubsection{A plea for forbearance} 

The introduction of the language and notation of line bundles into the classical theory of divisors and linear systems greatly clarifies many constructions and theorems. It also creates a notational nightmare: in the world of divisors, the group law is given by taking the sum of divisors, suggesting that we denote this operation by $+$; in the world of line bundles, the corresponding operation is tensor product, which seems like it should be denoted $\times$ or $\otimes$.

On top of that, we very often want to consider sections of a given line bundle $\cL$ with bounded singularities: if $D = \sum m_ip_i$ is a divisor, we define the line bundle $\cL(D)$ to be the sheaf of rational sections $\sigma$ of $\cL$ satisfying $\ord_{p_i}(\sigma) \geq -m_i$ for all $i$; as a line bundle, this is the same as $\cL \otimes \cO_C(D)$.

All of this means, unfortunately, that it's extremely difficult to be completely consistent with the notation for divisors and line bundles on a curve, and we ask the reader's forbearance if at times we mix up our conventions.




\section{Genus, Riemann-Roch and Serre Duality}

\subsection{The genus of a curve}

Any discussion of smooth projective curves pretty much has to start with the notion of the \emph{genus} of a curve, the sole discrete continuous invariant of curves. The most natural and intuitive is topological: if $C \subset \PP^r$ is a smooth curve, we can think of it as a submanifold of the complex projective space $\PP^r(\CC)$ with the classical topology; as such, it is a compact, oriented surface, and so its topological type is described by its genus---informally, the ``number of holes."

**** obligatory picture ****

This is not a fully satisfactory definition for a couple reasons: for one thing, it doesn't a priori apply to curves over fields other than $\CC$; and in any case it's not clear how we can relate this invariant to the algebra of the curve. There is a fix, however. To start with, we can relate the topological genus of a curve directly to its topological Euler characteristic: we have
$$
g(C) = 1 - \frac{\chi_{top}(C)}{2}.
$$
At the same time, by the Hopf index theorem, the topological Euler characteristic is the degree of the tangent sheaf, which is the dual of the canonical bundle; thus
$$
g(C) = \frac{\deg(K_C)}{2} + 1,
$$
which we can adopt as the definition of $g(C)$.

This may seem an awkward characterization of the genus, and it is; but there are various other characterizations, whose equivalence with the topological requires some more machinery to establish. We will give some here, and then use the tools established in the following section to prove equivalence.

\begin{enumerate}

\item\label{genus 1forms} Another characterization of the genus of $C$ is simple to state, though the proof that it's equivalent to the preceding ones is non-trivial (it will be a special case of the \emph{Riemann-Roch formula}, described in the following section): the genus of a curve $C$ is simply the dimension of the vector space of regular 1-forms on $C$.

\item The genus of a curve $C$ may also be characterized in terms of the Euler characteristic $\chi(\cO_C) = h^0(\cO_C) - h^1(\cO_C)$ of the structure sheaf $\cO_C$: since $h^0(\cO_C) = 1$, we have
$$
g = 1 - \chi(\cO_C).
$$

\item\label{genus Hilbert} Finally, if $C \subset \PP^r$ is embedded in projective space as a curve of degree $d$, its Hilbert polynomial will be
$$
p_C(m) = dm - g + 1,
$$
which can also be used as a characterization of the genus: $g = 1 - p_C(0)$.

\end{enumerate}

There are two basic facts from which the equivalence of these various characterizations of the genus follow: the \emph{Riemann-Roch formula}, and \emph{Kodaira-Serre duality}; we'll discuss these now.

\subsection{The Riemann-Roch formula}

The classical Riemann-Roch formula is simple to state: if $C$ is a smooth, connected projective curve of genus $g$, and $D$ a divisor of degree $d$ on $C$, it says that
$$
h^0(D) = d - g + 1 + h^0(K_C - D).
$$

There are a number of general consequences. To start with, if we take $D=0$, this tells us that $h^0(K) = g$, proving the characterization~(\ref{genus 1forms}) above. Another example: since $h^0(D) = 0$ for any divisor $D$ of negative degree, we have the

\begin{corollary}\label{nonspecial RR}
For any divisor of degree $d \geq 2g-1$, we have
$$
h^0(D) = d - g + 1.
$$
\end{corollary}

As a consequence of this, we can apply Proposition~\ref{very ample} to conclude the

\begin{corollary}\label{degree 2g+1 embedding}
Let $D$ be a divisor of degree $d$ on a smooth, connected projective curve of genus $g$. If $d \geq 2g$, the complete linear series $|D|$ is base point free; and if $d \geq 2g+1$ the associated map $\phi_D : C \to \PP^{d-g}$ is an embedding.
\end{corollary}

We also use Riemann-Roch in the simple case of Corollary~\ref{nonspecial RR} to determine the Hilbert polynomial of a projective curve. To do this, let $C \subset \PP^r$ be a smooth curve of degree $d$ and genus $g$, and consider the exact sequence of sheaves
$$
0 \rTo \cI_{C/\PP^r}(m) \rTo \cO_{\PP^r}(m) \rTo \cO_C(m) \rTo 0
$$
and the corresponding exact sequence
$$
 H^0(\cO_{\PP^r}(m)) \rTo^{\rho_m} H^0(\cO_C(m)) \rTo H^1(\cI_{C/\PP^r}(m)) \rTo 0.
$$
The \emph{Hilbert function} $h_C$ of $C$  is defined by
$$
h_C(m) = \rank(\rho_m);
$$
 since $H^1(\cI_{C/\PP^r}(m)) = 0$ for large $m$, this is equal to $h^0(\cO_C(m))$, which by Riemann Roch equals $md-g+1$, again for large $m$. Thus, the Hilbert polynomial of $C \subset \PP^r$ is $p_C(m) = dm-g+1$, establishing the characterization~(\ref{genus Hilbert}) above.
 
 Note that the Riemann-Roch formula does not give us a formula for the dimension $h^0(D)$ of a complete linear series $|D|$ on a curve $C$ in terms of the degree of $D$ and the genus of $C$, on account of the ``error term" $h^0(K_C - D)$. Given the formula, though, it's natural to identify the classes of divisors $D$ for which $h^0(K_C - D)>0$; these are called \emph{special divisors}, or \emph{special divisor classes}. Classically, the dimension $h^0(K_C-D) = h^1(D)$ was called the \emph{superabundance} of $D$: the idea was that a divisor of degree $d$ had, at a minimum, $d-g+1$ sections and $h^1(D)$ represented the number of ``extra " sections. (Interestingly, even though the introduction of cohomology was still almost a century away, the ranks of cohomology groups $h^1$ had classical names, often involving the term superabundance---a premonition of the Riemann-Roch theorem in general.

\subsection{Serre duality}

In general, if $\cF$ and $\cG$ are coherent sheaves on a scheme $X$, we have for every $i$ and $j$ a cup product map
$$
H^i(\cF) \otimes H^j(\cG) \to H^{i+j}(\cF \otimes \cG).
$$

Suppose now that $C$ is a smooth, projective curve, and that $D$ is any divisor on $C$. Serre duality says first of all that
$$
h^1(K) = 1
$$
and then that the cup product map
$$
H^1(D) \otimes H^0(K-D) \to H^1(K)
$$
is a perfect pairing; that is, we have a natural isomorphism
$$
H^1(D) = H^0(K-D)^*.
$$

\subsection{A partial proof}

Proving either the Riemann-Roch formula, in the form given above, or Serre duality, requires a fair amount of work. As it happens, though, if we combine them we arrive at the statement
$$
h^0(D) - h^1(D) = d-g+1
$$
or in other words, for any line bundle $\cL$ of degree $d$ on $C$,
$$
\chi(\cL) = d-g+1
$$
which is pretty easy to prove. To see this, observe that for any line bundle $\cL$ on $C$ and any point $p \in C$ we have an exact sequence of sheaves
$$
0 \to \cL(-p) \to \cL \to \cL_p \to 0.
$$
It follows that $\chi(\cL(-p)) = \chi(\cL) - 1$, so that Riemann-Roch for $\cL$ is equivalent to Riemann-Roch for $\cL(-p)$. Since any divisor can be obtained from 0 by adding and subtracting points, the Riemann-Roch formula for an arbitrary $\cL$ follows from the special case $\cL = \cO_C$.


\section{The canonical map}

Given the central role played by the canonical divisor class, it is natural to look at the geometry of the map $\phi_K : C \to \PP^{g-1}$ associated to the complete canonical series $|K|$. To do this, we first have to single out a special class of curves:

\begin{definition}
A curve $C$ of genus $g \geq 2$ is said to be \emph{hyperelliptic} if there exists a map $f : C \to \PP^1$ of degree 2; equivalently, if there exists a line bundle $\cL$ on $C$ of degree 2 with $h^0(\cL) = 2$.
\end{definition}

The basic theorem about the canonical map is then

\begin{proposition}
Let $C$ be a smooth, projective curve of genus $g\geq 2$. The canonical map $\phi_K : C \to \PP^{g-1}$ is an embedding if and only if $C$ is not hyperelliptic.
\end{proposition}

\begin{proof}
This is just a direct application of Proposition~\ref{very ample} to the Riemann-Roch formula. To see that $\phi_K$ is an embedding, we have to show that for any pair of points $p, q \in C$ we have
$$
h^0(K_C(-p-q)) = h^0(K_C)-2 = g-2.
$$
Applying Riemann-Roch, we see that this is the case if and only if $h^0(\cO_C(p+q)) = 1$ for any $p,q \in C$, which is the case if and only if $C$ is not hyperelliptic.
\end{proof}

Note that if $C$ is hyperelliptic, the map $\phi_K$ factors through the degree 2 map $\pi : C \to \PP^1$: if $\{p,q\} \subset C$ is a fiber of this map, we have $h^0(\cO_C(p+q)) = 2$ and hence $\phi_K(p) = \phi_K(q)$. The image of the map $\phi_K$ is a nondegenerate curve of degree $g-1$ in $\PP^{g-1}$, which we will see is a \emph{rational normal curve}. This observation implies in particular that if $C$ is hyperelliptic of genus $g \geq 2$, then the line bundle $\cL$ of degree 2 with $h^0(\cL) = 2$ is in fact unique.

Hyperelliptic curves are very special, at least when $g \geq 3$ (in case $g=2$ the canonical map itself is a degree 2 map $C \to \PP^1$; all curves of genus 2 are hyperelliptic): in the family of all curves, as we'll see, they comprise a closed subvariety. In particular, the behavior of linear series and maps on a hyperelliptic curve is very different from that of series on a general curve; when we discuss the geometry of curves of low genus in the Chapter~\ref{}, we will exclude  the hyperelliptic case, and deal with this case in a separate chapter.

For non-hyperelliptic curves, however, the geometry of the canonical map, and its image, the canonical curve, are the keys to understanding the curve. We'll see this in detail in many cases in the following chapter; for now, we mention one highly useful result along these lines.

\fix{add here: canonical series on plane curves cut by $|\cO_{\PP^2}(d-3)|$; consequence that no smooth plane curve can be hyperelliptic}

\fix{maybe move initial discussion of hyperelliptic curves from Ch. 6 to a section here}

\fix{also add to this chapter: differentials on plane curves $C$, possibly with nodes or more general singularities; adjoint conditions; algorithm for determining the complete linear system associated to a divisor $D$ on $C$}

\subsection{The geometric Riemann-Roch theorem}

Let's state this first in a relatively simple case: let $C$ be a nonhyperelliptic curve, embedded in $\PP^{g-1}$ by its canonical series and let $D = p_1+\dots + p_d$ be a divisor consisting of $d$ distinct points; let $\overline D$ be the span of the points $p_i \in C \subset \PP^{g-1}$. Since the hyperplanes in $\PP^{g-1}$ containing $\{p_1,\dots,p_d\}$ correspond (up to scalars) to sections of $K_C$ vanishing at the points $p_1$, we see that
$$
h^0(K_C-D) = g - 1 - \dim \overline D.
$$
Plugging this into the Riemann-Roch formula, we arrive at the statement
$$
r(D) = d - 1 - \dim \overline D;
$$
or in other words, \emph{the dimension of the linear series $|D|$ in which the divisor $D$ moves is equal to the number of linear relations on the points $p_i$ on the canonical curve}. Thus, for example, if $D = p_1+p_2+p_3$, we see that $D$ moves in a pencil if and only if the points $p_i$ are collinear.

We can extend this statement to the case of arbitrary effectived divisors $D$ (and even hyperelliptic curves) if we define our terms correctly. To do this, suppose $f : C \to \PP^d$ is any map, and $D \subset C$ any divisor. We define the \emph{span} of the divisor $D$ under the map $f$ to be the intersection
$$
\overline{f(D)} = \bigcap_{H \mid f^{-1}(H)\supset D} H 
$$
of all hyperplanes in $\PP^d$ whose inverse image in $C$ contains $D$. In these terms, we can state the \emph{geometric Riemann-Roch} as

\begin{theorem}
If $C$ is any curve of genus $g \geq 2$,  $\phi : C \to \PP^{g-1}$ its canonical map and $D \subset C$ any effective divisor of degree $d$, then
$$
r(D) = d - 1 - \dim \overline{\phi(D)}.
$$
\end{theorem}

 \section{The Riemann-Hurwitz formula}
 
 The Riemann-Hurwitz formula does a simple thing: given a (nonconstant) map $f : C \to X$ of smooth projective curves, it relates the genera of $C$ and $X$ via the geometry of the map. The proof is likewise simple.
 
 To set it up, we first introduce the notion of \emph{ramification} of such a map. Suppose $p \in C$ is any point, with $q = f(p) \in X$ its image in $X$. In terms of a suitable choice of local coordinates $z$ on $C$ around $p$ and $w$ on $X$ around $q$, we can write the map simply as $z \mapsto w = z^m$ for some integer $m > 0$; the integer $m-1$ is then called the \emph{ramification index} of $f$ at $p$, and denoted $\ram(f,p)$. To characterize this without invoking local coordinates, we can say that for any point $q \in X$ we have
 $$
 f^{-1}(q) = \sum_{p\in C \mid f(p)=q} (\ram(f,p)+1)\cdot p
 $$
 as subschemes (or, equivalently, divisors) on $C$. The \emph{ramification divisor} of $f$ is defined to be the divisor
 $$
 R = \sum_{p \in C} \ram(f,p)\cdot p \; \in \;  \Div(C);
 $$
 the \emph{branch divisor} is the corresponding divisor
 $$
 B = \sum_{q \in X} \Big(\sum_{p \in f^{-1}(q)} \ram(f,p) \Big)\cdot q \; \in \; \Div(X).
 $$
 Note that $R$ and $B$ have the same degree $\sum_{p \in C} \ram(f,p)$. Note also that just to define the divisor $R$ requires we know that \emph{not every point of $C$ is a ramification point}; that is, $\ram(f,p) = 0$ at a general point $p \in C$, which may be false in characteristic $p > 0$; thus the argument sketched here requires the hypothesis of characteristic 0.
 
 To relate the genera of $X$ and $C$ in this situation, start by choosing any rational 1-form $\omega$ on $X$; let $\eta = f^*(\omega)$ be its pullback to $C$. For simplicity, we will assume that the zeroes and poles of $\omega$ lie outside the branch divisor $B$, so that $\omega$ will be regular and nonzero at each branch point. (We can certainly find such a form, and in any event the calculation goes through without this assumption, albeit with more complicated notation.) We know that $\deg(\omega) = 2g(B) - 2$ and $\deg(\eta) = 2g(C) - 2$; we will relate these two by describing the zeroes and poles of $\eta$ in terms of those of $\omega$.
 
 First of all, since the zeroes of $\omega$ lie outside the branch divisor $B$, for every zero of $\omega$ of multiplicity $m$ we have exactly $d$ zeroes of $\eta$, each with multiplicity $m$; and likewise for the poles of $\omega$. Meanwhile, at every point of $B$ the form $\omega$ is regular and nonzero; and by the description of the map $f$ in terms of local coordinates we see that $\eta$ will have a zero of multiplicity $\ram(f,p)$ at every point $p \in R$. In sum, then, we have
 $$
 \deg((\eta)) = d\cdot \deg((\omega)) + \sum_{p \in C} \ram(f,p) \cdot p,
 $$
 or in other words,
 $$
 2g(C) - 2 = d(2g(X)-2) + \deg R;
  $$
  this is the Riemann-Hurwitz formula. Note that if the target curve $X \cong \PP^1$ has genus 0, the formula reads
  $$
  g(C) = -d  + 1+ \frac{\deg R}{2}
  $$
 or equivalently
 $$
 \deg R = \deg B  = 2d + 2g(C) - 2.
 $$
 
 \subsection{An application}
 
 Let $V$ be the vector space of homogeneous polynomials of degree $d$ in two variables; that is, $V = H^0(\cO_{\PP^1}(d))$. In the projectivization $\PP V \cong \PP^d$, let $\Delta$ be the locus of polynomials with a repeated factor. It's not hard to see that $\Delta$ in a hypersurface; we ask, what is its degree?
 
 To answer this, let $L \subset \PP V$ be a general line---that is, a general pencil of divisors of degree $d$ on $\PP^1$. The linear system $L$ defines a map $\phi_L : \PP^1 \to \PP^1$ of degree $d$, and the locus $L \cap \Delta$ is exactly the branch divisor $B$.
 It follows then that
 $$
 \deg(\Delta) = \deg(L \cap \Delta) = \deg B = 2d-2,
 $$
 answering our question. Note that since $L$ is general, both $L\cap \Delta$ and $B$ are reduced divisors, so we can interpret the above simply as an equality of cardinality of sets; but if we keep track of multiplicities the argument goes through for an arbitrary $L \not\subset \Delta \subset \PP^d$.
 
 \section{Hyperelliptic Curves}
 
 
 This will be a somewhat anomalous section, in keeping with its subject matter. In the world of curves, hyperelliptic curves are outliers: they behave differently from other curves, and the techniques used to analyze them are correspondingly different from how we approach more general curves. As you'll see in following chapters, many theorems about curves contain the hypothesis ``non-hyperelliptic," with the corresponding result for hyperelliptic curves arrived at directly by ad hoc methods.
 
 In this section, we'll lay out the basic facts about hyperelliptic curves. There will be a further discussion of them in Chapter~\ref{ScrollsChapter}, focussing on the algebra and geometry of their projective embeddings; but the analysis here will cover most of the questions we'll be asking about curves in general in the next four chapters. 
 
 It should also be said that this section does not have to be read in sequence; if you're curious to see some of the questions we ask about curves in general (and some of the answers, and the techniques for arriving at them), you can skip ahead to the following two chapters and return to this section whenever you feel like it.
 
 \subsection{The equation of a hyperelliptic curve}
 
  By definition, a hyperelliptic curve $C$ is one admitting a degree two map $\pi : C \to \PP^1$. The local geometry of such a map is pretty simple: at all but finitely many points $p \in C$ the map $\pi$ is a local isomorphism (``local" here in the complex analytic/classical or \'etale topology, not the Zariski topology!); at any other point $p \in C$, the map is given in terms of local coordinates on $C$ and $\PP^1$ simply by $z \mapsto z^2$. In particular, both the ramification divisor and the branch divisor are reduced, meaning by the Riemann-Hurwitz formula there will be exactly $2g+2$ branch points $q_1,\dots,q_{2g+2} \in \PP^1$.
  
  In fact, we claim that \emph{there is a unique smooth projective curve $C$ expressible as a 2-sheeted cover of $\PP^1$ branched over a given collection of points $p_1,\dots,p_{2g+2} \in \PP^1$}. \fix{proof to be filled in. Why not make this assertion a theorem?}
  
  Given this, it's not hard to write down the equation of this curve: if the coordinate of the point $p_i \in \PP^1$ is $\lambda_i$, it is the smooth projective model of the affine curve 
  $$
C^\circ = \big\{ (x,y) \in \AA^2 \; \mid \; y^2 = \prod_{i=1}^{2g+2} (x - \lambda_i) \big\}.
$$ 
Note that we're choosing a coordinate $x$ on $\PP^1$ with the point $x = \infty$ at infinity not among the $q_i$, so that the pre-image of $\infty \in \PP^1$ is two points $r, s \in C$. Concretely, we see that as $x \to \infty$, the ratio $y^2/x^{2g+2} \to 1$, so that 
$$
\lim_{x \to \infty} \; \frac{y}{x^{g+1}} \; = \; \pm 1;
$$
  the two possible values of this limit correspond to the two points $r,s \in C$.
  
  It's worth pointing out that $C$ is \emph{not} simply the closure of the affine curve $C^\circ \subset \AA^2$ in either $\PP^2$ or $\PP^1 \times \PP^1$: as you can see from a direct examination of the equation, each of these closures will be singular at the (unique) point at infinity.
  
  \begin{exercise}
  In the case $g=1$, show that the closure $\overline{C^\circ}$ of $C^\circ \subset \AA^2$ in either $\PP^2$ or $\PP^1 \times \PP^1$ consists of the union of $C^\circ$ with one additional point, with that point a tacnode of $\overline{C^\circ}$ in either case.
  \end{exercise}
  
  In fact, it's possible to give explicitly a projective model of the curve $C$: if we divide the points $q_1,\dots,q_{2g+2} \in \PP^1$ into two sets of $g+1$---say, for example, $q_1,\dots,q_{g+1}$ and $q_{g+2}, \dots, q_{2g+2}$---then we can describe the smooth projective curve $C$ as the closure in $\PP^1 \times \PP^1$ of the  locus
  $$
  \big\{ (x,y) \in \AA^2 \; \mid \; y^2\prod_{i=1}^{g+1} (x - \lambda_i) = \prod_{i=g+2}^{2g+2} (x - \lambda_i) \big\};
  $$
  in projective coordinates, this is
   $$
  C \; = \; \big\{ (X,Y) \in \PP^1 \times \PP^1 \; \mid \; Y_1^2\prod_{i=1}^{g+1} (X_1 - \lambda_iX_0) = Y_0^2\prod_{i=g+2}^{2g+2} (X_1 - \lambda_iX_0) \big\}.
  $$
  (No local analysis is needed to see that $C \subset \PP^1 \times \PP^1$ is smooth: it is a curve of bidegree $(2,g+1)$ in $\PP^1 \times \PP^1$, and the genus formula tells us that such a curve has arithmetic genus $g$.)
  
%  \
%  
%  \
%  
%  \
%  
%  \
% 
% We start by establishing some basic facts about hyperelliptic curves. Many of these follow from general theorems like Riemann-Roch; but since they can be established by direct examination we will carry that out here.
%
%Suppose $C$ is a smooth, projective hyperelliptic curve of genus $g \geq 2$. By definition, $C$ admits a degree 2 map $\pi : C \to \PP^1$; and as we've observed (\ref{**}) this map is unique.
%
%By Riemann-Hurwitz, \fix{attibution?} the map $\pi : C \to \PP^1$ will have $2g+2$ distinct simple branch points, say $\lambda_1,\dots,\lambda_{2g-2} \in \PP^1$. An open subsect $C^\circ$ of $C$ can then be realized as the smooth projective completion of the affine curve given as
%$$
%C^\circ = \big\{ (x,y) \in \AA^2 \; \mid \; y^2 = \prod_{i=1}^{2g+2} (x - \lambda_i) \big\}.
%$$ 
%\fix{if two of the $\lambda_i$ coinncide, then the curve develops a singular point. Much of what we will do carries over to the singular case.} \fix{say the smooth model has 2 points at $\infty$.} Note that if we simply take the closure of this locus in $\PP^2$, the resulting curve will be highly singular at the point $[1,0,0]$, as can be seen either  directly by making an appropriate change of variables, or by invoking the genus formula for plane curves: if the closure were smooth, it would have genus $\binom{2g+1}{2}$. We can, however, complete the curve simply in $\PP^1 \times \PP^1$, for example by setting \fix{this is a rabbit from a hat. Consider either saying that by the previous section, if there's an emb in P3 then its on P1 x P1 as a divisor of type
%2,g+1; and then "finding" this embedding as below; or moving this page to the early place where hyperelliptic curves are first mentioned.}
%$$
%y' = \frac{y}{\prod_{i=1}^{g+1} (x - \lambda_i)};
%$$
%we can then write the equation of a still smaller open subset of $C$ as
%$$
%{y'}^2 \cdot \prod_{i=1}^{g+1} (x - \lambda_i) \; = \; \prod_{i=g+2}^{2g+2} (x - \lambda_i).
%$$
%If we now take the closure of this locus in $\PP^1 \times \PP^1$, we get a curve of type $(2,g+1)$ on $\PP^1 \times \PP^1$; this curve is smooth, as can be seen again either directly in coordinates or by invoking the genus formula for curves on $\PP^1 \times \PP^1$. In other words,
%$$
%C \; = \; V\Big(Y_0^2\cdot \prod_{i=1}^{g+1} (X_1 - \lambda_iX_0) - Y_1^2 \cdot \prod_{i=g+2}^{2g+2} (X_1 - \lambda_iX_0) \Big)
%$$

  
  \subsection{Differentials on a hyperelliptic curve}

Next, let's describe the space of regular differentials on $C$. For this, it's convenient to work with the affine model $C^\circ = V(f) \subset \AA^2$, where
$$
f(x,y) = y^2 - \prod_{i=1}^{2g-2} (x - \lambda_i).
$$
Again, we'll denote the two points at infinity---that is, the two points of $C \setminus C^\circ$---as $r$ and $s$; for convenience, we'll denote the divisor $q+r$ by $D$.

To start, consider the simple differential $dx$ on $C$. (Technically, we should write this as $\pi^*dx$, since we mean the pullback to $C$ of the differential $dx$ on $\PP^1$, but for simplicity of notation we'll suppress the $\pi^*$.)  This is clearly regular on $C^\circ$, with zeros at the ramification points $q_i = (\lambda_i, 0)$. But it does not extend to a regular differential on all of $C$: it will have double poles at $r$ and $s$.  This can be seen directly: the differential $dx$ extends to a rational differential on $\PP^1$, and in terms of the local coordinate $w = 1/x$ around the point $x = \infty$ on $\PP^1$, we have
$$
dx = d\left(\frac{1}{w}\right) = \frac{-dw}{w^2}
$$
so $dx$ has a double pole at the point at $\infty$; since the map $\pi$ is a local isomorphism near $r$ and $s$ the pullback of $dx$ to $C$ likewise has double poles at the points $r$ and $s$.


We could also see that $dx$ must have poles by degree considerations: as we said, $dx$ has $2g+2$ zeros and no poles in $C^\circ$, while the degree of $K_C$ is $2g-2$, meaning that there must be  a total of four poles at the points $r$ and $s$. In any event, we have an expression for the canonical divisor class on $C$: denoting by $R = q_1 + \dots + q_{2g+2}$ the sum of the ramifications points of $\pi$, we have
$$
K_C \sim (dx) \sim R - 2D;
$$
this is a case of the Riemann-Hurwitz formula above.

So, given that $dx$ has poles at $r$ and $s$, how do we find regular differentials on $C$? One thing to do would be simply to divide by $x^2$ (or any quadratic polynomial in $x$) to kill the poles. But that just introduces new poles in the finite part $C^\circ$ of $C$. Instead, we want to multiply $dx$ by a rational function with zeros at $p$ and $q$, but \emph{whose poles occur only at the points where $dx$ has zeroes}---that is, the points $q_i$.  A natural choice is simply the reciprocal of the partial derivative $f_y = \partial f/ \partial y = 2y$, which vanishes exactly at the points $r_i$, and has correspondingly a pole of order $g+1$ at each of the points $r$ and $s$ (reason: the involution $y\to -y$ fixes $C^\circ$ and $x$, and exchanges the points $p,q$). In other words, the differential
$$
\omega = \frac{dx}{f_y}
$$
is regular, with divisor
$$
(\omega) = (g-1)r + (g-1)s = (g-1)D.
$$
The remaining regular differentials on $C$ are now easy to find: Since $x$ has only a simple pole
at the two points at infinity we can  multiply $\omega$ by any $x^k$ with $k = 0, 1, \dots, g-1$. Since this gives us $g$ independent differentials, these  form a basis for $H^0(K_C)$.

\subsection{The canonical map of a hyperelliptic curve}\label{hyperelliptic special}

Given that a basis for $H^0(K_C)$ is given by 
$$
H^0(K_C) = \langle \omega, x\omega,\dots,x^{g-1}\omega \rangle,
$$
we see that the canonical map $\phi : C \to \PP^{g-1}$ is given by $[1,x,\dots,x^{g-1}]$. In other words, the canonical map $\phi$ is simply the composition of the map $\pi : C \to \PP^1$ with the Veronese embedding $\PP^1 \hookrightarrow \PP^{g-1}$ of $\PP^1$ into $\PP^{g-1}$ as a rational normal curve of degree $g-1$.

Note that as a consequence of this fact, we see that \emph{a hyperelliptic curve $C$ has a \emph{unique} linear series $g^1_2$ of degree 2 and dimension 1}, that is, a unique map of degree 2 to $\PP^1$. 
Finally, we can give an explicit description of special linear series on a hyperelliptic curve: if $D = \sum p_i$ is any effective divisor on $C$, we can pair up points $p_i$ that are conjugate under the involution $\iota$ exchanging sheets of the degree 2 map $C \to \PP^1$; each conjugate pair is a divisor of the unique $g^1_2$ on $C$, and so we can write
$$
D \sim r\cdot g^1_2 + q_1 + \dots + q_{d-2r},
$$
where no two of the points $q_i$ are conjugate under $\iota$. Now the geometric form of the Riemann-Roch formula tells us that the dimension $r(D)$ of the complete linear series $|D|$ is exactly $r$, so that in fact 
$$
|D| = |r\cdot g^1_2| + + q_1 + \dots + q_{d-2r};
$$
that is, the points $q_i$ are base points of the linear series $D$.

One key observation is that, according to this analysis, \emph{no special linear series on a hyperelliptic curve can be very ample}; the map associated to any special series factors through the degree 2 map $C \to \PP^1$. This is in marked contrast to the case of non-hyperelliptic curves, for which the embeddings of minimal degree in projective space are given by special linear series. 


\section{Moduli problems}

We'll conclude this introductory chapter with a discussion of one of the central constructions of algebraic geometry in the present context: the notion of a \emph{moduli}, or \emph{parameter space}.

It is a fundamental aspect of algebraic geometry that the objects we deal with often vary in families, and can often be parametrized by a ``universal" such family. This notion of objects varying with parameters underlies many of the constructions and theorems we'll be discussing in this book, and so it seems like a good idea to establish the basic facts about moduli spaces in general, and the particulars of the ones we'll be dealing with here. 

\subsection{What is a moduli problem?}

Briefly, a \emph{moduli problem} consists of two things: a class of objects, or isomorphism classes of objects; and a notion of what it means to have a \emph{family} of these objects parametrized by a given scheme $B$. To make this relatively explicit, the four main examples of moduli problems we'll be discussing here are:

\begin{enumerate}
\item  smooth curves: objects are isomorphism classes  of smooth, projective curves $C$ of a given genus $g$; by a family we'll mean a flat, smooth, projective morphism $\cX \to B$.

\item the Hilbert scheme: objects are smooth, projective curves $C \subset \PP^r$ of degree $d$ and genus $g$; a family over $B$ will be a subscheme $\cX \subset B \times \PP^r$, smooth, projective and flat over $B$, whose fibers are curves of degree $d$ and genus $g$

\item effective divisors on a given curve: objects are effective divisors of a given degree $d$ on a given smooth, projective curve $C$; families over $B$ will be subschemes $\cD \subset B \times C$ flat over $B$ with fibers of degree $d$

\item line bundles on a given curve: objects are line bundles of a given degree $d$ on a given curve $C$; by a family of such line bundles over a scheme $B$ we'll mean a line bundle on the product $B \times C$ whose restriction to each fiber over $B$ has degree $d$, modulo tensor product with line bundles pulled back from $B$.
\end{enumerate}

Given a moduli problem, our goal will be to describe a corresponding \emph{moduli space}. By this we mean a scheme $M$ whose points are in natural one-to-one correspondence with the objects in our moduli problem; or, to put it differently, we want to realize the set of objects in our moduli problem as the underlying set of a scheme $M$.

The issue here, as it so often is,  is the word ``natural." If we're working over $\CC$, for example, all positive dimensional varieties have the same cardinality ($\aleph_1$), so saying that we have a bijection between the points of a variety $M$ and the set of isomorphism classes of curves isn't much of a characterization of $M$. Rather, we need some basic condition on the bijection, for which the word ``natural" is a stand-in. 

In the pre-Grothendieckian world of varieties, it was easy to express this condition, though the result (as we'll see) was not always satisfactory. Given a family of the objects in our moduli problem over a variety $B$, we get a map from the underlying set of $B$ to the underlying set of $M$; and the requirement was simply that this map of sets defined a regular morphism of varieties.

In the post-Grothendieckian world of schemes, however, this doesn't work: a morphism of schemes is not determined by the associated map of sets. \fix{let's give a better reason: more examples exist; and you can do infinitesimal computations}
The solution to this is to recast the condition above in functorial terms. We observe first that a moduli problem defines a functor $\cM$ from the category of schemes to the category of sets: we can associate to every scheme $B$ the set of families of objects parametrized by $B$; a morphism $B' \to B$ of schemes gives rise, via pullback, \fix{define?} to a set map $\cM(B) \to \cM(B')$. By a \emph{fine moduli space} for the moduli problem, then, we'll mean a scheme $M$ that represents this functor, meaning that we have an isomorphism of functors
$$
\cM \to {\rm Mor}(\bullet, M)
$$
In other words, for every scheme $B$ we have a bijection between families of our objects over $B$ and maps from $B$ to $M$. In particular, applying this to $B = \Spec \CC$, we have a bijection between the set of objects and the closed points of $M$; and for any family over an arbitrary scheme $B$, the map from $B(\CC)$ to $M(\CC)$ sending each closed point  $b \in B$ to the point in $M(\CC)$ corresponding to the fiber over $b$ is the underlying set map of a morphism $B \to M$ of schemes.

Note that by Yoneda's Lemma, this condition uniquely determines the space $M$---if one exists. In fact, it doesn't always; for the four examples above, a fine moduli space exists for three of them, but for the moduli problem of smooth curves it doesn't. We'll defer the discussion of why this is, and what we can do about it, until Chapter~\ref{Moduli chapter}, when we'll talk in more detail about this problem.

Looking ahead, we'll discuss the third and fourth example in Chapter~\ref{}, where we'll describe the moduli spaces for effective divisors of given degree $d$ on a given curve $C$ (the symmetric powers of the curve) and for line bundles of a given degree on $C$ (the \emph{Jacobian} and \emph{Picard variety} of $C$). These, as we'll see, are smooth, irreducible projective varieties of dimensions $d$ and $g$ respectively.

We'll take up the moduli space $M_g$ of smooth curves in Chapter~\ref{Moduli chapter}, where we'll see that this space (or rather the closest approximation to it we can cook up) is irreducible of dimension $3g-3$ for $g \geq 2$, though not smooth or projective.

Finally, the Hilbert scheme will be described (to the extent that we can!) in Chapter~\ref{HilbertSchemesChapter}; this will turn out to be much wilder and more varied in its behavior than any of the above.


%The definition is straightforward: by a family of smooth, projective curves $C$ of a given genus $g$ over a given base variety $B$, we mean a flat morphism
%$\cC \to B$ whose fibers are smooth, projective curves $C$ of genus $g$


\input footer.tex