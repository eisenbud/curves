%header and footer for separate chapter files

\ifx\whole\undefined
\documentclass[12pt, leqno]{book}
\usepackage{graphicx}
\usepackage{eps-to-pdf}
\input style-for-curves.sty
%\input sl-macros.sty
\usepackage{hyperref}
\usepackage{showkeys} %This shows the labels.
\usepackage{msribib}
\usepackage{pdfpages}
\usepackage{draftwatermark}
\SetWatermarkText{DRAFT:\ \today}
\SetWatermarkScale{2}
\SetWatermarkColor[gray]{0.9}

%\usepackage{SLAG,msribib,local}
%\usepackage{amsmath,amscd,amsthm,amssymb,amsxtra,latexsym,epsfig,epic,graphics}
%\usepackage[matrix,arrow,curve]{xy}
%\usepackage{graphicx}
%\usepackage{diagrams}
%
%%\usepackage{amsrefs}
%%%%%%%%%%%%%%%%%%%%%%%%%%%%%%%%%%%%%%%%%%
%%\textwidth16cm
%%\textheight20cm
%%\topmargin-2cm
%\oddsidemargin.8cm
%\evensidemargin1cm
%
%%%%%%Definitions
%\input preamble.tex
%\input style-for-curves.sty
%\def\TU{{\bf U}}
%\def\AA{{\mathbb A}}
%\def\BB{{\mathbb B}}
%\def\CC{{\mathbb C}}
%\def\QQ{{\mathbb Q}}
%\def\RR{{\mathbb R}}
%\def\facet{{\bf facet}}
%\def\image{{\rm image}}
%\def\cE{{\cal E}}
%\def\cF{{\cal F}}
%\def\cG{{\cal G}}
%\def\cH{{\cal H}}
%\def\cHom{{{\cal H}om}}
%\def\h{{\rm h}}
% \def\bs{{Boij-S\"oderberg{} }}
%
%\makeatletter
%\def\Ddots{\mathinner{\mkern1mu\raise\p@
%\vbox{\kern7\p@\hbox{.}}\mkern2mu
%\raise4\p@\hbox{.}\mkern2mu\raise7\p@\hbox{.}\mkern1mu}}
%\makeatother

%%
%\pagestyle{myheadings}

%\input style-for-curves.tex
%\documentclass{cambridge7A}
%\usepackage{hatcher_revised} 
%\usepackage{3264}
   
\errorcontextlines=1000
%\usepackage{makeidx}
\let\see\relax
\usepackage{makeidx}
\makeindex
% \index{word} in the doc; \index{variety!algebraic} gives variety, algebraic
% PUT a % after each \index{***}

\overfullrule=5pt
\catcode`\@\active
\def@{\mskip1.5mu} %produce a small space in math with an @

\title{A Chapter from ``The Practice of Algebraic Curves"}
\author{\copyright David Eisenbud and Joe Harris}
%%\includeonly{%
%0-intro,01-ChowRingDogma,02-FirstExamples,03-Grassmannians,04-GeneralGrassmannians
%,05-VectorBundlesAndChernClasses,06-LinesOnHypersurfaces,07-SingularElementsOfLinearSeries,
%08-ParameterSpaces,
%bib
%}

\date{\today}
%%\date{}
%\title{Curves}
%%{\normalsize ***Preliminary Version***}} 
%\author{David Eisenbud and Joe Harris }
%
%\begin{document}

\begin{document}
\maketitle

\pagenumbering{roman}
\setcounter{page}{5}
%\begin{5}
%\end{5}
\pagenumbering{arabic}
\tableofcontents
\fi


\chapter{Linear Systems}\label{linear systems chapter}
\fix{we need the $H^1 \cO^*$ characterization of the group of inv sheaves}

Morphisms of a smooth curve $C$ (or indeed of any scheme) to a projective space are conveniently studied using the closely related notions of Divisors, linear systems and invertible sheaves. 

\section{Morphisms to projective space, and families of Cartier divisors}

Let $\phi: C\to \PP^{r}$ be a morphism from a smooth curve $C$. If $H\subset \PP^r$ is a hyperplane that does not contain $\phi(C)$, then the preimage of $\phi(C)\cap H$ is a finite sets of points on $C$, with multiplicities when $H$ is tangent to $\phi(C)$ or passes through a singular point of $\phi(C)$. Such a set of points with non-negative integer multiplicities is called an \emph{effective divisor} on $C$; more generally, a \emph{divisor} (sometimes called a \emph{Weil divisor})on a scheme $X$ is an integral linear combination of codimension 1 subvarieties, and it is called \emph{effective} if the coefficients are all non-negative. The divisors that arise as the pullbacks of general hyperplanes are special: since a hyperplane is defined by just one equation, which is locally given by the vanishing of a function, the pullback of a hyperplane will be locally defined by the vanishing of a single function that is a nonzerodivisor; that is, it is an  \emph{effective Cartier divisor}. See \cite[pp. 140-146]{H} for more information; on a smooth curve every divisor is Cartier, so the difference between Weil and Cartier divisors will not be an issue for us.) 

The  word ``local'' scattered through the previous paragraph is needed because, if $X$ is a projective variety, then the only algebraic functions $X\to \CC$ are constant functions. (Proof: the image of a projective variety
is again projective, and the only projective subvarieties of an affine variety are points.)

If we are given the family of divisors on $C$ that are the preimages of the intersections of hyperplanes with  $\phi(C)$, we can recover the morphism $\phi$ set-theoretically: it takes a point $p\in C$ to the point of projective space that is the intersection of those
hyperplanes whose preimages contain $p$. 

The relationship of two divisors on $C$ that are preimages of intersections of $\phi(C)$ with hyperplanes is simple to describe: If hyperplanes
$H, H'\subset \PP^r$ are defined by the linear form $h, h'$  then $1/h$ has a simple pole along $E$---we may say that it ``vanishes along $E$'' to degree $-1$.
In this sense the divisor $H-H'$ on $\PP^n$ is defined by the rational function $\lambda= h'/h$. If neither $H$ nor $H'$ contain $C$ then the pullback of $\lambda$ is a well-defined, nonzero rational function on $C$, and the divisor 
$\phi^{-1}(\phi(C)\cap H') - \phi^{-1}(\phi(C)\cap H)$ is defined by the pullback  $\phi^*(\lambda) := \lambda \circ f$. Thus the divisors arising from a given morphism to $\PP^{r}$ differ by the divisors of zeros minus poles of rational functions on $C$. 

If $C$ is a smooth curve then the local ring $\sO_{C,p}$ of $C$ at a point $p$ is a discrete valuation ring, and if $\pi$ is a generator of the maximal ideal of $\sO_{C,p}$, then any rational
function $\lambda$ on $C$ can be expressed uniquely as $u\pi^k$ where $u\in \sO_{C,p}$ is a unit and $k\in \ZZ$. We say that
the \emph{order} of $\lambda$ at $p$, and write $k = \ord_p \lambda$. We associate $\lambda$ to the divisor
$$
(\lambda) := \sum_{p\in C} (\ord_p\lambda)p.
$$
The \emph{class group} of $C$ is defined to be the the group of divisors on $C$ modulo the divisors of rational functions.
Thus the divisors on $C$ that are preimages of intersections of $\phi(C)$ with different hyperplanes all belong to the same
\emph{divisor class}, and form a linear system in the sense of the following section.

\section{Morphisms and linear systems}
We want to understand morphisms to $\PP^r$ more than set-theoretically, and we want to be able to produce them from data on $C$. For this we use the notion of linear system (sometimes called linear series). 

\begin{definition}
 A \emph{linear system} on a scheme $X$ is a pair $\sV  = (\sL, V)$ where $\sL$ is an invertible sheaf on $X$ and
 $V$ is a vector space of global sections of $\sL$. 
\end{definition}

We will spend the next pages unpacking this notion. Our goal is to explain and prove:
\begin{theorem}\label{morphisms and linear systems}
There is a natural bijection between the set of nondegenerate morphisms $\phi : C \to \PP^r$ modulo $PGL_{r+1}$, and basepoint-free linear systems of dimension $r$ on $C$.
\end{theorem}

Here ``nondegenerate" means the image of the morphism $\phi$ is not contained in any hyperplane. 



\subsection{Invertible sheaves}
Recall first that a \emph{coherent sheaf} $\sL$ on a scheme $X$ may be defined by
giving 
\begin{itemize}
 \item An open affine cover $\{U_{i}\}$ of $X$; 
 \item For each $i$, a finitely generated $\sO_{X}(U_{i})$-module $L_{i}$;
 \item For each $i,j$, an isomorphism $\sigma_{i,j}: L_{i}\mid_{U_{i}\cap U_{j}} \to L_{j}\mid_{U_{i}\cap U_{j}}$
 satisfying the compatibility conditions $\sigma_{j,k}\sigma_{i,j} = \sigma_{i,k}$. 
 \end{itemize}

A \emph{global section} of $\sL$ is a family of elements $t_{i}\in F_{i}$ such that 
$\sigma_{i,j} t_{i} = t_{j}$. Such a section may be realized as the image of the constant function 1 under
a homomorphism of sheaves $\sO_{X} \to \sL$. By Theorem \cite[Thm III.5.2]{H} the space $H^{0}(\sL)$  of global sections is
a finite-dimensional vector space. For example, $H^{0}(\sO_{X}) = \CC$ because the only globally defined
functions on $X$ are the constant functions.

The coherent sheaf $\sL$ is said to be an \emph{invertible sheaf} on $X$ if there is an open cover as above with the additional property
that $F_{i} \cong \sO_X(U_{i})$, the free module on one generator. 

If $\sigma \in H^0\sL$ is a global section of an invertible sheaf
on $X$, and $p\in X$ is a point, then $\sigma(p)$ is in the stalk of $\sL$ at $p$, a module isomorphic to $\sO_{X,p}$. Since the isomorphism is not canonical, $\sigma$ does not define a function on $X$ at $p$; but since any two isomorphisms
differ by a unit in $\sO_{X,p}$, the vanishing locus, denoted $(\sigma)_0$ of $\sigma$ \emph{is} a well-defined subscheme of $X$. Moreover, if $X$ is integral, then the ratio of two global sections is a well-defined rational function, so the divisor class of 
$(\sigma)_0$ is independent of the choice of $\sigma$.

\begin{proposition}
 The invertible sheaves on $X$ form a group under $\otimes_{X}$, called the 
\emph{Picard group of $X$}, denoted $\Pic(X)$. 
\end{proposition}
\begin{proof}
 If $\sF, \sG$ are invertible sheaves then so are $\sF\otimes_{X}\sG$ and  $\Hom_{X}(\sF, \sG)$, as one sees immediately by
restricting to the open sets where $\sF$ and $\sG$ are isomorphic to $\sO_{X}$. Moreover the natural isomorphisms
$$
\sF(U) \otimes_{X} \Hom(\sF(U), \sO_{X}(U)) \to \sO_{X}(U)\quad s \otimes f \mapsto f(s)
$$ 
patch together to define a global isomorphism 
$$
\sF \otimes_{X} \Hom(\sF, \sO_{X}) \to \sO_{X}
$$
justifying the definition
$\sF^{-1} := \Hom(\sF, \sO_{X})$ and thus the name ``invertible sheaf''. 
\end{proof}
 
If $D\subset X$ is an effective divisor, then we define $\sO_{X}(-D)$ to be the ideal sheaf of $D$. If $D$ is locally defined by the vanishing of a (locally defined) nonzerodivisor in $\sO_{X}$, (that is, $D$ is a Cartier divisor), then
$\sO_{X}(-D)$ is an invertible
sheaf.
We write $\sO_{X}(D)$ for the inverse, $\sO_{X}(-D)^{-1}$. The dual of the inclusion
$\sO_{X}(-D)\subset \sO_{X}$ is a map $\sO_{X} \to \sO_{X}(D)$ sending the global section $1\in \sO_{X}$ to a section
$\sigma\in \sO_{X}(D)$ that vanishes precisely on $D$.

\begin{example} [Invertible sheaves on $\PP^{r}$]\label{linear systems on Pr} If $H\subset \PP^{r}$ is a hyperplane defined by the vanishing of a linear form $\ell = \ell(x_{0}, \dots x_{r})$ then the ideal sheaf $\sO_{\PP^{r}}(-1) := \sI_{H/\PP^{r}}\subset \sO_{\PP^{r}}$ is generated on the open affine set 
$U_{i}:= \{x_{i}\neq 0\} \cong \AA^{r}$
by $\ell/x_{i}$, and is thus an invertible sheaf. 
Moreover, if $H'$ is the hyperplane defined by another linear form $\ell'$, then 
$$
\frac{\ell'}{\ell}\cdot\sI_{H/\PP^{r}} = \sI_{H'/\PP^{r}} 
$$
\fix{check that this is out notation for ideal sheaf}
so the sheaves $\sI_{H/\PP^{r}}$ and $\sI_{H'/\PP^{r}} $ are isomorphic, justifying the name $\sO_{\PP^{r}}(-1)$.

The $p$-th tensor power of $\sO_{\PP^{r}}(-1)$ is called $\sO_{\PP^{r}}(-d)$; it is isomorphic to the
ideal sheaf of any hypersurface of degree $d$. Because polynomials satisfy the unique factorization property,
every effective divisor $D\subset \PP^{r}$ is a hypersurface of some degree $d$, so
$\sO_{\PP^{r}}(-D) \cong \sO_{\PP^{r}}(-d)$. Note that if $d>0$ then $H^{0}(\sO_{\PP^{r}}(-D)) = 0$, since it may be realized
as the sheaf of locally defined functions vanishing on $D$, and there are no such
globally defined functions except 0.

We take $\sO_{\PP^{r}}(d)$ to be the inverse of $\sO_{\PP^{r}}(-d)$. If $D$ is the hypersurface defined by 
a form $F$ of degree $d$, then $\sO_{\PP^{r}}(-D)$ is generated on $U_{i}$ by $F/(x_{i}^{d})$, so
$\sO_{\PP^{r}}(D)$ is generated on $U_{i}$ by $x_{i}^{d}/F$.
Starting from the inclusion 
$
\sO_{\PP^{r}}(-D) \subset \sO_{\PP^{r}}
$
and taking inverses, we see that 
$
\sO_{\PP^{r}} \subset \sO_{\PP^{r}}(D)
$
and the global section $1\in H^0(\sO_{\PP^{r}})\subset H^0(\sO_{\PP^{r}}(D))$, restricted to
$U_{i}$, is $F/(x_{0}^{d})$ times the local generator of $\sO_{\PP^{r}}(D)$ and thus vanishes on $D$.
 Because every
rational function on $\PP^{r}$ has degree 0, and any two global sections differ by a rational
function, it follows that every global section of $\sO_{\PP^{r}}(d)$ vanishes on a divisor of degree $d$. Thus
we may identify $H^{0}(\sO_{\PP^{r}}(d))$ with the ${n+d\choose n}$-dimensional vector space of forms of degree $d$ on $\PP^{r}$.
\end{example}

The proof of Theorem~\ref{morphisms and linear systems} is contained in the material of the next two subsections:

\subsection{The morphism to projective space coming from a linear system} 
For any $\CC$-vector space $V$ of dimension $r+1$ with basis $x_{0}, \dots, x_{r}$, we write $\Sym(V) \cong \CC[x_{0},\dots, x_{r}]$ for the symmetric algebra on $V$, and
$\PP(V)\cong \PP^{r}_{\CC}$ to be the projective space ${\rm Proj}(\Sym(V))$, which is naturally isomorphic to the
space of lines in $V^{*}$. Note that the isomorphism $\PP(V)\cong \PP^{r}_{\CC}$ is well-defined up to the action
of $\Aut(\PP^r) = PGL(r+1)$.


Given a linear system $\sV:=(\sL, V)$  of dimension $r$ on a scheme $X$, where
$\sL$ is an invertible sheaf on $X$ and $V = \langle\sigma_{0}, \dots \sigma_{r}\rangle$ is a vector space of global sections, we define the \emph{base locus} of $\sV$ to be the closed subscheme 
$$
B_\sV := \bigcap_{i= 0}^{r}\{\sigma_{i} = 0\}.
$$
Let $W:=X\setminus B_\sV$ be the open subscheme where not all sections $\sigma_{i}$ vanish.

For any point $q\in W$ we  may choose an open neighborhood $W'\subset W$ of $q$, and an identification 
$$
t: \sL\mid_{W'} \rTo^{\cong} \sO_{W'}
$$
and define $\phi_{\sV}: W' \to \PP(V)$ by 
$$
W'\ni p \mapsto \bigl(t(\sigma_{0}(p)),\dots, t(\sigma_{r}(p))\bigr) \in \PP(V).
$$
This  is a morphism on $W'$. A change of neighborhoods $W'$ or of identifications $t$ would multiply
each value $t(\sigma_{i}(p))$ by a unit, the same one for each $i$, and thus the construction would define the same morphism. It follows that the morphisms
defined on different $W'$ agree on overlaps, and thus define a morphism $W \to \PP(V) \cong \PP^r$. This is the reason
that the dimension of $\sV$ is defined to be $r=\dim V -1$ instead of $\dim V$.

The most useful linear series are those that define morphisms defined on all of $X$. This happens when $B_\sV = \emptyset$,
that is, for every point $q\in X$, there is a section $\sigma \in V$ such that $\sigma$ does not vanish at $x$. In this case we say that $(\sL, \sV)$ is \emph{basepoint free}.

\begin{example}\label{Veronese definition}
The morphism from $\PP^r$ defined by the complete linear system $|\cO_{\PP^r}(d)|$ has target
$\PP^{r+d\choose r}$, and takes a point $x_0,\dots x_r$ to the point whose coordinates are all the monomials of
degree $d$ in $x_0,\dots x_r$. It is called the \emph{$d$-th Veronese morphism} of $\PP^r$. For example on $\PP^1$, this has the form
$$
(x_0,x_1) \mapsto (x_0^d,\ x_0^{d-1}x_1,\ \dots,x_1^d).
$$
The image of $\PP^1$ under this morphism is called the \emph{rational normal curve} of degree $d$; in the case $d=2$ is the
\emph{plane conic}, and if $d=3$ it is called the \emph{twisted cubic}. Veronese himself studied the image of $\PP^2$
by the Veronese morphism of degree 2 now simply called \emph{the Veronese surface}.
\end{example}

\begin{exercise}\label{here there be basepoints}
 Show that there is no non-constant morphism $\PP^r\to \PP^s$ when $s<r$ by showing that any nontrivial linear
 system of dimension $<r$ has a non-empty base locus.
\end{exercise}

\subsection{The linear system coming from a morphism to projective space}

Conversely, suppose that we are given a morphism $\phi: X\to \PP^{r}$. With notation as in Example~\ref{linear systems on Pr} we may choose an open affine cover $W_{i,j}$ of $X$ such that $\phi(W_{i,j})\subset U_{j}$. Composing the regular
functions
$x_{0}/x_{j},\dots, x_{r}/x_{j}$ with $\phi$ we get functions $\sigma_{0},\dots,\sigma_{r}$ on $W_{i,j}$.  The function $\sigma_{j}$, is the image under $\phi^*: \sO_{U_j} \to \sO_{W_{i,j}}$ of the function $x_j/x_j = 1$ on $U_{j}$, so it $\sigma_j = 1\in \sO_{W_{i,j}}$. In particular, the module $\sL_{\phi^{-1}(U_j)}$ generated by the rational functions 
$$
\{(\sigma_i)_{\phi^{-1}(U_j)} = \phi^*(x_i/x_j)\}_{0\leq i\leq n}
$$
 is a free $\sO_{W_{i,j}}$-module on 1 generator. On the preimage of $U_j\cap U_k$ these sections differ by the common unit $\phi^*(x_k/x_j)$, and thus the collection of these modules defines an invertible sheaf $\sL$ on $X$ together with an
$r+1$-dimensional space of global sections $\sV := \langle \sigma_0,\dots \sigma_r\rangle$ that forms a basepoint free linear system. Note that the subscheme  $\{\sigma_k = 0\} \subset W_{i,j}$  is the scheme-theoretic preimage of the
the hyperplane $\{x_k = 0\}\subset \PP^r$. This completes the explanation and proof of Theorem~\ref{morphisms and linear systems}


\subsection{More about linear systems}

Let $\sV = (\sL, V)$ be a linear sysytem on $X$.  The linear system is said to be \emph{complete} if $V = H^0(\cL)$; in this case it is sometimes denoted $|\cL|$. If $\cL \cong \cO_C(D))$, we also write it as $|D|$. 
The \emph{dimension of $\sV$} is $\dim V -1$ . If $D$ is any divisor on $C$ we write $r(D)$ for the dimension of the complete linear series $|D|$; that is, $r(D) = h^0(\cO_C(D)) - 1$. Finally, a linear system of dimension 1 is called a \emph{pencil}, a linear system of dimension 2 is called a \emph{net} and, less commonly, a three-dimensional linear system is called a \emph{web}. 

A linear system $\sV = (\sL, V)$ is called \emph{basepoint free} if it defines a morphism to $\PP(V)$, or equivalently if the
the sections in $V$ generate $\sL$ locally at each point of $X$. It is called \emph{very ample}  if it is basepoint-free and defines an embedding. If $D$ is a Cartier divisor on $X$, then we say that $D$ is \emph{very ample} if the complete linear system $|D|$ is versy ample, and we say that $D$ is \emph{ample} if $mD$ is very ample for some integer $m>0$.

Given a linear system $\sV = (\sL, V)$ and an effective divisor $D$ on $C$, we'll  set
$$
V(-D) := \{ \sigma \in V \mid \sigma(D) = 0 \}.
$$
The difference $\dim V - \dim V(-D)$ is called the \emph{number of conditions imposed by $D$ on the linear system $\sV$}; we say that $D$ \emph{imposes independent conditions} on $\sV$ if $\dim V - \dim V(-D) = \deg(D)$.


Via the correspondence of Theorem~\ref{morphisms and linear systems}, the statements about the geometry of a morphism $\phi : C \to \PP^r$ can be formulated as statements about the relevant linear systems. We will see this in many instances throughout this book. It will be most convenient to formulate this in terms of the vector space $H^{0}(\sL)$ of global sections of $\sL$, and we write $h^{0}(\sL)$ for the dimension of this vector space. Here is a first example:

\begin{proposition}\label{very ample}\cite[Thm. IV.3.1]{H}
Let $\cL$ be an invertible sheaf on a smooth curve $C$. The complete linear system $|\cL|$ is base-point-free iff
$$
h^0(\cL(-p)) = h^0(\cL) - 1 \quad \forall p \in C;
$$
and in this case the associated morphism $\phi_\cL$ is an embedding, so $|\cL|$ is very ample, iff
$$
h^0(\cL(-p-q)) = h^0(\cL) - 2 \quad \forall p, q \in C.
$$
\end{proposition} 

\begin{proof}
The statement $h^0(\cL(-p-q)) = h^0(\cL) - 2$ for $p \neq q$ implies that $\phi_\cL(p) \neq \phi_\cL(q)$. The tangent space of $C$ at $p$ is $(\sI_C(p)/\sI_C(p)^2)^*$, so the condition that there is a section of $\sL$ that vanishes at $p$, but does not vanish
to order 2, implies that the differential $d\phi_\cL$ is injective at $p$.
\end{proof}

\fix{this uses a lot: even given the identification of the tangent space with $m/m^2$ we really only get an analytic isomorphism.
To deduced the algebraic one we'd need a finiteness principal: projective maps with finite fibers are finite. Should we say  some of this??}

We can also relate the geometry of the morphism associated to a incomplete linear system $V \subset H^0(\cL)$ to the geometry of the morphism associated to the complete linear system $|\cL|$. In general, if $V \subset W \subset H^0(\cL)$ are a pair of nested linear systems, we have a linear map $W^* \to V^*$ dual to the inclusion $V \hookrightarrow W$, and a corresponding linear projection $\pi : \PP W^* \dashrightarrow \PP V^*$, with indeterminacy locus the subspace $\PP(Ann(V)) \subset \PP W^*$. In this case, we have 
$$
\phi_V = \pi \circ \phi_W;
$$
that is, we have the diagram 

\begin{diagram}
& & \PP W^* \\
& \ruTo^{\phi_W} & \dDashto_\pi \\
C & \rTo^{\phi_V} & \PP V^*.
\end{diagram}

Note that in this case, given that $W$ is base-point-free, the condition that $V$ be base-point-free is equivalent to saying that the center $\PP(Ann(V))$ of the projection $\pi$ is disjoint from $\phi_W(C)$.

By way of language, we will say that a curve $C \subset \PP^r$ embedded by a complete linear series is \emph{linearly normal}; this is equivalent to saying that the restriction map
$$
H^0(\cO_{\PP^r}(1)) \to H^0(\cO_{C}(1))
$$
is surjective, which is in turn equivalent to saying that $C$ is not the regular  projection of a nondegenerate curve $\tilde C \subset \PP^{r+1}$.

\begin{exercise}
Extend the statement of Proposition~\ref{very ample} to incomplete linear systems; that is, prove that the morphism associated to a linear system $(\cL, V)$ is an embedding iff
$$
\dim\big( V \cap H^0(\cL(-p-q))\big) = \dim V - 2 \quad \forall p, q \in C.
$$
\end{exercise}

\begin{exercise}
An automorphism of $\PP^r$ takes hyperplanes to hyperplanes. Deduce that it is given by the linear system
$\sV = \sO_{\PP^r}(1), H^0(\sO_{\PP^r}(1))$, and use this to show that $\Aut \PP^r = PGL(r+1)$. 
\end{exercise}

\begin{exercise}
Show that, if $s<r$, then the image of any morphism $\PP^r \to \PP^s$ is a single point.
\end{exercise}

For another example of the relationship between linear series on curves and morphisms of curves to projective space, consider a smooth curve $C \subset \PP^r$ embedded in projective space, and assume that $C$ is linearly normal. If $\phi : C \to C$ is any automorphism, we can ask whether $\phi$ is induced by an automorphism of $\PP^r$; in other words, does there exist an automorphism $\Phi : \PP^r \to \PP^r$ such that $\Phi(C) = C$ and $\Phi|_C = \phi$? The answer is expressed in the following exercise.

\begin{exercise}\label{projective automorphism}
In the circumstances above, the automorphism $\phi$ is induced by an automorphism of $\PP^r$ if and only if $\phi$ carries the invertible sheaf $\cO_{C}(1)$ to itself; that is, $\phi^*(\cO_{C}(1)) = \cO_{C}(1)$.
\end{exercise}

\begin{example}
Consider the morphism of $\PP^1to \PP^d$ given by the complete linear system $|\cO_{\PP^1}(d)|$; this is called the \emph{rational normal curve}. Since there is a unique invertible sheaf of each degree $n$ on $C$, and the curve is linearly normal, we see that \emph{every automorphism of a rational normal curve $C \subset \PP^d$  is projective}, so ``the'' rational normal curve of degree $d$ is well-defined up to an automorphism of $\PP^d$.  A similar statement holds
for the image of any Veronese morphism.
\end{example}

If $\sL, \sL'$ are linear systems on a smooth curve $C$ and $D = (\sigma)_0,D' = (\sigma')_0$ are the divisors of zeros of sections of $\sL$ and $\sL'$ respectively, then $D+D'$ is the divisor of zeros of the section $\sigma\otimes \sigma'$ of
$\sL\otimes \sL'$.

We  often want to consider sections of a given invertible sheaf $\cL$ with bounded singularities: if $D = \sum m_ip_i$ is a divisor, we define the invertible sheaf $\cL(D)$ to be the sheaf of rational sections $\sigma$ of $\cL$ satisfying $\ord_{p_i}(\sigma) \geq -m_i$ for all $i$; as a line bundle, this is the same as $\cL \otimes \cO_C(D)$.

If $\phi:X \to \PP^r$ is a generically finite morphism, then the \emph{degree of $\phi$} is the number of points in the preimage of a general point of $\phi(X)$. Thus, for example, if $D := \sum_{p\in C} n_pp$ is a divisor on a smooth curve, and the linear system $|D|$ is basepoint free, then the degree of the morphism associated to $|D|$ is $\deg D := \sum_{p\in C} n_p$.

\subsection{The most interesting linear system}

The most important invertible sheaf on a smooth variety $X$ is the sheaf of global sections of the top exterior power of the  the cotangent bundle of $X$, called the canonical sheaf $\omega_X$ of $X$ (for canonical sheaves more generally, see Chapter~\ref{dualizing sheaf}). A section of 
$\omega_X$ is thus a differential form of degree equal to the dimension of $X$, and the divisor class
of such a form is usually denoted $K_C$. 
\begin{theorem}
 The canonical sheaf of $\PP^{r}$ is $\sO_{\PP^{n}}(-r-1)$. 
\end{theorem}
\begin{proof}
Let $x_{0}, \dots, x_{r}$ be the projective coordinates on $\PP^{r}$ and let  $U = \PP^{r}\setminus H$ be the affine open set where $x_{0} \neq 0$. Thus $U \cong \AA^{r}$ with coordinates $z_{1 := }x_{1}/x_{0}, \dots, z_{r}:=x_{r}/x_{0}$. The space of $r$-dimensional differential forms on $U$ is spanned by $d(x_{1}/x_{0})\wedge\cdots\wedge d(x_{r}/x_{0})$, which is regular everywhere in $U$. In view of the formula
$$
d\frac{x_{i}}{x_{0}} = \frac{x_{0}dx_{i}-x_{i}dx_{0}}{x_{0}^{2}}
$$
we get
$$
d(x_{1}/x_{0})\wedge\cdots\wedge d(x_{r}/x_{0}) = \frac{dx_{1}\wedge\cdots\wedge dx_{r}}{x_{0}^{r}}-
\sum_{i=1}^{r} x_{i} \frac{ dx_{1}\wedge\cdots \wedge \widehat{d_{x_{i}}}\wedge \cdots \wedge dx_{r}}{x_{0}^{r+1}}
$$
which has a pole of order $r+1$ along the locus $H$ defined by $x_{0}$. Thus the divisor of this differential form
is $-(r+1)H$, and this is the canonical class.
\end{proof}

\begin{fact}
A different derivation: there is a short exact sequence of sheaves of differentials, called the Euler sequence:
$$
0\to \Omega_{\PP^{r}} \to \sO_{\PP^{r}}^{r+1} (-1) \to \sO_{\PP^{r}} \to 0.
$$. 
Taking exterior powers, we see that
$$
\bigwedge^{r}\Omega_{\PP^{r}} \otimes \bigwedge^{1}\sO_{\PP^{r}} = \bigwedge^{r+1} (\sO_{\PP^{r}}^{r+1} (-1)) = \sO_{\PP^{r}}(-r-1).
$$
\end{fact}

Computations of the canonical sheaf on a variety usually involve comparing the variety to another variety, such as projective space, where the canonical sheaf is already known. The most useful results of this type are  the \emph{adjunction formula}
and the \emph{Hurwitz' Theorem}.

\begin{proposition}\label{adjunction}(Adjunction Formula)
 Let $X$ be a variety that is a Cartier divisor on a variety $Y$. If the canonical divisor of $S$ is $K_{Y}$, then
 $K_{X}$ is the restriction to $X$ of the divisor $K_{Y}+X$.
\end{proposition}
This is a special case of \cite[****]{H}.
\begin{proof}
 There is an exact sequence of sheaves
 $$
0\to  \sI_{X/Y}\mid_{X} \to \Omega_{Y}\mid _{X} \to \Omega_{X} \to 0
 $$
 where $\Omega_{X}$ is the sheaf of differential forms on $X$ (see \cite[Theorem ***]{E}), and
$ \sI_{X/Y}\mid_{X} = \sO_{Y}(-X)\mid_{X} = \sO_{X}(-X)$. The proposition follows by taking top exterior powers.
\end{proof}

\begin{corollary}\label{canonical of plane curve}
If $C\subset \PP^{2}$ is a smooth plane curve of degree $d$, then $\omega_{C} = \sO_{C}(d-3)$; more generally, if
$X\subset \PP^{r}$ is a complete intersection of hypersurfaces of degrees $d_{1},\dots, d_{c}$ then
$\omega_{X} = \sO_{X}(\sum_{i}d_{i }-r-1).$
\end{corollary}

 Given a (nonconstant) morphism $f : C \to X$ of smooth projective curves, the Riemann-Hurwitz formula computes the canonical sheaf  $C$ in terms of that of  $X$ and the local geometry $f$. To do this we define the
\emph{ramification index} of $f$ at $p$,  denoted $\ram(f,p)$, by the formula of divisors
$$
 f^{-1}(f(p)) = \sum_{p\in C \mid f(p)=q} (\ram(f,p)+1)\cdot p
 $$
In terms of a suitable choice of local coordinates $z$ on $C$ around $p$ and $w$ on $X$ around $f(p)$, we can write the morphism as $z \mapsto w = z^m$ for some integer $m > 0$, and $\ram(f,p) = m-1$.

It follows from complex analysis (or the separability of field extensions in characteristic 0) that there are only finitely many
points on $C$ where $\ram(f,p) \neq 0$ (this would be false in characteristic $>0$ in the case where the
induced extension of fraction fields was inseparable.) Thus we may define the \emph{ramification divisor} of $f$ to be the divisor
 $$
 R = \sum_{p \in C} \ram(f,p)\cdot p \; \in \;  \Div(C).
 $$
 and the \emph{branch divisor} to be
 $$
 B = \sum_{q \in X} \Big(\sum_{p \in f^{-1}(q)} \ram(f,p) \Big)\cdot q \; \in \; \Div(X).
 $$
 Note that $R$ and $B$ have the same degree $\sum_{p \in C} \ram(f,p)$. 

 
\begin{theorem}(Hurwitz' Theorem) \cite[****]{H} \label{Hurwitz}
If $f:C\to X$ is a non-constant morphism of smooth curves, with ramification divisor $R$, then 
$$
\omega_{C} = f^{*}\omega_{X}(-R).
$$
\end{theorem}
 
 
\begin{proof}
Choose a rational 1-form $\omega$ on $X$, and $\eta = f^*(\omega)$ be its pullback to $C$. For simplicity, we will assume that the zeroes and poles of $\omega$ lie outside the branch divisor $B$, so that $\omega$ will be regular and nonzero at each branch point. (Since we have the freedom to multiply by any rational function on $X$ we can certainly find such a form, and in any event the calculation goes through without this assumption, albeit with more complicated notation.) 

Since the zeroes of $\omega$ lie outside the branch divisor $B$, for every zero of $\omega$ of multiplicity $m$ we have exactly $d$ zeroes of $\eta$, each with multiplicity $m$; and likewise for the poles of $\omega$. Meanwhile, at every point of $B$, the form $\omega$ is regular and nonzero. At a point $p$ where (locally) $f$ has the form $z \mapsto w = z^{e}$
and $\omega = dw,\ \eta dz$ we have $\eta = z^{e-1}dz$; that is $\eta$ has a zero of multiplicity $\ram(f,p)$ at  $p$.
Thus the divisor $K_{C}$ of $\eta$ is
$K_{C} = df^{*}(K_{X})+R$.
\end{proof}

\begin{example}
 Let $V$ be the vector space of homogeneous polynomials of degree $d$ in two variables; that is, $V = H^0(\cO_{\PP^1}(d))$. In the projectivization $\PP(V^{*}) \cong \PP^d$, let $\Delta$ be the locus of polynomials with a repeated factor. Since $\Delta$ is defined by the vanishing of the discriminant, it is a hypersurface. What is its degree?
 
 To answer this, let $W^{*}\subset V^{*}$ be a general 2-dimensional linear subspace---that is, a general pencil of forms of degree $d$ on $\PP^1$. The linear system $\sW = (\sO_{\PP^{d}}, W^{*})$ defines a morphism $\phi_{\sW} : \PP^{1} \to \PP(W) \cong \PP^{1}$ and the fiber over the point of $\PP(W)$ corresponding to a form $f$ of degree $d$ is the divisor $f = 0\subset \PP^{1}$. Thus the locus of polynomials in $W$ with a multiple root is the branch locus of $\phi_{\sW}$, where we count an $m$-fold root $m-1$ times.
 By Hurwitz' formula, the degree of the branch locus $B$ of a degree $d$ morphism from $\PP^{1}$ to $\PP^{1}$ is
 $$
 \deg B = \deg \omega_{\PP^{1}} - d\deg \omega_{\PP^{1}} = 2d-2.
 $$
 \end{example}
 
\begin{fact}
A famous result asserted by Franchetta and proved by **** is that the canonical sheaf (and its powers) are the \emph{only} sheaves that can be chosen uniformly among all, or even almost all, smooth curves. For a more precise statement, see ****.
\end{fact}

\section{Genus, Riemann-Roch and Serre Duality}

We will henceforward assume that the reader is acquainted with sheaf cohomology, at least sufficiently to write
$H^i(X; \s(F)$ or  $H^i(\sF)$ (our preferred form) without blushing. If $D$ is a divisor on a scheme $X$ we will often
abbreviate $H^i(\sO_X(D))$ to $H^i(D)$, and we write $h^i(\sF)$ or $h^{i}(D)$ for $\dim_{\CC}H^i(\sF)$ or $\dim_{\CC}H^i(D)$. Because $h^{i}(\sF)$, for $i>0$, often appears as a kind of ``error term'' in formulas when one would like to compute
$H^{0}(\sF)$, vanishing theorems have an important place in all of algebraic and analytic geometry. We will use the simplest of these often:

\begin{theorem}(Serre Vanishing Theorem)\label{Serre vanishing}
If $\sF$ is a coherent sheaf on $\PP^{n}$ then $H^{i}(\sF(d))= 0$ for all $i>0$ and $d\gg 0$.
\end{theorem}

\subsection{The genus of a curve}

The sole topological invariant of a smooth projective curve $C$ is its genus. We can think of $C$ as a submanifold of the complex projective space $\PP^r(\CC)$ with the classical topology; as such, it is a compact, oriented surface, and its genus is the rank of its first integral homology, $H^{1}(C; \ZZ)$---informally, the ``number of holes'':

\includegraphics[scale = 1]{RiemannSurface}
**** Riemann Surface of genus 3, from Wikimedia ****

Of course this definition does not apply to curves over fields other than $\CC$, and doesn't relate the genus to the algebra of the curve. However, we can relate the topological genus of a curve directly to its topological Euler characteristic
$
\chi_{top}(C) = 2-2g.
$
By the Hopf index theorem, the topological Euler characteristic is the degree of the tangent sheaf, or equivalently, minus the degree of the cotangent sheaf $\omega_{C}$; that is, $\deg K_{C} = 2g-2$, and thus

$$
g(C) = \frac{\deg(K_C)}{2} + 1.
$$
(This formula serves to define the genus of a smooth projective curve over any field).

Other characterizations of the genus require more machinery to establish. We will give some here, and use  tools from the following section to prove equivalence.

\begin{enumerate}

\item\label{genus 1forms} $g(C)$ is the dimension of the vector space of regular 1-forms (that is, global sections of the
cotangent sheaf) on $C$.

\item The (Zariski) Euler characteristic of the structure sheaf of $C$ is $\chi(\cO_C) = h^0(\cO_C) - h^1(\cO_C)$. Since $h^0(\cO_C) = 1$, 
$$
g(C) = 1 - \chi(\cO_C).
$$

Recall that if $X\subset \PP^{r} = \PP(V)$ is any projective scheme, the \emph{homogeneous coordinate ring} of $X$
is the ring $S/I(X)$ where $S = \Sym V \cong \CC[x_{0}, \dots, x_{r}]$ and $I(V)\subset S$ is the ideal of homogeneous
forms that vanish on $X$.

\item\label{genus Hilbert} Suppose that $C \subset \PP^r = \PP(V)$ is a smooth curve of degree $d$  with homogeneous coordinate ring
$S_C$, then the function $d \mapsto \dim_\CC (S_C)_d$ is equal to a polynomial function $p_C(m)$ for large $d$. We have:
$$
p_C(m) =  dm - g + 1,
$$
so $g(C) = 1 - p_C(0)$. 
\end{enumerate}

\subsection{The Riemann-Roch Theorem}

To prove that these formulas for the genus are correct, we use \trr and Serre duality (sometimes called Kodaira-Serre duality, since Kodaira was responsible for the analytic version.)

\begin{theorem}[Riemann-Roch Theorem]\label{RR}
 If $C$ is a smooth, connected projective curve of genus $g$, and $D$ a divisor of degree $d$ on $C$ then
$$
h^0(D) = d - g + 1 + h^0(K_C - D).
$$
\end{theorem}

For example, if we take $D=0$, this tells us that $h^0(K) = g$, proving the characterization~(\ref{genus 1forms}) above. Also, since $h^0(D) = 0$ for any divisor $D$ of negative degree, the formula gives the dimension of $h^{0}(D)$ when $\deg D$ is large:

\begin{corollary}\label{nonspecial RR}
For any divisor of degree $d \geq 2g-1$, we have
$$
h^0(D) = d - g + 1.
$$
\end{corollary}

Using this, we can apply Proposition~\ref{very ample} to show that all high degree divisors come from embeddings:

\begin{corollary}\label{degree 2g+1 embedding}
Let $D$ be a divisor of degree $d$ on a smooth, connected projective curve of genus $g$. If $d \geq 2g$, the complete linear series $|D|$ is base point free; and if $d \geq 2g+1$ the associated morphism $\phi_D : C \to \PP^{d-g}$ is an embedding, so that
$D$ is the preimage of the intersection of $C$ with a hyperplane in $ \PP^{d-g}$.
\end{corollary}

Since the complement of a hyperplane in projective space is an affine space, we get an affine embedding result too:

\begin{corollary}
 If $C$ is any smooth, connected projective curve and $\emptyset \neq \Gamma \subset C$ a finite subset then $C \setminus \Gamma$ is affine.
\end{corollary}
\begin{proof}
Let $D$ be the divisor defined by $\Gamma$ By Corollary~\ref{degree 2g+1 embedding} a high multiple of $D$ is very ample,
and gives an embedding $\phi: C\to \PP^n$ such that the preimage of the intersection of $C$ with some hyperplane $H$
is a multiple of $D$. It follows that $C\setminus \Gamma$ is embedded in $\PP^n\setminus H$.
\end{proof}


We can  use \TRR in the simple case of Corollary~\ref{nonspecial RR} to determine the Hilbert polynomial of a projective curve. To do this, let $C \subset \PP^r$ be a smooth curve of degree $d$ and genus $g$, and consider the exact sequence of sheaves
$$
0 \rTo \cI_{C/\PP^r}(m) \rTo \cO_{\PP^r}(m) \rTo \cO_C(m) \rTo 0
$$
and the corresponding exact sequence
$$
 H^0(\cO_{\PP^r}(m)) \rTo^{\rho_m} H^0(\cO_C(m)) \rTo H^1(\cI_{C/\PP^r}(m)) \rTo 0.
$$

The \emph{Hilbert function} $h_C$ of $C$  is defined by
$$
h_C(m) = \dim_{\CC} (S_{C})_{m} = \rank(\rho_m).
$$
By Theorem~\ref{Serre vanishing} we have $H^1(\cI_{C/\PP^r}(m)) = 0$ for large $m$, so $h_{C}(m) = h^0(\cO_C(m))$, for large $m$, which, by \trr, equals $md-g+1$, again for large $m$. Thus, the Hilbert polynomial of $C \subset \PP^r$ is $p_C(m) = dm-g+1$, establishing the characterization~(\ref{genus Hilbert}).
 
The Riemann-Roch formula does \emph{not} give us a formula for the dimension $h^0(D)$ when $h^0(K_C - D)>0$; such divisors $D$ are called \emph{special divisors}, or \emph{special divisor classes}. The existence or non-existence of divisors $D$ with given $h^{0}(D)$ and $h^{1}(D)$ often serves to distinguish one curve from another, and will be an important part of our study.
 
 
\begin{fact}
 Classically, the dimension $h^0(K_C-D) = h^1(D)$ was called the \emph{superabundance} of $D$: the idea was that a divisor of degree $d$ had, at a minimum, $d-g+1$ sections and $h^1(D)$ represented the number of ``extra" sections. Even though the introduction of cohomology was still almost a century away, the ranks of cohomology groups $h^1$ had classical names, often involving the term superabundance---a premonition of the Riemann-Roch theorem in general.
\end{fact}
 
\begin{fact}
If $k$ is a field that is not algebraically closed there may be genus 0 curves that are not isomorphic to $\PP^1$. However, they must be``forms'' of $\PP^1$ in the sense that they become isomorphic to $\PP^1$ after extension of scalars to 
the algebraic closure $\overline k$ of $k$. The unique example with $k = \RR$ is the conic $x^2+y^2+z^2 = 0$. Indeed, any form of $\PP^1$ over any field $k$ can all be embedded in $\PP_{k}^2$ (by using the anti-canonical linear system.

The curve $\PP_k^1$ itself may be described as the scheme of left ideals of $k$-vector-space dimension 1 in the ring of
$2\times 2$ matrices over $k$ (such an ideal can be embedded in the matrix ring as a linear combination of the 2 columns in an appropriate sense). More generally, any scheme that is a form of $\PP^1$ over $k$
may be described as the scheme of 1-dimensional left ideals in a central simple ($=$ Azumaya) algebra over $k$---though as a set this scheme has no $k$-rational points unless the algebra is the algebra of $2\times 2$ matrices!
\end{fact}
\subsection{Serre duality}

In general, if $\cF$ and $\cG$ are coherent sheaves on a scheme $X$, we have for every $i$ and $j$ a cup product map
$$
H^i(\cF) \otimes H^j(\cG) \to H^{i+j}(\cF \otimes \cG).
$$

\begin{theorem}[Serre Duality]\label{sd} Let $C$ be a smooth connected projective curves with canonical divisor $K$. We have
 $$
h^1(K) = 1
$$
and the cup product map
$$
H^1(D) \otimes H^0(K-D) \to H^1(K)
$$
is a perfect pairing; that is, it induces a natural isomorphism
$$
H^1(D) = H^0(K-D)^*.
$$
\end{theorem}

\subsection{A partial proof}

Combining \TRR and Serre Duality we get:
\begin{corollary}
 If $C$ is a smooth, connected projective curve and $D$ is a divisor on $C$ then
\end{corollary}
$$
\chi(\sO_C(C)) := h^0(D) - h^1(D) = d-g+1
$$
or in other words, for any invertible sheaf $\cL$ of degree $d$ on $C$,
$$
\chi(\cL) = d-g+1
$$
which is pretty easy to prove. To see this, observe that for any invertible sheaf $\cL$ on $C$ and any point $p \in C$ we have an exact sequence of sheaves
$$
0 \to \cL(-p) \to \cL \to \cL_p \to 0.
$$
It follows that $\chi(\cL(-p)) = \chi(\cL) - 1$, so that Riemann-Roch for $\cL$ is equivalent to Riemann-Roch for $\cL(-p)$. Since any divisor can be obtained from 0 by adding and subtracting points, the Riemann-Roch formula for an arbitrary $\cL$ follows from the special case $\cL = \cO_C$.


\section{The canonical morphism}

Given the central role played by the canonical divisor class, it is natural to look at the geometry of the morphism $\phi_K : C \to \PP^{g-1}$ associated to the complete canonical series $|K|$.  By the Riemann-Roch theorem, 
$h^{0}(K) = g(C)$, so $|K|$ cannot define a non-constant morphism unless $g(C)\geq 2$, and cannot define an embedding unless $g(C)\geq 3$.

\begin{definition}
A curve $C$ of genus $g \geq 2$ is said to be \emph{hyperelliptic} if there exists a morphism $f: C \to \PP^1$ of degree 2. \end{definition}

%equivalently, if there exists a invertible sheaf $\cL$ on $C$ of degree 2 with $h^0(\cL) = 2$.

\begin{proposition}
The canonical morphism $\phi_K : C \to \PP^{g-1}$ is an embedding if and only if $C$ is not hyperelliptic.
\end{proposition}

\begin{proof}
By Corollary~\ref{degree 2g+1 embedding} we have to show that for any pair of points $p, q \in C$ we have
$$
h^0(K_C(-p-q)) = h^0(K_C)-2 = g-2.
$$
Applying \trr we see that this would fail if and only if $h^0(\cO_C(p+q)) \geq 2$ for some $p,q \in C$, and by Lemma~\ref{deg 2 morphism} $|p+q|$ would define a degree 2 morphism to $\PP^{1}$. 
\end{proof}

\begin{lemma}\label{deg 2 morphism}
Let $C$ be a smooth, projective curve of genus $g\geq 2$. Any invertible sheaf of degree 2 on $C$ defines a morphism to $\PP^{1}$. In particular, if $g(C) = 2$ then the canonical series $|K_{C}|$ defines a 2 to 1 morphism to $\PP^{1}$.
\end{lemma}

\begin{proof}
 If this happens,
we claim that $\cO_C(p+q)$ is basepoint free, so that $C$ is hyperelliptic. To finish the proof, by Corollary~\ref{degree 2g+1 embedding} it suffices to show that
 an invertible sheaf $\sL$ of degree 1 on $C$ must have $h^{0}(\sL)\leq 1$.
 
Suppose that $\sigma_{0}, \sigma_{1}$ were two linearly independent sections of $\sL$. Each $\sigma_{i}$ vanishes at a unique point $p_{i}$. If $p_{0}= p_{1}$ then a linear combination of $\sigma_{0}, \sigma_{1}$ would be a section vanishing to order $\geq 2$, which is impossible, so $\sL$ is basepoint free, and defines a degree 1 morphism $C\to \PP^{1}$. Such a morphism must be an isomorphism (because $\PP^{1}$ is normal), contradicting $g(C) \geq 2$.
\end{proof}

\fix{the following argument is only set-theoretic. Admit this or make it precise}
Note that if $C$ is hyperelliptic, the morphism $\phi_K$ factors through the degree 2 morphism $\pi : C \to \PP^1$: if $\{p,q\} \subset C$ is a fiber of this morphism, we have $h^0(\cO_C(p+q)) = 2$ and hence $\phi_K(p) = \phi_K(q)$. The image of the morphism $\phi_K$ is a nondegenerate curve of degree $g-1$ in $\PP^{g-1}$, which we will see is a \emph{rational normal curve}. This observation implies in particular that if $C$ is hyperelliptic of genus $g \geq 2$, then the invertible sheaf $\cL$ of degree 2 with $h^0(\cL) = 2$ is in fact unique.

Among curves with $g \geq 3$ the hyperelliptic curves are very special: in the family of all curves, as we'll see, they comprise a closed subvariety. Also, the behavior of linear series and morphisms on a hyperelliptic curve is very different from that of series on a general curve; when we discuss the geometry of curves of low genus in the Chapter~\ref{}, we will exclude  the hyperelliptic case, and deal with this case in a separate chapter.

For non-hyperelliptic curves, however, the geometry of the canonical morphism, and its image, the canonical curve, are the keys to understanding the curve. We'll see this in detail in many cases in the following chapter; for now, we mention one highly useful result along these lines.

\fix{add here: canonical series on plane curves cut by $|\cO_{\PP^2}(d-3)|$; consequence that no smooth plane curve can be hyperelliptic}

\fix{maybe move initial discussion of hyperelliptic curves from Ch. 6 to a section here}

\fix{maybe add to this chapter: differentials on plane curves $C$, possibly with nodes or more general singularities; adjoint conditions; algorithm for determining the complete linear system associated to a divisor $D$ on $C$}

\subsection{The geometric Riemann-Roch theorem}

Let's state this first in a relatively simple case: let $C$ be a nonhyperelliptic curve, embedded in $\PP^{g-1}$ by its canonical series and let $D = p_1+\dots + p_d$ be a divisor consisting of $d$ distinct points; let $\overline D$ be the span of the points $p_i \in C \subset \PP^{g-1}$. Since the hyperplanes in $\PP^{g-1}$ containing $\{p_1,\dots,p_d\}$ correspond (up to scalars) to sections of $K_C$ vanishing at all the points $p_i$, we see that
$$
h^0(K_C-D) = g - 1 - \dim \overline D.
$$
Plugging this into the Riemann-Roch formula, we arrive at the statement
$$
r(D) = d - 1 - \dim \overline D;
$$
or in other words, \emph{the dimension of the linear series $|D|$ in which the divisor $D$ moves is equal to the number of linear relations on the points $p_i$ on the canonical curve}. Thus, for example, if $D = p_1+p_2+p_3$, we see that $D$ moves in a pencil if and only if the points $p_i$ are collinear.

We can extend this statement to the case of arbitrary effective divisors $D$ (and even hyperelliptic curves) if we define our terms correctly. To do this, suppose $f : C \to \PP^d$ is any morphism, and $D \subset C$ any divisor. We define the \emph{span} of  $f(D)$ to be the intersection
$$
\overline{f(D)} = \bigcap_{H \mid f^{-1}(H)\supset D} H 
$$
of all hyperplanes in $\PP^d$ whose preimage in $C$ contains $D$. 

\begin{theorem}[Geometric Riemann-Roch Theorem]\label{geometric RR}
If $C$ is any curve of genus $g \geq 2$,  $\phi : C \to \PP^{g-1}$ its canonical morphism and $D \subset C$ any effective divisor of degree $d$, then
$$
r(D) = d - 1 - \dim \overline{\phi(D)}.
$$
\end{theorem}

 \fix{I moved the hyperelliptic section to be with the curves of genus 2}
 

\section{Moduli problems}

It is a fundamental aspect of algebraic geometry that the objects we deal with often vary in families, and can often be parametrized by a ``universal" such family. For example, the family of plane curves of degree $d$ may be thought of as the projective
space $\PP(H^{0} \sO_{\PP^{2}}(d))$, and similarly with hypersurfaces in any projective space. This notion of objects varying with parameters underlies many of the constructions and theorems we will discuss. 

\subsection{What is a moduli problem?}

Briefly, a \emph{moduli problem} consists of two things: a class of objects, or isomorphism classes of objects; and a notion of what it means to have a \emph{family} of these objects parametrized by a given scheme $B$. To make this relatively explicit, the four main examples of moduli problems we'll be discussing here are:

\begin{enumerate}
\item  smooth curves: objects are isomorphism classes  of smooth, projective curves $C$ of a given genus $g$. A family over $B$ is a subscheme $\cX \subset B \times \PP^r$, smooth, over $B$, whose fibers are curves of genus $g$.

\item the Hilbert scheme: objects are subchemes of $ \PP^r$ with a given Hilbert polynomials. A family  is a subscheme $\cX \subset B \times \PP^r$, with $cX$ flat over $B$, whose fibers have the given Hilbert polynomial. We will be interested in the case of Hilbert polynomial $p(m) = dm-g+1$ and the open subscheme corresponding to smooth projective curves $C \subset \PP^r$ of degree $d$ and genus $g$.

\item effective divisors on a given curve: objects are effective divisors of a given degree $d$ on a given smooth, projective curve $C$. A family over $B$ will be a subscheme $\cD \subset B \times C$ flat over $B$, with fibers of degree $d$

\item invertible sheaves on a given curve $C$: objects are invertible sheaves of a given degree $d$ on $C$. A family over $B$ is an invertible sheaf on the product $B \times C$ whose restriction to each fiber over $B$ has degree $d$. We identify two such sheaves if they differ by tensor product with an invertible sheaf pulled back from $B$.
\end{enumerate}

Given a moduli problem, our goal will be to describe a corresponding \emph{moduli space}. By this we mean a scheme $M$ whose points are in \emph{natural} one-to-one correspondence with the objects in our moduli problem. This will realize the objects of the moduli problem as the points of the underlying set of the scheme $M$.

If the moduli space in question and the base of the family are varieties, then the crucial condition that the correspondence be \emph{natural} is simple to express: that given a family of the objects in our moduli problem over a variety $B$, the map from underlying set of $B$ to the underlying set of $M$ taking each fiber
to the corresponding point of $M$ should be a morphism of varieties.  But in the world of schemes the set-theoretic mapping does not determine the morphism of schemes (think, for example, of the morphisms from $\Spec(\CC[x]/x^{2})$ into the plane with the closed point mapping to the origin. The situation is even worse when the moduli space itself is not a variety.)

To deal with the general case, we recast the naturality condition in functorial terms. We observe first that a moduli problem defines a functor $\cM$ from the category of schemes to the category of sets: the value of the functor at a scheme $B$ is the set of families of objects parametrized by $B$; a morphism $B' \to B$ of schemes gives rise, via pullback, to a map of sets $\cM(B) \to \cM(B')$. We define a \emph{fine moduli space} for the moduli problem to be a scheme $M$ that represents this functor, in the sense that there is an isomorphism of functors
$$
\cM \to {\rm Mor}(\bullet, M)
$$
In other words, for every scheme $B$ we have a bijection between families of our objects over $B$ and morphisms from $B$ to $M$. In particular, applying this to $B = \Spec \CC$, we have a bijection between the set of objects and the closed points of $M$; and for any family over an arbitrary scheme $B$, the map from $B(\CC)$ to $M(\CC)$ sending each closed point  $b \in B$ to the point in $M(\CC)$ corresponding to the fiber over $b$ is the underlying map of a morphism $B \to M$ of schemes.

If a fine moduli space for a given problem exists at all, then Yoneda's Lemma shows that it is unique up to a unique isomorphsm. This is a real problem: there is no fine moduli space for the first and most important of the examples above---the isomorphism classes of smooth curves---though there is for the others. We'll defer the discussion of why this is, and what we can do about it, until Chapter~\ref{Moduli chapter}.

Looking ahead, we'll discuss the third and fourth example in Chapter~\ref{}, where we'll describe the moduli spaces for effective divisors of given degree $d$ on a given curve $C$ (the symmetric powers of the curve) and for invertible sheaves of a given degree on $C$ (the \emph{Jacobian} and \emph{Picard variety} of $C$). These, as we'll see, are smooth, irreducible projective varieties of dimensions $d$ and $g$ respectively.

We'll take up the moduli space $M_g$ of smooth curves in Chapter~\ref{Moduli chapter}, where we'll see that this space (or rather the closest approximation to it we can cook up) is irreducible of dimension $3g-3$ for $g \geq 2$, though not smooth or projective.

Finally, the Hilbert scheme will be described (to the extent that we can!) in Chapter~\ref{HilbertSchemesChapter}; this will turn out to be much wilder and more varied in its behavior than any of the above.


%The definition is straightforward: by a family of smooth, projective curves $C$ of a given genus $g$ over a given base variety $B$, we mean a flat morphism
%$\cC \to B$ whose fibers are smooth, projective curves $C$ of genus $g$


\input footer.tex