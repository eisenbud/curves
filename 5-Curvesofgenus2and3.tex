%header and footer for separate chapter files

\ifx\whole\undefined
\documentclass[12pt, leqno]{book}
\usepackage{graphicx}
\input style-for-curves.sty
\usepackage{hyperref}
\usepackage{showkeys} %This shows the labels.
%\usepackage{SLAG,msribib,local}
%\usepackage{amsmath,amscd,amsthm,amssymb,amsxtra,latexsym,epsfig,epic,graphics}
%\usepackage[matrix,arrow,curve]{xy}
%\usepackage{graphicx}
%\usepackage{diagrams}
%
%%\usepackage{amsrefs}
%%%%%%%%%%%%%%%%%%%%%%%%%%%%%%%%%%%%%%%%%%
%%\textwidth16cm
%%\textheight20cm
%%\topmargin-2cm
%\oddsidemargin.8cm
%\evensidemargin1cm
%
%%%%%%Definitions
%\input preamble.tex
%\input style-for-curves.sty
%\def\TU{{\bf U}}
%\def\AA{{\mathbb A}}
%\def\BB{{\mathbb B}}
%\def\CC{{\mathbb C}}
%\def\QQ{{\mathbb Q}}
%\def\RR{{\mathbb R}}
%\def\facet{{\bf facet}}
%\def\image{{\rm image}}
%\def\cE{{\cal E}}
%\def\cF{{\cal F}}
%\def\cG{{\cal G}}
%\def\cH{{\cal H}}
%\def\cHom{{{\cal H}om}}
%\def\h{{\rm h}}
% \def\bs{{Boij-S\"oderberg{} }}
%
%\makeatletter
%\def\Ddots{\mathinner{\mkern1mu\raise\p@
%\vbox{\kern7\p@\hbox{.}}\mkern2mu
%\raise4\p@\hbox{.}\mkern2mu\raise7\p@\hbox{.}\mkern1mu}}
%\makeatother

%%
%\pagestyle{myheadings}

%\input style-for-curves.tex
%\documentclass{cambridge7A}
%\usepackage{hatcher_revised} 
%\usepackage{3264}
   
\errorcontextlines=1000
%\usepackage{makeidx}
\let\see\relax
\usepackage{makeidx}
\makeindex
% \index{word} in the doc; \index{variety!algebraic} gives variety, algebraic
% PUT a % after each \index{***}

\overfullrule=5pt
\catcode`\@\active
\def@{\mskip1.5mu} %produce a small space in math with an @

\title{Personalities of Curves}
\author{\copyright David Eisenbud and Joe Harris}
%%\includeonly{%
%0-intro,01-ChowRingDogma,02-FirstExamples,03-Grassmannians,04-GeneralGrassmannians
%,05-VectorBundlesAndChernClasses,06-LinesOnHypersurfaces,07-SingularElementsOfLinearSeries,
%08-ParameterSpaces,
%bib
%}

\date{\today}
%%\date{}
%\title{Curves}
%%{\normalsize ***Preliminary Version***}} 
%\author{David Eisenbud and Joe Harris }
%
%\begin{document}

\begin{document}
\maketitle

\pagenumbering{roman}
\setcounter{page}{5}
%\begin{5}
%\end{5}
\pagenumbering{arabic}
\tableofcontents
\fi


\chapter{Hyperelliptic curves and curves of genus 2 and 3}\label{genus 2 and 3 chapter}

\section{Hyperelliptic Curves}\label{hyperelliptic}
 
 
Hyperelliptic curves are outliers: they behave differently from other curves, and the techniques used to analyze them are different from the techniques used for more general curves. Many theorems about curves contain the hypothesis ``non-hyperelliptic," with the corresponding result for hyperelliptic curves arrived at directly by ad hoc methods. Because the methods of this section will not be used in other cases, it could be skipped in first reading
 
There will be a further discussion of hyperelliptic curves in Chapter~\ref{ScrollsChapter}, focussing on the algebra and geometry of their projective embeddings; the analysis here will cover most of the questions we'll be asking about curves in general in the next four chapters. 
  
 \subsection{The equation of a hyperelliptic curve}
 
By definition, a hyperelliptic curve $C$ a curve of genus $\geq 2$ admitting a degree two map $\pi : C \to \PP^1$. Because the degree is only 2, each point in $\PP^1$ has either two distinct preimages, or only one, a ramification point. 
The ramification index of such a point $p$, defined as 1 less than the length of the fiber $\pi^{-1}(\pi(p)$, is 1, and there can be no higher ramification. Because at all but finitely many points $p \in C$ the map $\pi$ is a local isomorphism (``local" here in the complex analytic/classical or \'etale topology, not the Zariski topology!); at any other point $p \in C$, the map is given in terms of local analytic coordinates on $C$ and $\PP^1$ simply by $z \mapsto z^2$. In particular, both the ramification divisor and the branch divisor (as defined in Chapter~\ref{linear systems chapter}) are reduced. Thus by the Riemann-Hurwitz formula there are exactly $2g+2$ branch points $q_1,\dots,q_{2g+2} \in \PP^1$. These points determine the curve:
  
 
\begin{theorem}\label{hyperelliptic existence}
There is a unique smooth projective hyperelliptic curve $C$ expressible as a 2-sheeted cover of $\PP^1$ branched over any set of $2g+2$ distinct points.
\end{theorem}

\begin{proof} 
We can easily construct such a curve, postponing for a moment the uniqueness:
If the coordinate of the point $p_i \in \PP^1$ is $\lambda_i$, it is the smooth projective model of the affine curve 
  $$
C^\circ = \big\{ (x,y) \in \AA^2 \; \mid \; y^2 = \prod_{i=1}^{2g+2} (x - \lambda_i) \big\}.
$$ 
Note that we're choosing a coordinate $x$ on $\PP^1$ with the point $x = \infty$ at infinity not among the $q_i$, so that the pre-image of $\infty \in \PP^1$ is two points $r, s \in C$. Concretely, we see that as $x \to \infty$, the ratio $y^2/x^{2g+2} \to 1$, so that 
$$
\lim_{x \to \infty} \; \frac{y}{x^{g+1}} \; = \; \pm 1;
$$
  the two possible values of this limit correspond to the two points $r,s \in C$.
  
  It's worth pointing out that $C$ is \emph{not} simply the closure of the affine curve $C^\circ \subset \AA^2$ in either $\PP^2$ or $\PP^1 \times \PP^1$: as you can see from a direct examination of the equation, each of these closures will be singular at the (unique) point at infinity.
  
   The remainder of the proof of Theorem~\ref{hyperelliptic existence} will be done in a more general situation in Section~\ref{branched covers} below.
   
 \begin{exercise}
  In the case $g=1$, show that the closure $\overline{C^\circ}$ of $C^\circ \subset \AA^2$ in either $\PP^2$ or $\PP^1 \times \PP^1$ consists of the union of $C^\circ$ with one additional point, with that point a tacnode of $\overline{C^\circ}$ in either case.
  \end{exercise}
  
It is also possible to give a projective model of the hyperelliptic curve $C$ with given branch divisor: if we divide the points $q_1,\dots,q_{2g+2} \in \PP^1$ into two sets of $g+1$---say, for example, $q_1,\dots,q_{g+1}$ and $q_{g+2}, \dots, q_{2g+2}$---then $C$ is the closure in $\PP^1 \times \PP^1$ of the  locus
  $$
  \big\{ (x,y) \in \AA^2 \; \mid \; y^2\prod_{i=1}^{g+1} (x - \lambda_i) = \prod_{i=g+2}^{2g+2} (x - \lambda_i) \big\};
  $$
  in projective coordinates, this is
   $$
  C \; = \; \big\{ (X,Y) \in \PP^1 \times \PP^1 \; \mid \; Y_1^2\prod_{i=1}^{g+1} (X_1 - \lambda_iX_0) = Y_0^2\prod_{i=g+2}^{2g+2} (X_1 - \lambda_iX_0) \big\}.
  $$
  (No local analysis is needed to see that $C \subset \PP^1 \times \PP^1$ is smooth: it is a curve of bidegree $(2,g+1)$ in $\PP^1 \times \PP^1$, and the formula for the genus of a curve in $\PP^1 \times \PP^1$ derived in Section~\ref{rational quartics section} tells us that such a curve has arithmetic genus $g$.)
  
 The map $\iota : C \to C$ that exchanges the two points in each reduced fiber of the map $C \to \PP^1$ and fixes the ramification points is algebraic: in terms of the last representation of $C$, it is given by $([X_0,X_1], [Y_0,Y_1]) \mapsto  ([X_0,X_1], [Y_0,-Y_1]) $). The map $\iota$ is called the \emph{hyperelliptic involution} on $C$.

  
  \subsection{Differentials on a hyperelliptic curve}

We can give a very concrete description of the differentials, and thus the canonical linear system, on a hyperelliptic curve $C$ by working with the affine model $C^\circ = V(f) \subset \AA^2$, where
$$
f(x,y) = y^2 - \prod_{i=1}^{2g+2} (x - \lambda_i).
$$
We will again denote the two points at infinity---that is, the two points of $C \setminus C^\circ$ by $r$ and $s$; for convenience, we'll denote the divisor $r+s$ by $D$.

To start, consider the differential $dx$ on $C$. (Technically, we should write this as $\pi^*dx$, since we mean the pullback to $C$ of the differential $dx$ on $\PP^1$, but for simplicity of notation we'll suppress the $\pi^*$.)  The function $x$ is regular on $C^\circ$, and is a local parameter over points other than the $\lambda_i$; from the local description of the map $\pi$, we see that $dx$ is regular on $C^\circ$  with simple zeros at the ramification points $q_i = (\lambda_i, 0)$. But it does not extend to a regular differential on all of $C$: it will have double poles at $r$ and $s$.  This can be seen directly: the differential $dx$ extends to a rational differential on $\PP^1$, and in terms of the local coordinate $w = 1/x$ around the point $x = \infty$ on $\PP^1$, we have
$$
dx = d\left(\frac{1}{w}\right) = \frac{-dw}{w^2}
$$
so $dx$ has a double pole at the point at $\infty$; since the map $\pi$ is a local isomorphism near $r$ and $s$ the pullback of $dx$ to $C$ likewise has double poles at the points $r$ and $s$. Thus the canonical
divisor of $C$ is 
$$
K_C \sim (dx) \sim R - 2D,
$$
where $R$ denotes the ramification divisor, in this case the sum of the ramification points. 
This is a special case of the Riemann-Hurwitz of Chapter~\ref{linear systems chapter}.

By the Riemann-Roch Theorem, the degree of $K_C$ is $2g-2$, and this confirms the existence of
a total of 4 poles of $dx$ at the two points at infinity. By symmetry we see again that each of those point is a double pole.

How can we find differentials that are regular everywhere on $C$? If we divide $dx$ by $x^2$ (or any quadratic polynomial in $x$) to kill the poles we  introduce new poles in the finite part $C^\circ$ of $C$. 

Instead, we want to multiply $dx$ by a rational function with zeros at $p$ and $q$, but \emph{whose poles occur only at the points where $dx$ has zeroes}---that is, the points $q_i$.  A natural choice the reciprocal of the partial derivative $f_y := \partial f/ \partial y = 2y$, which vanishes at the points $q_i$, and has  a pole of order $g+1$ at each of the points $r$ and $s$ (reason: the involution $y\to -y$ fixes $C^\circ$ and $x$, and exchanges the points $r$ and $s$). In other words, as long as $g \geq 1$, the differential
$$
\omega = \frac{dx}{f_y}
$$
is regular, with divisor
$$
(\omega) = (g-1)r + (g-1)s = (g-1)D.
$$
The remaining regular differentials on $C$ are now easy to find: Since $x$ has only a simple pole
at the two points at infinity we can  multiply $\omega$ by any $x^k$ with $k = 0, 1, \dots, g-1$. Since this gives us $g$ independent differentials, we see that the differentials
$$
\omega, x\omega, \dots, x^{g-1}\omega
$$
  form a basis for $H^0(K_C)$.

\subsection{The canonical map of a hyperelliptic curve}\label{hyperelliptic special}

Given that a basis for $H^0(K_C)$ is given by 
$$
H^0(K_C) = \langle \omega, x\omega,\dots,x^{g-1}\omega \rangle,
$$
we see that the canonical map $\phi : C \to \PP^{g-1}$ is given by $[1,x,\dots,x^{g-1}]$. In other words, the canonical map $\phi$ is  the composition of the map $\pi : C \to \PP^1$ with the Veronese embedding $\PP^1 \hookrightarrow \PP^{g-1}$ of $\PP^1$ into $\PP^{g-1}$ as a rational normal curve of degree $g-1$.

Note that as a consequence of this fact, we see that \emph{a hyperelliptic curve $C$ has a \emph{unique} linear system $g^1_2$ of degree 2 and dimension 1}: the $g^1_2$ is characterized as the complete linear system $|D|$, where $D$ is the sum of any two points mapping to the same point under the canonical map. Equivalently, we see that $C$ admits  a unique map of degree 2 to $\PP^1$. 

Finally, we can give an explicit description of special linear system on a hyperelliptic curve: if $D = \sum p_i$ is any effective divisor on $C$, we can pair up points $p_i$ that are conjugate under the involution $\iota$ exchanging sheets of the degree 2 map $C \to \PP^1$; each conjugate pair is a divisor of the unique $g^1_2$ on $C$, and so we can write
$$
D \sim r\cdot g^1_2 + q_1 + \dots + q_{d-2r},
$$
where no two of the points $q_i$ are conjugate under $\iota$. Now the geometric form of the Riemann-Roch formula tells us that the dimension $r(D)$ of the complete linear system $|D|$ is exactly $r$, so that in fact 
$$
|D| = |r\cdot g^1_2| + q_1 + \dots + q_{d-2r};
$$
that is, the points $q_i$ are base points of the linear system $|D|$.

We see in particular that no special linear system on a hyperelliptic curve can be very ample: the map associated to any special system factors through the degree 2 map $C \to \PP^1$. This is in marked contrast to the case of non-hyperelliptic curves of genus $g \geq 3$, for which, as we shall see in Chapter~\ref{BNChapter}, the embeddings of minimal degree in projective space are given by special linear system.

   \section{Interlude: branched covers with specified branch divisor}\label{branched covers}
   
   We will describe here the classification of branched covers $\pi : C \to B$ of a given curve $B$ with specified branch points. To give ourselves a concrete goal, we will address the question: given a curve $B$ and a collection $\Delta = \{p_1,\dots,p_b\} \subset B$ of points in $B$, how many branched covers $\pi : C \to B$ of degree $d$ are there with specified branching over each of the points $p_i$, up to isomorphism over $B$? Our analysis will consist of two steps: first, we will reduce the problem to the classification of topological covering spaces of the complement $U = B \setminus \Delta$; we will then use our knowledge of the fundamental group of $U$ to enumerate such covering spaces.
   
\begin{theorem}
 Let $B$ be a smooth curve, let $\Delta\subset B$ be a finite set of points, and let $U := B\setminus Delta$.
If $\pi^\circ : V \to U$ is a topological covering space then $V$ may be given the structure of a Riemann surface in a unique way so that the map $\pi^\circ$ is holomorphic; and $V$ may be compactified to a compact Riemann surface in $C$ a unique way such that the map $\pi^\circ$ extends to a holomorphic map $\pi : C \to B$.
\end{theorem} 

\begin{proof}
The space $V$ inherits the structure of a complex manifold from $U$ because if $D \subset U$ is any simply connected coordinate chart, then the preimage $({\pi^\circ})^{-1}(D)$ is a disjoint union of $d$ copies of $D$, and we may use them as coordinate charts on $V$. 
   
To compactify $V$ we observe that if $D^* = \{ z \in \CC \mid 0 < |z| < 1 \}$ is a punctured disc, then
the map $D\to D: z \mapsto z^n$ restricts to a connected $n-fold$ covering space $D^*\to D^*$. 
Since $\pi_1(D^*) = \ZZ$, any connected covering space $E$ of degree $n$ is homeomorphic to this one
by a homeomorphism inducing the identity on the target of $\pi$.
If we  define a holomorphic structure on $E$ by pulling back the one on $D$, then
this homeomorphism is biholomorphic.

Thus if $D_i$ is a small neighborhood of the point $p_i \in B$ biholomorphic to a disc, then the preimage  in $V$ of the punctured disc $D_i^* := D_i \cap U$ is a disjoint union of punctured discs, and $V$ can then be compactified to a compact Riemann surface in a unique way by completing each one to a full disc.
\end{proof}
   
 The problem of classifying smooth curves $C$ that have a map $\pi : C \to B$ of degree $d$ is thus the problem becomes one of classifying covering spaces of $U$. 
    
 \subsection{Branched covers of $\PP^1$} 

We continue with the notation $U = B\setminus \Delta$, now supposing that $B = \PP^1_\CC$, the Riemann sphere. Again, let $\pi:V\to U$ be a covering space.

Choose a base point $p_0 \in U$, and draw simple, non-intersecting arcs $\gamma_i$ joining $p_0$ to $p_i$ in $U$. If $\Sigma$ is the complement of the union of these arcs in the sphere, then the preimage of $\Sigma$ in $V$ will be the disjoint union of $d$ copies of $\Sigma$, called the \emph{sheets} of the cover; label these $\Sigma_1,\dots,\Sigma_d$
   
  \includepdf{"pic41"}
   
   Now, a covering space $V \to U$ of $U$ is determined by its \emph{monodromy}, which is the map
   $$
   \pi_1(U, p_0) \to S_d
   $$
   associating to each path  $\beta$ in $U$ from $p_0$ to $p_0$  the permutation of the points of $\pi^{-1}(p_0)$ given by sending a point $q \in \pi^{-1}(p_0)$ to the endpoint of the unique lift of $\beta$ starting at $q$. Let $\beta_i$ be the path starting at $p_0$, going out along the arc $\gamma_i$ until just short of $p_i$, going once around $p_i$ and then going back to $p_0$ along the same path $\gamma_i$, as drawn in the Figure. Let $\tau_i$ i the corresponding permutation of $\{1,2,\dots,d\}$. The paths $\beta_i$  generate the fundamental group $\pi_1(U, p_0)$, with the one relation that the product $\beta_1\cdot \dots \cdot \beta_b = Id$ is the identity; the permutations $\tau_1,\dots,\tau_b$ thus determine the covering space $V$.
   
\includepdf{"pic42"} 
 
If we assume that the covering
$\pi : C \to B$ is simply branched, so each $\tau_i$ is a transposition; and if we assume that
$C$, and thus $V$, is connected, then the subgroup $\langle \tau_1, \dots, \tau_b \rangle \subset S_d$ is transitive. If we were to relabel the sheets $\Sigma_\alpha$ according to some permutation $\sigma$, the effect would be to conjugate each $\tau_i$ by $\sigma$. 

Summarizing we have proven:   
   \begin{lemma}\label{branched cover classification}
   Let $p_1,\dots, p_b \in \PP^1$ be any $b$ given points. There is a natural bijection between 
   \begin{enumerate}
   \item the set of  simply branched covers $\pi : C \to \PP^1$ of degree $d$, branched over the points $p_i$, up to isomorphism over $\PP^1$; and
   \item the set of $b$-tuples of transpositions $\tau_1, \dots, \tau_b \in S_d$ such that $\prod \tau_i = id$ and such that $\tau_1, \dots, \tau_b$ generate a transitive subgroup of $S_d$, modulo simultaneous conjugation by $S_d$.
   \end{enumerate}
   \end{lemma}

In the case $d=2$ that is relevant to hyperelliptic curves, we note that there is only one transposition in $S_2$. Thus there is a unique double cover of $\PP^1$ with given branch points $p_1,\dots,p_b$. The product
condition shows again that the number of branch points must be even. . This completes the proof of Theorem~\ref{hyperelliptic existence}.
\end{proof}

We can use the same technique to count the number of 3 to 1 branched covers  $C \to \PP^1$ with given branch points, using that fact that every odd permutation $\tau \in S_3$ is a transposition. Thus if $b$ is even and  $\tau_1,\dots,\tau_{b-1} \in S_3$ are arbitrary transpositions, then the product 
$\tau_1\cdot \cdots,\tau_{b}$ is also a
 transposition. It follows that the number of $b$-tuples of transpositions $\tau_1,\dots,\tau_{b} \in S_3$ with $\prod \tau_i$ equal to the identity is $3^{b-1}$. The requirement that the group generated by the $\tau_i$ is transitive eliminates just the three cases where all the $\tau_i$ are equal. The group $S_3$ acts on the set of $b$-tuples of permuations without stabilizing any $b$-tuple, so every cover corresponds to exactly 6 collections $\tau_1,\dots,\tau_b$. In sum, the number of simply branched three-sheeted covers of $\PP^1$ with specified branch points $q_1,\dots,q_b \in \PP^1$ is
$$
\frac{3^{b-1} - 3}{6} \; = \; \frac{3^{b-2} - 1}{2} 
$$


\subsection{More general covers}\label{general covers}

As we indicated, there are natural extensions of this basic theory to covers with arbitrary branching, and to covers of curves of any genus. For the first, suppose that instead of specifying ``simply branched"---meaning that the fiber over any branch point consists of $d-2$ simple points and one double point---we specified that the fiber over a given branch point $q_i$ consisted of $m_1$ simple points, $m_2$ double points, $m_3$ triple points and so on. This amounts to specifying a conjugacy class in $S_d$, and instead of associating to $q_i$ a transposition, we want to specify an element $\tau_i$ of this conjugacy class. Again, the permutations $\tau_i$ have to have product the identity, and to generate a transitive subgroup of $S_d$; and again, they are determined up to simultaneous conjugation by an element of $S_d$.

\begin{exercise}
Find the number of 3-sheeted covers $C \to \PP^1$ of genus $g$ with simple branching except for one point of total ramification (that is, one point with just a single preimage point.)
\end{exercise}

Similarly, we can enumerate covers of curves $B$ of genus $g>0$ by a modification of the construction above. Choosing a base point $p \in B$ we can draw $2g$ loops $\alpha_1,\dots,\alpha_{g},\beta_1, \dots, \beta_g$ based at $p$ and disjoint except for $p$ so that the complement $B \setminus \cup \alpha_i$ is a disc with boundary $\alpha_1, \beta_1, \alpha_1^{-1}, \beta_1^{-1}, \dots, \alpha_g, \beta_g, \alpha_g^{-1}, \beta_g^{-1}$. If we draw arcs $\gamma_i$ joining $p$ to each of the branch points $q_i$, we can associate to a cover a collection of permutations $\sigma_1, \dots, \sigma_b, \mu_1,\dots,\mu_g, \nu_1,\dots,\nu_g$ (with $\sigma_i$ a transposition, if we are assuming simple branching, or of specified conjugacy class in general, and the $\mu_i$ and $\nu_i$ arbitrary). These, up to conjugacy, determine the cover, and have to satisfy the relation
$$
\prod_{i=1}^b \sigma_i = \prod_{\alpha=1}^g \; [\mu_\alpha, \nu_\alpha]
$$
and generate a transitive subgroup of $S_d$

\begin{exercise}
Let $B$ be a curve of genus $h$. How many unramified double covers of $B$ are there? 
\end{exercise}


\begin{exercise} Let $E$ be a curve of genus 1, and $q_1,\dots,q_b \in E$. How many double covers $C \to E$ are there branched over the $q_i$?
\end{exercise}


\begin{exercise} Let $E$ be a curve of genus 1, and $q, q' \in E$. How many triple covers $C \to E$ are there simply branched over $q$ and $q'$?
\end{exercise}


In general, the numbers of branched covers of a given curve with given branching are called \emph{Hurwitz numbers}; they are of interest to physicists, for reasons we can't fathom.



\section{Curves of genus 2}

Since  curves of genus 2 are hyperelliptic, everything we said above applies to them; in particular, the canonical map $\phi_K : C \to \PP^1$ on a curve of genus 2 is the expression of $C$ as a double cover of $\PP^1$, branched over 6 points in $\PP^1$, which are unique up to automorphisms of $\PP^1$. 

In this section, we'll consider other maps from a hyperelliptic curves $C$ to projective space, starting with maps $C \to \PP^1$.

\subsection{Maps of $C$ to $\PP^1$}\label{genus 2 pencil}

Of course, $C$ may be expressed as a degree 2 cover of $\PP^1$ by taking the map associated to the canonical system $|K_C|$. But what about other degrees? For example, can we express $C$ as a three-sheeted cover of $\PP^1$?

The answer is ``yes," and in fact we can do so in many ways. First, start with an invertible sheaf $L$ of degree 3 on $C$. Since $3 > 2g-2$, Riemann-Roch tells us immediately that $h^0(L) = 2$, and we see that there are two possibilities:

\begin{enumerate}
\item First, if the linear system $|L|$ has a base point $p \in C$, then $h^0(L(-p)) = 2$, and hence $L$ must be of the form $L = K_C(p)$. Conversely, if $L = K_C(p)$, then $h^0(L(-p)) = h^0(L)$, which is to say $p$ is a base point of $|L|$.
\item On the other hand, if $L$ is not of the form $L = K_C(p)$, then $|L|$ does not have a base point, and so defines a degree 3 map $\phi_L : C \to \PP^1$.
\end{enumerate}

Do both possibilities occur? Certainly the first does; there's a one-parameter family of invertible sheaves of the form $K_C(p)$. But we know that the variety $\Pic^3(C)$ is 2-dimensional, so we see that the general invertible sheaf of degree 3 does  give an expression of $C$ as a 3-sheeted cover of $\PP^1$; in fact there exists a 2-parameter family of such maps.

There are plenty of higher-degree maps as well: an invertible sheaf of degree $d \geq 4 = 2g+2$ is automatically base-point free, and gives a map to $\PP^{d-2}$, from which we can project in many ways
to $\PP^1$.

\subsection{Maps of $C$ to $\PP^2$} Next consider maps of our curve $C$ of genus 2 to the plane. By theRiemann-Roch theorem, an invertible sheaf $L$ of degree 4 on $C$ will have $h^0(L) = 3$; and since $h^0(L(-p)) = 2$ for any point $p \in C$ (again by Riemann-Roch), we see that the linear system $|L|$ will give a regular map $\phi_L : C \to \PP^2$. We ask now about the geometry of this map.

There are three possibilities:

\begin{enumerate}
\item First, suppose $L = K_C^2$ is  the square of the canonical sheaf on $C$. We have then a map
$$
\Sym^2 H^0(K_C) \to H^0(L).
$$
Recalling that the elements of $H^0(K_C)$ may be written as $\omega, x\omega$ we see that
the map is injective, and since both sides are 3-dimensional vector spaces, we have equality. In other words, every divisor $D \sim K_C^2$ is the sum of two divisors $D_1, D_2 \in |K_C|$ in the canonical system. We conclude that the map $\phi_L$ is the composition of the canonical map $\phi_K : C \to \PP^1$ with the Veronese embedding $\nu_2 : \PP^1 \to \PP^2$ of $\PP^1$ as a conic curve in the plane and the map $\phi_L$ is generically 2-to-1 onto the conic.

\item Suppose now that $L$ is not equal to $K_C^2$; equivalently, $M = L \otimes K_C^{-1}$ is an invertible sheaf
of the form $\sO_C(p+q)$ for some pair of points $p,q$. Since $h^0(M) = 1$, this pair is unique, and $L = K_C(p+q)$ for a unique pair of points $p, q \in C$.

If $p \neq q$ then every section of $L$ vanishing at $p$ vanishes at $q$ and vice versa, so that $\phi_L(p) = \phi_L(q)$. However, for any effective divisor $D = r+s$ of degree 2 on $C$  other than $p+q$, we have $h^0(L(-D)) = 1$, so apart from the fact that $\phi_L(p) = \phi_L(q)$, the map $\phi_L$ is an embedding. We'll see in Exercise~\ref{nodal quartic} below that the point $\phi_L(p) = \phi_L(q)$ is an ordinary node of the image curve $\phi_L(C)$. Thus if $L = K_C(p+q)$, with $p \neq q$ and $p+q \neq K_C$, then the map $\phi_L : C \to \PP^2$ is a birational embedding of $C$ as a quartic plane curve with one node, the node being the common image of $p$ and $q$.

\item In the remaining case $L = K_C(2p)$, where $p \in C$ is any point such that $2p \not\sim K_C$. Here the map $\phi_L$ is one-to-one with vanishing differential at $p$, and the image curve $\phi_L(C)$ has  a cusp at the point $\phi_L(p)$.
\end{enumerate}

To summarize: the map $\phi_L : C \to \PP^2$ associated to an invertible sheaf $L$ of degree 4 on $C$ is either
\begin{enumerate}
\item two-to-one onto a plane conic curve, if $L = K_C^2$;
\item birational onto a plane quartic curve with a cusp, if $L = K_C(2p)$ with $2p \not\sim K_C$; and
\item birational onto a plane quartic curve with a node, if $L = K_C(p+q)$ with $p \neq q$ and $p+q \not\sim K_C$.
\end{enumerate}

Note that the last case is the ``general" one, meaning it holds for $L$ in an open subset of $\Pic^4(C)$; the second case holds for a one-dimensional locus in $\Pic^4(C)$, and the first case holds for just one point in $\Pic^4(C)$.

\begin{exercise}\label{nodal quartic}
Let $L \in \Pic^4(C)$ be an invertible sheaf of the form $L = K_C(p+q)$ with $p \neq q$ and $p+q \not\sim K_C$. Show that
\begin{enumerate}
\item $h^0(L(-2p)) = h^0(L(-2q)) = 1$, and
\item $h^0(L(-2p-2q)) = 0$.
\end{enumerate}
Deduce from this that the map $\phi_L$ is an immersion, and that the tangent lines to the two branches of $\phi_L(C)$ at the point $\phi_L(p) = \phi_L(q)$ are distinct, meaning the point $\phi_L(p) = \phi_L(q)$ is a node of $\phi_L(C)$.
\end{exercise}


\subsection{Embeddings in $\PP^3$}

By Corollary~\ref{degree 2g+1 embedding} any invertible sheaf $L$ of degree 5 is very ample. 
Write $\phi_L : C \to \PP^3$ for the map given by the complete linear system $|L|$. Since $\phi_L$ is an embedding, we'll also denote the image $\phi_L(C) \subset \PP^3$ by $C$ and write $\cO_C(1)$ for $L$.

What degree surfaces in $\PP^3$ contain the curve $C$? We start with degree 2, and consider the restriction map
$$
H^0(\cO_{\PP^3}(2)) \to H^0(\cO_C(2)) = H^0(L^2).
$$
The space on the left has dimension 10; by the Riemann-Roch Theorem we have $h^0(L^2) = 2\cdot5 - 2 + 1 = 9$. It follows that $C$ lies on a quadric surface $Q$. Since $C$ a plane or a union of planes, any quadric containing $C$ is irreducible; if there were more than one such, Bezout's Theorem would imply that $\deg(C) \leq 4$. Thus $Q$ is unique.

We might ask at this point: is $Q$ smooth or a quadric cone? The answer depends on the choice of invertible sheaf $L$. 

\begin{proposition}\label{genus 2 embedding}
Let $C \subset \PP^3$ be a smooth curve of degree 5 and genus 2 and $Q \subset \PP^3$ the unique quadric containing $C$. If $L = \cO_C(1) \in \pic^5(C)$, then $Q$ is singular if and only if we have
$$
L \cong K^2(p)
$$
for some point $p \in C$; in this case, the point $p$ is the vertex of $Q$.
\end{proposition}

Note that the variety $\pic^5(C)$ has dimension 2, while the sheaves of the form $K^2(p)$ form a one-dimensional subfamily. Thus in general $Q$ will be smooth; the set of invertible sheaves $L$
for which the quadric is singular is 1-dimesnional, and thus of codimension one.

\begin{proof}
First, suppose that the invertible sheaf $L \cong K^2(p)$ for some $p \in C$. Then $L(-p) \cong K^2$, so that the map $\pi : C \to \PP^2$ given by projection from $p$ is the map $\phi_{K^2} : C \to \PP^2$ given by the square of the canonical sheaf. As we've seen, the map $\phi_{K^2}$ is two-to-one onto a conic curve $E \subset \PP^2$, so that the curve $C$ lies on the cone $Q$ over $E$ with vertex $p$, and this is the unique quadric surface containing $C$.

Next, let's consider the case where $L$ is not of the form $K^2(p)$. Set $M = LK^{-1}$, so that we can write
$$
L = K \otimes M,
$$
where by hypothesis $M$ is not of the form $K(p)$. As we saw in Section~\ref{genus 2 pencil}, this means that the pencil $|M|$ gives a degree 3 map $C \to \PP^1$.

This gives us a way of factoring the map $\phi_L : C \to \PP^3$: we have maps $\phi_K : C \to \PP^1$ of degree 2 and $\phi_M : C \to \PP^1$ of degree 3, and we can compose their product with the Segre embedding $\sigma : \PP^1 \times \PP^1 \to \PP^3$:
\begin{diagram}
& & \PP^1 & & & &\\
& \ruTo^{\phi_K} & & \luTo & & & \\
C & & \rTo^{\phi_K \times \phi_M} & & \PP^1 \times \PP^1 & \rTo^\sigma & \PP^3 \\
& \rdTo^{\phi_M} & & \ldTo & & & \\
& & \PP^1 & & & & \\
\end{diagram}

This description of the map $\phi_L$  shows  that \emph{$C$ is a curve of type $(2,3)$ on a smooth quadric $Q \subset \PP^3$}, completing the proof of Proposition~\ref{genus 2 embedding}.
\end{proof}

Whether the quadric $Q$ is smooth or not, we can describe a minimal set of generators of the homogeneous ideal $I(C) \subset \CC[x_0, x_1, x_2, x_3]$ similarly. First, we look at the restriction map
$$
H^0(\cO_{\PP^3}(3)) \to H^0(\cO_C(3));
$$
since the dimensions of these spaces are 20 and $15-2+1 = 14$ respectively, we see that  vector space of cubics vanishing on $C$ has dimension at least 6. Only four of these are multiples of the defining equation of $Q$ linear forms on $\PP^3$. It follows that there are at least two cubics vanishing on $C$ that are linearly independent modulo those vanishing on $Q$.

We can identify these cubics geometrically, and show that there are no more than 2 linearly independent curbics modulo the ideal of $Q$. Suppose first that $Q$ is smooth, so that $C$ is a curve of type $(2,3)$ on $Q$. In that case, if $L \subset Q$ is any line of the first ruling, the sum $C+L$ is the complete intersection of $Q$ with a cubic $S_L$, unique modulo the ideal of $Q$; conversely, if $S$ is any cubic containing $C$ but not containing $S$, the intersection $S \cap Q$ will be the union of $C$ and a line $L$ of the first ruling; thus, mod $I(Q)$, $S = S_L$. A similar argument applies in case $Q$ is a cone, and $L$ is any line of the (unique) ruling of $Q$.

\begin{exercise}
Show that for any pair of lines $L, L'$ of the appropriate ruling of $Q$, the three polynomials $Q$, $S_L$ and $S_{L'}$ generate the homogeneous ideal $I(C)$. Find relations among them. Write out the minimal resolution of $I(C)$.
\end{exercise}



\subsection{The dimension of $M_2$ via maps to projective space}

Each of the types of maps that we described from a curve $C$ of genus 2 to projective space suggests the dimension of the moduli space $M_2$ of curves of genus 2.

To start, we know that every curve $C$ of genus 2 is uniquely expressible as a double cover of $\PP^1$ branched at six points, modulo the group $PGL_2$ of automorphisms of $\PP^1$. The space of such double covers has dimension 6, and $\dim(PGL_2) = 3$, and since the group acts with finite stabilizers this suggests that $\dim(M_2) = 6-3 = 3$.

Similarly, we've seen that a curve $C$ of genus 2 is expressible as a 3-sheeted cover of $\PP^1$ (with eight branch points) in a 2-dimensional family of ways. As we saw in the preceding section, such a triple cover is determined up to a finite number of choices by its branch divisor, so the space of such triple covers has dimension 8; modulo $PGL_2$ it has dimension 5, and since every curve is expressible as a triple cover in a two-dimensional family of ways, we arrive again at $\dim M_2 = 5-2 = 3$.

We've also seen that $C$ can be realized as (the normalization of) a plane quartic curve with a node in a 2-dimensional family of ways. The space of plane quartics has dimension 14; the family of those with a node has codimension one (\ref{PlaneCurvesChapter}) and hence dimension 13. Since  the automorphism group $PGL_3$ of $\PP^2$ has dimension 8, we see that the family of nodal plane quartics modulo $PGL_3$ has dimension 5, and since every curve of genus 2 corresponds to a 2-parameter family of such curves, we have $\dim M_2 = 5-2=3$.

Finally, a curve of genus 2 may be realized as a quintic curve in $\PP^3$ in a two-parameter family of ways. To count the dimension of the family of such curves, note that each one lies on a unique quadric $Q$, and is of type $(2,3)$ on $Q$. Thus to specify such a curve we have to specify $Q$ (9 parameters) and then a bihomogeneous polynomial of bidegree $(2,3)$ on $Q \cong \PP^1 \times \PP^1$ up to scalars; these have $3\cdot 4 - 1 = 11$ parameters. Altogether, then, there is a 20-dimensional family of such curves; modulo the automorphism group $PGL_4$ of $\PP^3$, this is a 5-dimensional family. Again, every abstract curve $C$ of genus 2 corresponds to a 2-parameter family of these curves modulo $PGL_4$, so once more we have $\dim M_2 = 5 - 2 = 3$.

\section{Curves of genus 3}

If $C$ be a smooth projective curve of genus 3. The is an immediate bifurcation into two cases, hyperelliptic and non-hyperelliptic curves; we will discuss hyperelliptic curves of any genus in Section~\ref{hyperelliptic}, and so for the following we'll assume $C$ is nonhyperellitic. By our general theorem~\ref{canonical system is very ample}, this means that the canonical map $\phi_K : C \to \PP^2$ embeds $C$ as a smooth plane quartic curve; and conversely, by Proposition~\ref{Adjunction Formula} any smooth plane curve of degree 4 has genus 3 and is embedded by the complete canonical system (that is, $\cO_C(1) \cong K_C$). 

Note that this gives us a way to determine the dimension of the moduli space $M_3$ of smooth curves of genus $3$: if $\PP^{14}$ is the space of all plane quartic curves, and $U \subset \PP^{14}$ the open subset corresponding to smooth curves, we have a dominant map $U \to M_3$ whose fibers are isomorphic to the 8-dimensional affine group $PGL_3$. (Actually, the fiber over a point $[C] \in M_3$ is isomorphic to the quotient of $PGL_3$ by the automorphism group of $C$; but since $Aut(C)$ is finite this is still 8-dimensional.) We conclude, therefore, that
$$
\dim M_3 = 14 - 8 = 6.
$$

What about other linear system on $C$, and the corresponding models of $C$? By hypothesis $C$ is not a 2-sheeted cover of $\PP^1$. On the other hand, it is a 3-sheeted cover: if $L \in \pic^3(C)$ is an invertible sheaf of degree 3 then, by the Riemann-Roch Theorem, we have
$$
h^0(L) = 
\begin{cases}
2, &\text{if $L \cong K-p$ for some point $p \in C$; and} \\
1 &\text{otherwise.}
\end{cases}
$$
There is thus a 1-dimensional family of representations of $C$ as a 3-sheeted cover of $\PP^1$. These are  visible directly from the canonical model: a degree 3 map $\phi_{K-p} : C \to \PP^1$ is the composition of the canonical embedding $\phi_K : C \to \PP^2$ with a projection from $p$. 

There are other representations of $C$ as the normalization of a plane curve. By the Riemann-Roch Theorem, $C$ has no $g^2_3$, and the canonical system is the only $g^2_4$, but there are plenty of models as plane quintic curves: by Proposition~\ref{very ample}, if $L$ is any invertible sheaf of degree 5, the linear system $|L|$ will be a base-point-free $g^2_5$ as long as $L$ is not of the form $K+p$, so that $\phi_L$ maps $C$ birationally onto a plane quintic curve $C_0 \subset \PP^2$. These can also be described geometrically in terms of the canonical model: any such invertible sheaf $L$ is of the form $2K-p-q-r$ for some trio of  points $p, q, r \in C$ that are not colinear in the canonical model, and we see  that $C_0$ is obtained from the canonical model of $C$ by applying a Cremona transform with respect to the points $p, q$ and $r$, that is, by applying the birational transformation
of the plane defined by the linear series of conics through $p,q,r$.

We can also embed $C$ in $\PP^3$ as a smooth sextic curve by Theorem~\ref{g+3 theorem}. In this case we can also apply Proposition~\ref{very ample} to conclude that an invertible sheaf $L \in \pic^6(C)$ of degree 6 will be very ample if and only if it is not of the form $K+p+q$ for any $p, q \in C$. 

\begin{fact}
If $C \subset \PP^3$ is a smooth non-hyperelliptic curve of degree 3 and genus 6, then there exists a $3 \times 4$ matrix $M$ of linear forms on $\PP^3$ such that 
$$
C = \{ p \in \PP^3 \mid \rank(M(p)) \leq 2 \}.
$$
This is an application of the ``Hilbert-Burch Theorem"; see ~\cite{geomsyz}
\end{fact}


\section{Theta characteristics}

In this section we sketch the algebraic theory of theta-characteristics, mostly as it applies to curves of genus 3.

Suppose that $C \subset \PP^2$ is a smooth plane curve. A \emph{bitangent} to $C$ is a line $L \subset \PP^2$ that is either tangent to $C$ at two distinct points, or has contact of order $\geq 4$ with $C$ at a point. Alternatively, we can say that a bitangent  corresponds to an effective divisor of degree 2 on $C$ such that $2D$ is contained in the intersection of $C$ with a line $L \subset \PP^2$.

A naive dimension count suggests that a smooth plane curve should have a finite number of bitangents (it's one condition on a line $L \in {\PP^2}^*$ to be tangent to $C$, so it should be two conditions for it to be bitangent). Indeed, this is the case; By Bezout's Theorem a conic or cubic curve cannot have any bitangents, but it is known that every smooth curve of degree $d \geq 4$ has 
$$
12\binom{d+1}{4} - 4d(d-2),
$$
counted with appropriate multiplicities~\cite[p. 282]{Griffiths-Harris1978}. For example, a line tangent to $C$ at 3 points  counts as three bitangents. Accordingly, a smooth plane quartic has 28 bitangents.

The bitangents to a plane quartic $C$ have a special significance: since $4 = 2 \times 2$, if $D = p+q$ is a bitangent, then the divisor $2D$ comprises the complete intersection of $C$ with a line; in other words, we have a linear equivalence
$$
2D \sim K_C
$$
or equivalently the invertible sheaf $\cO_C(D)$ is a square root of the canonical sheaf of $C$. Because of their appearance in the theory of theta functions, Riemann named the square roots of the canonical sheaf theta-characteristics.

How many such square roots are there? If $\cL$ and $\cM$ are invertible sheaves with $\cL^2 = \cM^2 = K$, then $\cL$ and $\cM$ differ by an invertible sheaf of order 2; that is,
$$
\cM = \cL \otimes \cF, \quad \text{where} \quad \cF \otimes \cF \sim \cO_C.
$$
In other words, $\cF$ is an invertible sheaf of degree 0, and  corresponds to a point of order 2 in the Picard group $\Pic_0(C)$. Since we've seen that $\Pic_0(C) = Jac(C)$ is a complex torus of dimension g = 3---the quotient of $\CC^3$ by a lattice $\Lambda \cong \ZZ^6$---we see that there are $2^6 = 64$ such invertible sheaves, and thus, given that there is some invertible sheaf $\cL$ satisfying $\cL^2 \cong K_C$, there are exactly $64 = 2^{2g}$ of them.

 \begin{exercise}
 Let $C$ be a curve of genus 2, expressed as a 2-sheeted cover of $\PP^1$ with ramification points $p_1,\dots,p_6$
 \begin{enumerate}
 \item Show that the theta-characteristics on $C$ are either of the form $\cL = \cO_C(p_i)$ or of the form $\cL = \cO_C(p_i + p_j - p_k)$ with $i, j, k$ distinct. 
 \item Show that in the first case we have $h^0(\cL) = 1$, and in the second case we have $h^0(\cL) = 0$. 
 \item Finally, show that there are six of the former kind, and 10 of the latter, making $2^4 = 16$ in all.
 \end{enumerate}
 \end{exercise}
 
The reader will have noticed that the number 64 of theta-characteristics does not agree with the number 28 of bitangents. The reason is easy to see: bitangents correspond to \emph{effective} divisors $D$ with $2D \sim K$, while a theta characteristic $\cL$ may have $h^0(\cL) = 0$, that is, may not correspond to an effective divisor. What can we say about the dimensions $h^0(\cL)$ of the space of sections of the theta-characteristics on $C$?
 
 There is a beautiful partial answer to this question, which can be deduced from a remarkable fact: \emph{the dimension $h^0(\cL)$ of the space of sections of a theta characteristic mod 2 is invariant under deformation}:
  
 \begin{theorem}\label{locally constant sign} Let $\cC \to B$ be a family of smooth curves, and $\cL_b$ a family of theta characteristics on the curves in this family---in other words, an invertible sheaf $\cL$ on $\cC$ such that $(\cL|_{C_b})^2 \cong K_{C_b}$ for each $b \in B$. If we define a function $f : B \to \ZZ/2$  by
 $$
 f(b) = h^0(\cL|_{C_b}) \;  \; (\text{mod } 2)
 $$
then $f$ is locally constant.
\end{theorem}

We say that a theta-characteristic $\cL$ is \emph{even} or \emph{odd} according to the parity of $h^0(\cL)$. Given the irreducibility of the moduli space $M_g$,  Theorem~\ref{locally constant sign} says that all curves of genus $g$ have the same number of even theta characteristics given by: 

\begin{theorem}\label{number of theta characteristics}
If $C$ is any curve of genus $g$, then $C$ has $2^{g-1}(2^g - 1)$ even theta characteristics and $2^{g-1}(2^g+1)$ odd theta characteristics.
\end{theorem}

Note that in case of a nonhyperelliptic curve $C$ of genus 3, the dimension $h^0(\cL)$ of a theta characteristic $\cL$ cannot be $\geq 2$, so this says exactly that there are $2^{g-1}(2^g-1) = 28$ effective theta-characteristics corresponding to the 28 bitangents.

We'll sketch here the proofs of these theorems; for details, see~\cite{MumfordPaper} and~\cite{JHPaper}. 

The proof of Theorem~\ref{locally constant sign} follows, via an ingenious construction of Mumford's, from an elementary fact about quadratic forms in an even number of variables. To state this, suppose that $V$ is a complex vector space of dimension $2n$ and $Q$ a nondegenerate symmetric bilinear form on $V$. By an \emph{isotropic plane} we mean a linear space $\Lambda \subset V$ such that $Q(\Lambda, \Lambda) = 0$. 

\begin{fact}
 \begin{enumerate}
\item The maximal isotropic subspaces for $Q$ have dimension $n$;
\item The set of maximal isotropic subspaces for $Q$ is a subvariety of the Grassmannian $G(n,V)$, or dimension $\binom{n}{2}$ and having exactly two connected components (the ``rulings" of $Q$); and
\item If $\Lambda, \Lambda' \subset V$ are any two maximal isotropic subspaces, then
$$
\dim(\Lambda \cap \Lambda') \equiv n \text{ (mod 2)} \quad \iff \quad \Lambda, \Lambda' \text{ belong to the same ruling.}
$$
\end{enumerate} 
\end{fact}

The first of these assertions is completely elementary: since the map $\tilde Q : V \to V^*$ associated to the form $Q$ carries an isotropic subspace to its annihilator, there can't be an isotropic plane of dimension $>n$; and similarly if $\Lambda \subset V$ is any isotropic plane of dimension $<n$ we can include $\Lambda$ in a larger isotropic plane by adding any nullvector for the induced quadratic form on $\ann(\Lambda)/\Lambda$.

The second and third assertions are less elementary, but the reader may already have seen the first two nontrivial cases of each. When $n=2$, they amount to observing that a smooth quadric surface in $\PP^3$ has two rulings by lines, and lines of opposite rulings meet in a point, while lines of the same ruling are either disjoint or equal. When $n=3$, we can see both assertions via the identification of a smooth quadric in $\PP^5$ with the Grassmannian $\GG(1,3)$: the two families of maximal isotropic subspaces are the loci of lines containing a given point $p \in \PP^3$ and the loci of lines contained in a given plane $H \subset \PP^3$.

Now suppose that $C$ is a smooth curve of genus $g$, and $\cL$ an invertible sheaf on $C$ with $\cL^2 \cong K_C$; that is, a theta characteristic. Choose a divisor $D = p_1 + \dots + p_n$ of degree $n> g-1$ consisting of distinct points, and let $V$ be the $2n$-dimensional vector space
$$
V := H^0( \cL(D) / \cL(-D) ).
$$
(Note that the sheaf $ \cL(D) / \cL(-D) $ is supported on $D$, with stalk of dimension 2 at each $p_i \in D$.) We can define a quadratic form on $V$ by setting
$$
Q(\sigma, \tau) := \sum_i \text{Res}_{p_i}(\sigma \cdot \tau)
$$
where we use the isomorphism $\cL^2 \cong K_C$ to identify the product $\sigma\tau$ with a rational differential.

We now introduce two isotropic subspaces for $Q$: first, we set
$$
\Lambda := H^0( \cL / \cL(-D) );
$$
this is isotropic simply because the product $\sigma\tau$ of two elements of $\Lambda$ corresponds to a regular differential, and so has no residues. Second, we set
$$
\Lambda' := \im\left( H^0(\cL(D)) \to H^0( \cL(D) / \cL(-D) ) \right)
$$
Since the $H^0(\cL(-D)) = 0$, the map is injective and according to Riemann-Roch, $h^0(\cL(D)) = n$, so this is again an $n$-dimensional subspace of $V$; it's isotropic because the sum of the residues of a global rational differential on $C$ is 0. Finally, the payoff: we observe that
$$
H^0(\cL) \cong \Lambda \cap \Lambda',
$$
and Theorem~\ref{locally constant sign} follows.

As for Theorem~\ref{number of theta characteristics}, there are two ways we can go. One is to dig a little deeper and actually describe the configurations of odd and even theta characteristics as subsets of the set $S$ of all theta characteristics, which as we've seen is a principle homogeneous space for the group $\Jac(C)_2 \cong (\ZZ/2\ZZ)^{2g}$ of points of order 2 on the Jacobian. Alternatively, we can use the invariance of these configurations under deformation to reduce to the hyperelliptic case; the result in that case is expressed in the following exercise.

\begin{exercise}
Let $C$ be a hyperelliptic curve of genus $g$, expressed as a 2-sheeted cover of $\PP^1$ with ramification points $p_1,\dots,p_{2g+2}$. If $\cE$ is the (unique) $g^1_2$ on $C$, show that any theta characteristic on $C$ is represented by a divisor
$$
D = m\cdot \cE + \sum_{i \in I} p_i
$$
for some $m \geq -1$ and $I \subset \{1,\dots, 2g+2\}$ a subset of cardinality $g-1-2m$; and that in this case
$$
h^0(D) = m+1.
$$
Using this, count the number of even and odd theta characteristics on $C$ (and thus on any curve of genus $g$).
\end{exercise}

Of course, if we do describe the configurations $S^+$ and $S^-$ of even and odd theta characteristics, we get more information. We won't go in to that here---again, the result is in~\cite{MumfordPaper} and~\cite{JHPaper}---but we will mention one remarkable fact that we can derive from it.

As noted, if we choose any theta characteristic on a curve $C$, we may identify the set $S^-$ of odd theta-characteristics with a subset of the group $\Jac(C)_2$ of points of order 2 on the Jacobian of $C$. We might expect, then, that some 4-tuples of these points will add up to 0 in $\Jac(C)$; in other words, there should exist some 4-tuples $\cL_1,\dots,\cL_4 \in S^-$ such that
$$
\cL_1+ \dots +\cL_4 = 2K_C.
$$
What this means in the specific case of genus $g=3$ is: \emph{among the 28 bitangents to a smooth plane quartic curve $C$, there will be some subsets of 4 whose eight points of tangency form the intersection of $C$ with a plane conic curve}; if we know the configuration $S^-$, we should be able to say how many. Indeed, the number was first found by Salmon; it is 315.
\input footer.tex


