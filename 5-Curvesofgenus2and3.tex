%header and footer for separate chapter files

\ifx\whole\undefined
\documentclass[12pt, leqno]{book}
\usepackage{graphicx}
\input style-for-curves.sty
\usepackage{hyperref}
\usepackage{showkeys} %This shows the labels.
%\usepackage{SLAG,msribib,local}
%\usepackage{amsmath,amscd,amsthm,amssymb,amsxtra,latexsym,epsfig,epic,graphics}
%\usepackage[matrix,arrow,curve]{xy}
%\usepackage{graphicx}
%\usepackage{diagrams}
%
%%\usepackage{amsrefs}
%%%%%%%%%%%%%%%%%%%%%%%%%%%%%%%%%%%%%%%%%%
%%\textwidth16cm
%%\textheight20cm
%%\topmargin-2cm
%\oddsidemargin.8cm
%\evensidemargin1cm
%
%%%%%%Definitions
%\input preamble.tex
%\input style-for-curves.sty
%\def\TU{{\bf U}}
%\def\AA{{\mathbb A}}
%\def\BB{{\mathbb B}}
%\def\CC{{\mathbb C}}
%\def\QQ{{\mathbb Q}}
%\def\RR{{\mathbb R}}
%\def\facet{{\bf facet}}
%\def\image{{\rm image}}
%\def\cE{{\cal E}}
%\def\cF{{\cal F}}
%\def\cG{{\cal G}}
%\def\cH{{\cal H}}
%\def\cHom{{{\cal H}om}}
%\def\h{{\rm h}}
% \def\bs{{Boij-S\"oderberg{} }}
%
%\makeatletter
%\def\Ddots{\mathinner{\mkern1mu\raise\p@
%\vbox{\kern7\p@\hbox{.}}\mkern2mu
%\raise4\p@\hbox{.}\mkern2mu\raise7\p@\hbox{.}\mkern1mu}}
%\makeatother

%%
%\pagestyle{myheadings}

%\input style-for-curves.tex
%\documentclass{cambridge7A}
%\usepackage{hatcher_revised} 
%\usepackage{3264}
   
\errorcontextlines=1000
%\usepackage{makeidx}
\let\see\relax
\usepackage{makeidx}
\makeindex
% \index{word} in the doc; \index{variety!algebraic} gives variety, algebraic
% PUT a % after each \index{***}

\overfullrule=5pt
\catcode`\@\active
\def@{\mskip1.5mu} %produce a small space in math with an @

\title{Personalities of Curves}
\author{\copyright David Eisenbud and Joe Harris}
%%\includeonly{%
%0-intro,01-ChowRingDogma,02-FirstExamples,03-Grassmannians,04-GeneralGrassmannians
%,05-VectorBundlesAndChernClasses,06-LinesOnHypersurfaces,07-SingularElementsOfLinearSeries,
%08-ParameterSpaces,
%bib
%}

\date{\today}
%%\date{}
%\title{Curves}
%%{\normalsize ***Preliminary Version***}} 
%\author{David Eisenbud and Joe Harris }
%
%\begin{document}

\begin{document}
\maketitle

\pagenumbering{roman}
\setcounter{page}{5}
%\begin{5}
%\end{5}
\pagenumbering{arabic}
\tableofcontents
\fi


\chapter{Hyperelliptic curves and curves of genus 2 and 3}\label{genus 2 and 3 chapter}

\section{Hyperelliptic Curves}\label{hyperelliptic}
 
We met hyperelliptic curves in Chapter 2, and proved there that, in genus $g \geq 2$, the canonical
map is two to one onto the rational normal curve of degree $g-1$ in $\PP^{g-1}$. We used this to show that every special divisor on a hyperelliptic curve is a sum of divisors in the 
unique $g^1_2$ plus base points. We will begin this chapter with an explicit construction of hyperelliptic curves and use it to give a very concrete computation of the canonical series, reproving what we did in Chapter 2. Then we will consider the projective embeddings of curves of genus 2 (which are all hyperelliptic) and genus 3.
 
There will be a further discussion of hyperelliptic curves in Chapter~\ref{ScrollsChapter}, focussing on the algebra and geometry of their projective embeddings. 
  
 \subsection{The equation of a hyperelliptic curve}
 
Recall that a hyperelliptic curve $C$ a curve of genus $\geq 2$ admitting a degree two map $\pi : C \to \PP^1$, and we have seen that such a map is unique. Because the degree is only 2, each point in $\PP^1$ has either two distinct preimages, or only one, a ramification point with ramification index 1, where the map is given in terms of local analytic coordinates on $C$ and $\PP^1$ by $z \mapsto z^2$. In particular, both the ramification divisor and the branch divisor (as defined in Chapter~\ref{linear systems chapter}) are reduced. By the Hurwitz' formula there are exactly $2g+2$ branch points $q_1,\dots,q_{2g+2} \in \PP^1$. These points determine the curve:
 
\begin{theorem}\label{hyperelliptic existence}
There is a unique smooth projective hyperelliptic curve $C$ expressible as a 2-sheeted cover of $\PP^1$ branched over any set of $2g+2$ distinct points.
\end{theorem}

\begin{proof} 
We can easily construct such a curve, postponing for a moment the uniqueness:
If the coordinate of the point $p_i \in \PP^1$ is $\lambda_i$, it is the smooth projective model of the affine curve 
  $$
C^\circ = \big\{ (x,y) \in \AA^2 \; \mid \; y^2 = \prod_{i=1}^{2g+2} (x - \lambda_i) \big\}.
$$ 
Note that we're choosing a coordinate $x$ on $\PP^1$ with the point $x = \infty$ at infinity not among the $q_i$, so that the pre-image of $\infty \in \PP^1$ is two points $r, s \in C$. Concretely, we see that as $x \to \infty$, the ratio $y^2/x^{2g+2} \to 1$, so that 
$$
\lim_{x \to \infty} \; \frac{y}{x^{g+1}} \; = \; \pm 1;
$$
  the two possible values of this limit correspond to the two points $r,s \in C$.
  
  It's worth pointing out that $C$ is \emph{not} simply the closure of the affine curve $C^\circ \subset \AA^2$ in either $\PP^2$ or $\PP^1 \times \PP^1$: as you can see from a direct examination of the equation, each of these closures will be singular at the (unique) point at infinity.
  
   The remainder of the proof of Theorem~\ref{hyperelliptic existence} will be done in a more general situation in Section~\ref{branched covers} below.
  
To give a smooth projective model of the hyperelliptic curve $C$ with given branch divisor, we divide the $2g+2$ branch points  into two sets of $g+1$---say, for example, $q_1,\dots,q_{g+1}$ and $q_{g+2}, \dots, q_{2g+2}$. We can then take $C$ to be the closure in $\PP^1 \times \PP^1$ of the  locus
  $$
  \big\{ (x,y) \in \AA^2 \; \mid \; y^2\prod_{i=1}^{g+1} (x - \lambda_i) = \prod_{i=g+2}^{2g+2} (x - \lambda_i) \big\};
  $$
  in projective coordinates, this is
   $$
  C \; = \; \big\{ (X,Y) \in \PP^1 \times \PP^1 \; \mid \; Y_1^2\prod_{i=1}^{g+1} (X_1 - \lambda_iX_0) = Y_0^2\prod_{i=g+2}^{2g+2} (X_1 - \lambda_iX_0) \big\}.
  $$
To see that $C \subset \PP^1 \times \PP^1$ is smooth we note that it is a curve of bidegree $(2,g+1)$ in $\PP^1 \times \PP^1$, and the formula for the genus of a curve in $\PP^1 \times \PP^1$ derived in Example~\ref{Div of quadric} tells us that such a curve has arithmetic genus $g$, and thus no singular points.
  
 The map $\iota : C \to C$ that exchanges the two points in each reduced fiber of the map $C \to \PP^1$ and fixes the ramification points is algebraic: in terms of the last representation of $C$, it is given by $([X_0,X_1], [Y_0,Y_1]) \mapsto  ([X_0,X_1], [Y_0,-Y_1]) $). The map $\iota$ is called the \emph{hyperelliptic involution} on $C$.

  \subsection{Differentials on a hyperelliptic curve}\label{hyperelliptic differentials}

We can give a pleasantly concrete description of the differentials, and thus the canonical linear system, on a hyperelliptic curve $C$ by working with the affine model $C^\circ = V(f) \subset \AA^2$, where
$$
f(x,y) = y^2 - \prod_{i=1}^{2g+2} (x - \lambda_i).
$$
We will again denote the two points at infinity---that is, the two points of $C \setminus C^\circ$ by $r$ and $s$; for convenience, we'll denote the divisor $r+s$ by $D$. We write $\pi:C \to \PP^1$ for the morphism that, on $C^\circ$, sends $(x,y) \in C$ to $x$.

We can construct a differential form on $C$ by following the proof of Hurwitz' Theorem of Chapter~\ref{linear systems chapter}.
Let $dx$ denote the usual different on $\PP^1$ having a double poll at infinity, and consider $\pi^*dx$ on $C$.  The function $x$ is regular on $C^\circ$, and is a local parameter over points other than the $\lambda_i$; from the local description of the map $\pi$, we see that $\pi^*dx$ is regular on $C^\circ$  with simple zeros at the ramification points $q_i = (\lambda_i, 0)$. Since $dx$ has a double pole at the point at $\infty\in \PP^1$ and $\pi$ is a local isomorphism near $r$ and $s$, the differential $\pi^*dx$ has double poles at the points $r$ and $s$. Thus the canonical
divisor of $C$ is 
$$
(*) \qquad K_C \sim (dx) \sim R - 2D,
$$
where $R$ denotes the ramification divisor, in this case the sum of the ramification points. .

How can we find differentials that are regular everywhere on $C$? If we divide $dx$ by $x^2$ (or any quadratic polynomial in $x$) to kill the poles we  introduce new poles in the finite part $C^\circ$ of $C$. 

Instead, we want to multiply $dx$ by a rational function with zeros at $p$ and $q$, but whose poles occur only at the points where $dx$ has zeroes---that is, the points $q_i$.  A natural choice is the reciprocal of the partial derivative $f_y := \partial f/ \partial y = 2y$, which vanishes at the points $q_i$, and has  a pole of order $g+1$ at each of the points $r$ and $s$ (reason: the involution $y\to -y$ fixes $C^\circ$ and $x$, and exchanges the points $r$ and $s$). In other words, as long as $g \geq 1$, the differential
$$
\omega = \pi^*(\frac{dx}{f_y})
$$
is regular, with divisor
$$
(\omega) = (g-1)r + (g-1)s = (g-1)D.
$$
The remaining regular differentials on $C$ are now easy to find: Since $x$ has only a simple pole
at the two points at infinity we can  multiply $\omega$ by any $x^k$ with $k = 0, 1, \dots, g-1$. Since this gives us $g$ independent differentials, we see that the differentials
$$
\omega, x\omega, \dots, x^{g-1}\omega
$$
  form a basis for $H^0(K_C)$.

With this description of the differentials, we can see clearly why the , canonical map of a hyperelliptic curves is two to one onto a rational normal curve, as proven in Chapter~\ref{2-RR}:
the relations on 
$$
\omega, x\omega, \dots, x^{g-1}\omega
$$
are the relations on $x^i$, and we see that the canonical image is the rational normal curve of degree $g-1$.


\section{Interlude: branched covers with specified branch divisor}\label{branched covers}
   

We will describe here the classification of branched covers $\pi : C \to B$ of a given curve $B$ with specified branch points. To give ourselves a concrete goal, we will address the question: given a curve $B$ and a collection $\Delta = \{p_1,\dots,p_b\} \subset B$ of points in $B$, how many branched covers $\pi : C \to B$ of degree $d$ are there with specified branching over each of the points $p_i$, up to isomorphism over $B$? This is called
the Hurwitz number of the configuration, and its computation in general is the subject of a large and active literature; see for example
\cite{ELSV}.

Our analysis will consist of two steps: first, we will reduce the problem to the classification of topological covering spaces of the complement $U = B \setminus \Delta$; we will then use our knowledge of the fundamental group of $U$ to enumerate such covering spaces.
   
\begin{theorem}
 Let $B$ be a smooth curve, let $\Delta\subset B$ be a finite set of points, and let $U := B\setminus \Delta$.
If $\pi^\circ : V \to U$ is a topological covering space then $V$ may be given the structure of a Riemann surface in a unique way so that the map $\pi^\circ$ is holomorphic; and $V$ may be compactified to a compact Riemann surface in $C$ a unique way such that the map $\pi^\circ$ extends to a holomorphic map $\pi : C \to B$.
\end{theorem} 

\begin{proof}
The space $V$ inherits the structure of a complex manifold from $U$ because if $D \subset U$ is any simply connected coordinate chart, then the preimage $({\pi^\circ})^{-1}(D)$ is a disjoint union of $d$ copies of $D$, and we may use them as coordinate charts on $V$. 
   
To compactify $V$ we observe that if $D^* = \{ z \in \CC \mid 0 < |z| < 1 \}$ is a punctured disc, then
the map $D\to D: z \mapsto z^n$ restricts to a connected $n$-fold covering space $D^*\to D^*$. 
Since $\pi_1(D^*) = \ZZ$, any connected covering space $E$ of degree $n$ is homeomorphic to this one
by a homeomorphism inducing the identity on the target of $\pi$.
If we  define a holomorphic structure on $E$ by pulling back the one on $D$, then
this homeomorphism is biholomorphic.

Thus if $D_i$ is a small neighborhood of the point $p_i \in B$ biholomorphic to a disc, then the preimage  in $V$ of the punctured disc $D_i^* := D_i \cap U$ is a disjoint union of punctured discs, and $V$ can then be compactified to a compact Riemann surface in a unique way by completing each one to a full disc.
\end{proof}
   
 The problem of classifying smooth curves $C$ that have a map $\pi : C \to B$ of degree $d$ is thus the problem becomes one of classifying covering spaces of $U$. 
    
 \subsection{Branched covers of $\PP^1$} 

We continue with the notation $U = B\setminus \Delta$, now supposing that $B = \PP^1_\CC$, the Riemann sphere. Again, let $\pi:V\to U$ be a covering space.

Choose a base point $p_0 \in U$, and draw simple, non-intersecting arcs $\gamma_i$ joining $p_0$ to $p_i$ in $U$. If $\Sigma$ is the complement of the union of these arcs in the sphere, then the preimage of $\Sigma$ in $V$ will be the disjoint union of $d$ copies of $\Sigma$, called the \emph{sheets} of the cover; label these $\Sigma_1,\dots,\Sigma_d$.

%\centerline{\includegraphics[height=2in]{"pic41"}}
   
A covering space $V \to U$ of $U$ is determined by its \emph{monodromy}, which is the map
   $$
   \pi_1(U, p_0) \to S_d
   $$
associating to each path  $\beta$ in $U$ from $p_0$ to $p_0$  the permutation of the points of $\pi^{-1}(p_0)$ given by sending a point $q \in \pi^{-1}(p_0)$ to the endpoint of the unique lift of $\beta$ starting at $q$. Let $\beta_i$ be the path starting at $p_0$, going out along the arc $\gamma_i$ until just short of $p_i$, going once around $p_i$ and then going back to $p_0$ along the same path $\gamma_i$, as drawn in the Figure. 

\centerline{ \includegraphics[height=2in]{"pic42"}}

Let $\tau_i$ be the permutation of $\{1,2,\dots,d\}$ corresponding to the path $\beta_i$. The fundamental group of $U$ is the free group of rank $p-1$, generated by the paths $\beta_1,\dots \beta_p$ modulo the relation $\prod_{i=1}^p \beta_i = 1$
which comes from the fact that the sphere minus the part enclosed by the paths $\beta_i$ is 
contractible. In particular, the permutations $\tau_1,\dots,\tau_b$ thus determine the covering space $V$.
 
If we assume that the covering
$\pi : C \to B$ is simply branched, so each $\tau_i$ is a transposition; and if we assume that
$C$, and thus $V$, is connected, then the subgroup $\langle \tau_1, \dots, \tau_b \rangle \subset S_d$ is transitive. If we were to relabel the sheets $\Sigma_\alpha$ according to some permutation $\sigma$, the effect would be to conjugate each $\tau_i$ by $\sigma$. 

Summarizing we have proven:   
   \begin{lemma}\label{branched cover classification}
   Let $p_1,\dots, p_b \in \PP^1$ be any $b$ given points. There is a natural bijection between 
   \begin{enumerate}
   \item the set of  simply branched covers $\pi : C \to \PP^1$ of degree $d$, branched over the points $p_i$, up to isomorphism over $\PP^1$; and
   \item the set of $b$-tuples of transpositions $\tau_1, \dots, \tau_b \in S_d$ such that $\prod \tau_i = id$ and such that $\tau_1, \dots, \tau_b$ generate a transitive subgroup of $S_d$, modulo simultaneous conjugation by $S_d$.
   \end{enumerate}
   \end{lemma}
In the case $d=2$ that is relevant to hyperelliptic curves, we note that there is only one transposition in $S_2$. Thus there is a unique double cover of $\PP^1$ with given branch points $p_1,\dots,p_b$. The product
condition shows again that the number of branch points must be even. . This completes the proof of Theorem~\ref{hyperelliptic existence}.
\end{proof}

We can use the same technique to count the number of 3 to 1 branched covers $C \to \PP^1$ with given simple branch points, using that fact that every odd permutation $\tau \in S_3$ is a transposition. Thus if $b$ is even and  $\tau_1,\dots,\tau_{b-1} \in S_3$ are arbitrary transpositions, then the product 
$\tau_1\cdot \cdots\tau_{b-1}$ is also a
 transposition. It follows that the number of $b$-tuples of transpositions $\tau_1,\dots,\tau_{b} \in S_3$ with $\prod \tau_i$ equal to the identity is $3^{b-1}$. The requirement that the group generated by the $\tau_i$ is transitive eliminates just the three cases where all the $\tau_i$ are equal. The group $S_3$ acts on the set of $b$-tuples of permutations without stabilizing any $b$-tuple, so every cover corresponds to exactly 6 sequences
  $\tau_1,\dots,\tau_b$. In sum, the number of simply branched three-sheeted covers of $\PP^1$ with specified branch points $q_1,\dots,q_b \in \PP^1$ is
$$
\frac{3^{b-1} - 3}{6} \; = \; \frac{3^{b-2} - 1}{2} 
$$


\subsection{More general covers}\label{general covers}

As we indicated, there are natural extensions of this basic theory to covers with arbitrary branching, and to covers of curves of any genus. For the first, suppose that instead of specifying ``simply branched"---meaning that the fiber over any branch point consists of $d-2$ simple points and one double point---we specified that the fiber over a given branch point $q_i$ consisted of $m_1$ simple points, $m_2$ double points, $m_3$ triple points and so on. This amounts to specifying a conjugacy class in $S_d$, and instead of associating to $q_i$ a transposition, we want to specify an element $\tau_i$ of this conjugacy class. Again, the permutations $\tau_i$ have to have product the identity, and to generate a transitive subgroup of $S_d$; and again, they are determined up to simultaneous conjugation by an element of $S_d$.


Similarly, we can enumerate covers of curves $B$ of genus $g>0$ by a modification of the construction above. To do this we must first describe the fundamental group of $B$.
\begin{fact}
  (see \cite[Ch 13]{Munkres}):
Choosing a base point $p \in B$ we can draw $2g$ loops $\alpha_1,\dots,\alpha_{g},\beta_1, \dots, \beta_g$ based at $p$ and disjoint except for $p$ so that the complement $B \setminus \cup \alpha_i$ is a disc $D$ with boundary $\alpha_1, \beta_1, \alpha_1^{-1}, \beta_1^{-1}, \dots, \alpha_g, \beta_g, \alpha_g^{-1}, \beta_g^{-1}$,
and the fundamental group of $B$ is the free group on the classes $\{\alpha_i, \beta_i\}_{i=1,\dots,g}$
modulo the single relation that the product of the commutators is 1:
$$
\prod_{i=1}^g \alpha_i\beta_i\alpha_i^{-1}\beta_i^{-1} = 1.
$$
\end{fact}
We may assume that the base point and branch points of the covering are all within $D$, so if we draw arcs $\gamma_i$ in $D$ joining $p$ to each of the branch points $q_i$, we can associate to a cover a collection of permutations $\sigma_1, \dots, \sigma_b, \mu_1,\dots,\mu_g, \nu_1,\dots,\nu_g$ (with $\sigma_i$ a transposition, if we are assuming simple branching, or of specified conjugacy class in general, and the $\mu_i$ and $\nu_i$ arbitrary). The sequence of these, up to conjugacy, determine the cover, and have to satisfy the relation
$$
\prod_{i=1}^b \sigma_i = \prod_{\alpha=1}^g \; [\mu_\alpha, \nu_\alpha]
$$
and generate a transitive subgroup of $S_d$.


\section{Curves of genus 2}

Since  curves of genus 2 are hyperelliptic, everything we said above applies to them; in particular, the canonical map $\phi_K : C \to \PP^1$ on a curve of genus 2 is the expression of $C$ as a double cover of $\PP^1$, branched over 6 points in $\PP^1$, which are unique up to automorphisms of $\PP^1$. 

In this section, we'll consider other maps from a hyperelliptic curves $C$ to projective space, starting with maps $C \to \PP^1$.

\subsection{Maps of $C$ to $\PP^1$}\label{genus 2 pencil}

The curve $C$ has a unique degree 2 morphism to $\PP^1$  associated to the canonical system $|K_C|$. But there are many other morphisms to $\PP^1$. For example, there is a 2-parameter
family of maps of degree 3:

 Let $L$ be an invertible sheaf of degree 3 on $C$. Since $3 > 2g-2$, Riemann-Roch tells us immediately that $h^0(L) = 2$, and there are two possibilities:

\begin{enumerate}
\item If the linear system $|L|$ has a base point $p \in C$, then $h^0(L(-p)) = 2$, and hence $L$ must be of the form $L = K_C(p)$. Conversely, if $L = K_C(p)$, then $h^0(L(-p)) = h^0(L)$, which is to say $p$ is a base point of $|L|$. There is a 1-parameter family of such $L$.

\item If $L$ is not of the form $L = K_C(p)$, then $|L|$ does not have a basepoint, and so defines a degree 3 map $\phi_L : C \to \PP^1$.
\end{enumerate}

Since the variety $\Pic^3(C)$ has dimension $g= 2$ the general invertible sheaf of degree 3 is of the second kind, and this gives a 2-parameter family of such maps.

There are plenty of higher-degree maps as well: an invertible sheaf of degree $d \geq 4 = 2g$ is basepoint free
, and gives a map to $\PP^{d-2}$, from which we can project in many ways
to $\PP^1$.

\subsection{Maps of $C$ to $\PP^2$} 
Next consider maps of $C$ of genus 2 to the plane. By theRiemann-Roch theorem, an invertible sheaf $L$ of degree 4 on $C$ will have $h^0(L) = 3$ and is basepoint free by Corollary~\ref{degree 2g+1 embedding} so the linear system $|L|$  gives a morphism $\phi_L : C \to \PP^2$. The invertible sheaf $\sL\otimes \omega_C^{-1}$ is either
$\omega_C$ or nonspecial; in either case, by the Riemann-Roch theorem, it has at least one sections, 
so we may write $\sL\otimes \omega_C^{-1} = \sO_C(p+q)$ for some points $p,q$. There are two possibilities:

\begin{enumerate}
\item If $p+q =  K_C$, then $\sL = \omega_C^2$. Since
the elements of $H^0(\omega_C)$ may be written as $\omega, x\omega$, the map
$$
\Sym^2 H^0(\omega_C) \to H^0(\sL).
$$
 is injective, and since both sides are 3-dimensional vector spaces, they are equal. In other words, every divisor $D \sim 2K_C$ is the sum of two divisors $D_1, D_2 \in |K_C|$. We conclude that the map $\phi_L$ is the composition of the canonical map $\phi_K : C \to \PP^1$ with the Veronese embedding $\nu_2 : \PP^1 \to \PP^2$ of $\PP^1$ as a conic in the plane and the map $\phi_L$ is generically 2-to-1 onto the conic.

\item \label{p+q not g12} If $p+q \neq  K_C$  then $h^0(p+q) = 1$, so the pair $p,q$ is unique. Furthermore,
 $h^0(\sL-p) = 2 =  h^0(\sL(-p-q))$ so 
 $H^0(\sL(-p)) = H^0(\sL(-q))$ and $\phi_\sL(p) = \phi_\sL(q)$. 
% However, for any effective divisor $D = r+s$ of degree 2 on $C$  other than $p+q$, we have
%  $h^0(\s(-D)) = 1$, so the map $\phi_\s$ is birational, and the image curve is a plane cubic with one double point.
By the genus formula, the $\delta$ invariant of this point must be 1. By Exercise~\ref{delta=1 characterization}
 this is an ordinary node (if $p\neq q$) or cusp (if $p=q$).

\end{enumerate}

Thus  for $\sL$ in an open subset of $\Pic^4(C)$ the image is a quartic with a node; for a one-dimensional locus in $\Pic^4(C)$, the image is a cubic with a cusp; and for one point in $\Pic^4(C)$ the image is a conic.

\subsection{Embeddings in $\PP^3$}

By Corollary~\ref{degree 2g+1 embedding} any invertible sheaf $L$ of degree 5 is very ample. 
Write $\phi_\sL : C \to \PP^3$ for the map given by the complete linear system $|L|$. Since $\phi_L$ is an embedding, we'll also denote the image $\phi_L(C) \subset \PP^3$ by $C$ and write $\cO_C(1)$ for $L$.

What degree surfaces in $\PP^3$ contain the curve $C$? We start with degree 2, and consider the restriction map
$$
H^0(\cO_{\PP^3}(2)) \to H^0(\cO_C(2)) = H^0(L^2).
$$
The space on the left has dimension 10; by the Riemann-Roch Theorem we have $h^0(L^2) = 2\cdot5 - 2 + 1 = 9$. It follows that $C$ lies on a quadric surface $Q$. Since $C$ a plane or a union of planes, any quadric containing $C$ is irreducible; if there were more than one such, Bezout's Theorem would imply that $\deg(C) \leq 4$. Thus $Q$ is unique.

We might ask at this point: is $Q$ smooth or a quadric cone? The answer depends on the choice of invertible sheaf $L$. 

\begin{proposition}\label{genus 2 embedding}
Let $C \subset \PP^3$ be a smooth curve of degree 5 and genus 2 and $Q \subset \PP^3$ the unique quadric containing $C$. If $L = \cO_C(1) \in \pic^5(C)$, then $Q$ is singular if and only if we have
$$
L \cong K^2(p)
$$
for some point $p \in C$; in this case, the point $p$ is the vertex of $Q$.
\end{proposition}

Note that the variety $\pic^5(C)$ has dimension 2, while the sheaves of the form $K^2(p)$ form a one-dimensional subfamily. Thus in general $Q$ will be smooth; the set of invertible sheaves $L$
for which the quadric is singular is 1-dimesnional, and thus of codimension one.

\begin{proof}
First, suppose that the invertible sheaf $L \cong K^2(p)$ for some $p \in C$. Then $L(-p) \cong K^2$, so that the map $\pi : C \to \PP^2$ given by projection from $p$ is the map $\phi_{K^2} : C \to \PP^2$ given by the square of the canonical sheaf. As we've seen, the map $\phi_{K^2}$ is two-to-one onto a conic $E \subset \PP^2$, so that the curve $C$ lies on the cone $Q$ over $E$ with vertex $p$, and this is the unique quadric surface containing $C$.

If $L$ is not of the form $K^2(p)$, then we can write
$$
L = \omega \otimes M,
$$
where by hypothesis $M$ is not of the form $\omega(p)$. As we saw in Section~\ref{genus 2 pencil}, this means that the pencil $|M|$ gives a degree 3 map $C \to \PP^1$.

This gives us a way of factoring the map $\phi_L : C \to \PP^3$: we have maps $\phi_K : C \to \PP^1$ of degree 2 and $\phi_M : C \to \PP^1$ of degree 3, and we can compose their product with the Segre embedding $\sigma : \PP^1 \times \PP^1 \to \PP^3$:
\begin{diagram}
& & \PP^1 & & & &\\
& \ruTo^{\phi_K} & & \luTo & & & \\
C & & \rTo^{\phi_K \times \phi_M} & & \PP^1 \times \PP^1 & \rTo^\sigma & \PP^3 \\
& \rdTo^{\phi_M} & & \ldTo & & & \\
& & \PP^1 & & & & \\
\end{diagram}

This description of the map $\phi_L$  shows  that \emph{$C$ is a curve of type $(2,3)$ on a smooth quadric $Q \subset \PP^3$}, completing the proof of Proposition~\ref{genus 2 embedding}.
\end{proof}

To describe a minimal set of generators of the homogeneous ideal $I(C) \subset \CC[x_0, x_1, x_2, x_3]$ in either case, we look at the restriction map
$$
H^0(\cO_{\PP^3}(3)) \to H^0(\cO_C(3)).
$$
Since the dimensions of these spaces are 20 and $15-2+1 = 14$ respectively, we see that  vector space of cubics vanishing on $C$ has dimension at least 6. Only four of these are multiples of the defining equation of $Q$ linear forms on $\PP^3$. It follows that there are at least two cubics vanishing on $C$ that are linearly independent modulo those vanishing on $Q$.

We can identify these cubics geometrically. Suppose first that $Q$ is smooth, so that $C$ is a curve of type $(2,3)$ on $Q$. In that case, if $L \subset Q$ is any line of the first ruling, the sum $C+L$ is the complete intersection of $Q$ with a cubic $S_L$, unique modulo the ideal of $Q$; conversely, if $S$ is any cubic containing $C$ but not containing $S$, the intersection $S \cap Q$ will be the union of $C$ and a line $L$ of the first ruling; thus, mod $I(Q)$, $S = S_L$. A similar argument applies in case $Q$ is a cone, and $L$ is any line of the (unique) ruling of $Q$. In Exercise~\ref{ideal of genus 2 degree 5} you may show that there are no more cubics containing C

\subsection{The dimension of the families of genus 2 curves}

Each of the types of maps that we described from a curve $C$ of genus 2 to projective space suggests
a computation of a 3-dimensional family of genus 2 curves, and indeed, as we willl explain in Chapter~\ref{ModuliChapter}, there is a moduli space of this dimension.

Since every curve of genus 2 is uniquely expressible as a double cover of $\PP^1$ branched at six points, modulo the group $PGL_2$ of automorphisms of $\PP^1$. The space of such double covers has dimension 6, and $\dim(PGL_2) = 3$, and since the group acts with finite stabilizers this gives a family of dimension $6-3 = 3$.

Also, each curve  of genus 2 is expressible as a 3-sheeted cover of $\PP^1$ (with eight branch points) in a 2-dimensional family of ways. As we saw in Section~\ref{branched covers}, such a triple cover is determined up to a finite number of choices by its branch divisor, so the space of such triple covers has dimension 8; modulo $PGL_2$ it has dimension 5, and since every curve is expressible as a triple cover in a two-dimensional family of ways, we arrive again at a family of dimenision $ 5-2 = 3$.

We've also seen that each curve of genus 2 can be realized as (the normalization of) a plane quartic  with a node in a 2-dimensional family of ways. The space of plane quartics has dimension 14; the family of those with a node has codimension one (Section~\ref{local severi geometry}) and hence dimension 13. In It is a fact that the automorphism group of any curve of genus $\geq 2$ is finite, and since  the automorphism group $PGL_3$ of $\PP^2$ has dimension 8, this suggests that that the family of nodal plane quartics modulo $PGL_3$ has dimension 5, and since every curve of genus 2 corresponds to a 2-parameter family of such curves, this again suggests a family of dimension $ 5-2=3$.

Finally, a curve of genus 2 may be realized as a quintic curve in $\PP^3$ in a two-parameter family of ways. To count the dimension of the family of such curves, note that each one lies on a unique quadric $Q$, and is of type $(2,3)$ on $Q$. Thus to specify such a curve we have to specify $Q$ (9 parameters) and then a bihomogeneous polynomial of bidegree $(2,3)$ on $Q \cong \PP^1 \times \PP^1$ up to scalars; these have $3\cdot 4 - 1 = 11$ parameters. Thus there is a 20-dimensional family of such divisors; modulo the automorphism group $PGL_4$ of $\PP^3$, this is a 5-dimensional family. Again, every abstract curve $C$ of genus 2 corresponds to a 2-parameter family of these curves modulo $PGL_4$, so once more this suggests a family of dimension $ 5 - 2 = 3$.

\section{Curves of genus 3}

Let $C$ be a smooth projective curve of genus 3. Since we have already discussed hyperelliptic curves, 
we will assume  that $C$ is nonhyperellitic. By  Theorem~\ref{canonical series is very ample}, the canonical map $\phi_K : C \to \PP^2$ embeds $C$ as a smooth plane quartic curve. Conversely, by Proposition~\ref{Adjunction Formula} any smooth plane curve of degree 4 has genus 3 and is embedded by the complete canonical series. 

Again, the automorphism group of a curve of genus 3 is finite. Since the space of quartics is 14-dimensional, and $\dim PGL(3) = 8$, this suggests that
there is a 6-dimensional family of curves of genus 3, and in Chapter~\ref{ModuliChapter}
we will see that there is indeed a 6-dimensional moduli space.

\subsection{Other representations of a curve of genus 3}
Since we have assumed that $C$ is not hyperelliptic there is no degree 2 cover of $\PP^1$. On the other hand, there are degree 3 covers: if $L \in \pic^3(C)$ is an invertible sheaf of degree 3 then, by the Riemann-Roch Theorem, we have
$$
h^0(L) = 
\begin{cases}
2, &\text{if $L \cong K-p$ for some point $p \in C$; and} \\
1 &\text{otherwise.}
\end{cases}
$$
There is thus a 1-dimensional family of representations of $C$ as a 3-sheeted cover of $\PP^1$. These are  visible directly from the canonical model: a degree 3 map $\phi_{K-p} : C \to \PP^1$ is the composition of the canonical embedding $\phi_K : C \to \PP^2$ with a projection from $p$. 

There are other representations of $C$ as the normalization of a plane curve. By the Riemann-Roch Theorem, $C$ has no $g^2_3$, and the canonical system is the only $g^2_4$, but there are plenty of models as plane quintic curves: by Proposition~\ref{very ample}, if $L$ is any invertible sheaf of degree 5, the linear system $|L|$ will be a basepoint free $g^2_5$ as long as $L$ is not of the form $K+p$, so that $\phi_L$ maps $C$ birationally onto a plane quintic curve $C_0 \subset \PP^2$. These can also be described geometrically in terms of the canonical model: any such invertible sheaf $L$ is of the form $2K-p-q-r$ for some trio of  points $p, q, r \in C$ that are not colinear in the canonical model, and we see  that $C_0$ is obtained from the canonical model of $C$ by applying a Cremona transform with respect to the points $p, q$ and $r$, that is, by applying the birational transformation
of the plane defined by the linear series of conics through $p,q,r$.

Since curves of genus 3 cannot be embedded in the plane, the lowest possible degree of an embedding is 6. Proposition~\ref{very ample} implies that a divisor $D$ of degree 6 is very ample if and only if it is not of the form $K+p+q$ for any $p, q \in C$ and Theorem~\ref{g+3 theorem} assures us that indeed a general invertible sheaf of degree 6 is very ample.  

If $C\subset \PP^3$ is a curve of genus 3 embedded as a curve of degree 6, then $C$ cannot lie on a singular quadric since by Example~\ref{Div of quadric} it would
have to be a complete intersection of the quadric with cubic, and then such a curve has genus 4. If $C$ lies on a smooth quadric
in class $(a,b)$ then $a$ or $b$ would be 2, so $C$ would be hyperelliptic, and conversely any curve in class $(2,4)$ 
is a hyperelliptic curve of genus 3, degree 6. 

Thus if $C$ is not hyperelliptic, then $C$ does not lie on a quadric surface. We have $h^0(\sO_{\PP^3}(3)) = 20$ while, by the Riemann-Roch formula, $h^0(\sO_C(3)) = 18-3+1= 16$, so $C$ lies on (at least) 4 cubics. Each of these cubics must be irreducible, so any two of them
intersect in a curve of degree 9 containing $C$ and another component or components $D$ of degree totaling 3. By Bertini's theorem
if we choose two \emph{general} cubics containing $C$, then each of the components of $D$ will be smooth. We shall see in Theorem~\ref{liaison genus formula} that the arithmetic genus of $D$ must be 0; thus $D$ must be a twisted cubic curve. The ideal of the twisted cubic is generated by the $2\times 2$ minors of a matrix of the form
$$
\begin{pmatrix}
 \ell_0& \ell_1&\ell_2\\
 \ell_1& \ell_2&\ell_3\\
\end{pmatrix}
$$
where the $\ell_i$ are linear forms,
and it follows that the two cubics can be written as the two $3\times 3$ minors involving the first two rows of  a matrix of the form
$$
\begin{pmatrix}
 \ell_0& \ell_1&\ell_2\\
 \ell_1& \ell_2&\ell_3\\
\ell_4& \ell_5&\ell_6\\
 \ell_7& \ell_8&\ell_9\\
\end{pmatrix}
$$
where $\ell_4,\dots,\ell_9$ are linear forms as well.
We shall see in Chapter~\ref{SyzygiesChapter}, as an application of the Hilbert-Burch theorem, that the ideal of $C$ is in fact generated by the 4 $3\times 3$ minors of this matrix, whose columns generate
the syzygies of the ideal of the curve.


\section{Theta characteristics}

In this section we sketch the algebraic theory of theta-characteristics, mostly as it applies to curves of genus 3.

Suppose that $C \subset \PP^2$ is a smooth plane curve. A \emph{bitangent} to $C$ is a line $L \subset \PP^2$ that is either tangent to $C$ at two distinct points, or has contact of order $\geq 4$ with $C$ at a point. Alternatively, we can say that a bitangent  corresponds to an effective divisor of degree 2 on $C$ such that $2D$ is contained in the intersection of $C$ with a line $L \subset \PP^2$.

A naive dimension count suggests that a smooth plane curve should have a finite number of bitangents (it's one condition on a line $L \in {\PP^2}^*$ to be tangent to $C$, so it should be two conditions for it to be bitangent). Indeed, this is the case; by Bezout's Theorem a conic or cubic curve cannot have any bitangents, but as we will show in Section~\ref{plane curve pluecker} every smooth curve of degree $d \geq 4$ has 
$$
12\binom{d+1}{4} - 4d(d-2),
$$
counted with appropriate multiplicities. For example, a line tangent to $C$ at 3 points  counts as three bitangents. Accordingly, a smooth plane quartic has 28 bitangents.

The bitangents to a plane quartic $C$ have a special significance: since $4 = 2 \times 2$, if $D = p+q$ is a bitangent, then the divisor $2D$ comprises the complete intersection of $C$ with a line; in other words, we have a linear equivalence
$$
2D \sim K_C
$$
or equivalently the invertible sheaf $\cO_C(D)$ is a square root of the canonical sheaf of $C$. Because of their appearance in the theory of theta functions, Riemann named the square roots of the canonical sheaf theta-characteristics.

How many such square roots are there? If $\cL$ and $\cM$ are invertible sheaves with $\cL^2 = \cM^2 = K$, then $\cL$ and $\cM$ differ by an invertible sheaf of order 2; that is,
$$
\cM = \cL \otimes \cF, \quad \text{where} \quad \cF \otimes \cF \sim \cO_C.
$$
In other words, $\cF$ is an invertible sheaf of degree 0 and, having fixed $\sL$,  corresponds to a point of order 2 in the Picard group $\Pic_0(C)$. Since we've seen that $\Pic_0(C) = \Jac(C)$ is a complex torus of dimension g = 3---the quotient of $\CC^3$ by a lattice $\Lambda \cong \ZZ^6$---we see that there are $2^6 = 64$ such invertible sheaves, and thus, given that there is some invertible sheaf $\cL$ satisfying $\cL^2 \cong K_C$, there are exactly $64 = 2^{2g}$ of them.

  
The reader will have noticed that the number 64 of theta-characteristics does not agree with the number 28 of bitangents. The reason is easy to see: bitangents correspond to \emph{effective} divisors $D$ with $2D \sim K$, while a theta characteristic $\cL$ may have $h^0(\cL) = 0$, that is, may not correspond to an effective divisor. 
This situation also occurs in other genera; see Exercise~\ref{theta char on genus 2} for a case that is directly accessible.
What can we say about the dimensions $h^0(\cL)$ of the space of sections of the theta-characteristics on $C$?
 
 There is a beautiful partial answer to this question, which can be deduced from a remarkable fact: the dimension $h^0(\cL)$ of the space of sections of a theta characteristic mod 2 is invariant under deformation:
We will only
sketch this development. See~\cite{MumfordPaper} and~\cite{JHPaper} for a full treatment.

 \begin{theorem}\label{locally constant sign} Let $\cC \to B$ be a family of smooth curves, and $\cL_b$ a family of theta characteristics on the curves in this family---in other words, an invertible sheaf $\cL$ on $\cC$ such that $(\cL|_{C_b})^2 \cong K_{C_b}$ for each $b \in B$. If we define a function $f : B \to \ZZ/2$  by
 $$
 f(b) = h^0(\cL|_{C_b}) \;  \; (\text{mod } 2)
 $$
then $f$ is locally constant.
\end{theorem}

We say that a theta-characteristic $\cL$ is \emph{even} or \emph{odd} according to the parity of $h^0(\cL)$. Given the irreducibility of the moduli space $M_g$,  Theorem~\ref{locally constant sign} suggests that all curves of genus $g$ have the same number of even theta characteristics. 

\begin{theorem}\label{number of theta characteristics}
If $C$ is a curve of genus $g$, then of the $2^{2g}$ theta-characteristics on $C$ there are $2^{g-1}(2^g + 1)$ even theta characteristics and $2^{g-1}(2^g-1)$ odd theta characteristics.
\end{theorem}

Using Theorem ~\ref{locally constant sign} and the connectedness of the moduli space of curves, 
the result may be reduced to the case when $C$ is hyperelliptic, and in this case it can be proven directly, as follows:

\begin{proof}[Proof when $C$ is hyperelliptic]
 Let $C$ be a hyperelliptic curve of genus $g$ expressed as a 2-sheeted cover of $\PP^1$, with ramification points $p_1,\dots,p_{2g+2}$. If we denote the class of the unique $g^1_2$ on $C$ by $E$, and $D$ is any theta characteristic, then $D+E$ will be effective, and so we can write
$$
D \sim mE + F
$$
with $-1 \leq m \leq \frac{g-1}{2}$ and $F$ the sum of $g-1-2m$ distinct points $p_i$. Moreover, this representation is unique, except in case $m=-1$; in that case, we note that the sum of $g+1$ of the Weierstrass points of $C$ is linearly equivalent to the sum of the other $g+1$. Thus the total number of theta characteristics is a sum of binomial coefficients; if $g$ is odd, it is
$$
\binom{2g+2}{0} + \binom{2g+2}{2} + \binom{2g+2}{4} + \dots + \binom{2g+2}{g-1} + \frac{1}{2}\binom{2g+2}{g+1}
$$ 
and similarly if $g$ is even it is
$$
\binom{2g+2}{1} + \binom{2g+2}{3} + \binom{2g+2}{5} + \dots + \binom{2g+2}{g-1} + \frac{1}{2}\binom{2g+2}{g+1}.
$$ 
In either case, we are adding up every other entry in the $2g+2$th row of Pascal's triangle, starting from the left and ending up with one half of the middle term. This sum is one half of the sum of every other entry in the whole row. But the sum of every other entry in the whole row is half the sum of all the entries in that row; thus the number of theta characteristics is $\frac{1}{4} \cdot 2^{2g+2} = 2^{2g}$.
\end{proof}

Note that in case of a nonhyperelliptic curve $C$ of genus 3, the dimension $h^0(\cL)$ of a theta characteristic $\cL$ cannot be $\geq 2$, so this says exactly that there are $2^{g-1}(2^g-1) = 28$ effective theta-characteristics corresponding to the 28 bitangents.

The proof of Theorem~\ref{locally constant sign} follows, via an ingenious construction of Mumford's, from an elementary fact about quadratic forms in an even number of variables.
To state this, suppose that $V$ is a complex vector space of dimension $2n$ and $Q$ a nondegenerate symmetric bilinear form on $V$. By an \emph{isotropic plane} we mean a linear space $\Lambda \subset V$ such that $Q(\Lambda, \Lambda) = 0$. 

\begin{fact}
 \begin{enumerate}
\item The maximal isotropic subspaces for $Q$ have dimension $n$;
\item The set of maximal isotropic subspaces for $Q$ is a subvariety of the Grassmannian $G(n,V)$, or dimension $\binom{n}{2}$ that has exactly two connected components (the ``rulings" of $Q$); and
\item If $\Lambda, \Lambda' \subset V$ are any two maximal isotropic subspaces, then
$$
\dim(\Lambda \cap \Lambda') \equiv n \text{ (mod 2)} \quad \iff \quad \Lambda, \Lambda' \text{ belong to the same ruling.}
$$
\end{enumerate} 
See~\cite[pp. 735--740]{Griffiths-Harris1978}.
\end{fact}

The first of these assertions is completely elementary: since the map $\tilde Q : V \to V^*$ associated to the form $Q$ carries an isotropic subspace to its annihilator, there can't be an isotropic plane of dimension $>n$; and similarly if $\Lambda \subset V$ is any isotropic plane of dimension $<n$ we can include $\Lambda$ in a larger isotropic plane by adding any vector $v$ with $\overline Q(v,v) = 0$ for the induced bilinear form $\overline Q$ on $\ann(\Lambda)/\Lambda$.

The second and third assertions are less elementary, but the reader may already have seen the first two nontrivial cases of each: 

\begin{example}
\begin{itemize}

\item When $n=2$ the form $Q$ corresponds to a smooth quadric surface in $\PP^3$, and the lines on this surface correspond to the isotropic 2-planes in $\CC^4$. There are two rulings by lines, and lines of opposite rulings meet in a point, while lines of the same ruling are either disjoint or equal. 

\item When $n=3$, the Grassmannian $\GG(1,3)$, in its Pl\"ucker embedding, is a smooth quadric in $\PP^5$. The projective 2-plane of lines containing a given point $p \in \PP^3$, and that of lines contained in a given plane $H \subset \PP^3$
correspond to linear subspaces of dimension 3 in $\CC^6$, which are the isotropic planes in the two distinct components. Two of these families of lines intersect in either a point, or coincide; that is, the linear subspaces either intersect  in a line or a 3-plane. See Exercise~\ref{G13}.

\end{itemize}
\end{example}

%\fix{is each of these families parametrized by a projective space also in higher dim?

Now suppose that $C$ is a smooth curve of genus $g$, and $\cL$ an invertible sheaf on $C$ with $\cL^2 \cong K_C$; that is, a theta characteristic. Choose a divisor $D = p_1 + \dots + p_n$ of degree $n> g-1$ consisting of distinct points, and let $V$ be the $2n$-dimensional vector space
$$
V := H^0( \cL(D) / \cL(-D) ).
$$
The sheaf $ \cL(D) / \cL(-D) = \cL \otimes \cO_C(-2D) $ is supported on $D$, with stalk 
isomorphic to $\cO_p/\gm_{C,p}^2$ of dimension 2 at each $p \in D$. We can define a bilinear form on $V$ by setting
$$
Q(\sigma, \tau) := \sum_i \text{Res}_{p_i}(\sigma \cdot \tau)
$$
where we use the isomorphism $\cL^2 \cong K_C$ to identify the product $\sigma\tau$ with a rational differential.

We now introduce two isotropic subspaces for $Q$: first, we set
$$
\Lambda := H^0( \cL / \cL(-D) ).
$$
This is isotropic because the product $\sigma\tau$ of two elements of $\Lambda$ corresponds to a regular differential, and so has no residues.  Second, we set
$$
\Lambda' := \im\left( H^0(\cL(D)) \to H^0( \cL(D) / \cL(-D) ) \right)
$$
Since the $H^0(\cL(-D)) = 0$, the map is injective and according to Riemann-Roch, $h^0(\cL(D)) = n$, so this is again an $n$-dimensional subspace of $V$; it's isotropic because the sum of the residues of a global rational differential on $C$ is 0. Finally, 
$$
H^0(\cL) \cong \Lambda \cap \Lambda',
$$
and Theorem~\ref{locally constant sign} follows.

As for Theorem~\ref{number of theta characteristics}, it is possible to describe the configurations of odd and even theta characteristics as subsets of the set $S$ of all theta characteristics, which as we've seen is a principle homogeneous space for the group $\Jac(C)_2 \cong (\ZZ/2\ZZ)^{2g}$ of points of order 2 on the Jacobian \cite{JHPaper}. However, we can instead
Theorem~\ref{locally constant sign} and the fact that there exists an irreducible family of 
curves and their Jacobians connecting an arbitrary curve to a hyperelliptic curve, to use reduce to the hyperelliptic case; the necessary result in that case is expressed (in the genus 2 case) in
Exercise~\ref{theta char on genus 2}.
% \fix{we will have theta on all hyperelliptic directly}

There is more to say about the configuration of theta characteristics. For example:
\begin{fact}
 As noted, if we choose any theta characteristic on a curve $C$, we may identify the set $S^-$ of odd theta-characteristics with a subset of the group $\Jac(C)_2$ of points of order 2 on the Jacobian of $C$. We might expect, then, that some 4-tuples of these points will add up to 0 in $\Jac(C)$; in other words, there should exist some 4-tuples $\cL_1,\dots,\cL_4 \in S^-$ such that
$$
\cL_1+ \dots +\cL_4 = 2K_C.
$$
What this means in the specific case of genus $g=3$ is that among the 28 bitangents to a smooth plane quartic curve $C$, there are some subsets of 4 whose eight points of tangency form the intersection of $C$ with a plane conic. From the more detailed knowledge of the configuration $S^-$ we can say how many. Indeed, the number was first found by Salmon; it is 315 (\cite{MR0115124}.
\end{fact}

\subsection{Counting theta-characteristics}

One way to count the number of odd and even theta characteristics on a curve of genus $g$ is  to describe these explicitly in the case of a hyperelliptic curve and then use Theorem\ref{locally constant sign} to deduce the corresponding statements for any smooth curve of genus $g$. We start with some preliminary calculations:

\begin{lemma}\label{summing binomials}
For any positive integer $n$,
\begin{enumerate}
\item 
$$
\sum_{k=0}^n \binom{2n}{2k} \; = \; \sum_{k=0}^{n-1} \binom{2n}{2k+1} \; = \; 2^{2n-1}
$$
\item 
$$
\sum_{k=0}^n \binom{4n}{4k} = 2^{4n-2}  + (-1)^n 2^{2n-1} \quad \text{and} \quad \sum_{k=0}^{n-1} \binom{4n}{4k+2} = 2^{4n-2} - (-1)^n  2^{2n-1}
$$
\item 
$$
\sum_{k=0}^n \binom{4n+2}{4k+1} = 2^{4n} + (-1)^n 2^{2n} \quad \text{and} \quad \sum_{k=0}^{n-1} \binom{4n}{4k+3} = 2^{4n} - (-1)^n  2^{2n}
$$
\end{enumerate}
\end{lemma}

\begin{proof}
The first is completely elementary; by the binomial theorem, we have
$$
2^{2n} = (1+1)^{2n} = \sum_{l = 0}^{2n} \binom{2n}{l} \quad \text{and} \quad 0 = (1-1)^{2n} = \sum_{l = 0}^{2n} (-1)^l\binom{2n}{l}
$$
and taking the sum and the difference of these two equations yields the desired result.

The second  follows similarly by applying the binomial theorem to the expression $(1 + i)^{4n} = (-1)^n2^{2n}$. Equating the real parts, we have
$$
\sum_{k=0}^n \binom{4n}{4k} - \sum_{k=0}^{n-1} \binom{4n}{4k+2} = (-1)^n2^{2n}
$$
while by the first part of the lemma we have
$$
\sum_{k=0}^n \binom{4n}{4k} + \sum_{k=0}^{n-1} \binom{4n}{4k+2} = 2^{4n-1};
$$
taking the sum and difference of these two equations yields the desired formula.

Finally, for the third we apply the binomial theorem to the expression $(1 + i)^{4n+2} = (-1)^n2^{2n+1}i$. Equating the imaginary parts, this gives
$$
\sum_{k=0}^n \binom{4n+2}{4k+1} - \sum_{k=0}^{n-1} \binom{4n+2}{4k+3} \; = \; (-1)^n2^{2n+1}
$$
whereas by the first part of the lemma,
$$
\sum_{k=0}^n \binom{4n+2}{4k+1} + \sum_{k=0}^{n-1} \binom{4n+2}{4k+3} \; = \; (-1)^n 2^{4n+1}
$$
and as before taking the sum and difference of these two yields the result.
\end{proof}

We will count the number of theta characteristics on a hyperelliptic curve in terms of sums of subsets of the ramification points, so we need to know what linear equivalences exist among sums of these subsets:

\begin{lemma}\label{ramification point relations}
Let $C$ be the hyperelliptic curve of genus $g$ expressed as a 2-sheeted cover of $\PP^1$ with ramification points $p_1,\dots,p_{2g+2}$. The divisor class of
 any half of the ramification points is equal to the divisor class of the other half, but there are no
 smaller relations. More precisely,
 let $I_1,I_2$ be subsets of $\{1,\dots 2g+2\}$
set 
$$
D_i = \sum_{j\in I_i} p_j.
$$
The divisors $D_1,D_2$ are linearly equivalent if and only they
have the same cardinality and $I_1\cup I_2 = \{1,\dots, 2g+2\}$.\end{lemma}

\begin{proof}
Subtracting whatever points $D_1$ and $D_2$ have in common we may suppose
that $I_1\cap I_2 = \emptyset$. If $D_1\sim D_2$, it follows at once that they have the same degree, $d\leq g+1$, and we must show that either $d=0$ or $d=g+1$.

We have $D_1\sim D_2$ iff $D_1+D_2\equiv 2D_1$. If
$d\leq g$ we have $r(2D_1) = d$: for $d<g$ this is
the extremal case of Clifford's Theorem, while for $d = g$ this follows simply from the 
Riemann-Roch formula. Thus every divisor in $|2D_1|$ is a sum of $d$ fibers of the 
2 to 1 map of $C$ to $\PP^1$, and for such a divisor to be a sum of distinct points $p_i$
must be 0.

Now suppose that $d=g+1$, so that $D_1+D_2 = \sum_{j=1}^{2g+2}p_i$. Recall from equation
$(*)$ in Section~\ref{hyperelliptic differentials} that the canonical class on $C$ is
equivalent to $\sum_{i=1}^{2g+2}p_i-2p -2q$, where $p+q$ is a fiber of the 2 to 1 map
from $C$ to $\PP^1$. We deduce
$$
D_1+D_2 = K_C+2\g^1_2 = (g+1)\g^1_2 = 2D_1,
$$
concluding the argument.
\end{proof}


Returning to the counting, let $C$ be the hyperelliptic curve of genus $g$ expressed as a 2-sheeted cover of $\PP^1$, with ramification points $p_1,\dots,p_{2g+2}$. 

First of all, if we denote the class of the unique $g^1_2$ on $C$ by $E$, and $D$ is any theta characteristic, then $D+E$ will be effective, and so we can write
$$
D \sim mE + F
$$
with $-1 \leq m \leq \frac{g-1}{2}$ and $F$ the sum of $g-1-2m$ distinct points $p_i$. Moreover, this representation is unique, except in case $m=-1$; in that case, we note that the sum of $g+1$ of the Weierstrass points of $C$ is linearly equivalent to the sum of the other $g+1$. Thus the total number of theta characteristics is a sum of binomial coefficients; if $g$ is odd, it is
$$
\binom{2g+2}{0} + \binom{2g+2}{2} + \binom{2g+2}{4} + \dots + \binom{2g+2}{g-1} + \frac{1}{2}\binom{2g+2}{g+1}
$$ 
and similarly if $g$ is even it is
$$
\binom{2g+2}{1} + \binom{2g+2}{3} + \binom{2g+2}{5} + \dots + \binom{2g+2}{g-1} + \frac{1}{2}\binom{2g+2}{g+1}.
$$ 
In either case, we are adding up every other entry in the $2g+2$th row of Pascal's triangle, starting from the left and ending up with one half of the middle term. This sum is exactly one half of the sum of every other entry in the whole row; by the first part of Lemma~\ref{summing binomials} this is exactly $\frac{1}{4} \cdot 2^{2g+2} = 2^{2g}$.

Finally, we can add up the number of even and odd theta characteristics separately simply by taking every other term in the sums above; using the second and third parts of Lemma~\ref{summing binomials} (in case $g$ is odd and even, respectively) we can conclude that $C$ has $2^{g-1}(2^g-1)$ odd theta characteristics and $2^{g-1}(2^g+1)$ even theta characteristics. By Theorem~\ref{locally constant sign} (and the connectedness of the moduli space $M_g$), this count then holds for all curves of genus $g$, establishing Theorem~\ref{number of theta characteristics}.

\section{Exercises}
 \begin{exercise}
  We have seen that a curve $C$ of genus $g=1$ is expressible as a 2-sheeted cover of $\PP^1$ branched over four points; that is, as the smooth projective curve associated to the affine curve $C^\circ$ given by $y^2 - \prod_{i=1}^4 (x-\lambda_i)$. Show that the closure $\overline{C^\circ}$ of $C^\circ \subset \AA^2$ in either $\PP^2$ or $\PP^1 \times \PP^1$ consists of the union of $C^\circ$ with one additional point, with that point a tacnode of $\overline{C^\circ}$ in either case.
  
  Hint: Observe that in either case the complement $\overline{C^\circ} \setminus C^\circ$ consists of a single point, with two points of $C$ mapping to it; now use the genus formula in either $\PP^2$ or $\PP^1 \times \PP^1$. 
  \end{exercise}

\begin{exercise}
Find the number of 3-sheeted covers $C \to \PP^1$ of genus $g$ with simple branching except for one point of total ramification (that is, one point with just a single preimage point.)

Hint: such a cover is specified by giving $2g+2$ transpositions, not all equal, whose product is a nontrivial 3-cycle, modulo simultaneous conjugation. We have already worked out the number of such tuples whose product is the identity; just subtract.
\end{exercise}


\begin{exercise}
Let $B$ be a curve of genus $h$. How many unramified double covers of $B$ are there?

Hint: Topologically, such covers are in 1-1 correspondence with subgroups of index 2 in $\pi_1(C)$; and such a subgroup is necessarily the preimage of a subgroup of index 2 in the abelianization $H_1(C, \ZZ) \cong \ZZ^{2g}$.
\end{exercise}

\begin{exercise}
Show that unramified double covers of a smooth curve $C$ are in one-to-one correspondence
with invertible sheaves $\sL$ on $C$ such that $\sL^2 \cong \sO_C$, that is with the 2-torsion points
of $\Jac(C)$.

Hint: If $f : X \to C$ is an unramified double cover, consider the direct image $f_*(\cO_X)$. This is a locally free sheaf of rank 2 on $C$, on which the group $\ZZ/2$ acts; the $+1$-eigenspace is the structure sheaf $\cO_C$, and the $-1$-eigenspace is an invertible sheaf $\sL$ on $C$ such that $\sL^2 \cong \sO_C$.
\end{exercise}


\begin{exercise} Let $E$ be a curve of genus 1, and $q_1,\dots,q_b \in E$. How many double covers $C \to E$ are there branched over the $q_i$?

Hint: By our analysis, to specify such a cover, we have to specify the monodromy around representative loops generating $H_1(E, \ZZ) \cong \ZZ^2$; thus there are four possibilities.
\end{exercise}

\fix{Do we do anywhere the correspondence between double covers and square roots of the branch divisor?}

%\begin{exercise} Let $E$ be a curve of genus 1, and $q, q' \in E$. How many triple covers $C \to E$ are there simply branched over $q$ and $q'$?
%\end{exercise}

\begin{exercise}\label{ideal of genus 2 degree 5} 
Show that for any pair of lines $L, L'$ of the appropriate ruling of $Q$, the three polynomials $Q$, $S_L$ and $S_{L'}$ generate the homogeneous ideal $I(C)$. Find relations among them. Write out the minimal resolution of $I(C)$.
\end{exercise}


\begin{exercise}\label{theta char on genus 2} %\fix{we will do the case of arb g}
 Let $C$ be a curve of genus 2, expressed as a 2-sheeted cover of $\PP^1$ with ramification points $p_1,\dots,p_6$
 \begin{enumerate}
 \item Show that the theta-characteristics on $C$ are either of the form $\cL = \cO_C(p_i)$ or of the form $\cL = \cO_C(p_i + p_j - p_k)$ with $i, j, k$ distinct. 
 \item Show that in the first case we have $h^0(\cL) = 1$, and in the second case we have $h^0(\cL) = 0$. 
 \item Finally, show that there are six of the former kind, and 10 of the latter, making $2^4 = 16$ in all.
 \end{enumerate} 
 \end{exercise}
 
 
\begin{exercise}\label{nodal quartic}
Let $C$ be a  curve of genus 2 and let $\sL\in \Pic^4(C)$ be an invertible sheaf of the form $L = K_C(p+q)$ with $p \neq q$ and $p+q \not\sim K_C$ as in~\ref{p+q not g12}. Show that
\begin{enumerate}
\item $h^0(L(-2p)) = h^0(L(-2q)) = 1$, and
\item $h^0(L(-2p-2q)) = 0$.
\end{enumerate}
Deduce from this that the map $\phi_L$ is an immersion, and that the tangent lines to the two branches of $\phi_L(C)$ at the point $\phi_L(p) = \phi_L(q)$ are distinct, meaning the point $\phi_L(p) = \phi_L(q)$ is a node of $\phi_L(C)$.

Hint: the first statement implies that the map $\phi_L$ is an immersion at $p$ and $q$, while the second says that the images of the differential $d\phi_L$ at $p$ and $q$ are distinct. 
\end{exercise}


 
 
\begin{exercise}\label{G13}
We can represent any line in $\PP^3$ by 2 points on it, and using their coordinates as the two rows of a 
$2\times 4$ matrix. The \emph{Pl\"ucker coordinates} of the line are the six $2\times 2$ minors
$$
\{p_{i,j}\}_{0\leq i<j\leq 3}
$$
of this matrix. They are independent, up to a common scalar multiple, of the two points chosen, and define the \emph{Pl\"ucker embedding} of the Grassmannian $\GG(1,3)$ in $\PP^5$.

The minors $p_{i,j}$  satisfy a nonsingular quadratic equation: if we stack two copies of the $2\times 2$
matrix to produce a $4\times 4$ matrix, its determinant is zero, and the Laplace expansion of this determinant
is the \emph{Pl\"ucker equation}
$$
p_{0,1}p_{2,3}-p_{0,2}p_{1,3}+p_{0,3}p_{1,2} = 0.
$$

\begin{enumerate}
\item Show that the quadratic form
$
Q = p_{0,1}p_{2,3}-p_{0,2}p_{1,3}+p_{0,3}p_{1,2}
$
is nonsingular, and deduce that it generates the ideal of $\GG(1,3)$ in $\PP^5$.
\item
Write the bilinear form corresponding to $Q$ as the determinant of a matrix, and deduce that 
two points in $\GG(1,3)$ correspond to vectors that pair to 0 iff and only if they correspond to lines that intersect.
\item Deduce that a maximal isotropic plane for $Q$ corresponds either to the set of lines containing a given point or the set of lines contained in a given plane; and that two such sets of lines meet in a single point or coincide.
\end{enumerate}
\end{exercise}


\input footer.tex


