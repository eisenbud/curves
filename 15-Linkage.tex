%header and footer for separate chapter files

\ifx\whole\undefined
\documentclass[12pt, leqno]{book}
\usepackage{graphicx}
\input style-for-curves.sty
\usepackage{hyperref}
\usepackage{showkeys} %This shows the labels.
%\usepackage{SLAG,msribib,local}
%\usepackage{amsmath,amscd,amsthm,amssymb,amsxtra,latexsym,epsfig,epic,graphics}
%\usepackage[matrix,arrow,curve]{xy}
%\usepackage{graphicx}
%\usepackage{diagrams}
%
%%\usepackage{amsrefs}
%%%%%%%%%%%%%%%%%%%%%%%%%%%%%%%%%%%%%%%%%%
%%\textwidth16cm
%%\textheight20cm
%%\topmargin-2cm
%\oddsidemargin.8cm
%\evensidemargin1cm
%
%%%%%%Definitions
%\input preamble.tex
%\input style-for-curves.sty
%\def\TU{{\bf U}}
%\def\AA{{\mathbb A}}
%\def\BB{{\mathbb B}}
%\def\CC{{\mathbb C}}
%\def\QQ{{\mathbb Q}}
%\def\RR{{\mathbb R}}
%\def\facet{{\bf facet}}
%\def\image{{\rm image}}
%\def\cE{{\cal E}}
%\def\cF{{\cal F}}
%\def\cG{{\cal G}}
%\def\cH{{\cal H}}
%\def\cHom{{{\cal H}om}}
%\def\h{{\rm h}}
% \def\bs{{Boij-S\"oderberg{} }}
%
%\makeatletter
%\def\Ddots{\mathinner{\mkern1mu\raise\p@
%\vbox{\kern7\p@\hbox{.}}\mkern2mu
%\raise4\p@\hbox{.}\mkern2mu\raise7\p@\hbox{.}\mkern1mu}}
%\makeatother

%%
%\pagestyle{myheadings}

%\input style-for-curves.tex
%\documentclass{cambridge7A}
%\usepackage{hatcher_revised} 
%\usepackage{3264}
   
\errorcontextlines=1000
%\usepackage{makeidx}
\let\see\relax
\usepackage{makeidx}
\makeindex
% \index{word} in the doc; \index{variety!algebraic} gives variety, algebraic
% PUT a % after each \index{***}

\overfullrule=5pt
\catcode`\@\active
\def@{\mskip1.5mu} %produce a small space in math with an @

\title{Personalities of Curves}
\author{\copyright David Eisenbud and Joe Harris}
%%\includeonly{%
%0-intro,01-ChowRingDogma,02-FirstExamples,03-Grassmannians,04-GeneralGrassmannians
%,05-VectorBundlesAndChernClasses,06-LinesOnHypersurfaces,07-SingularElementsOfLinearSeries,
%08-ParameterSpaces,
%bib
%}

\date{\today}
%%\date{}
%\title{Curves}
%%{\normalsize ***Preliminary Version***}} 
%\author{David Eisenbud and Joe Harris }
%
%\begin{document}

\begin{document}
\maketitle

\pagenumbering{roman}
\setcounter{page}{5}
%\begin{5}
%\end{5}
\pagenumbering{arabic}
\tableofcontents
\fi


\chapter{Linkage and the canonical sheaves of singular curves}
\label{LiaisonChapter}\label{linkageChapter}\label{LinkageChapter}


\section{Introduction} \label{LinkageIntro}

\emph{In this chapter, curves are purely 1-dimensional projective schemes---not necessarily reduced or irreducible.}

Linkage is an equivalence relation on varieties and schemes of a given dimension embedded in a common space. It was a key element
in the classification of curves in $\PP^3$ for which Max Noether and  Georges-Henri Halphen received in the Steiner prize of the Prussian Academy of Sciences in 1880, and it was a necessary ingredient in the work of Clebsch, Brill, Noether and Macaulay toward
a version of the Riemann-Roch theorem couched in terms of the algebra of plane curves near the end of the 19-th century. 
It was put on a firm modern footing in~\cite{MR364271}, and this foundation was used for further progress in projective geometry by Hartshorne, Rao and others. In this Chapter we will explain some of these developments, starting with a simple example,
and including the algebra necessary for a formulation in the natural generality of purely 1-dimensional schemes.

As we have seen, any plane curve is arithmetically Cohen-Macaulay, and its arithmetic genus is determined by its degree. Similarly, a curve in  $\PP^3$ that is a complete intersection of surfaces of degrees $d,e$
is arithmetically Cohen-Macaulay (Theorem~\ref{Lasker}) and has arithmetic genus determined by $d,e$.
 Next simplest, perhaps
is a curve $C$ that is \emph{directly linked} to a complete intersection, which means roughly that its union $X = C\cup D$
with a complete intersection
curve $D$ is again a complete intersection (see Definition~\ref{linkage def} for the general definition). We will see that, once again, such a curve $C$ is arithmetically Cohen-Macaulay,
and its genus is determined by the degrees of the equations of $X$ and the degree and genus of $D$. 
Allowing sequences of direct links we define an equivalence relation called  \emph{linkage} or \emph{liaison}, 
and curves in the \underline{li}nkage {\underline c}lass of a {\underline c}omplete {\underline i}ntersection are often said to be ``licci." 

A famous theorem of Hartshorne and Rao \cite{MR520926} shows that the linkage class of a curve $C\subset \PP^3$
is classified by the finite dimensional graded module 
$$
D(C) :=H^1_*(\sI_C):=\bigoplus_{m\in \ZZ} H^1(\sI_C(m)),
$$
called the \emph{deficiency module} of $C$. (This is a module over the homogeneous coordinate ring $S = H^0_*(\cO_{\PP^3})$.) The correspondence is explicit: from a finite-dimensional graded module over $S$ one can
actually construct curves.

In the first sections of this chapter we will examine the equivalence relation on curves in $\PP^3$ that is defined by linkage. Much of the story extends to the case of singular curves. This extension requires an understanding of the
dualizing sheaves of singular curves, to which we turn in Section~\ref{duality}. We conclude the chapter with an analysis of the adjoint ideal, completing a result from Chapter~\ref{PlaneCurvesChapter},  and to the Riemann-Roch theorem for general curves and coherent sheaves.

Aside from the classification result above, linkage is useful in analyzing Hilbert schemes. We will exploit this systematically in cases of low degree and genera in Chapter~\ref{HilbertSchemesChapter}, and we begin
this chapter with what is perhaps the simplest example, computing the dimension of the component of
the Hilbert scheme $Hilb_{3m+1}(\PP^3)$ that is the closure of the open subset $\cH^\circ$  parametrizing twisted cubics (see Proposition~\ref{hilb of twisted cubics} for another proof).

\section{Linkage of twisted cubics}
The simplest example of linkage is that of the union of a twisted cubic and one of its secant lines, pictured in
Figure~\ref{cubicAndLine}, and we will start with that.

\begin{figure}\label{cubicAndLine}
\centerline {\includegraphics[height=3in]{"main/Fig15-1-TwistAndShout"}}
 \caption{A quadratic cone (red) intersecting a smooth quadric (yellow) in the union of a vertical line and a twisted cubic (credit: Herwig Hauser)}
\end{figure}

Any twisted cubic curve $C\subset \PP^{3}$ lies on a nonsingular quadric in class $(1,2)$. Adding a line $L$ of class
$(1,0)$ we get a  divisor of class $(2,2)$, the class of the complete intersection of two quadrics. Since $L$ is also 
a complete intersection, $C$ is licci.

We can make the relation of $L$ and $C$ explicit as follows: The ideal of $C$
is minimally generated by the three $2\times 2$ minors of the matrix
$$
\begin{pmatrix}
 x_0&x_1&x_2\\
 x_1&x_2&x_3
\end{pmatrix}\,.
$$
The minor $Q_{1,2}$ involving the first two columns and the minor $Q_{2,3}$ involving the last two columns 
both vanish on the line $L: x_1 = x_2 = 0$, which meets the twisted cubic in the two points
$x_{0}= x_{1}=x_{2}=0$ and $x_{1} = x_{2} = x_{3}= 0$. Thus $L$ is a secant line to $C$. 
A general linear combination $Q$ of $Q_{1,2}$ and $Q_{2,3}$ defines a smooth quadric, which is thus isomorphic to $\PP^1\times \PP^1$. The curve $C$ necessarily lies in the divisor class $(1,2)$ (or, symmetrically, $(2,1)$), and the line in class $(1,0)$ (respectively, $(0,1)$), summing to the 
complete intersection $(2,2)$ of Q with (say) $Q_{1,2}$. See Figure~\ref{cubicAndLine}.

Conversely, if two irreducible quadrics $Q_{1}, Q_{2}$ both contain a twisted cubic $C$ then, by B\'ezout's theorem,
$Q_{1}\cap Q_{2}$ is the union of $C$ with a line. If at least one of the quadrics is smooth, we are in the
situation above.
 
 This suggests that we set up an incidence correspondence between twisted cubics and their secant lines. Let $\PP^9$ denote the projective space of quadrics in $\PP^3$, and consider
$$
\Phi = \{ (C, L, Q, Q') \in \cH^\circ \times \GG(1,3) \times \PP^9 \times \PP^9 \; \mid \; Q \cap Q' = C \cup L \}.
$$

We'll analyze $\Phi$ by considering the projection maps to $\cH^\circ$ and $\GG(1,3)$; that is, by looking at the diagram
\begin{diagram}[small]
& &  \Phi & & \\
& \ldTo^{\pi_1} & & \rdTo^{\pi_2} & \\
\cH^\circ & & & & \GG(1,3)
\end{diagram}

Consider  the projection map $\pi_2 : \Phi \to \GG(1,3)$ on the second factor. By what we just said, the fiber over any point $L \in \GG(1,3)$ is an open subset of $\PP^6 \times \PP^6$, where $\PP^6$ is the space of quadrics containing $L$. Since $\dim \GG(1,3) = 4$ we see that $\Phi$ is irreducible of dimension $4 + 2\times 6 = 16$. On the other hand, the map $\pi_1 : \Phi \to \cH^\circ$ is surjective, with fiber over a curve $C$ an open subset of $\PP^2 \times \PP^2$, where $\PP^{2}$ is the projective space of quadrics containing $C$; we conclude that $\cH^\circ$ is irreducible of dimension 12, in accord with our 
computation in Proposition~\ref{hilb of twisted cubics} of the space of twisted cubics as $PGL_4/PGL_2$.


\section{Linkage of smooth curves in $\PP^3$}\label{SLinkage}\label{linkage section}

If the union of two smooth curves in $\PP^3$ is a complete intersection of surfaces, then the degrees and genera
of the curves are related:

\begin{theorem}\label{liaison genus formula-first version} Let $C, {C_2}\subset \PP^3$ be distinct smooth irreducible curves of degrees $c,d$ whose union is the complete intersections of two surfaces $S,T$, with $S$ smooth. If the degrees of $S,T$ are $s,t$ respectively, then
$$
\begin{aligned}
&\deg C_1+\deg C_2 = st\\
&g(C_1) - g({C_2}) = \frac{s+t-4}{2}(\deg C_1-\deg {C_2}).
\end{aligned}
 $$
\end{theorem}
In words, the difference between the genera of $C_1$ and ${C_2}$ is proportional to the difference in their degrees, with constant of proportionality $(s+t-4)/2$. For example, in the case of the complete intersection of two quadrics
described above, the multiplier $(s+t-4)/2 = 0$, and indeed the line and the twisted cubic have the same genus.
The relation of degrees and genera is true more generally, as we shall see in the next section, but the special
case is already useful.

\begin{proof}
B\'ezout's theorem implies that the degree of $S\cap T = C_1\cup C_2$ is $st$, whence the first formula.

By the adjunction formula in $\PP^3$ the canonical divisor of $S$ has class $K_S = (s-4)H$. Thus, from the 
adjunction formula on the surface $S$ we get
$$
g(C_i) = \frac{C_i^2+C_i\cdot K_S}{2}+1 = \frac{C_i^2+(s-4) \deg C_i}{2}+1.
$$
Subtracting, we get
$$
g(C_1)-g(C_2) = \frac{C_1^2-C_2^2+(s-4) (\deg C_1-\deg C_2)} {2}.
$$
Because $C_1+C_2$ is in the class $tH$ on $S$ we have
$$
C_1^2-C_2^2 = (C_1-C_2)(C_1+C_2) = t(\deg C_1-\deg C_2)
$$
and substituting this into the previous formula we get the second formula of the theorem.
\end{proof}

\begin{remark}
Linkage is closely related to linear equivalence; here is a special case:
Suppose that $S$ is a smooth surface in $\PP^3$, and $C\subset S$ is a curve. If $T$ is a sufficiently general surface of degree $t$
containing $C$ then the curve $C'$ that is the link of $C$ with respect to $S,T$ lies in the class $tH-C$. If we link again with
respect to another surface $T'$ of degree $t'$ we thus arrive at $C'' = C+(t-t')H$. Thus if $t=t'$ we get a curve in the same 
linear equivalence class as $C$. Moreover, since every rational function on $S$ is the restriction to $S$ of the ratio of two forms of the same degree on $\PP^3$,
the set of curves on $S$ that can be obtained from $C$ by two linkages with surfaces $T, T'$ of the same degree is exactly the 
linear series $|C|$ on $S$. This idea is generalized in the notion of a \emph{basic double link}; see 
Exercise~\ref{Basic double links}.
\end{remark}

\section{Linkage of purely 1-dimensional schemes in $\PP^3$}
To say that the union of distinct reduced irreducible curves $C\cup C'$  is a complete intersection $X$ means that 
the ideal $I_X =I_C\cap I_{C'}$. Since  $I_C\cap I_{C'}$ contains $I_CI_{C'}$, the ideal quotient
$(I_X:I_C) := \{F \mid FI_C\subset I_X\}$
contains $I_{C'}$. 

On the other hand, if $F \notin I_{C'}$ and we choose $G\in I_C\setminus I_{C'}$, then $FG\notin I_{C'}$, so 
$F\notin (I_X:I_C)$, and thus
$(I_X:I_C) = I_{C'}$. It turns out that this relationship is the key to the formulas connecting the degrees and genera of $C,C'$, which hold 
for arbitrary purely 1-dimensional subschemes of $\PP^3$, as we shall see in Theorem~\ref{direct linkage}. 

\begin{definition}\label{linkage def}
Let $C,C'$ be purely 1-dimensional subschemes of $\PP^3$. We say that $C'$ is \emph{directly linked} to $C$ if there is a complete
intersection $X$ containing $C,C'$ and $(I_X:I_C) = I_{C'}$. We say that $C'$ is \emph{linked} to $C$ if they are connected by a chain of such
direct linkages, and we say that $C'$ is \emph{evenly linked} to $C$ if the chain involves an even number of direct linkages.
\end{definition}

Note that in this setting, $C$ and $C'$ can have components in common. For example, a ``rope"---that is, the subscheme $C \subset \PP^3$ defined by the square of the ideal $\cI_{L/\PP^3}$ of a line $L \subset \PP^3$---is linked to the reduced line $L$ in the complete intersection of two quadrics, as the reader will be invited to verify in Exercise~\ref{line and rope}. (This makes sense, since a rope is a flat limit of twisted cubics.)

As in the smooth case treated above, direct linkage is a symmetric relationship:
\begin{proposition}\label{link unmixed}
Let $C_1\subset \PP^3$ be a purely 1-dimensional subscheme with saturated homogeneous ideal $I_1$ and suppose that $C_1$ is contained in a complete intersection of
hypersurfaces $X := S\cap T$. The ideal $I_2 = (I_{X}:I_1)$ is a saturated ideal, defining a purely 1-dimensional subscheme and 
$I_1 = (I_{X}: I_2)$ as well.
\end{proposition}
 
\begin{proof}
Since $X$ is a complete intersection, the ideal $I_{X}$ is unmixed of codimension 2
(\cite[Proposition 18.13]{Eisenbud1995}), and it follows
that $I_2 = (I_{X}:I_1)$ is unmixed of codimension 2 as well. In particular it is saturated.
Thus it suffices to prove that $I_1 = (I_{X}: I_2)$ after localizing at a codimension 2 prime $P$
that contains $I_{X}$. 

Write $R$ for the localization at $P$ of the homogeneous coordinate ring of $X$. 
Because $I_{X}$ is a complete intersection, the ring $R$
 is zero-dimensional and Gorenstein.
By \cite[Propositions 21.1 and 21.5]{Eisenbud1995}, every finitely generated $R$-module is reflexive. Since 
$$
I_{C_2}R= \ann_{R}(I_{C_1}R) \cong Hom_R(R/I_{C_1}R, R)
$$
the proposition follows.
\end{proof}

We define the equivalence relation of \emph{linkage} among purely 1-dimensional subschemes of $\PP^{3}$ as the
closure of the relation of direct linkage, and we say that two such subschemes are \emph{evenly linked}
if they are connected by an even number of direct linkages.

\section{Degree and genus of linked curves}

The degrees and (arithmetic) genera 
of directly linked schemes are related exactly as in the simple case above:

\begin{theorem}\label{direct linkage}\label{linked genus formula}
If $C_1,{C_2}\subset \PP^3$ are purely 1-dimensional schemes that are directly linked by surfaces $S,T$ of degrees $s,t$  then 
$$
\begin{aligned}
&\deg C_1+\deg C_2 = st\\
&p_a(C_1) - p_a({C_2}) = \frac{s+t-4}{2}(\deg C_1-\deg {C_2}).
\end{aligned}
 $$
\end{theorem}

Since we have left the realm of smooth curves and surfaces, we will need a more sophisticated duality theory, and we
postpone the proof to explain the necessary ideas.



\subsection{Dualizing sheaves for singular curves}\label{duality}

Recall that in Chapter~\ref{RiemannRochChapter} we claimed that the canonical sheaf of a smooth curve---the sheaf of differential forms---was ``the most important invertible sheaf'' after the structure sheaf. In the general setting of Cohen-Macaulay schemes, the analogue of the canonical sheaf is called the dualizing sheaf.
The general definition of the dualizing sheaf is not very illuminating; what is useful is how it is constructed and its cohomological properties relating to duality.
However, having a definition may be comforting. 

\begin{definition}
Let $X$ be a projective scheme  of pure dimension $d$ over $\CC$. The \emph{dualizing sheaf} for $X$ is a coherent sheaf $\omega_X$ 
with a \emph{residue map} $\eta: H^d(\omega_X) \to \CC$ such that for every coherent sheaf  $\sF$ the composite map
$$
H^d(\sF) \times \Hom(\sF, \omega_X) \to H^d(\omega_X) \rTo^\eta \CC
$$
is a perfect pairing. 
\end{definition}

Thus if $\cF = \sL$ is an invertible sheaf on a projective curve $C$ then $\Hom(\sL, \omega_X) = \sL^{-1}\otimes \omega$,
so this relation becomes Serre duality: $H^{1}(\sL)$ is the dual of $H^{0}(\sL^{-1}\otimes \omega_{C})$.

It follows from the definition that the pair $(\omega_{X}, \eta)$ is unique up to canonical isomorphism if it exists:
The module $H^{0}_{*}(\omega_{X}) = \oplus_{n\in \ZZ}(Hom_{X}(\sO_{X}(n), \omega_{X}))$
is determined as the graded vector space dual of $H^{d}_{*}(\sO_{X})$, and the choice of $\eta$ simply fixes the isomorphism. It may not be apparent that such a sheaf exists, but we will give a construction in Section~\ref{dualizing sheaves section} below.
 
 Several properties of the dualizing sheaf on a purely 1-dimensional scheme
are the same as in the smooth case, and follow easily from the definition:

\begin{proposition}\label{similarities}
If $C$ is a purely 1-dimensional projective scheme over $\CC$, then:
\begin{enumerate}

\item $\sHom_C(\omega_C, \omega_C) = \sO_C$, and  thus if $C$ is integral then the generic rank of $\omega_C$ is 1.

\item For any invertible sheaf $\sL$ on $C$ we have: 
$$
H^1(\sL^{-1})= H^0(\sL\otimes \omega_{C});\quad %\hbox{ and }
H^0(\sL^{-1} ) = H^1(\sL\otimes \omega_{C}).
$$
In particular,
$$
\chi(\sL^{-1}) = -\chi(\sL\otimes \omega_{C}).
$$

\end{enumerate}
\end{proposition}

\begin{proof}
\noindent{\bf 1:} We claim that the natural map $\sO_C \to \sHom_C(\omega_C, \omega_C)$ is an isomorphism.
Since the map is globally defined, it suffices to prove that it is an isomorphism locally. 

Choose a Noether
normalization of $C$, that is, a finite map $f: C\to \PP^1$. 
We shall see in Theorem~\ref{canonical as Hom} below that $\omega_C \cong \sHom_{\PP^1}(\sO_C,\omega_{\PP^1})$, regarded as a sheaf on $C$ (in Theorem~\ref{canonical as Hom} this is the sheaf $f^{!}\omega_{\PP^{1}}$).
Since $C$ is purely 1-dimensional, $\sO_{C}$ is torsion-free as an $\sO_{\PP^{1}}$-module, and is thus locally 
free. 
Also, since $\PP^{1}$ is smooth, $\omega_{\PP^{1}}$ is locally
isomorphic to $\sO_{\PP^{1}}$.
But if $B$ is any commutative ring and $A$ is a $B$-algebra that is finitely generated and free as a $B$-module, then the composite map 
$
A \to Hom_{A}(Hom_B(A,B),Hom_B(A,B)) \cong Hom_{B}(Hom_B(A,B), B)
$
sending an element $a$ to the multipication by $a$ and thence to the map $f\mapsto f(a)$ may be identified with the isomorphism of $A$ to 
it's double dual as a $B$-module,
completing the proof of {\bf 1}. See Exercise~\ref{adjointness} for a generalization of this last step.


\noindent{\bf 2:} The definition of $\omega_C$ shows that, if $\sL$ is an invertible sheaf, then
$$
\begin{aligned}
H^1(\sL^{-1}) &= \Hom_C(\sL^{-1}, \omega_C))\\
&= H^0(\sHom_C(\sL^{-1}, \omega_C))\\
&= H^0(\sL\otimes_C \omega_{C}).
\end{aligned}
$$
Using item 1, we have:
$$
\begin{aligned}
H^0 (\sL^{-1})
&= H^0 (\sL^{-1}\otimes_{C}\sO_{C})\\
&= H^0 \bigl(\sL^{-1}\otimes_{C}\sHom_C(\omega_{C}, \omega_{C})\bigr) \\
&= \Hom_{C}(\sL\otimes_C \omega_{C},  \omega_{C}))\\
&= H^{1}(\sL\otimes_{C}\omega_{C}) 
\end{aligned}
$$
 \end{proof}


\section{The construction of dualizing sheaves}\label{dualizing sheaves section}
Dualizing sheaves do exist on any projective Cohen-Macaulay scheme. We have already seen constructions
in three different cases:
\begin{itemize}
 \item If $X$ is a smooth scheme of dimension $d$ over $\CC$ then $\omega_{X}= \wedge^{d} \Omega_{X/\CC}$
is a dualizing sheaf~(\cite[Hartshorne[Section III.7]{Hartshorne1977} or  \cite[p. 648, 708]{Griffiths-Harris1978}). 
\item If $f: X\to Y$ is a map of smooth curves, then $\omega_{X} = f^{*}(\omega_{Y})(\ram_{X/Y})$, where
$\ram$ denotes the ramification divisor.
\item If $X\subset Y$ is a Cartier divisor on a surface, then $\omega_{X} = \omega_{Y}(X)\mid_{X}$.
\end{itemize}

How can such different looking formulas all be correct? Grothendieck provided a general scheme
that unifies them and gives many more. 
To understand what is needed for the general case, we first consider a setting generalizing Hurwitz' theorem.
Suppose that $X\to Y$ is a finite map of projective schemes, and that $\sF$ is a coherent sheaf on $X$.

If we restrict ourselves to open affine subsets
$U := \Spec A\subset X$ mapping to $V:= \Spec B\subset Y$ via the map of rings $f^{*}:B\to A$, then 
$F := \sF_{U}$ is an $A$-module. Moreover,
$f_*{F} := f_*(\sF)(V)$ is just $F$ regarded as a $B$-module via the map $f^{*}$.

For any $B$-module $M$ the module $\Hom_{B}(A, M)$ has a natural structure of $A$-module,
where $(a\phi)(m)$ is defined to be $\phi(am)$. The functor $f^{!}(-) := \Hom_{B}(A, -): mod_{B}:\to mod_{A}$ 
defined in this way is the right adjoint of the functor $f_{*}(-): mod_{A}\to mod_{B}$, 
which means that there is a natural isomorphism of functors
$$
\Hom_{B}(f_{*}F, -) \cong \Hom_{A}(F, \Hom_{B}(A, -)) = Hom_{A}(F, f^{!}(-)).
$$

In particular if $Y$ has dualizing sheaf $\omega_Y$ and we set $o_Y := \omega_Y(V)$, then
$
\Hom_{B}(f_{*}F, o_{Y}) \cong \Hom_{A}(F, f^{!}o_{Y}),
$
Also, there is a natural transformation $\eta$ from $ f_{*}f^{!}$ to the identity functor given by the formula
$$
\eta: f_{*}f^{!}(M) = f_{*}(\Hom_{B}, (A,M)) = \Hom_{B}, (A,M) \to M:\quad \eta(\phi) = \phi(1)
$$
for any $A$-module $M$. 
The transformation $\eta$ called the \emph{counit} of the adjoint pair $(f_{*}, f^{!})$.

We also write  $f^{!}(-)$ for the sheafification of  the functor 
$\Hom_{B}(A,-)$. Again $f^{!}$ is right adjoint to $f_{*}$ on coherent sheaves,
and again there are natural maps $\eta:f_{*}f^{!}(\sF) \to\sF$. 

\begin{theorem}\label{canonical as Hom}
Let $f: X \to Y$ be a finite map of $d$-dimensional projective schemes. If $Y$ has a dualizing sheaf $\omega_{Y}$,
with residue map $\eta_{Y}: H^{d}(\omega_{Y}) \to \CC$,
then 
 $$
\omega_{X} := f^{!}\omega_{Y},
$$
with residue map 
$$
\rho_{X}: H^{d}(f^{!}\omega_{X}) = H^{d}(f_{*}f^{!}\omega_{X}) \rTo^{H^{d}(\eta)} H^{d}(\omega_{Y}) \rTo^{\rho_{Y}} \CC,
$$ 
where $\eta$ is the counit of the adjoint pair $(f_{*},f^{!})$, is a dualizing sheaf on $X$. 
\end{theorem}

\begin{proof}
Let $\sF$ be a coherent sheaf on $X$. Since $f$ is finite, we have $H^{d}(\sF) = H^{d}(f_{*}(\sF))$. Thus,
since $f^{!}(-)$ is a right adjoint of $f_{*}$ there are natural isomorphisms
$$
H^{d}(\sF) = H^{d}(f_{*}\sF) \cong^{\rho_{Y}} \Hom_{Y}(f_{*}\sF, \omega_{Y})^{\vee} \cong 
\Hom_{X}(\sF, f^{!}\omega_{Y})^{\vee},
$$
where $\cong^{\rho_{Y}}$ is the isomorphism induced by $\rho_{Y}$. One can check that the composite
isomorphism is the one induced by $\rho_{X}$, so $\omega_X = f^{!}(\omega_{Y})$, completing the proof.\end{proof}

\begin{fact}
 Given this theorem, it seems natural to look for an adjoint functor $f^{!}$ for a wider class of morphisms
$f$ but\dots in most cases, for example when $f$ is the inclusion of a divisor on a smooth surface, no such functor exists on the category of coherent sheaves! However, an adjoint functor
$f^{!}$ does exist on the derived category, where it is the right adjoint to $Rf_{*}$, leading to a theory
of dualizing complexes. 

Fortunately for the reader who is mostly interested in curves, this level of 
complication is unnecessary, and there is an intermediate level of generality that suffices
for all the purposes of this book and more: 

\begin{theorem}\label{general adjunction}
Suppose that $f: X\to Y$ is a finite map of projective schemes. If $Y$ is
Gorenstein with dualizing module $\omega_{Y}$, then
$$
f^{!}(\omega_{Y}) \cong \sExt^{\dim Y-\dim X}_{Y}(\sO_{X}, \omega_{Y})
$$
is a dualizing module for $X$.
\end{theorem}

The hypothesis is satisfied for any $Y$ that is smooth, or even locally a complete intersection.The
reason this works is that the complex
$f^{!} \omega_{Y}$ 
can be identified with its one nonvanishing cohomology module,
 $\sExt^{\dim Y-\dim X}_{Y}(\sO_{X}, \omega_{Y})$. 

See for example \cite{AltmanKleiman} for a thorough and accessible exposition.\fix{check A-K for precise hypotheses}
\end{fact}

\subsection{Proof of Theorem~\ref{direct linkage}}

\begin{proof}
 Let $X$ be the complete intersection of surfaces of degrees $s,t$ containing $C$, and let $R_X = S/(F,G)$ be its homogeneous coordinate ring, where
$S = \CC[x_0,\dots,x_3]$ is the homogeneous coordinate ring of $\PP^3$.
From the free resolution
$$
0\rTo S(-s-t) \rTo^{
\begin{pmatrix}
 G \\ -F
\end{pmatrix}}
 S(-s)\oplus S(-t) \rTo^{
\begin{pmatrix}
 F & G
\end{pmatrix}}
 S \rTo R_{X} \rTo 0
$$
 and Theorem~\ref{general adjunction} we see that
 $$
\omega_X =  \sExt^2_C(\sO_X, \omega_{\PP^3}) =\sExt^2(\sO_X, \sO_{\PP^3}(-4)) = \sO_X(s+t-4).
 $$
Note that for any ideals $J\subset I$ in a ring $A$ we have $Hom_A(A/I, A/J) \cong (J:I)/J$, where the isomorphism
sends a homomorphism $\phi$ to the element $\phi(1)$. From Theorem~\ref{general adjunction} we have 
$$
\omega_C = \sHom_X(\sO_C, \omega_X) = \sHom_X(\sO_C, \sO_X)(s+t-4) = \frac{\sI_X:\sI_C}{\sI_X}(s+t-4),
$$
where we have identified $\sO_C$ with its pushforward under the inclusion map $C\to X$. 

By Proposition~\ref{similarities}
$\chi(\omega_C(m)) = -\chi(\sO_C(-m))$. It follows that the leading coefficient of the Hilbert polynomial of $\omega_C$, is 
equal to $\deg C$, and thus
$$
st = \deg \sO_X = \deg \sO_{C'}+\deg \sO_C,
$$
as required by the formula for the sum of the degrees.

From Theorem~\ref{canonical as Hom} (or Theorem~\ref{general adjunction}) we see that $\chi(\sO_X) = st(4-s-t)/2$. Since $\sO_{C'} = \sO_{\PP^3}/(\sI_X : \sI_C)$ and
$(\sI_X : \sI_C)/(\sI_X) = \omega_C(4-s-t)$ we have
$$
\begin{aligned}
&\frac{4-s-t}{2}(\deg C+\deg C')\\
&=\frac{4-s-t}{2}st\\
&=\chi(\sO_X) \\
&=  \chi(\sO_{C'})+\chi(\omega_C(4-s-t)) \\
&= \chi(\sO_{C'})-\chi(\sO_C(s+t-4)) \\&= \chi(\sO_{C'})-(s+t-4)\deg C-\chi(\sO_C)\\
&= (1-p_a(\sO_{C'})) - (1-p_a(\sO_C)) -(s+t-4)\deg C,
\end{aligned}
$$
whence 
$$
p_a(\sO_C) -p_a(\sO_{C'}) = \frac{(s+t-4)}{2} (\deg C - \deg C'). 
$$
 \end{proof}

Linkage also behaves in a simple way with respect to deficiency modules. The following result was attributed in \cite{MR364271} to Daniel Ferrand:

\begin{theorem}\label{HR}
If $C,C'$ are purely 1-dimensional subschemes of $\PP^3$ that are directly linked by a complete intersection of degrees $s,t$ then
$$
D(C') = Hom_\CC(D(C), \CC) (-s-t+4).
$$ 
as graded modules over the homogeneous coordinate ring of $\PP^3$.
\end{theorem}

\begin{proof}
Suppose that the homogeneous ideal of $C$ is generated by forms of degree $a_i, i=1,\dots,s$. Since $C$ is locally Cohen-Macaulay,
the local rings $\sO_{C,p}$ have projective dimension 2 as modules over $\sO_{\PP^3, p}$, and $\sI_{C,p}$ has projective dimension 1.
Thus we have an exact sequence
$$
0\to \sE \to \oplus_i\sO_{\PP^3}(-a_i) \to \sI_C \to 0.
$$
Since the first and second cohomology groups of the twists of $\sO_{\PP^3}$ vanish, we deduce an isomorphism
$$
D(C) := \oplus_{m\in \ZZ} H^1(\sI_C(m)) \cong \oplus_{m\in \ZZ} H^2(\sE(m)).
$$

Let $X$ be the complete intersection of two hypersurfaces, of degrees $s,t$, containing $C$. From the inclusion we deduce a
map of resolutions
$$
\begin{diagram}[small]
0&\rTo& \sE &\rTo& \oplus_i\sO_{\PP^3}(-a_i)                                         &\rTo&\sO_{\PP^3}&\rTo &\sO_C &\rTo& 0\\
&&\uTo&&\uTo&&\uTo_{=}&&\uTo\\
0&\rTo& \sO_{\PP^3}(-s-t) &\rTo& \sO_{\PP^3}(-s)\oplus \sO_{\PP^3}(-t) &\rTo& \sO_{\PP^3}&\rTo& \sO_X &\rTo& 0\\
\end{diagram}
$$
We dualize this diagram, form the mapping cone, and twist by $-s-t$. Note that $\Hom_{\PP^3}(\sO_C, \sO_{\PP^3}) = 0$. 
Also, since the vertical map $\sO_{\PP^3}\to \sO_{\PP^3}$ on the right
is the identity we may cancel these terms in the mapping cone. Noting that $\omega_C = \sExt^2(\sO_C, \sO_{\PP^3}(-4))$ the result is a diagram with 
exact rows:
$$
\begin{diagram}[small]
 0&\lTo&\omega_C(-s-t+4)&\lTo&\sE^*(-s-t) &\lTo&  \oplus_i\sO_{\PP^3}(a_i-s-t)&\lTo&  0\\
 &&\dTo^\phi&&\dTo&&\dTo\\
 0&\lTo&\sO_X&\lTo&\sO_{\PP^3} &\lTo& \sO_{\PP^3}(-t)\oplus \sO_{\PP^3}(-s) \\
 &&\dTo\\
 &&\sO_{C'}\\
 &&\dTo\\
 && 0
\end{diagram}.
$$
\fix{Xianglong attributed this to Ferrand. Reference?? Jan 2: DE writes to Xianglong to ask}
The map $\phi$ is a monomorphism because $(\sI_X:\sI_C)/\sI_X \cong \omega_C(-s-t+4)$, as explained above, so the column on the left is a short exact sequence.
We can now write a resolution of $\sI_{C'}$ as the mapping cone:
$$
\begin{diagram}
0\leftarrow \sI_{C'} \leftarrow \sO_{\PP^3}(-t)\oplus \sO_{\PP^3}(-s) \oplus \sE^*(-s-t) \leftarrow \oplus_i\sO_{\PP^3}(a_i-s-t)\leftarrow  0.
\end{diagram}
$$
From this we see that 
$$
H^1(\sI_{C'}(m)) \cong H^1(\sE^*(-s-t+m)) \cong Hom_\CC( H^2(\sE(s+t-m-4)), \CC)
$$
where the last equality is from Serre duality on $\PP^3$. Summing over $m$ we see that
$D(C') \cong \Hom_\CC(D(C)(s+t-4), \CC)$,
and since Serre duality is functorial, the isomorphism holds not only as graded vector spaces, but as graded $S$-modules. \end{proof}

Sometimes the following consequence is a useful way to compute the deficiency module:

\begin{proposition}\label{deficiency as dual of Ext}
If $C$ is a purely 1-dimensional subscheme of $\PP^3$ with homogeneous ideal $I = I_C$ then 
$$
D(C) \cong Hom_\CC (Ext^3(S/I, S), \CC)(-4), \CC)
$$
as graded modules over the homogeneous coordinate ring $S$ of $\PP^3$.
\end{proposition}

\begin{proof}
We may choose a surjection  $\psi:  \oplus_iS(-a_i)\rTo I$, and choose the map
$\phi: \oplus_i\sO_{\PP^3}(-a_i)\rTo\sI_C$
in the proof of Theorem~\ref{HR}
to be the corresponding map of sheaves, so that
$\sE$ is the sheafification of the graded module $E = \ker \psi$.

Since $I$ is a saturated ideal,
 the depth of $S/I$ is at least 1, so $\pd\ S/I\leq 3$, and $I$ has a free resolution of the form
$$
0\rTo G \rTo F \rTo \oplus_iS(-a_i)  \rTo S\rTo S/I \rTo 0.
$$
where $G\to F$ is a free presentation of $E$. and there is an exact sequence
$$
0 \to E^* \to F^* \to G^* \to Ext^3_S(S/I, S) \to 0.
$$
By Theorem~\ref{Auslander-Buchsbaum}, the fact that $C$ is Cohen-Macaulay implies that the projective dimension of each of its
local rings is $\leq 2$,  and it follows that
$Ext^3_S(S/I, S)$ has finite length. Writing $\widetilde{(\phantom{-})}$ for the sheafification functor,
we have a short exact sequence of sheaves 
$$
0\to \sE^* \to \widetilde{F^*} \to \widetilde{G^*}\to 0.
$$
From this we see that 
$$
Ext^3_S(S/I,S) = H^1_*(\sE^*) = Hom_\CC( H^2_*(\sE(-4)),\CC) = H^1_*(\sI)(-4),
$$
proving the assertion.
\end{proof}

Proposition~\ref{deficiency as dual of Ext} is actually a special case of the local duality isomorphism between local cohomology and the dual of Ext; see for example \cite[Theorem A.1.9]{MR2103875}.

\section{The linkage equivalence relation}
As an immediate consequence of Theorem~\ref{HR} we have:
\begin{corollary}(Hartshorne)
 If two curves $C,C'$ are linked by an even length chain of direct linkages, then 
 $D(C)$ and $D(C')$ are isomorphic up to a shift in grading.\qed
\end{corollary}

As we mentioned at the beginning of this Chapter, the converse is also true: the Hartshorne-Rao modules, up to shift in grading, provide a complete invariant of
linkage. Even more precise results are known (and the characteristic 0 hypothesis is largely unnecessary); here is a sample:

\begin{fact}
\begin{theorem}
Let $S = \CC[x_0, \dots, x_3]$ be the homogeneous coordinate ring of $\PP^3$, and let $M$ be a graded $S$-module of finite length.
\begin{enumerate}
\item There is a smooth curve $C$ with $D(C) = M(m)$ for some integer $m$.
\item There is a minimum value of $m$ such that $M(m) = D(C_0)$ for some purely one-dimensional scheme $C_0$.
\end{enumerate}
\end{theorem}

Moreover, each Liaison class has a relatively simple structure, known as the \emph{Lazarsfeld-Rao property}.
We say that $C'$ is obtained from $C$ by an \emph{ascending double link} if $I_{C'} = fI_C+(g)$ for some regular sequence
contained in $I_C$---see Exercise~\ref{Basic double links}. 

\begin{theorem}\cite{MR1087803}\label{LR property}
Let $M = D(C_0)$ the the Hartshorne Rao invariant of a purely 1-dimensional subscheme of $\PP^3$, and suppose that
$M$ is minimal in the sense that no $M(m)$ with $m>0$ is the invariant of a purely 1-dimensional scheme. 
\begin{enumerate}
 \item Any curve in $\PP^3$ with $D(C) = M$ is a deformation of $C_{0}$ through curves with invariant $M$.
 \item Every curve in the even linkage class of $C_0$ is the result of a series of ascending double links followed by a deformation.
\end{enumerate}
\end{theorem}

In \cite{MR714753} it is shown that general curves  in $\PP^3$ that have 
reasonably large degree compared to their genus are minimal in the sense 
of  Theorem~\ref{LR property}.
\end{fact}

\section{Comparing the canonical sheaf with that of the normalization}

In Chapter~\ref{PlaneCurvesChapter} we boasted in that we could effectively compute
linear series on a smooth curve $C$ given any plane curve $C_0$ with normaliztion $C$, 
and we showed how to do this when the plane curve has only nodes. To complete the discussion 
we need to compare the canonical sheaf of $C$ with
that of $C_0$; that is, we need a formula for the adjoint ideal of any curve singularity.



\begin{theorem}\label{general adjoint}
If $\nu: C \to C_0$ is the normalization of a reduced connected projective curve then the 
adjoint ideal 
$$
\fA_{C/C_0} :=\ann_{\sO_{C_0}}\frac{\omega_{C_0}}{\nu_* \omega_C}
$$
is equal to the conductor ideal
$$
\mathfrak f_{C/C_0}: = \ann_{\sO_{C_0}}\frac{\nu_* \sO_C}{\sO_{C_0}}	
$$
Moreover, if $C_0$ is a plane curve, then 
$$
\delta(C_0) = \length \frac{\nu_* \sO_C}{\sO_{C_0}} = \length \frac {\sO_{C_0}}{\mathfrak f_{C/C_0}}.
$$
\end{theorem}

Here $\delta(C_0)$ is the ``number of nodes equivalent to the singularities of $C_0$'' as
explained in Chapter~\ref{RiemannRochChapter}.  The property in the last statement of the
theorem  was first noted (for plane curves) in Daniel Gorenstein's thesis under Oscar Zariski\footnote{Gorenstein is better remembered for his work on the classification of finite simple groups.}. This is the
reason why Grothendieck gave the name ``Gorenstein" to Cohen-Macaulay rings that have cyclic canonical 
modules--see~\cite{Bass}. 
This condition is always satisfied for singularities of curves in the plane, and we exploited this fact in Chapter~\ref{PlaneCurvesChapter}.
See Example~\ref{nongorenstein} for a singularity that is not locally planar, and behaves differently.


\begin{proof}[Proof of Theorem~\ref{general adjoint}]
  Let 
$\rho: C_0 \to \PP^1$ be a finite morphism. Both $\rho_*\nu_*\sO_C$ and $\rho_*\sO_{C_0}$
are torsion free coherent sheaves over $\sO_{\PP^1}$, and are thus locally free. Since $\rho_*$ is left-exact,
the inclusion $\sO_{C_0} \subset \nu_*\sO_C$ pushes forward to an inclusion
$$
\alpha: \rho_*\sO_{C_0} \hookrightarrow \rho_*\nu_*\sO_C
$$
and since $\sO_C$ is equal to $\sO_{C_0}$ generically on $\PP^1$, the cokernel $\coker \alpha$ has finite length; indeed, it is supported on the 
image in $\PP^{1}$ of the singular locus of $C_0$. Since the maps $\nu$ and $\rho$ are finite, we may harmlessly think of both 
$\sO_{C_0}$ and $\sO_C$ as coherent sheaves on $\PP^1$, and we will simplify the notation by dropping $\nu_*$ and $\rho_*$.
Taking duals into $\omega_{\PP^{1}} = \sO_{\PP^1}(-2)$ and defining
$\alpha^\vee := \Hom_{\PP^1}(\alpha, \sO_{\PP^1}) $ we get a map that fits into the long exact sequence
of $\sExt_{\PP^1}(-, \omega_{\PP^{1}})$:
$$
\begin{diagram}[small]
 0&\rTo& 
 \Hom_{\PP^1}(\coker \alpha,\omega_{\PP^{1}})
&\rTo&
\omega_{C}
& \rTo^{\alpha^\vee}&
\omega_{C_0}\\
&\rTo&
\sExt^1_{\PP^1}(\coker \alpha,\omega_{\PP^{1}})
&\rTo&
\sExt^1_{\PP^1}(\sO_C,\omega_{\PP^{1}})
&\rTo&
\cdots
\end{diagram}
$$
Since $\coker \alpha$ has finite support, $ \Hom_{\PP^1}(\coker \alpha,\omega_{\PP^{1}}) = 0$ and
$\sExt^1_{\PP^1}(\coker \alpha, \omega_{\PP^{1}})$ has the same length and the same annihilator
as $\coker \alpha$. Furthermore, because $\sO_C$ is locally free as an $\sO_{\PP^1}$-module, the term
$\sExt^1_{\PP^1}(\sO_C, \omega_{\PP^{1}})$ vanishes, and we get the more manageable exact sequence
$$
\begin{diagram}
 0&\rTo&
\omega_{C}
& \rTo^{\alpha^\vee}&
\omega_{C_0}
&\rTo&
\sExt^1_{\PP^1}(\coker \alpha, \sO_{\PP^1})
&\rTo&
0
\end{diagram}
$$
It follows that the sheaves $\nu_*\sO_C/\sO_{C_0}$ and $\omega_{C_0}/\nu_*\omega_C$ have the same
length $\delta(C_0)$. Note that the \emph{conductor} $\mathfrak f_{C/C_0}$ of $C/C_0$, which is defined to be the annihilator of $\sO_C/\sO_{C_0}$ 
in $\sO_{C_0}$, is at the same time an ideal sheaf of $\sO_{C_0}$ and an ideal sheaf of $\sO_C$ via the inclusion $\sO_{C_0}\subset \sO_C$. The argument above shows that $\mathfrak f_{C/C_0}$ is also the annihilator ideal of $\omega_{C_0}/\omega_C$. By definition, this is the adjoint ideal of $C_0$, proving the first statement
of the theorem.

A further simplification occurs when $C_0\subset \PP^2$ is a plane curve, or more generally any 
Gorenstein curve.
If the defining equation of 
$C_0$ is the form $F$ of degree $d$, then there is a locally free resolution of  $\sO_C$ of the form
$$
0\rTo \sO_{\PP^2}(-d) \rTo^F \sO_{\PP^2} \rTo \sO_C \rTo 0
$$
and thus $\omega_C \cong \sExt^1_{\PP^2}(\sO_{C_0}, \omega_{\PP^2})$
is the cokernel of the map $\sO_{\PP^2}(-3) \to \sO_{\PP^2}(d-3)$ given by multiplication by $F$. Thus
$\omega_{C_0}$ is locally cyclic, and  since the support of $\omega_{C_0}/\omega_C$ is finite, this module is
 globally cyclic, so
$\omega_{C_0}/\omega_C \cong \sO_{C_0}/\mathfrak f_{C/C_0}$.
Since the length of $\omega_{C_0}/\omega_C$ is equal to the length of 
$\sO_{C}/\sO_{C_0}$, the last statement of the theorem follows.
\end{proof}

\begin{example}
 Working locally, consider the germ of a node, represented by the ring $R_0:= k[[x,y]]/(xy)$, and
 the projection to a line represented by the inclusion 
 $$
 P:= k[[t]] \subset R_0: t\mapsto x+y.
 $$
 The normalization of $R_0$ is the map $R_0 \to R := k[[x]]e_1\times k[[y]]e_2$,
 where $e_1= x/t, e_2= y/t$ are orthogonal idempotents. Writing 
 $Q_1 = k((t)) \subset Q:= k((x))\times k((y))$
 for the map of total quotient rings, we know that, because the extension is separable, the trace map
 $Tr := Tr_{Q/Q_1}: Q \to Q_1$ generates $\Hom_{Q_1}(Q,Q_1)$ as a $Q$-vector space. Thus we may write the
 elements of $\omega_{R_0} = \Hom_{P}(R_0,P)$ and $\omega_R = \Hom_{P}(R,P)$ as
 multiples of $Tr$ by elements of $Q$.
 
 Since $R \cong Pe_1\oplus Pe_2$ as a $P$-module, the module $\Hom_P(R,P)$ is
 generated by the two projections, and it is easy to check that these are the maps
 $(x/t)Tr$ and $(y/t)Tr$. One can also check easily that 
 $$
 g := \frac{x-y}{t^2} Tr \in \Hom_P(R_0,R).
 $$
 Since
$xg = x/t$ and $yg = y/t$ in $Q$ we have
 $xg = x/tTr$ and $yg= y/tTr$, the generators of $\Hom_{P}(R,P)$. 
 The ring $R_0$, regarded as a $P$-module, is freely generated by 1 and $x$.
 Immediate computation shows that $gTr(1) = 0$ while $gTr(x) = 1$.
 Furthermore $(x-y)g = 1$ in $Q$,  and thus $\frac{1}{2}(x-y)gTr$ takes 1 to 1
 and $x$ to $t$, proving that $gTr$  generates
 $\Hom_P(R_0,P)$. We also see directly from this that the adjoint ideal
 of $\omega_{R_0}/\omega_R$ is the conductor ideal $(x,y)R_0 = (x,y)R = \mathfrak f_{R/R_0}$,
 as shown by  Theorem~\ref{general adjoint}. 
\end{example}

\begin{example}\label{nongorenstein}
We showed earlier that the adjoint ideal of a planar triple point is the square of the ideal of the point. Consider now
  the germ of a spatial triple point singularity $R_0 := k[[x,y,z]]/(xy, xz,yz)$, the vertex of the cone over 3
 non-colinear points in the plane, with normalization $R = k[[x]]\times k[[y]]\times k[[z]]$.
 If we set $t = x+y+z\in R_0\subset R$, then the three idempotents that generated $R$ as an $S$-module
 can be written as $x/t, y/t, z/t$. We have $x^2/t = xt/t= x$ and similarly for $y$ and $z$, so
 the conductor is given by $\ff_{R/R_0} = (x,y,z)R = (x,y,z)R_0$.
 
 The ring $R$ is also finite over $S := k[[x,y,z]]$, and is isomorphic as such to 
 $S/(x,y)\oplus S/(x,z) \oplus S/(y,z)$, so we can compute its canonical module from the
 resolution that is the direct sum of the three Koszul complexes resolving these summands, and 
 as such the canonical module $\omega_R = \Ext^2(R,S)$ is isomorphic to $R$ as an $S$-module.
 

 The $S$-module $R_0$, on the other hand,   has free resolution
 $$
 0\rTo S^2\rTo^{
	\begin{pmatrix}
 0&z\\
 y&-y\\
 -x&0
\end{pmatrix}}
 S^3 \rTo^{
\begin{pmatrix}
 xy&xz&yz
\end{pmatrix}}
 S \rTo R_0 \rTo 0. 
 $$
The canonical module of $R_0$ (which would be the germ of the canonical module of a global curve with such a singularity) is thus 
 $$
 \omega_{R_0} = \Ext^2(R_0, S) = 
 \coker 
\begin{pmatrix}
 0&y&-x\\z&-y&0
\end{pmatrix}
$$
a module requiring 2 generators.
\end{example}
The inclusion $\omega_{R} \subset \omega_{R_0}$ is  induced by the 
dual of the last map in a comparison map between the two resolutions. Computation
confirms the conclusion of Theorem~\ref{general adjoint} that $\fA_{R/R_0} = \ff_{R/R_0} = (x,y,z)$. However, the length of $R/R_0$ is 2, and the length of $R_0/\ff_{R/R_0} = R_0/(x,y,z)$ is  1, so the
last conclusion of Theorem~\ref{general adjoint} fails, along with its hypothesis.

\section{A general Riemann-Roch theorem}

Using dualizing sheaves for singular curves, we can
state a general version of the Riemann-Roch theorem.  We will not make use of the generality further, so we only sketch the argument.

We first need a general definition of the degree of a sheaf:

\begin{definition}
 If $\sF$ is a sheaf of generic rank $r$ on a projective reduced and irreducible curve $C$ over $\CC$ we define the degree
 of $\sF$ to be $\chi(\sF) -r\chi(\sO_C)$.
\end{definition}

\begin{fact}
The degree of $\sF$ is actually the degree of a divisor class called the first Chern class of $\sF$. See
for example \cite[Chapter 5]{3264}
for more information. 
\end{fact}

Thus:
\begin{proposition}\label{general RR without duality}
$$
 \chi(\sF) = \deg \sF + r\chi(\sO_C) = \deg \sF + r(1-p_a(C)).
 $$
\end{proposition}
 
This statement would be
a tautology if there were no other way to compute $\deg(\sF)$, but there is:

\begin{fact} Let $C$ be a reduced and irreducible projective curve, let $\sF$ be a coherent sheaf on $C$ of generic rank $r$, and let $\sL$ be an invertible sheaf on $C$.
\begin{enumerate}
\item If $\sF$ is generated by its global sections and $\sigma_1,\dots, \sigma_r$ is a maximal generically independent
collection of global sections of $\sF$,  then $M = \coker(\sO_C^r \rTo^{(\sigma_1,\dots, \sigma_r)}\sF)$
has finite support, and
$$
\deg (\sF) = r \chi(\sO_C)+
\dim_\CC H^0(M)\,;
$$

\item $\deg (\sL \otimes \sF) = \deg (\sF) +r\deg (\sL).$

\end{enumerate}
Thus if $\sO_C(1)$ is a very ample invertible sheaf on $C$ and $m$ is a sufficiently large integer so that
$\sF(m)$ is generated by global sections, then the degree of $\sF(m)$ and the degree of $\sO_C(1)$ are computed by the formula in item (1)
and $\deg \sF = \deg(\sF(m)) - m \rank(\sF) \deg (\sO_C(1))$.

The proof of these assertions uses nothing more than the additivity of the Euler characteristic.
\end{fact}
%
%\begin{proof}
%The Euler characteristic is additive on short exact sequences. For example if $\sG$ is a coherent sheaf of finite support, then $\sG$ can be written as an iterated extension of
%copies of the skyscraper sheaves with 1-dimensional fibers at various closed points $p_i\in C$ we see that $\chi(\sG) = \dim_\CC(H^0(\sG))$. Applying additivity again, we get item 1.
%
%For item 2, note first that we can write $\sL$ as $\sL_1\otimes_C(\sL_2)^{-1}$ for very ample invertible sheaves
%$\sL_i$ such that $\sL_i\otimes_C\sF$ is generated by global sections, so it is enough to prove (2) when $\sL$ is
%a very ample invertible sheaf or its inverse. 
%
%If $\sL$ is very ample and $\sL\otimes \sF$ is generated by global sections,
%and the generic rank of $\sF$ is $r$, then by choosing $r$ general sections we get a short exact sequence
%$0\rTo \sO_C^r \rTo^{(\sigma_1,\dots, \sigma_r)}\sF \rTo \sG\rTo 0$
%where $\sG$ has finite support. Since $\sG \otimes \sL \cong \sG$ for any invertible sheaf $\sL$, it suffices to 
%prove the given formula when $\sF \cong \sL^{\oplus r}$, or even for $\sL$ itself. Since $\sL$ is very ample,
%it has the form $\sO_C(D)$ for some effective divisor $D$ supported on the smooth locus of $C$, 
%and, applyng $\chi$ to the exact sequences 
%$$ 
%\begin{aligned}
%&0\to \sO_C \to \sO_C(D)  \to \sO_D \to 0\\
%&0\to \sO_C(-D) \to \sO_C  \to \sO_D \to 0\\
%\end{aligned}
%$$
%the result follows (this is the usual proof of the usual Riemann-Roch theorem in the smooth case.)
%\end{proof}

Using the dualizing property of $\omega_C$ we can reformulate Theorem~\ref{general RR without duality} as
\begin{theorem}\label{general RR with duality}
If $C$ is a reduced and irreducible curve and $\sF$ is a coherent sheaf on $C$, then
$$
h^0(\sF) = \deg(\sF) + \rank(\sF)(1-p_a(C)) + h^0(\Hom_C(\sF, \omega_C)).
$$
\end{theorem}



\section{Exercises}

\begin{exercise}
 Verify that the genus formula in Theorem~\ref{direct linkage} agrees with the usual calculation of degrees and genera for divisors on a quadric of
 classes $(a,b)$ and $(d-a, d-b)$.
\end{exercise}

\begin{exercise}
 Let $C$ be a reduced and irreducible projective curve, and let $\sE$ be a locally free sheaf of rank $r$ on $C$. Show that
 $\deg(\sE) = \deg(\wedge^r(\sE))$.
 
 Hint: First show that any locally free sheaf on $C$ is an iterated extension of invertible sheaves.
\end{exercise}

\begin{exercise}
Let $C$ be the disjoint union of 3 skew lines. 

\begin{figure}
\inprogress
\centerline {\includegraphics[height=2.5in]{"main/Fig15-2-SinglyRuledHyperboloid"}}
\caption{A family of disjoint lines on a smooth quadric. [Being redrawn]}
\label{Fig15.2}
\end{figure}

\begin{enumerate}
 \item prove that $C$ lies on a unique quadric, and that $H^2(\sI_C) = 0$
 \item compute the Hartshorne-Rao module $D(C)$.
 \item show that if $\Gamma$ is the union of 3 points in $\PP^3$ then
 $H^1\sI(\Gamma) = 0$ if and only if the three points are colinear.
 \item Using the exact sequence in cohomology coming from the short exact sequence
$$
0\to \sI_C \rTo^{\ell} \sI_C(1) \to \sI_\Gamma(1) \to 0
$$
where $\ell$ is a linear form, show that the map of vector spaces
$$
H^1(\sI_C) \rTo^{\ell} H^1(\sI_C(1))
$$
has rank$<2$ if and only if $\ell$ vanishes on 3 collinear points on the three lines (including the case when $\ell$ vanishes identically on one of the lines).
Conclude that if a different union $C'$ of 3 skew lines is linked to $C$, then $C'$ lies on the same quadric as $C$.
\end{enumerate}
See~\cite{Migliore} for more examples of this type.
\end{exercise}

\begin{exercise}
 Compute the Hilbert function of the Hartshorne-Rao module of a curve of type $(a,b)$ on a smooth quadric surface.
 
 Hint: The ideal sheaf of the curve on the quadric $Q$ is an extension of the ideal sheaf of the quadric in $\PP^3$
 with the ideal sheaf of the curve on the quadric, which is 
 $$
 \sO_Q(-a,-b) = \pi_1^*(\sO_{\PP^1}(-a)) \otimes \pi_2^*(\sO_{\PP^1}(-b)),
 $$
 where $\pi_1, \pi_2$ are the projections to $\PP^1$. Use the K\"unneth formula
 $$
 H^1(\sO_Q(p,q)) = H^1(\sO_{\PP^1}(p)) \otimes H^0(\sO_{\PP^1}(q)) \oplus
  H^0(\sO_{\PP^1}(p)) \otimes H^1(\sO_{\PP^1}(q))
 $$
  to compute the necessary cohomology.
\end{exercise}

\begin{exercise} (Liaison addition)\label{Liaison addition}
(From the unpublished Brandeis thesis of Phillip Schwartau):
Suppose that $I, J$ are saturated ideals defining purely 1-dimensional subschemes of $\PP^3$
and that $f,g$ is a regular sequence with $f\in I$ and $g\in J$.
Prove that $g I \cap fJ = (fg)$, and conclude that if $I,J$ are saturated codimension 2 ideals
 defining purely 1-dimensional schemes $C,C'$ in $\PP^3$
 then  $(gI+fJ)$ is a saturated ideal defining a purely 1-dimensional
scheme $C''$ with $D(C'') = D(C)(-\deg g) \oplus D(C')(-\deg f)$.

Hint: Use the exact sequence 
$$
0\to (fg) \to gI \oplus fJ \to gI+fJ \to 0
$$
and the corresponding exact sequence of quotients by these ideals.
\end{exercise}


\begin{exercise}(Basic double links)\label{Basic double links}
The special case of the construction in Exercise~\ref{Liaison addition} in which $C'$ is trivial is already interesting. 

\begin{enumerate}
 \item Show that if $I$ is a saturated ideal of codimension 2
 defining a purely 1-dimensional scheme $C$ in $\PP^3$
 and $(f, g)$ is a regular sequence with $g\in I$, 
 then  $fI+(g)$ defines a scheme $C'$ with $D(C') = D(C)(-\deg f)$.

 \item Show directly that, with notation as above, $C'$ is directly linked to $C$
 in two steps. Since the degrees of the generators of $D(C')$ are more positive, this
 is sometimes called an \emph{ascending double link}. Geometrically it amounts to taking the
union of $C$ with some  components that are complete intersections.
 \end{enumerate}

\end{exercise}

\begin{exercise}~\label{adjointness}
Here is a more general form of the last step in the proof of part {\bf 1} of
Propositions~\ref{similarities}. Suppose that $B\to A$ is a homomorphism of rings, $X$ is an $A$-module and $Y$ is a 
$B$-module. Show that there is a natural transformation
$$
\phi: \Hom_A(X, \Hom_B(A, Y)) \cong \Hom_B(X,Y)
$$
and that if $X = \Hom_B(A,Y)$, then the map
$$
A\to \Hom_A(\Hom_B(A,Y), \Hom_B(A, Y)) 
$$
taking an element  $a\in A$ to multiplication by $a$ on the $A$-module  $\Hom_B(A, Y)$
is sent by $\phi$ to the evaluation map $\alpha \mapsto \alpha(a)$ for
$\alpha \in \Hom_B(A,Y)$.

 
\end{exercise}

\subsection{Ropes and Ribbons}
The simplest way to construct well-behaved nonreduced curves is
to take neighborhoods of smooth ones. Ropes and ribbons are examples of this sort:


\begin{definition}
The \emph{rope defined from a curve $C \subset \PP^n$} is the scheme $V(I^2_C)$ defined by the square of the ideal $C$.
\end{definition}

\begin{exercise}\label{hilbert function of rope}
If $C$ is the rope defined from a line $L\subset \PP^3$ then the Hilbert function $h_C(m)$ and Hilbert polynomial $p_C(m)$ are both equal to $3m+1$. Thus $C$ has degree 3 and
arithmetic genus 0. Note that the degree can also be computed as the degree of 
a general hyperplane section, since this is defined by the square of the ideal of a point
in $\PP^2$.

Hint: Count the monomials of each degree in square of the ideal of a line. 
\end{exercise}

\begin{exercise} To see why the rope in Exercise~\ref{hilbert function of rope} should look like a twisted cubic, show that it is the flat limit of a twisted cubic as follows:
 Let $X \subset \PP^3$ be the twisted cubic with parametrization $x_i = s^it^{3-i}$. Consider the one-parameter subgroup of $PGL_4$ given in homogeneous coordinates $x_0,\dots, x_3$ on $\PP^3$ by
$$
A_t : (x_0,\dots,x_3) \mapsto (tX_0, X_1, X_2,tX_3).
$$
Show that the flat limit, as $t\to 0$, of the twisted cubics $A_t(C)$ is the rope $V(X_0^2, X_0X_1,X_1^2)$.

Hint: Use the description of $I(X)$ as the ideal of $2\times 2$ minors of
$$
\begin{pmatrix}
 x_0 &x_1&x_2\\
x_1& x_2& x_3
\end{pmatrix}
$$
\end{exercise}
 
 
\begin{exercise}\label{line and rope}
We saw in Section~\ref{LinkageIntro} that a twisted cubic curve is linked to a line by the complete intersection
 of two quadrics. Show that the same is true for the rope of Exercise~\ref{hilbert function of rope} 
\end{exercise}
 
If $C$ is the rope defined from a line in $\PP^2$, then the Zariski tangent space to $C$ at
any point is 2-dimensional; that is, it looks like a ribbon. More generally:

\begin{definition}
By a \emph{ribbon} $X \subset \PP^n$ we will mean a scheme of pure dimension 1 and multiplicity 2 whose support is a smooth, irreducible curve $C \subset \PP^n$ and whose Zariski tangent space at every point is 2-dimensional.
\end{definition}


\begin{figure}
\centerline {\includegraphics[width=2.4in]{"main/Fig15-3"}}
\caption{A ribbon can be thought of as the first infinitesimal neighborhood
of a curve on a smooth surface. {Silvio: the two curves don't actually intersect. possible to indicate?}}
\label{Fig15.3}
\end{figure}

\begin{exercise}
Suppose that $C\subset \PP^n$ is a ribbon. Show that $C$ is contained in the rope defined
from $C_{\rm red}$, and show that the degree of $C$ is twice the degree of $C_{\rm red}$.

Hint: look at hyperplane sections of $C$.
\end{exercise}

Unlike ropes, there are many different ribbons $C$ with the same smooth curve $C_{\rm red}$,
and they can have different arithmetic genera. 
Suppose that $C\subset \PP^3$ is a ribbon such that $X = C_{\red}$ is the line $V(x_0,x_1)$.
Since $C$ is contained in the rope defined from $X$ we must have $(x_0^2, x_0x_1,x_1^2) \subset I(C)$. The tangent space to $C$ at a point $(0,0,s,t)$ meets the line  $X' = V(x_2, x_3)$
at some point $(F(s,t),G(s,t),0,0)$, so $F$ and $G$ define a morphism $\PP^1 \to \PP^1$;
thus they are homogeneous polynomials of the same degree $d$.  It follows that $I(C)$ also contains the element
$x_0 G(x_2, x_3) - x_1F(x_2,x_3)$. Show that the ideal of $C$ is obtained by adding this form to the ideal of the rope, that is,
$$
I_C \; = \; \big(X_0^2, X_0X_1, X_1^2, F(X_3,X_4)X_0 + G(X_3,X_4)X_1\big)
$$
In case $d=1$, show that $C$ lies on a smooth quadric 
\subsection{General adjunction}

The next two exercises illustrate Theorem~\ref{general adjunction}:

\begin{exercise}\label{codimension0}
Show that if $C\to D$ is a map of smooth curves with ramification index $e$ at $p\in C$, and $t$ is a local
analytic parameter at $p$, then 
locally analytically at $p$ the sheaf $\sHom_C(\sO_C, \omega_D)$ is $\sO_C(e)$.
\end{exercise}

\begin{exercise}\label{codimension1}
 Show that if $C\subset S$ is a Cartier divisor on a surface $S$ with canonical sheaf $\omega_S$, 
 then $\sExt^1(\sO_C, \omega_S) \cong \sO_C\otimes \sO_S(C)$, and thus $K_C = (K_S+C)\cap C$.
\end{exercise}


%footer for separate chapter files

\ifx\whole\undefined
%\makeatletter\def\@biblabel#1{#1]}\makeatother
\makeatletter \def\@biblabel#1{\ignorespaces} \makeatother
\bibliographystyle{msribib}
\bibliography{slag}

%%%% EXPLANATIONS:

% f and n
% some authors have all works collected at the end

\begingroup
%\catcode`\^\active
%if ^ is followed by 
% 1:  print f, gobble the following ^ and the next character
% 0:  print n, gobble the following ^
% any other letter: normal subscript
%\makeatletter
%\def^#1{\ifx1#1f\expandafter\@gobbletwo\else
%        \ifx0#1n\expandafter\expandafter\expandafter\@gobble
%        \else\sp{#1}\fi\fi}
%\makeatother
\let\moreadhoc\relax
\def\indexintro{%An author's cited works appear at the end of the
%author's entry; for conventions
%see the List of Citations on page~\pageref{loc}.  
%\smallbreak\noindent
%The letter `f' after a page number indicates a figure, `n' a footnote.
}
\printindex[gen]
\endgroup % end of \catcode
%requires makeindex
\end{document}
\else
\fi


