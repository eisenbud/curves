%header and footer for separate chapter files

\ifx\whole\undefined
\documentclass[12pt, leqno]{book}
\usepackage{graphicx}
\input style-for-curves.sty
\usepackage{hyperref}
\usepackage{showkeys} %This shows the labels.
%\usepackage{SLAG,msribib,local}
%\usepackage{amsmath,amscd,amsthm,amssymb,amsxtra,latexsym,epsfig,epic,graphics}
%\usepackage[matrix,arrow,curve]{xy}
%\usepackage{graphicx}
%\usepackage{diagrams}
%
%%\usepackage{amsrefs}
%%%%%%%%%%%%%%%%%%%%%%%%%%%%%%%%%%%%%%%%%%
%%\textwidth16cm
%%\textheight20cm
%%\topmargin-2cm
%\oddsidemargin.8cm
%\evensidemargin1cm
%
%%%%%%Definitions
%\input preamble.tex
%\input style-for-curves.sty
%\def\TU{{\bf U}}
%\def\AA{{\mathbb A}}
%\def\BB{{\mathbb B}}
%\def\CC{{\mathbb C}}
%\def\QQ{{\mathbb Q}}
%\def\RR{{\mathbb R}}
%\def\facet{{\bf facet}}
%\def\image{{\rm image}}
%\def\cE{{\cal E}}
%\def\cF{{\cal F}}
%\def\cG{{\cal G}}
%\def\cH{{\cal H}}
%\def\cHom{{{\cal H}om}}
%\def\h{{\rm h}}
% \def\bs{{Boij-S\"oderberg{} }}
%
%\makeatletter
%\def\Ddots{\mathinner{\mkern1mu\raise\p@
%\vbox{\kern7\p@\hbox{.}}\mkern2mu
%\raise4\p@\hbox{.}\mkern2mu\raise7\p@\hbox{.}\mkern1mu}}
%\makeatother

%%
%\pagestyle{myheadings}

%\input style-for-curves.tex
%\documentclass{cambridge7A}
%\usepackage{hatcher_revised} 
%\usepackage{3264}
   
\errorcontextlines=1000
%\usepackage{makeidx}
\let\see\relax
\usepackage{makeidx}
\makeindex
% \index{word} in the doc; \index{variety!algebraic} gives variety, algebraic
% PUT a % after each \index{***}

\overfullrule=5pt
\catcode`\@\active
\def@{\mskip1.5mu} %produce a small space in math with an @

\title{Personalities of Curves}
\author{\copyright David Eisenbud and Joe Harris}
%%\includeonly{%
%0-intro,01-ChowRingDogma,02-FirstExamples,03-Grassmannians,04-GeneralGrassmannians
%,05-VectorBundlesAndChernClasses,06-LinesOnHypersurfaces,07-SingularElementsOfLinearSeries,
%08-ParameterSpaces,
%bib
%}

\date{\today}
%%\date{}
%\title{Curves}
%%{\normalsize ***Preliminary Version***}} 
%\author{David Eisenbud and Joe Harris }
%
%\begin{document}

\begin{document}
\maketitle

\pagenumbering{roman}
\setcounter{page}{5}
%\begin{5}
%\end{5}
\pagenumbering{arabic}
\tableofcontents
\fi


\chapter{Linkage of curves in $\PP^3$}
\label{DualityChapter}\label{LinkageChapter}
\fix{DualityChapter should refer to a different chapter!}

\def\length{{\rm length}}
\def\red{{\rm red}}
\section{Introduction} 
In this Chapter we will study  invariants associated to a free resolution, or syzygies, of the homogeneous coordinate ring of a curve in projective space, with an emphasis on their relation to the complete intersections containing the curve---this is the theory of \emph{linkage}. The theory is most powerful in the case of curves in $\PP^3$, so we will concentrate on this case. Throughout this Chapter, the word \emph{curve} will refer to a purely 1-dimensional projective scheme.

Recall that two curves in $\PP^3$ without common components are \emph{directly linked} if their union is a complete intersection. In this section we will study the generalization of this notion, and the equivalence relation it generates, to the case of arbitrary purely 1-dimensional subschemes of $\PP^3$. A simple example is the linkage of a twisted cubic and one of its secant lines, which together form the complete intersection of two quadrics.

We have already used the relation of \emph{ linkage} in Chapter~\ref{HilbertSchemesChapter} \fix{ref?} in a special case case of smooth curves without common components. In this setting it is obvious that the relation is symmetric, and that the degrees of the two curves add up to the degree of the complete intersection. We showed in \ref{***} that the genera of the two curves is related by the formula~\ref{***}. 

Linkage was first studied extensively by Halphen~\cite{***} and taken up in the 1940's by Ap\'ery and Gaeta~\cite{****}. The subject was modernized and generalized by Peskine and Szpiro in~\cite{PeskineSzpiro}. Hartshorne~\cite{****} and his student
Rao~\cite{****} made decisive breakthroughs, showing that a simple invariant classifies curves up to linkage; and Lazarsfeld and Rao~\cite{****} explained how to describe a given linkage equivalence class. A thorough exposition of the subject in the general case can be found in the book of Migliore~\cite{Mig}. Note that the linkage relation is often called by its French name, \emph{liaison}.

Notation:  We write $S := k[x_0,\dots,x_3]$ for the homogeneous coordinate ring of $\PP^3$ and $\gm = (x_0,\dots,x_3)$ for its
irrelevant ideal. If $X\subset \PP^3$ is a subscheme we write $I_X$ for the homogeneous ideal of $X$ and $S_X:=S/I_X$ for the homogeneous coordinate ring of $X$. \fix{except for the restriction to 3 dimensions, this notation should be standard in the book already...} In this section, the word \emph{curve} will mean a closed subscheme of pure dimension 1 in a projective space.

\section{General definition and basic results}\label{linkage definition}

Here is the definition of (complete intersection) linkage:
\begin{definition}
If $X$ and $Y$ are curves  of codimension $c$ in a complete intersection scheme $P$ then $X$ and $Y$ are \emph{directly linked} if there exists a codiimension $c$ complete intersection $Z \subset P$ containing $X\bigcup Y$ such that $I_{X} = I_{Z}:I_{Y}$. In this case we say that $X$ is directly linked to $Y$ by $Z$. 

More generally, we say that $X$ and $Y$ are \emph{evenly} (respectively oddly) linked if they are
connected by an even (respectively odd) number of direct linkages.\end{definition}

The relation of direct linkage is symmetric in $X$ and $Y$, and satisfies the same formulas for degree and genus as in the special case we treated in Theorem~\ref{****}:

\begin{theorem}\label{justification of general linkage} The relation of direct linkage is symmetric. Moreover, if $X,Y\subset \PP^{3}$ are purely 1-dimensional subschemes and $X$ is linked to $Y$ by the complete intersection $Z$ of surfaces of degrees $d_{1}, d_{2}$, then
\begin{enumerate}
\item $Y$ is linked to $X$ by $Z$; that is, linkage is symmetric.
 \item $\deg X+\deg Y = \deg Z = d_{1}d_{2}$.
 \item  The arithmetic genera of $X$ and $Y$ are related by
$$
p_{a}(Y) - p_{a}(X) =\frac{(d_{1}+d_{2}-4)}{2} (\deg Y - \deg X)
$$
\end{enumerate}
 \end{theorem}
 
 The proofs involve several important results from commutative algebra:
 
\begin{theorem} \label{double colon}
\begin{enumerate}
\item For any ideals $G,I$ in a commutative Noetherian ring $R$, the associated primes of  $J = G:I$ are  are among the associated primes of $G$. Moreover, if $G$ is unmixed (that is, all primary components have the same dimension) then  
the associated primes of $J$ are precisely the associated primes of $G$ whose primary components do not contain $I$.
Moreover, the associated primes of $G:(G:I)$ are the primary components of $G$ whose associated primes contain $I$.

\item (\emph{symmetry}) If $R$ is Gorenstein, $G$ is a complete intersection in $R$, and $I\subset R$ is an ideal containing $G$, then $G:(G:I)$ is the 
intersection of the primary components of $I$ that have the same codimension as $I$. 

\item Under these hypotheses, the sum of the multiplicities of $R/I$ and $R/(G:I)$ is the multiplicity of $R/G$.

\item 
$
\omega_{R/I} = \Hom(R/I, \omega_{R/G})\cong (G:I)/G.
$\\
\end{enumerate}
\end{theorem}

In case $R = k[x_{0},\dots,x_{n}]$ and both $I$ and $G$ are graded, with $G = (f_{1}, \dots f_{c})$ the intersection of forms of degree
$d_{1}+\cdots + d_{c}$, then $\omega_{R/G} = R/G(\sum d_{i} -n-1)$ as graded modules by Example~\ref{canonical of CI} so, \emph{as graded modules},
$$
\omega_{R/I} = \Hom(R/I, \omega_{R/G})\cong (G:I)/G(\sum d_{i} -n-1).
$$


\begin{proof}[Proof of Theorem~\ref{double colon}]
 1) If $G = \bigcap Q_{i}$ is an irredundant primary decomposition of $G$ then
$G:I = \bigcap_i (Q_{i}:I)$. If $I \subseteq Q_i$, then $Q_{i}:I = R$, and this term can be omitted. If $P_i$ is the associated prime of $Q_i$
and $I \not\subseteq Q_i$ then $ Q_i \subset Q_i:I\subset P_i$ (since $P_i$ is the set of zerdivisors mod $Q_i$), 
so $\sqrt {Q_i:I} = P_i$. Furthermore, if $xy\in Q_i:I$ and $x\notin P_i$, then $x$ is a nonzerodivisor mod $Q_i$, so from
$xyI\subset Q_i$ we deduce $yI\subset Q_i$; that is, $y\in Q_i:I$. This shows that $Q_i:I$ is $P_i$-primary. 
Finally, if $I\not\subseteq P_i$, then $I$ contains a nonzerodivisor mod $Q_i$, so $Q_i:I = Q_i$.

This proves that $G:I$ has a primary decomposition whose terms are primary to the associated primes of the primary
components of $G$ that do not contain $I$, so the associated primes of $G:I$ are among these. If some $(Q_i:I)$ were contained
in the intersection of the others, then we would have $P_i^n \subset \cap_{j\neq i}P_j$, and $P_i$ would be contained in one
of the $P_j$ with $j\neq i$, and this is impossible if all the $P_i$ have the same codimension.
 
 2) Since $G$ is a complete intersection, it is unmixed, and it follows from part 1 that $G:I$ and $G:(G:I)$ are unmixed too. Further,
 the primary components of $G:I$ have the form $Q_i:I$, where the $Q_i$ are the primary components of $G$ that
 do not contain $I$. Now $Q_i$ contains $Q_i:I$ if and only if $Q_i=Q_i:I$, and this happens if and only if
  $I\not\subseteq P_i$,  the associated prime of $Q_i$. Since $G$ is unmixed and $G\subseteq I$, this proves that the associated primes of 
  $G:(G:I)$ are exactly the associated primes of $G$ that are also associated primes of $I$.
  
 Now suppose that if $P$ is a minimal prime of $G$ and $Q \subset I \subset P$, where $Q$ is the $P$-primary
  component of $G$. Since $P$ is minimal over $I$, it is an associated prime. Write $Q'$ for the $P$-primary component
  of $I$. By the argument above, the $P$-primary component of $G:(G:I)$ is $Q:(Q:Q')$, and we must show that this is the
  same as $Q'$.
  
  Since both $Q'$ and $Q:(Q:Q')$ are $P$-primary, it suffices to prove this after localizing at $P$, so we may assume
  that $R$ is a local ring, with $\dim R/Q = \dim R/Q' = 0$. Since $R/Q$ is a localization of $R/G$ it is again a complete
  intersection, and thus Gorenstein (more generally, it is true that the localization of any Gorenstein ring is Gorenstein, but
  we do not need this.)
  Furthermore, $Q:Q' = \Hom(R/Q', R/Q)$, and similarly $Q:(Q:Q') = \Hom(\Hom(R/Q', R/Q), R/Q) = R/Q'$ by duality.
  
3) Since 
\begin{align*}
 \length (R/G) &= \length(R/(G:I))+\length((G:I)/G)\\
 &= \length(R/(G:I) +\length R/I,
\end{align*}
we see from the associativity formula for multiplicity that when $I$ has the same codimension as $G$, then the multiplicity of $I$ plus that of $G:I$ is the multiplicity of $G$. In the graded case, this means that
$\deg \Proj(R/I) +\deg \Proj(R/(G:I)) = \deg \Proj(R/G)$.

4)See Proposition~\ref{computation of omega}.
\end{proof}

 \begin{proof}[Proof of Theorem~\ref{justification of general linkage}]
 By hypothesis, 
$X,Y\subset Z$ are unmixed of dim 1, with $\cI_{Y} = (\cI_Z:\cI_X)$. By Part 1 of Lemma~\ref{double colon}, we have
$\cI_{X} = (\cI_Z:\cI_Y)$ as well, proving symmetry.

1) This is  the formula of part 3 of Lemma~\ref{double colon}, interpreted in the graded case.

2)  By Proposition~\ref{computation of omega}
\begin{align*}
 \omega_{Y} &= \cHom_{\cO_Z}(\cO_X, \omega_Z) = 
\cHom_{\cO_Z}(\cO_X, \cO_Z(d_1+d_2-4))\\
&=  \frac{(\cI_Z:\cI_X)}{\cI_{Z}}(d_1+d_2-4).
\end{align*}
Thus there is an exact sequence
$$
0\to \omega_{Y}(-d_1-d_2+4) \to \cO_Z \to \cO_X \to 0,
$$
whence
$$
 \chi(\cO_Z) =
 \chi(\omega_{Y}(-d_1-d_2+4) +
 \chi(\cO_X).
 $$
 Applying the adjunction formula twice, and using the Riemann-Roch Theorem, together with the formula
 $\deg Z = \deg X + \deg Y$, we see that the arithmetic genus of $Z$ is
 $$
 \chi(\sO_Z) = -\frac{(\deg X + \deg Y)(d_1+d_2-4)}{2}.
 $$
Furthermore, by the Riemann-Roch Theorem~\ref{****}, $\chi(\sO_X) = 1-p_a(X)$ while
\begin{align*}
\chi(\omega_{Y}(-d_1-d_2+4)) &= 2p_a(Y) -2 + \deg(Y)(-d_1-d_2+4) + 1- p_a(Y)\\
&= p_a(Y)-1- \deg(Y)(d_1+d_2+4).
\end{align*}
Putting this together we get
$$
-\frac{(\deg X+\deg Y)(d_1+d_2-4)}{2} = 1-p_a(X) + p_a(Y)-1 - \deg(Y)(d_1+d_2+4)
$$
and thus 
$$
p_a(Y) -p_a(X) = \frac{(\deg Y-\deg X)(d_1+d_2-4)}{2}
$$
as claimed.
 \end{proof}

\section{The Hartshorne-Rao module}

The main theorem on linkage of curves in $\PP^3$ is due to Hartshorne~\cite{****} and Rao ~\cite{****}. If $X$ is a purely 1-dimensional projective scheme, then $S_X$ is locally Cohen-Macaulay , and thus $H^1(\sI_X(i)$ is nonzero for only finitely many values of $i\in \ZZ$, so
the vector space
$$
M(X) := H^1_*(\sI_X) := \oplus_{d\in \ZZ}H^1(\sI_X(d)),
$$
which is a graded module over the homogeneous coordinate ring of $\PP^n$, has finite length (equivalently, finite dimension as a 
vector space over the ground field.) \fix{put this into the local coho section?}
There are two other ways to look at $M(X)$ that are sometimes useful:
$$
M(X) = H^1_\gm(S_X) = Ext^3(S_X,S(-4))^\vee.
$$
The first of these equalities follows immediately from the exact sequence
$$
0\to I_X\to S \to \bigoplus_{d\in \ZZ} H^0(\sO_X(d) \to H^1_\gm(S_X) \to 0
$$
and the corresponding sequence in which $H^1_\gm(S_X)$ is replaced by $\oplus_{d\in \ZZ}H^1(\sI_X(d))$.
while the second is a special case of local duality for sheaves on $\PP^3$; see Section~\ref{local coho section}

\begin{theorem}[Hartshorne-Rao\cite{****}]\label{Hartshorne-Rao}
Write $S$ for the homogeneous coordinate ring
of $\PP^3$, and suppose that $X,Y\subset \PP^3$ are subschemes of pure dimension 1. If $X,Y$ are directly linked by a complete
intersection of surfaces of degree $d_1,d_2$ then, as graded $S$-modules,
$$
M(Y)\cong M(X)^\vee(-d_1-d_2+4).
$$
Moreover, $X,Y$ are evenly linked if and only if 
$$
M(Y) \cong M(X)(t)
$$
 for some integer $t$. 
 
 Every graded module of finite length is isomorphic, up to a shift in grading,
to the module $H^1_*(\sI_X)$ for some smooth curve $X$.
\end{theorem}
Note that the module $M(X)$ appears in the exact sequence in cohomology coming from the surjection $\sO_{\PP^3} \to \sO_X$:
$$
0\to H^0_*(\sI_X) \to H^0_*(\sO_{\PP^3}) \to H^0_*(\sO_X) \to H^1_*(\sI_X) \to 0
$$
Thus $M(X) = H^1_*(\sI_X) = 0$ if and only if the linear series cut by hyperplanes of degree $d$ is complete for all $d$, that is,
$X$ is arithmetically Cohen-Macaulay.

We will prove Hartshorne's half of the Hartshorne-Rao Theorem:

\begin{theorem} [Hartshorne \cite{****}]\label{Hartshorne}
If $X,Y\subset \PP^3$ are purely 1-dimensional subschemes that are directly linked 
through the complete intersection $Z$, which is given by forms $f_1, f_2$ of degrees $d_1,d_2$ respectively
then
$$
M(X)^\vee \cong M(Y)(d_1+d_2-4)
$$
where $M(X)^\vee$ denotes the vector space dual of $M(X)$ equipped with the natural module structure.
 \end{theorem}

\begin{proof}
Let $S$ be the homogeneous coordinate ring of $\PP^3$. The ring $S/I_X$ may not be Cohen-Macaulay,
but because it is purely 1-dimensional,  $X = \Proj S/I_X$ is locally Cohen-Macaulay of codimension 2, and thus its
second syzygy sheafifies to a vector bundle $\sE$ on $\PP^3$, and we see that 
$$
M(X) = \oplus_{d \in \ZZ} H^1(\sI_X(d)) \cong \oplus_{d \in \ZZ} H^2(\sE(d)).
$$

The natural surjection $S/(f_1,f_2) \to S/I_X$ lifts to a map of free resolutions, and sheafifying we get a diagram with exact rows
and right-hand column:
{\scriptsize
$$ 
\begin{diagram}[]
0&\rTo &\sE &\rTo &F_1& \rTo& \sO_{\PP^3}&\rTo &\sO_X &\rTo &0\\
&&\uTo&&\uTo&&\uTo^1&&\uTo\\
0&\rTo&\sO_{\PP^3}(-d_1-d_2) &\rTo &\sO_{\PP^3}(-d_1)\oplus \sO_{\PP^3}(-d_2)& \rTo& \sO_{\PP^3}&\rTo &\sO_Z &\rTo &0\\
&&&&&&&&\uTo\\
&&&&&&&&\sI_X/(f_1,f_2)\\
&&&&&&&& \uTo\\
&&&&&&&& 0.
\end{diagram}
$$
}
By Proposition~\ref{computation of omega},  $I_X/(f_1,f_2) \cong \omega_{S/I_Y}(-d_1-d_2+4)$, so the mapping cone of the map of complexes above has first homology
$\omega_Y(-d_1-d_2-4)$. Dropping the two copies of $\sO_{\PP^3}$ and the identity map between them, we get a locally free resolution
$$
\begin{diagram}[small]
0&\rTo&\sO_{\PP^3}(-d_1-d_2)&\rTo&\sE\oplus \sO_{\PP^3}(-d_1)\oplus \sO_{\PP^3}(-d_2)&\rTo&F_1
\end{diagram}
$$
of $\omega_Y(-d_1-d_2+4)$.

Let $R := \oplus_{d\in \ZZ} H^0(\sO_Y(d))$.
The natural map $S/I_Y \to R$ has cokernel $\oplus_{d\in \ZZ}H^1\sI_Y(d)= M(Y)$, which has finite length. Thus
$$
\omega_{S/I_Y} = \Ext_S^2(S/I_Y, \omega_S) = \Ext_S^2(R, \omega_S).
$$
Moreover $R$ is Cohen-Macaulay, so also $\Ext_S^2(\omega_{S/I_Y}, \omega_S) = R$. Dualizing the resolution above 
 and sheafifying, we get an exact sequence of sheaves
$$
0\to F_1^* \to \sE^*\oplus \sO_{\PP^3}(d_1)\oplus \sO_{\PP^3}(d_2) \to \sO_{\PP^3}(d_1+d_2) \to R(d_1+d_2) \to 0.
$$
It follows that the image of $\sE^*\oplus \sO_{\PP^3}(d_1)\oplus \sO_{\PP^3}(d_2) $ in $\sO_{\PP^3}(d_1+d_2)$
is $\sI_Y(d_1+d_2)$. By Serre duality, 
$M(X) = \oplus_{d\in \ZZ} H^2(\sE(d)) = \bigl(\oplus_{d\in \ZZ} H^1(\sE^*(-d-4))\bigr)^\vee$, the dual over the ground field. From the above sequence we see that 
$$
M(X)^\vee \cong \oplus_{d\in \ZZ} H^1(\sE^*(-d-4))\oplus_{d\in \ZZ}  = M(Y).
$$ 
Thus $M(X)^\vee = M(Y)(d_1+d_2-4)$ as required.
 \end{proof}

There are sometimes large families of curves having a given Hartshorne-Rao module, sometimes very few. Here are three simple examples:

\begin{example}[Two lines]\label{2 lines}
 Let $X\subset \PP^3$ be the union of two disjoint lines, $L_1,L_2$. Supposing that the lines are given by equations $x_0=x_1=0$ and $x_2=x_3=0$ respectively. Since $H^0_*(\sO_X) \cong k[x_2,x_3] \times k[x_0,x_1]$ the exact sequence
 $$
 (*)  0\rTo I_X \rTo S \rTo^{\hbox{restriction}} H^0_*(\sO_X) \rTo M(X) \rTo  0,
 $$
where the map labeled restriction sends each variable to the variable with the same name shows that $M(X) = M(X)_0 =k$.
We can see directly that $X$ is linked in two steps to any other union of 2 disjoint lines $L_1',L_2'$. Indeed there are two lines $K_1,K_2$ (or possibly $K = K_1 = K_2$ as a double line) meeting each of the 4 lines $L_1,L_2,L_1', L_2'$ in two points (or possibly 1 with multiplicity 2). (Proof: any three disjoint lines lie on a  unique quadric, which must be smooth since the lines are disjoint, and the lines lie in the same ruling. The 4th line pierces that quadric in 2 points or is tangent to it; the two lines from the opposite ruling through those two points (or the double line in the case of tangency) meet all 4 lines.) The unions $Z = L_1\cup L_2\cup K_1\cup K_2$ and 
$Z' = L_1'\cup L_2'\cup K_1\cup K_{2}$ are each the complete intersection of 2 quadrics, each of which may be taken to be the union of two planes; for example 
$$
Z = \bigl(\overline{L_1,K_1} \cup \overline{L_2, K_2}\bigr)\bigcap\bigl( \overline{L_1,K_2} \cup \overline{L_2, K_1}\bigr)
$$
\fix{add a picture!}
It is not hard to show that any curve of type $(a,a+2)$ on a smooth quadric is also linked to $X$.\end{example}

\begin{example}[Three lines]\label{3 lines} Let $X\subset \PP^{3}$ be the union of 3 disjoint lines.
Since a line imposes 3 conditions on a quadric to contain it, and since there is a 10-dimensional vector space of quadratic forms in 4 variables, $X$ is contained in at least 1 quadric $Q$. Since no two of the lines can lie on a plane, $Q$ is irreducible; and since any two lines on an irreducible singular quadric in $\PP^{3}$ meet, $X$ must be smooth. Recall that $Q$ has two linear equivalence classes of lines, and lines from one class all meet the lines from the other class; thus the three lines are all linearly equivalent on $Q$.

\begin{proposition}[Migliore]
If $X'\subset \PP^{3}$ is another union of 3 disjoint lines then $X$ is linked to $X'$ if and only if $X'\subset Q$ as well. Moreover, $X$ is directly linked if $X'$ is in the opposite linear equivalence class, and evenly linked in two steps if $X'$ is in the same equivalence class.
\end{proposition}

For more results in this direction, see~\cite{Migliore paper****} from which the argument below is taken.

\begin{proof}
First, if $X'\subset Q$ is in the opposite equivalence class as $X$, then $X+X' \sim 3H$ as divisors on $Q$, where $H$ is the hyperplane section. Thus $X+X' = X\cup X'$ is the complete intersection of $Q$ with a cubic surface, proving that $X$ and $X'$ are directly linked.

On the other hand, if $X'\subset Q$ is in the same equivalence class as $X$, then the union $Y$ of three lines in the opposite equivalence class is linked to both $X$ and $X'$.

From the exact sequences analogous to (*) in Example~\ref{2 lines} we see that 
$M(X)_{0} \cong k^{2}\cong M(X)_{1}$, and $M(X)_{d} = 0$ for $d\neq 0,1$. Each linear form $\ell$ on $\PP^{3}$ induces a map $m_{\ell}: M(X)_{0}\to M(X)_{1}$ by multiplication. Let
$$
Q' (X):= \{\ell \mid m_{\ell}\hbox{ has nonzero kernel on }M(X)\} \subset \PP^{3*};
$$
where $\PP^{3*}$ is the projective space of linear forms on $\PP^{3}$, that is, the dual projective space to $\PP^{3}$. 

We next show that if $X'$ is linked to $X$ then $Q'(X) = Q'(X')$. By Hartshorne's theorem, if $X'$ is  linked to $X$ then
$M(X') \cong M(X)$ up to twist or $M(X') \cong M(X)^{\vee}$ up to twist. In the first case it is obvious
that  $Q'(X') = Q'(X)$ and this is also true in the second case, because the multiplication map 
$$
m_{\ell}: M(X')_{0}\cong M(X)_{1}^{\vee} \cong k^{2} \rTo M(X')_{1}\cong M(X)_{0}^{\vee}\cong k^{2}
$$
is simply $m_{\ell} ^{\vee}$.

It remains to show that $Q'(X)$ determines $Q$. We claim that $Q'$ is the set of linear forms vanishing on one of the lines of $Q$. Let $L_{\ell}\subset \PP^{3}$ be the hyperplane on which $\ell$ vanishies. Since a hyperplane meets $Q$ in a plane conic, $\ell\in Q'$ iff $L_{\ell}\cap Q$ is a divisor of type $L+L'\subset Q$, where 
$L,L'$ belong to opposite rulings. Thus $Q'$ is the set of linear forms whose hyperplanes meet $Q$ in singular curves, that is, the set of tangent hyperplanes, also known as the dual variety to $Q$. Since the dual of the dual is the original variety, the dual of $Q'$ is $Q$. 

Finally, we must show that 
$$
Q' = \{\ell \mid L_{\ell} \hbox{ contains a line of }Q\}.
$$
First, suppose that $L_{\ell}$ contains one of the components $L_{1}$ of $X = L_{1}\cup L_{2}\cup L_{3}$.
We may write 
$$
M(X)_{0} = ke_{1}\oplus ke_{2}\oplus ke_{3}/k(e_{1}+e_{2}+e_{3})
$$
where $e_{i}$ is a rational function that is nonzero on $L_{i}$ and zero on $L_{j}$ for $j\neq i$. It follows
that $m_{\ell}(e_{1})$ = 0, so $\ell\in Q'$.

Next suppose $L_{\ell}$ does not contain any of the $L_{i}$. For any linear form $\ell$ there is an exact sequence of sheaves
$$
0\rTo \sI_{X/\PP^{3}} \rTo^{\ell} \sI_{X/\PP^{3}}(1) \rTo \sI_{(X\cap L_{\ell'})/L_{\ell}}(1) \rTo 0.
$$
and thus, from the long exact sequence in cohomology,
$$
0\rTo H^{0}(\sI_{(X\cap L_{\ell})/L_{\ell}}(1)) \rTo H^{1}(\sI_{X/\PP^{3}} ) \rTo H^{1}(\sI_{X/\PP^{3}}(1)); 
$$
that is, 
$$
H^{0}(\sI_{(X\cap L_{\ell})/L_{\ell}}(1)) \cong \ker m_{\ell}: M(X)_{0} \to M(X)_{1}.
$$
However, $H^{0}(\sI_{(X\cap L_{\ell})/L_{\ell}}(1)) \neq 0$ if and only if there is a linear form
$\ell'$, not a multiple of $\ell$, that vanishes on $X\cap L_{\ell}$; that is, if the three points of $X\cap L_{\ell}$
are colinear. Since $Q$ is a quadric, this is the same as saying that $Q\cap L_{\ell}$ contains a line, completing the argument.
\end{proof}
\end{example}

For double lines not lying on an irreducibly quadric, see Example~\ref{double lines of higher genus}
\section{Construction of curves with given \\Hartshorne-Rao module}

Some of the main remaining results about linkage of curves in $\PP^3$ depend on careful general position arguments, and we merely
sketch them. In the following $S = k[x_0,x_1,x_2,x_3]$.

\begin{theorem}(Rao\cite{***}
Let $M$ be a graded $S$-module of finite length. There is a smooth curve in $\PP^3$ with Hartshorne-Rao module $M(t)$ for some twist $t\in \ZZ$
\end{theorem}
\begin{proof}[Sketch of Proof] 
Recall that the homogeneous coordinate
ring $S_X$ of $X$ has resolution of the form
$$
\GG: 0\rTo F_3\rTo^A F_2\rTo^\phi F_1\rTo S
$$
with $\rank F_1 = \rank \phi+1$
ignoring shifts of the grading, we have $\coker A^* = \Ext^3_S(S_X,S)  = M(X)^\vee$. The dual of $\GG$ is not exact, but maps to the 
resolution $\LL$ of $M(X)^\vee$. The dual of $\LL$ is a resolution (of $M(X)$), and has the form
$$
\LL^*: 0\rTo F_3 \rTo F_2\rTo^\psi L_2\rTo L_1\rTo L_0.
$$
It turns out that if we take a sufficiently general projection $p: \L_2\to L_2'$, with $\rank L_2' = \rank \psi+1$, then
the cokernel of $\psi' = p\circ \psi$ is torsion free of rank 1. Thus this cokernel is equal to an ideal $I$, up to some twist, 
and we get a resolution of $S/I$ of the the form
$$
0\rTo F_3(t) \rTo F_2(t) \rTo^\psi L_2'(t)\rTo S
$$
proving that $M(S/I) = M$. Possibly after twisting further, an application of Bertini's theorem shows that $S/I$ will be the coordinate ring of a smooth curve.
\end{proof}

It is nevertheless the case that \emph{not} every twist of every module occurs as the Rao module of a curve, even when we allow the curve to be an arbitrary purely 1-dimension subscheme; see Corolllary~\ref{twist by 1} below.

\section{Curves on a surface}
For curves on a surface,
the relation of even linkage reduces to
that of linear equivalence:

\begin{proposition}
Let $S$ be a surface in $\PP^3$, and let $X,Y\subset S$ be purely 1-dimensional schemes. The schemes $X$ and $Y$ are directly linked on $S$ if and only if
there is a rational function $f$ on $S$ such that the divisor $f$ is $X-Y$. Thus $X$ and $Y$ are evenly linked if and only if 
$X~Y$, and they are oddly linked if and only if $-X\sim Y$.
\end{proposition}

\begin{proof} Suppose first that $X,Y$ are directly linked. After passing to an affine open set we have
$(a):I_X = I_Y$ for some nonzerodivisor $a$. Write $K(S)$ for the sheaf of rational functions on $S$, so that
when restricted to an affine open set $U$ we have 
$$
K(S)|_U = \{a/b\mid a,b \in \sO_S(U), b \hbox{a nonzerdivisor}\}.
$$  
In this context the divisor $-X$ is the divisor associated to the
fractional ideal $I^{-1} := \{q\in K(S) \mid qI \subset \sO_S(U)\}$.
Write $Z$ for the divisor of $a$. Since $I_X$ contains a nonzerodivisor, $a:I_X = aI_X^{-1}$; that is,
$-X+Z = Y$, so indeed $-X$ is linearly equivalent to $Y$. Iterating this argument we see that if $X,Y$ are evenly linked then
they are linearly equivalent, and if oddly linked then $-X$ is linearly equivalent to $Y$, as claimed.

Conversely, suppose that $X \sim Y$. Passing to an open affine subset, this means that there are regular functions $g,h$
such that $gI_X = hI_Y$. Since $I_X$ and $I_Y$ contain nonzerodivisors on $S$, we may multiply and assume $h\in I_Y$.
We know from **** that $(h:(h:I_Y)) = I_Y$, and it follows that 
$$
I_X = (g/h)(h:(h:I_Y))= g:(h:I_Y)
$$
as required.
\end{proof}

\section{Liaison Addition and Basic Double Links}

Phillip Schwartau discovered a simple way to construct a curve $Z$ whose Rao invariant $M(Z)$ is the direct sum of Rao 
invariants $M(X), M(Y)$ for given curves $X,Y$:

\begin{proposition}[Liaison Addition]\ref{Schwartau}\label{Schwartau}
 Let $X,Y$ be purely 1-dimensional subschemes of $\PP^3$, and let $f\in I_Y,\ g\in I_X$ be forms such that $f,g$ is a regular
 sequence. The ideal $fI_X+gI_Y$ is unmixed of codimension 2 and the scheme $Z$ it defines has Rao invariant
 $$
 M(Z) \cong M(X)(-\deg f)\oplus M(Y)(-\deg g).
 $$
\end{proposition}

\begin{proof}
We write $S = k[x_0,\dots,x_3]$ for the homogeneous coordinate ring of $\PP^3$, with maximal homogeneous ideal $\gm$ and set $J = fI_X\oplus gI_Y$.
 Since 
 $$
 (fg) \subset fI_X\cap gI_Y \subset (f)\cap (g) = (fg)
 $$
 we have in fact $(fg) = fI_X\cap gI_Y $ and thus an exact sequence
 $$
 0\to S/(fg) \rTo S/fI_X\oplus S/gI_Y \rTo S/J \rTo 0,
 $$
 from which we see that $J$ has codimension 2. 
 If $J\subset P\subset S$ is were an associated prime of $J$ having codimension 3 in $S$, then localizing at $P$
 we would find $\depth(S/(fg))\leq 1$; contradicting the fact that $(S/fg)_P$ is Cohen-Macaulay of dimension 2.
 Thus $J$ is unmixed. Further, since $S/fg$ is Cohen-Macaulay of dimension 3,
 we have $H^1_\gm(S/fg) = H^2_\gm(S/fg) = 0$ and thus
\begin{align*}
  M(Z) = H^1_\gm(S/J)= H^1_\gm(S/fI)  &\oplus H^1_\gm(S/gJ)\\
  = H^1_\gm(S/I)(-\deg f) &\oplus H^1_\gm(S/J)(-\deg g)\\
  = M(X)(-\deg f) &\oplus M(Y)(-\deg g)
\end{align*}
\end{proof}

In the case $Y = \emptyset, I_Y = S, f = 1$, the Hartshorne-Rao Theorem implies that $M(Z) = M(X)(-\deg g)$.
In particular, every negative twist of a module that is the Hartshorne-Rao invariant of a curve is again the
Hartshorne-Rao invariant of a curve.

This case was exploited by
Lazarsfeld and Rao under the name \emph{Basic double link}, and under a mild additional hypothesis the linking sequence can be made explicit:

\begin{proposition}[Basic Double Links]\label{basic link}
Let $X$ be a purely 1-dimensional subschemes of $\PP^3$, and let $f,g$ be a regular sequence of forms, with $g \in I_X$. The ideal $fI_X+gS$ is unmixed of codimension 2 and in the even linkage class of $I$. Moreover, if $I_X+fS$ has codimension 3, then the scheme $Z$ it defines is linked in two steps to $I$: for any $h\in I_X$ such that $g,h$ is a regular sequence, 
$$
fI_X + gS = (g,fh):\bigl((g,h):I_X\bigr).
$$
\end{proposition}

\begin{proof} By Theorem~\ref{Schwartau} the  ideal $J := fI_X+gS$ is unmixed and has the same Hartshorne-Rao invariant as $I_X$,
so by Theorem~\ref{Hartshorne-Rao} it is evenly linked to $I_X$.

if $r(I_X) \subset (g,fh)$ so that $r\in (g,h):I_X$, then $rJ\subset (g,fh)$, so $J \subset (g,fh):\bigl((g,h):I_X\bigr)$.
Thus to prove the equality in the case when $I_X+fS$ has codimension 3, it suffices to do so after localizing at each of the associated primes of $J$. 
By Proposition ~\ref{Schwartau}, $J: = fI_X+gS$ is unmixed of codimension 2, so it suffices to prove the equality after
localizing at a codimension 2 prime $P$. By our hypothesis, either $f\notin P$ or $I_X\not\subset P$

If $f\notin P$ then $J_P = (I_X)_P$ and 
$$
\biggl((g,fh):\bigl((g,h):I_X\bigr)\biggr)_P = \biggl((g,h):\bigl((g,h):I_X\bigr)\biggr)_P = (I_X)_P
$$
by the assumption that $g,h$ is a regular sequence and the symmetry of linkage, Theorem~\ref{justification of general linkage}.

On the other hand, if $I_X \not\subset P$ then after localizing the equality becomes
$$
(g,f) = (g,fh):(g,h)
$$
which holds because $g,h$ is a regular sequence. 
\end{proof}

\begin{corollary}\label{twist by 1} If $M = M(X) \neq 0$ for some purely 1-dimensional scheme $X\subset \PP^{3}$, then $M_{d} \neq 0$ for some $d\geq -1$. Thus
for any nonzero graded $S$-module $M$ of finite length,
there is a maximal integer $d$ such that $M(d)$ occurs as a Rao module. Moreover, if $M$ occurs as a
Rao module, then for all $e\leq d$ the module
$M(e)$ also occurs.
\end{corollary}

\begin{proof} If $M = M(X)$ then for $d \leq -1$ we have $M_{d} = H^{0}(\sO_{X}(d)$.  Since 
$\bigoplus_{d \in \ZZ}H^{0}(\sO_{X}(d)$ is an $S$-module of depth $\geq 1$, we have
$M_{d}\geq M_{d-1}$ for all $d\leq -1$. In particular, if $M\neq 0$ and $M_{d}\neq 0$ for some
$d\leq -2$, the $M_{-1} \neq 0$, and the conclusion follows.
 If $Y$ is obtained from $X$ by a basic double link with $\deg g = 1$, then $M(Y) = M(X)(-1).$
\end{proof}

A much sharper result is given in \cite{M-D-P}: $M_{n} = 0$ unless 
$$
 g+1-((d-2)(d-3)/2) \leq n \leq (d(d-3)/2)-g.
 $$

%\bibitem[M-D-P]{M-D-P} M. Martin-Deschamps and D.~ Perrin, Sur les bornes du module de Rao.
%C.R. Acad. Sci. Paris t 317 (1993)1159--1162.

The main result of Lazarsfeld and Rao gives a description of a given linkage class:

\begin{theorem}[Structure of a linkage class](\cite{Lazarsfeld-Rao})\label{description of a linkage class}
Let $M= M(X)$ be the Rao module of a purely 1-dimensional subscheme, and suppose that $M(1)$ does not occur as a Rao module. All the curves $Y$ that are evenly linked to $X$ are obtained from $X$ by a series of basic double links followed by a deformation.
\end{theorem}

\section{Arithmetically Cohen-Macaulay Curves}

Before discussing the proof of Theorem~\ref{Hartshorne-Rao}, we examine the case $M(X) = 0$, which was first elucidated by Gaeta. 

\begin{theorem}[Gaeta \cite{****}]\label{Gaeta}
If $X$ is a curve in $\PP^3$  then $X$ is in the (even and odd) linkage class of a complete intersection if and only if the homogeneous
coordinate ring of $X$ is Cohen-Macaulay. Moreover, if $I_X$ can be generated by $n$ elements, then $X$ is linked to
a complete intersection in $n-2$ steps.
\end{theorem}

We first prove that even and odd linkage are the same in this case, and that any two complete intersection curves
are evenly linked:

\begin{lemma}
If $f,g$ and $f,h$ are regular sequences, then $(f,g)$ and $(f,h)$ are directly linked. Moreover, 
any  two complete intersections are both evenly and oddly linked.
\end{lemma}
\begin{proof}
Since $f,g$ and $f,h$ are regular sequences, so is $f,gh$. We claim that
$$
(f,h) = (f,gh):(f,g).
$$
Indeed if $ag = bf+cgh\in (f,gh)$ then $(a-ch)g = bf$ so $a-ch \in (f)$, whence $a\in (f,h)$.

It follows that if $m,n$ are independent linear forms, neither a divisor of $f$ or $g$, then each consecutive pair
in the sequence of complete intersections
$$
(f,g), (fm,g), (fm,gn), (m,gn),(m,n)
$$
are directly linked, so $(f,g)$ is evenly linked to $(m,n)$. The sequence 
$$
(f,g),(f,mg), (f,mng), (f,g)
$$
shows that $(f,g)$ is also oddly linked to itself, completing the proof.
\end{proof}
\fix{can these things be done with geometric links?}

Before proving Theorem~\ref{Gaeta} we need one more result from commutative algebra:

\begin{theorem}[Hilbert-Burch\cite{****}]\label{Hilbert-Burch}
Let $A$ be a homogeneous $n\times (n-1)$ matrix of forms in $S := k[x_{0},\dots, x_{r}]$ and
Let  $I: = I_{n-1}(A)$ be the ideal generated by the $n-1\times n-1$ minors of $A$.
\begin{enumerate}
 
\item If $I \neq S$ then $\codim I \leq 2$.
 \item If $I$ has codimension 2, then
$S/I$ is Cohen-Macaulay. Moreover, if $\Delta_{i}$ is the determinant of the matrix obtained
from $A$ by omitting the $i$-th column, then 
$$
0\rTo S^{n-1} \rTo^{A} S^{n}\rTo^{\scriptsize
\begin{pmatrix}
 \Delta_{1}& -\Delta_{1}&\dots&\pm \Delta_{n}
\end{pmatrix}}
S
$$
is a resolution of $S/I$, and its dual is a resolution of $\omega_{S/I}$.
\item Furthermore, every graded Cohen-Macaulay factor ring of $S$ of codimension 2
arises in this way.
\end{enumerate}
\end{theorem}

\begin{proof}
If we augment $A$ to an $n\times n$ matrix by repeating the $i$-th column, the determinant is zero.
The product of the row of signed minors 
$$\begin{pmatrix}
 \Delta_{1}& -\Delta_{2}&\dots&\pm \Delta_{n}
\end{pmatrix}
$$ 
with the $i$-th row of $A$ is the Cauchy expansion of this determinant. Thus the give sequence of maps
forms a complex. The fact that it is a resolution, and that its dual is a resolution, follows from a general result on finite free complexes,
\cite[Theorem *****]{E}. This shows that $S/I = S/(\Delta_{1}, \dots \Delta_{n})$ is Cohen-Macaulay.

Now suppose that $S/I$ is a homogeneous factor ring of $S$ that is Cohen-Macaulay and of codimension 2.
By Theorem~\ref{AB}, The minimal free resolution of $S/I$ as an $S$-module has the form 
$$
\FF: \quad 0\rTo S^{m} \rTo^{A} S^{n} \rTo^{B} S
$$
where $n$ is the minimal number of generators of $I$. Tensoring with the quotient field of $S$, we
get a complex of vector spaces that is exact, so $m = n-1$, and we see that $A$ is a homogeneous
$n\times (n-1)$ matrix. Again by \cite[Theorem *****]{E}, the ideal of $n-1\times n-1$ minors of $A$ has
codimension 2, and the dual of the resolution is a resolution of $\omega_{I}$. Write $\Delta$ for the
row of signed minors of $A$. Both $\Delta^*$ and $B^*$ can be regarded as the kernel of $A^*$,
so $\Delta = uB$ for some unit, and we are done.
\end{proof}
We remark that the complex $\FF$ in the proof of Theorem~\ref{Hilbert-Burch} is a special case of the Eagon-Northcott complex,
to be treated in the next chapter. The following Corollary is the corresponding special case of \cite[Theorem ***]{BE-annihilator}.

\begin{corollary}\label{annihilator codim 2}
Let $A$ be a homogeneous $n\times (n-1)$ matrix of forms in $S := k[x_{0},\dots, x_{n}]$ and
Let  $I: = I_{n-1}(A)$ be the ideal generated by the $n-1\times n-1$ minors of $A$.
If the codimension of $I$ is (at least) 2, then the annihilator of the cokernel of $A*: S^{n}\to S^{n-1}$ is exactly $I$.
\end{corollary}

\begin{proof}
The dual $\FF^*$ of the complex $\FF$ in the proof of Theorem~\ref{Hilbert-Burch} is a resolution of $\coker A^*$, and thus any
element of $s$ that anniliates the kernel induces a map of complexes that is homotopic to 0. Dualizing again, we see
that it induces the zero map on $S/I$---that is, it lies in $I$. The same argument applied to $F$ itself shows that any element of
$I$ annihilates $\coker A^*$.
\end{proof}

\begin{proof}[Proof of Gaeta's Theorem]
From Theorem~\ref{Hartshorne} if follows that if $X\subset \PP^3$ is in the linkage class of a complete intersection then $M(X) = 0$, so the homogeneous coordinate ring of $X$ is Cohen-Macaulay.

For the converse we prove a more general version: Suppose that $I\subset S= k[x_0,\dots,x_r]$ is a homogeneous ideal of codimension 2, generated by $n$ elements, such that $S/I$ is Cohen-Macaulay; we will show that $I$ can be linked in $n-2$ steps to a complete intersection. Let $A$ be the presentation matrix of $I$ so that, as in the Hilbert-Burch Theorem, $A$ has $n$ rows and $n-1$ columns, and $I$ is equal to the ideal of
$(n-1)\times (n-1)$ minors of $A$. 

Replacing the generators of $I$ by appropriate linear combinations, and making a corresponding change of generators of the 
module $S^n$ in the complex $\FF$ of Theorem~\ref{Hilbert-Burch}, we may assume that the first two generators, which are the 
$(n-1)\times(n-1)$ subdeterminants $\Delta_1, \Delta_2$ of $A$ omitting the first two rows, form a regular sequence. 

We now compute the linked ideal $(\Delta_1,\Delta_2): I$. Let $A'$ be $(n-2)\times (n-1)$ matrix obtained from $A$ by
deleting the first two rows. We may interpret the columns of $A'$, as generating the syzygies
of $I/(\Delta_1, \Delta_2)$. By Corollary~\ref{annihilator codim 2}, the ideal $I'$ generated by the  $(n-2)\times(n-2)$ of $A'$ is
the annihilator of the module $I/(\Delta_1, \Delta_2)$; that is, $I' = (\Delta_1,\Delta_2): I$ is directly linked to $I$. Moreover, $I'$ has codimension 2 because
the laplace expansions express the regular sequence $\Delta_1, \Delta_2$ in terms of these minors. By Theorem~\ref{Hilbert-Burch},
$S/I'$ is Cohen-Macaulay, and $I'$ has $n-1$ generators, so we are done by induction.
\end{proof}

\section{The structure of an even linkage class}
The structure present within a given (even) linkage class was illuminated in the work of Lazarsfeld and Rao\cite{Laz-Rao}, which proved a version of a conjecture of Harris; we close this chapter by sketching a result of their paper. But first, an elementary result:

\begin{proposition} 
Let $M$ be a graded $S$-module of finite length. The set of $t\in\ZZ$ such that there is a curve in $\PP^3$ with Rao module
$M(t)$ is bounded below. 
\end{proposition}

We say that a curve $X$ is \emph{minimal} in its even linkage class if, for all curves $Y$ in the even linkage class of $X$ we have
$M(Y) \cong M(X)(t)$ with $t\geq 0$.

\begin{proof} 
If suffices to show that if $M = M(C)$ for some curve $C\subset \PP^3$ then $\max\{n\mid M_n \neq 0\} \geq 0$. Let
$S_C$ be the homogeneous coordinate ring of $C$, and let $\tilde S_C = \oplus_n H^0(\sO_C(n))$. Since
$(S_C)_n = 0$ for $n<0$, we see that $M(C) = \tilde S_C/S_C$ agrees with $\tilde S_C$ in negative degrees. But
$\depth \tilde S_C\geq 2$, so $\dim (\tilde S_C)_n\geq \dim (\tilde S_C)_{n-1}$ for all $n<0$, and since $M_n=0$ for $n$ sufficiently
large, the conclusion follows.
\end{proof}

Much sharper bounds are known; see for example \cite{Perrin-Martin-Deschamps1990}.

\begin{theorem}[Structure of Linkage]\label{Lazarsfeld-Rao}
Any two minimal curves in an even linkage class are connected by a deformation.
Every curve in an even linkage class is obtained from a minimal curve by a sequence of basic double links, followed by a deformation.
\end{theorem}

\begin{proof}[Sketch of the proof]
 $$
0\to F_1^* \to \sE^*\oplus \sO_{\PP^3}(d_1)\oplus \sO_{\PP^3}(d_2) \to \sO_{\PP^3}(d_1+d_2) \to R(d_1+d_2) \to 0.
$$

\end{proof}

\begin{example} \label{double lines of higher genus}
The minimal elements in a linkage class may be unique and may not be reduced. From the formula
for the degree of a linked curve in Theorem~\ref{justification of general linkage} we see that
any curve of degree 2 must be minimal in its linkage class, and can only be linked to another 
minimal curve in its class by the complete intersection of two quadrics.

Consider the double line $C$ with ideal 
$$
I_{C}= (x_{0}^{2},\ x_0x_1,\ x_{1}^{2},\ x_{0}F_{0}(x_{2},x_{3})+x_{1}F_{1}(x_{2},x_{3}), )
$$
supported on the line $C_{\red}$ with ideal $(x_{0}, x_{1})$, where $F_{0}, F_{1}$ are relatively prime forms
of degree $d\geq 1$. It is not hard to show that this is a curve of degree 2 with arithmetic genus $-d$.

We first claim that the free resolution of the homogeneous coordinate ring $S_C$ over $S = k[x_0,\dots,x_3]$ is
\begin{align*}
 S&\lTo^{
\begin{pmatrix}
x_0^2&x_0x_1&x_1^2&x_0F+x_1G
\end{pmatrix}
}
S^3(-2)\oplus S(-d-1)
\\ \\
&
\lTo^{
\begin{pmatrix}
0&-F&x_1&0\\
F&-G&-x_0&x_1\\
G&0&0&-x_0\\
-x_1&x_0&0&0
\end{pmatrix}
}
S^2(-d-2)\oplus S^2(-3)
\lTo^{
\begin{pmatrix}
 x_0\\
 x_1\\
 F\\
 G
 \end{pmatrix}
}
S(-d-3)
\end{align*}
To prove that this is a resolution we use Theorem~\ref{WMACE}. It is easy to check that this sequence of maps forms a finite free complex.
 The only non-obvious fact needed to apply the Theorem 
is that the $3\times 3$ minors of the middle
matrix generate an ideal of codimension $\geq 2$, and in fact it clearly contains $(x_0^3, x_1^3)$.

From the resolution we
see that 
$$
M(C) = \Ext^3(S_C, S(-4))^\vee = S(d-1)/JS(d-1),
$$
where $J = (x_0,x_1,F,G)$. Since $S/(x_{0}, x_{1}, F_{0}, F_{1})$ has socle in degree $2d-2$, we see that 
$$
M(C) = M(C)^\vee = S(d-1)/JS(d-1).
$$
as well.
\end{example}


In earlier chapters we analyzed various families of curves in $\PP^3$ by linking the curves to simpler curves. One of the main
results of Lazarsfeld and Rao verified a conjecture of Joe Harris that this won't work for general curves of high genus.
Using the Maximal Rank theorem of Eric Larson~\cite{Larson}, we can give a bound:

\begin{theorem}
If $C$ is a general smooth projective curve of large genus, or if $C$ has genus $\geq 10$ and is embedded in $\PP^3$ by a general linear series,
then $C$ is minimal in its linkage class, and thus any Curve in the even linkage class of $C$ is of at least as high genus and degree as $C$.
\end{theorem}


The first version is proven in \cite{Lazarsfeld-Rao}. We prove the version with a general line bundle:
\begin{proof}
What Lazarsfeld and Rao actually prove is that if $e(C) := \max\{e \mid H^1(\sO_C(e)) \neq 0\}$, and $C$ lies on no surface o degree $\leq e+3$, then $C$ is minimal in its even linkage class (if $C$ lies on no surface of degree $\leq e+4$, then $C$ is (up to automorphisms of 
$\PP^3$) the only curve $C'$ with Rao module $M(C') = M(C)$ is $C$ itself.

Now suppose that $C$ is a general curve of genus $g$, embedded in $\PP^3$ as a nondegenerate curve,
by a general line bundle of degree $d$.
 By Petri's Theorem~\ref{****},
the line bundle $\sO_C(2)$ is nonspecial \fix{insert pf}, so $e(C) =1$. By the maximal rank theorem \cite{Larson},
$C$ lies on no surface of degree $e+3 = 4$ if and only if 
$$
4d-g+1 = H^0(\sO_C(4)) \geq H^0 (\sO_{\PP^3}(4) = 35.
$$ 
By the Brill-Noether Theorem~\ref{****},  $d\geq\lceil (3/4)g\rceil+3$.
For $g = 10$ we have $d\geq 11$, so $4d-g+1 = 35$.
and for $g>10$ the difference $(4d-g+1)-35$ only grows.
\end{proof}

%footer for separate chapter files

\ifx\whole\undefined
%\makeatletter\def\@biblabel#1{#1]}\makeatother
\makeatletter \def\@biblabel#1{\ignorespaces} \makeatother
\bibliographystyle{msribib}
\bibliography{slag}

%%%% EXPLANATIONS:

% f and n
% some authors have all works collected at the end

\begingroup
%\catcode`\^\active
%if ^ is followed by 
% 1:  print f, gobble the following ^ and the next character
% 0:  print n, gobble the following ^
% any other letter: normal subscript
%\makeatletter
%\def^#1{\ifx1#1f\expandafter\@gobbletwo\else
%        \ifx0#1n\expandafter\expandafter\expandafter\@gobble
%        \else\sp{#1}\fi\fi}
%\makeatother
\let\moreadhoc\relax
\def\indexintro{%An author's cited works appear at the end of the
%author's entry; for conventions
%see the List of Citations on page~\pageref{loc}.  
%\smallbreak\noindent
%The letter `f' after a page number indicates a figure, `n' a footnote.
}
\printindex[gen]
\endgroup % end of \catcode
%requires makeindex
\end{document}
\else
\fi
